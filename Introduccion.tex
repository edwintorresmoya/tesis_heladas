\chapter{Introducción}
La agricultura en la Sabana de Bogotá es de gran importancia para el abastecimiento alimentario de vairas ciudades, incluyendo Bogotá. En la sabana de Bogotá históricamente se han presentado fenómenos de bajas temperaturas, algunos de estos llamados heladas. Hoy en día las áreas en las que se puede hacer agricultura se han venido reduciendo, ya que hay una competencia de los terrenos con la construcción, adicionalmente la frontera agrícola se encuentra restringida por la delimitación de páramos. Los aumentos de temperatura que se han experimentado en la Sabana de Bogotá se han venido convirtiendo en un problema para los productores, ya que las variedades de papa tradicionales no están adaptadas a las temperaturas extremas que se están presentando en la Sabana de Bogotá.\\

Por esta razón nace la idea de este trabajo. Ya que, la predicción de las altas y las bajas temperaturas puede ser una herramienta que ayude a los agricultores. Desafortunadamente las medidas para afrontar las heladas no son tan fácil de implemetar, porque se requiere una inversión previa en infraestructura y en la muchos casos estas medidas están fuera del alcance de un gran grupo de agricultores. A pesar de estas situaciones, este trabajo tiene como finalidad mejorar la predicción de un modelo meteorológico con la finalidad de ofrecer mejores pronósticos, ya que cualquier aporte que mejore la situación de los productores en la Sabana de Bogotá será un gran avance.\\

Cuando se emprendió este trabajo se tenia claro cuáles eran los objetivos, pero no sabíamos las dificultades por las que se iba a pasar. Por esta razón este trabajo se divide en una parte de exploración que es el capítulo 1, otra parte que hace parte de la limpieza de datos de la estaciones automáticas capitulo 2 y comparación de los resultados del modelo meteorológico con los datos reales para definir dominios, mejores parametrizaciones y tiempos de inicio capitulo 3.\\



\chapter{Objetivos}

\section{Objetivo general}

\begin{itemize}
    \item Analizar las condiciones meteorológicas asociadas con heladas y temperaturas altas nocivas para el cultivo de papa (\textit{Solanum tuberosum}) en la Sabana de Bogotá a partir de datos de estaciones meteorológicas y de un modelo regional de clima.
    
\end{itemize}

\section{Objetivos específicos}

\begin{itemize}
    \item Identificar las características de ocurrencia bajo las cuales se presentan las heladas y temperaturas altas nocivas para el cultivo de papa en la Sabana de Bogotá teniendo en cuenta duración, hora del día, mes del año y año.
    
    \item Identificar las características meteorológicas generales bajo las cuales se presentan las heladas y temperaturas altas nocivas para el cultivo de papa en la Sabana de Bogotá teniendo en cuenta la hora del día, mes del año y año.
    
    \item Identificar las características meteorológicas específicas bajo las cuales se presentan las heladas y temperaturas altas nocivas para el cultivo de papa en la Sabana de Bogotá a través de casos de estudio.
    
    \item Realizar un diagnóstico de la habilidad de un modelo regional de clima para reproducir las condiciones meteorológicas en que se presentan las heladas y las temperaturas altas nocivas para el cultivo de papa en casos de estudio usando diferentes parametrizaciones de la capa límite planetaria.
\end{itemize}