\chapter{Capítulo 4}

%%%%%%%%%%%%%%%%%%%%%%%%%%%%%%%%%%%%%%%%%%%%%%%%%%%%%%%%%%%%%%
%%%%%%%%%%%%%%%%%%%%%%%%%%%%%%%%%%%%%%%%%%%%%%%%%%%%%%%%%%%%%%

%%%%%%%%%%%%%%%%%%%%%%%%%%%%%%%%%%%%%%%%%%%%%%%%%%%%%%%%%%%%%%%%%%%
%%%%%%%%%%%%%%%%%%%%%%%%%%%%%%%%%%%%%%%%%%%%%%%%%%%%%%%%%%%%%%%%%%%
%%%%%%%%%%%%%%%%%%%%%%%%%%%%%%%%%%%%%%%%%%%%%%%%%%%%%%%%%%%%%%%%%%%

%\begin{comment}
\section{Determinación de los casos de estudio}%Código busqueda_fechas.py

%Se determinaron las fechas y horas en las cuales la temperatura estuvo bajo 0\celc\ y sobre 20\celc\ en las estaciones convencionales presentes en la zona, con la finalidad de determinar las mejores fechas para realizar el estudio. Como se vio anteriormente las estaciones automáticas tienden a sobre estimar los valores de la temperatura del aire, por esta razón, el límite de las altas temperaturas para las estaciones automáticas será de 25\celc.\\

Para realizar esta fase del estudio se va a tomar como referencia la estación Tibatitatá, ya que este lugar es unos de sitios donde se hace la calibración de las estaciones y además cuenta con datos de una estación convencional y una estación automática. Los criterios para la selección de los casos fueron:

\begin{itemize}
    \item Encontrar un período de altas y bajas temperaturas para un mes típico.
    
    \item Encontrar un período de altas y bajas temperaturas para un mes no típico.
    
    \item Encontrar el fenómeno de mas larga duración para altas y bajas temperaturas.
    
    \item Encontrar un período de altas y bajas temperaturas para un mes con influencia de un evento El Niño o La Niña.
\end{itemize}

Los meses en los que se presentan normalmente las heladas son diciembre, enero y febrero (esta información se puede en la sección de Caracterización de las heladas y altas temperaturas (Sección \ref{area_caracterizacion_heladas_extremas}) como resultado se presentan los siguientes cuatro casos:



\begin{itemize}
\item{Caso 1}%Este caso corresponde al mas largo, fuerte y está en el último mes de un periodo el niño ggweather.com/enso/oni.html

El primer caso de estudio se encuentra entre las fechas 31 de enero del 2007 hasta el 5 de febrero del 2007 que corresponde a un mes neutro según \citet{NOAA-ORI}. Se seleccionó este caso porque en el día 4 de febrero del 2007 se presentó la temperatura más baja reportada desde el año 1996 en la Sabana de Bogotá, ver Figura   \ref{fig:tmp_autom_tibaitata}. Adicionalmente, se presentaron altas temperaturas en algunas zonas,  ver Tabla \ref{table:caso1}. Según la estación Tibaitatá automática, la helada tuvo una duración de 5 horas comenzando a las 2 a.m. y finalizando a las 7 a.m., esto la convierte en la helada más larga para nuestro período de estudio que va desde el año 2007 hasta el año 2017. Por otra parte para esta misma estación, se registraron altas temperaturas desde las 1 pm hasta las 3 pm.

\item{Caso 2}%Mes no común en un niño (el periodo del 2015 fue niño todo)

El segundo caso seleccionado corresponde a una helada que se presentó en un mes atípico del año, en una temporada reportada en fase neutra de ENSO según \citet{NOAA-ORI}. Este caso se encuentra entre el 29 de agosto del 2014 hasta el 2 de septiembre del 2014. La helada se presentó el día 30 de agosto del 2014, ver Figura \ref{fig:tmp_autom_Tibaitatá_2014}. Esta helada fue registrada por varias estaciones convencionales, aunque las temperaturas no descendieron tanto como en el caso 1, ver Tabla \ref{table:caso2}.  Las bajas temperaturas que se registraron en la Estación Tibaitatá Automática duraron desde las 4 am hasta las 5 am. No se registraron temperaturas superiores a 25 \celc\ en la estación automática, pero las estaciones convencionales sí registraron valores superiores a 20 \celsius\ a las 3 p.m. el 30 de agosto, a las 3 p.m. del 31 de agosto y 4 p.m. del 1 de septiembre.

\begin{figure}[H]
    \centering
    \caption{Valores de temperatura para la estación Tibaitatá automática para las fechas entre el 29 de agosto hasta el 2 de septiembre del 2014. Las líneas verticales discontinuas corresponden a la temperatura de 0\celc\ en las horas 4 y 5 a.m. del 30 de octubre del 2014. Las líneas verticales de puntos corresponden a una temperatura de 20\celc\ a las 3 p.m. el 30 de agosto, a las 3 p.m. del 31 de agosto y 4 p.m. del 1 de septiembre.}
    \includegraphics[draft=false, scale=0.5]{casos_altos_bajas/grafica_minimas_tmp_tibaitata_helada_2014.png}
    \label{fig:tmp_autom_Tibaitatá_2014}
\end{figure}


En la figura \ref{carta_caso2_20140828} podemos observar que el día 28 se estaban presentando huracanes, estos estaban golpeando las costas orientales de Norte América. La mayor parte del territorio colombiano estaba en un área de baja presión. Estos huracanes estaban siendo fortalecidos por las ondas tropicales. La ZCIT no se encontraba sobre el territorio nacional. El día 29 de agosto se puede observar que se encontraba la ZCIT en la latitud 10\degree N, pero esta no estaba sobre el territorio nacional (Figura \ref{carta_caso2_20140829}). El día 30 de agosto de 2014 sobre la mayoría del territorio colombiano se encontraba en un área de baja presión. Las condiciones presentadas en estos días favorecieron la formación del huracán Cristobal.

\begin{figure}
\begin{subfigure}[normla]{0.5\textwidth}
\caption{Carta de superficie para el día 28 de agosto de 2014, hora 2:53 UTC.}
\includegraphics[draft=false, scale=0.25]{cartas_superficie/cs_20140828.png}
\label{carta_caso2_20140828}
\end{subfigure}
~
\begin{subfigure}[normla]{0.5\textwidth}
\caption{Carta de superficie para el día 29 de agosto de 2014, hora 2:32 UTC.}
\includegraphics[draft=false, scale=0.25]{cartas_superficie/cs_20140829.png}
\label{carta_caso2_20140829}
\end{subfigure}

~
\centering
\begin{subfigure}[normla]{0.5\textwidth}
\caption{Carta de superficie para el día 30 de agosto de 2014, hora 2:56 UTC.}
\includegraphics[draft=false, scale=0.25]{cartas_superficie/cs_20140830.png}
\label{carta_caso2_20140830}
\end{subfigure}

\caption{Análisis de superficie para el entre el 28 y 30 de agosto de 2014.}
\label{carta_caso2_20140829}
\end{figure}

El radiosondeo del 30 de agosto tuvo problemas ya que llegó hasta la altura de 400 mb (Figura \ref{radioson_201408}). De la fracción del radiosondeo se puede observar una inversión térmica hasta los 500 mb de altura. Es importante resaltar que estas condiciones de inversión están asociadas a heladas en la Sabana de Bogotá. Estas condiciones son similares a las presentadas para el caso 1.\\

%La tasa de enfriamiento es de 6.4°C por kilómetro. La transferencia de calor del suelo es más alta que el del aire. Por esta razón se enfría más rápido el suelo que la atmósfera. La transferencia de calor está dada por la cercanía que tengan las moléculas, en el sólido la transferencia de calor es más alta porque las moléculas están juntas, en el agua están juntas pero no tanto como en el sólido, y las moléculas que se encuentran más separadas son las moléculas de aire, por lo tanto la transferencia de calor es más baja.

\begin{figure}
\centering
\includegraphics[draft=false, scale=0.25]{radiosondeos/radio_sondeo_20140830.png}
\caption{Radiosondeo para el día 30 de agosto de 2014 a las 7 am HL. Tomado de la \textcolor{blue}{\href{http://weather.uwyo.edu/upperair/sounding.html}{Universidad de Wyoming.}}}
\label{radioson_201408}
\end{figure}
%% Discusión sobre las imágenes del goes para el caso 2 del 201408

En la Figura \ref{fig:goes_total_caso2} se puede observar que sobre la zona de estudio no se presentó una cobertura nubosa. La Figura \ref{fig:goes_caso2_2} permite identificar una nubosidad sobre el mar Caribe, especialmente sobre Haití y República Dominicana. Sobre Colombia se puede observar una formación nubosa en el sur de l país. Basado en estas condiciones podemos concluir que sobre el área de interés que es la Sabana de Bogotá no había cobertura nubosa en las horas de la mañana.

\begin{figure}[H]
    \begin{subfigure}[normla]{0.5\textwidth}
\caption{Radiación infrarroja de onda corta.}
\includegraphics[draft=false,     scale=0.23]{{goes/201408/goes13.2014.242.001519.BAND_02}.jpg}
\label{fig:goes_caso2_2}
\end{subfigure}
~
\begin{subfigure}[normla]{0.5\textwidth}
\caption{Vapor de agua}
\includegraphics[draft=false,     scale=0.23]{{goes/201408/goes13.2014.242.001519.BAND_03}.jpg}
\label{fig:goes_caso2_3}
\end{subfigure}
    ~
\centering
\begin{subfigure}[normla]{0.3\textwidth}
\caption{Radiación infrarroja de onda larga.}
\includegraphics[draft=false,         scale=0.21]{{goes/201408/goes13.2014.242.001519.BAND_04}.jpg}
\label{fig:goes_caso2_4}
\end{subfigure}
    
    	
    \caption{Imágenes del GOES 12: canal 2 (a), canal 3 (b) y canal 4(c). Para el día 30 de agosto de 2014 a las 00 UTC (29 de febrero de 2007 a las 19:00 HL).}
    \label{fig:goes_total_caso2}	
\end{figure}

\begin{table}[H]
\centering
\caption{Temperaturas máximas diarias y mínimas diarias de las estaciones convencionales que registraron temperatura sobre 20\celsius\ o debajo de 0\celsius\ o ambas y temperaturas máxima diaria y mínima diaria reistradas para la estación automática Tibaitatá, para el caso 2 del día 30 de agosto del 2014.}
\begin{tabular}{p{3.5cm}p{2cm}lll}
Temperatura igual o menor a 0\celsius\ & Temperatura sobre 20\celc & Código   & Nombre de la estación & Municipio \\ \hline
-3.6 & &  21205870 &   Salitre El [21205870] &      Bojacá \\
-1.6 & &  21205880 &         Flores Chibcha  &      Madrid \\
-0.2 & &  21205920 &  Suasuque    [21205920] &        Sopó \\
 0.0 & &  21205940 &   Villa Inés [21205940] &  Facatativá \\
 & 20.6 &  21206560 &  INEM Kennedy [21206560] &  Bogotá D.C. \\
 & 21.0 &  21205710 &         Jardín Botánico  &  Bogotá D.C. \\
 & 21.0 &  21206690 &      Col Miguel A. Caro  &  Bogotá D.C. \\
 0.0 & 21.4 &  21205980 &         Providencia Gja  & Tenjo \\
 & 21.4   & 21206620 &       Col H Duran Dussan  &  Bogotá D.C. \\
 0.0 & 22.0 &  21205420 &     Tibaitatá &     Mosquera \\
 -1.0 & 22.0 &  21206990 &    Tibaitatá [Automática] &    Mosquera \\
\end{tabular}

\label{table:caso2}
\end{table}




\item{Caso 3}

El tercer caso seleccionado corresponde a una alta temperatura presentada en el mes de agosto en un evento El Niño según \citet{NOAA-ORI}. Se seleccionó este caso porque se presentó en un mes poco habitual. El período de estudio se encuentra entre el 24 y el 28 de agosto del 2015. Se registraron temperaturas superiores a 20\celc\ en dos estaciones, la estación convencional Tibaitatá, la cual presentó una temperatura cerca a los 20\celc\ y la Estación Tibaitatá Automática registró valores superiores a 25\celc, ver Tabla \ref{table:caso3}. No se presentaron temperaturas iguales o inferiores a 0\celc. La duración de este evento fue de una hora, iniciando a las 9 am y finalizando a las 10 am. Se observa en la Figura \ref{fig:tmp_autom_Tibaitatá_201508} que en todos los días estudiados se presentaron temperaturas superiores a 20\celsius.

\begin{figure}[H]
    \centering
    \caption{Valores de temperatura para la estación Tibaitatá automática para las fechas entre el 24 y 28 de agosto del 2015. Las líneas verticales de puntos corresponden a una temperatura de 20\celsius, donde se puede observar que en todos los días estudiados la temperatura superó este umbral.}
    \includegraphics[draft=false, scale=0.5]{casos_altos_bajas/grafica_minimas_tmp_tibaitata_helada_201508.png}
    \label{fig:tmp_autom_Tibaitatá_201508}
\end{figure}

En la Figura \ref{carta_caso3} no se encuentra la ZCIT, pero sí se puede observar la zona de \textit{Monsoon trought} (MONSOON TROF). \textit{Monsoon trought} es una parte de la ZCIT que representa un área donde los vientos del norte y los vientos del sur se encuentran, esta zona es mostrada como una línea que conecta áreas de baja presión \citep{Wang2006}.

Basado en el análisis de superficie podemos observar que el día 25 de agosto de 2015 sobre Colombia había un sistema de baja presión y este sistema se encontraba sobre la zona de estudio (Figura \ref{carta_caso3_20150825}). El día 26 de agosto de 2015 en el mar Caribe un sistema de baja presión se convirtió en un huracán (Erika), sobre Colombia no hubo sistemas de bajas presiones, pero la zona del \textit{Monsoon trought} se encontraba en la parte Norte de Colombia (Latitud 10\degree N). El día 27 de agosto de 2015 se encontraba sobre Colombia el \textit{Monsoon trought} en las costas pacíficas con dirección a la Región de la Guajira y no había ningún sistema de altas o bajas sobre sobre Colombia. Adicionalmente para los días analizados no se presentó cobertura nubosa sobre el territorio colombiano.\\



\begin{figure}[H]
\begin{subfigure}[normla]{0.5\textwidth}
\caption{Carta de superficie para el día 25 de agosto de 2015, hora 2:16 UTC.}
\includegraphics[draft=false, scale=0.25]{cartas_superficie/cs_20150825.png}
\label{carta_caso3_20150825}
\end{subfigure}
~
\begin{subfigure}[normla]{0.5\textwidth}
\caption{Carta de superficie para el día 26 de agosto de 2015, hora 2:24 UTC.}
\includegraphics[draft=false, scale=0.25]{cartas_superficie/cs_20150826.png}
\label{carta_caso3_20150826}
\end{subfigure}

~
\centering
\begin{subfigure}[normla]{0.5\textwidth}
\caption{Carta de superficie para el día 27 de agosto de 2015, hora 2:27 UTC.}
\includegraphics[draft=false, scale=0.25]{cartas_superficie/cs_20150827.png}
\label{carta_caso3_20150827}
\end{subfigure}

\caption{Análisis de superficie entre el 25 y 27 de agosto de 2015.}
\label{carta_caso3}
\end{figure}

Como se puede observar en el radiosondeo (Figura \ref{radiosondeo_201508}) se observa una pequeña inestabilidad en los primeros metros del radiosondeo, pero posteriormente la atmósfera se estabiliza. Presenta alta humedad entre los 650 y los 700 mb. A una altura superior a los 650 mb la atmósfera tiene menor humedad.


\begin{figure}[H]
\centering

\includegraphics[draft=false, scale=0.25]{radiosondeos/radio_sondeo_20150827.png}
\caption{Radiosondeo para el día 27 de agosto de 2015 a las 7 am HL. Tomado de la \textcolor{blue}{\href{http://weather.uwyo.edu/upperair/sounding.html}{Universidad de Wyoming.}}}
\label{radiosondeo_201508}
\end{figure}

%% Discución sobre las imágenes del goes para el caso 3 del 201508

\begin{figure}[H]
    \begin{subfigure}[normla]{0.5\textwidth}
\caption{Radiación infrarroja de onda corta.}
\includegraphics[draft=false,     scale=0.23]{{goes/201508/goes13.2015.239.001519.BAND_02}.jpg}
\label{fig:goes2}
\end{subfigure}
~
\begin{subfigure}[normla]{0.5\textwidth}
\caption{Vapor de agua}
\includegraphics[draft=false,     scale=0.23]{{goes/201508/goes13.2015.239.001519.BAND_03}.jpg}
\label{fig:goes3}
\end{subfigure}
    ~
\centering
\begin{subfigure}[normla]{0.3\textwidth}
\caption{Radiación infrarroja de onda larga.}
\includegraphics[draft=false,         scale=0.21]{{goes/201508/goes13.2015.239.001519.BAND_04}.jpg}
\label{fig:goes4}
\end{subfigure}
    
    	
    \caption{Imágenes del GOES 13: canal 2 (a), canal 3 (b) y canal 4(c). Para el día 27 de agosto de 2015 a las 00 UTC (26 de agosto de 2015 a las 19:00 HL).}
    \label{fig:goes_total}	
\end{figure}


\begin{table}[H]
\centering
\caption{Temperaturas máximas registradas diarias de las estaciones convencionales que registraron temperaturas sobre 20\celc y temperatura máxima diaria registrada para la estación automática Tibaitatá, para el caso 4 correspondiente al día 27 de agosto del 2015.}
\begin{tabular}{p{5cm}lll}
Temperatura iguales o superiores a 20\celc & Código   & Nombre de la estación & Municipio \\ \hline
22.0           & 21206620 & Col H Duran Dussan   & Bogotá \\
20.1           & 21205980 & Providencia Gja   & Tenjo \\
19.6           & 21205420 & Tibaitatá   & Mosquera \\
26.0         & 21206990 & Tibaitatá[Automática] & Mosquera
\end{tabular}
\label{table:caso3}


\end{table}


\item{Caso 4}

%\textit{Se creó una carpeta en agrometeo llamada /home/agrometeo/wrf/resultados/zona\_201508, para procesar el wps y los archivos usados se descargaron en Downloads}


La cuarta fecha seleccionada corresponde a una alta temperatura presentada en el mes de septiembre de 2015 en una temporada bajo la influencia de un evento El Niño según \citet{NOAA-ORI}. Se seleccionó este caso ya que fue uno de los que presentó el intervalo de tiempo con mas horas sobre 25 \celc\ el cual comenzó a las 10:33 a.m. y finalizó a las 3:21 p.m., durando cuatro horas y media, ver Figura \ref{fig:tmp_autom_Tibaitatá_201509}. El período de estudio se encuentra entre el 06 de septiembre del 2015 y el 09 de septiembre del 2015. Varias estaciones registraron temperaturas sobre 20 \celsius ninguna estación presentó valores menores o iguales a 0\celc, ver la Tabla \ref{table:caso4}. Adicionalmente, se presentó temperaturas sobre 20\celsius\ el 6 de septiembre de 2015 a las 3 p.m. y el 7 de septiembre  de 2015 a las 12 m.

%El día 20150907 no estaba disponible en los datos del GFS, por esta razón no se usaron estos datos. Pero si estaban los datos del día analizado que corresponde al 20150908.
%ftp://nomads.ncdc.noaa.gov/GFS/analysis_only/201509/20150907/

\begin{figure}[H]
    \centering
    \caption{Valores de temperatura para la estación Tibaitatá automática para las fechas entre el 6 y el 9 de septiembre del 2015. Las líneas verticales de puntos corresponden a una temperatura de 20\celsius, donde se puede observar que en todos los días estudiados la temperatura superó este umbral.}
    \includegraphics[draft=false, scale=0.5]{casos_altos_bajas/grafica_minimas_tmp_tibaitata_helada_201509.png}
    \label{fig:tmp_autom_Tibaitatá_201509}
\end{figure}



\begin{figure}[H]
\begin{subfigure}[normla]{0.5\textwidth}
\caption{Carta de superficie para el día 25 de agosto de 2015, hora 2:16 UTC.}
\includegraphics[draft=false, scale=0.25]{cartas_superficie/cs_20150904.png}
\label{carta_caso4_20150825}
\end{subfigure}
~
\begin{subfigure}[normla]{0.5\textwidth}
\caption{Carta de superficie para el día 26 de agosto de 2015, hora 2:24 UTC.}
\includegraphics[draft=false, scale=0.25]{cartas_superficie/cs_20150905.png}
\label{carta_caso4_20150826}
\end{subfigure}

~
\centering
\begin{subfigure}[normla]{0.5\textwidth}
\caption{Carta de superficie para el día 27 de agosto de 2015, hora 2:27 UTC.}
\includegraphics[draft=false, scale=0.25]{cartas_superficie/cs_20150906.png}
\label{carta_caso4_20150827}
\end{subfigure}

\caption{Análisis de superficie para los días entre el 4 y 6 de septiembre de 2015.}
\label{carta_caso4}
\end{figure}

Como se puede observar en el radiosondeo \ref{radiosondeo_201508} podemos ver una pequeña inestabilidad en los primeros metros del radiosondeo, pero posteriormente se observa que la atmósfera se estabiliza. Presenta alta humedad entre los 650 y los 700 mb, probablemente a esa altura se encuentre nubosidad. A una altura superior a los 650 mb la atmósfera tiene menor humedad.


\begin{figure}[H]
\centering
\includegraphics[draft=false, scale=0.25]{radiosondeos/radio_sondeo_20150908.png}
\caption{Radiosondeo del 8 de septiembre de 2015 a las 7 am HL. Tomado de la \textcolor{blue}{\href{http://weather.uwyo.edu/upperair/sounding.html}{Universidad de Wyoming.}}} 
\label{radiosondeo_201508}
\end{figure}

%% Discución sobre las imágenes del goes para el caso 4 del 201508

\begin{figure}[H]
    \begin{subfigure}[normla]{0.5\textwidth}
\caption{Radiación infrarroja de onda corta.}
\includegraphics[draft=false,     scale=0.23]{{goes/201509/goes13.2015.251.001519.BAND_02}.jpg}
\label{fig:goes2}
\end{subfigure}
~
\begin{subfigure}[normla]{0.5\textwidth}
\caption{Vapor de agua}
\includegraphics[draft=false,     scale=0.23]{{goes/201509/goes13.2015.251.001519.BAND_03}.jpg}
\label{fig:goes3}
\end{subfigure}
    ~
\centering
\begin{subfigure}[normla]{0.3\textwidth}
\caption{Radiación infrarroja de onda larga.}
\includegraphics[draft=false, scale=0.21]{{goes/201509/goes13.2015.251.001519.BAND_04}.jpg}
\label{fig:goes4}
\end{subfigure}
    
    	
    \caption{Imágenes del GOES 13: canal 2 (a), canal 3 (b) y canal 4(c). Para el día 27 de agosto de 2015 a las 00 UTC (26 de agosto de 2015 a las 19:00 HL).}
    \label{fig:goes_total}	
\end{figure}

\begin{table}[H]
\centering
\caption{Temperaturas máximas diarias de las estaciones convencionales que registraron temperaturas sobre 20\celc\ y temperatura máxima registrada para la estación automática Tibaitatá, el 8 septiembre del 2015.}
\begin{tabular}{llll}
Temperatura sobre 20\celc & Código   & Nombre de la estación & Municipio \\ \hline
 20.3 &  21205790 &          Apto El Dorado  &  Bogotá D.C. \\
 20.6 &  21206620 &       Col H Duran Dussan  &  Bogotá D.C. \\
 20.8 &  21206560 &  INEM Kennedy &  Bogotá D.C. \\
 21.6 &  21205980 &         Providencia Gja  &        Tenjo \\
 21.6 &  21206260 &       C.Univ.Arop-UDCA  &  Bogotá D.C. \\
 20.8 &  21205420 &     Tibaitatá &     Mosquera \\
 26.7 & 21206990 & Tibaitatá[Automática] & Mosquera\\
\end{tabular}

\label{table:caso4}



\end{table}
\color{blue}
Las horas en las que se presenta las menores temperaturas es cerca de las 5 am (hora local) como se vio en el Capítulo 1, este estudio se enfocó en estudiar las bajas y las altas temperaturas, por esta razón luego de identificar el día en el que se presentaron los eventos de altas o bajas temperaturas para los diferentes casos se inició 12 horas antes de las 5 am hora local (hora local). Para el caso 1 el evento se presentó el 4 de febrero del 2007, para el caso 2 el evento se presentó el 30 de agosto de 2014, el caso 3 fue el 27 de agosto de 2015 y el 8 de septiembre de 2015, las horas de inicio se encuentran en la Tabla \ref{table:fechas_de_inicio}.\\

\begin{table}[H]
\centering
\caption{Fechas de inicio de modelación para cada uno de los casos en horas UTC y hora local.}
\begin{tabular}{lll}
Caso & Fecha de inicio UTC & Fecha de inicio hora local \\ \hline
1 & 2007-02-03 18:00 & 2007-02-05 13:00 \\
2 & 2014-08-29 18:00 & 2014-08-29 13:00 \\
3 & 2015-08-26 18:00 & 2015-08-27 13:00 \\
4 & 2015-09-07 18:00 & 2015-09-07 13:00
\end{tabular}
\label{table:fechas_de_inicio}
\end{table}
\color{black}

\section{Búsqueda de las mejores parametrizaciones}
\label{busqueda_mejores_param}

Para la evaluación de la mejor parametrización se tuvo como base la configuración que el IDEAM usa para hacer las simulaciones sobre todo el territorio colombiano. Se probaron diferentes parametrizaciones con cada una de sus opciones, estas pruebas no fueron combinatorias, ya que se requiere demasiado tiempo para realizar todas las posibles combinaciones (ver Tabla \ref{tabla_parametrizacion_opciones}). Las pruebas se hicieron de tal manera que se cambia una opción de parametrización a la vez mientras que todas las demás opciones de parametrizaciones son iguales a las de la configuración IDEAM-Colombia. Las parametrizaciones a las cuales se les probaron sus diferentes opciones son: capa límite planetaria (bl\_pbl\_physics) la cual se encarga de los flujos superficiales de (calor, humedad y momento) y su difusión vertical; parametrización de cúmulos (cu\_physics); microfísica de nubes (mp\_physics), la cual se encarga de resolver los procesos que tienen que ver con el vapor de agua, nubes y el proceso de precipitación; parametrizaciones de radiación de onda larga y corta (ra\_lw\_phyisics y ra\_sw\_phyisics), las cuales se encargan de las tendencias de la temperatura atmosférica y los flujos de radiación de la superficie; parametrización de suelo (sf\_surface\_physics), la cual se encarga de modelar las temperaturas del suelo y parametrizaciones de superficie (sf\_sfclay\_surface\_physics) se encarga de determinar los coeficientes de intercambio entre el suelo y la atmósfera. También se realizaron dos simulaciones adicionales, una con la configuración IDEAM-Colombia y otra con la configuración IDEAM-Bogotá, las cuales se muestran en el Anexo \ref{anexo:namelist_mejor_parametrizacion}.\\

%Es importante recordar que la parametrización de capa límite planeraria se encarga de w
%https://www.climatescience.org.au/sites/default/files/physics-3.9-new-pt1.pdf
%rainc = comes from cumulus scheme
%rainnc = comes from microphysics scheme

Las opciones evaluadas para cada parametrización se encuentran en la Tabla \ref{tabla_parametrizacion_opciones}. En esta tabla podemos observar que se van a evaluar 10 opciones de parametrización para la capa límite planetaria, 9 opciones para cúmulos, 17 opciones para microfísica, 7 opciones para la radiación de onda larga, 7 opciones para la radiación de onda corta, 3 opciones para superficie y 1 opción para el suelo; para las opciones del suelo se probaron las opciones 0, 1, 3, 4, 31 y 99, pero sólo la opción 1 generó resultados. Adicionalmente, se evaluará las configuraciones propuestas por el IDEAM para Colombia y Bogotá. Para un total de 56 diferentes combinaciones.

%En total se realizaron 88 diferentes modelaciones, pero algunas de estas modelaciones no fueron aceptadas por el modelo WRF, ya que para que fueran ejecutadas exitosamente era necesario realizar otros cambios en la \textit{namelist.input}, pero para asegurar la homogeneidad en los resultados se realizaron los cambios que están en la tabla \ref{tabla_parametrizacion_opciones}.

A continuación se muestra un ejemplo de cómo se realizaron los cambios en las configuraciones del \texttt{namelist.input}. Se tomó como base la configuración del IDEAM para Colombia y se probaron las opciones para una sola parametrización a la vez. Como ejemplo se va a mostrar una parte de la  configuración del \texttt{namelist.input} IDEAM para Colombia y cómo se realizaron los cambios para la parametrización de microfísica (mp\_physics) para las opciones 0, 1 y 51.\\

\begin{table}[H]
\caption{Parametrizaciones y opciones usadas, en el WRF versión 9.1.1.}
\label{tabla_parametrizacion_opciones}
\begin{tabular}{lp{8cm}}
Parametrización & Opciones \\ \hline
Capa límite planetaria (bl\_pbl\_physics) & 0, 1, 5, 6, 7, 8, 9, 11, 12, 99 \\ %10
Cúmulos (cu\_physics)      & 0, 1, 2, 6, 11, 14, 16, 93, 99 \\ %9
Microfísica (mp\_physics)      & 0, 1, 2, 3, 5, 6, 7, 8, 9, 10, 11, 13, 14, 16, 19, 32, 51 \\%17
Radiación de onda larga (ra\_lw\_phyisics) & 0, 1, 3, 4, 5, 7, 31 \\%7
Radiación de onda corta (ra\_sw\_phyisics) & 0, 1, 2, 3, 4, 5, 7 \\%7 
Superficie (sf\_sfclay\_physics) & 0, 1, 91\\ %3
Suelo (sf\_surface\_physics) & 1 \\ % 1
IDEAM para Bogotá (IDEAM-Bogotá) & \\
IDEAM para Colombia (IDEAM-Colombia) & \\ % Total 56
\end{tabular}
\end{table}
%\begin{tabular}{lp{8cm}}
%Parametrización & Opciones \\ \hline
%Capa límite planetaria (bl\_pbl\_physics) & 0, 1, 2, 3, 4, 5, 6, 7, 8, 9, 10, 11, 12, 99 \\ %14
%Cúmulos (cu\_physics)      & 0, 1, 2, 3, 4, 5, 6, 7, 11, 14, 16, 93, 99 \\ %13
%Microfísica (mp\_physics)      & 0, 1, 2, 3, 5, 6, 7, 8, 9, 10, 11, 13, 14, 15, 16, 19, 30, 32, 50, 51, 52 \\%21
%Radiación de onda larga (ra\_lw\_phyisics) & 0, 1, 2, 3, 4, 5, 7, 31, 99 \\%9
%Radiación de onda corta (ra\_sw\_phyisics) & 0, 1, 2, 3, 4, 5, 7, 31, 99 \\%9 
%Superficie (sf\_sfclay\_physics) & 0, 1, 2, 3, 4, 5, 7, 10, 91\\ %9
%Suelo (sf\_surface\_physics) & 0, 1, 2, 3, 4, 5, 7, 8, 31, 99\\ % 10
%IDEAM para Bogotá (IDEAM-Bogotá) & \\
%IDEAM para Colombia (IDEAM-Colombia) & \\ % Total 88
%\end{tabular}


\begin{figure}[H]

\center
\texttt{
...\\
\&physics\\
mp\_physics~~~~~~~~~~~~~~~=~3,~3,\\
ra\_lw\_physics~~~~~~~~~~~~=~1,~~1,\\
ra\_sw\_physics~~~~~~~~~~~~=~1,~~1,\\
%radt~~~~~~~~~~~~~~~~~~~~~=~30,~30,\\
sf\_sfclay\_physics~~~~~~~~=~1,~~1,\\
sf\_surface\_physics~~~~~~~=~1,~~1,\\
bl\_pbl\_physics~~~~~~~~~~~=~1,~~1,\\
bldt~~~~~~~~~~~~~~~~~~~~~~=~1,~~1,\\
cu\_physics~~~~~~~~~~~~~~~~=~1,~~1,\\
...\\
}
\centering
\includegraphics[draft=false, scale=0.0001]{graph/ideam_ej_1.png}
\caption{Sección del \texttt{namelist.input} IDEAM-Colombia.}
\label{imag_tab_ej}

\end{figure}




%\texttt{
%...\\
%\&physics\\
%mp\_physics~~~~~~~~~~~~~~~=~\underline{0},~\underline{0},\\
%ra\_lw\_physics~~~~~~~~~~~~=~1,~~1,\\
%ra\_sw\_physics~~~~~~~~~~~~=~1,~~1,\\
%radt~~~~~~~~~~~~~~~~~~~~~=~30,~30,\\
%sf\_sfclay\_physics~~~~~~~~=~1,~~1,\\
%sf\_surface\_physics~~~~~~~=~1,~~1,\\
%bl\_pbl\_physics~~~~~~~~~~~=~1,~~1,\\
%bldt~~~~~~~~~~~~~~~~~~~~~~~=~1,~~1,\\
%cu_physics~~~~~~~~~~~~~~~~~=~1,~~1,\\
%...\\
%}




\begin{figure}[H]


\center
\texttt{
...\\
\&physics\\
mp\_physics~~~~~~~~~~~~~~~=~\underline{0,~~0},\\
ra\_lw\_physics~~~~~~~~~~~~=~1,~~1,\\
ra\_sw\_physics~~~~~~~~~~~~=~1,~~1,\\
%radt~~~~~~~~~~~~~~~~~~~~~=~30,~30,\\
sf\_sfclay\_physics~~~~~~~~=~1,~~1,\\
sf\_surface\_physics~~~~~~~=~1,~~1,\\
bl\_pbl\_physics~~~~~~~~~~~=~1,~~1,\\
bldt~~~~~~~~~~~~~~~~~~~~~~=~1,~~1,\\
cu\_physics~~~~~~~~~~~~~~~~=~1,~~1,\\
...\\
}
\centering
\includegraphics[draft=false, scale=0.018]{ideam_ej_1.png}
\caption{Sección del \texttt{namelist.input} para el primer ejemplo de la microfísica con la opción 0 (mp\_physics-0).}
\label{imag_tab_ej2}

\end{figure}



\begin{figure}[H]

\center
\texttt{
...\\
\&physics\\
mp\_physics~~~~~~~~~~~~~~~=~\underline{1,~~1},\\
ra\_lw\_physics~~~~~~~~~~~~=~1,~~1,\\
ra\_sw\_physics~~~~~~~~~~~~=~1,~~1,\\
%radt~~~~~~~~~~~~~~~~~~~~~=~30,~30,\\
sf\_sfclay\_physics~~~~~~~~=~1,~~1,\\
sf\_surface\_physics~~~~~~~=~1,~~1,\\
bl\_pbl\_physics~~~~~~~~~~~=~1,~~1,\\
bldt~~~~~~~~~~~~~~~~~~~~~~=~1,~~1,\\
cu\_physics~~~~~~~~~~~~~~~~=~1,~~1,\\
...\\
}
\centering
\includegraphics[draft=false, scale=0.018]{ideam_ej_1.png}
\caption{Sección del \texttt{namelist.input} para el segundo ejemplo de la microfísica con la opción 1 (mp\_physics-1).}
\label{imag_tab_ej3}


\end{figure}


\begin{figure}[H]

\center
\texttt{
...\\
\&physics\\
mp\_physics~~~~~~~~~~~~~~~=~\underline{51,~51},\\
ra\_lw\_physics~~~~~~~~~~~~=~1,~~1,\\
ra\_sw\_physics~~~~~~~~~~~~=~1,~~1,\\
%radt~~~~~~~~~~~~~~~~~~~~~=~30,~30,\\
sf\_sfclay\_physics~~~~~~~~=~1,~~1,\\
sf\_surface\_physics~~~~~~~=~1,~~1,\\
bl\_pbl\_physics~~~~~~~~~~~=~1,~~1,\\
bldt~~~~~~~~~~~~~~~~~~~~~~=~1,~~1,\\
cu\_physics~~~~~~~~~~~~~~~~=~1,~~1,\\
...\\
}
\centering
\includegraphics[draft=false, scale=0.018]{ideam_ej_1.png}
\caption{Sección del \texttt{namelist.input} para el segundo ejemplo de la microfísica con la opción 51 (mp\_physics-1).}
\label{imag_tab_ej3}

\end{figure}





Este proceso se repitió para cada uno de los cuatro casos de estudio mencionados en la sección "Búsqueda de las mejores parametrizaciones". Para cada caso, se realizaron 56 simulaciones. Las fechas y horas de inicio de las modelaciones se encuentran en la Tabla \ref{tabla_inicio_fin_casos}.\\
 
Se repitió el proceso realizado para la selección de los dominios el cual consiste en seleccionar las combinaciones parametrización-opción que tengan un coeficiente de Pearson en la columna  $Pearson$ mayor a 0.8 y valores inferiores a 0.3 en la columna $NRMSE$ como lo sugiere \citet{Agua2016} y que se encuentr entre los mejores 5 resultados de la columna $ET$, estos resultados se encuentran en la Tabla \ref{tabla_mejores_param_opciones}.

 
 
 \begin{table}[H]
 \centering
 \caption{Fechas y horas UTC de inicio y finalización de los diferentes casos.}
 \label{tabla_inicio_fin_casos}
\begin{tabular}{lll}
Caso & Inicio & Finalización \\ \hline
1 & 2007-02-03 18:00 & 2007-02-05 00:00 \\
2 & 2014-08-29 18:00 & 2014-09-01 00:00 \\
3 & 2015-08-27 18:00 & 2015-08-30 00:00 \\
4 & 2015-09-07 18:00 & 2015-09-10 00:00 \\
\end{tabular}
\end{table}


Es importante resaltar que el número de estaciones con las que se trabaja varía entre los casos. Por ejemplo: en el caso 1 se usaron 10 estaciones, en el caso 2 se usaron 15 estaciones, en el caso 3 se usaron 16 estaciones y en el caso 4 se usaron 15 estaciones debido a la disponibilidad de los datos.\\
%Esto implica que si existiera la combinación parametrización-opción perfecta tendría que tener 56 valores en la frecuencia


\begin{longtable}{lll}

\caption{Conteo de las mejores nueve simulaciones para 10 estaciones. La simulación ideal tendría un valor de frecuencia igual a 10 que sería el número de combinaciones simulaciones-dominios (ver Tabla \ref{tab:estado_hydras}).}
\label{tab_estaciones_5_tiempo}\\
\hline
Caso & Estación & Valores \\

\midrule
\endhead
\midrule
\multicolumn{3}{r}{{Continúa en la siguiente página.}} \\
\midrule
\endfoot

\bottomrule
\endlastfoot

Caso 1 & Pmo Guerrero  & 5 \\
Caso 1 & Subia Automatica  & 5 \\
Caso 1 & Ciudad Bolívar  & 5 \\
Caso 1 & Hda Sta Ana Autom  & 4 \\
Caso 1 & Tibaitatá Automatica  & 5 \\
Caso 1 & Sta Cruz De Siecha  & 5 \\
Caso 1 & La Capilla Autom  & 5 \\
Caso 1 & Chinavita Automatica  & 5 \\
Caso 1 & Pmo Guacheneque  & 5 \\
Caso 2 & Univ Nacional  & 5 \\
Caso 2 & Chinavita Automatica  & 5 \\
Caso 2 & La Boyera Automatica  & 5 \\
Caso 2 & Apto El Dorado  & 5 \\
Caso 2 & Tibaitatá Automatica  & 5 \\
Caso 2 & San Cayetano Autom   & 0 \\
Caso 2 & Bosque Intervenido    & 5 \\
Caso 2 & Hda Sta Ana Autom  & 5 \\
Caso 2 & Esc La Unión Automatica & 5 \\
Caso 2 & Pmo Guerrero  & 5 \\
Caso 2 & Pmo Guacheneque  & 5 \\
Caso 3 & Hda Sta Ana Autom  & 5 \\
Caso 3 & Nueva Generacion  & 5 \\
Caso 3 & Chinavita Automatica  & 5 \\
Caso 3 & Esc La Unión Automatica & 5 \\
Caso 3 & Pmo Guerrero  & 5 \\
Caso 3 & Villa Teresa Automatica  & 5 \\
Caso 3 & Bosque Intervenido    & 5 \\
Caso 3 & Univ Nacional  & 5 \\
Caso 3 & Tibaitatá Automatica  & 5 \\
Caso 3 & Sta Cruz De Siecha  & 5 \\
Caso 3 & Apto El Dorado  & 5 \\
Caso 3 & Pmo Rabanal Automatica   & 0 \\
Caso 3 & San Cayetano Autom   & 0 \\
Caso 3 & Pmo Chingaza  & 0 \\
Caso 3 & Ideam Bogotá  & 0 \\
Caso 4 & La Capilla Autom  & 5 \\
Caso 4 & Ideam Bogotá  & 0 \\
Caso 4 & Chinavita Automatica  & 5 \\
Caso 4 & Tibaitatá Automatica  & 5 \\
Caso 4 & Hda Sta Ana Autom  & 5 \\
Caso 4 & Apto El Dorado  & 5 \\
Caso 4 & Sta Cruz De Siecha  & 5 \\
Caso 4 & Pmo Guerrero  & 5 \\
Caso 4 & San Cayetano Autom   & 0 \\
Caso 4 & Univ Nacional  & 5 \\
Caso 4 & Pmo Chingaza  & 1 \\
Caso 4 & Villa Teresa Automatica  & 5 \\
Caso 4 & Pmo Rabanal Automatica   & 1 \\
Caso 4 & Bosque Intervenido    & 0 \\

\end{longtable}

En algunos casos cuando se configuraba el modelo para hacer la simulación este no realizaba los cálculos, ya que para ciertas opciones de la parametrización es necesa

\begin{table}[H]
\caption{Frecuencia de ocurrencia de la combinación parametrización-opción mayor a 0.8 en $Pearson$ e inferior a 0.3 en $NMRSE$ que se encuentra entre los 5 mejores resultados. * Para la combinación sf\_surface\_physics se evaluaron diferentes opciones pero sólo de la opción 1 se obtuvieron resultados.}
\centering
\label{tabla_mejores_param_opciones}
\begin{tabular}{lr}
\toprule
Combinación parametrización-opción               & Frecuencia       \\
\midrule
bl\_pbl\_physics-1     &      1 \\
bl\_pbl\_physics-5     &      7 \\
bl\_pbl\_physics-6     &      5 \\
bl\_pbl\_physics-7     &     13 \\
bl\_pbl\_physics-8     &     15 \\
bl\_pbl\_physics-9     &      5 \\
bl\_pbl\_physics-12    &      9 \\
bl\_pbl\_physics-99    &      6 \\ \hline
cu\_physics-2         &      9 \\
cu\_physics-3         &      1 \\
cu\_physics-5         &      4 \\
cu\_physics-6         &      7 \\
cu\_physics-11        &      1 \\
cu\_physics-14        &      3 \\
cu\_physics-16        &      4 \\
cu\_physics-99        &      3 \\ \hline
mp\_physics-0         &      8 \\
mp\_physics-2         &      3 \\
mp\_physics-3         &      1 \\
mp\_physics-8         &      1 \\
mp\_physics-11        &      3 \\
mp\_physics-13        &      3 \\
mp\_physics-16        &      1 \\ \hline
ra\_lw\_physics-3      &      6 \\
ra\_lw\_physics-4      &      2 \\
ra\_lw\_physics-5      &      2 \\
ra\_lw\_physics-7      &     13 \\
ra\_lw\_physics-31     &      7 \\ \hline
ra\_sw\_physics-1      &      1 \\
ra\_sw\_physics-2      &     12 \\
ra\_sw\_physics-3      &     14 \\
ra\_sw\_physics-4      &     13 \\
ra\_sw\_physics-5      &      8 \\ \hline
sf\_sfclay\_physics-1  &      2 \\
sf\_sfclay\_physics-91 &     17 \\ \hline
*sf\_surface\_physics-1 &      1 \\
IDEAM-Bogotá         &      1 \\
IDEAM-Colombia       &      4 \\
\bottomrule
\end{tabular}
\end{table}

En la Tabla \ref{tabla_mejores_param_opciones} podemos observar que para las parametrizaciones de capa límite planetaria hay diferentes resultados ya que de las 14 opciones evaluadas se obtuvieron resultados en 8. De estas 8 favorables, dos presentaron valores sobre 10, esto implica que un cambio en esta parametrización produce cambios importantes en los resultados de la temperatura.\\ 

Para la parametrización de cúmulos podemos observar que las frecuencias son inferiores a 10, esto puede implicar que esta parametrización no tenga un papel importante en el momento de la simulación de temperatura extrema.\\

En la parametrización de microfísica de nubes es importante resaltar que, en general, se presentaron números bajos y que no hay un gran efecto de la microfísica en la simulación de las temperaturas extremas pues el mejor resultado fue cuando esta parametrización está desactivada.\\

Aunque en las parametrizaciónes de radiación de onda larga y corta fueron evaluadas 9 combinaciones sólo se obtuvieron 4 combinaciones con buenos resultados respectivamente. Se puede observar que estas buenas combinaciones presentaron valores de frecuencia superiores a diez en 3 de 4 combinaciones, respectivamente. Esto implica que las modificaciones en esta parametrización mejora los resultados de los datos modelados de temperatura.\\


Es interesante que dentro de los mejores resultados aparezca la configuración del IDEAM-Colombia y que sea superior a la configuración IDEAM-Bogotá, esto implica que es una buena configuración y que en algunos casos está representando bien las temperaturas extremas, en comparación con la configuración IDEAM-Bogotá que también aparece en los resultados pero con valores inferiores.\\

Como resultado podemos observar que para la parametrización de la capa límite planetaria (bl\_pbl\_physics) la mejor parametrización es la 8, estas parametrizaciones corresponden a \textit{Bougeault–Lacarrere Scheme (BouLac)}. En la parametrización de cúmulos (cu\_physics) la mejor parametrización es la 2, estas parametrizaciones corresponden a parametrización \textit{Moisture–advection–based Trigger for Kain–Fritsch Cumulus Scheme}, \textit{Grell 3D Ensemble Scheme}. En la microfísica de nubes (mp\_physics) se obtuvo que la mejor es la parametrización 0, esto significa que se debe mantener desactivada la parametrización de microfísica. En la parametrización de la onda larga (ra\_lw\_physics) se observa que la mejor parametrización es la opción 7, la cual corresponde a \textit{Fu–Liou–Gu Shortwave and Longwave Schemes}. En la parametrización de la onda corta (ra\_sw\_physics) se observa que la mejor parametrización es la opción 3 \textit{CAM Shortwave and Longwave Schemes}. En la parametrización de superfície (sf\_sfclay\_physics) la única combinación que presentó buenos resultados fue la opción 91, la cual corresponde a \textit{MM5 Similarity Scheme}. Y para finalizar, la configuración del IDEAM con la configuración para Colombia presentó buenos resultados, esto quiere decir que fue una buena decisión haber tomado la configuración del IDEAM para Colombia y experimentar con ella, ya que presenta buenos resultados.\\

Posteriormente a la obtención de los resultados de las mejores opciones se realizaron dos nuevas configuraciones usando los resultados obtenidos. La primera configuración se realizó sobre la configuración del IDEAM-Colombia, junto con las mejores opciones de la Tabla \ref{tabla_mejores_param_opciones} creándose de la configuración IDEAM-Colombia-modificada (icm). Esta configuración fue evaluada con los resultados obtenidos, ver Tabla \ref{tabla_parametrizacion_opciones_2}.

\begin{table}[H]
\caption{Parametrizaciones y opciones usadas en el WRF versión 9.1.1.}
\label{tabla_parametrizacion_opciones_2}
\begin{tabular}{lp{8cm}}
Parametrización & Opciones \\ \hline
bl\_pbl\_physics & 0, 1, 2, 3, 4, 5, 6, 7, 8, 9, 10, 11, 12, 99 \\ %14
cu\_physics      & 0, 1, 2, 3, 4, 5, 6, 7, 11, 14, 16, 93, 99 \\ %13
mp\_physics      & 0, 1, 2, 3, 5, 6, 7, 8, 9, 10, 11, 13, 14, 15, 16, 19, 30, 32, 50, 51, 52 \\%21
ra\_lw\_phyisics & 0, 1, 2, 3, 4, 5, 7, 31, 99 \\%9
ra\_sw\_phyisics & 0, 1, 2, 3, 4, 5, 7, 31, 99 \\%9 
sf\_sfclay\_physics & 0, 1, 2, 3, 4, 5, 7, 10, 91\\ %9
sf\_surface\_physics & 0, 1, 2, 3, 4, 5, 7, 8, 31, 99\\ % 10
IDEAM-Colombia & \\
IDEAM-Bogotá & \\ 
icm & \\ 
icm-mp\_physics 3 &
\end{tabular}
\end{table}

La segunda configuración usó las mejores combinaciones de la Tabla \ref{tabla_mejores_param_opciones} pero para la parametrización de microfísica de nubes se uso la configuración del IDEAM que es la configuración 3 que corresponde a la WSM3 \textit{WRF Single–moment 3–class and 5–class Schemes}. Es importante dejar activada esta parametrización ya que esta es la que se encarga de resolver el vapor de agua y el proceso de precipitación, estos procesos no se estaban teniendo en cuenta con la opción 0, se espera que los resultados de esta investigación sean usados como insumo para la modelación de cultivos, por esta razón no se puede obviar el vapor de agua y el proceso de precipitación. Adicionalmente es importante resaltar que en las fechas de los casos se presentaron precipitaciones en la Sabana de Bogotá, como se puede observar en la Tabla \ref{tabla_lluvias}. \\

\begin{table}[H]
\centering
\caption{Resumen de la precipitación de 96 estaciones convencionales para los diferentes casos. Número de estaciones convencionales que presentaron lluvia para las fechas. Valor promedio de la precipitación de las estaciones que presentaron lluvias. Y valor máximo de la precipitación registrado.}
\label{tabla_lluvias}
\begin{tabular}{llp{4cm}ll}
Caso & Fecha & Número de estaciones con precipitación & Promedio & Valor máximo \\ \hline
	1 & 2007-02-03 & 0  & NaN & NaN \\
	2 & 2007-02-03 & 5  & 0.3 & 0.5 \\
	3 & 2007-02-03 & 23 & 2.5 & 13.5\\
	4 & 2007-02-03 & 19 & 1.1 & 3.8 \\

\end{tabular}
\end{table}

En la Tabla \ref{tabla_lluvias} podemos observar que para en el caso 1 no se presentaron precipitaciones en ninguna de las estaciones convencionales del área de estudio, en el caso 2 podemos observar que se presentó precipitación en 5 estaciones y la precipitación fue de menos de un milímetro, el caso 3 fue el evento en el que más estaciones reportaron precipitaciones, con el promedio más alto y presentó los valores más altos, y en el caso 4 también se presentaron precipitaciones que fueron mayores a 1 milímetro.

%Posterior a estos resultados se procedió a resolver los empates para el caso de la capa límite planetaria y para la parametrización de cúmulos. Para esto se tomaron las mejores opciones para las parametrizaciones que no presentaron empates como: mp\_physics-11, ra\_lw\_physics-7, ra\_sw\_physics-3 y sf\_sfclay\_physics-91; estas fueron modificadas dentro de las opciones de la \texttt{namelist.input} del IDEAM para Colombia, creándose de esta forma la IDEAM-Colombia-modificada (icm).\\


%Para buscar cuál es la mejor de los empates se procedió a hacer 6 modelaciones más con las siguientes combinaciones:

%\begin{itemize}
%    \item icm: bl\_pbl\_physics-5 y cu\_physics-0
%    \item icm\_pbl-5\_cu-5: bl\_pbl\_physics-5 y cu\_physics-5
%    \item icm\_pbl-5\_cu-14: bl\_pbl\_physics-5 y cu\_physics-14
%    \item icm\_pbl-7\_cu-0: bl\_pbl\_physics-7 y cu\_physics-0
%    \item icm\_pbl-7\_cu-5: bl\_pbl\_physics-7 y cu\_physics-5
%    \item icm\_pbl-7\_cu-14: bl\_pbl\_physics-7 y cu\_physics-14
%\end{itemize}

%Luego de la obtención de los resultados estas nuevas modelaciones fueron de nuevo evaluadas con todas las opciones de la Tabla \ref{tabla_parametrizacion_opciones} más las 6 nuevas simulaciones. En resumen se evaluaron 93 formas diferentes de configuración del modelo, ver Tabla \ref{tabla_parametrizacion_opciones_2}. Como resultado de este proceso se obtuvo la tabla \ref{tabla_parametrizacion_opciones_2}.\\

%Se usaron 9 estaciones para el caso 1, en el caso 2 se usaron 13 estaciones, en el caso 3 se usaron 15 estaciones y en el caso 4 se usaron 16 estaciones.\\ %% Las estaciones están en la carpeta Extraccion_dominios/estaciones_usadas_year_XXXX.csv



Como resultado de esta nueva evaluación se obtuvo la Tabla \ref{tabla_mejores_param_opciones_2}.\\
% la que mejores resultados tuvo fué una frecuencia de 12 pero ahora con la modificación la mejor es de 25
\begin{table}[H]
\caption{Frecuencia de ocurrencia de la combinación parametrización-opción mayor a 0.8 en Pearson e inferior a 0.3 en NMRSE que se encuentra entre los 5 mejores resultados. * Sólo se evaluó una opción, porque las demás configuraciones no generaron resultados en el WRF.}
\label{tabla_mejores_param_opciones_2}
\centering
\begin{tabular}{lr}
\toprule
Combinación parametrización-opción               & Frecuencia       \\
\midrule

bl\_pbl\_physics-5     &      3 \\
bl\_pbl\_physics-6     &      1 \\
bl\_pbl\_physics-7     &      4 \\
bl\_pbl\_physics-8     &      6 \\
bl\_pbl\_physics-9     &      4 \\
bl\_pbl\_physics-12    &      7 \\
bl\_pbl\_physics-99    &      4 \\ \hline
cu\_physics-2         &      4 \\
cu\_physics-5         &      2 \\
cu\_physics-6         &      4 \\
cu\_physics-11        &      1 \\
cu\_physics-14        &      2 \\
cu\_physics-16        &      3 \\
cu\_physics-99        &      2 \\ \hline
mp\_physics-0         &      1 \\
mp\_physics-2         &      1 \\
mp\_physics-11        &      2 \\
mp\_physics-13        &      2 \\ \hline
ra\_lw\_physics-3      &      3 \\
ra\_lw\_physics-5      &      1 \\
ra\_lw\_physics-7      &      6 \\
ra\_lw\_physics-31     &      6 \\ \hline
ra\_sw\_physics-2      &      5 \\
ra\_sw\_physics-3      &      6 \\
ra\_sw\_physics-4      &      5 \\
ra\_sw\_physics-5      &      4 \\ \hline
*sf\_sfclay\_physics-91 &      9 \\ \hline
sf\_surface\_physics-1 &      1 \\ \hline
ideam-Bogotá         &      1 \\
ideam-colombia       &      2 \\
ideam-icm            &     24 \\
ideam-icm\_3          &     19 \\


\bottomrule
\end{tabular}
\end{table}

%icm ideam-mejor
%icm\_pbl-5\_cu-5 ideam-cu\_5
%icm\_pbl-5\_cu-14 ideam-cu\_14\_pbl\_5
%icm\_pbl-7\_cu-0 ideam-mejor7?? 
%icm\_pbl-7\_cu-5 cu\_physics\_5\_5
%icm\_pbl-7\_cu-14 ideam-cu\_14  cu_physics_14


Como se puede observar en la Tabla \ref{tabla_mejores_param_opciones_2}, la nueva configuración usada tiene la más alta frecuencia entre todas las combinaciones parametrización-opción. Cuando se comparan los resultados de la Tabla \ref{tabla_mejores_param_opciones} con la Tabla \ref{tabla_mejores_param_opciones_2}, se puede observar que efectivamente hay una respuesta al cambio de las parametrizaciones, ya que para algunas combinaciones se disminuye la frecuencia, porque la nueva combinación es contada en los 5 mejores resultados por estación.\\

Para el caso de simulaciones de temperaturas extremas la mejor combinación de las parametrizaciones corresponde a la configuración icm cuya configuración del \textit{namelist.input} está en la Figura \ref{imag_tab_ej3}.\\

\begin{figure}[H]
\center
\texttt{
...\\
\&physics\\
mp\_physics~~~~~~~~~~~~~~~=~0,~0,\\
ra\_lw\_physics~~~~~~~~~~~~=~7,~~7,\\
ra\_sw\_physics~~~~~~~~~~~~=~3,~~3,\\
%radt~~~~~~~~~~~~~~~~~~~~~=~30,~30,\\
sf\_sfclay\_physics~~~~~~~~=~91,~91,\\
sf\_surface\_physics~~~~~~~=~1,~~1,\\
bl\_pbl\_physics~~~~~~~~~~~=~8,~~8,\\
bldt~~~~~~~~~~~~~~~~~~~~~~=~1,~~1,\\
cu\_physics~~~~~~~~~~~~~~~~=~2,~~2,\\
...\\
}
\centering
\includegraphics[draft=false, scale=0.0018]{ideam_ej_1.png}
\caption{Sección del \texttt{namelist.input} para la combinación icm.}
\label{imag_tab_ej3}


\end{figure}


\section{Resultados}
%Como conclusión podemos observar que los cambios en las parametrizaciones del WRF no resulta acumulativo, es decir que si se encuentran las mejores opciones para las diferentes parametrizaciones no garantiza que esto se vea reflejado en un mejor resultado.\\

%La configuración usada por el IDEAM para ofrece buenos resultados para la representación de las altas y bajas temperaturas.\\

%El modelo WRF tienen dificultades para representar temperaturas bajo 0\celsius, pero presenta buenos resultados cuando se trabaja con altas temperaturas. El modelo WRF con las configuraciones halladas no es capaz de representar una helada, pero si es capaz de representar un fenómeno de altas temperaturas.\\

El método usado para la selección de la mejor parametrización-opción resultó de utilidad, ya que se obtuvo la mayor frecuencia para la configuración icm. La mejor configuración evaluada para la representación de las altas y bajas temperaturas corresponde a la combinación icm.\\

Las respuestas del modelo se vieron influenciadas por el cambio de cada parametrizaciones. Es difícil determinar en qué proporción se mejoran los resultados dependiendo de la parametrización, pero sí se puede decir que para el caso de las temperaturas extremas las parametrizaciones más importantes son las de radiación, suelo y capa límite planetaria.\\


\end{itemize} % Es usado para los item de los casos OJO Dejar
%\end{comment}

%\begin{comment}

\section{Búsqueda del mejor tiempo para iniciar las simulaciones}

El modelo WRF toma un tiempo para la calibración y los ajustes de las condiciones iniciales y frontera \citep{Jimenez2014}. \citet{Skamarock2008} reconoce que se debe dejar un tiempo de 6 horas antes de la hora objetivo, ya que este tiempo permite realizar un ajuste a la topografía. \citet{cortes2010} resalta que los modelos de mesoescala presentan un problema con el arranque inicial el cual está relacionado con el proceso de iniciar los movimientos verticales y las circulaciones divergentes del modelo, por esta razón \citet{cortes2010} descartó las primeras 12 horas de simulación. \citet{Hu2010} eliminó las primeras 12 horas de las simulaciones el cual le pertmitió obtener mejores resultados. \citet{Arango2011} establecieron que un \textit{spin-up} de 6 horas es muy bajo para la determinación de la temperatura, por esta razón proponen un \textit{spin-up} de 10 a 15 horas. \citet{Draxl2007} simularon vientos para un período de tiempo de un mes y usó un \textit{spin-up} de 48 horas. \citet{Corrales2015} no tuvo en cuenta el período de \textit{spin-up} para evaluar las temperaturas en México.\\

Con la finalidad de buscar el mejor tiempo en el cuál se deben iniciar las simulaciones antes del evento se planteó la evaluación de diferentes periodos de tiempos de inicio antes del fenómeno de estudio (ver Tabla \ref{tab_horas_inicio}). Los tiempos de finalización fueron los mismos para cada caso (ver Tabla \ref{tabla_fechas_finalizacion}).\\ 

\begin{table}[h]
\caption{Simulaciones y horas de inicio antes del fenómeno.}
\label{tab_horas_inicio}
\centering
\begin{tabular}{ll}
Simulación & Horas antes del evento \\ \hline
1 & 0 \\
2 & 6 \\
3 & 12 \\
4 & 18 \\
5 & 24 \\
6 & 30 \\
7 & 36 \\
8 & 42 \\
9 & 48 \\
10 & 54 \\
11 & 60 \\
12 & 66 \\
13 & 72 \\
14 & 78 \\
15 & 84 \\
16 & 108
\end{tabular}
\end{table}

Se tomaron los cuatro casos y fueron configurados con la combinación icm e icm-mp\_physics 3, las fechas y las horas de inicio se encuentran en la Tabla \ref{tabla_fechas_inicio}. Los datos usados para proveer condiciones iniciales y de frontera corresponden a los datos de análisis del GFS los cuales están disponibles cada 6 horas. La fecha y hora de finalización no cambió, las fechas se encuentran en la Tabla \ref{tabla_fechas_finalizacion}.

\begin{table}[H]
\caption{Fechas de inicio para los 4 diferentes casos y las simulaciones realizadas}
\label{tabla_fechas_inicio}
\begin{tabular}{lllll}
 Simulación & Caso 1 & Caso 2 & Caso3 & Caso 4 \\ \hline
 1 & 2007-02-04 06:00 & 2014-08-30 06:00 & 2015-08-27  06:00 & 2015-09-08  06:00\\
 2 & 2007-02-04 00:00 & 2014-08-30 00:00 & 2015-08-27  00:00 & 2015-09-08  00:00\\
 3 & 2007-02-03 18:00 & 2014-08-29 18:00 & 2015-08-26  18:00 & 2015-09-07  18:00\\
 4 & 2007-02-03 12:00 & 2014-08-29 12:00 & 2015-08-26  12:00 & 2015-09-07  12:00\\
 5 & 2007-02-03 06:00 & 2014-08-29 06:00 & 2015-08-26  06:00 & 2015-09-07  06:00\\
 6 & 2007-02-03 00:00 & 2014-08-29 00:00 & 2015-08-26  00:00 & 2015-09-07  00:00\\
 7 & 2007-02-02 18:00 & 2014-08-28 18:00 & 2015-08-25  18:00 & 2015-09-06  18:00\\
 8 & 2007-02-02 12:00 & 2014-08-28 12:00 & 2015-08-25  12:00 & 2015-09-06  12:00\\
 9 & 2007-02-02 06:00 & 2014-08-28 06:00 & 2015-08-25  06:00 & 2015-09-06  06:00\\
10 & 2007-02-02 00:00 & 2014-08-28 00:00 & 2015-08-25  00:00 & 2015-09-06  00:00\\
11 & 2007-02-01 18:00 & 2014-08-27 18:00 & 2015-08-24  18:00 & 2015-09-05  18:00\\
12 & 2007-02-01 12:00 & 2014-08-27 12:00 & 2015-08-24  12:00 & 2015-09-05  12:00\\
13 & 2007-02-01 06:00 & 2014-08-27 06:00 & 2015-08-24  06:00 & 2015-09-05  06:00\\
14 & 2007-02-01 00:00 & 2014-08-27 00:00 & 2015-08-24  00:00 & 2015-09-05  00:00\\
15 & 2007-01-31 18:00 & 2014-08-26 18:00 & 2015-08-23  18:00 & 2015-09-04  18:00\\
16 & 2007-01-30 18:00 & 2014-08-25 18:00 & 2015-08-22  18:00 & 2015-09-03  18:00\\
\end{tabular}
\end{table}

\begin{table}[H]
\caption{Fechas de finalización para los 4 diferentes casos}
\label{tabla_fechas_finalizacion}
\begin{tabular}{lllll}
  & Caso 1 & Caso 2 & Caso3 & Caso 4 \\ \hline
 Fin & 2007-02-05 00:00 & 2014-09-01 00:00 & 2015-08-30 00:00 & 2015-09-10 00:00 \\
 
\end{tabular}
\end{table}



Como se vio en anteriores secciones (Caracterización de las heladas y altas temperaturas (Sección \ref{area_caracterizacion_heladas_extremas}), las horas más probables de ocurrencia de una helada son entre la 1 am y las 6 am y las horas más probables para que se presenten altas temperaturas son entre las 11 am y las 2 pm. Por esta razón las modelaciones tomaron como la hora 0 las 6 horas UTC del día en el que se presentó el evento, esto equivale a inicial la modelación a la 1 a.m. (hora local ), la cual corresponde a la simulación 1, ver Tabla \ref{tab_horas_inicio}. Las diferentes horas y el tiempo de cómputo usado para estos cálculos se encuentran en la Tabla \ref{tabla:tiempos_inicializacion}.\\


\begin{table}[]
\centering
\caption{Simulaciones, tiempo entre el caso de estudio y el inicio de la simulación (Tiempo (h)) y tiempo usado para computar cada una de las simulaciones (Tiempo de computo horas).}
\label{tabla:tiempos_inicializacion}
\begin{tabular}{llll}
Simulación    & Tiempo (h)    & Tiempo de computo (horas) \\ \hline
  1   &  0            &     2           \\
  2   &  6            &     3.33           \\
  3   &  12           &     3           \\
  4   &  18           &     4           \\ 
  5   &  24           &     4           \\ 
  6   &  30           &     5           \\ 
  7   &  36           &     6.3          \\ 
  8  &  42           &     6.5          \\ 
 9   &  48           &     7.33          \\
 10   &  60           &     9          \\
  11  &  54           &     7          \\
   12 & 66           &     8.5          \\
   13 &  72         &     9          \\
  14  &  78         &     9.5          \\
  15  &  84           &     10           \\
  16  &  108          &     13           \\


            


 
\end{tabular}
\end{table}

Para encontrar las mejores combinaciones se usó la misma metodología que se usó para escoger la parametrización-opción en la sección anterior (Sección \ref{busqueda_mejores_param}). Para la selección de la mejor simulación, se escogieron las combinaciones que tuvieron valores superiores a 0.8 en el coeficiente de correlación de Pearson, valores inferiores a 0.3 en la columna $NMRSE$ y que estuviera entre los 5 mejores resultados de la columna $ET$ por estación Tabla \ref{tabla:resultado_tiempo}.\\

% (ver Anexo \ref{anexo:resultados_comparaciones_estadisticos_tiempos}).\\

\begin{table}[H]
    \centering
    \caption{Frecuencia de ocurrencia de la mejor simulación.}
    \label{tabla:resultado_tiempo}
\begin{tabular}{rlrr}
\toprule
 Simulación &  Frecuencia \\
\midrule
Simulación 1  &     41 \\
Simulación 2  &     39 \\
Simulación 3  &     45 \\
Simulación 4  &     22 \\
Simulación 5  &     13 \\
Simulación 6  &      8 \\
Simulación 7  &     16 \\
Simulación 8  &      6 \\
Simulación 9  &      4 \\
Simulación 10 &      2 \\
Simulación 11 &     12 \\
Simulación 12 &      7 \\
Simulación 13 &      5 \\
Simulación 15 &      4 \\
Simulación 16 &      4 \\

%Simulación 1  &     43 \\
%Simulación 2  &     33 \\
%Simulación 3  &     41 \\
%Simulación 4  &     18 \\
%Simulación 5  &     14 \\
%Simulación 6  &     11 \\
%Simulación 7  &     15 \\
%Simulación 8  &      8 \\
%Simulación 9  &      5 \\
%Simulación 10 &      2 \\
%Simulación 11 &     10 \\
%Simulación 12 &     14 \\
%Simulación 13 &      5 \\
%Simulación 14 &      1 \\
%Simulación 15 &      6 \\
%Simulación 16 &     11 \\
\bottomrule
\end{tabular}
\end{table}
%\justify

Es importante aclarar que en la tabla \ref{tabla:resultado_tiempo} se esperaría que los valores de frecuencia fueran igual a el número de estaciones por cada caso multiplicado por dos, ya que se evaluaron dos combinaciones y multiplicado por 5 porque se escogieron las 5 mejores para cada estación. Pero esto no sucede, ya que cuando se seleccionaron las 5 mejores por estación, hubo casos en los que hubo menos de 5, como se muestra en la Tabla \ref{tab_estaciones_5_tiempo}, la cual muestra la frecuencia de las 5 mejores para la configuración icm. Un ejemplo de este comportamiento se puede observar en el Caso 2 para la estación Ideam Bogotá  la cual tiene una frecuencia de 0, esto implica que no hubo ninguna Simulación que cumpliera los filtros, en otras palabras ninguna de las simulaciones se ajustó lo suficiente a los datos de la estación automática.\\

\begin{longtable}{lll}

\caption{Conteo de las 5 mejores simulaciones para las estaciones. La simulación ideal tendría un valor de frecuencia igual a 5 que sería la máxima frecuencia que se puede reportar por cada estación.}
\label{tab_estaciones_5_tiempo}\\
\hline
Caso & Estación & Valores \\

\midrule
\endhead
\midrule
\multicolumn{3}{r}{{Continúa en la siguiente página.}} \\
\midrule
\endfoot

\bottomrule
\endlastfoot


Caso 1 & La Capilla Autom  & 5\\
Caso 1 & Ciudad Bolívar  & 5\\
Caso 1 & Hda Sta Ana Autom  & 5\\
Caso 1 & La Boyera Automática  & 5\\
Caso 1 & Chinavita Automática  & 5\\
Caso 1 & Subia Automática  & 5\\
Caso 1 & Tibaitatá Automática  & 5\\
Caso 1 & Sta Cruz De Siecha  & 5\\
Caso 1 & Pmo Guacheneque  & 5\\
Caso 2 & Hda Sta Ana Autom  & 5\\
Caso 2 & Esc La Unión Automática & 5\\
Caso 2 & Apto El Dorado  & 5\\
Caso 2 & La Capilla Autom  & 5\\
Caso 2 & Chinavita Automática  & 5\\
Caso 2 & La Boyera Automática  & 5\\
Caso 2 & Sta Cruz De Siecha  & 0\\
Caso 2 & Univ Nacional  & 5\\
Caso 2 & Bosque Intervenido    & 5\\
Caso 2 & Pmo Guerrero  & 5\\
Caso 2 & Pnn Chingaza Autom   & 5\\
Caso 2 & Pmo Guacheneque  & 5\\
Caso 2 & Ideam Bogotá  & 0\\
Caso 2 & San Cayetano Autom   & 0\\
Caso 3 & Tibaitatá Automática  & 5\\
Caso 3 & Sta Cruz De Siecha  & 5\\
Caso 3 & Hda Sta Ana Autom  & 5\\
Caso 3 & La Capilla Autom  & 5\\
Caso 3 & Chinavita Automática  & 5\\
Caso 3 & Univ Nacional  & 5\\
Caso 3 & Pmo Chingaza  & 0\\
Caso 3 & Apto El Dorado  & 5\\
Caso 3 & Nueva Generacion  & 5\\
Caso 3 & Esc La Unión Automática & 5\\
Caso 3 & Ideam Bogotá  & 0\\
Caso 3 & Pmo Rabanal Automática   & 0\\
Caso 3 & Villa Teresa Automática  & 5\\
Caso 3 & Pmo Guerrero  & 5\\
Caso 3 & San Cayetano Autom   & 5\\
Caso 4 & La Capilla Autom  & 5\\
Caso 4 & Tibaitatá Automática  & 5\\
Caso 4 & Bosque Intervenido    & 2\\
Caso 4 & Chinavita Automática  & 5\\
Caso 4 & Pmo Rabanal Automática   & 5\\
Caso 4 & Sta Cruz De Siecha  & 5\\
Caso 4 & Apto El Dorado  & 5\\
Caso 4 & Ideam Bogotá  & 0\\
Caso 4 & San Cayetano Autom   & 5\\
Caso 4 & Nueva Generacion  & 5\\
Caso 4 & Villa Teresa Automática  & 5\\
Caso 4 & Pmo Chingaza  & 0\\
Caso 4 & Univ Nacional  & 5\\
Caso 4 & Pmo Guerrero  & 5\\


\end{longtable}


Basado en los resultados de la Tabla \ref{tabla:resultado_tiempo} se puede observar que la simulación 3 presenta los mejores resultados ya que tienen una frecuencia de 45. Estas simulación corresponde la hora de inicio 18 UTC del día anterior al evento.\\

%Adicionalmente, se realizaron los diagramas de Taylor para la estación Tibaitatá en los diferentes casos (Figura \ref{fig:taylor_tiempo}) las demás gráficas se encuentran en el Anexo \ref{anexo:graficas_taylor_tiempos_4casos}.
%
%\begin{figure}[H]
%    \centering
%    
%\begin{subfigure}[normla]{0.4\textwidth}
%\includegraphics[draft=false, scale=0.25]{graficas_taylor_tiempo_casos/taylor_20070221206990.png}
%\caption{Estación Tibaitatá Automática  código 21206990 caso 1.}
%\end{subfigure}
%~
%\begin{subfigure}[normla]{0.4\textwidth}
%\includegraphics[draft=false, scale=0.25]{graficas_taylor_tiempo_casos/taylor_20140821206990.png}
%\caption{Estación Tibaitatá Automática  código 21206990 caso 2.}
%\end{subfigure}
%~
%\begin{subfigure}[normla]{0.4\textwidth}
%\includegraphics[draft=false, scale=0.25]{graficas_taylor_tiempo_casos/taylor_20150821206990.png}
%\caption{Estación Tibaitatá Automática  código 21206990 caso 3.}
%\end{subfigure}
%~
%\begin{subfigure}[normla]{0.4\textwidth}
%\includegraphics[draft=false, scale=0.25]{graficas_taylor_tiempo_casos/taylor_20150921206990.png}
%\caption{Estación Tibaitatá Automática  código 21206990 caso 4.}
%\end{subfigure}
%    
%    \caption{Diagramas de Taylor para la estación Tibaitatá para los cuatro casos evaluados.}
%    \label{fig:taylor_tiempo}
%\end{figure}
%
%En la Figura \ref{fig:taylor_tiempo} se observa una buena dispersión de los datos. En estas gráficas podemos observar como los números pequeños se encuentran más cerca a los datos reales, esto quiere decir que los resultados resultan mejores si el inicio de la modelación se encuentra más cerca a la fecha de interés, ya que como lo nombra \citet{Kovacik2007} los modelos necesitan cierto tiempo para lograr un equilibrio térmico e hidrológico entre el suelo y la atmósfera. Por esta razón teniendo en cuenta la bibliografía citada el mejor tiempo para iniciar el modelo es 12 horas antes del fenómeno lo que equivale a la simulación 3.

\section{Conclusiones}

La configuración icm presenta los mejores resultados para realizar simulaciones de altas y bajas temperaturas. Configuración icm:\\

\begin{figure}[H]
\centering
\texttt{
...\\
\&physics\\
mp\_physics~~~~~~~~~~~~~~~=~0,~0,\\
ra\_lw\_physics~~~~~~~~~~~~=~7,~~7,\\
ra\_sw\_physics~~~~~~~~~~~~=~3,~~3,\\
%radt~~~~~~~~~~~~~~~~~~~~~=~30,~30,\\
sf\_sfclay\_physics~~~~~~~~=~91,~91,\\
sf\_surface\_physics~~~~~~~=~1,~~1,\\
bl\_pbl\_physics~~~~~~~~~~~=~8,~~8,\\
bldt~~~~~~~~~~~~~~~~~~~~~~=~1,~~1,\\
cu\_physics~~~~~~~~~~~~~~~~=~2,~~2,\\
...\\
}
\centering
\includegraphics[draft=false, scale=0.0018]{ideam_ej_1.png}
\caption{Sección del \texttt{namelist.input} para la combinación icm.}
\label{imag_tab_ej3}
\end{figure}

La simulación 3 que inicia 12 horas antes del evento es la mejor opción para iniciar el modelo ya que se debe tener un tiempo en el que se debe estabilizar para posteriormente generar mejores resultados.

%\end{comment}
