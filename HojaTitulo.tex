%\newpage
%\setcounter{page}{1}
%\begin{center}
%\begin{figure}
%\centering%
%\epsfig{file=HojaTitulo/EscudoUN.eps,scale=1}%
%\end{figure}
%\thispagestyle{empty} \vspace*{2.0cm} \textbf{\huge
%T\'{\i}tulo de la tesis  o trabajo de investigaci\'{o}n}\\[6.0cm]
%\Large\textbf{Nombres y apellidos completos del autor}\\[6.0cm]
%\small Universidad Nacional de Colombia\\
%Facultad, Departamento (Escuela, etc.)\\
%Ciudad, Colombia\\
%A\~{n}o\\
%\end{center}

\newpage{\pagestyle{empty}\cleardoublepage}

\newpage
\begin{center}
%\thispagestyle{empty} \vspace*{0cm} \textbf{\huge Efectos potenciales de Temperaturas Extremas del Aire en la Sabana de Bogotá Sobre la Producción de un Cultivo de Importancia para la Seguridad Alimentaria: Un estudio con un Modelo Regional de Pronóstico del Tiempo Atmosférico y Un Modelo de Simulación de Cultivo. }\\[3.0cm]
\thispagestyle{empty} \vspace*{0cm} \textbf{\huge Cracterización de las temperaturas extremas del aire para el cultivo de papa en la Sabana de Bogotá}\\[3.0cm]
\Large\textbf{Edwin Torres Moya}\\[3.0cm]
%\small Tesis o trabajo de grado presentada(o) como requisito parcial para optar al
%t\'{\i}tulo de:\\
%\textbf{Indicar el t\'{\i}tulo que se obtendr\'{a}. Por ejemplo, Magister en Ingenier\'{\i}a Qu\'{\i}mica}\\[2.5cm]
%Directores:\\
%Ph.D., Astrid Baquero Bernal y M.Sc Eliecer Díaz Almanza\\[2.0cm]
Directora:\\
Ph.D., Astrid Baquero Bernal\\[2.0cm]
%L\'{\i}nea de Investigaci\'{o}n:\\
%Nombrar la l\'{\i}nea de investigaci\'{o}n en la que enmarca la tesis  o trabajo de investigaci\'{o}n\\
Grupo de Investigaci\'{o}n:\\
Grupo de simulación del sistema climático terrestre\\[2.5cm]
Universidad Nacional de Colombia\\
Facultad de Ciencias, Meteorología\\
Bogotá, Colombia\\
A\~{n}o 2018\\
\end{center}
