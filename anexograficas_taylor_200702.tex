\chapter{Gráficas de los diagramas de Taylor de las diferentes combinaciones para el 2007.}
\label{anexograficas_taylor_200702}

\newpage
\begin{figure}[H]
    
\begin{subfigure}[normla]{0.5\textwidth}
\caption{Diagrama de Taylor para la Estación Subia Automatica código 21195160.}
\includegraphics[draft=false, scale=0.5]{../taylor_simulaciones/200702_m/taylor_21195160.png}
\end{subfigure}
~
\begin{subfigure}[normla]{0.5\textwidth}
\caption{Diagrama de Taylor para la Estación Hda Sta Ana Autom código 21206790.}
\includegraphics[draft=false, scale=0.5]{../taylor_simulaciones/200702_m/taylor_21206790.png}
\end{subfigure}
~
\begin{subfigure}[normla]{0.5\textwidth}
\caption{Diagrama de Taylor para la Estación Pmo Guerrero código 21206930.}
\includegraphics[draft=false, scale=0.5]{../taylor_simulaciones/200702_m/taylor_21206930.png}
\end{subfigure}
~
\begin{subfigure}[normla]{0.5\textwidth}
\caption{Diagrama de Taylor para la Estación Ciudad Bolivar código 21206940.}
\includegraphics[draft=false, scale=0.5]{../taylor_simulaciones/200702_m/taylor_21206940.png}
\end{subfigure}
~
\begin{subfigure}[normla]{0.5\textwidth}
\caption{Diagrama de Taylor para la Estación Pmo Guacheneque código 21206950.}
\includegraphics[draft=false, scale=0.5]{../taylor_simulaciones/200702_m/taylor_21206950.png}
\end{subfigure}
~
\begin{subfigure}[normla]{0.5\textwidth}
\caption{Diagrama de Taylor para la Estación Sta Cruz De Siecha código 21206980.}
\includegraphics[draft=false, scale=0.5]{../taylor_simulaciones/200702_m/taylor_21206980.png}
\end{subfigure}
~
\end{figure}
           
\begin{figure}[H]\ContinuedFloat
\begin{subfigure}[normla]{0.5\textwidth}
\caption{Diagrama de Taylor para la Estación Tibaitata Automatica código 21206990.}
\includegraphics[draft=false, scale=0.5]{../taylor_simulaciones/200702_m/taylor_21206990.png}
\end{subfigure}
~
\begin{subfigure}[normla]{0.5\textwidth}
\caption{Diagrama de Taylor para la Estación La Boyera Automatica código 24015110.}
\includegraphics[draft=false, scale=0.5]{../taylor_simulaciones/200702_m/taylor_24015110.png}
\end{subfigure}
~
\begin{subfigure}[normla]{0.5\textwidth}
\caption{Diagrama de Taylor para la Estación Chinavita Automatica código 35075070.}
\includegraphics[draft=false, scale=0.5]{../taylor_simulaciones/200702_m/taylor_35075070.png}
\end{subfigure}
~
\begin{subfigure}[normla]{0.5\textwidth}
\caption{Diagrama de Taylor para la Estación La Capilla Autom código 35085080.}
\includegraphics[draft=false, scale=0.5]{../taylor_simulaciones/200702_m/taylor_35085080.png}
\end{subfigure}
~

    
    \caption{Caso 1 }
    \label{fig:my_label}
\end{figure}


    

%%%%%%%%%%%%%%%%%%%%%%%%%%%%%%%%%%%%%%%%%%%%%%%%%%%%%%%%%
%%%%%%%%%%%%%%%%%%%%%%%%%%%%%%%%%%%%%%%%%%%%%%%%%%%%%%%%%
%%%%%%%%%%%%%%%%%%%%%%%%%%%%%%%%%%%%%%%%%%%%%%%%%%%%%%%%%
%%%%%%%%%%%%%%%%%%%%%%%%%%%%%%%%%%%%%%%%%%%%%%%%%%%%%%%%%\\



\begin{figure}[H]
\begin{subfigure}[normla]{0.5\textwidth}
\caption{Diagrama de Taylor para la Estación Esc La Union Automaticacódigo 21201200.}
\includegraphics[draft=false, scale=0.5]{../taylor_simulaciones/201408_m/taylor_21201200.png}
\end{subfigure}
~
\begin{subfigure}[normla]{0.5\textwidth}
\caption{Diagrama de Taylor para la Estación Univ Nacional código 21205012.}
\includegraphics[draft=false, scale=0.5]{../taylor_simulaciones/201408_m/taylor_21205012.png}
\end{subfigure}
~
\begin{subfigure}[normla]{0.5\textwidth}
\caption{Diagrama de Taylor para la Estación Apto El Dorado código 21205791.}
\includegraphics[draft=false, scale=0.5]{../taylor_simulaciones/201408_m/taylor_21205791.png}
\end{subfigure}
~
\begin{subfigure}[normla]{0.5\textwidth}
\caption{Diagrama de Taylor para la Estación Hda Sta Ana Autom código 21206790.}
\includegraphics[draft=false, scale=0.5]{../taylor_simulaciones/201408_m/taylor_21206790.png}
\end{subfigure}
~
\begin{subfigure}[normla]{0.5\textwidth}
\caption{Diagrama de Taylor para la Estación Pmo Guerrero código 21206930.}
\includegraphics[draft=false, scale=0.5]{../taylor_simulaciones/201408_m/taylor_21206930.png}
\end{subfigure}
~
\begin{subfigure}[normla]{0.5\textwidth}
\caption{Diagrama de Taylor para la Estación Pmo Guacheneque código 21206950.}
\includegraphics[draft=false, scale=0.5]{../taylor_simulaciones/201408_m/taylor_21206950.png}
\end{subfigure}
~
\end{figure}
           
\begin{figure}[H]\ContinuedFloat
\begin{subfigure}[normla]{0.5\textwidth}
\caption{Diagrama de Taylor para la Estación Ideam Bogota código 21206960.}
\includegraphics[draft=false, scale=0.5]{../taylor_simulaciones/201408_m/taylor_21206960.png}
\end{subfigure}
~
\begin{subfigure}[normla]{0.5\textwidth}
\caption{Diagrama de Taylor para la Estación Sta Cruz De Siecha código 21206980.}
\includegraphics[draft=false, scale=0.5]{../taylor_simulaciones/201408_m/taylor_21206980.png}
\end{subfigure}
~
\begin{subfigure}[normla]{0.5\textwidth}
\caption{Diagrama de Taylor para la Estación Tibaitata Automatica código 21206990.}
\includegraphics[draft=false, scale=0.5]{../taylor_simulaciones/201408_m/taylor_21206990.png}
\end{subfigure}
~
\begin{subfigure}[normla]{0.5\textwidth}
\caption{Diagrama de Taylor para la Estación San Cayetano Autom  código 23125170.}
\includegraphics[draft=false, scale=0.5]{../taylor_simulaciones/201408_m/taylor_23125170.png}
\end{subfigure}
~
\begin{subfigure}[normla]{0.5\textwidth}
\caption{Diagrama de Taylor para la Estación La Boyera Automatica código 24015110.}
\includegraphics[draft=false, scale=0.5]{../taylor_simulaciones/201408_m/taylor_24015110.png}
\end{subfigure}
~
\begin{subfigure}[normla]{0.5\textwidth}
\caption{Diagrama de Taylor para la Estación Pnn Chingaza Autom  código 35025080.}
\includegraphics[draft=false, scale=0.5]{../taylor_simulaciones/201408_m/taylor_35025080.png}
\end{subfigure}
~
\end{figure}
           
\begin{figure}[H]\ContinuedFloat
\begin{subfigure}[normla]{0.5\textwidth}
\caption{Diagrama de Taylor para la Estación Bosque Intervenido   código 35025090.}
\includegraphics[draft=false, scale=0.5]{../taylor_simulaciones/201408_m/taylor_35025090.png}
\end{subfigure}
~
\begin{subfigure}[normla]{0.5\textwidth}
\caption{Diagrama de Taylor para la Estación Chinavita Automatica código 35075070.}
\includegraphics[draft=false, scale=0.5]{../taylor_simulaciones/201408_m/taylor_35075070.png}
\end{subfigure}
~
\begin{subfigure}[normla]{0.5\textwidth}
\caption{Diagrama de Taylor para la Estación La Capilla Autom código 35085080.}
\includegraphics[draft=false, scale=0.5]{../taylor_simulaciones/201408_m/taylor_35085080.png}
\end{subfigure}


         
\caption{Diagramas de Taylor para el Caso 2}
\label{caso2}
\end{figure}
Para la figura de Sta Cruz de Siecha 21206980 presentó valores negativos de R2. Debido a que quedaron pocos valores de la estación automática luego de la corrección.





%
%
%



\begin{figure}
\begin{subfigure}[normla]{0.5\textwidth}
\caption{Diagrama de Taylor para la Estación Esc La Union Automaticacódigo 21201200.}
\includegraphics[draft=false, scale=0.5]{../taylor_simulaciones/201508_m/taylor_21201200.png}
\end{subfigure}
~
\begin{subfigure}[normla]{0.5\textwidth}
\caption{Diagrama de Taylor para la Estación Univ Nacional código 21205012.}
\includegraphics[draft=false, scale=0.5]{../taylor_simulaciones/201508_m/taylor_21205012.png}
\end{subfigure}
~
\begin{subfigure}[normla]{0.5\textwidth}
\caption{Diagrama de Taylor para la Estación Apto El Dorado código 21205791.}
\includegraphics[draft=false, scale=0.5]{../taylor_simulaciones/201508_m/taylor_21205791.png}
\end{subfigure}
~
\begin{subfigure}[normla]{0.5\textwidth}
\caption{Diagrama de Taylor para la Estación Nueva Generacion código 21206600.}
\includegraphics[draft=false, scale=0.5]{../taylor_simulaciones/201508_m/taylor_21206600.png}
\end{subfigure}
~
\begin{subfigure}[normla]{0.5\textwidth}
\caption{Diagrama de Taylor para la Estación Hda Sta Ana Autom código 21206790.}
\includegraphics[draft=false, scale=0.5]{../taylor_simulaciones/201508_m/taylor_21206790.png}
\end{subfigure}
~
\begin{subfigure}[normla]{0.5\textwidth}
\caption{Diagrama de Taylor para la Estación Villa Teresa Automatica código 21206920.}
\includegraphics[draft=false, scale=0.5]{../taylor_simulaciones/201508_m/taylor_21206920.png}
\end{subfigure}
~
\end{figure}
           
\begin{figure}[H]\ContinuedFloat
\begin{subfigure}[normla]{0.5\textwidth}
\caption{Diagrama de Taylor para la Estación Pmo Guerrero código 21206930.}
\includegraphics[draft=false, scale=0.5]{../taylor_simulaciones/201508_m/taylor_21206930.png}
\end{subfigure}
~
\begin{subfigure}[normla]{0.5\textwidth}
\caption{Diagrama de Taylor para la Estación Ideam Bogota código 21206960.}
\includegraphics[draft=false, scale=0.5]{../taylor_simulaciones/201508_m/taylor_21206960.png}
\end{subfigure}
~
\begin{subfigure}[normla]{0.5\textwidth}
\caption{Diagrama de Taylor para la Estación Sta Cruz De Siecha código 21206980.}
\includegraphics[draft=false, scale=0.5]{../taylor_simulaciones/201508_m/taylor_21206980.png}
\end{subfigure}
~
\begin{subfigure}[normla]{0.5\textwidth}
\caption{Diagrama de Taylor para la Estación Tibaitata Automatica código 21206990.}
\includegraphics[draft=false, scale=0.5]{../taylor_simulaciones/201508_m/taylor_21206990.png}
\end{subfigure}
~
\begin{subfigure}[normla]{0.5\textwidth}
\caption{Diagrama de Taylor para la Estación San Cayetano Autom  código 23125170.}
\includegraphics[draft=false, scale=0.5]{../taylor_simulaciones/201508_m/taylor_23125170.png}
\end{subfigure}
~
\begin{subfigure}[normla]{0.5\textwidth}
\caption{Diagrama de Taylor para la Estación Bosque Intervenido   código 35025090.}
\includegraphics[draft=false, scale=0.5]{../taylor_simulaciones/201508_m/taylor_35025090.png}
\end{subfigure}
~
\end{figure}
           
\begin{figure}[H]\ContinuedFloat
\begin{subfigure}[normla]{0.5\textwidth}
\caption{Diagrama de Taylor para la Estación Pmo Chingaza código 35035130.}
\includegraphics[draft=false, scale=0.5]{../taylor_simulaciones/201508_m/taylor_35035130.png}
\end{subfigure}
~
\begin{subfigure}[normla]{0.5\textwidth}
\caption{Diagrama de Taylor para la Estación Chinavita Automatica código 35075070.}
\includegraphics[draft=false, scale=0.5]{../taylor_simulaciones/201508_m/taylor_35075070.png}
\end{subfigure}
~
\begin{subfigure}[normla]{0.5\textwidth}
\caption{Diagrama de Taylor para la Estación Pmo Rabanal Automatica  código 35075080.}
\includegraphics[draft=false, scale=0.5]{../taylor_simulaciones/201508_m/taylor_35075080.png}
\end{subfigure}
~
\begin{subfigure}[normla]{0.5\textwidth}
\caption{Diagrama de Taylor para la Estación La Capilla Autom código 35085080.}
\includegraphics[draft=false, scale=0.5]{../taylor_simulaciones/201508_m/taylor_35085080.png}
\end{subfigure}
~


\caption{Diagramas de Taylor para el Caso 3}
\label{caso3}
\end{figure}

%%%
En el diagrama de la estación Esc La Unión 21201200, se observa que la desviación estándar de los datos de la estación no son altos. Esto sugiere que los datos de la estación no son buenos. Los datos de la estación presentan poca variación \\

Los datos modelados de la estación el Dorado 21205791 presentaron valores altos de desviación estándar y los datos de la estación meteorológica automática presentó valores bajos. Los datos de la estación tienen una desviación estándar más baja

La estación Univ Nacional 21205012 no se puede observar ninguna razón para sospechar.\\

La estación Villa Teresa 21206920 y Chingaza 35035130 en el caso 3 se puede observar que presenta una baja desviación estándar. Observando la gráfica donde se compara la estación se puede observa que tiene problemas. La estación Villa Teresa Y Chingaza no fue graficada.



\begin{figure}[H]
\begin{subfigure}[normla]{0.5\textwidth}
\caption{Diagrama de Taylor para la Estación Esc La Unión 21201200.}
\includegraphics[draft=false, scale=0.5]{../comparacion_grafica/201508_21201200.png}
\end{subfigure}
~
\begin{subfigure}[normla]{0.5\textwidth}
\caption{Diagrama de Taylor para la estación el Dorado 21205791.}
\includegraphics[draft=false, scale=0.5]{../comparacion_grafica/201508_21205791.png}
\end{subfigure}
~
\begin{subfigure}[normla]{0.5\textwidth}
\caption{Diagrama de Taylor para la Estación Univ Nacional 21205012.}
\includegraphics[draft=false, scale=0.5]{../comparacion_grafica/201508_21205012.png}
\end{subfigure}
~
\begin{subfigure}[normla]{0.5\textwidth}
\caption{Diagrama de Taylor para la estación Villa Teresa 21206920.}
\includegraphics[draft=false, scale=0.5]{../comparacion_grafica/201508_21206920.png}
\end{subfigure}
\end{figure}

\begin{figure}[H]
\begin{subfigure}[normla]{0.5\textwidth}
\caption{Diagrama de Taylor para la Estación Chingaza 35035130.}
\includegraphics[draft=false, scale=0.5]{../comparacion_grafica/201508_35035130.png}
\end{subfigure}


\end{figure}

%%%
\newpage

\begin{figure}[H]


\begin{subfigure}[normla]{0.5\textwidth}
\caption{Diagrama de Taylor para la Estación Univ Nacional código 21205012.}
\includegraphics[draft=false, scale=0.5]{../taylor_simulaciones/201509_m/taylor_21205012.png}
\end{subfigure}
~
\begin{subfigure}[normla]{0.5\textwidth}
\caption{Diagrama de Taylor para la Estación Apto El Dorado código 21205791.}
\includegraphics[draft=false, scale=0.5]{../taylor_simulaciones/201509_m/taylor_21205791.png}
\end{subfigure}
~
\begin{subfigure}[normla]{0.5\textwidth}
\caption{Diagrama de Taylor para la Estación Nueva Generacion código 21206600.}
\includegraphics[draft=false, scale=0.5]{../taylor_simulaciones/201509_m/taylor_21206600.png}
\end{subfigure}
~
\begin{subfigure}[normla]{0.5\textwidth}
\caption{Diagrama de Taylor para la Estación Hda Sta Ana Autom código 21206790.}
\includegraphics[draft=false, scale=0.5]{../taylor_simulaciones/201509_m/taylor_21206790.png}
\end{subfigure}
~
\begin{subfigure}[normla]{0.5\textwidth}
\caption{Diagrama de Taylor para la Estación Villa Teresa Automatica código 21206920.}
\includegraphics[draft=false, scale=0.5]{../taylor_simulaciones/201509_m/taylor_21206920.png}
\end{subfigure}
~
\begin{subfigure}[normla]{0.5\textwidth}
\caption{Diagrama de Taylor para la Estación Pmo Guerrero código 21206930.}
\includegraphics[draft=false, scale=0.5]{../taylor_simulaciones/201509_m/taylor_21206930.png}
\end{subfigure}
~
\end{figure}
           
\begin{figure}[H]\ContinuedFloat
\begin{subfigure}[normla]{0.5\textwidth}
\caption{Diagrama de Taylor para la Estación Ideam Bogota código 21206960.}
\includegraphics[draft=false, scale=0.5]{../taylor_simulaciones/201509_m/taylor_21206960.png}
\end{subfigure}
~
\begin{subfigure}[normla]{0.5\textwidth}
\caption{Diagrama de Taylor para la Estación Sta Cruz De Siecha código 21206980.}
\includegraphics[draft=false, scale=0.5]{../taylor_simulaciones/201509_m/taylor_21206980.png}
\end{subfigure}
~
\begin{subfigure}[normla]{0.5\textwidth}
\caption{Diagrama de Taylor para la Estación Tibaitata Automatica código 21206990.}
\includegraphics[draft=false, scale=0.5]{../taylor_simulaciones/201509_m/taylor_21206990.png}
\end{subfigure}
~
\begin{subfigure}[normla]{0.5\textwidth}
\caption{Diagrama de Taylor para la Estación San Cayetano Autom  código 23125170.}
\includegraphics[draft=false, scale=0.5]{../taylor_simulaciones/201509_m/taylor_23125170.png}
\end{subfigure}
~
\begin{subfigure}[normla]{0.5\textwidth}
\caption{Diagrama de Taylor para la Estación Bosque Intervenido   código 35025090.}
\includegraphics[draft=false, scale=0.5]{../taylor_simulaciones/201509_m/taylor_35025090.png}
\end{subfigure}
~
\begin{subfigure}[normla]{0.5\textwidth}
\caption{Diagrama de Taylor para la Estación Pmo Chingaza código 35035130.}
\includegraphics[draft=false, scale=0.5]{../taylor_simulaciones/201509_m/taylor_35035130.png}
\end{subfigure}
~
\end{figure}
           
\begin{figure}[H]\ContinuedFloat
\begin{subfigure}[normla]{0.5\textwidth}
\caption{Diagrama de Taylor para la Estación Chinavita Automatica código 35075070.}
\includegraphics[draft=false, scale=0.5]{../taylor_simulaciones/201509_m/taylor_35075070.png}
\end{subfigure}
~
\begin{subfigure}[normla]{0.5\textwidth}
\caption{Diagrama de Taylor para la Estación Pmo Rabanal Automatica  código 35075080.}
\includegraphics[draft=false, scale=0.5]{../taylor_simulaciones/201509_m/taylor_35075080.png}
\end{subfigure}
~
\begin{subfigure}[normla]{0.5\textwidth}
\caption{Diagrama de Taylor para la Estación La Capilla Autom código 35085080.}
\includegraphics[draft=false, scale=0.5]{../taylor_simulaciones/201509_m/taylor_35085080.png}
\end{subfigure}



\caption{Diagramas de Taylor para el Caso 4}
\label{caso4}
\end{figure}

Para la estación Villa Teresa 21206920 la desviación estándar de la estación automática es mucho menor que los valores modelados.\\

Para la estación Sta Cruz de Siecha 21206980 creo que no está bien que sea descartada.\\

Para la estación Chingaza 35035130 se observa que los datos de la estación tiene una baja desviación estándar.\\



\begin{figure}[H]
\begin{subfigure}[normla]{0.5\textwidth}
\caption{Diagrama de Taylor para la Estación Villa Teresa 21206920.}
\includegraphics[draft=false, scale=0.5]{../comparacion_grafica/201509_21206920.png}
\end{subfigure}
~
\begin{subfigure}[normla]{0.5\textwidth}
\caption{Diagrama de Taylor para la estación Sta Cruz de Siecha 21206980.}
\includegraphics[draft=false, scale=0.5]{../comparacion_grafica/201509_21206980.png}
\end{subfigure}
~
\begin{subfigure}[normla]{0.5\textwidth}
\caption{Diagrama de Taylor para la Estación Chingaza 35035130.}
\includegraphics[draft=false, scale=0.5]{../comparacion_grafica/201509_35035130.png}
\end{subfigure}

\end{figure}




Profe estas son las gráficas de todas las simulaciones la verdad yo sí quería usarlas todas. Segú mi criterio las estaciones que se deben sacar son:

Para el primer caso: ninguna\\
Para el segundo caso: 21206980 Sta Cruz de Siecha\\
Para el tercer caso: 21201200 Esc La Unión,  21205791 El Dorado, 21205012 Univ Nacional, 21206920 Villa Teresa, 35035130 Pmo Chingaza.\\
Para el cuarto caso: 21206920 Villa teresa, 2120698 Sta Cruz de Siecha, 35035130 Chingaza.

%Profe si sacamos esta estaciones la tabla da de esta forma.
%
%
%\begin{table}[H]
%    
%\begin{tabular}{lr}
%\toprule
%
%Combinación simulación-opción               &   Frecuencia     \\
%\midrule
%bl\_pbl\_physics-0     &      2 \\
%bl\_pbl\_physics-5     &     10 \\
%bl\_pbl\_physics-6     &      8 \\
%bl\_pbl\_physics-7     &     14 \\
%bl\_pbl\_physics-8     &     12 \\
%bl\_pbl\_physics-9     &      2 \\
%bl\_pbl\_physics-12    &     11 \\
%bl\_pbl\_physics-99    &      7 \\
%cu\_physics-0         &      6 \\
%cu\_physics-2         &      2 \\
%cu\_physics-5         &      6 \\
%cu\_physics-6         &      5 \\
%cu\_physics-11        &      4 \\
%cu\_physics-14        &      6 \\
%cu\_physics-16        &      4 \\
%cu\_physics-93        &      1 \\
%cu\_physics-99        &      4 \\
%ideam-colombia       &      6 \\
%mp\_physics-0         &     10 \\
%mp\_physics-1         &      1 \\
%mp\_physics-2         &      2 \\
%mp\_physics-3         &      2 \\
%mp\_physics-8         &      1 \\
%mp\_physics-11        &      5 \\
%mp\_physics-13        &      4 \\
%mp\_physics-32        &      8 \\
%ra\_lw\_physics-3      &      4 \\
%ra\_lw\_physics-5      &      4 \\
%ra\_lw\_physics-7      &     10 \\
%ra\_lw\_physics-31     &      3 \\
%ra\_sw\_physics-2      &     17 \\
%ra\_sw\_physics-3      &     17 \\
%ra\_sw\_physics-4      &     15 \\
%ra\_sw\_physics-5      &     14 \\
%sf\_sfclay\_physics-91 &      8 \\
%\bottomrule
%\end{tabular}
%
%    
%\end{table}
%
%Esta tabla es la tabla resumen de la frecuencia de los resultados sin tener en cuenta las estaciones anteriormente mencionadas. Estos resultados son más parecidos a los que ya se habían obtenido.\\
%
%Dado que sea así queda de la siguiente forma:\\
%
%bl\_pbl = 14 (Estaba)\\
%cu\_physics = 0, 5, 14 (Estaban todas)\\
%ra\_lw = 7 (Estaba)\\
%ra\_sw = 2 y 3 (Nuevo empate)\\


Me tocaría hacer unas nuevas corridas pero no hay lío con ello.
\begin{figure}
    \centering
    \includegraphics[draft=false, scale=0.3]{graph/lebel_3.png}
    \caption{Etiquetas para cada una de las gráficas}
    \label{fig:my_label}
\end{figure}