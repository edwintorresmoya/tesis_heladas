\chapter{Capítulo 5}
\section{Descripción de algunas variables en días de las altas y bajas temperaturas a nivel horario}

%Se tomó la configuración icm para cada uno de los casos y se realizó una comparación visual de los resultados de las simulaciones contra los datos reales de las estaciones. Estos datos fueron comparados a nivel horario. La comparación se realizó para los días de los casos descritos anteriormente. Para explicar el comportamiento de las estaciones se presentará la estación automática Tibaitatá. 

%A nivel horario se ve en la figura \ref{subfig:temp_total} que entre las 0 am y las 6 am se presentan las horas más bajas, %posteriormente se observa un aumento de la temperatura el cuál se encuentra relacionado con el amanecer. Durante el día podemos %observar que la más alta temperatura se encuentra entre las 12 m y las 2 pm. Posteriormente se observa un descenso de la %temperatura que inicia a las 6 pm.\\
%
%A partir de los todos los datos de temperatura de la estación Tibaitatá se seleccionaron los días en los que se presentaron %heladas, estos días fueron seleccionados y gaficados en la figura \ref{subfig:temp_min}. En esta figura se puede observar que %las temperaturas bajo 0\celsius\ se comienzan a presentar desde la 1 am y que las temperaturas más bajas se presentaron entre %las 6 y 6:30 am. Luego se presentan altas temperaturas, casi que en todos los casos las temperaturas son iguales o superiores a %17\celsius\ y en una gran proporción es superiór a 20\celsius. Luego se observa que la temperatura desciende rápidamente. Se %observan datos atípicos en la figura \ref{subfig:temp_min} entre las 2:00 pm y las 5:00 pm.\\
%
%Se extrajeron los datos de las altas temperaturas  y se graficaron horariamente esto se aprecia en la figura %\ref{subfig:temp_max}. Si comparamos esta figura con la figura de las temperaturas bajas \ref{subfig:temp_min} podemos observar %que hay una mayor dispersión de los datos cuando hay temperaturas altas. Esto se aprecia porque enntre las 0 am y las 6 am se %presentan temperaturas sobre 9\celsius. Según la figura se puede concluír que las altas temperaturas se pueden presentar desde %las 9 am hasta las 4 pm. Se observan dos valores que presentan altas temperaturas muy altas, estos datos se deben analizar %puntualmente ya que se puede tratar de un error en los sensores. Adicionalmente, se observa que los días que presentaron altas %temperaturas se pueden presentar temperaturas bajo 0\celsius.

%\begin{figure}
%    \centering
%    \begin{subfigure}[b]{0.35\textwidth}
%	\includegraphics[draft=false, scale=0.3]{grafica_var_dia/21206990_1.png}
%    \caption{Gráfica todos los datos de temperatura.}
%    \label{subfig:temp_total}
%	\end{subfigure}
%	~
%    \begin{subfigure}[b]{0.35\textwidth}
%	\includegraphics[draft=false, scale=0.3]{grafica_var_dia/21206990_3.png}
%    \caption{Gráfica de los datos en un día de helada.}
%    \label{subfig:temp_min}
%	\end{subfigure}
%		~
%    \begin{subfigure}[b]{0.35\textwidth}
%	\includegraphics[draft=false, scale=0.3]{grafica_var_dia/21206990_5.png}
%    \caption{Gráfica de los datos en un día de altas temperaturas.}
%    \label{subfig:temp_max}
%	\end{subfigure}
%	
%    \includegraphics{}
%    \caption{Gráficas de temperatura del aire de la estación Tibaitatá a nivel horario.}
%    \label{fig:temp_tibaitata}
%\end{figure}

Con la finalidad de hacer una caracterización de diferentes variables a nivel horario se usaron los datos de la estación Tibaitatá. Se marcaron los días en los que se presentaron heladas y altas temperaturas. A partir de estos datos se analizará los datos cuando se presenta una helada, cuando se presenta una alta temperatura y todos los datos sin ningún filtro, para las diferentes variables. Los datos de la variable temperatura fueron analizados anteriormente (Sección \ref{area_caracterizacion_heladas_extremas}).\\

La humedad relativa para la estación Tibaitatá presenta altos valores desde las 0 am hasta las 6 am, posteriormente se observa una disminución de los valores de la humedad relativa cuyo valor medio llega hasta 60\% y se amplían los rangos de humedad relativa, luego se presenta una disminución, Figura \ref{subfig:hum_total}.\\

Cuando se presentan las heladas se puede observar que la humedad alcanza valores medios de 37\% a las 12 m., Figura \ref{subfig:hum_min}. Entre las 4 am y las 6 am se observa que la humedad relativa es alta, luego se presenta un descenso.\\

En los días de altas temperaturas se puede observar una mayor dispersión en los valores de humedad Figura \ref{subfig:hum_max} en comparación con los días en los que se presentaron heladas. En algunos casos la humedad relativa es menor a 20\%. Luego de las 2 pm la humedad relativa presenta un aumento continuo. En el caso de las altas temperaturas la humedad presenta un mayor rango de valores.\\


\begin{figure}[H]
    \centering
    \begin{subfigure}[b]{0.45\textwidth}
       \caption{Valores de humedad relativa para todos los días, se usaron 96939 datos.}
	\includegraphics[draft=false, scale=0.4]{grafica_var_dia/21206990_hum_2m_1.png}
     \label{subfig:hum_total}
	\end{subfigure}
	~
    \begin{subfigure}[b]{0.45\textwidth}
       \caption{Humedad relativa para días con heladas, se usaron 546 datos.}
	\includegraphics[draft=false, scale=0.4]{grafica_var_dia/21206990_hum_2m_3.png}
     \label{subfig:hum_min}
	\end{subfigure}
		~
    \begin{subfigure}[b]{0.45\textwidth}
       \caption{Humedad relativa para días con altas temperaturas, se usaron 625 datos.}
	\includegraphics[draft=false, scale=0.4]{grafica_var_dia/21206990_hum_2m_5.png}
     \label{subfig:hum_max}
	\end{subfigure}
	
%    \includegraphics{}
    \caption{Humedad relativa para la estación Tibaitatá a nivel horario, entre el año 2007 hasta el año 2018.}
    \label{fig:hr_tibaitata}
\end{figure}


La radiación desde las 0 am hasta las 5 am presenta los mas bajos valores, posterior a estas horas se presenta un aumento de los valores de radiación como se ve en la Figura \ref{subfig:rad_total}. La hora en la que se presenta la mayor radiación es a las 12 m, luego de esta hora se presenta una disminución de los valores hasta las 6 pm, este es un comportamiento normal que se encuentra condicionado por el amancer y el atardecer.\\

Los valores de radiación para los días que hubo heladas se observa que los primero valores de radiación son registrados a las 7 am, después de esta hora se presenta un rápido aumento en los valores de la radiación, figura \ref{subfig:rad_min}. Adicionalmente a las 9, 10 y 11 am no se observan valores inferiores a 350$w/m^2$ antes de las 11 a.m., esto implica que los días de heladas son días despejados lo que concuerda con lo dicho por \citet{Snyder2005} quien asevera que los días en los que se presentan heladas son días despejados.\\

Cuando hay altas temperaturas podemos observar que los valores de radiación estuvieron más dispersos y alcanzaron valores más altos de radiación Figura \ref{subfig:rad_max} en comparación con la Figura \ref{subfig:rad_min} en la cual los máximos valores son graficados como valores fuera del diagrama de cajas y bigotes. Adicionalmente se observa en la Figura \ref{subfig:rad_min} que las 9, 10 y 11 a.m. sí se presentaron valores menores a 350$w/m^2$, comparado con la Figura \ref{subfig:rad_max} en la cual no se presentaron, esto implica que los días con altas temperaturas se puede presentar cierta cobertura nubosa.\\


\begin{figure}[H]
    \centering
    \begin{subfigure}[b]{0.45\textwidth}
    \caption{Valores de radiación para todos los días evaluados, se usaron 74333 datos.}
	\includegraphics[draft=false, scale=0.4]{grafica_var_dia/21206990_rad_1_1.png}
    \label{subfig:rad_total}
	\end{subfigure}
	~
    \begin{subfigure}[b]{0.45\textwidth}
    \caption{Radiación para días con heladas, se usaron 406 datos.}
	\includegraphics[draft=false, scale=0.4]{grafica_var_dia/21206990_rad_1_3.png}
    \label{subfig:rad_min}
	\end{subfigure}
		~
    \begin{subfigure}[b]{0.45\textwidth}
    \caption{Radiación para días con altas temperaturas, se usaron 411 datos.}
	\includegraphics[draft=false, scale=0.4]{grafica_var_dia/21206990_rad_1_5.png}
    \label{subfig:rad_max}
	\end{subfigure}
	
%    \includegraphics{}
	\caption{Radiación para la estación Tibaitatá a nivel horario, entre el año 2007 hasta el año 2018. La radiación medida en la estación automática es radiación de onda corta.}
    \label{fig:rad_tibaitata}
\end{figure}

%Hablar sobre la precipitación 

La estación automática HYDRAS reporta el valor promedio de la precipitación cada 10 minutos. Para poder realizar una comparación se sumaron todos los valores de precipitación cada hora, estos valores son horarios y no son acumulativos. La precipitación fue sometida a el proceso de control de calidad. Es importante notar que en Bogotá una de las precipitaciones más altas registradas por el Observatorio Ambiental fue de 56.4 mm en una hora, registrado en la estación El Bosque en el año 2010 \citep{OAB2011}.\\

La precipitación horaria en esta estación se presenta principalmente entre las 11 a.m. y las 6 p.m., aunque también se presentan precipitaciones con menor intensidad y frecuencia entre las 6 pm y las 11 pm, (Figura \ref{subfig:prec_total}). El día que se presentaron heladas no hubo valores de precipitación mayores a 1 mm, figura \ref{subfig:prec_min}, esto implica que para estos días no se presentaron precipitaciones importantes. En el caso de los días con altas temperaturas no se presentan precipitaciones desde las 0 am hasta las 3 pm, pero luego de estas horas en algunos casos se presentaron precipitaciones. Estas precipitaciones pueden ser el resultado de las altas temperaturas como lo nombra \citet{Fischer2015} en su artículo.

\begin{figure}[H]
    \centering
    \begin{subfigure}[b]{0.45\textwidth}
    \caption{Valores de precipitación para todos los días, se usaron 603002 datos.}
	\includegraphics[draft=false, scale=0.4]{grafica_var_dia/21206990_precip_1_1.png}
    \label{subfig:prec_total}
	\end{subfigure}
	~
    \begin{subfigure}[b]{0.45\textwidth}
    \caption{Precipitación para días con heladas, se usaron 3342 datos.}
	\includegraphics[draft=false, scale=0.4]{grafica_var_dia/21206990_precip_1_3.png}
    \label{subfig:prec_min}
	\end{subfigure}
		~
    \begin{subfigure}[b]{0.45\textwidth}
    \caption{Precipitación para días con altas temperaturas, se usaron 4107 datos.}
	\includegraphics[draft=false, scale=0.4]{grafica_var_dia/21206990_precip_1_5.png}
    \label{subfig:prec_max}
	\end{subfigure}
	
%    \includegraphics{}
    \caption{Precipitación para la estación Tibaitatá a nivel horario, entre el año 2007 hasta el año 2018.}
%    \label{fig:temp_tibaitata}
\end{figure}



%%%%%%% Hablar sobre el bulbo húmedo

El bulbo húmedo presenta valores que oscilan entre los 10\celsius\ entre las 0 y las 6, posteriormente se presenta un aumento logrando los valores máximos entre las 13 y 14 horas, luego se evidencia una disminución de los valores, Figura \ref{subfig:wb_total}. En los días que se presentaron heladas se evidencia un rápido salto entre las 7 y las 9 am, posterior a estas horas alcanza los valores máximos entre las 13 y las 14 horas, Figura \ref{subfig:wb_min}. Cuando hay altas temperaturas se puede observar que las temperaturas de bulbo húmedo en la mañana no son tan bajas como en los días de heladas, pero se presenta un rápido cambio entre las 7 y las 8 am, Figura \ref{subfig:wb_max}. En estas gráficas podemos observar que la temperatura de bulbo húmedo para los días de helada son bajos en la mañana al igual que en la noche, para las altas temperaturas se observa que las temperaturas en la mañana. En la noche no descienden tanto en comparación con los días de helada.

\begin{figure}[H]
    \centering
    \begin{subfigure}[b]{0.45\textwidth}
    \caption{Valores de bulbo húmedo para todos los días, se usaron 69945 datos.}
	\includegraphics[draft=false, scale=0.4]{grafica_var_dia/21206990_wb_1.png}
    \label{subfig:wb_total}
	\end{subfigure}
	~
    \begin{subfigure}[b]{0.45\textwidth}
    \caption{Valores de bulbo húmedo para días con heladas, se usaron 371 datos.}
	\includegraphics[draft=false, scale=0.4]{grafica_var_dia/21206990_wb_3.png}
    \label{subfig:wb_min}
	\end{subfigure}
		~
    \begin{subfigure}[b]{0.45\textwidth}
    \caption{Valores de bulbo húmedo para días con altas temperaturas, se usaron 455 datos.}
	\includegraphics[draft=false, scale=0.4]{grafica_var_dia/21206990_wb_5.png}
    \label{subfig:wb_max}
	\end{subfigure}
	
%    \includegraphics{}
    \caption{Bulbo húmedo para la estación Tibaitatá a nivel horario, entre el año 2007 hasta el año 2018.}
%    \label{fig:temp_tibaitata}
\end{figure}

%%%% Hablar sobre el punto de rocío

La temperatura de punto de rocío tiene un comportamiento similar al presentado para el bulbo húmedo, ya que hay bajas temperaturas entre 0 y 6 horas, posteriormente se presenta un aumento hasta las 14 horas y luego se presenta un descenso de la temperatura, Figura \ref{subfig:td_min}. En los días que se presentaron heladas podemos observar que hay bajas temperaturas en las horas de la mañana y se presenta un cambio brusco en la temperatura entre las 7 y 9 horas, posterior hay un aumento en los valores de temperatura cuyos valores más altos se alcanzan entre las 14 y 15 horas, Figura \ref{subfig:td_min}. Cuando se presentaron altas temperaturas podemos observar que la media de la temperatura no fue tan baja como la que se presentó los días con heladas, \ref{subfig:td_max}.

\begin{figure}[H]
    \centering
    \begin{subfigure}[b]{0.45\textwidth}
    \caption{Punto de rocío para todos los días, se usaron 69945 datos.}
	\includegraphics[draft=false, scale=0.4]{grafica_var_dia/21206990_Td_1.png}
    \label{subfig:td_total}
	\end{subfigure}
	~
    \begin{subfigure}[b]{0.45\textwidth}
    \caption{Punto de rocío para días con heladas, se usaron 371 datos.}
	\includegraphics[draft=false, scale=0.4]{grafica_var_dia/21206990_Td_3.png}
    \label{subfig:td_min}
	\end{subfigure}
		~
    \begin{subfigure}[b]{0.45\textwidth}
    \caption{Punto de rocío para días con altas temperaturas, se usaron 455 datos.}
	\includegraphics[draft=false, scale=0.4]{grafica_var_dia/21206990_Td_5.png}
    \label{subfig:td_max}
	\end{subfigure}
	
%    \includegraphics{}
    \caption{Punto de rocío para la estación Tibaitatá a nivel horario, entre el año 2007 hasta el año 2018.}
%    \label{fig:temp_tibaitata}
\end{figure}


\section{Comparación de temperaturas simuladas y observadas}

Para evaluar el comportamiento de las nuevas configuraciones del modelo WRF se realizaron gráficas que permitan comparar los datos de una estación automática, los datos de la configuración usada por el IDEAM para Colombia llamada IDEAM-Colombia y las nuevas configuraciónes llamadas icm e icm-mp\_physics 3. La comparación se realizó para los cuatro casos descritos anteriormente, y para la estación automática Tibaitatá. Se escogió esta estación porque esta estación es una de las estaciones de referencia en la Sabana de Bogotá para el IDEAM.\\

Como resultado de las gráficas podemos observar que las simulaciones realizadas por WRF para el caso 1 (Figura \ref{caso1_tiba_wrf}) no reprodujeron éxitosamente las temperaturas las temperaturas más bajas y más altas registradas por la estación. De las tres configuraciones del modelo evaluadas, la que presentó los resultados menos lejanos a las observaciones fue la configuración icm.\\

Para el caso 2 (Figura \ref{caso2_tiba_wrf}), se puede observar que los datos modelados presentan un mejor ajuste con los datos observados, excepto en las temperaturas diarias más bajas. Para el día 31 de agosto, la temperatura mínima del modelo fue mas baja que la observada llegando a valores de -4\celc\ y los valores de la estación llegó hasta una temperatura de 6.5\celc, las tres configuraciones presentaron resultados similares.\\


En el caso 3 Figura \ref{caso3_tiba_wrf} podemos observar que los datos de la estación automática presenta varios cambios repentinos en la temperatura antes de las 6 am (hora local). En cuanto a las temperaturas mínimas de las simulaciones para el 28 de agosto, la configuración IDEAM-Colombia fue la que presentó resultados menos lejanos con una diferencia de 2.5\celsius. En el día 28 de agosto la temperatura máxima registrada por la estación automática presentó un pico de temperatura, las temperaturas modeladas no reprodujeron este pico. Las temperaturas mínimas del día 29 de agosto fueron sobre estimadas en promedio 4\celsius\ y la configuración que presentó valores menos lejanos fue IDEAM-Colombia. El valor máximo para el día 29 tuvo una buena aproximación para la configuración icm e icm-mp\_physics 3.\\

En el caso 4 (Figura \ref{caso4_tiba_wrf}) podemos observar que los datos de temperatura de la estación automática no presentan tanta variación como en el caso 3. Se puede observar que para el día 8 de septiembre la temperatura mínima modelada de las combinaciónes icm e icm-mp\_physics 3 para el dominio 2 presentaron los valores menos lejanos a la temperatura registrada por la estación 1.75\celsius. Por otra parte, fueron las combinaciónes icm e icm-mp\_physics 3 las que presentaron la mejor aproximación para las altas temperaturas del mismo día. El día 9 de septiembre todas las simulaciones produjeron valores de temperatura más bajos que los valores de la estación en promedio 2.5\celsius\ y las configuraciones que presentaron los valores menos lejanos fueron IDEAM-Colombia e icm-physics 3. Las altas temperaturas para el día 9 de septiembre que presentaron los resultados menos lejanos fue la combinación icm que tuvo una diferencia de 1\celsius.\\

\begin{figure}[H]
    
\begin{subfigure}[normla]{0.4\textwidth}
\caption{Caso 1 helada del 4 de febrero de 2007.}
%"comparacion_real_wrf_200702.py"5
\label{caso1_tiba_wrf}
\includegraphics[draft=false, scale=0.4]{comparacion_grafica/200702_21206990.png}
\end{subfigure}
~
\begin{subfigure}[normla]{0.4\textwidth}
\caption{Caso 2 helada del 30 de octubre de 2014.}
\label{caso2_tiba_wrf}
\includegraphics[draft=false, scale=0.4]{comparacion_grafica/201408_21206990.png}
\end{subfigure}
~
\centering
\begin{subfigure}[normla]{0.4\textwidth}
\caption{Caso 3 altas temperaturas para el 28 de octubre de 2015.}
\label{caso3_tiba_wrf}
\includegraphics[draft=false, scale=0.4]{comparacion_grafica/201508_21206990.png}
\end{subfigure}
~
\centering
\begin{subfigure}[normla]{0.4\textwidth}
\caption{Caso 4 altas temperaturas para el 8 de septiembre de 2015.}
\label{caso4_tiba_wrf}
\includegraphics[draft=false, scale=0.4]{comparacion_grafica/201509_21206990.png}
\end{subfigure}
~

    \caption{Temperatura del aire a dos metros para los cuatro casos escogidos de acuerdo a los valores registrados en la estación automática Tibaitatá y los valores simulados con las configuraciones ideam-Colombia, icm e icm-mp\_physics 3.}
    \label{fig:wrf_temp_tibaitata}

\end{figure}



La configuración icm presenta, en general valores mínimos más bajos y máximos más altos en comparación con los datos la simulación ideam-Colombia r icm-mp\_physics 3. Si comparamos la configuración icm y icm-mp\_physics 3 podemos observar que la configuración icm presentó los valores mínimos más bajos. Esto implica que si se quiere simular valores extremos de temperatura la combinación icm presenta mejores resultados.\\

Comparando los resultados de las simulaciones frente a los datos reales se puede apreciar que la configuración icm e icm-mp\_physics 3 presenta un mejor ajuste a las temperaturas altas en comparación con las temperaturas mínimas.\\


\section{Comparación de otras variables}

Además del análisis realizado para la variable de temperatura del aire se evaluó cuál fue el desempeño de la combinación icm, icm-mp\_physics 3 e IDEAM-Colombia con respecto a otras variables medidas por las estaciones automáticas de la zona de estudio tales como precipitación, humedad relativa, radiación, bulbo húmedo, punto de rocío y rapidez del viento. Se va a mostrar los datos de la estación Tibaitatá, ya que es una estación de referencia para el IDEAM, las gráficas de las demás estaciones se encuentran en el Anexo \ref{anexo:graficas_otras_variables_wrf}.\\

\subsection{Humedad relativa}


En el caso 1, la noche entre el 3 y 4 de febrero de 2007, podemos observar que los valores de humedad según la estación automática estuvieron cercanos a 100\% por un periodo cercano a 8 horas. Posteriormente, la humedad presentó un rápido descenso el día 4 de febrero (Figura \ref{caso1_tiba_wrf_hum}). Los datos modelados lograron reproducir el valor máximo de humedad relativa, pero el modelo no fue capaz de reproducir la duración del período en el que la humedad estuvo cercana al 100\%. Las tres configuraciones no fueron capaces de reproducir la humedad mínima. La configuración icm e icm-mp\_physics 3 presentaron un comportamiento similar, ya que el dominio 1 presentó valores superiores al 100\% y fue el dominio 2 el que alcanzó valores de 100\% y no excedió este valor.\\

En el caso 2 (Figura \ref{caso2_tiba_wrf_hum}) podemos observar un mejor ajuste entre el día 29 de octubre a las 12 y el 30 de octubre a las 12 (Hora local). Los datos reportados por la estación automática muestran que entre el día 29 y 30 de octubre de 2014 la humedad llegó hasta 100\%. En el siguiente día la humedad llegó hasta 90\% (Figura \ref{caso2_tiba_wrf_hum}). Las configuraciones icm-mp\_physics 3 e IDEAM-Colombia presentaron los resultados menos lejanos hasta las 4 pm del 30 de agosto de 2014, ya que posteriormente presentó una sobreestimación. La noche entre el 30 y 31 no presentó un buen ajuste ya que la humedad fue sobreestimada aproximadamente en 10\% de humedad relativa. La configuración que los resultados menos lejanos fue icm-mp\_physiscs 3 IDEAM-Colombia, de nuevo la configuración icm sobre estimó la humedad en las horas de la noche.\\

En el caso 3 podemos observar que hay un buen ajuste de los datos modelados con los datos reales de las estaciones automáticas. La estación automática presenta un pico de baja humedad el día 28 de octubre del 2015, este valor no fue representado por las simulaciones (Figura \ref{caso2_tiba_wrf_hum}). En la noche entre el 28 y 29 de octubre del 2015 la configuración IDEAM-Colombia e icm-mp\_physics 3 presentaron valores cercanos a el valor de la estación. Los valores mínimos de humedad para el día 29 de octubre de 2015 no fueron bien representados por las simulaciones, ya que las simulaciones no fueron capaces de modelar el valor mínimo de la estación automática. La configuración que mejor ajuste presenta es la IDEAM-Colombia.\\

En el caso 4 podemos observar que los datos observados y los datos simulados presentaron un buen ajuste. El día 8 de septiembre de 2015 la estación automática presentó valores de 40\% y las tres configuraciones no fueron capaces de generar estos valores, la configuración que presentó los valores más cercanos fue IDEAM-Colombia, Figura \ref{caso4_tiba_wrf_hum}. Es interesante notar cómo los valores máximos fueron muy bien representados por la configuración IDEAM-Colombia.\\

En general para esta sección podemos observar que las configuraci IDEAM-Colombia e icm-mp\_physics 3 representa mejor la humedad relativa.


\begin{figure}[H]

\begin{subfigure}[normla]{0.4\textwidth}
\caption{Caso 1 helada del 4 de febrero de 2007.}
\label{caso1_tiba_wrf_hum}
\includegraphics[draft=false, scale=0.4]{comparacion_graficas_otras_var/200702_21206990_humedad.png}
\end{subfigure}
~
\begin{subfigure}[normla]{0.4\textwidth}
\caption{Caso 2 helada del 30 de octubre de 2014.}
\label{caso2_tiba_wrf_hum}
\includegraphics[draft=false, scale=0.4]{comparacion_graficas_otras_var/201408_21206990_humedad.png}
\end{subfigure}
~
\centering
\begin{subfigure}[normla]{0.4\textwidth}
\caption{Caso 3 altas temperaturas para el 28 de octubre de 2015.}
\label{caso3_tiba_wrf_hum}
\includegraphics[draft=false, scale=0.4]{comparacion_graficas_otras_var/201508_21206990_humedad.png}
\end{subfigure}
~
\centering
\begin{subfigure}[normla]{0.4\textwidth}
\caption{Caso 4 altas temperaturas para el 8 de septiembre de 2015.}
\label{caso4_tiba_wrf_hum}
\includegraphics[draft=false, scale=0.4]{comparacion_graficas_otras_var/201509_21206990_humedad.png}
\end{subfigure}

    \caption{Humedad relativa para los cuatro casos escogidos de acuerdo a los valores registrados en la  estación automática Tibaitatá y los valores simulados con las configuraciones icm, icm-mp\_phyiscs 3 e IDEAM-Colombia.}
    \label{fig:wrf_hum_tibaitata}
\end{figure}


\subsection{Radiación}


La radiación es una de las variables que presenta más fallas por falta de datos observados, esto se evidencia en las gráfica \ref{fig:wrf_rad_tibaitata}. Para el caso 1 (\ref{caso1_tiba_wrf_rad}) hay disponibilidad de la mayoría de los datos observados, pero para los otros casos (Figura \ref{caso2_tiba_wrf_rad}, \ref{caso3_tiba_wrf_rad} y \ref{caso4_tiba_wrf_rad}) se observa que en las horas de la noche se está presentado fallas de registro.\\

Para los cuatro casos la radiación no fue correctamente modelada ya que los valores fueron sobreestimados por todas las configuraciones..\\

\begin{figure}[H]
    
\begin{subfigure}[normla]{0.4\textwidth}
\caption{Caso 1 helada del 4 de febrero de 2007.}
\label{caso1_tiba_wrf_rad}
\includegraphics[draft=false, scale=0.4]{comparacion_graficas_otras_var/200702_21206990_radiacion.png}
\end{subfigure}
~
\begin{subfigure}[normla]{0.4\textwidth}
\caption{Caso 2 helada del 30 de octubre de 2014.}
\label{caso2_tiba_wrf_rad}
\includegraphics[draft=false, scale=0.4]{comparacion_graficas_otras_var/201408_21206990_radiacion.png}
\end{subfigure}
~
\centering
\begin{subfigure}[normla]{0.4\textwidth}
\caption{Caso 3 altas temperaturas para el 28 de octubre de 2015.}
\label{caso3_tiba_wrf_rad}
\includegraphics[draft=false, scale=0.4]{comparacion_graficas_otras_var/201508_21206990_radiacion.png}
\end{subfigure}
~
\centering
\begin{subfigure}[normla]{0.4\textwidth}
\caption{Caso 4 altas temperaturas para el 8 de septiembre de 2015.}
\label{caso4_tiba_wrf_rad}
\includegraphics[draft=false, scale=0.4]{comparacion_graficas_otras_var/201509_21206990_radiacion.png}
\end{subfigure}

    \caption{Radiación de onda corta para los cuatro casos escogidos de acuerdo a los valores registrados en la estacón Tibaitatá y a los valores simulados con las configuraciones icm, icm-mp\_physics 3 e IDEAM-Colombia.} % La variable que se usó para la extracción del wrf fue SWDOWN = DOWNWARD SHORT WAVE FLUX AT GROUND SURFACE (W m-2)
    \label{fig:wrf_rad_tibaitata}
\end{figure}


\subsection{Precipitación}

En la Figura \ref{fig:wrf_prec_tibaitata} se puede observar que en ninguno de los días de los cuatro casos se presentaron lluvias y las dos configuraciones del modelo generaron precipitaciones en 3 de los 4 casos analizados. En el caso 1 (Figura \ref{caso1_tiba_wrf_prec}) se observa que la configuración icm e IDEAM-Colombia representó correctamente la no precipitación, mientras que la configuración icm-mp\_physics presentó equivocadamente valores de precipitación. Para el caso 2 se observa que no hubo precipitación y la configuración icm presentó una precipitación acumulada superior a los 2 mm y la configuración icm-mp\_physics 3 presentó una precipitación acumulada de 0.5 mm(Figura \ref{caso2_tiba_wrf_prec}). En el caso 3, todas las configuraciones presentaron valores de precipitación siendo mayores los de la configuración icm (Figura \ref{caso3_tiba_wrf_prec}). Y en el caso 4 sucede algo similar al caso 3 en el cual ambas configuraciones presentaron valores de precipitación y fue la configuración icm-mp\_physics 3 fue la que presentó una mayor sobreestimación, ver Figura \ref{caso4_tiba_wrf_prec}.\\

%La precipitación del modelo resulta de la suma de dos variables internas de las salidas una de ellas es RAINC la cual viene del esquema de cúmulos y la otra es RAINNC la cual viene de la microfísica de nubes. Ya que los resultados mostraron que la mejor opción es mantener parametrizaciones apagadas mp\_physics y cu\_physics, como resultado no se van a obtener valores de precipitación en las salidas de la configuración icm.\\

\begin{figure}[H]
    
\begin{subfigure}[normla]{0.4\textwidth}
\caption{Caso 1 helada del 4 de febrero de 2007.}
\label{caso1_tiba_wrf_prec}
\includegraphics[draft=false, scale=0.4]{comparacion_graficas_otras_var/200702_21206990_rain.png}
\end{subfigure}
~
\begin{subfigure}[normla]{0.4\textwidth}
\caption{Caso 2 helada del 30 de octubre de 2014.}
\label{caso2_tiba_wrf_prec}
\includegraphics[draft=false, scale=0.4]{comparacion_graficas_otras_var/201408_21206990_rain.png}
\end{subfigure}
~
\centering
\begin{subfigure}[normla]{0.4\textwidth}
\caption{Caso 3 altas temperaturas para el 28 de octubre de 2015.}
\label{caso3_tiba_wrf_prec}
\includegraphics[draft=false, scale=0.4]{comparacion_graficas_otras_var/201508_21206990_rain.png}
\end{subfigure}
~
\centering
\begin{subfigure}[normla]{0.4\textwidth}
\caption{Caso 4 altas temperaturas para el 8 de septiembre de 2015.}
\label{caso4_tiba_wrf_prec}
\includegraphics[draft=false, scale=0.4]{comparacion_graficas_otras_var/201509_21206990_rain.png}
\end{subfigure}

    \caption{Figuras de comparación entre los datos de precipitación para la estación Tibaitatá en los cuatro casos escogidos. La línea continua representa los datos de la estación meteorológica automática del IDEAM, la línea gris representa los valores modelados teniendo en cuenta la configuración del IDEAM-Colombia y la línea roja representa los datos de la configuración hallada icm. Los triángulos representan los resultados de el dominio 1 y los pentágonos representan los resultados del dominio 2.}
    \label{fig:wrf_prec_tibaitata}
\end{figure}


%\subsection{Bulbo húmedo}
%
%En el caso 1 \ref{caso1_tiba_wrf_wb} los valores modelados no representaron correctamente los valores mínimos, el día 3 de febrero del 2007 la configuración icm tuvo una mejor aproximación a los valores maximos y el día 4 de febrero de 2007 la configuración IDEAM-Colombia fue la que presentó la mejor aproximación a los valore máximos reales. En el caso 2 las ambas configuraciones presentaron resultados similares a los valores generados por la estación automática, para el día 30 de octubre del 2014 los valores de las configuraciones no presentaron valores similares a los valores de la estación automática, ver Figura \ref{caso2_tiba_wrf_wb}. En el caso 3 el valor del bulbo húmedo para los modelos tuvo una buena aproximación, el valor mínimo del 28 de octubre del 2015 no fue correctamente representado, pero el valor máximo fue correctamente representado por la configuración icm, ver Figura \ref{caso3_tiba_wrf_prec}. Y en el caso 4 se observa que la configuración icm presentó un mejor ajuste a los valores de la estación automática, el día 8 de septiembre del 2015 ambas configuraciones presentaron una subestimación y fue más evidente en la configuración icm, ver Figura \ref{caso4_tiba_wrf_wb}.\\
%
%%La precipitación del modelo resulta de la suma de dos variables internas de las salidas una de ellas es RAINC la cual viene del esquema de cúmulos y la otra es RAINNC la cual viene de la microfísica de nubes. Ya que los resultados mostraron que la mejor opción es mantener parametrizaciones apagadas mp\_physics y cu\_physics, como resultado no se van a obtener valores de precipitación en las salidas de la configuración icm.\\
%
%\begin{figure}[H]
%    
%\begin{subfigure}[normla]{0.4\textwidth}
%\caption{Caso 1 helada del 4 de febrero de 2007.}
%\label{caso1_tiba_wrf_wb}
%\includegraphics[draft=false, scale=0.4]{comparacion_graficas_otras_var/200702_21206990_wetbulb.png}
%\end{subfigure}
%~
%\begin{subfigure}[normla]{0.4\textwidth}
%\caption{Caso 2 helada del 30 de octubre de 2014.}
%\label{caso2_tiba_wrf_wb}
%\includegraphics[draft=false, scale=0.4]{comparacion_graficas_otras_var/201408_21206990_wetbulb.png}
%\end{subfigure}
%~
%\centering
%\begin{subfigure}[normla]{0.4\textwidth}
%\caption{Caso 3 altas temperaturas para el 28 de octubre de 2015.}
%\label{caso3_tiba_wrf_wb}
%\includegraphics[draft=false, scale=0.4]{comparacion_graficas_otras_var/201508_21206990_wetbulb.png}
%\end{subfigure}
%~
%\centering
%\begin{subfigure}[normla]{0.4\textwidth}
%\caption{Caso 4 altas temperaturas para el 8 de septiembre de 2015.}
%\label{caso4_tiba_wrf_wb}
%\includegraphics[draft=false, scale=0.4]{comparacion_graficas_otras_var/201509_21206990_wetbulb.png}
%\end{subfigure}
%
%    \caption{Figuras de comparación entre los datos de bulbo húmedo para la estación Tibaitatá en los cuatro casos escogidos. La línea continua representa los datos de la estación meteorológica automática del IDEAM, la línea gris representa los valores modelados teniendo en cuenta la configuración del IDEAM-Colombia y la línea roja representa los datos de la configuración hallada icm. Los triángulos representan los resultados de el dominio 1 y los pentágonos representan los resultados del dominio 2.}
%    \label{fig:wrf_wb_tibaitata}
%\end{figure}


\subsection{Punto de rocío}

En el caso 1 (Figura \ref{caso1_tiba_wrf_dp}) se observa que los resultados de las simulaciones no se ajustaron a los valores de la estación automática. Para el caso 2, se observa que hubo mejor respuesta de los datos modelados en comparación con los datos del caso 1, los días 29 y 30 de agosto del 2014 los valores máximos tuvieron una mejor aproximación, ya que se presentaron los valores de punto de rocío pero con un lapso de 3 horas promedio por parte de la configuración icm e icm-mp\_physics 3, ver Figura \ref{caso2_tiba_wrf_dp}. En el caso 3, sucede algo similar al caso 2, ya que en los primeros dos días de modelación los valores máximos con la configuración icm presentan los valores de punto de rocío, pero con un lapso de retraso de 5 horas en promedio, ver Figura \ref{caso3_tiba_wrf_dp}. En el caso 4 los valores modelados no presentan similitud con los valores de la estación automática, ver Figura \ref{caso4_tiba_wrf_prec}.\\



\begin{figure}[H]
    
\begin{subfigure}[normla]{0.4\textwidth}
\caption{Caso 1 helada del 4 de febrero de 2007.}
\label{caso1_tiba_wrf_dp}
\includegraphics[draft=false, scale=0.4]{comparacion_graficas_otras_var/200702_21206990_dewpoint.png}
\end{subfigure}
~
\begin{subfigure}[normla]{0.4\textwidth}
\caption{Caso 2 helada del 30 de octubre de 2014.}
\label{caso2_tiba_wrf_dp}
\includegraphics[draft=false, scale=0.4]{comparacion_graficas_otras_var/201408_21206990_dewpoint.png}
\end{subfigure}
~
\centering
\begin{subfigure}[normla]{0.4\textwidth}
\caption{Caso 3 altas temperaturas para el 28 de octubre de 2015.}
\label{caso3_tiba_wrf_dp}
\includegraphics[draft=false, scale=0.4]{comparacion_graficas_otras_var/201508_21206990_dewpoint.png}
\end{subfigure}
~
\centering
\begin{subfigure}[normla]{0.4\textwidth}
\caption{Caso 4 altas temperaturas para el 8 de septiembre de 2015.}
\label{caso4_tiba_wrf_dp}
\includegraphics[draft=false, scale=0.4]{comparacion_graficas_otras_var/201509_21206990_dewpoint.png}
\end{subfigure}

    \caption{Temperatura de punto de rocío para los cuatro casos escogidos de acuerdo a los valores registrados en la estación automática Tibaitatá y a los valores simulados con la configuraciones icm, icm-mp\_physics 3 e IDEAM-Colombia.}
    \label{fig:wrf_dp_tibaitata}
\end{figure}


\subsection{Rapidez del viento}

Los valores de rapidez del viento para la estación Tibaitatá no pasaron el control de calidad, por esta razón se trabajó con la estación La Boyera, ya que esta se encuentra a una altura similar a la de Tibaitatá.\\

En el caso 1 se observa que las simulaciones no representaron correctamente los valores de la rapidez del viento para la estación La Boyera, solo el día 3 de febrero del 2007 la rapidez máximas alcanzaron valores similares, ver Figura \ref{caso1_tiba_wrf_vv}. Para el caso 2 se observa que hubo un mejor ajuste de los datos modelados frente a los valores de la estación automática, hay un mejor ajuste entre los valores mínimos en comparación con los valores máximos que tienen una diferencia de 2\celsius\ en promedio, ambas configuraciones del modelo presentan resultados similares, ver Figura \ref{caso2_tiba_wrf_vv}. En el caso 3 podemos observar que los datos modelados presentaron un mejor ajuste y esta vez la configuración IDEAM-Colombia presentó un buen ajuste frente a las rapidez máxima el día 29 de octubre del 2015, ver Figura \ref{caso3_tiba_wrf_vv}. Y en el caso 4 se puede observar que en los valores de rapidez baja los valores modelados presentan un mejor ajuste, ver Figura \ref{caso4_tiba_wrf_vv}.\\

%La precipitación del modelo resulta de la suma de dos variables internas de las salidas una de ellas es RAINC la cual viene del esquema de cúmulos y la otra es RAINNC la cual viene de la microfísica de nubes. Ya que los resultados mostraron que la mejor opción es mantener parametrizaciones apagadas mp\_physics y cu\_physics, como resultado no se van a obtener valores de precipitación en las salidas de la configuración icm.\\

\begin{figure}[H]
    
\begin{subfigure}[normla]{0.4\textwidth}
\caption{Caso 1 helada del 4 de febrero de 2007 para la estación La Boyera.}
\label{caso1_tiba_wrf_vv}
\includegraphics[draft=false, scale=0.4]{comparacion_graficas_otras_var/200702_24015110_vel_viento.png}
\end{subfigure}
~
\begin{subfigure}[normla]{0.4\textwidth}
\caption{Caso 2 helada del 30 de octubre de 2014 para la estación Tibaitatá.}
\label{caso2_tiba_wrf_vv}
\includegraphics[draft=false, scale=0.4]{comparacion_graficas_otras_var/201408_21206990_vel_viento.png}
\end{subfigure}
~
\centering
\begin{subfigure}[normla]{0.4\textwidth}
\caption{Caso 3 altas temperaturas para el 28 de octubre de 2015 para la estación Tibaitatá.}
\label{caso3_tiba_wrf_vv}
\includegraphics[draft=false, scale=0.4]{comparacion_graficas_otras_var/201508_21206990_vel_viento.png}
\end{subfigure}
~
\centering
\begin{subfigure}[normla]{0.4\textwidth}
\caption{Caso 4 altas temperaturas para el 8 de septiembre de 2015 para la estación Tibaitatá.}
\label{caso4_tiba_wrf_vv}
\includegraphics[draft=false, scale=0.4]{comparacion_graficas_otras_var/201509_21206990_vel_viento.png}
\end{subfigure}

    \caption{Rapidez del viento para los cuatro casos escogidos de acuerdo a los valores registrados en las estaciones automáticas la Boyera y Tibaitatá y los valores simulados con las configuraciones icm, icm-mp\_physics 3 e IDEAM-Colombia.}
    \label{fig:wrf_vv_tibaitata}
\end{figure}


\section{Comparación estadística entre la configuración icm, icm-mp\_physics 3 e IDEAM-Colombia}

Para el caso de las heladas, la información sobre temperatura, humedad, y rapidez del viento en capas cercanas a la superficie juega un papel importante en la toma de decisiones para la protección de los cultivos \citep{prabha2008}. Por esta razón, se realizó una comparación estadística de las variables temperatura, humedad relativa, y rapidez del viento para los cuatro casos de los datos modelados y los datos de las estaciones automáticas.\\

%Para la extracción de estas variables se usaron las siguientes fórmulas.\\

Para saber la frecuencia horaria en cada año que se presentaron heladas y altas temperaturas se realizaron gráficas de la cantidad de horas en las que se presentaron temperaturas por encima de 0\celc\ y por debajo de 25\celc (Figura \ref{subfig:temp_horarias_ext}) usando los valores validados de las estaciones automáticas.\\

\begin{figure}[H]
    \centering

	\begin{subfigure}[b]{0.45\textwidth}
        \caption{Número de horas por año que la temperatura estuvo por debajo de 0\celc.}
	\includegraphics[draft=false, scale=0.45]{prabha/grafica2/bajas_tmp.png}
    \label{subfig:tmp_0}
	\end{subfigure}
	~
		\begin{subfigure}[b]{0.45\textwidth}
        \caption{Número de horas por año que la temperatura estuvo por encima de 25\celc.}
	\includegraphics[draft=false, scale=0.45]{prabha/grafica2/altas_tmp.png}
    \label{subfig:tmp_25}
	\end{subfigure}
	~

\caption{Número de horas por año en las cuales la temperatura estuvo por debajo de 0\celc\ o por encima de 25\celc.}	
\label{subfig:temp_horarias_ext}	
\end{figure}

Para el caso de las temperaturas bajo 0\celc\ podemos observar que hay varios picos estos picos están asociados a años en los que se presentó eventos el Niño (2006/2007, 2009/2010 y 2014/2015/2016; \citep{NOAA-ORI}). La estación que presentó la mayor cantidad de valores de horas bajo 0\celc\ en la Figura \ref{subfig:tmp_0} es la estación Hda Santa Ana en el municipio de Nemocón a 2572 msnm. La estación que presentó la mayor cantidad de horas anuales con temperaturas por encima de los 25\celc\ es la estación La Capilla Autom en el municipio de a 1917 msnm, la estación Hda Santa Ana se encuentra en un valle entre montañas, lo que implica que las masas de aire frío se puedan depositar en este lugar. La estación La Capilla Autom se encuentra en el município de La Capilla el cual tiene una temperatura promedio de 18\celc, entonces temperaturas por encima de 25\celc\ no son extrañas en esta región.\\

Para poder realizar una comparación de los valores obtenidos con el modelo WRF y los datos observados se usó el promedio del error (PE), Ecuación \ref{eq:mbe}. Este estadístico nos ayuda a determinar cuando el modelo presenta sobreestimación o subestimación. Este estadístico fue calculado a nivel horario.\\

\begin{equation}\label{eq:mbe}
MBE = \mathlarger{\frac{1}{n} \sum_{i=1}^n (x'_i - x_i)}
\end{equation}

Donde $n$ es el número de datos, $x_i$ corresponde a los registros de las estaciones automáticas, $x'_i$ corresponde a los datos modelados.\\

%%%% Voy acá

Se calculó el PE para todas las estaciones con la configuración icm, estos valores fueron promediados y se les calculó la desviación estándar a este Grupo de Estaciones (GE). Las desviaciones estándar de GE se encuentran representadas por líneas verdes. Adicionalmente, se calculó el PE para la estación Tibaitatá usando las configuraciones IDEAM-Colombia, icm e icm-mp\_physics 3, con la finalidad de comparar el desempeño de estas configuraciones. Este procedimiento se realizó para cada caso y  para las siguientes variables: temperatura del aire, punto de rocío, bulbo húmedo y rapidez del viento.\\


%Para el cálculo del sesgo horario se usó la fórmula de la ecuación \ref{eq:mbe} y los valores se promediaron dependiendo la cantidad de días de cada caso. El análisis para todas las estaciones se realizó con la configuración icm. Se tomaron todas las estaciones y se excluyó la estación Tibaitatá, ya que solo para esta estación se analizará la configuración IDEAM-Colombia e icm a forma de ejemplo.\\

%Para este análisis se tomaron los valores horarios de las variables temperatura del aire, punto de rocío, bulbo húmedo y rapidez del viento, estos valores fueron promediados y se les sacó la desviación estándar. Dentro de este grupo de datos no se incluyó la estación Tibaitatá, ya que esta fue analizada por separado.\\

%Se realizó una valoración del rendimiento de la configuración IDEAM-Colombia, icm e icm-mp\_physics 3 frente a los datos de temperatura de las estaciones automáticas. Se tomó como referencia la estación Tibaitatá ya que es una de las estaciones de referencia del IDEAM. Se incluyó la configuración IDEAM-Colombia para evaluar las nuevas configuraciones icm e icm-mp\_physicss 3 frente a la configuración del IDEAM.\\

\subsection{Caso 1}

Los valores de rapidez del viento de la estación Tibaitatá en el caso 1 no fueron suficientes para hacer una gráfica, por esa razón solo para el caso 1 se tendrá en cuenta la estación La Boyera, ubicada en Ubaté para analizar la rapidez del viento.
	
\begin{figure}[H]
    \centering
    \begin{subfigure}[b]{0.45\textwidth}
        \caption{PE para la temperatura del aire.}
	\includegraphics[draft=false, scale=0.45]{prabha/grafica4abcd_final/200702_tmp_2m.png}
    \label{subfig:tmp_0_caso1}
	\end{subfigure}
	~
	    \begin{subfigure}[b]{0.45\textwidth}
	        \caption{PE para el punto de rocío.}
	\includegraphics[draft=false, scale=0.45]{prabha/grafica4abcd_final/200702_Td.png}

    \label{subfig:td_caso1}
	\end{subfigure}
	~
	    \begin{subfigure}[b]{0.45\textwidth}
	\caption{PE para el bulbo húmedo.}
	\includegraphics[draft=false, scale=0.45]{prabha/grafica4abcd_final/200702_wb.png}
    \label{subfig:wb_caso1}
	\end{subfigure}
	~
	    \begin{subfigure}[b]{0.45\textwidth}
	\caption{PE para la rapidez del viento.}	
	\includegraphics[draft=false, scale=0.45]{prabha/grafica4abcd_final/200702_vel_vi10.png}
    
    \label{subfig:vel_caso1}
	\end{subfigure}
	~

\caption{PE para el caso 1 comprendido entre el 3 y el 4 de febrero de 2007. Los triángulos representan el dominio 2 y los cuadrados representan el dominio 1. Las figuras verdes representan los valores del GE donde las líneas verdes representan la desviación estándar.}	
\label{subfig:mbe_caso1}	
\end{figure}

%Cómo se calculó el punto de rocío y la tmp de bulbo húmedo y su importancia

La temperatura del aire para GE presenta un buen comportamiento a través del día, pero entre las 7 y 19 horas hay una tendencia a que WRF subestime los valores de temperatura llegando hasta -3\celc, en la mayoría de las horas hay una desviación estándar homogénea entre los 4 y -4. Para las configuraciones icm, icm-mp\_physics 3 en la estación Tibaitatá podemos observar que subestiman los valores de temperatura llegando hasta -6\celc\ a las 14 horas y en la noche los valores son sobreestimados alcanzando valores de 8\celc\ a las 4 horas, la configuración que presenta los valores más lejanos a 0 es icm-mp\_physics 3, ver Figura \ref{subfig:tmp_0_caso1}.\\


El punto de rocío para el GE presentó una sobreestimación en la mayoría de las horas la cual alcanza valores de 7\celc\ a las 6 horas y desde las 19 horas hasta las 21 horas, que se ve disminuida a las 13 horas llegando a valores de -2\celc, la desviación estándar estuvo entre 5 y -5. Para las combinaciones icm, icm-mp\_physics 3 e IDEAM-Colombia en el día se presentó una subestimación que llegó hasta -9.5\celc\ y en las horas de la noche se presentó una sobreestimación que alcanzó valores de 7\celc, en general las diferentes configuraciones se comportaron de una forma similar, pero se observó que la configuración IDEAM-Colombia presentó los valores más negativos, en comparación con las otras configuraciones, ver Figura \ref{subfig:td_caso1}.\\

El bulbo húmedo del GE presentó una sobre estimación generalizada excepto entre las 8 y 9 horas, la desviación estándar estuvo entre 5 y -5\celsius, ver Figura \ref{subfig:wb_caso1}. Las configuraciones icm e icm-mp\_physics 3 para la estación Tibaitatá presentaron una sobre estimación generalizada excepto a las 13 horas y entre las 17 y 19 horas que se presentaron valores negativos los cuales llegaron a -3\celc. La configuración IDEAM-Colombia para la estación Tibaitatá presentó los valores más lejanos a 0 en el PE en la variable de bulbo húmedo.\\

La variable rapidez del viento podemos observar que para el grupo de estaciones en las horas de la mañana hubo una sobreestimación que llegó hasta los 4 $m/s$ a las 12 horas y en las horas de la noche mejoró el comportamiento, la desviación estándar fue más alta entre las 4 y 12 horas. Las combinaciones icm, icm-mp\_physics 3 e IDEAM-Colombia en La Boyera entre las 8 y 20 horas presentaron una sobre estimación cuyo valor máximo fue de 3\celc\ las demás horas fueron subestimadas alcanzando valores de -4\celc, Figura \ref{subfig:vel_caso1}.\\


\subsection{Caso 2}

\begin{figure}[H]
    \centering
    \begin{subfigure}[b]{0.45\textwidth}
        \caption{PE para la temperatura del aire.}
	\includegraphics[draft=false, scale=0.45]{prabha/grafica4abcd_final/201408_tmp_2m.png}
    \label{subfig:tmp_0_caso2}
	\end{subfigure}
	~
	    \begin{subfigure}[b]{0.45\textwidth}
	        \caption{PE para el punto de rocío.}
	\includegraphics[draft=false, scale=0.45]{prabha/grafica4abcd_final/201408_Td.png}

    \label{subfig:td_caso2}
	\end{subfigure}
	~
	    \begin{subfigure}[b]{0.45\textwidth}
	\caption{PE para el bulbo húmedo.}
	\includegraphics[draft=false, scale=0.45]{prabha/grafica4abcd_final/201408_wb.png}
    \label{subfig:wb_caso2}
	\end{subfigure}
	~
	    \begin{subfigure}[b]{0.45\textwidth}
	\caption{PE para la rapidez del viento.}	
	\includegraphics[draft=false, scale=0.45]{prabha/grafica4abcd_final/201408_vel_vi10.png}
    
    \label{subfig:vel_caso2}
	\end{subfigure}
	~

\caption{PE para el caso 2 comprendido entre el 29 y 31 de agosto de 2014. Los triángulos representan el dominio 2 y los cuadrados representan el dominio 1. Las figuras verdes representan los valores del GE donde las líneas verdes representan la desviación estándar.}	
\label{subfig:mbe_caso2}	
\end{figure}

%Cómo se calculó el punto de rocío y la tmp de bulbo húmedo y su importancia

El GE presenta subestimaciónes entre 0\celsius\ y -4\celsius\ durante las 24 horas siendo menor a las 13 horas con valores promedio de 0\celc, la desviación estándar de mantiene en promedio constante con valores entre 3.5 y -3.5\celc. Las combinaciones icm, icm-mp\_physics 3 e IDEAM-Colombia varían entre 2 y -2\celc\ alrededor del valor de 0\celc\ (figura \ref{subfig:tmp_0_caso2}).\\

Para el punto de rocío, se puede observar que GE presentó sobreestimación en la mayoría de las horas y la desviación estándar aumentó desde las 21 horas hasta las 6 horas (Figura \ref{subfig:td_caso2}). Las combinaciones icm, icm-mp\_physics 3 e IDEAM-Colombia presentaron una subestimación que llegó hasta -6\celc\ y en las horas de la noche se presentó una sobreestimación similar a la que presentó el grupo de estaciones llegando hasta 4\celc.\\

Para la temperatura de bulbo húmedo, el GE presentaron valores cercanos a 0\celc\ con tendencia a la sobreestimación llegando a valores promedio de 1\celc, la desviación estándar se mantuvo relativamente constante entre 2.3 y -2.3\celsius\ (Figura \ref{subfig:wb_caso2}). Las combinaciones icm, icm-mp\_physics 3 e IDEAM-Colombia para la estación Tibaitatá presentaron una sobreestimación más alta en comparación con GE, en las horas de la mañana se presentaron valores subestimados que alcanzaron -3\celc.\\

Para la rapidez del viento, podemos observar que tanto GE como las combinaciones icm, icm-mp\_physics 3 e IDEAM-Colombia presentaron un comportamiento similar, en el cual están entre 0 y 2 $m/s$ y se presenta una sobreestimación entre las 13 y 19 horas que alcanza valores máximos promedios de 3 $m/s$, la desviación estándar de GE presenta una disminución entre las 21 horas y las 5 horas, ver Figura \ref{subfig:vel_caso2}.\\


\subsection{Caso 3}

\begin{figure}[H]
    \centering
    \begin{subfigure}[b]{0.45\textwidth}
        \caption{PE para la temperatura del aire.}
	\includegraphics[draft=false, scale=0.45]{prabha/grafica4abcd_final/201508_tmp_2m.png}
    \label{subfig:tmp_0_caso3}
	\end{subfigure}
	~
	    \begin{subfigure}[b]{0.45\textwidth}
	        \caption{PE para el punto de rocío.}
	\includegraphics[draft=false, scale=0.45]{prabha/grafica4abcd_final/201508_Td.png}

    \label{subfig:td_caso3}
	\end{subfigure}
	~
	    \begin{subfigure}[b]{0.45\textwidth}
	\caption{PE para el bulbo húmedo.}
	\includegraphics[draft=false, scale=0.45]{prabha/grafica4abcd_final/201508_wb.png}
    \label{subfig:wb_caso3}
	\end{subfigure}
	~
	    \begin{subfigure}[b]{0.45\textwidth}
	\caption{PE para la rapidez del viento.}	
	\includegraphics[draft=false, scale=0.45]{prabha/grafica4abcd_final/201508_vel_vi10.png}
    
    \label{subfig:vel_caso3}
	\end{subfigure}
	~

\caption{PE para el caso 3 comprendido entre el 27 y 29 de octubre de 2015.. Los triángulos representan el dominio 2 y los cuadrados representan el dominio 1. Las figuras verdes representan los valores del GE donde las líneas verdes representan la desviación estándar.}	
\label{subfig:mbe_caso3}	
\end{figure}

%Cómo se calculó el punto de rocío y la tmp de bulbo húmedo y su importancia

El GE presenta una sobreestimación entre las 8 y las 18 horas alcanzando valores de 4\celc\ y en las demás horas presenta subestimación que llega a valores de -4\celc\ a las 2 y 6 horas, la desviación estándar se mantiene constante a lo largo de las horas del día entre 4 y -4\celsius\ (Figura \ref{subfig:tmp_0_caso3}). Las combinaciones icm, icm-mp\_physics 3 e IDEAM-Colombia presentan la mayor subestimación a las 9 horas llegando a valores de -8\celc, en la mayoría de las horas del día se presenta sobreestimación que llega hasta valores de 5\celc. Se puede observar que la configuración que presenta más sobreestimación es IDEAM-Colombia y la configuración que presenta la más alta subestimación es la configuración idm-mp\_physics 3.\\


La temperatura del punto de rocío para el GE presentó sobreestimación en todas las horas y la desviación estándar se mantuvo constante entre 2.5 y -2.5\celsius, (Figura \ref{subfig:td_caso3}). Las combinaciones icm, icm-mp\_physics 3 e IDEAM-Colombia presentaron subestimación que llegó hasta -3\celc\ y en las horas de la noche se presentó una sobreestimación similar a la que presentó el grupo de estaciones llegando hasta 5.5\celc.\\

Para la temperatura de bulbo húmedo se presentó un comportamiento similar al presentado en la temperatura del aire, ya que para el GE hubo una sobreestimación de la temperatura de bulbo húmedo entre 7 y 22 horas cuyos valores ascendieron en promedio hasta 2.4\celsius, para posteriormente presentar una subestimación que llegó hasta -1\celsius, la desviación estándar se mantuvo relativamente constante entre 1.7\celc\ y -1.7\celc (Figura \ref{subfig:wb_caso3}). Las combinaciones icm, icm-mp\_physics 3 e IDEAM-Colombia presentaron los menores valores a las 9 horas, llegando a valores cercanos de -4\celsius, posteriormente se presentó un aseso hasta llegar al pico máximo con valores de 4\celsius\ a las 21 horas.\\

Para la rapidez del viento podemos observar que el GE presentó una sobreestimación y solo entre las 0 horas y 23 horas los valores estuvieron cercanos a 0 $m/s$, la desviación estándar se mantuvo en promedio entre 3 y -3 $m/s$ (Figura \ref{subfig:vel_caso3}). Las combinaciones icm, icm-mp\_physics 3 e IDEAM-Colombia tuvieron en la mayoría de las horas presentaron sobreestimación, la configuración IDEAM-Colombia presentó una subestimación a las 8 horas y la configuración icm-mp\_physics presentó los valores más altos a las 21 horas.\\


\subsection{Caso 4}

\begin{figure}[H]
    \centering
    \begin{subfigure}[b]{0.45\textwidth}
        \caption{PE para la temperatura del aire.}
	\includegraphics[draft=false, scale=0.45]{prabha/grafica4abcd_final/201509_tmp_2m.png}
    \label{subfig:tmp_0_caso4}
	\end{subfigure}
	~
	    \begin{subfigure}[b]{0.45\textwidth}
	        \caption{PE para el punto de rocío.}
	\includegraphics[draft=false, scale=0.45]{prabha/grafica4abcd_final/201509_Td.png}

    \label{subfig:td_caso4}
	\end{subfigure}
	~
	    \begin{subfigure}[b]{0.45\textwidth}
	\caption{PE para el bulbo húmedo.}
	\includegraphics[draft=false, scale=0.45]{prabha/grafica4abcd_final/201509_wb.png}
    \label{subfig:wb_caso4}
	\end{subfigure}
	~
	    \begin{subfigure}[b]{0.45\textwidth}
	\caption{PE para la rapidez del viento.}	
	\includegraphics[draft=false, scale=0.45]{prabha/grafica4abcd_final/201509_vel_vi10.png}
    
    \label{subfig:vel_caso4}
	\end{subfigure}
	~

\caption{PE para el caso 3 comprendido entre el 7 y el 9 de septiembre de 2015. Los triángulos representan el dominio 2 y los cuadrados representan el dominio 1. Las figuras verdes representan los valores del GE donde las líneas verdes representan la desviación estándar.}	
\label{subfig:mbe_caso4}	
\end{figure}

%Cómo se calculó el punto de rocío y la tmp de bulbo húmedo y su importancia

El GE presentaron sobreestimación de la temperatura del aire entre las 8 y 16 horas alcanzando valores de 4\celc, en las demás horas presenta subestimación la cual alcanzó sus valores mínimos a las 15 horas, la desviación estándar en promedio estuvo entre 4.5 y -4.5\celc, (Figura \ref{subfig:tmp_0_caso2}). Las combinaciones icm, icm-mp\_physics 3 e IDEAM-Colombia estuvieron presentaron resultados entre -6.5\celc\ y 3\celc\, el mínimo valor fue registrado a las 15 horas por la configuración icm-mp\_physics 3 y el mayor valor fue reportado por la configuración IDEAM-Colombia.\\

Para la temperatura de punto de rocío se puede observar que el GE, presentó una sobre estimación desde las 7 horas hasta las 23 horas y posteriormente presentó valores cercanos a 0\celc, la desviación estándar fue en promedio 1\celc\ (Figura \ref{subfig:td_caso2}). Las combinaciones icm, icm-mp\_physics 3 e IDEAM-Colombia presentaron subestimación desde las 6 hasta las 17 horas, posteriormente se evidenció una sobreestimación que llegó hasta 4\celsius. 

Para el bulbo húmedo el GE, los valores variaron entre -3.5 y 2.2\celc\ presentando sobreestimación en las horas de la mañana y subestimación en las horas de la tarde, hubo una desviación estándar de 3\celsius\ en las horas del día y en las horas de la noche la desviación disminuyó hasta valores de 2\celc\ (Figura \ref{subfig:wb_caso2}). Las combinaciones icm, icm-mp\_physics 3 e IDEAM-Colombia presentaron variaciones entre 2.5\celc\ y -4.5\celc. 

En la variable rapidez del viento, podemos observar que GE y las combinaciones icm, icm-mp\_physics 3 e IDEAM-Colombia se presentaron un comportamiento similar donde las altas sobreestimaciones se presentaron entre las 16 y las 18 horas llegando a valores de 3.7 $m/s$ (Figura \ref{subfig:vel_caso2}). Se puede observar que al igual que en todos los casos en las horas de la noche se presentan valores cercanos a 0 $m/s$ lo que implica que en la noche los valores de rapidez del viento presentan más similitud entre los valores modelados y los valores reales.\\

En los casos 2, 3 y 4 podemos observar que para el PE para la temperatura del aire del GE entre las 6 horas y las 17 horas hay un pico, posterior el PE se estabiliza. El GE en los casos 3 y 4 para el PE para la temperatura se observar que entre las 6 horas y las 17 horas hay sobreestimación y posterior a estas horas hay subestimación.\\

Para la rapidez del viento vemos que en los casos 2, 3 y 4 podemos observar que aproximadamente entre las 6 horas y las 21 horas hay altos valores de PE, en las demás horas el error se disminuye.\\

El GE para todos los casos presentó los valores del PE para la temperatura, temperatura de bulbo húmedo y punto de rocío se presentó que los promedios del dominio 1 fueron mayores a los valores del dominio 2.\\

%Los mayores valores de MBE para la temperatura se presentaron para el caso 1 en las horas de la noche, este comportamiento se puede observar en la gráfica \ref{caso1_tiba_wrf}. En el caso 2 se evidenció que había una gran desviación estándar y que los datos en general tienden a ser subestimados. En los casos de altas temperaturas se puede observar que el conjunto de datos se comporta de una forma similar, ya que en el día se presenta una sobre estimación y en las horas de la noche hay una subestimación los valores en ambos casos oscilan entre 4 y -4\celc\ y las desviaciones estandar hacen que los valores lleguen desde -7\celc\ hasta 7\celc. En el caso 1, 3 y 4 la configuración icm presentó mejores resultados que la del IDEAM-Colombia.\\

%En general para los 4 casos de estudio la rapidez del viento fue siempre sobre estimada, solo para la estación La Boyera se puede observar una subestimación clara. Se evidencia en los 4 casos una sobre estimación entre las 16 y 17 horas en todos los casos y en las horas de la noche se observa una mejora en los resultados.\\

%La temperatura de bulbo húmedo para los casos 1 y 2 se comportó de una forma similar ya que en general se sobre estimó con algunas horas en las que fue sub estimada en las horas de la mañana. En los casos 3 y 4 se presentó una sobre estimación en el día que para el caso 3 se prolongó hasta las 23 horas. hay una gran desviación estándar y  los valores promedios oscilaron entre los 4 y los -2\celc.\\


\section{Comparación de temperaturas extremas observadas y simuladas con configuraciones icm e icm-mp\_physics 3 e IDEAM-Colombia.}

Para cada caso se realizó una comparación entre los datos de la configuración icm e icm-mp\_physics 3 del dominio 2 frente a los valores diarios de temperaturas extremas (máximas y mínimas) y la temperatura a las 18 horas y 22 horas. Estas horas fueron escogidas ya que según \citet{snyder2005frost} estas horas son importantes para la predicción de las temperaturas extremas del día siguiente \citet{prabha2008}, ver Figura \ref{subfig:tmp_ext_caso4_d01}.\\

En la Figura \ref{subfig:tmp_ext_caso4_d01} se puede observar el comportamiento de los datos de la configuración icm frente a los datos registrados de las estaciones automáticas. Para el Caso 1 (Figura \ref{subfig:tmp_ext_caso1_d01}) se observa que la configuración icm presenta valores similares a los de la estación automática, pero para las temperaturas mínimas se puede observar que el modelo no es capaz de reproducir correctamente los valores temperatura del aire.\\

Para el Caso 2 se observa que en general los datos de temperatura de la configuración icm estuvieron cercanos a los datos de las estaciones, pero existe un sesgo en el que el modelo da como resultado valores inferiores a los valores reportados en las estaciones automáticas, ver Figura \ref{subfig:tmp_ext_caso2_d01}.\\

Para el Caso 3 se observa que hay una distribución homogénea de los datos, ver Figura \ref{subfig:tmp_ext_caso3_d01}. En el Caso 3 se puede observar que hay temperaturas máximas que el la configuración icm está sobreestimando. Para los datos extremos podemos observar que el modelo posee cierta dificultad, ya que algunos fueron sobreestimados y otros fueron subestimados. Por el contrario los valores mínimos estuvieron mejor representados por la configuración icm.\\

En el Caso 4, se  puede observar que hay temperaturas máximas que fueron subestimadas por la configuración, pero en general se observa una buena respuesta del modelo frente a los datos de las estaciones automáticas.\\

En los Casos 1 y 2 que son los casos de las heladas podemos observar cierta dificultad de la configuración para representar valore inferiores a 0\celsius, pero para estos dos casos presentó un buen comportamiento en para las altas temperaturas. Para los Casos 3 y 4 se observa un mejor ajuste de las temperaturas modeladas y las temperaturas observadas.\\



\begin{figure}[H]
    \centering
    \begin{subfigure}[b]{0.45\textwidth}
        \caption{Temperaturas a las 18, 22, máximas y mínimas para el caso 1.}
	\includegraphics[draft=false, scale=0.45]{prabha/grafica7/200702_ideam_i_d02.png}
    \label{subfig:tmp_ext_caso1_d01}
	\end{subfigure}
	~
	    \begin{subfigure}[b]{0.45\textwidth}
        \caption{Temperaturas a las 18, 22, máximas y mínimas para el caso 2.}
	\includegraphics[draft=false, scale=0.45]{prabha/grafica7/201408_ideam_i_d02.png}

    \label{subfig:tmp_ext_caso2_d01}
	\end{subfigure}
	~
	    \begin{subfigure}[b]{0.45\textwidth}
        \caption{Temperaturas a las 18, 22, máximas y mínimas para el caso 3.}
	\includegraphics[draft=false, scale=0.45]{prabha/grafica7/201508_ideam_i_d02.png}
    \label{subfig:tmp_ext_caso3_d01}
	\end{subfigure}
	~
	    \begin{subfigure}[b]{0.45\textwidth}
        \caption{Temperaturas a las 18, 22, máximas y mínimas para el caso 4.}
	\includegraphics[draft=false, scale=0.45]{prabha/grafica7/201509_ideam_i_d02.png}
    
    \label{subfig:tmp_ext_caso4_d01}
	\end{subfigure}
	~

\caption{Comparación de las temperaturas del aire a las 18, 22, máximas y mínimas entre los valores de icm frente a los valores de las estaciones automáticas.}	
\label{subfig:tmp_ext_d01}	
\end{figure}


La configuración icm-mp\_physics 3 en los Casos 1 y 2 (Figuras \ref{subfig:tmp_ext_caso1_d01_phy} y \ref{subfig:tmp_ext_caso2_d01_phy}) presentan un comportamiento similar a la configuración icm. Pero para los Casos 3 y 4 podemos observar que la configuración icm-mp\_physics 3 tienen dificultad para generar valores de temperatura menores a 5\celc\ lo que implica que esta configuración posee dificultad para simular los valores mínimos de temperatura (Figuras \ref{subfig:tmp_ext_caso3_d01_phy} y \ref{subfig:tmp_ext_caso4_d01_phy}).

\begin{figure}[H]
    \centering
    \begin{subfigure}[b]{0.45\textwidth}
        \caption{Temperaturas a las 18, 22, máximas y mínimas para el caso 1.}
	\includegraphics[draft=false, scale=0.45]{prabha/grafica7/200702_ideam_3_d02.png}
    \label{subfig:tmp_ext_caso1_d01_phy}
	\end{subfigure}
	~
	    \begin{subfigure}[b]{0.45\textwidth}
        \caption{Temperaturas a las 18, 22, máximas y mínimas para el caso 2.}
	\includegraphics[draft=false, scale=0.45]{prabha/grafica7/201408_ideam_3_d02.png}

    \label{subfig:tmp_ext_caso2_d01_phy}
	\end{subfigure}
	~
	    \begin{subfigure}[b]{0.45\textwidth}
        \caption{Temperaturas a las 18, 22, máximas y mínimas para el caso 3.}
	\includegraphics[draft=false, scale=0.45]{prabha/grafica7/201508_ideam_3_d02.png}
    \label{subfig:tmp_ext_caso3_d01_phy}
	\end{subfigure}
	~
	    \begin{subfigure}[b]{0.45\textwidth}
        \caption{Temperaturas a las 18, 22, máximas y mínimas para el caso 4.}
	\includegraphics[draft=false, scale=0.45]{prabha/grafica7/201509_ideam_3_d02.png}
    
    \label{subfig:tmp_ext_caso4_d01_phy}
	\end{subfigure}
	~

\caption{Comparación de las temperaturas del aire a las 18, 22, máximas y mínimas entre los valores de icm-mp\_physics 3 frente a los valores de las estaciones automáticas.}	
\label{subfig:tmp_ext_icmphysiscs3}	
\end{figure}


La configuración IDEAM-Colombia presentó un comportamiento similar al presentado por la configuración icm-mp\_physics 3, ver Figura \ref{subfig:tmp_ext_icmphysiscs3}.\\

\begin{figure}[H]
    \centering
    \begin{subfigure}[b]{0.45\textwidth}
        \caption{Temperaturas a las 18, 22, máximas y mínimas para el caso 1.}
	\includegraphics[draft=false, scale=0.45]{prabha/grafica7/200702_ideam_c_d02.png}
    \label{subfig:tmp_ext_caso1_d01_ideamcol}
	\end{subfigure}
	~
	    \begin{subfigure}[b]{0.45\textwidth}
        \caption{Temperaturas a las 18, 22, máximas y mínimas para el caso 2.}
	\includegraphics[draft=false, scale=0.45]{prabha/grafica7/201408_ideam_c_d02.png}

    \label{subfig:tmp_ext_caso2_d01_ideamcol}
	\end{subfigure}
	~
	    \begin{subfigure}[b]{0.45\textwidth}
        \caption{Temperaturas a las 18, 22, máximas y mínimas para el caso 3.}
	\includegraphics[draft=false, scale=0.45]{prabha/grafica7/201508_ideam_c_d02.png}
    \label{subfig:tmp_ext_caso3_d01_ideamcol}
	\end{subfigure}
	~
	    \begin{subfigure}[b]{0.45\textwidth}
        \caption{Temperaturas a las 18, 22, máximas y mínimas para el caso 4.}
	\includegraphics[draft=false, scale=0.45]{prabha/grafica7/201509_ideam_c_d02.png}
    
    \label{subfig:tmp_ext_caso4_d01_ideamcol}
	\end{subfigure}
	~

\caption{Comparación de las temperaturas del aire a las 18, 22, máximas y mínimas entre los valores de IDEAM-Colombia frente a los valores de las estaciones automáticas.}	
\label{subfig:ideam_colombia}	
\end{figure}



%\begin{figure}[H]
%    \centering
%    \begin{subfigure}[b]{0.45\textwidth}
%        \caption{icm.}
%	\includegraphics[draft=false, scale=0.45]{prabha/grafica7/201602_ideam_i_d02.png}
%    \label{subfig:tmp_ext_caso1_d01_ideamcol}
%	\end{subfigure}
%	~
%	    \begin{subfigure}[b]{0.45\textwidth}
%        \caption{colombia.}
%	\includegraphics[draft=false, scale=0.45]{prabha/grafica7/201602_ideam_c_d02.png}
%
%    \label{subfig:tmp_ext_caso2_d01_ideamcol}
%	\end{subfigure}
%	~
%	    \begin{subfigure}[b]{0.45\textwidth}
%        \caption{icm-3.}
%	\includegraphics[draft=false, scale=0.45]{prabha/grafica7/201602_ideam_3_d02.png}
%    \label{subfig:tmp_ext_cas31_d01_ideamcol}
%	\end{subfigure}
%	~
%    \begin{subfigure}[b]{0.45\textwidth}
%        \caption{icm.}
%	\includegraphics[draft=false, scale=0.45]{prabha/grafica7/201712_ideam_i_d02.png}
%    \label{subfig:tmp_ext_caso1_d01_ideamcol}
%	\end{subfigure}
%	~
%	    \begin{subfigure}[b]{0.45\textwidth}
%        \caption{colombia.}
%	\includegraphics[draft=false, scale=0.45]{prabha/grafica7/201712_ideam_c_d02.png}
%
%    \label{subfig:tmp_ext_caso2_d01_ideamcol}
%	\end{subfigure}
%	~
%	    \begin{subfigure}[b]{0.45\textwidth}
%        \caption{icm-3}
%	\includegraphics[draft=false, scale=0.45]{prabha/grafica7/201712_ideam_3_d02.png}
%    \label{subfig:tmp_ext_cas31_d01_ideamcol}
%	\end{subfigure}
%	~
%
%
%
%\caption{Comparación de las temperaturas del aire a las 18, 22, máximas y mínimas entre los valores de IDEAM-Colombia frente a los valores de las estaciones automáticas.}	
%\label{subfig:ideam_colombia}	
%\end{figure}




\section{Conclusiones}

\begin{itemize}

    \item Los datos de radiación de las estación automática presenta fallas en las hora de la noche.
    
    \item Los resultados de radiación para las configuraciones IDEAM-Colombia e icm en general tienden a mostrar valores mayores a los reportados por la estación automática.
    
    \item En general los resultados de la precipitación presentan eventos que las observaciones no muestran, esto sucede con todas las configuraciones. Las sobreestimaciónes fueron más altas para las configuraciones icm e icm-mp\_physics 3.
    
    \item Las configuraciones icm-mp\_physics 3, icm e IDEAM-Colombia presentaron buenos resultados cuando se simularon altas temperaturas.
    
    \item La configuración icm presentó los resultados simulados menos lejanos a los valores observados con respecto a los valores de temperatura mínima. 
    
    \item En el Caso 1 fue el caso que presentó la mayor dificultad para reproducir el comportamiento con el modelo WRF en especial para las variables temperatura del aire, punto de rocío y rapidez del viento.
\end{itemize}
% de acá se sacó la formula de dew point dewpoint https://iridl.ldeo.columbia.edu/dochelp/QA/Basic/dewpoint.html
