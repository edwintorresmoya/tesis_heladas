\documentclass[11pt]{book}
\raggedbottom % Esta función es para que se reduzcan los espacios entre los párrafos.
\usepackage[utf8]{inputenc} % Usado par la recepción de las tildes
\usepackage{fullpage}
\usepackage{gensymb}%Sacar nuevo símbolos
\usepackage{apacite} % Citar con APA
\usepackage[hidelinks]{hyperref}
\usepackage{natbib}
\usepackage{csvsimple}%To read the documents in csv

\usepackage[final]{graphicx}
\usepackage{subcaption}%Usado para crear suplots
\graphicspath{ {graph/} }%Ubicación de las gráficas
\def\celc{$^{\circ}$C}%Código para el grados celsius

\usepackage[final]{graphicx}
\usepackage{subcaption}%Usado para crear suplots

\usepackage{float}%Control de los nombres debajo de la gráfica
\usepackage{pdflscape}%Lanscape
\usepackage{lscape}%Lanscape
\usepackage{longtable} % usada para crear grandes tablas
\usepackage[usenames, dvipsnames]{color} %Uso del color
\usepackage{booktabs} % para tomar las tablas desde python
\usepackage{multirow}
\usepackage{ulem} % para hacer tachaduras
\usepackage[draft]{todonotes}
% Para tablas grandes
\usepackage{longtable}
\usepackage{lipsum}
\usepackage{comment}% Usado para comentar
%Para el loop de las gráficas
\usepackage{relsize}
\usepackage[T1]{fontenc} % usado para poder usar el símbolo de < >
\usepackage{upgreek}
\usepackage[none]{hyphenat} % para evitar el corte de las palabras

\usepackage{appendix} % Usado para los anexos
\renewcommand{\appendixname}{Anexos}
\renewcommand{\appendixtocname}{Anexos}
\renewcommand{\appendixpagename}{Anexos}


\renewcommand{\tablename}{\textbf{Tabla}}
\renewcommand{\figurename}{\textbf{Figura}}
\renewcommand{\listtablename}{Lista de Tablas}
\renewcommand{\listfigurename}{Lista de Figuras}
\renewcommand{\contentsname}{Contenido}


%\usepackage[spanish]{babel}
\usepackage[spanish,es-tabla]{babel} % para cambiar cuadro por tabla
%\usepackage[draft]{article}



\begin{document}
\sloppy  % para que no se corten las palabras 
%\newpage
%\setcounter{page}{1}
%\begin{center}
%\begin{figure}
%\centering%
%\epsfig{file=HojaTitulo/EscudoUN.eps,scale=1}%
%\end{figure}
%\thispagestyle{empty} \vspace*{2.0cm} \textbf{\huge
%T\'{\i}tulo de la tesis  o trabajo de investigaci\'{o}n}\\[6.0cm]
%\Large\textbf{Nombres y apellidos completos del autor}\\[6.0cm]
%\small Universidad Nacional de Colombia\\
%Facultad, Departamento (Escuela, etc.)\\
%Ciudad, Colombia\\
%A\~{n}o\\
%\end{center}

\newpage{\pagestyle{empty}\cleardoublepage}

\newpage
\begin{center}
\thispagestyle{empty} \vspace*{0cm} \textbf{\huge Efectos potenciales de Temperaturas Extremas del Aire en la Sabana de Bogotá Sobre la Producción de un Cultivo de Importancia para la Seguridad Alimentaria: Un estudio con un Modelo Regional de Pronóstico del Tiempo Atmosférico y Un Modelo de Simulación de Cultivo. }\\[3.0cm]
\Large\textbf{Edwin Torres Moya}\\[3.0cm]
%\small Tesis o trabajo de grado presentada(o) como requisito parcial para optar al
%t\'{\i}tulo de:\\
%\textbf{Indicar el t\'{\i}tulo que se obtendr\'{a}. Por ejemplo, Magister en Ingenier\'{\i}a Qu\'{\i}mica}\\[2.5cm]
%Directores:\\
%Ph.D., Astrid Baquero Bernal y M.Sc Eliecer Díaz Almanza\\[2.0cm]
Directora:\\
Ph.D., Astrid Baquero Bernal\\[2.0cm]
%L\'{\i}nea de Investigaci\'{o}n:\\
%Nombrar la l\'{\i}nea de investigaci\'{o}n en la que enmarca la tesis  o trabajo de investigaci\'{o}n\\
Grupo de Investigaci\'{o}n:\\
Grupo de simulación del sistema climático terrestre\\[2.5cm]
Universidad Nacional de Colombia\\
Facultad de Ciencias, Meteorología\\
Bogotá, Colombia\\
A\~{n}o 2018\\
\end{center}
 % Portada


\tableofcontents
\newpage

%\chapter{Introducción}
La agricultura en la Sabana de Bogotá es de gran importancia para el abastecimiento alimentario de vairas ciudades, incluyendo Bogotá. En la sabana de Bogotá históricamente se han presentado fenómenos de bajas temperaturas, algunos de estos llamados heladas. Hoy en día las áreas en las que se puede hacer agricultura se han venido reduciendo, ya que hay una competencia de los terrenos con la construcción, adicionalmente la frontera agrícola se encuentra restringida por la delimitación de páramos. Los aumentos de temperatura que se han experimentado en la Sabana de Bogotá se han venido convirtiendo en un problema para los productores, ya que las variedades de papa tradicionales no están adaptadas a las temperaturas extremas que se están presentando en la Sabana de Bogotá.\\

Por esta razón nace la idea de este trabajo. Ya que, la predicción de las altas y las bajas temperaturas puede ser una herramienta que ayude a los agricultores. Desafortunadamente las medidas para afrontar las heladas no son tan fácil de implemetar, porque se requiere una inversión previa en infraestructura y en la muchos casos estas medidas están fuera del alcance de un gran grupo de agricultores. A pesar de estas situaciones, este trabajo tiene como finalidad mejorar la predicción de un modelo meteorológico con la finalidad de ofrecer mejores pronósticos, ya que cualquier aporte que mejore la situación de los productores en la Sabana de Bogotá será un gran avance.\\

Cuando se emprendió este trabajo se tenia claro cuáles eran los objetivos, pero no sabíamos las dificultades por las que se iba a pasar. Por esta razón este trabajo se divide en una parte de exploración que es el capítulo 1, otra parte que hace parte de la limpieza de datos de la estaciones automáticas capitulo 2 y comparación de los resultados del modelo meteorológico con los datos reales para definir dominios, mejores parametrizaciones y tiempos de inicio capitulo 3.\\



\chapter{Objetivos}

\section{Objetivo general}

\begin{itemize}
    \item Analizar las condiciones meteorológicas asociadas con heladas y temperaturas altas nocivas para el cultivo de papa (\textit{Solanum tuberosum}) en la Sabana de Bogotá a partir de datos de estaciones meteorológicas y de un modelo regional de clima.
    
\end{itemize}

\section{Objetivos específicos}

\begin{itemize}
    \item Identificar las características de ocurrencia bajo las cuales se presentan las heladas y temperaturas altas nocivas para el cultivo de papa en la Sabana de Bogotá teniendo en cuenta duración, hora del día, mes del año y año.
    
    \item Identificar las características meteorológicas generales bajo las cuales se presentan las heladas y temperaturas altas nocivas para el cultivo de papa en la Sabana de Bogotá teniendo en cuenta la hora del día, mes del año y año.
    
    \item Identificar las características meteorológicas específicas bajo las cuales se presentan las heladas y temperaturas altas nocivas para el cultivo de papa en la Sabana de Bogotá a través de casos de estudio.
    
    \item Realizar un diagnóstico de la habilidad de un modelo regional de clima para reproducir las condiciones meteorológicas en que se presentan las heladas y las temperaturas altas nocivas para el cultivo de papa en casos de estudio usando diferentes parametrizaciones de la capa límite planetaria.
\end{itemize} 
%%Actividad: Realizar una revisión bibliográfica sobre modelos físicos regionales de pronóstico del tiempo usados para el pronóstico de las temperaturas máxima y mínima.

\chapter{Capítulo 1}
\section{Introducción}

La papa (\textit{Solanum tuberosum} L.) es el tercer cultivo de mayor importancia en el mundo después del arroz y el trigo. La producción global excede las 300 millones de toneladas, por esta razón el cultivo de la papa es de gran importancia en términos de seguridad alimentaria \citep{birch2012crops}.\\

La temperatura óptima para el crecimiento y desarrollo del cultivo de papa se encuentra en un rango entre 14 y 22\celc \ , cuando las temperaturas del aire están por fuera de estos rangos el rendimiento puede decrecer dependiendo del estado fenológico en el que se encuentre \citep{Hijmans2003}. La temperatura del aire es considerada como el factor meteorológico sobre el que menos se tiene control y que afecta significativamente el crecimiento y rendimiento de los cultivos de papa \citep{hancock2014physiological}. Respecto al efecto del calentamiento global en la producción de papa, se ha predicho una disminución entre el 18 y 32\% para el año 2050 \citep{hijmans2003effect}. \\

El cultivo de la papa en Colombia es uno de los cultivos más importantes, en el 2017 el Fondo Nacional de Fomento de la Papa (Fedepapa) aseguró que la producción nacional en el 2017 fue de 2.7 millones de toneladas de las  cuales el 90\% de la papa que se produce en Colombia y el 10\% es importada \citep{Portafolio2017}. Una de las zonas de mayor importancia para la producción de papa es el Altiplano Cundiboyacence. Según \citet{Barrientos2014} e \citet{IDEAM2009} el cambio climático en la zona se ha manifestado con un incremento de la temperatura media y ampliación del rango de mínimos y máximos. Este tipo de cambios probablemente conllevará a reducir la productividad \citep{Hatfield2015}.\\

\section{Generalidades del cultivo de papa}

\subsection{Estadística agrícola del cultivo de papa en Colombia}

El \citet{DANE2002} reportó que el principal departamento en producción de papa en Colombia es Cundinamarca, seguido por Boyacá, Nariño y Antioquia, estos departamentos suman el 85\% de la producción total del país. En el 2014 Cundinamarca se ubicó como el departamento con mayor área sembrada de papa en Colombia con 163358 hectáreas (ha), correspondientes al 38.4\% del total del área sembrada \citep{MADR2014}. En el primer período de 2017, Cundinamarca fue el mayor productor del cultivo de papa, debido a que tiene la mayor área sembrada, área cosechada, producción y rendimiento de papa, seguido por Boyacá, Nariño y Antioquia como se muestra el la Tabla \ref{table:1}, la cual fue realizada con la información proveniente del enlace \textcolor{blue}{ \href{https://www.datos.gov.co/Agricultura-y-Desarrollo-Rural/Cadena-Productiva-Papa-Area-Producci-n-Y-Rendimien/pnsj-t3kh}{Cadena Productiva Papa-Área Producción Y Rendimiento}} \citep{madr2017}.\\

Basado en la anterior información, Cundinamarca es el departamento con mayor importancia en la producción de papa. Según los datos de \citet{madr2017} el área de siembra y  el área cosechada poseen un comportamiento similar a través de los años con algunas diferencias (ver Figura \ref{gra:papa_cund_1}), en el 2007 se presenta una diferencia entre los valores de área sembrada y de área cosechada en Cundinamarca. En el departamento de Cundinamarca las áreas de siembra y cosecha han variado en el tiempo y han presentado 4 períodos de altos valores en el 2006,  2010, 2011, 2013 y uno recientemente en el 2017. La producción  presentó un aumento entre el segundo semestre del año 2008 y el primer semestre del año  2010, seguido por una disminución hasta el segundo semestre del año 2013 (ver Figura \ref{gra:papa_cund_2}). A partir del segundo semestre del año 2014 se presentó un aumento casi continuo, comportamiento similar al presentado en el rendimiento. Al parecer las tendencias son al alza con respecto a los valores de área, producción y rendimiento. En cuanto a la producción de papa en Cundinamarca se registró una alta producción para el año 2010 y luego la producción cayó, actualmente se encuentra con una tendencia al aumento. El rendimiento presenta un comportamiento similar al presentado por la producción, ya que para el año 2010 se presentó una alza seguido de un decrecimiento en el rendimiento, para finalmente presentar un aumento (ver Figura \ref{gra:papa_cund_3}). %No entiendo por qué la profe lo quiere quitar, ¿dado el caso de eliminarlo tendría que eliminar la Figura?


\begin{table}[H] % el H mayúscula significa que el archivo debe ir en ese lugar, de lo contrario la tabla se baja
\begin{center}
	\caption{Estadística agrícola del cultivo de papa en el primer semestre del 2017 para los principales departamentos productores. Fuente \citep{madr2017}.}
    \label{table:1}
	\begin{tabular}{cccccccccc}
Departamento&Área Sembrada(ha)&Área Cosechada(ha)&Producción(t)&Rendimiento(t/ha)&\\
\hline
Cundinamarca&32036&32034&812218&20.8&\\
Boyacá&25975&25971&462780&14.49&\\
Nariño&20391&20391&357156&14.75&\\
Antioquia&3515&3504&77618&17.25&\\
Santander&3409&3409&55823&15.87&\\
Cauca&1826&1819&43940&12.21&\\
Norte de Santander&1650&1545&34707&20.53&\\
Tolima&1235&1158&18830&18.09&\\
Caldas&487&487&6892&16.46&\\
Valle del Cauca&247&228&3249&14.3&\\
Putumayo&29&29&287&10.07&\\
Huila&20&20&106&5.23&\\
Meta&1&1&17&17.6&\\
Quindío&1&1&7&6.5&\\
		\end{tabular}
		
\end{center}

\end{table}

\begin{figure}[H]
	\begin{center}
		\begin{subfigure}[normla]{0.65\textwidth}
		\includegraphics[draft=false, scale=0.55]{papa_cund_png_1.png}	
% * <etorresm@unal.edu.co> 2018-08-30T20:47:37.838Z:
%
% ^.
% * <etorresm@unal.edu.co> 2018-08-30T20:47:36.502Z:
%
% ^.
% * <etorresm@unal.edu.co> 2018-08-30T20:47:35.509Z:
%
% > draft=false
%
% ^.
		\caption{Área sembrada en Cundinamarca.}
		\label{gra:papa_cund_1}	
		\end{subfigure}
		~
		\begin{subfigure}[normla]{0.65\textwidth}
		\includegraphics[draft=false, scale=0.55]{papa_cund_png_2.png}	
		\caption{Producción de papa en Cundinamarca.}
		\label{gra:papa_cund_2}	
		\end{subfigure}
		~
		\begin{subfigure}[normla]{0.65\textwidth}
		\includegraphics[draft=false, scale=0.55]{papa_cund_png_3.png}	
		\caption{Rendimiento en Cundinamarca.}
		\label{gra:papa_cund_3}	
		\end{subfigure}

	\end{center}
	
	\caption{Valores semestrales de área sembrada, área cosechada, producción y promedio del rendimiento del cultivo de papa para Cundinamarca desde el segundo semestre de 2006 al primer semestre de 2017, basado en datos de \citet{madr2017}}.
	 \label{gra:papa_cund}
\end{figure}


\subsection{Área de estudio}

%Se realizó una tabla que resume los rendimientos de cada municipio esta tabla la pueden encontrar en la tabla llamada \texttt{tab\_pivot.csv}.

Gran parte de los municipios de Cundinamarca que son productores de papa hacen parte de la Sabana de Bogotá. La Sabana de Bogotá se ubica en la parte central de la cordillera Oriental de Colombia y se ubica a una altura promedio de 2500 metros sobre el nivel del mar (msnm). Constituye una provincia geomorfológica en la cual se diferencian dos zonas: una zona plana, ubicada hacia la parte central del área, y otra zona de relieve montañoso que alcanza alturas hasta de 3700 msnm \citep{hermelin2007entorno}.\\

En la Figura \ref{gra:areas} se muestra el área de estudio delimitada con la línea azul. Las áreas en color verde y punteadas corresponden a las áreas de páramo definidas por \citet{Cortes-Duque2013}, las áreas azules corresponden a los embalses y los puntos corresponden a las estaciones del Instituto de Hidrología, Meteorología y Estudios Ambientales (IDEAM) las cuales se encuentran el la tabla del Anexo \ref{anexo:estaciones_autom_y_conv_usadas}. El área de estudio cuenta con 162 estaciones meteorológicas convencionales y 31  estaciones meteorológicas automáticas.\\ % La delimitación de los páramos no solo obedece a un criterio de altura sino a un trabajo completo dirigido por el IAvH %\citet{Cortes-Duque2013}


Basado en la información suministrada por el \citet{madr2017} correspondiente al período 2006 a 2017, se espacializaron los valores de áreas promedio de siembra de papa (Figura \ref{subfig:a1}). En la Figura \ref{subfig:a1} podemos observar que el municipio con mayor producción es Tausa, seguido de Villapinzón, Zipaquirá, Sesquilé, Chocontá, Carmen de Carupa, San Cayetano y Une. Para hacer un estimado de la participación del cultivo de papa en cada uno de los municipios se tomó el área promedio de producción de cada municipio, se dividió entre el área total del municipio y se multiplicó por 100 para calcular el porcentaje de área sembrada. En la Figura \ref{subfig:a2}, se ve que Tausa es el municipio con mayor participación del cultivo de papa, seguido por Villapinzón, Chocontá, Sesquilé, Zipaquirá, Cogua, Susa, San Cayetano, Sibaté, Une y Funza. Es importante resaltar que Tausa es el municipio más importante en producción de papa, pero la mayor parte de su terreno se encuentra dentro del Páramo de Guerrero y en otra gran área está el Embalse del Neusa, como se puede observar en la Figura \ref{subfig:a1}, por esta razón el área de estudio no cubre este municipio (ver Figura \ref{gra:areas}).


\begin{figure}[H]
	\begin{center}
		\includegraphics[draft=false, scale=0.2]{municipios_paramos.png}
		\label{gra:areas_paramo}
	\end{center}
    	\caption{Área de estudio}
	\label{gra:areas}
\end{figure}
	
%%%%%%%%%%%% 

\begin{figure}[H]
	\begin{center}
		\begin{subfigure}[normla]{0.4\textwidth}
	\includegraphics[draft=false, scale=0.3]{promedio.png}
		\caption{Áreas promedio de siembra en cada municipio de Cundinamarca en hectáreas.}
		\label{subfig:a1}
		\end{subfigure}
		~
				\begin{subfigure}[normla]{0.4\textwidth}
	\includegraphics[draft=false, scale=0.3]{papa-area.png}
		\caption{Relación entre área con cultivo de papa y área total del municipio en porcentaje (\%)}
		\label{subfig:a2}
		\end{subfigure}

	\end{center}
	\caption{Mapas de las áreas de producción de papa, hecho a partir de la información del \citet{madr2017}}
	\label{gra:areas_promedio}	
\end{figure}


\subsection{Comportamiento de los precios}

La oferta de la papa está condicionada por diferentes factores tales como las condiciones climáticas de la zona, los precios del período anterior, tecnología de producción, costos de producción y el ataque de plagas y enfermedades \citep{Barrientos2014}. La papa se produce principalmente en dos períodos del año en el centro del país y tiene las características de una producción discontinua, ya que presentan un comportamiento de bajadas y subidas atribuido al teorema de la telaraña \citep{BarrientosF.2011, Ezekiel1938}:\\

%\citep{Cortes-Duque2013}

``El precio elevado de un producto en el año `$n$' estimula a los agricultores a que aumenten masivamente la superficie cultivada en el año `$n+1$'. Esto produce una caída de los precios en el año `$n+2$' y, posteriormente, una reducción de la superficie cultivada en el año `$n+3$'" tomado de \citet{Cartay1999}.\\

%En la Figura \ref{gra:papa_precio} podemos observar la alta variación de los precios a través del año y una variación en las áreas sembradas.\\ %% http://lema.rae.es/dpd/srv/search?key=periodo


%Las heladas provocan una disminución en la producción de papa, pero la demanda continúa igual, por esta razón los precios aumentan. Por esta razón se tomaron los datos históricos de las áreas sembradas y los precios y se ubicaron en un solo gráfico, ver gráfico \ref{gra:papa_precio}. Adicionabrotaciónlmente se realizaron estas Figuras para poder observar si existía alguna periodicidad de las áreas o los precios.\\


%\begin{figure}[H]
%	\begin{center}
%	\includegraphics[draft=false, scale=.5]{papa_cund_2.png}
%	\caption{Figura del área sembrada en Cundinamarca y el precio por kilogramo de papa en Bogotá, fuente \citep{madr2017, Agronet2018},}
%	 \label{gra:papa_precio}
%\end{center}
%end{figure}


\section{Temperatura del aire en la Sabana de Bogotá} %Temperaturas extremas y su relación con la papa}
%
\citet{Ruiz2012} afirman que en los últimos 40 años, la temperatura del aire en Bogotá ha aumentado entre 1 y 2\celc \ . Se esperaría que este fenómeno no esté localizado solamente en Bogotá sino en toda la Sabana de Bogotá. Por esta razón, se tomaron todos los valores de temperatura media diarios reportados por las estaciones convencionales del IDEAM de la zona de estudio. Con estos valores se realizó un histograma para cada año y adicionalmente se representó el promedio de ese año con una línea, ver Figura \ref{graph:evol_tmp_sabana}.\\

En el año 1971 la temperatura del aire promedio diaria fue de 12.54\celc \  \ como lo indica la Figura \ref{graph:histo_tiempo_1971}. En el año 2016 se obtuvo una temperatura promedio para la Sabana de Bogotá de 13.88\celc \ , ver Figura \ref{graph:histo_tiempo_2016}. A través de los años se ve un aumento en la temperatura del aire diaria promedio, ver Figura \ref{graph:histo_tiempo}. En promedio, desde el año 1971 hasta el 2016 se ha presentado un aumento de la temperatura del aire medio diario de 1.34\celc \  (Figura \ref{graph:histo_tiempo}).\\


% La temperatura promedio para la Sabana de Bogotá en el año de 1971 era de 12.54\celc \ . En la Figura \ref{graph:evol_tmp_sabana} podemos observar el cambio en la temperatura promedio de la temperatura del aire, la Figura comienza desde el año 1971 con un color azul aguamarina y finaliza en el 2016 con un color naranja, los colores van cambiando a través de los años de fríos a cálidos. En la Figura podemos observar cómo las líneas azules se encuentran a la izquierda, ya que el valor promedio de temperatura para el año 1971 fue de 12.54\celc \  y para el año 2016 el color corresponde a las líneas de colores naranja y rojo y la temperatura promedio de la zona fue de 13.74\celc \ . La Figura \ref{graph:histo_tiempo_2016} es el histograma correspondiente al año 2016 y sirve de referencia para poder observar en comparación con la Figura \ref{graph:histo_tiempo} y ver el aumento de la temperatura promedio de la Sabana de Bogotá.

%%%Voy acá en la revisión, tengo que cambiar la posición de los labels

\begin{figure}[H]
	
	\begin{subfigure}[b]{\linewidth}
	\begin{center}
	\includegraphics[draft=false, scale=0.6]{grafica_tiempo_1971.png}
		
		\caption{Histograma y promedio de la temperatura del aire en el año 1971}		
		\label{graph:histo_tiempo_1971}
		
	\end{center}
	\end{subfigure}

		
	\begin{subfigure}[b]{\linewidth}
	\begin{center}
	\includegraphics[draft=false, scale=0.6]{grafica_tiempo_2016.png}
		
		\caption{Histograma y promedio de la temperatura del aire en el año 2016}		
		\label{graph:histo_tiempo_2016}
		
	\end{center}
	\end{subfigure}
    
    \begin{subfigure}[b]{\linewidth}
	\begin{center}
	\includegraphics[draft=false, scale=0.6]{grafica_tiempo.png}
		
		\caption{Histogramas de la temperatura del aire a través del tiempo y promedios para cada año en el período 1971 a 2016.}		
		\label{graph:histo_tiempo}
		
	\end{center}
	\end{subfigure}
	
	
	\caption{Histogramas de la temperatura promedio del aire en la Sabana de Bogotá y promedios, basado en datos de las estaciones convencionales del IDEAM, pertenecientes al área de estudio. Creada a partir de datos desde el año 1971, iniciando con un colores azul aguamarina y finaliza en el 2016 con un color naranja donde se puede observar el aumento de los valores de temperatura.}		
	\label{graph:evol_tmp_sabana} 
		
\end{figure}

\section{Temperatura y su relación con el cultivo de papa}

\subsection{Altas temperaturas y su efecto fisiológico en plantas}

Las altas temperaturas producen estrés en las plantas, esto se produce cuando la temperatura es lo suficientemente alta como para causar un daño en los tejidos de las plantas e influenciar substancialmente en el crecimiento y en el metabolismo de las plantas. Cada especie posee un rango de termo-sensibilidad en el cual se puede desarrollar de una forma óptima. Sin embargo, un aumento en la temperatura fuera de este rango afectará las plantas en su morfología, fisiología, bioquímica y aún a nivel molecular. Un evento de estrés por calor en las plantas puede producir quemazón en las hojas, ramas y tallos, senescencia y abscisión, inhibición del crecimiento de tallos y raíces, aborto de yemas, reducción del rendimiento, entre otras \citep{Pareek1997}. Como consecuencia, del efecto de la alta temperatura sobre las plantas se producirán especies reactivas de oxígeno, las cuales inducirán un estrés oxidativo en las membranas lipídicas, desnaturalización de las proteínas, produciendo daños en los ácidos nucleicos y la clorofila, fuga de electrolitos y por último muerte celular \citep{Khan2015}.

\subsection{Bajas temperaturas y su efecto fisiológico en plantas}

Las temperaturas más bajas en la Sabana de Bogotá se presentan cuando hay una helada. Las heladas en el mundo son principalmente de dos tipos: la helada advectiva y la helada radiativa.\\

Una helada advectiva es producida por las masas de aire frío que se depositan en un área, remplazando el aire que estaba a una temperatura mayor, generalmente se presentan en latitudes medias y latitudes altas. Este tipo de  heladas están asociadas a condiciones nubladas, vientos moderados a fuertes y no se evidencia una inversión en la tropósfera. A menudo las temperaturas caen bajo cero y se mantienen bajas todo el día \citep{Prabha2008a, IDEAM2012a}. Por ejemplo, \citet{Prabha2008a} define una helada advectiva como noches en la que la velocidad del viento excede los 2 $m\ s^{-1}$ y existen temperaturas bajo 0\celc \.\\

Una helada radiativa es producida bajo condiciones de cielo despejado donde la pérdida de calor de las plantas y del suelo es mayor que el calor recibido durante el día, por esta razón la temperatura desciende y cae por debajo de 0\celc. Está caracterizada por cielos despejados, vientos con baja velocidad, inversiones de temperatura, bajas temperaturas de punto de rocío y temperaturas menores a 0\celc \   \citep{Halley2003, IDEAM2012a}. Existen dos subcategorías de heladas radiatívas la helada blanca y la helada negra. La helada blanca sucede cuando hay temperaturas por debajo de 0\celc \  y el vapor de agua se condensa sobre la superficie de las hojas, entonces se forma una película de hielo sobre las hojas que se llama comúnmente escarcha. Cuando la humedad es muy alta es más probable que se forme una superficie de hielo sobre las hojas y por tanto se forme una helada blanca. Una helada negra sucede cuando la temperatura cae bajo 0\celc \ y no se forma hielo sobre la superficie. En el momento de la formación de escarcha sobre las hojas libera energía, debido al cambio de estado, por esta razón la helada blanca produce menos daño que la helada negra \citep{Snyder2010}.\\

Cuando las temperaturas son lo suficientemente bajas se puede producir la formación de hielo. La formación de hielo produce daño en las plantas. Hay poca o nula evidencia acerca de los efectos de la duración de heladas sobre las plantas. Esto podría significar que los efectos dañinos para las plantas no dependen de la duración del evento sino de la temperatura que se alcance. Cuando la tasa de enfriamiento de la temperatura del aire es muy rápida, se produce un mayor efecto en las plantas \citep{Snyder2010}. Según \citet{Snyder2005} se pueden presentar dos tipos de congelamiento en la planta:

\begin{enumerate}
\item Formación de los cristales dentro del protoplasma (congelamiento intracelular)
\item Formación de cristales dentro de la planta pero fuera del protoplasma (congelamiento extracelular)
\end{enumerate}

En las plantas, cuando la temperatura es menor a 0\celc \  y existen sustancias que actúen como nucleador heterogéneo, el agua se congelará, esto por lo general sucede en las cavidades extracelulares en las cuales el agua se congelará primero. Caso contrario ocurre dentro de las células ya que su contenido de solutos es alto y esto las protege de la congelación. Si las condiciones de temperatura se mantienen o la temperatura desciende, entonces la presión de vapor del hielo será más baja que la presión de vapor del agua en estado líquido. Como resultado, el agua líquida dentro de la célula pasará a través de la membrana semipermeable y se depositará en los cristales de hielo fuera de la célula. Cuando se presenta un congelamiento en los tejidos de las plantas, las células van muriendo debido a la fuga de agua de las células en dirección a las masas de hielo extraceluar \citep{levitt2012chilling, pearce2001plant}.

%Las moléculas de agua se unen para formar un núcleo estable de hielo (nucleación homogénea), o puede ser catalizado por otra sustancia (nucleación heteogénea). Una nucleación homogénea es muy poco probable a temperaturas sobre 0\celc \ , en contraste en un clima húmedo una nucleación heterogénea es difícil de evitar \citep{pearce2001plant}. 

Si una célula de una planta no tiene un nucleador hetereogéneo, la temperatura a la cual se congelaría es de -40\celc \ . En general, el hielo es usualmente extracelular en las hojas congeladas \citep{pearce2001plant}. Algunas de las sustancias que pueden actuar como un nucleador heterogéneo según \citet{pearce2001plant} son:
\begin{enumerate}
\item Bacterias nucleadoras del hielo \textit{Ice nucleation-active} (INA). \textit{Pseudomonas syringae y Erwinia herbicola}.
\item Otras moléculas y estructuras biológicas.
\item Residuos orgánicos o inorgánicos.
\end{enumerate}

%Uno de los daños más frecuentes en las plantas es causado por el congelamiento de las plantas. 

\subsection{Las temperaturas extremas y el cultivo de papa}

Las respuestas de las plantas de papa a temperaturas extremas dependen del estado fenológico en el cual se encuentren. Según \citet{Hatfield2008} en el cultivo de papa en las fases de brotación y emergencia la temperatura del aire debe estar en el rango 4-16\celc\ y la temperatura óptima para el desarrollo después de la emergencia es de 16-20\celsius. Es en la etapa de emergencia cuando las plantas son más susceptibles a daños por temperaturas extremas.\\

La incidencia de las temperaturas extremas sobre los cultivos de papa tiene impacto en la producción. Por ejemplo, valores de temperatura en el suelo sobre 25\celc \  reducen la emergencia de las plantas, el número de plantas que nacen y el número de tallos por planta \citep{birch2012crops, sale1979growth}. Cuando la temperatura del aire es superior a 27\celc \  en los primeros estados de desarrollo, se produce una disminución de la cantidad de hojas en las plantas de papa \citep{birch2012crops, sale1979growth}. Cuando las temperaturas exceden los 29\celc \  se reduce el área foliar y el peso húmedo, esto detiene la producción de los tubérculos \citep{prange1990reduction}. Estos casos de altas temperaturas en la Sabana de Bogotá están relacionados con temporadas secas, cielos despejados y fuertes vientos \citep{IDEAM2017c}.\\

Como ejemplo tenemos la temporada de heladas que se presentaron en 1995 en las planicies del Altiplano Cundiboyacense,  Nariño y Antioquia; causaron la destrucción de cerca del 70\% de la cosecha de papa. Esto representó una pérdida aproximada de 56000 millones de pesos \citep{Romero1996}. Por otra parte, el primero y segundo día de febrero del 2007 se presentaron temperaturas de -4.7\celc \  en los municipios de Mosquera, Sopó, Tenjo, Subachoque, Funza, Sesquilé y Nemocón; este fenómeno afectó cerca de 7500 hectáreas de papa \citep{ElTiempo2007}.\\

%Una de las adaptaciones de las plantas durante períodos fríos es el endurecimiento. El endurecimiento está más relacionado con el aumento de contenido de solutos en los tejidos de las plantas que impiden el congelamiento de los tejidos disminuyendo la temperatura de congelamiento \citep{Snyder2010}.\\


%Por otro lado las bajas temperaturas producen estrés en las plantas, cuando el agua ya sea en la superficie de las hojas o dentro de las plantas se congela a este fenómeno lo llamamos helada \citep{Snyder2010}.\\

%Por esta razón \citet{Hijmans2003} propone que el cultivo de papa no puede estar por fuera de el rango entre 0\celc \  y 20\celc \  porque presentaría daños en la planta que se verían reflejado en la producción, ya que según  

%Para el cultivo de papa el rango óptimo de temperatura del aire está entre 20\celc \  y 0\celc \  \citep{Hijmans2003}, 

%Es importante notar que estas zonas siempre han sido de vocación agropecuaria algunos de estos tienen un mayor costo que otros. Y según \citep{Castro-Romero2014}, esto depende de la calidad de servicios ecosistémicos de las fincas, pero estos servicios ecosistémicos no son infinitos, es más rápidamente tienen a deteriorarse, como es el caso de la materia orgánica. %% Ojo usar esto.

\section{Comportamiento de las temperaturas extremas en la Sabana de Bogotá}

Para realizar este análisis se usaron los datos de las estaciones convencionales y automáticas. Basado en la información obtenida de la red HYDRAS para las estaciones automáticas se realizó la Tabla \ref{tab:hydras_inicio_fin} la cuál contiene las fechas de inicio y finalización de los datos válidos.\\

\begin{table}[h]
\centering
\caption{Fecha de inicio y finalización de la variable temperatura del aire de las estaciones automáticas de la red HYDRAS. Algunas estaciones no poseen valores válidos de temperatura estos son representados con un NaT.}
\label{tab:hydras_inicio_fin}
\begin{tabular}{lrlll}

\toprule
{} &       Código &                   Nombre &              Inicio &                 Fin \\
\midrule
1  &  35025080 &       PNN CHINGAZA AUTOM & 1996-01-24 08:00:00 & 2018-02-22 00:00:00 \\
2  &  35075080 &   PMO RABANAL AUTOMATICA & 1998-06-02 03:32:00 & 2018-02-22 00:00:00 \\
3  &  35085080 &         LA CAPILLA AUTOM & 2000-01-01 00:00:16 & 2018-02-21 23:08:00 \\
4  &  21206990 &     TIBAITATA AUTOMATICA & 2000-01-01 00:00:22 & 2018-02-22 00:00:00 \\
5  &  21206940 &           CIUDAD BOLIVAR & 2000-01-01 17:12:52 & 2014-07-22 03:02:00 \\
6  &  35035130 &             PMO CHINGAZA & 2000-01-01 23:37:48 & 2018-02-22 00:00:00 \\
7  &  21206920 &  VILLA TERESA AUTOMATICA & 2000-01-20 17:22:04 & 2018-02-22 00:00:00 \\
8  &  21205012 &            UNIV NACIONAL & 2003-05-20 17:42:00 & 2018-02-21 23:05:00 \\
9  &  21206930 &             PMO GUERRERO & 2004-04-02 10:44:52 & 2018-02-21 23:09:00 \\
10 &  21206790 &        HDA STA ANA AUTOM & 2005-02-08 02:00:00 & 2018-02-21 23:02:00 \\
11 &  23125170 &       SAN CAYETANO AUTOM & 2005-02-09 01:07:00 & 2018-02-22 00:00:00 \\
12 &  35075070 &     CHINAVITA AUTOMATICA & 2005-02-13 09:00:00 & 2018-02-22 00:00:00 \\
13 &  35027510 &           CALOSTROS BAJO & 2005-02-20 23:28:00 & 2018-02-22 00:00:00 \\
14 &  21206980 &       STA CRUZ DE SIECHA & 2005-04-21 17:59:00 & 2018-02-22 00:00:00 \\
15 &  21206950 &          PMO GUACHENEQUE & 2005-06-21 12:00:00 & 2018-02-22 00:00:00 \\
16 &  24015110 &     LA BOYERA AUTOMATICA & 2005-06-28 12:00:00 & 2018-02-22 00:00:00 \\
17 &  21195160 &         SUBIA AUTOMATICA & 2006-07-30 12:01:00 & 2012-06-12 19:43:00 \\
18 &  21206960 &             IDEAM BOGOTA & 2008-06-15 10:00:00 & 2018-02-21 23:26:00 \\
19 &  35025090 &       BOSQUE INTERVENIDO & 2009-04-28 03:00:00 & 2018-02-22 00:00:00 \\
20 &  21206600 &         NUEVA GENERACION & 2010-01-06 00:30:00 & 2018-02-21 23:01:00 \\
21 &  21201200 &  ESC LA UNION AUTOMATICA & 2010-06-11 15:30:00 & 2016-08-01 13:20:00 \\
22 &  21205791 &           APTO EL DORADO & 2014-08-29 10:03:00 & 2018-02-21 23:06:00 \\
23 &  21201580 &     PASQUILLA AUTOMATICA &                 NaT &                 NaT \\
24 &  21202270 &   PLUVIOMETRO AUTOMATICO &                 NaT &                 NaT \\
25 &  21202271 &       PLUVIOMETRO AUTOMA &                 NaT &                 NaT \\
26 &  21206710 &   SAN JOAQUIN AUTOMATICA &                 NaT &                 NaT \\
27 &  21209920 &         STA ROSITA AUTOM &                 NaT &                 NaT \\
28 &  26127010 &             EL ALAMBRADO &                 NaT &                 NaT \\
29 &  35025100 &           CALOSTROS BAJO &                 NaT &                 NaT \\
30 &  35027001 &          PLAZA DE FERIAS &                 NaT &                 NaT \\
31 &  35027002 &      PARQUE RAFAEL NUÑEZ &                 NaT &                 NaT \\
\bottomrule
\end{tabular}
\end{table}

Es importante notar que no todas las estaciones automáticas cuentan con datos de temperatura y los registros de las estaciones no comienzan en los mismos períodos de tiempo, ver Tabla \ref{tab:hydras_inicio_fin}. Además no todas las estaciones automáticas poseen datos de temperatura del aire, las estaciones sin datos de temperatura fueron marcadas con NaT. Se evidencia que nueve estaciones no poseen datos de temperatura del aire. Dos de las 22 estaciones reportan datos antes del año 2000. Tres de las 22 estaciones con datos no presentan datos para el año 2018.\\

Las temperaturas extremas del aire son aquellos valores que se encuentran fuera del rango entre 0\celc \  y 20\celc \ , ya que según \citet{Hijmans2003} las plantas de papa fuera de este rango presentan afectaciones.\\

Las Figuras \ref{graph:tmp_bajo0} y \ref{graph:tmp_sobre20} muestran todos los valores diarios máximos y mínimos de la temperatura del aire de las estaciones convencionales que están por fuera del rango entre 0 y 20\celc \  y los años en los cuales se presentó cada valor. En la Figura \ref{graph:tmp_bajo0} se puede observar que aproximadamente desde el 2009 ha ido disminuyendo el número de casos de bajas temperaturas y los valores de la temperatura han sido cada vez menos negativos. Para el caso de las temperaturas sobre 20\celc \ Figura\ref{graph:tmp_sobre20} se observa un aumento continuo de los eventos y en el 2003 se observa un incremento en los valores de temperatura.


\begin{figure}[H]
	\begin{subfigure}[b]{\linewidth}
	\begin{center}
	\includegraphics[draft=false, scale=0.7]{tmp_debajo0.png}
		
		\caption{Valores mínimos diarios de temperatura bajo 0\celc \  y año del evento. Los valores fueron registrados por las estaciones convencionales del IDEAM.}		
		\label{graph:tmp_bajo0}
		
	\end{center}
	\end{subfigure}
	
	\begin{subfigure}[b]{\linewidth}
	\begin{center}
	\includegraphics[draft=false, scale=0.7]{tmp_sobre20.png}
		
		\caption{Valores máximos diarios de temperatura sobre 20\celc \  y año del evento. Los valores fueron registrados por las estaciones convencionales del IDEAM.}		
		\label{graph:tmp_sobre20}
		
	\end{center}
	\end{subfigure}
	
	
	\caption{Valores diarios de temperatura del aire para las estaciones convencionales de la Sabana de Bogotá y que están fuera del rango comprendido entre 0 y 20\celc. }		
	\label{graph:limites}

\end{figure}

La figura \ref{graph:tmp_bajo_meses} muestra la distribución de las temperaturas mínimas bajo cero en los diferentes meses del año usando la información de las estaciones convencionales. Se han registrado temperaturas bajo cero en todos los meses del año, pero con mayor frecuencia en los meses de diciembre, enero y febrero. Por otra parte, las temperaturas sobre 20\celc \  se presentan durante todos los meses del año pero con mayor frecuencia entre diciembre y abril (Figura \ref{graph:tmp_sobre_meses}).


\begin{figure}[H]
	\begin{subfigure}[b]{\linewidth}
	\begin{center}
    		\caption{Frecuencia mensual de la temperatura mínima del aire bajo 0\celc \ }
	\includegraphics[draft=false, scale=0.7]{frec_bajas_tmp.png}
		

		\label{graph:tmp_bajo_meses}
		
	\end{center}
	\end{subfigure}
	
	\begin{subfigure}[b]{\linewidth}
	\begin{center}
    \caption{Frecuencia mensual de la temperatura máxima del aire sobre 20\celc \ }
	\includegraphics[draft=false, scale=0.7]{frec_altas_tmp.png}
		
		
		\label{graph:tmp_sobre_meses}
		
	\end{center}
	\end{subfigure}
	
	
	\caption{Frecuencia de ocurrencia de las temperaturas (a) mínimas diarias bajo 0\celc \  y (b) de las temperaturas sobre 20\celc \ de cada una de las 162 estaciones convencionales ubicadas en la Sabana de Bogotá en función de los meses del año.}
	\label{graph:tmp_meses}

\end{figure}
%Frecuencia de ocurrencia de valores de temperatura de cada una de las estaciones ubicadas en la Sabana de Bogotá bajo 0\celc \  en función de los meses del año



La Figura \ref{graph:tmp_hora} muestra la distribución horaria de la temperatura que está por fuera del rango 0 a 20\celc \ . Para este análisis se usó información de la estación automática Tibaitatá perteneciente a la red HYDRAS del IDEAM. Ya que fue necesario realizar una comparación entre las estaciones convencionales y las estaciones automáticas y esta estación posee ambos tipos de estación, además la estación convencional posee datos diurnos horarios que fueron facilitados por Agrosavia. Como se puede observar en la Figura \ref{graph:tmp_bajo_hora} la mayoría de eventos con temperatura bajo 0\celc \  se presentan entre las 0 horas y las 7 horas, presentando la mayor frecuencia a las 5 de la mañana. Para las horas en las que se presenta la mayor cantidad de eventos con temperaturas sobre 20\celc \ , podemos decir que se producen entre las 8 y las 17 horas teniendo la mayor frecuencia a las 12 horas (ver Figura \ref{graph:tmp_sobre_hora}).



\begin{figure}[H]
	\begin{subfigure}[b]{\linewidth}
	\begin{center}
	\includegraphics[draft=false, scale=0.7]{hora_heladas.png}
		
		\caption{Ocurrencia de temperaturas bajo 0\celc \  en las diferentes horas del día. Hora local.}		
		\label{graph:tmp_bajo_hora}
		
	\end{center}
	\end{subfigure}
	
	\begin{subfigure}[b]{\linewidth}
	\begin{center}
	\includegraphics[draft=false, scale=0.7]{hora_tmp_altas.png}
		
		\caption{Frecuencia de ocurrencia de temperaturas sobre 20\celc \  a en las diferentes horas del día. Hora local.}		
		\label{graph:tmp_sobre_hora}
		
	\end{center}
	\end{subfigure}
	
	
	\caption{Frecuencia de ocurrencia de las temperaturas horarias bajo 0\celc \  y sobre 20\celc.}		
	\label{graph:tmp_hora}

\end{figure}


\section{Comparación entre datos de una estación automática y una convencional en el evento de helada presentado en Febrero del 2007}

%%%acá voy

Las estaciones meteorológicas automáticas están tomando el lugar de las observaciones con las estaciones convencionales. Pero es claro que las estaciones meteorológicas automáticas necesitan de un periodo de observaciones en paralelo y todos los datos debe recibir un control de calidad para asegurar una homogeneidad en los datos \citep{Ying2004}, ya que en los análisis futuros la información provendrá solo de las estaciones automáticas.\\

Las estaciones convencionales poseen varios problemas uno de ellos es el alto costo que genera tener un operario registrando los datos, las estaciones convencionales presentan errores de transposición de los números (ejemplo: 12 es registrado por 21) o la transposición de máximos y mínimos.\\

Las estaciones automáticas también poseen problemas tales como el no registro de los datos, saltos en las series, sensores que presentan los mismos registros, valores atípicos de la variable, entre otros.  Los sensores de las estaciones automáticas permanecen en el campo hasta que se reportan problemas o el sensor cumple su tiempo útil. El tiempo de vida útil se establece en base de la experiencia de la red \citep{Shafer2000}.

Los errores pueden ser producidos por un ruido electrónico en las medidas, comunicaciones defectuosas, degradación del sensor, golpe de un rayo o cambios de los registros en los días \citep{Menne2001}.

La estación Tibaitatá genera datos de temperatura a dos metros. Para cada una de estas variable existen valores máximos, mínimos y promedios ver en la Figura \ref{subfig:b9} en esta figura se gráfico primero los valores promedio de temperatura, luego los valores de temperaturas máximas y para finalizar las temperaturas mínimas. Adicionalmente se realizó una figura donde se ubicó primero la temperatura mínima, seguido de la temperatura máxima y para finalizar la temperatura promedio, ver Figura \ref{subfig:b10}.


\begin{figure}[H]
	\centering
			\begin{subfigure}[b]{0.4\textwidth}

			\includegraphics[draft=false, scale=0.5]{grafica_info_4/prom_fondo.pdf}
			\caption{Temperatura del aire en el siguiente orden promedio, máxima y mínima.}
			\label{subfig:b9}

			\end{subfigure}
			%% % Simbolo usado para poner las Figuras una frente a la ootra
			\begin{subfigure}[b]{0.4\textwidth}

			\includegraphics[draft=false, scale=0.5]{grafica_info_4/prom_frente.pdf}
		\caption{Figura de la temperatura del aire en el siguiente orden mínima, máxima y promedio.}
			\label{subfig:b10}

			\end{subfigure}			

		
		\caption{Gráfico de las temperaturas máximas, mínimas y promedio para la estación Tibaitatá, desde el cinco de febrero hasta el siete de febrero del 2007.}
		\label{gra:conv_vali}
	
\end{figure}

En la Figura \ref{subfig:b9} se puede observar que las temperaturas máximas y mínimas no coinciden con los valores máximos y mínimos de la variable temperatura del aire, este patrón se repite en las diferentes estaciones analizadas. Existen lapsos de tiempo en los cuáles no se registran valores de temperatura. Según la Figura \ref{subfig:b10} la variable que posee la mayor cantidad de datos es la de temperatura promedio, sin embargo existen ciertos valores que son reportados por la temperatura máxima y mínima los cuales no son registrados en la temperatura promedio. Por esta razón si se quiere obtener la mejor serie de tiempo de temperatura se deben unir las temperatura máxima, mínima y promedio. Es importante notar que en ciertos momentos se puede obtener valores de temperatura promedio y valores máximos o mínimos al mismo tiempo. Basado en la experiencia de este trabajo cuando se presenten dos valores y uno de ellos es la temperatura promedio es preferible tomar la temperatura promedio, ya que las temperatura máxima y mínima suelen tener más errores que la promedio.\\


A partir de los datos de la estación convencional se quiso determinar cuánto es la diferencia entre la temperatura a dos metros y la temperatura sobre el suelo, ya que la altura de las plantas oscila entre cero y 80 centímetros aproximadamente, dependiendo del estado fenológico en el que se encuentre. Por esta razón se usaron los datos de la estación convencional Tibaitatá, facilitados por Agrosavia esta estación posee termómetros a diferentes alturas 5, 10, 50, 100 y 200 centímetros. Estos termómetros nos permiten hacer un análisis con respecto al perfil de temperatura, ya que los valores de temperatura que normalmente se usan en las estaciones se encuentra reportada a 2 metros, obedeciendo la normatividad de la \citet{WMO2010}. Por esta razón se tomaron los valores mínimos de estos termómetros y se promediaron, se obtuvo la Tabla \ref{tabla_minimas_convencional}. En la Tabla \ref{tabla_minimas_convencional} se puede observar que las temperaturas más bajas se encuentran a menor altura y podemos concluir que hay una diferencia de 2.8 \celsius entre el termómetro a 5 y 200 cm. Si en el sensor de temperatura a dos metros de altura presenta una temperatura de 0\celsius entonces, podemos pensar que la temperatura a 5 cm puede llegar a ser de -2.8\celsius.\\

\begin{table}[H]
\centering

\begin{tabular}{l|l}
\textbf{Promedio de la variable} & \textbf{\celc} \\ \hline
Temperatura a 5 cm               & 4.4         \\
Temperatura a 10 cm              & 5.1         \\
Temperatura a 50 cm              & 5.9         \\
Temperatura a 100 cm             & 6.7         \\
Temperatura a 200 cm             & 7.2        
\end{tabular}
\caption{Tabla resumen de las temperaturas mínimas reportados por la estación convencional}
\label{tabla_minimas_convencional}
\end{table}

%Los datos de la Tabla \ref{tabla_minimas_convencional} es una herramienta útil para poder determinar cuál es la diferencia entre las temperaturas a dos metros y a una altura menor. Todas las estaciones automáticas y convencionales generan resultados de la temperatura a dos metros, con los valores de la Tabla \ref{tabla_minimas_convencional} se puede hacer una aproximación de la diferencia de temperatura que se puede presentar a otras alturas. Por ejemplo en el caso que se presenten temperaturas de 0\celc en un termómetro a dos metros y la papa está en un estado fenológico temprano y tiene una altura de 10 cm, esperaríamos que la temperatura a la altura de la planta sea aún menor, cercano a -2.1\celc.


%%%%%%%%%%%%%%%%%%%%%%%%%%%%%%%%%%%%%%%%%%%%%%%%%%%%%%%%%%%%%%%%%%%%%%%%%%%%%%%%%
%%%%%%%%%%%%%%%%%%%%%%%%%%%%%%%%%%%%%%%%%%%%%%%%%%%%%%%%%%%%%%%%%%%%%%%%%%%%%%%%%


\subsection{Comparación entre una estación convencional y una automática}
\label{convencional_vs_automática}

Se realizó una comparación entre los valores de las estaciones convencionales y la estación automática de Tibaitatá, ver Figura \ref{subfig:b1}. La estación convencional y la estación automática se encuentran a cinco metros de distancia en el Centro de Investigación Tibaitatá de Agrosavia, coordenadas (4.688693, -74.205595). Se realizó una Figura de los datos de temperatura a dos metros mínimos y máximos que reporta la estación convencional y los datos de temperatura a dos metros que reporta la estación automática  entre las fechas 29 de enero de 2007 al 7 de febrero de 2007. Esta Figura se realizó para un período en el cuál se registraron altas y bajas temperaturas en la zona de estudio. Es importante resaltar que los valores de las estaciones convencionales son valores diarios, razón por la cuál los valores de máximas y mínima temperatura son ubicadas a las cero horas del día de estudio.

 \begin{figure}[H]
	\begin{center}
	\includegraphics[draft=false, scale=0.3]{grafica_info_4/conv_vs_auto_2007.pdf}
		\caption{Comparación de las temperaturas del aire a dos metros registrado por la Estación Convencional Tibatitatá (tmp\_2m) y las temperaturas del aire máxima (Máxima-conv) y temperatura del aire mínima (Mínima-conv) a dos metros reportada por la estación convencional Tibaitatá, para las fechas entre 29 de enero de 2007 al 7 de febrero de 2007.}
		\label{subfig:b1}
	\end{center}
\end{figure}

 En la Figura \ref{subfig:b1}, se puede observar que los valores de temperaturas mínimas de la estación convencional coinciden con los puntos más bajos de la estación automática en la mayoría de los casos y nunca el valor de la estación automática es menor que el valor de la estación convencional. Pero para el caso de las temperaturas altas podemos ver que en la mayoría de los casos los valores de la estación automática exceden los valores de la estación convencional. Esto quiere decir que los valores máximos de temperatura de las estaciones convencionales son mayores a los valores registrados por las estaciones convencionales.\\
 
 Es importante resaltar que los valores reportados por las estaciones convencionales son registrados basado en dos tipos de termómetros, para las temperaturas medias se usa un termómetro de mercurio y para las temperaturas mínimas se usa un termómetro de alcohol \citep{augter2013vergleich} (Figura \ref{gra:convencional_2}). Por otro lado en las estaciones automáticas la recolección de los datos se realiza mediante sensores (termocuplas) las cuales permiten un seguimiento más rápido de las condiciones atmosféricas que facilitan una rápida toma de decisiones \citep{Soares2017}.
 
% La temperatura promedio en las estaciones convencionales es calculada con la siguiente fórmula:
%
%\begin{equation}\label{fx:promedio}
%T_{promedio} = \frac{T_{I} + T_{II} +T_{II}}{4}
%\end{equation}

%donde $T_{I}$ es la temperatura observada a las 6:00, $T_{II}$ temperatura observada a las 12:00 y $T_{III}$ temperatura observada a las 18:00. Esta es una adaptación de la fórmula propuesta por \citet{kamtz1831lehrbuch}.\\



El funcionamiento de las estaciones automáticas y convencionales es diferente. Este fenómeno se ha presentado en otras investigaciones, por ejemplo \citet{auchmann2012physics} pantea una investigación para determinar si hay diferencias entre las estaciones automáticas y las estaciones convencionales, donde concluye que la diferencia entre los valores de temperatura radica en los tipos de abrigos meteorológicos usados. Por ejemplo \citet{augter2013vergleich} concluye que el cambio en el tipo de estaciones solo provoca pequeños cambios en la variable presión atmosférica y la temperatura, pero que no presentaron inhomogeneidades, la precipitación es ligeramente diferente, pero la mayor incertidumbre es reflejada en los sensores de lectura manual tales como la medición de humedad y brillo solar \citep{augter2013vergleich}. Pero otros autores como \citet{kamtz1831lehrbuch} han reportado diferencias en los valores de las mediciones, al igual que \citet{lacombe2010results} y \citep{kaspar2016climate} quienes aseveran que en ciertos casos la temperatura registrada por la estación automática registra valores más altos.\\

Los abrigos meteorológicos de las estaciones convencionales son basados en las modificaciones hechas por la Real Sociedad Meteorológica en 1884 (Figura \ref{gra:convencional_1}) y los sensores de las estaciones automáticas son operadas en Abrigos Laminares del tipo "LAM 630" (Figura \ref{gra:autom_1}).

\begin{figure}[H]
	
	\begin{subfigure}[normla]{0.4\textwidth}
	\centering
	\includegraphics[draft=false, scale=0.05]{grafica_info_4/abrigo_convencional.jpg}
		
		\caption{Abrigo meteorológico de una estación convencional}		
		\label{gra:convencional_1}
		
	\end{subfigure}
~
	\begin{subfigure}[normla]{0.4\textwidth}
	\centering
	\includegraphics[draft=false, scale=0.3]{grafica_info_4/abrigo_convencional_2.png}
		
	\caption{Termómetros dentro de un abrigo meteorológico convencional.}		
	\label{gra:convencional_2}
		
	\end{subfigure}
%%%%%%%%%%%%%%%%%%%%%%		
	\begin{subfigure}[normla]{0.4\textwidth}
	\centering
	\includegraphics[draft=false, scale=0.05]{grafica_info_4/abrigo_automatica_1.jpg}
		
		\caption{Estación automática.}		
		\label{gra:autom_1}
		
	\end{subfigure}
~
	\begin{subfigure}[normla]{0.4\textwidth}
	\centering
	\includegraphics[draft=false, scale=0.7]{grafica_info_4/abrigo_automatica_2.png}
		
		\caption{Detalle del abrigo meteorológico de una estación automática.}		
		\label{gra:autom_2}
		
	\end{subfigure}	
	
	
	\caption{Abrigos meteorológicos de una estación convencional y una estación automática ubicadas en el centro de investigación de Agrosavia, Tibaitata en Mosquera.}		
	\label{graph:evol_tmp_sabana} 
		
\end{figure}

El aumento de la temperatura en las estaciones automáticas es generado por el efecto de la radiación y la posición del sensor dentro del abrigo \citep{kaspar2016climate}. En caso de alta radiación y baja velocidad del viento se ha demostrado que el abrigo meteorológico LAM 630 registra valores de temperatura más alto en comparación con la garita de termómetro tipo Stevenson \citep{lacombe2010results, kaspar2016climate}.

%%%%%%%%%%%%%%%%%%%%%%%%%%%%%%%%%%%%%%%%%%%%%%%%%%%%%%%%%%%%%%%%%%%
%%%%%%%%%%%%%%%%%%%%%%%%%%%%%%%%%%%%%%%%%%%%%%%%%%%%%%%%%%%%%%%%%%%
\subsection{Caracterización de las heladas y altas temperaturas}
\label{area_caracterizacion_heladas_extremas}

Para analizar el comportamiento de las heladas y las altas temperaturas se tomaron todos los datos de la Estación Automática Tibaitatá y se graficaron diariamente, con el objetivo de observar el comportamiento diario de la temperatura (Figura \ref{gra:sin_filtro}). Para hacer una caracterización de las heladas se graficaron los días que se habían presentado heladas (Figura \ref{gra:helada}). Adicionalmente los días en los que se presentaron altas temperaturas también fueron graficadas (Figura \ref{gra:altas}).

%En varios textos se dice que la temperatura en los días de heladas cambia muy rápidamente. Para poder visualizar la forma como cambia la temperatura se realizaron las siguientes Figuras, las cuales son la representación de las diferentes temperaturas en el día, después del control de datos.

\begin{figure}[H]
\begin{subfigure}[b]{0.5\textwidth}
\begin{center}
\caption{Comportamiento horario de la temperatura horaria para la estación Tibaitatá.}
\includegraphics[draft=false, scale=0.4]{c_tmp/21206990_1.png}
\label{gra:sin_filtro}
        \end{center}
\end{subfigure}
~
%\begin{subfigure}[b]{0.5\textwidth}
%\begin{center}
%\includegraphics[draft=false, scale=0.4]{c_tmp/21206990_4.png}
%\caption{Antes de la helada}
%\label{gra:antes_helada}
%\end{center}    
%\end{subfigure}
%~
\begin{subfigure}[b]{0.5\textwidth}
\begin{center}
\caption{Comportamiento horario de la temperatura horaria para la estación Tibaitatá en los días de heladas.}
\includegraphics[draft=false, scale=0.4]{c_tmp/21206990_3.png}
\label{gra:helada}
\end{center}    
\end{subfigure}
~
%\begin{subfigure}[b]{0.5\textwidth}
%\begin{center}
%\includegraphics[draft=false, scale=0.4]{c_tmp/21206990_6.png}
%\caption{Antes de las altas temperaturas}
%\label{gra:antes_altas}
%\end{center}    
%\end{subfigure}
%~
\centering
\begin{subfigure}[b]{0.5\textwidth}
\begin{center}
\caption{Comportamiento horario de la temperatura horaria para la estación Tibaitatá en los días de altas temperaturas.}
\includegraphics[draft=false, scale=0.4]{c_tmp/21206990_5.png}
\label{gra:altas}
\end{center}    
\end{subfigure}
~

\caption{Comportamiento horario de la temperatura del aire de estación Tibaitatá código 21205990 con heladas y con altas temperaturas. Se les ajustó una línea de tendencia para poder observar los cambios en los comportamientos.}
\end{figure}  

En la Figura \ref{gra:sin_filtro} podemos observar un aumento de la temperatura a partir de las 6 am Hora Local (HL), se resgistran los valores máximos entre las 12 m. y 1 p.m. HL y luego de esto se observa un descenso de la temperatura. Las horas más frias se registran antes del amanecer, cerca a las 5 a.m. HL.\\

Cuando hay heladas podemos observar que el comportamiento de la temperatura varía (Figura \ref{gra:helada}), ya que se presenta el aumento de la temperatura a la misma hora que en la Figura \ref{gra:sin_filtro}, pero posteriormente la tasa de aumento de la temperatura es mayor. Y las temperaturas que se alcanzan algunas de ellas sobrepasan los 25\celsius. Posterior a este rápido aumento se presenta una rápida disminución de la temperatura. Se puede observar que se presentan registros de temperatura iguales o menores a 0\celsius desde las 9 a.m. HL.\\

Para el caso de las temperaturas altas podemos observar que hay unos valores que se encuentran sobre 30\celsius es muy probable que estos valores sean valores atípicos (Figura \ref{gra:altas}). Se puede observar que el comportamiento es similar a la Figura \ref{gra:helada} porque hay un aumento y un descenso rápido de temperatura. En la Figura \ref{gra:altas} podemos observar que existen registros de altas temperaturas desde las 9 am HL.\\

Estos resultados concuerdan con lo que se nombra en la investigación de \citep{Gomez2014} en el cual el modelo usado tiene en cuenta la rapidez como la temperatura desciende para de esta manera poder realizar la predicción de una helada.

%\end{comment}  

\section{Modelos usados para el pronóstico y simulación de la temperatura del aire}

Como se explicó anteriormente, la temperatura del aire tiene un rol central en el crecimiento y desarrollo de los cultivos, ya que influye sobre las tasas de crecimiento de las plantas, pero los extremos de temperatura del aire pueden generar daños en las plantas \citep{wheeler2000}. Por esta razón es importante realizar un pronóstico de estos eventos extremos, ya que son capaces de generar serios daños a los cultivos.\\

Una de las formas para determinar las temperaturas extremas es la utilización de modelos empíricos. \citet{Gomez2014} realizó un estudio sobre las heladas y su efecto en pasturas en el Valle de Ubaté y Chiquinquirá para el período del 2006 y 2007 para generar un pronóstico de las heladas del siguiente día a partir del día presente y se basó en la fórmula de \citet{Allen1957} y el estudio de \citet{Snyder2010}. Además, caracterizó el comportamiento de las heladas en la Sabana de Bogotá y reportó que en las épocas secas hay mayor probabilidad de heladas. El área de estudio comprendió un total del 216.907 ha, de las cuáles \citet{Gomez2014} encontró que 2.442 (1.13\%) hectáreas tienen una alta exposición frente a las heladas. Otro de los resultados de este estudio fue que las estimaciones realizadas por el modelo no fueron precisas, ya que en 5 de cada 14 oportunidades se subestimó la temperatura y en las demás ocasiones se sobrestimó.\\

El balance de energía del suelo fue usado por \citet{Rosenzweig2014} con el objetivo de establecer el rango de incertidumbres del cambio climático en la producción de alimentos en el mundo. En este estudio fue necesario realizar una estimación de la temperatura, por esta razón se usaron los modelos de agroecoistema, los cuales usan el balance de energía para poder hacer una aproximación de la temperatura del aire. Como resultado se obtuvo que hay una alta incertidumbre en la modelación del dióxido de carbono, nitrógeno y las altas temperaturas y su efecto sobre la producción agrícola.\\

Otro estudio que usa un balance energético fue el de \citet{Rossi2002}, hecho para la región al norte de Italia llamada Emilia Romagna. En esta zona, las heladas causan grandes pérdidas en los cultivos. El tipo de helada que afecta esta zona es principalmente de tipo advectivo, pero también hay presencia de heladas radiativas. Como resultado se encontró que en las heladas radiativas existe una gran estratificación del aire, haciendo más complejo el proceso de parametrización del intercambio de calor en la atmósfera que predice los perfiles de temperatura en la atmósfera. La caracterización de la estabilidad de la atmósfera es un elemento esencial en la predicción de la temperatura interna de los tejidos y brotes localizados en las partes productivas de los cultivos.\\

Los sistemas de información geoFigura (SIG) también han sido usados para la predicción de bajas temperaturas, ya que las mas bajas temperaturas son encontradas en los valles durante un evento de radiación de tipo advectivo \citep{Halley2003, Blennow1998}. \citet{Halley2003} realizó un estudio para el Norte de Tasmania, ya que desde 1992 la incidencia de heladas ha presentado un incremento y se ha visto afectada la producción de vino. Como resultado, se mostró que el modelo implementado logró explicar las heladas en un 61\% de la zona de estudio, pero se resalta que este modelo no cuantificó los flujos de radiación de onda larga, lo cual produce un sesgo.\\

El uso de redes neuronales también ha mostrado buenos resultados en la modelación de las heladas como en el caso de \citet{Smith2007} quien como resultado muestra que las predicciones fueron útiles incluso cuando las condiciones iniciales no son las mejores, pero cuando el horizonte de pronóstico aumenta incrementa los errores del pronóstico. \citet{Abhishek2012} concluyen que el proceso es demandante de recursos computacionales y que mediante este método se puede hacer predicción de otras variables meteorológicas tales como humedad, velocidad del viento, entre otros.\\

Para subsanar algunas de las falencias mencionadas anteriormente, en algunas investigaciones se han usado modelos regionales para el pronóstico de las heladas tal como lo hizo \citet{prabha2008evaluation} quienes usaron el modelo \textit{Weather Research and Forecasting Model} (WRF) para poder determinar  si la información de mesoescala puede ser una guía para la protección de los cultivos y poder generar información que ayude a reducir los efectos de las bajas temperaturas sobre los cultivos de arándano y durazno. Este estudio analizó heladas advectivas y radiativas en la zona del Sur de Georgia, E.E.U.U. en los años 2006 y 2007. En este estudio se tuvieron en cuenta las variables temperatura y velocidad del viento. \citet{prabha2008evaluation} concluyeron que el modelo, después de calibrado, permitió realizar un pronóstico de razonable precisión con respecto a las variaciones de la temperatura. El modelo WRF se usa actualmente en Colombia y es usado actualmente por el IDEAM para el pronóstico del tiempo en Colombia como lo reportan varios autores \citep{Arango2011, Mejia2012, Ruiz2014}.\\


%Para el pronóstico de las variaciones diarias de temperatura se han usado diferentes métodos por ejemplo, métodos empíricos \citep{Allen1957a, Kangieser1959}, métodos de balance de energía del suelo \citep{Rosenzweig2014a, Rossi2002} métodos basados en los sistemas de información geoFigura \citep{Halley2003} y métodos basados en redes neuronales \citep{Smith2007}. Una de las problemáticas de estos métodos es que no se tienen en cuenta los aspectos físicos y dinámicos de la atmósfera y su evolución temporal \citep{Prabha2008a}.\\

Una de las problemáticas de algunos modelos nombrados anteriormente (modelos empíricos, modelos de balance d energía, SIG  y redes neuronales) no tienen en cuenta aspectos dinámicos y físicos de la atmósfera y su evolución \citep{Prabha2008a}. Por esta razón y por el uso que se le ha dado en Colombia el modelo WRF se convierte en el modelo más interesante para la evaluación de las heladas. Basado en toda la información recopilada se creo una tabla para resumir las fortalezas y las limitaciones de cada uno de los modelos, ver Tabla \ref{tab:fort_deb_mod}.



\begin{table}[H]
\caption{Resumen de las ventajas y desventajas de los modelos usados para el pronóstico de temperaturas del aire}
\resizebox{\textwidth}{!}{\begin{tabular}{p{5cm}| p{5cm} p{5cm} p{5cm}}
Modelo                                        & Fortaleza                                                  & Limitación                            &Fuente                                                   \\ \hline
\multirow{2}{*}{Modelos empíricos}            & Fácil aplicación                                                               & Baja precisión en la predicción de la temperatura del aire       & \citep{Gomez2014, Allen1957, Snyder2010}                   \\
                                              & Se pueden hacer modificaciones al modelo de manera sencilla                     & Los modelos son creados para determinadas condiciones     &                                                                              \\ \hline
\multirow{2}{*}{Balance de energía del suelo} & Fácil aplicación siempre y cuando se tengan los valores de todas las variables                & Alta incertidumbre                                        & \citep{evett2011water,Rosenzweig2014, Rossi2002}          \\
                                              &                                                                                & Las variables no son fáciles de calcular                  &                                                                              \\ \hline
Sistemas de información geoFigura            & Predice muy bien las bajas temperaturas en los valles         & No  tiene en cuenta los flujos de radiación de onda larga & \citep{evett2011water, Halley2003, Blennow1998}              \\ \hline
Redes neuronales                              & Buenas predicciones en un corto horizonte de pronóstico (6 horas) & Pronóstico a un horizonte muy corto                       & \citep{Smith2007}                                           \\ \hline
Modelo \textit{Weather Research and Forecasting Model} (WRF)                   & Buenas predicciones en un horizonte de 2 días                                      & Alto gasto computacional                                  & \citep{prabha2008evaluation, Arango2011, Mejia2012, Ruiz2014} \\
                                              &                                                                                & Es necesario tener una alta capacidad de almacenamiento  &                                                                              \\ \hline


\end{tabular}}

\label{tab:fort_deb_mod}
\end{table}

En la Tabla \ref{tab:fort_deb_mod} podemos observar que algunos modelos presentan ciertas fortalezas y debilidades, pero dependiendo de las necesidades del usuario se debe realizar una valoración de cada uno de ellos para tomar las mejores decisiones.


\section{Conclusiones}

%El cultivo de papa es un cultivo de gran importancia en el país. El consumo interno del país es abastecido casi en su mayoría por la producción interna. Este cultivo es seriamente afectado por muchos factores agroclimáticos y dos de los más importantes son las temperatura extremas. 
Cundinamarca es el departamento con mayor producción en el país y la Sabana de Bogotá es la zona de Cundinamarca que más influencia tiene en la producción de papa y por sus condiciones topoFiguras es altamente susceptible a heladas.\\

Los meses con mayor probabilidad de heladas son diciembre, enero y febrero; y la hora en la que más se presentan bajas temperaturas es a las 5 am. Las temperaturas más altas se presentan en la mayoría de los meses pero principalmente entre los meses de diciembre y abril y la hora en la que más se presentan es a las 12 m.m.\\

Basado en los registros de las estaciones convencionales del IDEAM desde 1971 hasta 2016, se evidenció un aumento de la temperatura de 1.34\celc \ . Adicionalmente, se evidencian cambios en la frecuencia de las bajas y las altas temperaturas, ya que la frecuencia de las bajas temperaturas ha estado disminuyendo y las altas temperaturas han presentado un aumento en los valores registrados.\\

Existe una gran variedad de modelos de pronóstico del tiempo usados para el pronóstico de las temperaturas del aire, pero cada uno de ellos posee fortalezas y debilidades. Para escoger el mejor modelo se deberá tener en cuenta las necesidades y la capacidad de cómputo que se posea.\\


%\chapter{Capítulo 2}
%Actividad: Solicitar y realizar un control de calidad a la información meteorológica procedente de las estaciones en la Sabana de Bogotá pertenecientes a la red nacional de estaciones de monitoreo de la atmósfera administrada por el IDEAM.\\

\section{Control de calidad de la información de las estaciones meteorológicas}

Los datos del modelo se van a comparar con los datos de las estaciones convencionales ya que las estacinoes convencionales poseen datos horarios y esto permite una mejor comparación y ajuste. La información utilizada es proveniente de las estaciones convencionales y automáticas pertenecientes a la red nacional administrada por el Instituto de Hidrología, Meteorología y Estudios Ambientales de Colombia (IDEAM). La información recolectada a través de estaciones convencionales es medida directamente por un observador en horas específicas del día, en cambio las estaciones automáticas se tienen registros en tiempo real horarios, reportados en la plataforma HYDRAS.\\

%Se usaron los datos provenientes de las estaciones meteorológicas del IDEAM. La información usada proviene de estaciones convencionales y estaciones automáticas. En las estaciones convencionales una persona se encarga de tomar datos entre una y doce veces por día y las estaciones automáticas son estaciones automatizadas que reportan datos horarios al IDEAM, a través de la red HYDRAS.\\

Los datos de las estaciones automáticas y convencionales según el IDEAM tienen un proceso de control de calidad \citep{Torres2005}, pero se realizó un análisis exploratorio a los datos de esta institución y se encontraron algunas inconsistencia en los datos. Por esta razón se decidió realizar un proceso de control de calidad sobre los datos. El control de calidad se realizó usando el el software python 3.6.\\

Un no dato corresponde a un dato que no fue registrado o fue registrado incorrectamente y estará representado por las letras NaN.

\subsection{Control de calidad de la información de las estaciones convencionales}

Se solicitaron los datos de la red de estaciones convencionales del IDEAM desde el registro más antiguo hasta el registro más cercano a diciembre del 2017 de todas las estaciones presentes en Cundinamarca y Bogotá en formato tr5, ya que el formato tr5 posee un registro diario de la información, el formato tr8 no.\\

El formato tr5 es una matriz de datos que están separados por espacios, la explicación del uso de estos datos fueron brindados por el IDEAM y un documento llamado "Manejo de los archivos de texto del banco de datos" \citep{pedraza2015} con base en este documento se creó un código que permite facilitar la labor de interpretación de los datos, código llamado \texttt{pre\_procesamiento\_ideam.py}.\\

El código \texttt{pre\_procesamiento\_ideam.py} lee los datos y los organiza como lo sugiere \citet{pedraza2015}, adicionalmente, agrega la georreferenciación de las estaciones basada en el \textit{Shapefile} que proporciona el IDEAM en su página de Internet en la sección  \textcolor{blue}{ \href{http://www.ideam.gov.co/solicitud-de-informacion}{Solicitud de información}} \textcolor{blue}{ \href{institucional.ideam.gov.co/jsp/info/institucional/publicaciones/CATALOGO_ESTACIONES_IDEAM_V10_AGOSTO2017.zip}{Catálogo Shape del IDEAM}}.\\


\subsubsection{Control de calidad de los datos de las estaciones convencionales}

Se analizaron los datos de las 162 estaciones convencionales del Anexo \ref{anexo:estaciones_autom_y_conv_usadas},  que se encuentran dentro del polígono en la Figura \ref{gra:areas}. Los valores que no pasan las pruebas no son tenidos en cuenta para el análisis posterior. Los pasos realizados para la validación fueron los siguientes:\\

% Gráfica de las estaciones y las ubicaciones

%\begin{figure}[H]
%	\begin{center}
%		\includegraphics[draft=false, scale=0.2]{municipios_paramos.png}
%		
%		\label{gra:areas_paramo}
%	\end{center}
%    	\caption{Área de estudio}
%	\label{gra:areas}
%\end{figure}
%
\begin{enumerate}


	\item Conteo de no valores.\\
	Para cada estación convencional se calcularon el número de no-valores.
    
    \item Prueba de límites fijos.\\
	Se realizó una prueba de rango con la finalidad de determinar cuáles datos se salen de estos límites como lo sugiere \citet{estevez2011}. Los valores de temperatura ($T$), deben estar entre -30 y 50\celc. Esta prueba fue superada por todos los datos de todas las estaciones convencionales.


	\item Detección de saltos.\\
	Se usó la metodología propuesta por \citet{vickers1997} quienes proponen usar el promedio móvil (ver Ecuación \ref{eq:prom-mobil}) y la desviación estándar móvil (ver Ecuación \ref{eq:desv-mobil}). Donde el valor $T_i$ no puede ser superior al promedio móvil mas o menos 2.5 veces la desviación estándar móvil (ver Ecuación \ref{eq:Saltos}).

	
%	El valor de la temperatura en la posición i-ésima debe ser superior al promedio móvil en la posición i-ésima menos dos veces y medio el valor de la desviación estándar móvil en la posición i-ésima, y el valor i-ésimo de la temperatura no debe superar el valor promedio móvil en la posición i-ésima más dos veces y medio el valor de la desviación estándar móvil en la posición i-ésima, como lo indica la ecuación \ref{eq:Saltos}, de lo contrario el dato será marcado como sospechoso.
	
\begin{equation}\label{eq:prom-mobil}
    x_{mob} = \frac{\sum\limits_{n=i-k}^{i+k} T_{n}}{2k + 1}
\end{equation}

\begin{equation}\label{eq:desv-mobil}
s_{mob} = \sqrt[2]{\frac{\sum\limits_{n=i-k}^{i+k} (T_{n} - x_{mob})^2}{2k}} 
\end{equation}
 
\begin{equation}\label{eq:Saltos}
	     x_{mob} - (2.5\cdot{s_{mob}}) < T_{i} < x_{mob} + (2.5\cdot{s_{mob}}) 
\end{equation}

%%% Gráfica de los saltos anuales


% \item Prueba de discontinuidad.\\
% Se realizó basado en el artículo de \citet{roggero2012} quien usa la desviación estándar como una medida de homogeneidad dentro de la serie de tiempo, teniendo en cuenta que las variables aquí medidas corresponden a un valor diario, por lo tanto podemos pensar en que se deben presentar ciertos valores de homogeneidad.\\

% Se construyó una serie de valores de temperatura ejemplo para poder validar la metodología. El ejemplo consistió en crear 400 datos artificiales de temperatura con un promedio $a$ y una desviación estándar $b$ con una distribución normal, a partir del dato 401 generar datos con un promedio $c;\ c > a$ y una desviación estándar $b$ con una distribución normal y a partir del dato 450 hasta el dato 1050 los datos tienen un promedio $a$ y una desviación estándar $b$, ver Figura \ref{subfig:discontinuidad} puntos azules. Cada 11 datos consecutivos se les calculó la desviación estándar estos pueden ser observados en la Figura \ref{subfig:discontinuidad} como puntos de color naranja. A los valores de desviación estándar obtenidos se les calculó el promedio y la desviación estándar; estos dos valores se sumaron y están representados por la linea azul en la Figura \ref{subfig:discontinuidad}.


% En este ejemplo se puede ver como en ciertos momentos la desviación estándar supera el límite y luego cuando los datos vuelven a la normalidad se presenta otro pico en la desviación estándar, en estos casos los datos que superen la línea azul, los datos serán marcados como sospechosos.



\end{enumerate}

\subsubsection{Visualización del control de calidad de las estaciones convencionales.}

Las estaciones convencionales diariamente miden temperaturas máxima, mínima y promedio. Para los 3 tipos de temperatura se aplicó el control de calidad generando como resultado un diagnóstico del estado de cada una de las estaciones.\\

Para poder visualizar más fácilmente el control de calidad de los datos se construyeron varias gráficas como se puede ver en la Figura \ref{gra:faltantes_saltos}.

\begin{figure}[H]
	\centering
	\begin{subfigure}[normla]{0.4\textwidth}
	\includegraphics[draft=false, scale=0.4]{validacion_convencionales/21205420_1_1.png}
		\caption{Conteo de no valores anuales.}
		\label{subfig:f1}
		\end{subfigure}
		~
    \begin{subfigure}[normla]{0.4\textwidth}
	\includegraphics[draft=false, scale=0.4]{validacion_convencionales/21205420_1_2.png}
		\caption{Conteo de no valores mensuales.}
		\label{subfig:f2}
		\end{subfigure}
		
    \begin{subfigure}[normla]{0.4\textwidth}
	\includegraphics[draft=false, scale=0.4]{validacion_convencionales/21205420_1_3.png}
		\caption{Conteo de saltos anuales.}
		\label{subfig:f3}
		\end{subfigure}
		~
    \begin{subfigure}[normla]{0.4\textwidth}
	\includegraphics[draft=false, scale=0.4]{validacion_convencionales/21205420_1_4.png}
		\caption{Conteo de saltos mensuales.}
		\label{subfig:f4}
		\end{subfigure}

	
	\caption{Visualización del control de calidad hecho a los valores de temperaturas diarias promedio de la estación convencional Tibaitatá. Se encuentran los datos faltantes anuales y mensuales y el conteo de los saltos anuales y mensuales.}
	%La subfigura a corresponde a los datos faltantes anuales, la subfigura b corresponde a los datos faltantes mensualmente, la subfigura c corresponde a los saltos anuales y la subgráfica d corresponde a los saltos que se presentan mensualmente.
	\label{gra:faltantes_saltos}	
\end{figure}

%Se realizaron figuras similares a la Figura \ref{gra:faltantes_saltos} y se encuentran en el Anexo \ref{anexo:gra_mensual_eventos_sb}.
En este caso para la estación Tibaitatá en la Figura \ref{subfig:f1} vemos que es una estación que desde 1980 tiene casi la totalidad de sus datos de temperatura. En los años 1996 y 2014 se presentó una disminución en la toma de registros de los datos. Con respecto a el conteo de no valores mensuales podemos observar que en la Figura \ref{subfig:f2} todos los meses poseen la misma cantidad de no valores. Con respecto a los saltos presentados a través de todos los años Figura \ref{subfig:f3} vemos que el máximo valor fué de 4 y se presentó 3 veces, este es un valor bajo de fallas. En el gráfico de los saltos mensuales vemos que la mayor cantidad de saltos se presentan en el mes de julio pero los valores continúan siendo bajos. En general las estaciones de la zona presentan un buen comportamiento, el fallo más común son los no valores. No se encontró una relación entre los no valores y los saltos y los meses. Un resumen de cada una de las estaciones se encuentra Anexo \ref{anexo:resumen_control_calidad_est_con2}.

%% Está creado el script para la adición de todas las imágenes que ya están echas generador_graficas_latex.py
\subsection{Control de calidad de las estaciones automáticas de la red HYDRAS}

La validación de las estaciones automáticas de la red HYDRAS fue diferente a la validación de las estaciones convencionales ya que estas estaciones toman datos horarios. Para realizar este trabajo se realizó el control de calidad de las estaciones automáticas dentro del polígono que se muestra en la Figura \ref{gra:areas}.

\subsubsection{Descarga de los datos}
Para la descarga de los datos se realizó un código llamado \texttt{descarga\_IDEAM\_hydras.py}, este código permite la descarga de los archivos de la red de estaciones del IDEAM siempre y cuando se tenga el número SID que es facilitado por la entidad, mediante un usuario y una contraseña.

\subsubsection{Control de calidad de la temperatura del aire}

Se realizó un preprocesamiento de la información proveniente de la red HYDRAS. La finalidad del preprocesamiento es generar archivos homogéneos y de esta forma facilitar la validación de los datos. El preprocesamiento consistió la generación de archivos homogéneos en cuanto a forma de las variables horarias de temperaturas mínimas, máxima, y media, humedad y rapidez del viento de una misma estación, con la finalidad de generar un grupo de archivos que permita una fácil manipulación.\\

La plataforma HYDRAS presenta la temperatura de tres formas diferentes las cuales son: "Temp Max Aire 2m", "Temp Min Aire 2m" y "Temp Aire 2m", las cuales hacen referencia a las temperaturas máximas, mínimas y promedio, respectivamente. Estas variables se unieron dándole prioridad a la temperatura promedio por ejemplo: si en un mismo momento se presentan datos provenientes de "Temp Max Aire 2m" y de "Temp Aire 2m" solo se tomará el valor de "Temp Aire 2m". La forma como se realizó el control de calidad fue el siguiente:


\begin{enumerate}
\item Conteo de no valores (No datos).\\
	Para cada estación automática se calculó el número de no-valores.

\item Prueba de límites fijos (P. Rango).\\
Lo siguiente fue realizar una prueba de rango similar a la usada en el primer paso de la validación de las estaciones convencionales, propuesta por \citet{estevez2011}, para quienes las temperaturas $T$ en grados Celsius no deben ser menores a -20\celc\ ni mayores a 40\celc. Los rangos de la temperatura fueron modificados teniendo en cuenta las temperaturas extremas registrados por las estaciones convencionales de modo que se sumó 10\celc\ al límite superior y se le restó 10\celc\ al límite inferior. Así el rango de no validez de los valores de temperatura es $-30$\celc $< T < 50$\celc.\\

\item Detección de saltos (P. Saltos).\\
Para la detección de los saltos se usó una metodología similar usada para la detección de saltos en las estaciones convencionales (ver Ecuación \ref{eq:Saltos}), metodología propuesta por \citet{vickers1997}.

Se usó una ecuación similar a la Ecuación \ref{eq:Saltos}. En este caso el factor de 2.5 es cambiado por 1 para ser más selectivo en el filtro, ver ecuación \ref{eq:Saltos_aut2}.

\begin{equation}\label{eq:Saltos_aut2}
	     x_{mob} - (1\cdot{s_{mob}}) < T_{i} < x_{mob} + (1\cdot{s_{mob}}) 
\end{equation}

\item Prueba de paso (P. Paso).\\
Según \citet{estevez2011} en dos datos consecutivos  de temperatura del aire con una hora de diferencia no puede presentarse una diferencia de 4\celc\ en valor absoluto. Datos que superen esta diferencia deberán ser marcados como valores sospechosos.\\

\item Prueba de variación (P. Variación).\\

Se realizó una prueba para evaluar si el sensor no presenta variación por un lapso de tiempo de 10 horas. Cuando el sensor de temperatura no presente variación en los datos de temperatura, estos deben ser marcados como sospechosos. Para esto se evalúa que la desviación estándar de una ventana de 10 horas debe ser superior a 0.01. Esta prueba  se basó en \citet{zahumensky2004} y \citet{Shafer2000} quienes proponen que si la desviación estándar de ciertas variables cae por debajo de un límite, estos datos se deben reportar como sospechosos. Esto porque, es probable que el sensor esté reportando el mismo valor (ver Ecuación \ref{eq:desv_est}).



\begin{center}
\begin{equation}\label{eq:desv_est}
	s_{mob} > 0.001
\end{equation}
\end{center}

\item Límites con error (Límites)\\

Se realizó un análisis de los datos detalladamente y se evidenció que en muchas de las estaciones se presentaron valores de temperatura entre 0\celc y 0.4\celc en horas del medio día. Por esta razón se decidió eliminar todos los valores de temperatura que se encuentran entre estos dos límites.

\newpage
\begin{landscape}



\begin{longtable}{p{5cm}rrrrrrr}


\caption{Porcentaje de valores que no superaron cada una de las pruebas de control de calidad y el total de los datos analizados para el control de calidad de los valores de temperatura del aire de las estaciones automáticas de la zona de estudio. Conteo de No valores (No. Datos), prueba de límites fijos (P. Rango), prueba de paso (P. Paso), prueba de variación (P. Variación), prueba de saltos (P. Saltos) y límites con error (Límites).}
\label{tab:cc_tmp}\\
\hline

                  Nombre &  No. Datos &  P. Rango (\%) &  P. Paso (\%)&  P. Variación (\%)&  P. Saltos (\%)&  Límites &Total datos \\
\midrule
\endhead
\midrule
\multicolumn{8}{r}{{Continúa en la siguiente página}} \\
\midrule
\endfoot
\bottomrule
\endlastfoot
        SUBIA AUTOMATICA &        0.0 &      0.02 &           0.85 &     0.03 &      18.40 &          0.01 &       110033 \\
 ESC LA UNION AUTOMATICA &        0.0 &      0.00 &           0.14 &     0.02 &      20.44 &          0.00 &       378885 \\
    PASQUILLA AUTOMATICA &        0.0 &    100.00 &           0.00 &    99.99 &       0.00 &          0.00 &        72784 \\
           UNIV NACIONAL &        0.0 &      7.39 &           9.41 &     0.22 &      16.13 &          0.10 &       136701 \\
    LA BOYERA AUTOMATICA &        0.0 &      2.02 &           5.25 &     0.09 &      13.66 &          0.42 &       247084 \\
      PNN CHINGAZA AUTOM &        0.0 &      1.04 &          11.54 &     0.22 &      19.64 &          4.83 &       209566 \\
      BOSQUE INTERVENIDO &        0.0 &      0.00 &           1.56 &     0.69 &      13.93 &          0.00 &       109860 \\
          APTO EL DORADO &        0.0 &      0.00 &           0.68 &     0.03 &      14.47 &          0.00 &        78636 \\
       HDA STA ANA AUTOM &        0.0 &      4.72 &          10.49 &     0.16 &      15.52 &          0.72 &       265428 \\
            PMO GUERRERO &        0.0 &      2.54 &           7.21 &     0.13 &      16.47 &          1.02 &       257602 \\
         PMO GUACHENEQUE &        0.0 &      5.79 &           7.46 &     0.14 &      20.06 &          0.15 &       265070 \\
            IDEAM BOGOTA &        0.0 &      1.57 &           3.36 &     0.01 &      15.79 &          0.11 &       227182 \\
      STA CRUZ DE SIECHA &        0.0 &      1.46 &           5.88 &     1.38 &      18.16 &          1.40 &       254552 \\
    TIBAITATA AUTOMATICA &        0.0 &      9.53 &          25.13 &     0.11 &      18.35 &          6.70 &       314116 \\
      SAN CAYETANO AUTOM &        0.0 &      0.86 &           3.79 &     0.06 &      20.03 &          0.67 &       189971 \\
            PMO CHINGAZA &        0.0 &      7.47 &           7.05 &     0.11 &      19.98 &          0.29 &       120632 \\
    CHINAVITA AUTOMATICA &        0.0 &      4.07 &          24.96 &     0.66 &      16.11 &          9.71 &       280902 \\
  PMO RABANAL AUTOMATICA &        0.0 &      5.39 &          11.33 &     1.67 &      18.14 &          4.69 &       253532 \\
        NUEVA GENERACION &        0.0 &      0.59 &           2.75 &     4.75 &      13.96 &          6.49 &        96604 \\
        SUBIA AUTOMATICA &        0.0 &      0.02 &           0.85 &     0.03 &      18.40 &          0.01 &       110033 \\
 ESC LA UNION AUTOMATICA &        0.0 &      0.00 &           0.14 &     0.02 &      20.44 &          0.00 &       378885 \\
    PASQUILLA AUTOMATICA &        0.0 &    100.00 &           0.00 &    99.99 &       0.00 &          0.00 &        72784 \\
           UNIV NACIONAL &        0.0 &      7.39 &           9.41 &     0.22 &      16.13 &          0.10 &       136701 \\
    LA BOYERA AUTOMATICA &        0.0 &      2.02 &           5.25 &     0.09 &      13.66 &          0.42 &       247084 \\
      PNN CHINGAZA AUTOM &        0.0 &      1.04 &          11.54 &     0.22 &      19.64 &          4.83 &       209566 \\
      BOSQUE INTERVENIDO &        0.0 &      0.00 &           1.56 &     0.69 &      13.93 &          0.00 &       109860 \\
          APTO EL DORADO &        0.0 &      0.00 &           0.68 &     0.03 &      14.47 &          0.00 &        78636 \\
       HDA STA ANA AUTOM &        0.0 &      4.72 &          10.49 &     0.16 &      15.52 &          0.72 &       265428 \\
            PMO GUERRERO &        0.0 &      2.54 &           7.21 &     0.13 &      16.47 &          1.02 &       257602 \\
         PMO GUACHENEQUE &        0.0 &      5.79 &           7.46 &     0.14 &      20.06 &          0.15 &       265070 \\
            IDEAM BOGOTA &        0.0 &      1.57 &           3.36 &     0.01 &      15.79 &          0.11 &       227182 \\
      STA CRUZ DE SIECHA &        0.0 &      1.46 &           5.88 &     1.38 &      18.16 &          1.40 &       254552 \\
    TIBAITATA AUTOMATICA &        0.0 &      9.53 &          25.13 &     0.11 &      18.35 &          6.70 &       314116 \\
      SAN CAYETANO AUTOM &        0.0 &      0.86 &           3.79 &     0.06 &      20.03 &          0.67 &       189971 \\
            PMO CHINGAZA &        0.0 &      7.47 &           7.05 &     0.11 &      19.98 &          0.29 &       120632 \\
    CHINAVITA AUTOMATICA &        0.0 &      4.07 &          24.96 &     0.66 &      16.11 &          9.71 &       280902 \\
  PMO RABANAL AUTOMATICA &        0.0 &      5.39 &          11.33 &     1.67 &      18.14 &          4.69 &       253532 \\
        NUEVA GENERACION &        0.0 &      0.59 &           2.75 &     4.75 &      13.96 &          6.49 &        96604 \\
 VILLA TERESA AUTOMATICA &        0.0 &      4.35 &           8.61 &     0.04 &      21.05 &          0.03 &       243016 \\
          CIUDAD BOLIVAR &        0.0 &      8.19 &          16.18 &     0.02 &      17.34 &          1.30 &       229394 \\
          CALOSTROS BAJO &        0.0 &     13.03 &          25.15 &     0.15 &      17.04 &          0.42 &       151190 \\
        LA CAPILLA AUTOM &        0.0 &      3.35 &           7.67 &     0.46 &      16.15 &          0.97 &       256730 \\
\end{longtable}
\end{landscape}

En la Tabla \ref{tab:cc_tmp} podemos observar que para las estaciones analizadas hay un bajo porcentaje de no valores, los porcentajes de valores fuera de rango son menores al 10\% bajos excepto en las estaciónes PASQUILLA AUTOMATICA y CALOSTROS BAJO. En la estación PASQUILLA AUTOMATICA el valor es del 100\% por esta razón esta estación no se usará para los análisis, adicionalmete presenta un alto porcentaje de valores que no superan la prueba de variación, esto quiere decir que es probable que el sensor se encuentre averiado y esté reportando los mismos valores. En los valores de la prueba de paso podemos observar que hay una mayor cantidad de valores que no pasan esta prueba. En la prueba de variación vemos que se presentan valores entre 10 a 25\%, esto quiere decir que hay una proporción similar de datos que no cumplen esta prueba.

\end{enumerate}

\subsubsection{Control de calidad de la humedad}

Para el control de calidad de la humedad relativa se realizaron los siguientes pasos.\\

\begin{enumerate}
\item Identificación de los no valores (No datos).\\
\item Prueba de paso (P. Diferencia). Se marcaron como datos sospechosos aquellos datos consecutivos cuya diferencia es mayor a 45\%, siguiendo los criterios de \citet{estevez2011}.\\
\item Prueba de límites fijos (P. Rango).
Se marcaron como sospechosos los valores de humedad relativa que son menores a 0.8\% y superiores a 100\%, siguiendo los criterios de \citet{estevez2011}.

\end{enumerate}

\begin{table}[H]
\begin{center}

\caption{Porcentaje de valores que no superaron cada una de las pruebas de control de calidad y el total de los datos analizados para el control de calidad de los valores de humedad relativa de las estaciones automáticas de la zona de estudio. Conteo de no valores (No datos), prueba de paso (P. Diferencia) y prueba de límites (P. Rango).}
\label{tabla:val_humedad}

\begin{tabular}{p{5cm}rrrr}
\toprule
      Nombre &  No datos (\%) &  P. Diferencia (\%) &  P. Rango (\%) & Total datos \\
\midrule
         SUBIA AUTOMATICA &      5.18 &           0.13 &      0.23 &         42187 \\
 ESC LA UNION AUTOMATICA &     29.07 &           0.01 &      0.02 &        389065 \\
    PASQUILLA AUTOMATICA &      0.00 &           0.00 &    100.00 &         72728 \\
           UNIV NACIONAL &      1.48 &           0.05 &      0.01 &         49439 \\
    LA BOYERA AUTOMATICA &      7.71 &           0.08 &      0.02 &         96263 \\
      PNN CHINGAZA AUTOM &      5.07 &           0.38 &     14.68 &         76796 \\
      BOSQUE INTERVENIDO &      0.45 &           0.65 &      7.18 &         45222 \\
          APTO EL DORADO &      0.73 &           0.01 &      0.00 &         30407 \\
       HDA STA ANA AUTOM &      4.22 &           4.21 &      3.73 &        101293 \\
            PMO GUERRERO &      3.44 &           0.08 &      0.25 &         96298 \\
         PMO GUACHENEQUE &      1.08 &           0.38 &      2.27 &         97235 \\
            IDEAM BOGOTA &      2.65 &           0.11 &      0.03 &         70847 \\
      STA CRUZ DE SIECHA &      6.89 &           0.21 &      0.11 &         97204 \\
    TIBAITATA AUTOMATICA &      5.35 &           2.03 &      3.95 &        106805 \\
      SAN CAYETANO AUTOM &      9.24 &           0.05 &      0.16 &         76799 \\
            PMO CHINGAZA &     22.37 &           3.23 &      8.40 &         51623 \\
    CHINAVITA AUTOMATICA &     12.00 &           0.08 &      0.20 &        104087 \\
  PMO RABANAL AUTOMATICA &      4.25 &           1.17 &      9.14 &         90622 \\
        NUEVA GENERACION &      0.96 &           2.34 &     12.05 &         52663 \\
 VILLA TERESA AUTOMATICA &      3.17 &           0.75 &      2.18 &         87711 \\
          CIUDAD BOLIVAR &      3.69 &           0.01 &      0.01 &         32543 \\
          CALOSTROS BAJO &      6.86 &           0.09 &     26.68 &         71950 \\
        LA CAPILLA AUTOM &      2.26 &           7.97 &      6.40 &         96145 \\
\bottomrule
\end{tabular}
\end{center}
\end{table}

En la Tabla \ref{tabla:val_humedad} podemos observar que tres estaciones tienen No datos en más del 10\% de sus datos, esto implica que en esta estaciones no se se encuentren todas las fechas de análisis. Las estaciones presentaron buenos resultados con respecto a la prueba de diferencia ya que ninguna estación falló en más del diez porciento. En las pruebas de rango se observa que la estación PASQUILLA AUTOMATICA posee los valores fuera de rango en un 100\%, esta estación también presentó falencias en cuanto a la variable de temperatura al igual que la estación CALOSTROS BAJO.

\subsubsection{Control de calidad de la precipitación}

En la plataforma HYDRAS hay dos variables de precipitación una llamada "Precipitación instantánea 10 min" y otra llamada "Precipitación acumulada diaria". Se realizó una entrevista con un funcionario del IDEAM encargado de la automatización de las estaciones y él asevera que las estaciones toman los valores de precipitación del día desde las 7 a.m hasta las 7 a.m. del siguiente día (Villareal, 2018; comunicación personal). En este estudio se va a realizar una comparación de los datos de las estaciones automáticas con los datos generados por un modelo, estos datos vienen horarios. Razón por la cuál se escogió trabajar con la variable "Precipitación instantánea 10 min".


%El modelo de pronóstico del tiempo fue configurado para generar resultados cada hora, por esta razón se decidió escoger la variable de precipitación que emite valores cada hora, es decir se descartó la variable de precipitación horaria. Los pasos de la validación fueron los siguientes:

Para el control de calidad de la precipitación instantánea cada 10 minutos se realizaron los siguientes pasos:

\begin{enumerate}
\item Identificación de los no valores (No datos).\\
Se marcaron los no-valores.
\item Prueba de limites (P Rango).\\
Se marcaron como sospechosos los valores que son menores que 0 mm y superiores a 120 mm \citep{estevez2011}.

\end{enumerate}


\begin{table}[H]
\begin{center}


\caption{Porcentaje de valores que no superaron cada una de las pruebas de control de calidad y el total de los datos analizados para el control de calidad de los valores de precipitación diezminutal de las estaciones automáticas de la zona de estudio. Conteo de los no valores (No datos), prueba de límites (P. Rango).}
\label{tabla:val_precipitacion}
\begin{tabular}{lrrr}
\toprule
      Nombre &  No datos (\%) &  P. Rango (\%) & Total datos \\
\midrule
         SUBIA AUTOMATICA &      5.37 &      0.00 &        253275 \\
 ESC LA UNION AUTOMATICA &      1.04 &      0.00 &        514372 \\
    PASQUILLA AUTOMATICA &      0.00 &      0.00 &         73007 \\
           UNIV NACIONAL &      1.47 &      0.01 &        296639 \\
    LA BOYERA AUTOMATICA &      8.04 &      1.14 &        577583 \\
      PNN CHINGAZA AUTOM &      1.67 &      0.05 &        441383 \\
      BOSQUE INTERVENIDO &      0.30 &      0.00 &        291415 \\
          APTO EL DORADO &      0.72 &      0.00 &        182447 \\
       HDA STA ANA AUTOM &      3.76 &      1.15 &        600981 \\
            PMO GUERRERO &      3.40 &      0.91 &        579799 \\
         PMO GUACHENEQUE &      5.51 &      0.00 &        609366 \\
            IDEAM BOGOTA &      2.94 &      0.01 &        442367 \\
      STA CRUZ DE SIECHA &      7.05 &      0.12 &        582126 \\
    TIBAITATA AUTOMATICA &      4.65 &      0.00 &        632398 \\
      SAN CAYETANO AUTOM &     21.44 &      0.01 &        531215 \\
            PMO CHINGAZA &      6.87 &      1.00 &        583055 \\
    CHINAVITA AUTOMATICA &     12.71 &      0.00 &        624527 \\
  PMO RABANAL AUTOMATICA &      4.12 &      0.00 &        543501 \\
        NUEVA GENERACION &      2.01 &      2.83 &        194150 \\
 VILLA TERESA AUTOMATICA &      5.79 &      1.32 &        530238 \\
          CIUDAD BOLIVAR &      0.02 &      0.71 &        471866 \\
          CALOSTROS BAJO &      1.59 &      0.07 &        431428 \\
        LA CAPILLA AUTOM &      1.91 &      0.65 &        559971 \\
\bottomrule
\end{tabular}
\end{center}
\end{table}

En la tabla \ref{tabla:val_precipitacion} podemos observar que las estaciones SAN CAYETANO AUTOM y CHINAVITA AUTOMATICA presentaron no valores en más del 10\% de los dato, esto implica que es probable que las fechas que se analizarán puede que no se encuentren. Ninguno de las estaciones automáticas posee más del 10\% de no datos fuera del rango, esto implica que esta variable está registando buenos valores.

%validación de la Radiación

\subsubsection{Control de calidad de la radiación directa}

\begin{enumerate}
\item Identificación de no valores. (No datos).\\


\item Prueba de límites absolutos. (P Rango).\\
Los valores analizados no pueden ser menores a 0 $W/m^2$ ni superiores a 1500 $W/m^2$ \citep{estevez2011}. El valor de 1500 $W/m^2$ es bastante menor a la constante solar la cual es de 1366 $W/m^2$ la cuál corresponde a el máximo valor obtenido en el tope de la atmósfera.

\item Prueba de limites relativos (P. Cielo despejado).\\
La radiación nunca puede ser mayor a los valores de la radiación extraterrestre o radiación con cielo despejado, la función para su estimación es usada y explicada en  \citet{Allen1994}.\\

\centering 
\begin{equation}
    Q_s = S(1-\alpha)(\frac{\bar{d}}{d})^2cos(\zeta)\tau_s
\end{equation}

Donde $S$ es la constante solar, $\alpha$ es el albedo, $d$ es la distancia de la tierra hasta el sol, $\bar{d}$ es la distancia promedio de la tierra al sol y $\tau_s$ es la transmisividad de la atmósfera de la atmósfera.\\

El factor $cos(\zeta)$ puede ser calculado a partír de:

\begin{equation}
    cos(\zeta) = sin(\psi)sin(\delta) + cos(\psi)cos(\delta)cos(h)
\end{equation}

Donde

\begin{equation}
    \delta = 23.45 · \frac{\pi}{180}cos(\frac{2\pi(d - d_{solst})}{d_{year}})
\end{equation}

Donde $d$ es el día juliano, $d_{solst}$ es el día del del solsticio (173) y $d_{year}$ es el número de días en el año (365.25).\\

Adicionalmente, $h$ es la hora local definido por:

\begin{equation}
    h = \frac{(t_{utc}-12)\pi}{12} + \frac{\lambda\pi}{180}
\end{equation}

Dónde $\lambda$ es la longitud y $t_{UTC}$ es el tiempo en UTC (en horas).\\

Los resultados de la función varían dependiendo de la latitud, longitud y fecha del año, para su estimación se usó el paquete \textcolor{blue}{ \href{http://pvlib-python.readthedocs.io/en/latest/index.html}{pvlib-python}}. Los valores que exceden la radiación con cielo despejado fueron marcados como sospechosos. 

\item Prueba de paso (P. Diferencia). Se marcaron como datos sospechosos aquellos datos consecutivos cuya diferencia es mayor a 55 $W/m^2$, siguiendo los criterios de \citet{estevez2011}.\\


\end{enumerate}

\begin{table}[H]
\begin{center}

\caption{Porcentaje de valores que no superaron cada una de las pruebas de control de calidad y el total de los datos analizados para el control de calidad de los valores de radiación de las estaciones automáticas de la zona de estudio. Conteo de los no valores (No. datos), prueba de rango (P. Rango), prueba de limites relativos (P. Cielo despejado), prueba de paso (P. Diferencia).}
\begin{tabular}{p{3cm}rrp{3cm}p{3cm}r}
\toprule
      Nombre &  No. datos (\%) &  P. Rango (\%) &  P. Cielo despejado (\%) &  P. Diferencia (\%) & Total datos \\
\midrule
         SUBIA AUTOMATICA &       5.51 &      0.40 &                5.95 &           0.84 &         42187 \\
           UNIV NACIONAL &       1.60 &      3.43 &                6.99 &           0.25 &         49463 \\
    LA BOYERA AUTOMATICA &       7.75 &     10.60 &               15.12 &           0.25 &         96239 \\
          APTO EL DORADO &       0.72 &      0.00 &               34.00 &           0.28 &         30407 \\
       HDA STA ANA AUTOM &       4.45 &     15.79 &               15.53 &           1.37 &         98798 \\
            IDEAM BOGOTA &       3.45 &      5.48 &                7.52 &           0.38 &         73175 \\
      STA CRUZ DE SIECHA &       7.04 &     28.34 &               11.75 &           0.51 &         99672 \\
    TIBAITATA AUTOMATICA &       4.49 &     16.85 &                9.05 &           0.22 &        105260 \\
      SAN CAYETANO AUTOM &      17.68 &     23.50 &               13.53 &           0.17 &         94415 \\
    CHINAVITA AUTOMATICA &       3.37 &      3.50 &               28.40 &           2.43 &         88063 \\
        NUEVA GENERACION &       0.88 &      1.05 &                4.03 &           0.22 &         52979 \\
 VILLA TERESA AUTOMATICA &       2.69 &     19.83 &               40.18 &           6.69 &         90744 \\
          CIUDAD BOLIVAR &       0.12 &     28.48 &               18.19 &           6.63 &         81024 \\
        LA CAPILLA AUTOM &       1.40 &     31.25 &               19.04 &          16.33 &        103284 \\
\bottomrule
\end{tabular}		
		\label{tabla:val_radiacion}
\end{center}
\end{table}

En la tabla \ref{tabla:val_radiacion} podemos observar que en el conteo de los no datos la estación que presenta valores superiores a 10\% es la estación SAN CAYETANO AUTOM la cuál también presentó problemas en los valores de precipitación, esto implica que ésta estación tuvo fallas en varios sensores. Con respecto a el porcentaje de las estaciones que estuvieron fuera del rango fueron más de la mitad, esto quiere decir que los sensores están dando valores que no corresponden a la realidad en una buen parte de los registros y lo mismo sucedió con la prueba de cielo despejado. La prueba de paso sólo en la estación LA CAPILLA AUTOM presentó valores supériores a 10\%.

\subsubsection{Control de calidad de la rapidez del viento}

Velocidad

\begin{enumerate}
\item Identificación de no valores. (No datos).\\
Se marcaron los no valores

\item Prueba de límites (P. Rango).\\
Los valores analizados no pueden ser menores a 0 $m/s$ ni superiores a 60.3 $m/s$, basado en \citet{estevez2011}.

\item Prueba de paso. (P. Diferencia).\\
Se marcaron como datos sospechosos aquellos datos consecutivos cuya diferencia es mayor a 10 $m/s$, siguiendo los criterios de \citet{estevez2011}.

\item Prueba de variación. (P. Variación).\\

Se realizó una prueba para evaluar si el sensor no presenta variación por un lapso de tiempo de 10 horas. Cuando el sensor de velocidad del viento no presente variación en los datos de rapidez del viento , estos deben ser marcados como sospechosos.

\end{enumerate}

\begin{table}[H]
\begin{center}

\caption{Porcentaje de valores que no superaron cada una de las pruebas de control de calidad y el total de los datos analizados para el control de calidad de los valores de rapidez del viento de las estaciones automáticas de la zona de estudio. Conteo de los no valores (No datos), prueba de límites (P. Rango), prueba de paso (P. Diferencia) y prueba de variación (P. Variación).}

\begin{tabular}{p{3cm}rrrrr}
\toprule
      Nombre &  No datos  (\%)&  P. Rango  (\%)&  P. Diferencia  (\%)&  P. Variación  (\%)& Total datos \\
\midrule
        SUBIA AUTOMATICA &      34.65 &      0.00 &           0.01 &          6.34 &        253275 \\
           UNIV NACIONAL &      12.44 &      0.05 &           0.08 &          2.66 &        296639 \\
    LA BOYERA AUTOMATICA &      13.39 &      0.01 &           1.73 &          2.17 &        577583 \\
      PNN CHINGAZA AUTOM &      10.61 &      0.24 &           0.23 &          1.71 &        456455 \\
      BOSQUE INTERVENIDO &      17.75 &      0.00 &           0.00 &          4.43 &        271543 \\
          APTO EL DORADO &       0.73 &      0.00 &           0.00 &          0.06 &        182447 \\
       HDA STA ANA AUTOM &       1.71 &      0.05 &           0.05 &          0.94 &        335182 \\
            PMO GUERRERO &       7.69 &      1.25 &           1.74 &          2.64 &        389119 \\
         PMO GUACHENEQUE &      14.57 &      2.15 &           2.83 &          2.01 &        335288 \\
            IDEAM BOGOTA &       3.66 &      0.15 &           0.18 &          0.60 &        424943 \\
      STA CRUZ DE SIECHA &      29.34 &      0.62 &           0.88 &          1.79 &        554206 \\
    TIBAITATA AUTOMATICA &      15.63 &      0.97 &           2.14 &          1.07 &        386397 \\
      SAN CAYETANO AUTOM &      28.99 &      0.00 &           0.01 &          1.69 &        314783 \\
            PMO CHINGAZA &       7.81 &      0.00 &           0.01 &          1.81 &        346319 \\
    CHINAVITA AUTOMATICA &      17.71 &      1.97 &           3.40 &          2.44 &        567578 \\
  PMO RABANAL AUTOMATICA &      14.31 &      1.58 &           1.42 &          2.26 &        545127 \\
        NUEVA GENERACION &       2.25 &      0.12 &           0.76 &          3.20 &        186294 \\
 VILLA TERESA AUTOMATICA &      20.84 &      0.02 &           0.03 &          8.19 &        454729 \\
          CIUDAD BOLIVAR &       9.53 &      0.01 &           0.01 &          3.22 &        242805 \\
        LA CAPILLA AUTOM &      12.42 &      0.98 &           1.43 &          0.47 &        336067 \\
\bottomrule
\end{tabular}		
		\label{tabla:val_vel_viento}
\end{center}
\end{table}

En la tabla \ref{tabla:val_vel_viento} en el conteo de los no valores observamos que 13 de las 19 estaciones poseen no valores en más del 10\% de los valores. Las estaciones presentaron fallas en menos del 10\% para las pruebas prueba de límites, prueba de paso y prueba de variación. Esta es una variable que presentó buenos resultados.

\subsubsection{Control de calidad de la dirección del viento}

Para el control de calidad de la velocidad del viento se realizaron los siguientes pasos:

\begin{enumerate}
\item Identificación de no valores (No datos).\\

\item Prueba de límites (P. Rango)\\
Los valores analizados no pueden ser menores a 0$^{\circ}$ ni superiores a 360$^{\circ}$.
\end{enumerate}

\begin{table}[H]
\begin{center}

\caption{Porcentaje de valores que no superaron cada una de las pruebas de control de calidad y el total de los datos analizados para el control de calidad de los valores de dirección del viento de las estaciones automáticas de la zona de estudio. Conteo de los no valores (No. datos) y prueba de límites (P. Rango).}

\begin{tabular}{lrrr}
\toprule
      Nombre &  No. datos  (\%)&  P. Rango  (\%)& Total datos \\
\midrule
         SUBIA AUTOMATICA &      34.66 &      0.00 &        253275 \\
           UNIV NACIONAL &      12.59 &      3.61 &        296639 \\
    LA BOYERA AUTOMATICA &      16.40 &      1.59 &        598319 \\
      PNN CHINGAZA AUTOM &      28.28 &      0.01 &        471959 \\
      BOSQUE INTERVENIDO &      17.75 &      0.00 &        271543 \\
          APTO EL DORADO &       0.73 &      0.00 &        182447 \\
       HDA STA ANA AUTOM &      44.41 &      0.02 &        592578 \\
            PMO GUERRERO &       9.09 &      0.17 &        389719 \\
         PMO GUACHENEQUE &      53.48 &      0.45 &        612787 \\
            IDEAM BOGOTA &       5.17 &      0.00 &        431567 \\
      STA CRUZ DE SIECHA &      31.92 &      5.61 &        574398 \\
    TIBAITATA AUTOMATICA &      16.39 &      0.06 &        386711 \\
      SAN CAYETANO AUTOM &      57.65 &      0.03 &        527759 \\
            PMO CHINGAZA &       7.83 &      0.00 &        346319 \\
    CHINAVITA AUTOMATICA &      53.26 &      0.02 &        623491 \\
  PMO RABANAL AUTOMATICA &      15.53 &      0.98 &        548729 \\
        NUEVA GENERACION &       2.47 &      0.05 &        186648 \\
 VILLA TERESA AUTOMATICA &      20.87 &      0.00 &        454739 \\
          CIUDAD BOLIVAR &      30.14 &      1.03 &        243487 \\
        LA CAPILLA AUTOM &      47.15 &      1.30 &        551054 \\
\bottomrule
\end{tabular}
		\label{tabla:val_vel_viento}
\end{center}
\end{table}

En la Tabla \ref{tabla:val_vel_viento} 15 estaciones presentan no valores en más del 10\% y 3 estaciones presentan no valores en más del 50\% de los datos, estos son los valores más altos registrados de no valores teniendo en cuenta que se analizaron 20 estaciones. En la prueba de rango ninguna de las estaciones presentó valores superiores al 10\%. Esta es una de las variables que más ha tenido problemas ya que los registros están incompletos.


\subsubsection{Ajustes de control de calidad de las estaciones automáticas}

Se realizó el control de calidad de la temperatura del aire y los datos que se marcaron como sospechosos no fueron tomados en cuenta a partir de este punto. 

Algunas de las estaciones luego de haber realizado el control de calidad de los datos continúan presentando algunos datos que pueden ser errados. Por esta razón se recomienda posteriormente a el proceso de control de calidad realizar una visualización de los datos con el control de calidad. Se realizó un resumen de los datos sospechosos observados después de la validación (Tabla \ref{tab:res_tab_autom}).

\begin{table}[H]

\caption{Tabla resumen del control de calidad de las estaciones automáticas}
\label{tab:res_tab_autom}
\begin{tabular}{lllp{2cm}lp{3cm}}
   & Código   & Nombre                  & Datos sospechosos & Fechas      & Observación                                 \\ \hline
1  & 35025080 & PNN CHINGAZA AUTOM      & X                 & 2008        & Un pico de altas temperaturas               \\ \hline
2  & 35075080 & PMO RABANAL AUTOMATICA  &                   &             &                                             \\ \hline
3  & 35085080 & LA CAPILLA AUTOM        & x                 & 2001 – 2004 & No variación                                \\ \hline
4  & 21206990 & TIBAITATA AUTOMATICA    & X                 & 2017, 2018  & Altas temperaturas en pocos datos           \\ \hline
5  & 21206940 & CIUDAD BOLIVAR          & X                 & 2010        & Altas temperaturas                          \\ \hline
6  & 35035130 & PMO CHINGAZA            & X                 & 2014        & Altas temperaturas                          \\ \hline
7  & 21206920 & VILLA TERESA AUTOMATICA &                   &             &                                             \\ \hline
8  & 21205012 & UNIV NACIONAL           & X                 & 2013, 2017  & Altas y bajas temperaturas anómalas         \\ \hline
9  & 21206930 & PMO GUERRERO            &                   &             &                                             \\ \hline
10 & 21206790 & HDA STA ANA AUTOM       & X                 & 2015        & Altas temperaturas                          \\ \hline
11 & 23125170 & SAN CAYETANO AUTOM      &                   &             &                                             \\ \hline
12 & 35075070 & CHINAVITA AUTOMATICA    & X                 & 2008 – 2013 & No variación                                \\ \hline
13 & 35027510 & CALOSTROS BAJO          &                   &             &                                             \\ \hline
14 & 21206980 & STA CRUZ DE SIECHA      & X                 & 2015        & No variación                                \\ \hline
15 & 21206950 & PMO GUACHENEQUE         &                   &             &                                             \\ \hline
16 & 24015110 & LA BOYERA AUTOMATICA    &                   &             &                                             \\ \hline
17 & 21195160 & SUBIA AUTOMATICA        &                   &             &                                             \\ \hline
18 & 21206960 & IDEAM BOGOTA            & X                 & 2013, 2014  & Altas y bajas temperaturas anómalas         \\ \hline
19 & 35025090 & BOSQUE INTERVENIDO      &                   &             &                                             \\ \hline
20 & 21206600 & NUEVA GENERACION        & X                 & 2010, 2013  & Sensor pegado y picos de altas temperaturas \\ \hline
21 & 21201200 & ESC LA UNION AUTOMATICA &                   &             &                                             \\ \hline
22 & 21205791 & APTO EL DORADO          &                   &             &                                            
\end{tabular}
\end{table}
\begin{comment}


\begin{enumerate}


\item 21195160 Subia Automática. Presenta homogeneidad en los datos. No hay datos bajo 0\celc hay unos pocos datos sobre 25\celc. Se deben eliminar los datos que sean superiores a 3.5 veces la desviación estándar más el promedio.


\item 21201200 Esc La Unión Automática. Presenta cambios extraños hasta antes del 2012, es posible que se halla realizado un emplazamiento. Los valores en general después del 2012 mejoran. No hay datos por debajo de 0 ni por encima de 25\celc. Ubicada en Bogotá. Se deben eliminar los datos que sean superiores a 4 veces la desviación estándar más el promedio y los valores que sean 4 veces la desviación estándar menos el promedio.

\item 21201580 Pasquilla Automática estación que no presenta datos.

\item 21205012 Univ Nacional. Presenta varios picos de datos en la temperatura para el año 2004, 2013 y 2017. Posee datos bajo 0 y sobre 25\celc.

\item 21205791 Apto el Dorado. Presenta homogeneidad en los datos. No presenta datos bajo 0\celc y hay datos sobre 25\celc. Se deben eliminar los datos que sean superiores a 4 veces la desviación estándar más el promedio y los valores que sean 4 veces la desviación estándar menos el promedio.

\item 21206600 Nueva Generación. Presenta un pico de altas temperaturas en el año 2013 y en el 2010 el registró continuamente 0\celc. Se deben eliminar los datos que sean superiores a 3 veces la desviación estándar más el promedio y los valores que sean 3 veces la desviación estándar menos el promedio.

\item 21206790 Hacienda Santa Ana Automática, Nemocón. Presenta un pico de temperatura en fechas cercanas al 2016 y un pico de temperaturas bajas en el 2010, presenta homogeneidad. Se deben eliminar los datos que sean superiores a 4 veces la desviación estándar más el promedio y los valores que sean 4 veces la desviación estándar menos el promedio.

\item 21206920 Villa Teresa Automática. Presenta varios picos, pero los valores de estos picos no están fuera de los rangos propuestos por \citep{estevez2011}. Esta estación presenta datos de temperatura de buena calidad.

\item 21206930 PMO Guerrero. Estación que presenta buenos datos, se considera que no es necesario realizar la corrección de los datos.

\item 21206940 Ciudad Bolívar, presenta homogeneidad aunque existe un pico en el año 2010. Se presentan algunos valores atípicos. Se deben eliminar los datos que sean superiores a 3.5 veces la desviación estándar más el promedio y los valores que sean 3.5 veces la desviación estándar menos el promedio.

\item 21206950 Pmo Guacheneque. Presenta una buena homogeneidad, pero existen algunos datos fuera de los rangos, en especial valores inferiores a -20\celc. Exiten algunos picos pero parecen dentro de lo normal. No se recomienda aplicar filtro por desviación estándar.

\item 21206960 IDEAM Bogotá. Estación que presenta unos picos para el 2013 y 2014. Es una estación con homogeneidad marcada. No hay datos debajo de 0\celc ni superiores a 25\celc. Se deben eliminar los datos que sean superiores a 4 veces la desviación estándar más el promedio y los valores que sean 4 veces la desviación estándar menos el promedio.

\item 21206980 Sta cruz de Siecha. Es una estación que presenta una alta homogeneidad. Hay valores debajo de 0\celc y valores sobre 25\celc. No aplicar el filtro de la desviación estándar.

\item 21206990 Tibaitata Automática. Estación con gran homogeneidad presenta valores atípicos, valores sobre 25\celc y valores sobre 30\celc, además de valores por debajo de 0\celc. Se deben eliminar los datos que sean superiores a 4 veces la desviación estándar más el promedio y los valores que sean 4 veces la desviación estándar menos el promedio.



\item 23125170 San Pablo de Borbur. Es una serie que presenta muchos vacíos, pero presenta un buen comportamiento. Existen algunos casos de datos atípicos. Se deben eliminar los datos que sean superiores a 4 veces la desviación estándar más el promedio y los valores que sean 4 veces la desviación estándar menos el promedio.

\item 24015110 La Boyera. Estación con buen comportamiento no se considera necesario realizar otra corrección.

\item 35025080 PNN Chingaza la Calera. Estación con un largo periodo de tiempo sin datos. Existen dos periodos de tiempo con datos, al inicio del segundo periodo de datos se registran altas temperaturas. Se deben eliminar los datos que sean superiores a 4 veces la desviación estándar más el promedio y los valores que sean 4 veces la desviación estándar menos el promedio.

\item 35025090 Bosque Intervenido. Estación con algunos lapsos de tiempo faltante, pero con una buena homogeneidad. Yo no haría cambio alguno.

\item 35025090 Calostros Bajos. Estación con algunos periodos de tiempo con datos faltantes, pero con una buena homogeneidad. No es necesario aplicar otro filtro.

\item 35035130 Calostro Bajo. Estación con algunos periodos de tiempo con datos faltantes.

\item 35075070 Chinavita Fusavita. Estación con un buen comportamiento, presenta algunos datos atípicos. Se deben eliminar los datos que sean superiores a 4 veces la desviación estándar más el promedio y los valores que sean 4 veces la desviación estándar menos el promedio.

\item 35075080 PMO  Rabanál Boyacá, Ventaquemada. Estación en la que existe un período sin datos.

\item 35085080 La capilla La Unión. Estación con buen comportamiento, un período de tiempo sin datos en el año 2012. Se deben eliminar los datos que sean superiores a 4 veces la desviación estándar más el promedio y los valores que sean 4 veces la desviación estándar menos el promedio.


En la Tabla \ref{tab:res_tab_autom} podemos observar que aún quedaron valores de temperatura del aire con problemas, pero muchos de los datos no validos fueron eliminados y se obtuvo una mejor serie de datos, la cuál se presta para hacer análisis. Además hay unas estaciones que luego de el control de calidad no presentan datos sospechosos. Es importante notar que sólo la estación UNIV NACIONAL presenta datos sospechosos en los últimos años (2017), esto implica que la calidad de los datos luego ha venido mejorando. Los datos atípicos pueden ser eliminados fácilmente, pero antes de eliminarlos se debe hacer un análisis estación por estación teniendo en cuenta la zona, temporadas del año y años ENSO, ya que los datos sospechosos pueden ser datos extremos y si se eliminan sería un error.
\end{enumerate}
\end{comment}
Un ejemplo de la limpieza de datos es la que se realizó en la estación Tibaitatá. Esta estación presentaba temperaturas que alcanzaban los 6000\celc y los -6000\celc (Fig. \ref{subfig:c1}), pero luego de la corrección se pudo observar una mejora en sus datos (Fig. \ref{subfig:c2}).

\begin{figure}[H]
	\begin{subfigure}[b]{0.5\textwidth}
	\begin{center}
	\caption{Valores sin filtros}
    \includegraphics[draft=false, scale=0.5]{autom_validadas/21206990v.png}

	
	\label{subfig:c1}
		\end{center}
	\end{subfigure}
	~
		\begin{subfigure}[b]{0.5\textwidth}
	\begin{center}
    \caption{Valores con filtros y desviaciones estándar}
		\includegraphics[draft=false, scale=0.5]{autom_validadas/21206990.png}

	
	\label{subfig:c2}
	\end{center}	
	\end{subfigure}
	
	
	
	%% las demás gráficas están en /media/edwin/6F71AD994355D30E/Edwin/Maestría Meteorologia/Tesis/graficas
\end{figure}
\subsection{Resultados}


Existen varios métodos para el control de calidad de los datos meteorológicos, muchos de ellos se pueden aplicar, en su mayoría, en lugares con diferentes latitudes, pero éstos no son optimizados regionalmente, como lo resalta \citet{espinar2012controlling}. Las condiciones del clima de una región específica es influenciada por factores de escalas mayores y locales. Para lograr una descripción confiable del clima es necesario tener datos sobre periodos de tiempo lo suficientemente largos que aseguren el registro de diferentes oscilaciones climáticas \citep{kaspar2016climate}. Anteriormente la mayoría de las estaciones eran de tipo convencional. Últimamente se ha presentado un cambio paulatino en la forma de recolección de la información, ya que se ha cambiado de estaciones convencionales a estaciones automáticas.\\

Es necesario realizar un control de calidad de las estaciones meteorológicas, especialmente en las estaciones automáticas. Las estaciones automáticas de la red HYDRAS presentan gran cantidad de errores, pero son una gran herramienta para la investigación ya que ofrecen datos horarios útiles para poder observar fenómenos de escalas de tiempo cortas. El protocolo y los códigos acá realizados se pueden convertir en una guía para el análisis de este tipo de datos ya que existe una gran cantidad de información en la red HYDRAS.\\

Posteriormente al control de calidad de los datos, estos ya están aptos para su uso y para ser usados en próximos análisis.

%%%Me hace falta las correcciones de la página 6 del pdf
%%Llegué hasta la 16 del informe3v01Noviembre
\chapter{Capítulo 3}



%\textbf{Actividad: Escoger un modelo regional de pronóstico atmosférico y su configuración, que sean útiles para pronósticos de temperaturas máxima y mínima en la Sabana de Bogotá a partir de distintas parametrizaciones y de configuraciones del modelo y del cálculo de distintos estadísticos que evalúen el desempeño.}\\

\section{Escogencia del modelo de pronóstico}

El modelo WRF es un sistema de cálculo numérico para simular la atmósfera que fue diseñado para cumplir objetivos de investigación y pronóstico, este modelo sirve en un amplio rango de escalas espaciales, desde decenas de metros hasta miles de kilómetros. Los usuarios de este modelo pueden producir simulaciones basadas en condiciones atmosféricas reales o condiciones idealizadas \citep{Pielke2002}. WRF es capaz de realizar una reducción de escala de un modelo de pronóstico global como el Global Forecast System (GFS). La reducción de escala toma las condiciones del modelo global y aumenta la resolución de los resultados del modelo teniendo en cuenta las características de la zona de estudio \citep{Sene2010}.\\

El modelo WRF tiene una aproximación no-hidrostática, esto quiere decir que tiene en cuenta la ecuación del momentum en la dirección vertical $(w)$, en comparación con el modelo hidrostático el cual no lo tiene en cuenta. Los modelos no-hidrostáticos son usados para el pronóstico de fenómenos de mesoescala o escalas menores \citep{Pielke2002, Sene2010} ya que tienen en cuenta los movimientos verticales del viento \citep{ArmentaPorras2013}.\\
%. El modelo hidrostático supone una homogeneidad en la columna de aire y está dado por la densidad y la gravedad \citep{Pielke2002, Sene2010}.

El modelo WRF es un modelo muy usado alrededor del mundo gracias a los buenos resultados obtenidos, como lo reporta \citet{Jimenez2014}, quien evaluó el modelo en condiciones de precipitaciones extremas obteniendo como resultado que el modelo presenta diferencias en los resultados dependiendo de la ubicación con respecto a la altura sobre el nivel del mar, presentando mejores resultados en zonas bajas. En el ámbito internacional, el modelo WRF se ha usado en varios países como es el caso de Perú, \citet{Saavedra2016} realizó un trabajo de modelación de fenómenos meteorológicos de mesoescala, que se presentan en el Valle de Cajamarca, Perú. Como resultado se obtuvo que la modelación reproduce de buena forma el ciclo diario de las temperaturas del aire a dos metros, pero las temperaturas mínimas del aire a dos metros fueron sobrestimadas en las partes altas de la zona de estudio, y subestimó la tasa de enfriamiento en el fondo de los valles, generando una mayor temperatura modelada, respecto a los valores reportados por estaciones ubicadas en las laderas. La parametrización usada en el esquema de capa límite planetaria fue la de \textit{Yonsei University} \citep{Hong2006}.\\

\citet{Fernandez2011} realizaron un estudio sobre la ciudad de Mendoza, Argentina con tres dominios espaciales de 36, 12 y 4 km donde se tuvieron en cuenta las condiciones orográficas para la delimitación de los mismos. Se usó el esquema de capa límite planetaria de \textit{Yonsei University} (YSU; \citep{Hong2006}. Como resultado, se encontró que los valores máximos a dos metros de altura tanto de temperatura como de humedad a son predichos correctamente.\\

\citet{Corrales2015} usó el modelo WRF para realizar un pronóstico de las temperaturas en México usando un único dominio y la parametrización \textit{Mellor-Yamada-Janic} (MYJ; \citep{Janjic1994} para la capa límite planetaria. Como resultado, obtuvo que hay algunas zonas donde el modelo es confiable para la predicción de las temperatura, lo cual ayudar a puede prevenir daños por heladas en un horizonte de pronóstico de 5 días.\\

\citet{Hu2010} usaron el modelo WRF para modelar las temperaturas en el centro de Estados Unidos, en los estados de Texas, Lousiana y parte de Arkansas. Ellos probaron las  parametrizaciones YSU, MYJ de capa límite (PLB) y encontraron que la parametrización MYJ produce temperaturas más altas de lo esperado y YSU produce mejores resultados simulando la temperatura a dos metros.\\

\citet{Parra2012} realizó un estudio para Ecuador donde la finalidad era simular la meteorología de un año completo de todo el país. Para esto, usaron el modelo WRF con 2 dominios y con la parametrización MYJ, obtuvieron como resultado que las temperaturas simuladas con la parametrización MYJ son coherentes con los fenómenos observados en estaciones en superficie.\\


A partir de los estudios mencionados anteriormente, se evidencia la necesidad de probar varias parametrizaciones e intentar lograr la mejor combinación de ellas. Para el caso de Colombia, \citet{Uribe2012} escogió 10 eventos de duración de un día. Se probaron dos parametrizaciones diferentes de la convección de cúmulus, con dos diferentes resoluciones espaciales y dos diferentes horizontes de pronóstico con la finalidad de encontrar la mejor combinación. Como resultado encontró que la parametrización por el método de Kain-Fritsch \citep{JohnS2004} con una resolución de 20 km presentó la mayor subestimación para la precipitación, mientras que la parametrización de Morrison \citep{Morrison2009} presenta los mejores resultados. Y al aumentar el horizonte de pronóstico de 36 a 48 horas se logra una mejor simulación de los valores de precipitación \citep{Uribe2012}.\\

En Colombia, el Instituto de Hidrología, Meteorología y Estudios Ambientales (IDEAM) ha implementado el modelo WRF desde el año 2007 para la predicción del tiempo atmosférico convirtiéndose en una herramienta muy importante para esta entidad \citep{Arango2011}. \citet{Mejia2012} realizaron validaciones del modelo WRF para la Sabana de Bogotá para la variable precipitación mediante una comparación con estaciones meteorológicas convencionales. El objetivo fue comparar e identificar con cuales de las condiciones iniciales y de frontera proveniente de los dos modelos numéricos GFS \citep{NOAA2016} y MM5 WRF lograba identificar de manera aceptable los patrones de precipitación. Los límites de la rejilla están entre las longitudes 65\degree W y 85\degree W y las latitudes 5\degree S y 15\degree N, con una resolución espacial de 20 km. Se encontró que el modelo WRF alimentado con los datos del modelo GFS presentó los mejores resultados. Este estudio presenta una metodología útil para la validación de modelos y la determinación del mejor modelo.\\

%Estos estudios fueron realizados en diferentes cultivos como \citet{prabha2008} que usó el cultivo de durazno, \citep{Gomez2014} quien realizó su estudio en pasturas, \citep{Saavedra2016} quien estudio en una zona productora de papa y maíz.\\

%Es importante resaltar que el modelo toma como referencia datos del suelo. Pero según \citet{Castro-Romero2014} en la Región Andina se presentan cambios por el uso del paisaje que dejan como resultado tan solo el 31\% de los bosques naturales y una degradación del 53\% en arbustales secos. Se estima que para el año 1998 el 69\% de los bosques andinos habían sido talados. Uno de estos sucede en Suesca - Cundinamarca \citet{Castro-Romero2014} reporta que debido al uso agropecuario intensivo que se le ha dado a los suelos es posible observar zonas desprovistas de cobertura vegetal y de suelo con estados muy avanzados de degradación, lo cual imposibilita su posible recuperación en años próximos. Ya que los ecosistemas se usan de una forma extractiva como,lo menciona \citet{Ernesto}, uno de los ecosistemas más importantes de Colombia el Páramo es usado como una mina del que se extrae carbón, materia orgánica de los suelos y biomasa de sus páramos. Pero adicionalmente \citep{Ernesto} cuantificó la tasa a la que se extrae y la estimó en $2.49e-20 \frac{seJ}{year}$. Estos cambios nos hacen reflexionar acerca de la importancia del mantenimiento de los suelos y que se deben hacer ajustes periódicos a los modelos ya que con estos estudios se demuestra que el suelo es dinámico.

%Un modelo de mesoescala es un modelo numérico de predicción del tiempo atmosférico, que se usa para hacer una predicción a escala de kilómetros y horas, basado en la dinámica atmosférica \citep{Uribe2012}.\\
%####

Basado en esta información dada en los parámetros anteriores se decidió usar el modelo regional del tiempo atmosférico WRF, ya que es un modelo que cumple con los requerimientos para este estudio y en el país se ha usado este modelo con resultados satisfactorios como los obtenidos por \citet{Mejia2012,Arango2011,Arango2014,Ruiz2014,Uribe2012,Rojas2011,ArmentaPorras2013}.\\

\iffalse
Los pasos a seguir para realizar una modelación con el modelo WRF consiste en:
\begin{enumerate}
\item Compilar el WPS
\item Compilar el WRF
\item Determinar el área de estudio
\item Selección de los dominios del modelo WRF para la inicializacion del modelo
\item Selección de las parametrizaciones que se van a emplear
\item Descargar los datos que van a alimentar el modelo
\item Realizar el pre-proceso con el WPS
\item Realizar el proceso de modelación con el WRF
\item Realización del pos-proceso con python3.6


\end{enumerate}
\fi

\section{Selección de un período de estudio para escoger la configuración dominio-subdominio del WRF}

El período escogido es el comprendido entre el 1 de febrero del año 2007 a las 00 UTC hasta el 5 de febrero del año 2007 a las 00 UTC. Para estas fechas se presentó un evento de helada en la Sabana de Bogotá y fue uno de los más severos de los últimos 15 años, ver Figura \ref{fig:tmp_autom_tibaitata_1}. Las heladas comenzaron a presentarse desde el 2 de febrero de 2007 a las 10 horas UTC.
En la Figura \ref{fig:tmp_autom_tibaitata_1} se puede observar que la temperatura más baja para el período de tiempo comprendido entre el 1 de enero de 2005 y el 31 de diciembre del 2017 se presentó a comienzos del 2007, esta fecha corresponde al 4 de febrero y la temperatura fue de -4.7 \celc.
\begin{figure}[H]
    \centering
    \includegraphics[draft=false, scale=0.6]{graph/grafica_minimas_tmp_tibaitata.png}
    \caption{Valores de temperatura para la estación Tibaitatá automática para el período de tiempo comprendido entre el 1 de enero de 2005 y el 31 de diciembre del 2017. La línea negra horizontal corresponde a la temperatura mínima registrada en esta estación, que es -4.7 \celc.}
    \label{fig:tmp_autom_tibaitata_1}
\end{figure}

La mayor cantidad de heladas en la zona Andina coincide con la primera época seca del año, entre los meses de diciembre y febrero \citep{IDEAM2012a}. Se seleccionó este caso porque el 4 de febrero de 2007 se presentó una de las temperaturas más bajas registradas para la Sabana de Bogotá. Un par de días antes de esta fecha, el 1 y 2 de febrero del 2007, se presentaron temperaturas de -4.7\celc \ en los municipios de Mosquera, Sopó, Tenjo, Subachoque, Funza, Sesquilé y Nemocón. Este evento afectó cerca de 7500 hectáreas de papa \citep{ElTiempo2007}. En la Tabla \ref{table:caso1} se evidencian los registros de las temperaturas máximas y mínimas de algunas estaciones convencionales para el 4 de febrero y el valor reportado por la estación automática Tibaitatá. Según la estación automática Tibaitatá, la helada tuvo una duración de 5 horas comenzando a las 2 a.m. y finalizando a las 7 a.m. (Figura \ref{fig:tmp_autom_tibaitata}). Adicionalmente, en el mismo día se presentó un evento con temperaturas sobre 20\celc\ que duró 5 horas, iniciando a las 11 a.m. y finalizando a las 4 p.m. El valor más alto registrado para esta estación en el período 31 de enero a 4 de febrero de 2007 fue de 26.3\celc\ (ver Figura \ref{gra:tmp_tiba_2007_tibaitata}).

\begin{table}[H]
\centering
\caption{Temperaturas máximas diarias y mínimas diarias de las estaciones convencionales que registraron temperatura sobre 20\celsius\ o debajo de 0\celsius\ o ambas y temperaturas máxima diaria y mínima diaria registradas para la estación automática Tibaitatá, para el caso 1 correspondiente al día 4 de febrero del 2007.}
\begin{tabular}{p{2cm}p{2cm}lll}
Temperatura bajo 0\celc & Temperaturas sobre 20\celc\ & Código   & Nombre de la estación & Municipio \\ \hline


-7.0    &       & 21205880 & Flores Chibcha        & Madrid    \\
 & 21.6 &  21205700 &     Guasca &       Guasca \\
 -7.4 & & 21205920 & SUASUQUE              & Sopó      \\
 & 22.0 &  21205790 &        Apto El Dorado &  Bogotá D.C. \\
 & 22.2 &  21206230 &         Vegas Las Hda &  Bogotá D.C. \\
 & 22.5 &  21205770 &     Base Aérea Madrid &       Madrid \\
-8.8 & 23.8 &  21205980 &       Providencia Gja &        Tenjo \\
 & 23.8 &  21206260 &     C. Univ .Agrop-UDCA &  Bogotá D.C. \\
 & 24.0 &  21206210 &    Flores Colombianas &        Funza \\
 & 24.2 &  21206660 &    Col Santiago Pérez &  Bogotá D.C. \\
-4.6 & 24.8 &  21205420 &  Tibaitatá &     Mosquera \\
-4.7 & 26.3 &  21206990 &  Tibaitatá [Automática]&     Mosquera \\
\end{tabular}

\label{table:caso1}
\end{table}

\begin{figure}[H]
    \centering
    \includegraphics[draft=false, scale=0.5]{casos_altos_bajas/grafica_minimas_tmp_tibaitata_helada.png}
    \caption{Valores de temperatura para la estación Tibaitatá automática para las fechas entre el 3 y el 5 de febrero de 2007. Las líneas verticales discontinuas corresponden a la temperatura de 0\celc\ en las horas 2 y 7 am del 4 de febrero del 2007. Las líneas verticales de puntos corresponden a una temperatura de 20\celc\ en las horas 11 am y las 4 pm.}
    \label{fig:tmp_autom_tibaitata}
\end{figure}

En la Figura \ref{gra:tmp_tiba_2007} podemos observar que entre los días 31 de enero de 2007 y 5 de febrero de 2007, en las estaciones Hda Santa Ana (Figura \ref{gra:tmp_tiba_2007_sta_ana}), Pmo Guerrero (Figura \ref{gra:tmp_tiba_2007_Pmo_guerrero}) y Tibaitatá Automática (Figura \ref{gra:tmp_tiba_2007_tibaitata}) se presentaron temperaturas por debajo de 0\celc\ y para las estaciones Tibaitatá Automática (Figura \ref{gra:tmp_tiba_2007_tibaitata}), Chinavita (Figura \ref{gra:tmp_tiba_2007_chinavita}) y La Capilla Automática (Figura \ref{gra:tmp_tiba_2007_capilla}) se presentaron temperaturas sobre 25\celc. En el período de estudio sólo en la estación Tibaitatá se presentaron tanto altas como bajas temperaturas. Se presentaron tres días con bajas temperaturas y cuatro días con altas temperaturas.\\

Algunas estaciones presentaron altas temperaturas pero sobre un umbral más bajos sobre 20\celsius\, y no presentaron temperaturas por debajo de 0\celsius, como las estaciones Estación Subia Automática (ura \ref{gra:tmp_tiba_2007_subia}), Estación Ciudad Bolivar (Figura \ref{gra:tmp_tiba_2007_cuidad_bolivar}) y Estación La Boyera Automática (Figura \ref{gra:tmp_tiba_2007_la_boyera}). Las altas temperaturas se encuentran asociadas a cielos despejados.\\

\begin{figure}[H]

    \centering


\begin{subfigure}[normla]{0.4\textwidth}
\subcaption{Estación Subia Automática}
\includegraphics[draft=false, scale=0.25]{grafica_altas_bajas/21195160.png}
\label{gra:tmp_tiba_2007_subia}
\end{subfigure}
~
\begin{subfigure}[normla]{0.4\textwidth}
\caption{Estación Hda Sta Ana Automática}
\includegraphics[draft=false, scale=0.25]{grafica_altas_bajas/21206790.png}
\label{gra:tmp_tiba_2007_sta_ana}
\end{subfigure}
~
\begin{subfigure}[normla]{0.4\textwidth}
\caption{Estación Pmo Guerrero}
\includegraphics[draft=false, scale=0.25]{grafica_altas_bajas/21206930.png}
\label{gra:tmp_tiba_2007_Pmo_guerrero}
\end{subfigure}
~
\begin{subfigure}[normla]{0.4\textwidth}
\caption{Estación Ciudad Bolivar}
\includegraphics[draft=false, scale=0.25]{grafica_altas_bajas/21206940.png}
\label{gra:tmp_tiba_2007_cuidad_bolivar}
\end{subfigure}
~
\begin{subfigure}[normla]{0.4\textwidth}
\caption{Estación Pmo Guacheneque}
\includegraphics[draft=false, scale=0.25]{grafica_altas_bajas/21206950.png}
\label{gra:tmp_tiba_2007_pmo_guacheneque}
\end{subfigure}
~
\begin{subfigure}[normla]{0.4\textwidth}
\caption{Estación Sta Cruz De Siecha}
\includegraphics[draft=false, scale=0.25]{grafica_altas_bajas/21206980.png}
\label{gra:tmp_tiba_2007_cruz_siecha}
\end{subfigure}
~
\end{figure}
           
\begin{figure}[H]\ContinuedFloat
\centering
\begin{subfigure}[normla]{0.4\textwidth}
\caption{Estación Tibaitatá Automática}
\includegraphics[draft=false, scale=0.25]{grafica_altas_bajas/21206990.png}
\label{gra:tmp_tiba_2007_tibaitata}
\end{subfigure}
~
\begin{subfigure}[normla]{0.4\textwidth}
\caption{Estación La Boyera Automática}
\includegraphics[draft=false, scale=0.25]{grafica_altas_bajas/24015110.png}
\label{gra:tmp_tiba_2007_la_boyera}
\end{subfigure}
~
\begin{subfigure}[normla]{0.4\textwidth}
\caption{Estación Chinavita Automática}
\includegraphics[draft=false, scale=0.25]{grafica_altas_bajas/35075070.png}
\label{gra:tmp_tiba_2007_chinavita}
\end{subfigure}
~
\begin{subfigure}[normla]{0.4\textwidth}
\caption{Estación La Capilla Automática}
\includegraphics[draft=false, scale=0.25]{grafica_altas_bajas/35085080.png}
\label{gra:tmp_tiba_2007_capilla}
\end{subfigure}
~
    
    
    \caption{Temperatura registrada por diferentes estaciones automáticas entre los días 31 de enero de 2007 y 5 de febrero de 2007, hora local. La línea horizontal azul corresponde a 0\celc, la línea roja más oscura corresponde a la temperatura a 20\celc y la línea roja clara corresponde a 25\celc.}
    \label{gra:tmp_tiba_2007}
\end{figure}



Por otro lado hubo estaciones que no presentaron temperaturas extremas, como las estaciones Estación Pmo Guacheneque (Figura \ref{gra:tmp_tiba_2007_pmo_guacheneque}) y Estación Sta Cruz De Siecha (Figura \ref{gra:tmp_tiba_2007_cruz_siecha}). En estas estaciones es probable que las condiciones meteorológicas no fueran favorables para que se presente los fenómenos de altas y bajas temperaturas.\\

En la Tabla \ref{table:caso1} se evidencia que de las 162 estaciones convencionales, hay más estaciones (11) con valores sobre 20\celc\ que estaciones con valores de temperatura por debajo de 0\celc (4). Para el caso de la estación convencional Tibaitatá se presentaron temperaturas mínimas de -4.6\celc\ y en el mismo día presentaron valores de temperatura de 24.8\celc, esto implica un cambio de temperatura de 29\celc\ en un mismo día. Los cambios extremos de temperatura afectan a los cultivos ya que pueden ocasionar esterilidad, abortos en granos cuajados y disminución en los períodos de llenado como lo señala \citet{Hatfield2015}.\\

Las estación Tibaitatá automáticas y Tibaitatá convencional, se encuentran ubicadas a escasos metros una de la otra. Como se observa la diferencia entre las temperaturas mínimas es baja comparada con la deferencia entre las temperaturas máximas. Esto es explicado en los artículos de \citet{lacombe2010results, kaspar2016climate} y nombrado anteriormente, (sección \ref{convencional_vs_automática}).

\section{Corrección por altura}

Se realizó una corrección de los valores de temperatura simulada por el modelo WRF basado en la altura real de las estaciones del IDEAM. Cada estación del IDEAM tiene un valor asociado de altura (Ver Tabla \ref{tab:alturas_ideam_alos_2} columna $ Altura\ IDEAM$) , pero algunos de estos no concuerdan con la realidad. Un ejemplo de esto es el caso de la estación llamada Pmo (Páramo) Guacheneque. El IDEAM reporta una altura de 2300 msnm pero según el modelo digital de elevación (DEM) ALOS-PALSAR creado por \citet{ASFDAAC2007}, la ubicación de la estación tiene una elevación de 3288 msnm. Por otra parte, según el Distrito Regional de Manejo Integrado generado por \citet{Corpochivor2011}, el área de este páramo está entre los 3000 y los 3500 msnm. De modo que el modelo de ALOS-PALSAR presenta una mejor aproximación a la altura de la zona donde se encuentra la estación, ver Tabla \ref{tab:alturas_ideam_alos_2}. Razón por la cual no se tomaron los valores de altura sugeridos por el IDEAM; en cambio se usó la altura brindada por el DEM de ALOS-PALSAR para cada una de las ubicaciones de las estaciones, estos valores de altura serán usados como la altura de referencia para las estaciones. Usando la ubicación de cada una de las estaciones se realizó una extracción de los valores de altura presentes en los archivos netCDF generados por el modelo WRF. La forma como se realiza la extracción de los valores de altura se encuentra en el código \textit{ext\_simulaciones\_200702.py} que fue creado a partir de la información del siguiente enlace \textcolor{blue}{\href{http://www.openwfm.org/wiki/How_to_interpret_WRF_variables}{link}}, donde se explica que la altura del terreno se calcula a partir de las alturas geopotenciales, la cual toma los valores de los archivos NetCDF generados por el WRF.\\

\begin{table}[H]
\centering
\caption{Alturas reportadas para las estaciones automáticas según el IDEAM y según el DEM de ALOS-PALSAR.}
\label{tab:alturas_ideam_alos_2}
\begin{tabular}{llrr}
\toprule
{} &                    Nombre &  Altura IDEAM (m) &  Altura ALOS-PALSAR (m) \\
\midrule
1    &         Subia Automatica  &      2075 &     2080 \\
2  &   Esc La Union Automatica &      3320 &     3393 \\
3  &     Pasquilla Automatica  &      3000 &     2982 \\
4  &   Pluviometro Automatico  &      2685 &     2632 \\
5  &       Pluviometro Automa  &      2685 &     2632 \\
6  &            Univ Nacional  &      2556 &     2579 \\
7  &           Apto El Dorado  &      2547 &     2567 \\
8  &         Nueva Generacion  &      2590 &     2574 \\
9  &   San Joaquin Automatica  &       757 &      679 \\
10 &        Hda Sta Ana Autom  &      2572 &     2590 \\
11 &  Villa Teresa Automatica  &      3624 &     3423 \\
12 &             Pmo Guerrero  &      3257 &     3253 \\
13 &           Ciudad Bolivar  &      2687 &     2837 \\
14 &          Pmo Guacheneque  &      2300 &     3288 \\
15 &             Ideam Bogota  &      2646 &     2679 \\
16 &       Sta Cruz De Siecha  &      3100 &     3194 \\
17 &     Tibaitata Automatica  &      2543 &     2560 \\
18 &        Sta Rosita Autom   &      2618 &     2887 \\
19 &      San Cayetano Autom   &      2807 &     3103 \\
20 &     La Boyera Automatica  &      2610 &     2640 \\
21 &           El Alambrado    &       151 &     3240 \\
22 &      Pnn Chingaza Autom   &      3205 &     3205 \\
23 &     Bosque Intervenido    &      2944 &     2919 \\
24 &           Calostros Bajo  &      2943 &     3421 \\
25 &          Plaza De Ferias  &      1670 &     1677 \\
26 &      Parque Rafael Nuñez  &      1671 &     1686 \\
27 &         Calostros Bajo    &      2943 &     3421 \\
28 &             Pmo Chingaza  &      3863 &     3856 \\
29 &     Chinavita Automatica  &      2012 &     1938 \\
30 &  Pmo Rabanal Automatica   &      3398 &     3379 \\
31 &         La Capilla Autom  &      1917 &     1903 \\
\bottomrule
\end{tabular}
\end{table}



Los valores modelados fueron corregidos teniendo en cuenta la diferencia de altura ($\Delta A$) entre la altura de los datos modelados provenientes de  las salidas del modelo WRF ($A_{wrf}$) y la altura real ($A_{real}$), como lo sugiere \citet{Pabon-Caicedo2001}. Se calcula la diferencia entre los resultados del modelo y la altura real; se multiplica por un factor de 6.5\celc\ por cada 1000 metros (ver Ecuación \ref{correccion_altura}). El resultado de esta operación se le suma a los valores de temperatura modelados. Las tablas de las correcciones de las alturas se encuentran en el Anexo \ref{anexo:correccion_altura}.

\begin{equation} \label{correccion_altura}
\Delta A = (A_{wrf} - A_{real}) \times 0.0065 \frac{^{\circ}C}{m}
\end{equation}
 

Para la comparación de los resultados obtenidos con WRF y los valores reales de las estaciones automáticas, se ajustó la metodología usada en el diagrama de Taylor. El diagrama de Taylor \citep{taylor2001summarizing} fue creado para la comparación de diferentes modelos basado en la comparación de los datos reales con los datos modelados. El diagrama de Taylor usa tres estadísticos, correlación de Pearson (Ecuación \ref{pearson_coef}), error cuadrado medio (Ecuación \ref{rmse_eq_1}) y desviación estándar (Ecuación \ref{sd_eq}).\\

\begin{equation}\label{pearson_coef}
\rho =\frac{\mathlarger{\sum_{i=1}^n} (x_i-\bar{x}) (x'_i-\bar{x'})}{\mathlarger{\sqrt{\sum_{i=1}^n (x_i-\bar{x})^2 (x'_i-\bar{x'})^2}}}
\end{equation}

\begin{equation}\label{rmse_eq_1}
RMSE =\frac{\sqrt{\sum_{i=1}^n (x_i-x'_i)^2}}{n}
\end{equation}

\begin{equation}\label{sd_eq}
sd_{x'} =\mathlarger{\sqrt{\frac{\mathlarger{\sum_{i=1}^n} (x'_i-\bar{x'}_i)^2}{n-1}}}
\end{equation}

En el diagrama de Taylor el coeficiente de Pearson está asociado al ángulo azimutal, este valor está entre -1 y 1, representado en rojo en la Figura \ref{fig_explicacion_taylor}. El error cuadrado medio está representado como la distancia entre los modelos a evaluar y la referencia que se encuentra simbolizada por una estrella, los valores van desde 0 hasta infinito, se encuentra representado por una línea concéntrica gris cuyo centro es la estrella roja. La desviación estándar que es proporcional a la distancia radial desde el origen, se encuentra representado por líneas azules, estos valores pueden ir desde cero hasta infinito \citep{taylor_diag_2018}.\\

\begin{figure}[h]
    \centering
	\label{fig_explicacion_taylor}
	\includegraphics[draft=false, scale=0.45]{graph/explicacion_taylor.png}
	\caption{Ejemplo de un diagrama de Taylor.}
\end{figure}

La metodología propuesta por \citet{taylor2001summarizing} es una buena forma para escoger un modelo, sin embargo, por ser un método gráfico deja de ser útil cuando la cantidad de modelos es grande o cuando la diferencia entre modelos es pequeña. Por esta razón para la selección de los mejores resultados se usará la correlación de Pearson y una versión del cuadrado medio del error, llamado error cuadrado medio normalizado (NRMSE) (Ecuación \ref{rmse_eq}), ya que este estadístico permite realizar la comparación de grupos de datos o modelos con diferentes escalas.\\

\begin{equation}\label{rmse_eq}
NRMSE =\frac{\sqrt{\frac{\sum_{i=1}^n (x_i-x'_i)^2}{n}}}{x'_{i(max)} - x'_{i(min)}}
\end{equation}

En la Figura \ref{gra:taylor_total_a} se muestran los diagramas de Taylor para las temperaturas simuladas para las ubicaciones correspondientes a las estaciones Sta Cruz De Siecha y Hda Sta Ana Automática, los demás diagramas se encuentran en el Anexo \ref{anexo:diag_taylor_estaciones_aut_val}. En la Figura \ref{subfig:taylor_11} se puede apreciar dispersión de las diferentes simulaciones evaluadas, pero es complejo cual de estos tiene el mejor comportamiento y en la Figura \ref{subfig:taylor_22} podemos observar que sólo se puede observar un punto en el cuál todos las simulaciones convergen y se hace imposible determinar cuál de éstas presenta la mejor opción. A partir de las Figuras \ref{subfig:taylor_11} y \ref{subfig:taylor_22} se concluye que es necesario usar una metodología que facilite la escogencia de los mejores resultados.\\




%\begin{equation}\label{sd_eq}
%sd_x =\mathlarger{\sqrt{\frac{\mathlarger{\sum_{i=1}^n} (x_i-\bar{x}_i)^2}{n-1}}}
%\end{equation}
%
%\begin{equation} \label{std_abs}
%    {STD}_{abs} = |sd_{x} - sd_{x'}|
%\end{equation}

Donde $n$ es el número de datos, $x$ corresponde a los registros de las estaciones automáticas, $x'$ corresponde a los datos modelados.\\


\begin{figure}[H]
	\begin{center}
	\begin{subfigure}[normla]{0.4\textwidth}
	\caption{Estación Sta Cruz De Siecha código 21206980.}
	\includegraphics[draft=false, scale=0.45]{taylor/taylor_21206980.png}
	\label{subfig:taylor_11}
	\end{subfigure}
		~
    \begin{subfigure}[normla]{0.4\textwidth}
    \caption{Estación Hda Sta Ana Automática código 21206790.}
	\includegraphics[draft=false, scale=0.45]{taylor/taylor_21206790.png}
	\label{subfig:taylor_22}
	\end{subfigure}
		~
			\end{center}
	\caption{Diagramas de Taylor que comparan la temperatura observada en 2 estaciones con los datos provenientes de los datos simulados en cada dominio o subdominio de las nueve simulaciones hechas con el modelo WRF. El período simulado fue entre el 3 de febrero de 2007 a las 18 UTC hasta el 5 de febrero de 2007 a las 00 UTC.}
	\label{gra:taylor_total_a}	
\end{figure}


%\subsection{Cuantificación de las }
%Se realizaron tres regresiones lineares donde el objetivo fue escalar todos los valores obtenidos de uno a cero, siendo uno los mejores resultados y cero los peores resultados. 

%Los valores del coeficiente de correlación de Pearson nunca dieron negativos en ninguna combinación.\\

%Las regresiones se realizaron usando los valores máximos y mínimos del coeficiente de Pearson. Se halló el valor mínimo del coeficiente de correlación de Pearson y a este valor 

%Por esta razón se propone hacer uso de regresiones lineares para poder escalar todos los índices y realizar un solo índice que permita una fácil comparación entre todos los resultados. Para este fin se tomaron todos los valores de Pearson generados para cada una de las estaciones se buscó el menor valor y el mayor valor a estos valores se les realizó una regresión linear entre cero y uno, respectivamente. Para el error medio cuadrático se buscaron los valores más altos y los más bajos y se ajustaron a una regresión linear entre 0 y 1, respectivamente. Para las desviaciones estándar se sacó el valor absoluto de la diferencia entre la desviación estándar de la referencia menos la desviación estándar de las simulaciones, para estos valores se buscó el mayor valor y el menor valor y se realizó una desviación estándar entre 0 y 1, respectivamente. Como se puede observar se busca que los mejores valores correspondan a los valores más altos, cercanos a 1 y las simulaciones que menos similitud presentan corresponderán a valores bajos, cercanos a 0. Posteriormente estos valores se suman y se dividen entre 3.

%Para determinar cuál es el mejor de los resultados se realizó lo siguiente


\section{Análisis de las condiciones meteorológicas del 4 de febrero de 2007}

De acuerdo con el análisis de superficie (Figura \ref{fig:carta1}) se puede observar que el día 31 de enero de 2007 sobre Colombia no había ningún sistema de alta o baja presión, la zona de convergencia intertropical (ZCIT) se hallaba a la altura de las costas del pacífico colombiano (latitud: 4\degree N), adicionalmente podemos observar que la nubosidad se ubicó en el sur de Colombia. El día 2 de febrero no se observa desplazamiento de la ZCIT de las costas del pacífico colombiano (latitud: 4\degree N), no se ve ningún sistema de presión sobre Colombia, pero sí se puede ver que la nubosidad se encuentra al sur del país (Figura \ref{fig:carta2}). El día 3 de febrero se observa que la ZCIT se ha movido en dirección sur ya que se encuentra a la altura de Ecuador, adicionalmente se observa una acumulación de nubosidad en la parte sur del país (Figura \ref{fig:carta3}). Y el día 4 de febrero se presentó un sistema de baja presión sobre Colombia y la nubosidad continuó presentándose sobre la parte sur del país (Figura \ref{fig:carta4}).\\

Basado en este análisis podemos concluir que para las fechas de estudio no hubo cobertura nubosa en el centro del país, donde se encuentra la zona de estudio, puesto que la nubosidad se concentraba en el sur del país, lo que implica que en estas fechas se tuvieron cielos despejados sobre la Sabana de Bogotá.

\begin{figure}[H]
	\centering
		\begin{subfigure}[normla]{0.9\textwidth}
		\caption{Análisis de superficie para el día 31 de enero de 2007, hora 02:46 UTC.}
	\includegraphics[draft=false, scale=0.4]{cartas/2007/07013102QPAA99.png}
		\label{fig:carta1}
		\end{subfigure}
\end{figure}
           
\begin{figure}[H]\ContinuedFloat
		\centering
				\begin{subfigure}[normla]{0.9\textwidth}
		\caption{Análisis de superficie para el día 2 de febrero de 2007, hora 02:47 UTC.}
	\includegraphics[draft=false, scale=0.4]{cartas/2007/07020202QPAA99.png}
		\label{fig:carta2}
		\end{subfigure}

    \centering
    \begin{subfigure}[normla]{0.9\textwidth}
		\caption{Análisis de superficie para el día 3 de febrero de 2007, hora 02:56 UTC.}
	\includegraphics[draft=false, scale=0.4]{cartas/2007/07020303QPAA99.png}
		\label{fig:carta3}
		\end{subfigure}
\end{figure}
           
\begin{figure}[H]\ContinuedFloat
\centering
			\begin{subfigure}[normla]{0.9\textwidth}
		\caption{Análisis de superficie para el día 4 de febrero de 2007, hora 02:38 UTC.}
	\includegraphics[draft=false, scale=0.4]{cartas/2007/07020402QPAA99.png}
		\label{fig:carta4}
		\end{subfigure}
		
	
	\caption{Análisis de superficie para el 31 de enero (a), 2 (b), 3 (b) y 4 (c) de febrero de 2007 .}
	\label{fig:carta_total}	
\end{figure}

En las imágenes satelitales del GOES 12 del 4 de febrero de 2007 (Figura \ref{fig:goes_total}) se puede observar la baja presencia de nubes para este día. En la imagen infrarroja se observa de nuevo la situación vista anteriormente en el análisis de superficie, ya que se puede ver que en el centro del país no hay cobertura nubosa y se presenta una alta cobertura en la zona sur del país. Las temperaturas extremas están asociadas con noches y mañanas despejadas, como se puede observar en las imágenes.

\begin{figure}[H]
    \begin{subfigure}[normla]{0.5\textwidth}
\caption{Radiación infrarroja de onda corta.}
\includegraphics[draft=false,     scale=0.23]{{goes/2007/goes12.2007.034.001515.BAND_02}.jpg}
\label{fig:goes2}
\end{subfigure}
~
\begin{subfigure}[normla]{0.5\textwidth}
\caption{Vapor de agua}
\includegraphics[draft=false,     scale=0.23]{{goes/2007/goes12.2007.034.001515.BAND_03}.jpg}
\label{fig:goes3}
\end{subfigure}
    ~
\centering
\begin{subfigure}[normla]{0.3\textwidth}
\caption{Radiación infrarroja de onda larga.}
\includegraphics[draft=false,         scale=0.21]{{goes/2007/goes12.2007.035.034515.BAND_04}.jpg}
\label{fig:goes4}
\end{subfigure}
    
    	
    \caption{Imágenes del GOES 12: canal 2 (a), canal 3 (b) y canal 4(c). Para el día 4 de febrero de 2007 a las 00 UTC (3 de febrero de 2007 a las 19:00 HL).}
    \label{fig:goes_total}	
\end{figure}

Por otro lado, los radiosondeos para estas fechas muestran una gran inversión térmica, desde el nivel de superficie 750 hPa hasta los 500 hPa (Figura \ref{fig:radiosondeo}). La temperatura a los 753 hPa fue -0.7 con una humedad relativa de 96\%, y a mayor altura (500 hPa) se evidencia una gran disminución en la humedad la cuál fue del 2\%. La inversión térmica que se presenta impide la mezcla de la masa de aire con una de mayor temperatura, esto favorece que la baja temperatura se mantenga.\\



\begin{figure}[H]
    \centering
    \includegraphics[draft=false, scale=0.5]{radiosondeos/radio_sondeo_20070204.png}
    \caption{Radiosondeo del 4 de febrero de 2007. a las 7 am HL. Tomado de la \textcolor{blue}{\href{http://weather.uwyo.edu/cgi-bin/sounding?region=samer&TYPE=GIF\%3ASKEWT&YEAR=2007&MONTH=02&FROM=0412&TO=0412&STNM=80222}{Universidad de Wyoming.}}} 
    \label{fig:radiosondeo}
\end{figure}

En este estudio se usaron los datos del modelo GFS, por esta razón se quiso evaluar cuáles fueron las condiciones meteorológicas que presenta el modelo para la fecha de interés. Por esta razón se realizó una gráfica que incluye las variables presión atmosférica (Variable llamada \textit{Mean sea level pressure} dentro de los archivos grib de GFS) y componentes del viento en U y V a 10 metros extraídas de análisis del modelo GFS (Variables llamadas \textit{10 metre U wind component} y \textit{10 metre U wind component}, respectivamente, dentro de los archivos grib de GFS) (Figura \ref{fig:gfs_noaa}). En esta gráfica podemos ver cómo los vientos confluyen sobre la parte más oriental del continente suramericano formando la zona de convergencia intertropical (ZCIT), adicionalmente se observan dos sistemas de alta y baja presión entre las latitudes 30\degree N y 40\degree N y las longitudes 50\degree W y la latitud 20\degree W, los cuales coinciden con lo visto en la Figura \ref{fig:goes4}. Por lo tanto, podemos concluir que, a nivel sinóptico, el modelo GFS está representando las condiciones de cielo despejado para el centro del país, donde se encuentra el área de estudio para estas fechas.

\begin{figure}
    
    \centering
    \includegraphics[draft=false, scale=0.4]{graph/gfs_noaa.png}
    \caption{Carta de superficie generada a partír de los datos del GFS. Para el día 4 de febrero de 2007 00 UTC.}
    \label{fig:gfs_noaa}
\end{figure}


\section{Períodos de registro de las estaciones meteorológicas automáticas.}

Se seleccionaron todas las estaciones meteorológicas automáticas que se encontraban dentro del área de estudio (área marcada con rojo en los mapas mostrados en la Figura \ref{gra:areas}). Las estaciones meteorológicas automáticas tienen diferentes períodos de registro para la variable temperatura. Las fechas de inicio y finalización de registro para las estaciones automáticas seleccionadas para esta parte del estudio se encuentran en la Tabla \ref{tab:inicio_final_hydras}, en algunos casos la fecha de estudio no posee registro para alguna de las estaciones, por esta razón no se van a tener en cuenta ciertas estaciones.

\begin{table}[H]
\centering
\caption{Fechas de inicio y finalización de los registros de las 31 estaciones automáticas de la red HYDRAS usadas. Las estaciones que no tienen datos de temperatura para ningún período tienen un NaT.}
\label{tab:inicio_final_hydras}
\begin{tabular}{lrlll}
\toprule
{} &       Código &                   Nombre &              Inicio &                 Finalización \\
\midrule
1  &  35025080 &       PNN CHingaza Autom & 1996-01-24 08:00:00 & 2018-02-22 00:00:00 \\
2  &  35075080 &   Pmo Rabanal Automática & 1998-06-02 03:32:00 & 2018-02-22 00:00:00 \\
3  &  35085080 &         La Capilla Autom & 2000-01-01 00:00:16 & 2018-02-21 23:08:00 \\
4  &  21206990 &     Tibaitatá Automática & 2000-01-01 00:00:22 & 2018-02-22 00:00:00 \\
5  &  21206940 &           Ciudad Bolivar & 2000-01-01 17:12:52 & 2014-07-22 03:02:00 \\
6  &  35035130 &             Pmo Chingaza & 2000-01-01 23:37:48 & 2018-02-22 00:00:00 \\
7  &  21206920 &  Villa Teresa Automática & 2000-01-20 17:22:04 & 2018-02-22 00:00:00 \\
8  &  21205012 &            Univ Nacional & 2003-05-20 17:42:00 & 2018-02-21 23:05:00 \\
9  &  21206930 &             Pmo Guerrero & 2004-04-02 10:44:52 & 2018-02-21 23:09:00 \\
10 &  21206790 &        Hda Sta Ana Autom & 2005-02-08 02:00:00 & 2018-02-21 23:02:00 \\
11 &  23125170 &       San Cayetano Autom & 2005-02-09 01:07:00 & 2018-02-22 00:00:00 \\
12 &  35075070 &     Chinativa Automática & 2005-02-13 09:00:00 & 2018-02-22 00:00:00 \\
13 &  35027510 &           Calostros Bajo & 2005-02-20 23:28:00 & 2018-02-22 00:00:00 \\
14 &  21206980 &       Sta Cruz De Siecha & 2005-04-21 17:59:00 & 2018-02-22 00:00:00 \\
15 &  21206950 &          Pmo Guacheneque & 2005-06-21 12:00:00 & 2018-02-22 00:00:00 \\
16 &  24015110 &     La Boyera Automática & 2005-06-28 12:00:00 & 2018-02-22 00:00:00 \\
17 &  21195160 &         Súbia Automática & 2006-07-30 12:01:00 & 2012-06-12 19:43:00 \\
18 &  21206960 &             IDEAM Bogotá & 2008-06-15 10:00:00 & 2018-02-21 23:26:00 \\
19 &  35025090 &       Bosque Intervenido & 2009-04-28 03:00:00 & 2018-02-22 00:00:00 \\
20 &  21206600 &         Nueva Generación & 2010-01-06 00:30:00 & 2018-02-21 23:01:00 \\
21 &  21201200 &  Esc La Unión Automática & 2010-06-11 15:30:00 & 2016-08-01 13:20:00 \\
22 &  21205791 &           Apto El Dorado & 2014-08-29 10:03:00 & 2018-02-21 23:06:00 \\
23 &  21201580 &     Pasquilla Automática &                 NaT &                 NaT \\
24 &  21202270 &   Pluviómetro Automático &                 NaT &                 NaT \\
25 &  21202271 &       Pluviómetro Automática&                 NaT &                 NaT \\
26 &  21206710 &   San Joaquin Automática &                 NaT &                 NaT \\
27 &  21209920 &         Sta Rosita Automática &                 NaT &                 NaT \\
28 &  26127010 &             El Alambrado &                 NaT &                 NaT \\
29 &  35025100 &           Calostros Bajo &                 NaT &                 NaT \\
30 &  35027001 &          Plaza De Ferias &                 NaT &                 NaT \\
31 &  35027002 &      Parque Rafael Núñez &                 NaT &                 NaT \\

\bottomrule
\end{tabular}
\end{table}

Para evaluar las simulaciones que definirán la configuración dominio-subdominio del modelo WRF, se va a tener en cuenta el período de estudio comprendido entre el 31 de enero y el 5 de febrero de 2007. En la Tabla \ref{tab:inicio_final_hydras} se evidencia que 9 estaciones no tienen datos de temperatura válidos y 5 estaciones tienen un inicio de registro de datos después de la fecha de interés. Por estas dos razones se descarta el uso de 14 estaciones.\\

%Es importante notar que para esta parte del estudio, además de que algunas de las estaciones seleccionadas no tienen los datos de temperatura correspondientes al período de estudio también hay otras estaciones con registros cuyos valores no pasaron satisfactoriamente el proceso de validación, ver Tabla \ref{tab:inicio_final_hydras}.\\

Entre el 31 de enero y el 4 de febrero de 2007, se presentaron heladas. Por ejemplo, en la Estación Tibaitatá Automática se presentaron valores de temperatura por debajo de 0\celc. También se presentaron temperaturas sobre los 25\celc\ para el período de analizado, ver Figura \ref{graph:tiba}. Las demás series de tiempo pertenecientes a diferentes estaciones con datos válidos se pueden observar en el Anexo \ref{anexo:series_tiempo_temperatura}.

\begin{figure}[H]
\centering
\includegraphics[draft=false, scale=0.4]{automaticas_periodos/21206990.png}
\caption{Valores de temperatura para la Estación Tibaitatá Automática entre el 31 de enero y el 05 de febrero de 2007. La línea horizontal interrumpida corresponde a la temperatura de 0\celsius\ y la línea horizontal de puntos representa la temperatura de 20\celsius,}
\label{graph:tiba}
\end{figure}

\begin{table}[H]
\centering
\caption{Estado de las estaciones automáticas para el período entre el 31 de enero y el 05 de febrero de 2007. ``Datos válidos'' corresponden a las estaciones que poseen los suficientes datos para ser comparadas con los resultados de simulaciones hechas con el modelo WRF; ``Datos no válidos'' corresponden a una estación que no posee datos de temperatura válidos para ser comparados, ``Fuera de rango'' corresponde a una estación que no tienen valores de temperatura para el período de estudio.}
\label{tab:estado_hydras}
\begin{tabular}{lrr}
\toprule
{} &       Nombre de la estación & Estado\\
\midrule
1  &       PNN Chingaza Automática  & Datos no válidos\\
2  &   Pmo Rabanal Automática  & Datos no válidos\\
3  &         La Capilla Automática  & Datos válidos\\
4  &     Tibaitatá Automática  & Datos válidos\\
5  &           Ciudad Bolivar  & Datos válidos\\
6  &             Pmo Chingaza  & Datos no válidos\\
7  &  Villa Teresa Automática  & Datos no válidos\\
8  &            Univ Nacional  & Datos no válidos\\
9  &             Pmo Guerrero  & Datos válidos\\
10 &        Hda Sta Ana Automática  & Datos válidos\\
11 &       San Cayetano Automática  & Datos no válidos\\
12 &     Chinavita Automática  & Datos válidos\\
13 &           Calostros Bajo  & Datos no válidos\\
14 &       Sta Cruz De Siecha  & Datos válidos\\
15 &          Pmo Guacheneque  & Datos válidos\\
16 &     La Boyera Automática  & Datos válidos\\
17 &         Subia Automática  & Datos válidos\\
18 &             IDEAM Bogotá  & Fuera de rango\\
19 &       Bosque Intervenido  & Fuera de rango\\
20 &         Nueva Generación  & Fuera de rango\\
21 &  Esc La Unión Automática  & Fuera de rango\\
22 &           Apto El Dorado  & Fuera de rango\\

\bottomrule
\end{tabular}
\end{table}

 Como se puede observar en la Tabla \ref{tab:estado_hydras} sólo 10 estaciones tienen datos que pueden servir para realizar una  comparación con los datos simulados. De modo que para todos los análisis venideros referentes a el período entre el 31 de enero y el 05 de febrero de 2007 se usarán 10 estaciones automáticas, ya que son las que poseen datos válidos.


%\begin{table}[H]
%\centering
%\caption{Tabla de las estaciones con su paisaje asociado, precipitación promedio mensual y la %desviación estándar de la temperatura.}
%\label{tab:condiciones_estaciones}
%\begin{tabular}{lrr}
%\toprule
%id & Nombre & Ubicación & Precipitación promedio anual (mm)& Altura & Desviación estándar\\
%\midrule
%1  &  PNN Chingaza Automática  &    Ladera Montaña       & 1078  &  3205   &      \\
%2  &   Pmo Rabanal Automática  &    Ladera Montaña       & 979   &  3379   &   \\
%3  &    La Capilla Automática  &    Ladera Pendiente     & 776   &  1903   &   6\\
%4  &     Tibaitatá Automática  &    Valle                & 189   &  2560   &   12.5\\
%5  &           Ciudad Bolivar  &    Cima de una montaña  & 534   &  2837   &   6.8\\
%6  &             Pmo Chingaza  &    Cima de una montaña  & 2342  &  3863   &      \\
%7  &  Villa Teresa Automática  &    Ladera               & 1023  &  3423   &      \\
%8  &            Univ Nacional  &    Valle                & 635   &  2579   &   \\
%9  &             Pmo Guerrero  &    Ladera               & 828   &  3253   &   7\\
%10 &   Hda Sta Ana Automática  &    Valle                & 427   &  2590   &   12\\
%11 &  San Cayetano Automática  &    Ladera               & 783   &  3103   &   \\
%12 &     Chinavita Automática  &    Ladera               & 1332  &  1938   & 7.5     \\
%13 &           Calostros Bajo  &    Ladera               & 1569  &  3421   &      \\
%14 &       Sta Cruz De Siecha  &    Valle                & 1207  &  3194   &   4.8   \\
%15 &          Pmo Guacheneque  &    Valle                & 995   &  3288   &   3.2\\
%16 &     La Boyera Automática  &    Valle al lado del río& 781   &  2640   &   9\\
%17 &         Subia Automática  &    Valle                & 932   &  2080   &   5.7\\
%18 &             IDEAM Bogotá  &    Valle                & 618   &  2679   &   \\
%19 &       Bosque Intervenido  &    Ladera Montaña       & 737   &  2919   &   \\
%20 &         Nueva Generación  &    Valle                & 1784  &  2574   &      \\
%21 &  Esc La Unión Automática  &    Valle intermontano   & 709   &  3393   &   \\
%22 &           Apto El Dorado  &    Valle                & 543   &  2567   &   \\
%23 &     Pasquilla Automática  &    Valle                & 300   &  2982   &   \\
%24 &   Pluviómetro Automático  &    Valle                & NaN   &  2632   &   \\
%25 &    Pluviómetro Automática &    Valle                & NaN   &  2632   &   \\
%26 &   San Joaquin Automática  &    Valle                & NaN   &   679   &   \\
%27 &    Sta Rosita Automática  &    Al lado de laguna    & NaN   &  2887   &   \\
%28 &             El Alambrado  &    Ladera Montaña       & NaN   &  3240   &   \\
%29 &           Calostros Bajo  &    Ladera Montaña       & 1569  &  3421   &      \\
%30 &          Plaza De Ferias  &    Valle                & NaN   &  1677   &   \\
%31 &      Parque Rafael Núñez  &    Valle                & NaN   &  1686   &   \\
%
%
%\bottomrule
%\end{tabular}
%\end{table}


\section{Búsqueda de la mejor combinación dominio-subdominios de anidamiento.}

Para proveer condiciones iniciales y de frontera al modelo WRF, se utilizaron los datos provenientes del análisis del modelo GFS con una resolución de 0.5$^{\circ}$, que corresponde a una resolución aproximada de 55 km. Los datos poseen 24 niveles de resolución vertical y son producidos cada 6 horas.\\

La ubicación de los dominios es importante ya que se debe evitar al máximo ubicar sus fronteras en la mitad de las montañas \citep{Skamarock2008}, pero debido a la complejidad del relieve colombiano, con tres cordilleras, esta tarea resulta compleja. La Sabana de Bogotá se encuentra en un Valle sobre la Cordillera Oriental, por esta razón algunos de los dominios que se van a tener en cuenta más adelante comprenden desde el valle entre la Cordillera Oriental y la Cordillera Central hasta una parte de la llanura del Orinoco.\\




\subsection{Parametrizaciones usadas}

Los esquemas de parametrización utilizados para la búsqueda de la mejor combinación dominio-subdominio de anidamiento se basaron en la configuración usada por el IDEAM para realizar el pronóstico del tiempo en Colombia, esta configuración ha sido una construcción del IDEAM basado en su experiencia, adicionalmente fue evaluada por \citet{ArmentaPorras2013} en su tesis de maestría. Los esquemas usados en esta configuración son:\\


Esquema de microfísica (mp\_physics) \textit{WRF Single–moment 3–class and 5–class Schemes}, opción 3. Esta parametrización procesa la formación de las gotas y cristales de hielo, su crecimiento y su posterior descenso al final del paso de tiempo. En esta parametrización toda el agua se condesa primero como agua en una nube, con pequeñas gotas. Luego, comienza un proceso llamado autoconversión, este proceso implica la formación de partículas por la agregación de muchas pequeñas gotas de la nube por la acción de nucleación que ejercen las sales. Las nubes pueden persistir por un largo período de tiempo sin precipitar \citep{Ghosh1998}. El esquema utiliza una relación de diagnóstico para la concentración de hielo basada en el contenido de masa de hielo, no en la temperatura \citep{ArmentaPorras2013}.\\

Esquema de capa límite planetaria (bl\_pbl\_physics) \textit{Yonsei University Scheme (YSU)}, opción 1. Esta parametrización actúa en la parte más baja de la atmósfera y su comportamiento está directamente influenciado por el contacto con la superficie planetaria. En esta capa la velocidad del viento, temperatura, humedad y otras variables presentan fluctuaciones rápidas. El esquema YSU es un esquema de primer orden no local, con un término de contragradiente en la ecuación de difusión \citep{Hu2010}.\\

Esquema de cúmulus (cu\_physics) \textit{Kain–Fritsch Scheme}, opción 1. Es un esquema de parametrización que es responsable de los efectos de las nubes convectivas. Es un intento de representar los flujos verticales de las corrientes ascendentes y descendentes no resueltas y los movimientos de compensación fuera de la nube \citep{ArmentaPorras2013}.\\

Esquema de radiación de onda corta (ra\_sw\_physics) \textit{RRTM Shortwave Scheme}, opción 1. Hace referencia a toda la energía que es recibida en el espectro de rango de luz visible, una parte de la radiación ultravioleta y una parte del infrarrojo cercano. Esta parametrización tiene en cuenta la interacción de onda larga con el vapor de agua, ozono, $CO_2$ y las nubes; además tiene en cuenta el albedo de las nubes \citep{Armstrong2000}.\\

Esquema de radiación de onda corta (ra\_lw\_physics) \textit{Dudhia Longwave Scheme}, opción 1. La radiación de onda larga es la energía radiativa emitida desde la tierra y su atmósfera hacia el espacio en forma de energía térmica, el rango de logitud de onda está entre los $4\ \mu m$ y los $100\ \mu m$. El esquema usado es un esquema que tiene en cuenta múltiples bandas y gases traza \citep{ArmentaPorras2013}.\\

Esquema de suelo (sf\_surface\_physics) \textit{1–layer Thermal Diffusion Scheme}, opción 1. Es un esquema que usa la información del esquema de superficie, esquema de radiación y precipitación del esquema de microfísica y el esquema de cúmulos para proveer flujos de calor y humedad en el suelo \citep{Skamarock2008}. Estos flujos tienen interacción con el esquema de capa límite planetaria \citep{shuman1978numerical}. \\

Esquema de superficie (sf\_sfclay\_physics) \textit{1–Revised MM5 Scheme}, opción 1. Es un esquema usado para calcular la velocidad de fricción y los coeficientes de intercambio que permiten el cálculo del calor en la superficie y los flujos de humedad entre el esquema de suelo y el esquema de capa límite planetaria \citep{Skamarock2008}.
\begin{comment}
%%%%% Esto no se ve es sólo como guía

  &physics
  mp_physics               = 3,     3,     3,
  ra_lw_physics            = 1,     1,     1,
  ra_sw_physics            = 1,     1,     1,
  radt                     = 30,   30,    30,
  sf_sfclay_physics        = 1,     1,     1,
  sf_surface_physics       = 1,     1,     1,
  bl_pbl_physics           = 1,     1,     1,
  bldt                     = 0,     0,     0,
  cu_physics               = 1,     1,     0,
  cudt                     = 5,     5,     5,
  isfflx                   = 1,
  ifsnow                   = 0,
  icloud                   = 1,
  surface_input_source     = 1,
  num_soil_layers          = 4,
  num_land_cat             = 24,
  sf_urban_physics         = 0,
  mp_zero_out              = 0,
  maxiens                  = 1,
  maxens                   = 3,
  maxens2                  = 3,
  maxens3                  = 16,
  ensdim                   = 144,
  /
\end{comment}



\subsection{Anidamientos}

Cuando se inicializa un modelo como el WRF se selecciona un dominio parental el cual posee un área y una resolución específica, si se desea obtener una resolución mas fina que la resolución inicial se debe hacer un anidamiento (\textit{nesting}). Un anidamiento consiste en generar subdominios con una resolución más fina, este anidamiento debe estar embebido dentro del dominio parental. El anidamiento permite obtener una alta resolución en áreas específicas \citep{Werner2017}.\\

Si el subdominio se encuentra embebido en el dominio parental, entonces el subdominio será alimentado por las condiciones de frontera del dominio parental. Se debe hacer un anidamiento cuando se desea simular fenómenos localizados, tales como temperaturas extremas en un área determinada. Existen dos tipos de anidamientos, ambos son alimentados por las condiciones de frontera del dominio parental, pero una configuración dentro del WRF (\texttt{feedback = 1}), usada en esta investigación, realiza una retroalimentación desde los datos secundarios hacia los datos del dominio parental \citep{Werner2017}, esto implica que los valores del dominio parental son modificados a partír de los resultados obtenidos en los subdominios; de esta forma mejora los resultados en los dominios parentales.\\

Los subdominios dentro de un dominio parental no se deben sobreponer parcialmente. La distancia mínima a la cuál se debe colocar un subdominio es a una distancia de 4 celdas del dominio parental \citep{Werner2017}. Para cada subdominio se recomienda una resolución tres veces más fina que la del dominio parental, ya que de esta forma resulta más eficiente el proceso de reducción de escala \citep{Werner2017}.\\

Se realizaron nueve simulaciones. La primera simulación corresponde a tres dominios (un dominio parental y dos subdominios; d01, d02 y d03) los cuales fueron evaluados con resoluciones de 36, 12 y 4 km, respectivamente (ver Figura \ref{subfig:dom3}). En la segunda simulación se evaluaron tres dominios (un dominio parental y dos subdominios; d01, d02 y d03), los cuales fueron evaluados con resoluciones de 18, 6 y 2 km, respectivamente, (ver Figura \ref{subfig:dom3}). La tercera simulación se evaluó con dos dominios (un dominio parental y un subdominio; d01, d02) con una resolución de 10 y 3.33 km, respectivamente (ver Figura \ref{subfig:dom2}). En la cuarta simulación se evaluaron dos dominios (un dominio parental y un subdominio; d01, d02) con una resolución de 10 y 3.33 km, respectivamente; en comparación con la tercera simulación éste posee unos límites más amplios como se puede ver en la Figura \ref{subfig:dom3.1}. En la quinta simulación se evaluaron dos dominios (un dominio parental y un subdominio; d01, d02) con una resolución de 12 y 4 km, respectivamente (Figura \ref{subfig:dom3.1}). En la sexta simulación se evaluaron dos dominios (un dominio parental y un subdominio; d01, d02) con una resolución de 15 y 5 km, respectivamente (Figura \ref{subfig:dom3.1}). En la séptima simulación se evaluaron dos dominios (un dominio parental y un subdominio; d01, d02) con una resolución de 18 y 6 km, respectivamente (Figura \ref{subfig:dom3.1}). En la octava simulación se evaluaron tres dominios (un dominio parental y dos subdominios; d01, d02 y d03) con una resolución de 12, 4 y 1.33 km, respectivamente (Figura \ref{subfig:dom4}). Y en la novena simulación se evaluaron tres dominios (un dominio parental y dos subdominios; d01, d02y d03) con una resolución de 10, 3.3 y 1.11 km, respectivamente (Figura \ref{subfig:dom4}).\






%{dominios_osm.png}

\begin{figure}[H]
	\begin{center}
		\begin{subfigure}[normla]{0.4\textwidth}
		\caption{Ubicación de tres dominios (un dominio parental y dos subdominios anidados)}
	\includegraphics[draft=false, scale=0.2]{dominios_osm.png}
		\label{subfig:dom3}
		\end{subfigure}
		\vspace{1\baselineskip}
		~
		\begin{subfigure}[normla]{0.4\textwidth}
		\caption{Ubicación con dos dominios (un dominio parental y un subdominio anidado)}
	\includegraphics[draft=false, scale=0.2]{dominios_osm2.png}
		\label{subfig:dom2}
		\end{subfigure}
		~
    	\begin{subfigure}[normla]{0.4\textwidth}
		\caption{Ubicación con dos dominios (un dominio parental y un subdominio anidado) y extensión de área más grande.}
	\includegraphics[draft=false, scale=0.2]{graph/dos_dom_grandes.png}
		\label{subfig:dom3.1}
		\end{subfigure}
		~
		\begin{subfigure}[normla]{0.4\textwidth}
		\caption{Ubicación con tres dominios (un dominio parental y dos subdominios anidados) y extensión de área más grande.}
	    \includegraphics[draft=false, scale=0.2]{graph/tres_dom_grandes.png}
		\label{subfig:dom4}
		\end{subfigure}      
		~	

	
	

	\end{center}
	\caption{Ubicación de los diferentes dominios y subdominios que se usaron para realizar cada una de las simulaciones evaluadas. El  polígono rojo corresponde al área de estudio y los recuadros de color corresponden a  los subdominios.}
	\label{gra:areas}	
\end{figure}

Con estas ubicaciones de las combinaciones dominio parental-subdominios se intentó buscar la mejor configuración posible, teniendo en cuenta las recomendaciones de las proporciones que se deben tener entre los dominios parentales y sus subdominios.\\

%%%%%%%%%%%%%%%%%%%%%%%%%%%%%%%%%%

Las salidas de las simulaciones son cada hora, con la finalidad de ser comparadas con los datos horarios de las estaciones automáticas HYDRAS. Para cada una de las simulaciones hechas con WRF se usaron 5 núcleos (\texttt{cores; Intel(R) Xeon(R) CPU E5620 @ 2.40GHz}) y 14 gigabytes de RAM. El tiempo empleado para la realización de cada simulación se encuentra en la Tabla \ref{tabla:dominios_1}. La versión usada del WRF es la 3.9.1.1 distribuida desde el 28 de agosto de 2017.

\begin{table}[H]
\caption{Simulaciones hechas para escoger la mejor combinación dominio parental-subdominios. Días = d y horas = h.}
\label{tabla:dominios_1}
\begin{center}
\begin{tabular}{c|p{5cm}p{2cm}cc}
Simulación & Combinación dominio parental-subdominios     & Resoluciones en km & Tiempo de cómputo & Figura  \\ \hline
1    & d01-d02-d03 & 36-12-4    & 77h & Figura \ref{subfig:dom3}\\ %(3d-5h)
2    & d01-d02-d03 & 18-6-2     & 164h& Figura \ref{subfig:dom3}\\ %(6d-20h)
3    & d01-d02     & 10-3.33     & 9h & Figura \ref{subfig:dom2}\\
4    & d01-d02     & 10-3.33     & 10.5h & Figura \ref{subfig:dom3.1}\\
5    & d01-d02     & 12-4     & 10.5h    & Figura \ref{subfig:dom3.1}   \\
6    & d01-d02     & 15-5    & 10.5h     & Figura \ref{subfig:dom3.1}  \\
7    & d01-d02     & 18-6     & 8.5h     & Figura \ref{subfig:dom3.1}  \\
8    & d01-d02-d03 & 12-4-1.33   & 17.5h &Figura \ref{subfig:dom4}\\
9    & d01-d02-d03 & 10-3.3-1.1  & 40 h &Figura \ref{subfig:dom4}
\end{tabular}
\end{center}

\end{table}

Se observa en la Tabla \ref{tabla:dominios_1} que la simulación 2 es la que más toma tiempo y la simulación 3 es la que menos toma tiempo. Si se comparan la simulación 1 con la 2 podemos observar que la diferencia entre las dos corresponde a diferencias de resolución, pero estas diferencias de resolución implican un aumento del doble de tiempo, ya que pasó de 3 días y 5 horas a 6 días y 20 horas. Respecto a las simulaciones 3 y 4 se nota que el aumento en el área produce un aumento en el tiempo de cómputo de una hora y media. Las simulaciones cuatro, cinco, seis y siete poseen los mismos tamaños de dominio-subdominio, pero la diferencia entre ellos son las resoluciones, entre más fina sea la resolución más tiempo de máquina es necesario. Si comparamos las simulaciones 1, 2, 8 y 9, que poseen 3 dominios (un dominio parental y dos subdominios) vemos que las simulaciones que poseen una resolución más fina gastan un menor tiempo, debido a que en el momento de realizar las simulaciones 1, 2 y 3 las máquinas estaban realizando otros trabajos, pero cuando se realizaron las simulaciones 8 y 9 las máquinas tenían dedicación exclusiva a este trabajo. Anexo \ref{anexo:namelist-input-wps} se encuentra las \textit{namelist.wps} y \textit{namelist.input} para cada una de las nueve simulaciones.




\subsection{Cuantificación de los mejores resultados.}
Con la finalidad de tener un método cuantitativo de clasificación del desempeño de las simulaciones frente a los datos reales se generó un índice basado en los estadísticos coeficiente de correlación de Pearson y el error cuadrático medio normalizado (NRMSE), llamado $ET$ (Ecuación \ref{fx_et}). Los estadísticos fueron calculados para cada combinación observaciones en estación-simulación. Se realizó una regresión linear de los estadísticos, con la finalidad de poder hacer una comparación entre los estadísticos.\\

Se usaron 10 estaciones y se realizaron 9 simulaciones. Para cada una de las 90 posibles combinaciones se realizó el siguiente cálculo. Para mostrar un ejemplo se creó la Tabla \ref{tabla:ejemplo_dominios}, los demás resultados se encuentran en el Anexo \ref{anexo:resultado_comparaciones_entre_estadisticos}.\\

El coeficiente de correlación de Pearson ($\rho$; Ecuación.\ref{pearson_coef}) puede dar resultados entre -1 y 1. Para los resultados analizados no se obtuvieron resultados negativos, entonces los mejores resultados son los valores cercanos a 1 y los resultados menos deseados son los cercanos a 0. Por esta razón al valor más alto de $\rho$ se le asignará el valor de 1 y a el menor valor se le asignará el valor de 0.\\

Se tomó el valor mínimo del coeficiente de correlación de Pearson presente en la columna $Pearson$, en la Tabla \ref{tabla:ejemplo_dominios} el valor es de 0.9355 a este valor en la columna $Pearson_{esc}$ se le asignó un valor de 0 y para el mayor valor del coeficiente de Pearson 0.9732 se le asignó un valor de 1. A partir de estos dos valores se calculó la pendiente teniendo en cuenta la formula $m = \frac{y_2 - y_1}{x_2 - x_1}$, donde $y_1 = 0, x_1 = 0.9355, y_2 = 1, x_2 = 0.9732$ por lo tanto $m = 26.5251$. La Ecuación de la recta es $y = mx + b$, entonces remplazamos y obtenemos que $b = -24.8143$, de esta forma se obtiene la ecuación con la que se van a calcular los datos de la columna $Pearson_{esc}$: $y = 26.5251x - 24.8143$.\\

Si tomamos el valor de la columna $Pearson$ de la tercera fila de la Tabla \ref{tabla:ejemplo_dominios} es de 0.9446. Usando la ecuación $y = 26.5251 (0.9446) - 24.8143$ obtenemos $y = 0.2414$ este valor correspondería a la fila 3 de la columna Pearson-esc. Todos los valores de la tabla fueron calculados con todas sus cifras, pero luego se realizó una aproximación a tres decimales.\\

Para el error cuadrático medio normalizado (NRMSE; Ecuación \ref{rmse_eq}) se realizó un proceso similar, pero es importante notar que los mejores resultados son los que están más cerca a 0 y los resultados con menor desempeño son valores positivos que pueden llegar hasta infinito. Por esta razón el valor mínimo del $NRMSE$ que es de 0.3274 se le asignó el valor de 1 en la columna $NRMSE_{esc}$ y para el valor máximo del $NRMSE$ que es de 0.9180 se le asignó el valor de 0 en la columna de $NRMSE_{esc}$, esto quiere decir que $y_1 =1, x_1 = 0.3274, y_2 = 0$ y $x_2 = 0.9180$.\\

%Para la desviación estándar se calculó la desviación estándar del modelo (sd) (Eq. \ref{sd_eq1}) y a los valores observados de temperatura del aire (Eq. \ref{sd_eq}), estos valores fueron restados y se les calculó el valor absoluto, estos datos están en la columna $STD_{abs}$ (Eq. \ref{std_{abs}}) de la Tabla \ref{tabla:ejemplo_dominios}. Los mejores resultados de la columna $STD_abs$ son aquellos cercanos a 0 y los resultados con menor desempeño son los valores positivos que pueden llegar hasta infinito. Por esta razón el valor mínimo de la $STD_{abs}$ 2.0088 se le asignó en la columna $STD_{esc}$ el valor de 1 y para el mayor valor de la columna $STD_{abs}$ 2.5282 se le asignó en la columna $STD_{esc}$ el valor de 0, esto quiere decir que $y_1 = 1, x_1 = 0.7565, y_2 = 0$ y $x_2 = 1.7257$.\\


Posteriormente se procedió a realizar la suma de los valores escalados $Pearson_{esc}$ y $NRMSE_{esc}$ y se dividió entre 2; el resultado de esta suma se encuentra en la columna $ET$ (Ecuación \ref{fx_et}). El grupo de resultados que tenga el mayor valor en la columna $ET$ será el que presente mayor similitud a los datos reales. En el ejemplo, para la estación Chinavita Automática es la simulación 6 dominio 2 la que presenta los mejores resultados.\\

\begin{equation} \label{fx_et}
    ET = \mathlarger{\frac{{NRMSE}_{esc} + Pearson_{esc}}{2}}
\end{equation}


\newpage
\begin{landscape}


\begin{longtable}{rrrrrrrrrrr}
\caption{Ejemplo de los resultados obtenidos de las comparaciones entre las diferentes simulaciones. * Corresponde al mayor valor en la columna $Pearson$, ** corresponde al menor valor de la columna $Pearson$ y los valores subrayados fueron usados como ejemplo.}
\label{tabla:ejemplo_dominios}\\
\hline
   Nombre & Simulación & Dominio &  Pearson &     NRMSE &  $NRMSE_{esc}$ &  $Pearson_{esc}$ &      ET \\
   
\\ \midrule % Ojo acá me presentó un error.
\endhead
\midrule
\multicolumn{3}{r}{{Continúa en la siguiente página.}} \\
\midrule
\endfoot

\bottomrule
\endlastfoot


Chinavita Automatica  &          1 &     d01 &   0.9654 &  0.8305 &   0.1481 &   0.7950 &  0.4715 \\
Chinavita Automatica  &          1 &     d02 &   0.9646 &  0.5980 &   0.5418 &   0.7714 &  0.6566 \\
Chinavita Automatica  &          1 &     d03 &   0.9549 &  0.3896 &   0.8947 &   0.5143 &  0.7045 \\
Chinavita Automatica  &          2 &     d01 &   0.9694 &  0.8115 &   0.1802 &   0.9002 &  0.5402 \\
Chinavita Automatica  &          2 &     d02 &   0.9662* &  0.9180 &  \textbf{0.0000} &   0.8141 &  0.4071 \\
Chinavita Automatica  &          2 &     d03 &   0.9507 &  0.4956 &   0.7152 &   0.4039 &  0.5596 \\
Chinavita Automatica  &          3 &     d01 &   0.9663 &  0.8022 &   0.1961 &   0.8180 &  0.5071 \\
Chinavita Automatica  &          3 &     d02 &   0.9622 &  0.4598 &   0.7759 &   0.7098 &  0.7428 \\
Chinavita Automatica  &          4 &     d01 &   0.9683 &  0.6330 &   0.4826 &   0.8702 &  0.6764 \\
Chinavita Automatica  &          4 &     d02 &   0.9547 &  0.3947 &   0.8860 &   0.5103 &  0.6982 \\
Chinavita Automatica  &          5 &     d01 &   0.9653 &  0.4724 &   0.7545 &   0.7924 &  0.7734 \\
Chinavita Automatica  &          5 &     d02 &   0.9644 &  0.5354 &   0.6478 &   0.7677 &  0.7078 \\
Chinavita Automatica  &          6 &     d01 &   0.9682 &  0.5470 &   0.6281 &   0.8672 &  0.7477 \\
Chinavita Automatica  &          6 &     d02 &   0.9732 &  0.5014 &   0.7054 &   \textbf{1.0000} &  \underline{0.8527} \\
Chinavita Automatica  &          7 &     d01 &   0.9690 &  0.7563 &   0.2738 &   0.8896 &  0.5817 \\
Chinavita Automatica  &          7 &     d02 &   0.9622 &  0.4988 &   0.7097 &   0.7087 &  0.7092 \\
Chinavita Automatica  &          8 &     d01 &   0.9679 &  0.8120 &   0.1795 &   0.8603 &  0.5199 \\
Chinavita Automatica  &          8 &     d02 &   0.9515 &  0.3401 &   0.9785 &   0.4236 &  0.7010 \\
Chinavita Automatica  &          8 &     d03 &   0.9509 &  0.3383 &   0.9817 &   0.4074 &  0.6945 \\
Chinavita Automatica  &          9 &     d01 &   0.9500 &  0.4243 &   0.8360 &   0.3857 &  0.6108 \\
Chinavita Automatica  &          9 &     d02 &   0.9373 &  0.3747 &   0.9199 &   0.0482 &  0.4840 \\
Chinavita Automatica  &          9 &     d03 &   0.9355** &  0.3274 &   \textbf{1.0000} &   \textbf{0.0000} &  0.5000 \\

\end{longtable}


\end{landscape}

En la Tabla \ref{tabla:ejemplo_dominios} se puede observar que el mayor resultado de la columna $ET$ es de 0.8527 y este valor es el resultado de la simulación 6 dominio 2 la cuál presentó el más alto valor de $\rho$ y un valor que no es el peor ni el mejor valor de $NRMSE$.\\

Para evaluar la mejor simulación, se marcaron los valores que tuvieran un coeficiente de Pearson en la columna  $Pearson$ superior a 0.8 y valores inferiores a 0.3 en la columna $NRMSE$ como lo sugiere \citet{Agua2016}. Se realizó una tabla de frecuencia de los resultados (Tabla \ref{tabla:resultado_dom_sum_dom}).\\


%\begin{table}[H]
%\centering
%\caption{Tabla de frecuencia de las simulaciones y dominios que presentaron valores en la suma mayores a 0.8 para la estación Chinavita %Automática.}
%\label{tabla:resultado_dom_sum_dom_capilla}
%\begin{tabular}{llrr}
%\toprule
%   Simulación & Dominio &   Valores \\
%\midrule
%
%         6 &     d02 &    1.0 \\
%         
%
%\bottomrule
%\end{tabular}
%\end{table}

%En la Tabla \ref{tabla:resultado_dom_sum_dom_capilla} podemos observar que sólo una simulación en el dominio 2 presentó valores superiores a 0.8 en la columna $ET$. Este ejemplo se realizó sólo con una estación, por esta razón no se puede obtener un valor superior a 1. Las tablas de los resultados totales se encuentran en el Anexo \ref{anexo:resultado_comparaciones_entre_estadisticos}. Y la tabla de frecuencia de los valores superiores a 0.8 es la Tabla \ref{tabla:resultado_dom_sum_dom}.

%Tabla generada para la evaluación de los modelos

\begin{table}[H]
\centering
\caption{Tabla de frecuencia de las simulaciones y dominios que presentaron valores superiores a 0.8 en la columna $Pearson$ e inferiores a 0.3  en la columna $NRMSE$. La combinación simulación-dominio ideal tendría un valor de frecuencia igual al numero de estaciones evaluadas, en este caso son 10, ya que es el número de estaciones automáticas con datos válidos para el 24 de febrero de 2007 (ver Tabla \ref{tab:estado_hydras}).}
\label{tabla:resultado_dom_sum_dom} % Extraído de extraccion_datos_dominios_20190506.py
\begin{tabular}{llrr}
\toprule
   Simulación & Dominio &   Frecuencia \\
\midrule
%         1 &     d02 &    3.0 \\
%         1 &     d03 &    1.0 \\
%         2 &     d01 &    2.0 \\
%         2 &     d02 &    5.0 \\
%         2 &     d03 &    2.0 \\
%         3 &     d01 &    3.0 \\
%         3 &     d02 &    3.0 \\
%         4 &     d01 &    1.0 \\
%         4 &     d02 &    4.0 \\
%         5 &     d01 &    3.0 \\
%         5 &     d02 &    2.0 \\
%         6 &     d01 &    3.0 \\
%         6 &     d02 &    4.0 \\
%         7 &     d01 &    4.0 \\
%         7 &     d02 &    3.0 \\
%         8 &     d01 &    3.0 \\
%         8 &     d02 &    2.0 \\
%         8 &     d03 &    2.0 \\
%         9 &     d01 &    2.0 \\
1   &   d01 &      3 \\
1   &   d02 &      6 \\
1   &   d03 &      7 \\
2   &   d01 &      2 \\
2   &   d02 &      5 \\
2   &   d03 &      6 \\
3   &   d01 &      7 \\
3   &   d02 &      6 \\
4   &   d01 &      6 \\
4   &   d02 &      7 \\
5   &   d01 &      6 \\
5   &   d02 &      7 \\
6   &   d01 &      3 \\
6   &   d02 &      7 \\
7   &   d01 &      3 \\
7   &   d02 &      6 \\
8   &   d01 &      6 \\
8   &   d02 &      5 \\
8   &   d03 &      6 \\
9   &   d01 &      5 \\
9   &   d02 &      6 \\
9   &   d03 &      9 \\


 
\bottomrule
\end{tabular}
\end{table}

En la Tabla \ref{tabla:resultado_dom_sum_dom} se observa que la simulación 9 con en el dominio d03 presenta los mejores resultados, esta simulación tarda 40 horas, por esta razón se descarta su uso, ver Tabla \ref{tabla:dominios_1}. Los siguientes mejores resultados los presentan las simulación 1 dominio 3, simulación 3 dominio 1, simulación 4 dominio 2, simulación 5 dominio 2 y simulación 6 dominio 2. Es importante notar que para algunas simulaciones los mejores resultados se obtienen de diferentes dominios, no precisamente en los dominios de mayor resolución.\\ 

Como se puede observar en la tabla \ref{tabla:resultado_dom_sum_dom} se presentan empates con una frecuencia de siete, para poder determinar con mayor facilidad cuál de las combinaciones simulación-dominio presentó mejor comportamiento se ordenaron las combinaciones de mejores a peores resultados basados en el índice $ET$ y se realizó una tabla de frecuencia de los cinco mejores resultados para cada estación, (ver Tabla \ref{tab:estado_hydras}). En este paso se usaron 10 estaciones, como se evaluaron los 5 mejores resultados en cada una de las estaciones la suma de la columna Valores debería ser igual a 50, pero la suma da como resultado 41. La razón para que los valores sean inferiores a lo esperado es que los resultados para algunas estaciones no superaron los filtros propuestos.\\

Se realizó una tabla con las estaciones evaluadas, en esta tabla en la columna Valores el máximo valor es 5, pero cuando los resultados de la estación no superaron los filtros entonces aparecen valores menores, como es el caso de la estación Chinavita Automática y la estación Hda Sta Ana Autom las cuales tuvieron valores de 0 y 5, respectivamente, ver Tabla \ref{Tabla_mejores_5_dominio}.\\

\begin{table}[]
\centering
\caption{Conteo de las mejores cinco simulaciones para 10 estaciones. La simulación ideal tendría un valor de frecuencia igual a 5.}
\begin{tabular}{ll}
Estación &  Valores \\ \hline
Tibaitata Automatica & 5 \\
Pmo Guerrero & 5\\
Ciudad Bolivar & 5\\
Subia Automatica & 5 \\
La Boyera Automatica & 5\\
Pmo Guacheneque & 5\\
Sta Cruz De Siecha & 5\\
Chinavita Automatica & 0 \\
La Capilla Autom & 5 \\
Hda Sta Ana Autom & 1\\
\end{tabular}
\label{Tabla_mejores_5_dominio}
\end{table}



\begin{table}[H] % Extraído de la función extraccion_datos_dominios_20190506.py
\centering
\caption{Conteo de las mejores nueve simulaciones para 10 estaciones. El máximo valor que se puede obtener para cada simulación es 10, ya que se evaluaron 10 estaciones.}
\begin{tabular}{lll}
Simulación & Dominio & Frecuencia \\ \hline
1   &  d02 &  2 \\
1   &  d03 &  1 \\
2   &  d01 &  2 \\
2   &  d02 &  5 \\
2   &  d03 &  2 \\
3   &  d01 &  3 \\
4   &  d02 &  4 \\
4   &  d01 &  1 \\
5   &  d01 &  3 \\
5   &  d02 &  1 \\
6   &  d01 &  2 \\
6   &  d02 &  1 \\
7   &  d01 &  3 \\
7   &  d02 &  2 \\
8   &  d01 &  2 \\
8   &  d02 &  1 \\
8   &  d03 &  2 \\
9   &  d01 &  2 \\
9   &  d02 &  1 \\
9   &  d03 &  1 \\


\end{tabular}
\label{Tabla_resum_dom}
\end{table}

En la Tabla \ref{Tabla_mejores_5_dominio}{Tabla_resum_dom} se evaluaron 10 estaciones y se tomaron los 5 mejores resultados para cada una de las estaciones se esperaría que la suma de la columna frecuencia fuera de 50, pero para las estaciones Chinativa Automática no hubo simulaciones que superaran los filtros y para la estación Hda Sta Ana Autom sólo hubo un caso que logró superar los filtros como se evidencia en la Tabla \ref{Tabla_mejores_5_dominio}. En la Tabla \ref{Tabla_resum_dom} se observa que la mejor simulacion es la simulación 2 dominio 2, pero esta simulacion 2 consume 164 horas (ver Tabla \ref{tabla:dominios_1}). Por lo tanto esta simulación no sera de gran utilidad si estos resultados se quieren usar para hacer pronósticos. La simulación que presenta buenos resultados según los resultados de las Tablas \ref{tabla:dominios_1}, \ref{tabla:resultado_dom_sum_dom} y \ref{Tabla_resum_dom} es la simulación 4 ya que presenta un consumo de máquina aceptable, presenta buenos resultados teniendo en cuenta las diferentes simulaciones y a nivel de los cinco mejores resultados.\\

La simulación 4 tiene una reducción de escala de un factor de 5 entre las resoluciones de los datos GFS y del dominio parental. La reducción de escala es un aspecto de gran importancia, como se menciona en el trabajo de \citet{Corrales2015} quienes realizaron una reducción de escala con un factor de 5, \citet{Parra2012} en sus investigación también usó un factor de 5 y \citep{ArmentaPorras2013} también usó una reducción de escala de 5.  Basado en estos trabajos podemos decir que la reducción de escala hecha a través de la simulación 4 ha es apta y proporciona buenos resultados.\\


%Basado en los resultados obtenidos previamente se observa que a medida que se inicia la modelación mas cerca a la fecha de interés, los resultados son mejores. Adicionalmente se observa que si se inicia el modelo en el mismo momento que sucede el fenómeno los resultados no son los mejores, esto se evidenció con los resultados de la simulación 16, pues esta no obtuvo los mejores resultados en la Tabla \ref{tabla:resultado_tiempo} y \ref{tabla_frecuencia_tiempos}.\\

%La simulación 14 presentó buenos resultados en Tablas \ref{tabla:resultado_tiempo} y 
%\ref{tabla_frecuencia_tiempos}, por esta razón el modelo se iniciará 12 horas antes del fenómeno de estudio. Es importante resaltar que estudios como los de \citet{Skamarock2008} y \citep{Arango2011} nombran que es necesario tomar un tener un momento de \textit{spin-up} y lo recomendado es de 12 horas.

\section{Conclusiones}

La escogencia de WRF como modelo de la atmósfer es aceptada porque WRF es un modelo regional de pronóstico de tiempo atmosférico que presenta buenas características para la realización de estudios de la temperatura, ya que es un modelo que tiene en cuenta aspectos físicos, dinámicos y su evolución, además es un modelo que ha sido probado y es usado actualmente en Colombia.\\

La mejor combinación simulación-dominio-resolución corresponde a la Simulación 4. Ya que esta simulación fue una de las que mejor representó la temperatura del aire  y se realizó en un tiempo de máquina promedio que sería aceptable para realizar pronósticos con 12 horas de anticipación a eventos de temperaturas mínimas y máximas.\\

%En la mayoría de los casos cuando la desviación estándar con la cuál se compara es superior a seis se presenta poca variabilidad en las series simuladas con el modelo WRF.\\

%El mejor momento para realizar el inicio de las modelaciones es a las XXXXX horas antes de los eventos de temperaturas ya que de esta manera aseguramos la mayor cantidad de aciertod, como de vio con la .\\

%Las variaciones realizadas en el modelo con respecto a los límites y las resoluciones evidencian cambios leves en las simulaciones evaluadas.



%%\begin{comment}

%%%%%%%%%%%%%%%%%%%%%%%%%%%%%%%%%%%%%%%%%%%%%%%%%%%%%%%%%%%%%%
%%%%%%%%%%%%%%%%%%%%%%%%%%%%%%%%%%%%%%%%%%%%%%%%%%%%%%%%%%%%%%

%%%%%%%%%%%%%%%%%%%%%%%%%%%%%%%%%%%%%%%%%%%%%%%%%%%%%%%%%%%%%%%%%%%
%%%%%%%%%%%%%%%%%%%%%%%%%%%%%%%%%%%%%%%%%%%%%%%%%%%%%%%%%%%%%%%%%%%
%%%%%%%%%%%%%%%%%%%%%%%%%%%%%%%%%%%%%%%%%%%%%%%%%%%%%%%%%%%%%%%%%%%

%\begin{comment}
\section{Determinación de los casos de estudio}%Código busqueda_fechas.py

%Se determinaron las fechas y horas en las cuales la temperatura estuvo bajo 0\celc\ y sobre 20\celc\ en las estaciones convencionales presentes en la zona, con la finalidad de determinar las mejores fechas para realizar el estudio. Como se vio anteriormente las estaciones automáticas tienden a sobre estimar los valores de la temperatura del aire, por esta razón, el límite de las altas temperaturas para las estaciones automáticas será de 25\celc.\\

Para realizar esta fase del estudio se va a tomar como referencia la estación Tibatitatá, ya que este lugar cuenta con datos de una estación convencional y una estación automática. Los criterios para la selección de los casos fueron:

\begin{itemize}
    \item Encontrar un período de altas y bajas temperaturas para un mes típico.
    
    \item Encontrar un período de altas y bajas temperaturas para un mes no típico.
    
    \item Encontrar el fenómeno de mas larga duración para altas y bajas temperaturas.
    
    \item Encontrar un período de altas y bajas temperaturas para un mes con influencia de un evento El Niño o La Niña.
\end{itemize}

Los meses en los que se presentan normalmente las heladas son diciembre, enero y febrero (esta información se puede en la sección de Caracterización de las heladas y altas temperaturas (Sección \ref{area_caracterizacion_heladas_extremas})) como resultado se presentan los siguientes cuatro casos:



\begin{itemize}
\item{Caso 1}%Este caso corresponde al mas largo, fuerte y está en el último mes de un periodo el niño ggweather.com/enso/oni.html

El primer caso de estudio se encuentra entre las fechas 31 de enero del 2007 hasta el 5 de febrero del 2007 que corresponde a un mes neutro según \citet{NOAA-ORI}. Se seleccionó este caso porque en el día 4 de febrero del 2007 se presentó la temperatura más baja reportada desde el año 1996 en la Sabana de Bogotá, ver Figura   \ref{fig:tmp_autom_tibaitata}. Adicionalmente, se presentaron altas temperaturas en algunas zonas,  ver Tabla \ref{table:caso1}. Según la estación Tibaitatá automática, la helada tuvo una duración de 5 horas comenzando a las 2 a.m. y finalizando a las 7 a.m., esto la convierte en la helada más larga para nuestro período de estudio que va desde el año 2007 hasta el año 2017. Por otra parte para esta misma estación, se registraron altas temperaturas desde las 1 pm hasta las 3 pm.

\item{Caso 2}%Mes no común en un niño (el periodo del 2015 fue niño todo)

El segundo caso seleccionado corresponde a una helada que se presentó en un mes atípico del año, en una temporada reportada en fase neutra de ENSO según \citet{NOAA-ORI}. Este caso se encuentra entre el 29 de agosto del 2014 hasta el 2 de septiembre del 2014. La helada se presentó el día 30 de agosto del 2014, ver Figura \ref{fig:tmp_autom_tibaitata_2014}. Esta helada fue registrada por varias estaciones convencionales, aunque las temperaturas no descendieron tanto como en el caso 1, ver Tabla \ref{table:caso2}.  Las bajas temperaturas que se registraron en la Estación Tibaitatá Automática duraron desde las 4 am hasta las 5 am. No se registraron temperaturas superiores a 25 \celc\ en la estación automática, pero las estaciones convencionales sí registraron valores superiores a 20 \celsius\ a las 3 p.m. el 30 de agosto, a las 3 p.m. del 31 de agosto y 4 p.m. del 1 de septiembre.

\begin{figure}[H]
    \centering
    \caption{Valores de temperatura para la estación Tibaitatá automática para las fechas entre el 29 de agosto hasta el 2 de septiembre del 2014. Las líneas verticales discontinuas corresponden a la temperatura de 0\celc\ en las horas 4 y 5 a.m. del 30 de octubre del 2014. Las líneas verticales de puntos corresponden a una temperatura de 20\celc\ a las 3 p.m. el 30 de agosto, a las 3 p.m. del 31 de agosto y 4 p.m. del 1 de septiembre.}
    \includegraphics[draft=false, scale=0.5]{casos_altos_bajas/grafica_minimas_tmp_tibaitata_helada_2014.png}
    \label{fig:tmp_autom_tibaitata_2014}
\end{figure}


En la figura \ref{carta_caso2_20140828} podemos observar que el día 28 se estaban presentando huracanes, estos estaban golpeando las costas orientales de Norte América. La mayor parte del territorio colombiano estaba en un área de baja presión. Estos huracanes estaban siendo fortalecidos por las ondas tropicales. La ZCIT no se encontraba sobre el territorio nacional. El día 29 de agosto se puede observar que se encontraba la ZCIT en la latitud 10\degree N, pero esta no estaba sobre el territorio nacional (Figura \ref{carta_caso2_20140829}). El día 30 de agosto de 2014 sobre la mayoría del territorio colombiano se encontraba en un área de baja presión. Las condiciones presentadas en estos días favorecieron la formación del huracán Cristobal.

\begin{figure}
\begin{subfigure}[normla]{0.5\textwidth}
\caption{Carta de superficie para el día 28 de agosto de 2014, hora 2:53 UTC.}
\includegraphics[draft=false, scale=0.25]{cartas_superficie/cs_20140828.png}
\label{carta_caso2_20140828}
\end{subfigure}
~
\begin{subfigure}[normla]{0.5\textwidth}
\caption{Carta de superficie para el día 29 de agosto de 2014, hora 2:32 UTC.}
\includegraphics[draft=false, scale=0.25]{cartas_superficie/cs_20140829.png}
\label{carta_caso2_20140829}
\end{subfigure}

~
\centering
\begin{subfigure}[normla]{0.5\textwidth}
\caption{Carta de superficie para el día 30 de agosto de 2014, hora 2:56 UTC.}
\includegraphics[draft=false, scale=0.25]{cartas_superficie/cs_20140830.png}
\label{carta_caso2_20140830}
\end{subfigure}

\caption{Análisis de superficie para el entre el 28 y 30 de agosto de 2014.}
\label{carta_caso2_20140829}
\end{figure}

El radiosondeo del 30 de agosto tuvo problemas ya que llegó hasta la altura de 400 mb (Figura \ref{radioson_201408}). De la fracción del radiosondeo se puede observar una inversión térmica hasta los 500 mb de altura. Es importante resaltar que estas condiciones de inversión están asociadas a heladas en la Sabana de Bogotá. Estas condiciones son similares a las presentadas para el caso 1.\\

%La tasa de enfriamiento es de 6.4°C por kilómetro. La transferencia de calor del suelo es más alta que el del aire. Por esta razón se enfría más rápido el suelo que la atmósfera. La transferencia de calor está dada por la cercanía que tengan las moléculas, en el sólido la transferencia de calor es más alta porque las moléculas están juntas, en el agua están juntas pero no tanto como en el sólido, y las moléculas que se encuentran más separadas son las moléculas de aire, por lo tanto la transferencia de calor es más baja.

\begin{figure}
\centering
\includegraphics[draft=false, scale=0.25]{radiosondeos/radio_sondeo_20140830.png}
\caption{Radiosondeo para el día 30 de agosto de 2014 a las 7 am HL. Tomado de la \textcolor{blue}{\href{http://weather.uwyo.edu/upperair/sounding.html}{Universidad de Wyoming.}}}
\label{radioson_201408}
\end{figure}
%% Discusión sobre las imágenes del goes para el caso 2 del 201408

En la Figura \ref{fig:goes_total_caso2} se puede observar que sobre la zona de estudio no se presentó una cobertura nubosa. La Figura \ref{fig:goes_caso2_2} permite identificar una nubosidad sobre el mar Caribe, especialmente sobre Haití y República Dominicana. Sobre Colombia se puede observar una formación nubosa en el sur de l país. Basado en estas condiciones podemos concluir que sobre el área de interés que es la Sabana de Bogotá no había cobertura nubosa en las horas de la mañana.

\begin{figure}[H]
    \begin{subfigure}[normla]{0.5\textwidth}
\caption{Radiación infrarroja de onda corta.}
\includegraphics[draft=false,     scale=0.23]{{goes/201408/goes13.2014.242.001519.BAND_02}.jpg}
\label{fig:goes_caso2_2}
\end{subfigure}
~
\begin{subfigure}[normla]{0.5\textwidth}
\caption{Vapor de agua}
\includegraphics[draft=false,     scale=0.23]{{goes/201408/goes13.2014.242.001519.BAND_03}.jpg}
\label{fig:goes_caso2_3}
\end{subfigure}
    ~
\centering
\begin{subfigure}[normla]{0.3\textwidth}
\caption{Radiación infrarroja de onda larga.}
\includegraphics[draft=false,         scale=0.21]{{goes/201408/goes13.2014.242.001519.BAND_04}.jpg}
\label{fig:goes_caso2_4}
\end{subfigure}
    
    	
    \caption{Imágenes del GOES 12: canal 2 (a), canal 3 (b) y canal 4(c). Para el día 30 de agosto de 2014 a las 00 UTC (29 de febrero de 2007 a las 19:00 HL).}
    \label{fig:goes_total_caso2}	
\end{figure}

\begin{table}[H]
\centering
\caption{Temperaturas máximas diarias y mínimas diarias de las estaciones convencionales que registraron temperatura sobre 20\celsius\ o debajo de 0\celsius\ o ambas y temperaturas máxima diaria y mínima diaria reistradas para la estación automática Tibaitatá, para el caso 2 del día 30 de agosto del 2014.}
\begin{tabular}{p{3.5cm}p{2cm}lll}
Temperatura igual o menor a 0\celsius\ & Temperatura sobre 20\celc & Código   & Nombre de la estación & Municipio \\ \hline
-3.6 & &  21205870 &   Salitre El [21205870] &      Bojacá \\
-1.6 & &  21205880 &         Flores Chibcha  &      Madrid \\
-0.2 & &  21205920 &  Suasuque    [21205920] &        Sopó \\
 0.0 & &  21205940 &   Villa Inés [21205940] &  Facatativá \\
 & 20.6 &  21206560 &  INEM Kennedy [21206560] &  Bogotá D.C. \\
 & 21.0 &  21205710 &         Jardín Botánico  &  Bogotá D.C. \\
 & 21.0 &  21206690 &      Col Miguel A. Caro  &  Bogotá D.C. \\
 0.0 & 21.4 &  21205980 &         Providencia Gja  & Tenjo \\
 & 21.4   & 21206620 &       Col H Duran Dussan  &  Bogotá D.C. \\
 0.0 & 22.0 &  21205420 &     Tibaitatá &     Mosquera \\
 -1.0 & 22.0 &  21206990 &    Tibaitatá [Automática] &    Mosquera \\
\end{tabular}

\label{table:caso2}
\end{table}




\item{Caso 3}

El tercer caso seleccionado corresponde a una alta temperatura presentada en el mes de agosto en un evento El Niño según \citet{NOAA-ORI}. Se seleccionó este caso porque se presentó en un mes poco habitual. El período de estudio se encuentra entre el 24 y el 28 de agosto del 2015. Se registraron temperaturas superiores a 20\celc\ en dos estaciones, la estación convencional Tibaitatá, la cual presentó una temperatura cerca a los 20\celc\ y la Estación Tibaitatá Automática registró valores superiores a 25\celc, ver Tabla \ref{table:caso3}. No se presentaron temperaturas iguales o inferiores a 0\celc. La duración de este evento fue de una hora, iniciando a las 9 am y finalizando a las 10 am. Se observa en la Figura \ref{fig:tmp_autom_tibaitata_201508} que en todos los días estudiados se presentaron temperaturas superiores a 20\celsius.

\begin{figure}[H]
    \centering
    \caption{Valores de temperatura para la estación Tibaitatá automática para las fechas entre el 24 y 28 de agosto del 2015. Las líneas verticales de puntos corresponden a una temperatura de 20\celsius, donde se puede observar que en todos los días estudiados la temperatura superó este umbral.}
    \includegraphics[draft=false, scale=0.5]{casos_altos_bajas/grafica_minimas_tmp_tibaitata_helada_201508.png}
    \label{fig:tmp_autom_tibaitata_201508}
\end{figure}

En la Figura \ref{carta_caso3} no se encuentra la ZCIT, pero sí se puede observar la zona de \textit{Monsoon trought} (MONSOON TROF). \textit{Monsoon trought} es una parte de la ZCIT que representa un área donde los vientos del norte y los vientos del sur se encuentran, esta zona es mostrada como una línea que conecta áreas de baja presión \citep{Wang2006}.

Basado en el análisis de superficie podemos observar que el día 25 de agosto de 2015 sobre Colombia había un sistema de baja presión y este sistema se encontraba sobre la zona de estudio (Figura \ref{carta_caso3_20150825}). El día 26 de agosto de 2015 en el mar Caribe un sistema de baja presión se convirtió en un huracán (Erika), sobre Colombia no hubo sistemas de bajas presiones, pero la zona del \textit{Monsoon trought} se encontraba en la parte Norte de Colombia (Latitud 10\degree N). El día 27 de agosto de 2015 se encontraba sobre Colombia el \textit{Monsoon trought} en las costas pacíficas con dirección a la Región de la Guajira y no había ningún sistema de altas o bajas sobre sobre Colombia. Adicionalmente para los días analizados no se presentó cobertura nubosa sobre el territorio colombiano.\\



\begin{figure}[H]
\begin{subfigure}[normla]{0.5\textwidth}
\caption{Carta de superficie para el día 25 de agosto de 2015, hora 2:16 UTC.}
\includegraphics[draft=false, scale=0.25]{cartas_superficie/cs_20150825.png}
\label{carta_caso3_20150825}
\end{subfigure}
~
\begin{subfigure}[normla]{0.5\textwidth}
\caption{Carta de superficie para el día 26 de agosto de 2015, hora 2:24 UTC.}
\includegraphics[draft=false, scale=0.25]{cartas_superficie/cs_20150826.png}
\label{carta_caso3_20150826}
\end{subfigure}

~
\centering
\begin{subfigure}[normla]{0.5\textwidth}
\caption{Carta de superficie para el día 27 de agosto de 2015, hora 2:27 UTC.}
\includegraphics[draft=false, scale=0.25]{cartas_superficie/cs_20150827.png}
\label{carta_caso3_20150827}
\end{subfigure}

\caption{Análisis de superficie entre el 25 y 27 de agosto de 2015.}
\label{carta_caso3}
\end{figure}

Como se puede observar en el radiosondeo (Figura \ref{radiosondeo_201508}) se observa una pequeña inestabilidad en los primeros metros del radiosondeo, pero posteriormente la atmósfera se estabiliza. Presenta alta humedad entre los 650 y los 700 mb. A una altura superior a los 650 mb la atmósfera tiene menor humedad.


\begin{figure}[H]
\centering

\includegraphics[draft=false, scale=0.25]{radiosondeos/radio_sondeo_20150827.png}
\caption{Radiosondeo para el día 27 de agosto de 2015 a las 7 am HL. Tomado de la \textcolor{blue}{\href{http://weather.uwyo.edu/upperair/sounding.html}{Universidad de Wyoming.}}}
\label{radiosondeo_201508}
\end{figure}

%% Discución sobre las imágenes del goes para el caso 3 del 201508

\begin{figure}[H]
    \begin{subfigure}[normla]{0.5\textwidth}
\caption{Radiación infrarroja de onda corta.}
\includegraphics[draft=false,     scale=0.23]{{goes/201508/goes13.2015.239.001519.BAND_02}.jpg}
\label{fig:goes2}
\end{subfigure}
~
\begin{subfigure}[normla]{0.5\textwidth}
\caption{Vapor de agua}
\includegraphics[draft=false,     scale=0.23]{{goes/201508/goes13.2015.239.001519.BAND_03}.jpg}
\label{fig:goes3}
\end{subfigure}
    ~
\centering
\begin{subfigure}[normla]{0.3\textwidth}
\caption{Radiación infrarroja de onda larga.}
\includegraphics[draft=false,         scale=0.21]{{goes/201508/goes13.2015.239.001519.BAND_04}.jpg}
\label{fig:goes4}
\end{subfigure}
    
    	
    \caption{Imágenes del GOES 13: canal 2 (a), canal 3 (b) y canal 4(c). Para el día 27 de agosto de 2015 a las 00 UTC (26 de agosto de 2015 a las 19:00 HL).}
    \label{fig:goes_total}	
\end{figure}


\begin{table}[H]
\centering
\caption{Temperaturas máximas registradas diarias de las estaciones convencionales que registraron temperaturas sobre 20\celc y temperatura máxima diaria registrada para la estación automática Tibaitatá, para el caso 4 correspondiente al día 27 de agosto del 2015.}
\begin{tabular}{p{5cm}lll}
Temperatura iguales o superiores a 20\celc & Código   & Nombre de la estación & Municipio \\ \hline
22.0           & 21206620 & Col H Duran Dussan   & Bogotá \\
20.1           & 21205980 & Providencia Gja   & Tenjo \\
19.6           & 21205420 & Tibaitatá   & Mosquera \\
26.0         & 21206990 & Tibaitatá[Automática] & Mosquera
\end{tabular}
\label{table:caso3}


\end{table}


\item{Caso 4}

%\textit{Se creó una carpeta en agrometeo llamada /home/agrometeo/wrf/resultados/zona\_201508, para procesar el wps y los archivos usados se descargaron en Downloads}


La cuarta fecha seleccionada corresponde a una alta temperatura presentada en el mes de septiembre de 2015 en una temporada bajo la influencia de un evento El Niño según \citet{NOAA-ORI}. Se seleccionó este caso ya que fue uno de los que presentó el intervalo de tiempo con mas horas sobre 25 \celc\ el cual comenzó a las 10:33 a.m. y finalizó a las 3:21 p.m., durando cuatro horas y media, ver Figura \ref{fig:tmp_autom_tibaitata_201509}. El período de estudio se encuentra entre el 06 de septiembre del 2015 y el 09 de septiembre del 2015. Varias estaciones registraron temperaturas sobre 20 \celsius ninguna estación presentó valores menores o iguales a 0\celc, ver la Tabla \ref{table:caso4}. Adicionalmente, se presentó temperaturas sobre 20\celsius\ el 6 de septiembre de 2015 a las 3 p.m. y el 7 de septiembre  de 2015 a las 12 m.

%El día 20150907 no estaba disponible en los datos del GFS, por esta razón no se usaron estos datos. Pero si estaban los datos del día analizado que corresponde al 20150908.
%ftp://nomads.ncdc.noaa.gov/GFS/analysis_only/201509/20150907/

\begin{figure}[H]
    \centering
    \caption{Valores de temperatura para la estación Tibaitatá automática para las fechas entre el 6 y el 9 de septiembre del 2015. Las líneas verticales de puntos corresponden a una temperatura de 20\celsius, donde se puede observar que en todos los días estudiados la temperatura superó este umbral.}
    \includegraphics[draft=false, scale=0.5]{casos_altos_bajas/grafica_minimas_tmp_tibaitata_helada_201509.png}
    \label{fig:tmp_autom_tibaitata_201509}
\end{figure}



\begin{figure}[H]
\begin{subfigure}[normla]{0.5\textwidth}
\caption{Carta de superficie para el día 25 de agosto de 2015, hora 2:16 UTC.}
\includegraphics[draft=false, scale=0.25]{cartas_superficie/cs_20150904.png}
\label{carta_caso4_20150825}
\end{subfigure}
~
\begin{subfigure}[normla]{0.5\textwidth}
\caption{Carta de superficie para el día 26 de agosto de 2015, hora 2:24 UTC.}
\includegraphics[draft=false, scale=0.25]{cartas_superficie/cs_20150905.png}
\label{carta_caso4_20150826}
\end{subfigure}

~
\centering
\begin{subfigure}[normla]{0.5\textwidth}
\caption{Carta de superficie para el día 27 de agosto de 2015, hora 2:27 UTC.}
\includegraphics[draft=false, scale=0.25]{cartas_superficie/cs_20150906.png}
\label{carta_caso4_20150827}
\end{subfigure}

\caption{Análisis de superficie para los días entre el 4 y 6 de septiembre de 2015.}
\label{carta_caso4}
\end{figure}

Como se puede observar en el radiosondeo \ref{radiosondeo_201508} podemos ver una pequeña inestabilidad en los primeros metros del radiosondeo, pero posteriormente se observa que la atmósfera se estabiliza. Presenta alta humedad entre los 650 y los 700 mb, probablemente a esa altura se encuentre nubosidad. A una altura superior a los 650 mb la atmósfera tiene menor humedad.


\begin{figure}[H]
\centering
\includegraphics[draft=false, scale=0.25]{radiosondeos/radio_sondeo_20150908.png}
\caption{Radiosondeo del 8 de septiembre de 2015 a las 7 am HL. Tomado de la \textcolor{blue}{\href{http://weather.uwyo.edu/upperair/sounding.html}{Universidad de Wyoming.}}} 
\label{radiosondeo_201508}
\end{figure}

%% Discución sobre las imágenes del goes para el caso 4 del 201508

\begin{figure}[H]
    \begin{subfigure}[normla]{0.5\textwidth}
\caption{Radiación infrarroja de onda corta.}
\includegraphics[draft=false,     scale=0.23]{{goes/201509/goes13.2015.251.001519.BAND_02}.jpg}
\label{fig:goes2}
\end{subfigure}
~
\begin{subfigure}[normla]{0.5\textwidth}
\caption{Vapor de agua}
\includegraphics[draft=false,     scale=0.23]{{goes/201509/goes13.2015.251.001519.BAND_03}.jpg}
\label{fig:goes3}
\end{subfigure}
    ~
\centering
\begin{subfigure}[normla]{0.3\textwidth}
\caption{Radiación infrarroja de onda larga.}
\includegraphics[draft=false, scale=0.21]{{goes/201509/goes13.2015.251.001519.BAND_04}.jpg}
\label{fig:goes4}
\end{subfigure}
    
    	
    \caption{Imágenes del GOES 13: canal 2 (a), canal 3 (b) y canal 4(c). Para el día 27 de agosto de 2015 a las 00 UTC (26 de agosto de 2015 a las 19:00 HL).}
    \label{fig:goes_total}	
\end{figure}

\begin{table}[H]
\centering
\caption{Temperaturas máximas diarias de las estaciones convencionales que registraron temperaturas sobre 20\celc\ y temperatura máxima registrada para la estación automática Tibaitatá, el 8 septiembre del 2015.}
\begin{tabular}{llll}
Temperatura sobre 20\celc & Código   & Nombre de la estación & Municipio \\ \hline
 20.3 &  21205790 &          Apto El Dorado  &  Bogotá D.C. \\
 20.6 &  21206620 &       Col H Duran Dussan  &  Bogotá D.C. \\
 20.8 &  21206560 &  INEM Kennedy &  Bogotá D.C. \\
 21.6 &  21205980 &         Providencia Gja  &        Tenjo \\
 21.6 &  21206260 &       C.Univ.Arop-UDCA  &  Bogotá D.C. \\
 20.8 &  21205420 &     Tibaitatá &     Mosquera \\
 26.7 & 21206990 & Tibaitatá[Automática] & Mosquera\\
\end{tabular}

\label{table:caso4}



\end{table}
\color{blue}
Las horas en las que se presenta las menores temperaturas es cerca de las 5 am (hora local) como se vio en el Capítulo 1, este estudio se enfocó en estudiar las bajas y las altas temperaturas, por esta razón luego de identificar el día en el que se presentaron los eventos de altas o bajas temperaturas para los diferentes casos se inició 12 horas antes de las 5 am hora local (hora local). Para el caso 1 el evento se presentó el 4 de febrero del 2007, para el caso 2 el evento se presentó el 30 de agosto de 2014, el caso 3 fue el 27 de agosto de 2015 y el 8 de septiembre de 2015, las horas de inicio se encuentran en la Tabla \ref{table:fechas_de_inicio}.\\

\begin{table}[H]
\centering
\caption{Fechas de inicio de modelación para cada uno de los casos en horas UTC y hora local.}
\begin{tabular}{lll}
Caso & Fecha de inicio UTC & Fecha de inicio hora local \\ \hline
1 & 2007-02-03 18:00 & 2007-02-05 13:00 \\
2 & 2014-08-29 18:00 & 2014-08-29 13:00 \\
3 & 2015-08-26 18:00 & 2015-08-27 13:00 \\
4 & 2015-09-07 18:00 & 2015-09-07 13:00
\end{tabular}
\label{table:fechas_de_inicio}
\end{table}
\color{black}
\section{Búsqueda de las mejores parametrizaciones}

Para la evaluación de la mejor parametrización se tuvo como base la configuración que el IDEAM usa hacer las simulaciones sobre todo el territorio colombiano. Se probaron diferentes parametrizaciones con cada una de sus opciones, estas pruebas no fueron combinatorias, ya que se requiere demasiado tiempo para realizar todas las posibles combinaciones (ver Tabla \ref{tabla_parametrizacion_opciones}). Las pruebas se hicieron de tal manera que se cambia una opción de parametrización a la vez mientras que todas las demás opciones de parametrizaciones son iguales a las de la configuración IDEAM-Colombia. Las parametrizaciones a las cuales se les probaron sus diferentes opciones son: capa límite planetaria (bl\_pbl\_physics) la cual se encarga de los flujos superficiales de (calor, humedad y momento) y su difusión vertical; parametrización de cúmulos (cu\_physics); microfísica de nubes (mp\_physics), la cual se encarga de resolver los procesos que tienen que ver con el vapor de agua, nubes y el proceso de precipitación; parametrizaciones de radiación de onda larga y corta (ra\_lw\_phyisics y ra\_sw\_phyisics), las cuales se encargan de las tendencias de la temperatura atmosférica y los flujos de radiación de la superficie; parametrización de suelo (sf\_surface\_physics), la cual se encarga de modelar las temperaturas del suelo y parametrizaciones de superficie (sf\_sfclay\_surface\_physics) se encarga de determinar los coeficientes de intercambio entre el suelo y la atmósfera. También se realizaron dos simulaciones adicionales, una con la configuración IDEAM-Colombia y otra con la configuración IDEAM-Bogotá, las cuales se muestran en el Anexo \ref{anexo:namelist_mejor_parametrizacion}.\\

%Es importante recordar que la parametrización de capa límite planeraria se encarga de w
%https://www.climatescience.org.au/sites/default/files/physics-3.9-new-pt1.pdf
%rainc = comes from cumulus scheme
%rainnc = comes from microphysics scheme

Las opciones evaluadas para cada parametrización se encuentran en la Tabla \ref{tabla_parametrizacion_opciones}. En esta tabla podemos observar que se van a evaluar 10 opciones de parametrización para la capa límite planetaria, 9 opciones para cúmulos, 17 opciones para microfísica, 7 opciones para la radiación de onda larga, 7 opciones para la radiación de onda corta, 3 opciones para superficie y 1 opción para el suelo; para las opciones del suelo se probaron las opciones 0, 1, 3, 4, 31 y 99, pero sólo la opción 1 generó resultados. Adicionalmente, se evaluará las configuraciones propuestas por el IDEAM para Colombia y Bogotá. Para un total de 56 diferentes combinaciones.

%En total se realizaron 88 diferentes modelaciones, pero algunas de estas modelaciones no fueron aceptadas por el modelo WRF, ya que para que fueran ejecutadas exitosamente era necesario realizar otros cambios en la \textit{namelist.input}, pero para asegurar la homogeneidad en los resultados se realizaron los cambios que están en la tabla \ref{tabla_parametrizacion_opciones}.

A continuación se muestra un ejemplo de cómo se realizaron los cambios en las configuraciones del \texttt{namelist.input}. Se tomó como base la configuración del IDEAM para Colombia y se probaron las opciones para una sola parametrización a la vez. Como ejemplo se va a mostrar una parte de la  configuración del \texttt{namelist.input} IDEAM para Colombia y cómo se realizaron los cambios para la parametrización de microfísica (mp\_physics) para las opciones 0, 1 y 51.\\

\begin{table}[H]
\caption{Parametrizaciones y opciones usadas, en el WRF versión 9.1.1.}
\label{tabla_parametrizacion_opciones}
\begin{tabular}{lp{8cm}}
Parametrización & Opciones \\ \hline
Capa límite planetaria (bl\_pbl\_physics) & 0, 1, 5, 6, 7, 8, 9, 11, 12, 99 \\ %10
Cúmulos (cu\_physics)      & 0, 1, 2, 6, 11, 14, 16, 93, 99 \\ %9
Microfísica (mp\_physics)      & 0, 1, 2, 3, 5, 6, 7, 8, 9, 10, 11, 13, 14, 16, 19, 32, 51 \\%17
Radiación de onda larga (ra\_lw\_phyisics) & 0, 1, 3, 4, 5, 7, 31 \\%7
Radiación de onda corta (ra\_sw\_phyisics) & 0, 1, 2, 3, 4, 5, 7 \\%7 
Superficie (sf\_sfclay\_physics) & 0, 1, 91\\ %3
Suelo (sf\_surface\_physics) & 1 \\ % 1
IDEAM para Bogotá (IDEAM-Bogota) & \\
IDEAM para Colombia (IDEAM-Colombia) & \\ % Total 56
\end{tabular}
\end{table}
%\begin{tabular}{lp{8cm}}
%Parametrización & Opciones \\ \hline
%Capa límite planetaria (bl\_pbl\_physics) & 0, 1, 2, 3, 4, 5, 6, 7, 8, 9, 10, 11, 12, 99 \\ %14
%Cúmulos (cu\_physics)      & 0, 1, 2, 3, 4, 5, 6, 7, 11, 14, 16, 93, 99 \\ %13
%Microfísica (mp\_physics)      & 0, 1, 2, 3, 5, 6, 7, 8, 9, 10, 11, 13, 14, 15, 16, 19, 30, 32, 50, 51, 52 \\%21
%Radiación de onda larga (ra\_lw\_phyisics) & 0, 1, 2, 3, 4, 5, 7, 31, 99 \\%9
%Radiación de onda corta (ra\_sw\_phyisics) & 0, 1, 2, 3, 4, 5, 7, 31, 99 \\%9 
%Superficie (sf\_sfclay\_physics) & 0, 1, 2, 3, 4, 5, 7, 10, 91\\ %9
%Suelo (sf\_surface\_physics) & 0, 1, 2, 3, 4, 5, 7, 8, 31, 99\\ % 10
%IDEAM para Bogotá (IDEAM-Bogota) & \\
%IDEAM para Colombia (IDEAM-Colombia) & \\ % Total 88
%\end{tabular}


\begin{figure}[H]

\center
\texttt{
...\\
\&physics\\
mp\_physics~~~~~~~~~~~~~~~=~3,~3,\\
ra\_lw\_physics~~~~~~~~~~~~=~1,~~1,\\
ra\_sw\_physics~~~~~~~~~~~~=~1,~~1,\\
%radt~~~~~~~~~~~~~~~~~~~~~=~30,~30,\\
sf\_sfclay\_physics~~~~~~~~=~1,~~1,\\
sf\_surface\_physics~~~~~~~=~1,~~1,\\
bl\_pbl\_physics~~~~~~~~~~~=~1,~~1,\\
bldt~~~~~~~~~~~~~~~~~~~~~~=~1,~~1,\\
cu\_physics~~~~~~~~~~~~~~~~=~1,~~1,\\
...\\
}
\centering
\includegraphics[draft=false, scale=0.0001]{graph/ideam_ej_1.png}
\caption{Sección del \texttt{namelist.input} IDEAM-Colombia.}
\label{imag_tab_ej}

\end{figure}




%\texttt{
%...\\
%\&physics\\
%mp\_physics~~~~~~~~~~~~~~~=~\underline{0},~\underline{0},\\
%ra\_lw\_physics~~~~~~~~~~~~=~1,~~1,\\
%ra\_sw\_physics~~~~~~~~~~~~=~1,~~1,\\
%radt~~~~~~~~~~~~~~~~~~~~~=~30,~30,\\
%sf\_sfclay\_physics~~~~~~~~=~1,~~1,\\
%sf\_surface\_physics~~~~~~~=~1,~~1,\\
%bl\_pbl\_physics~~~~~~~~~~~=~1,~~1,\\
%bldt~~~~~~~~~~~~~~~~~~~~~~~=~1,~~1,\\
%cu_physics~~~~~~~~~~~~~~~~~=~1,~~1,\\
%...\\
%}




\begin{figure}[H]


\center
\texttt{
...\\
\&physics\\
mp\_physics~~~~~~~~~~~~~~~=~\underline{0,~0},\\
ra\_lw\_physics~~~~~~~~~~~~=~1,~~1,\\
ra\_sw\_physics~~~~~~~~~~~~=~1,~~1,\\
%radt~~~~~~~~~~~~~~~~~~~~~=~30,~30,\\
sf\_sfclay\_physics~~~~~~~~=~1,~~1,\\
sf\_surface\_physics~~~~~~~=~1,~~1,\\
bl\_pbl\_physics~~~~~~~~~~~=~1,~~1,\\
bldt~~~~~~~~~~~~~~~~~~~~~~=~1,~~1,\\
cu\_physics~~~~~~~~~~~~~~~~=~1,~~1,\\
...\\
}
\centering
\includegraphics[draft=false, scale=0.018]{ideam_ej_1.png}
\caption{Sección del \texttt{namelist.input} para el primer ejemplo de la microfísica con la opción 0 (mp\_physics-0).}
\label{imag_tab_ej2}

\end{figure}



\begin{figure}[H]

\center
\texttt{
...\\
\&physics\\
mp\_physics~~~~~~~~~~~~~~~=~\underline{1,~1},\\
ra\_lw\_physics~~~~~~~~~~~~=~1,~~1,\\
ra\_sw\_physics~~~~~~~~~~~~=~1,~~1,\\
%radt~~~~~~~~~~~~~~~~~~~~~=~30,~30,\\
sf\_sfclay\_physics~~~~~~~~=~1,~~1,\\
sf\_surface\_physics~~~~~~~=~1,~~1,\\
bl\_pbl\_physics~~~~~~~~~~~=~1,~~1,\\
bldt~~~~~~~~~~~~~~~~~~~~~~=~1,~~1,\\
cu\_physics~~~~~~~~~~~~~~~~=~1,~~1,\\
...\\
}
\centering
\includegraphics[draft=false, scale=0.018]{ideam_ej_1.png}
\caption{Sección del \texttt{namelist.input} para el segundo ejemplo de la microfísica con la opción 1 (mp\_physics-1).}
\label{imag_tab_ej3}


\end{figure}


\begin{figure}[H]

\center
\texttt{
...\\
\&physics\\
mp\_physics~~~~~~~~~~~~~~~=~\underline{51,~51},\\
ra\_lw\_physics~~~~~~~~~~~~=~1,~~1,\\
ra\_sw\_physics~~~~~~~~~~~~=~1,~~1,\\
%radt~~~~~~~~~~~~~~~~~~~~~=~30,~30,\\
sf\_sfclay\_physics~~~~~~~~=~1,~~1,\\
sf\_surface\_physics~~~~~~~=~1,~~1,\\
bl\_pbl\_physics~~~~~~~~~~~=~1,~~1,\\
bldt~~~~~~~~~~~~~~~~~~~~~~=~1,~~1,\\
cu\_physics~~~~~~~~~~~~~~~~=~1,~~1,\\
...\\
}
\centering
\includegraphics[draft=false, scale=0.018]{ideam_ej_1.png}
\caption{Sección del \texttt{namelist.input} para el segundo ejemplo de la microfísica con la opción 51 (mp\_physics-1).}
\label{imag_tab_ej3}

\end{figure}





Este proceso se repitió para cada uno de los cuatro casos de estudio mencionados en la sección "Búsqueda de las mejores parametrizaciones". Para cada caso, se realizaron 56 simulaciones. Las fechas y horas de inicio de las modelaciones se encuentran en la Tabla \ref{tabla_inicio_fin_casos}.\\
 
Se repitió el proceso realizado para la selección de los dominios el cual consiste en seleccionar las combinaciones parametrización-opción que tengan un coeficiente de Pearson en la columna  $Pearson$ mayor a 0.8 y valores inferiores a 0.3 en la columna $NRMSE$ como lo sugiere \citet{Agua2016} y que se encuentr entre los mejores 5 resultados de la columna $ET$, estos resultados se encuentran en la Tabla \ref{tabla_mejores_param_opciones}.

 
 
 \begin{table}[H]
 \centering
 \caption{Fechas y horas UTC de inicio y finalización de los diferentes casos.}
 \label{tabla_inicio_fin_casos}
\begin{tabular}{lll}
Caso & Inicio & Finalización \\ \hline
1 & 2007-02-03 18:00 & 2007-02-05 00:00 \\
2 & 2014-08-29 18:00 & 2014-09-01 00:00 \\
3 & 2015-08-27 18:00 & 2015-08-30 00:00 \\
4 & 2015-09-07 18:00 & 2015-09-10 00:00 \\
\end{tabular}
\end{table}


Es importante resaltar que el número de estaciones con las que se trabaja varía entre los casos. Por ejemplo: en el caso 1 se usaron 10 estaciones, en el caso 2 se usaron 15 estaciones, en el caso 3 se usaron 16 estaciones y en el caso 4 se usaron 15 estaciones debido a la disponibilidad de los datos.\\
%Esto implica que si existiera la combinación parametrización-opción perfecta tendría que tener 56 valores en la frecuencia


\begin{longtable}{lll}

\caption{Conteo de las mejores nueve simulaciones para 10 estaciones. La simulación ideal tendría un valor de frecuencia igual a 10 que sería el número de combinaciones simulaciones-dominios (ver Tabla \ref{tab:estado_hydras}).}
\label{tab_estaciones_5_tiempo}\\
\hline
Caso & Estación & Valores \\

\midrule
\endhead
\midrule
\multicolumn{3}{r}{{Continúa en la siguiente página.}} \\
\midrule
\endfoot

\bottomrule
\endlastfoot

Caso 1 & Pmo Guerrero  & 5 \\
Caso 1 & Subia Automatica  & 5 \\
Caso 1 & Ciudad Bolivar  & 5 \\
Caso 1 & Hda Sta Ana Autom  & 4 \\
Caso 1 & Tibaitata Automatica  & 5 \\
Caso 1 & Sta Cruz De Siecha  & 5 \\
Caso 1 & La Capilla Autom  & 5 \\
Caso 1 & Chinavita Automatica  & 5 \\
Caso 1 & Pmo Guacheneque  & 5 \\
Caso 2 & Univ Nacional  & 5 \\
Caso 2 & Chinavita Automatica  & 5 \\
Caso 2 & La Boyera Automatica  & 5 \\
Caso 2 & Apto El Dorado  & 5 \\
Caso 2 & Tibaitata Automatica  & 5 \\
Caso 2 & San Cayetano Autom   & 0 \\
Caso 2 & Bosque Intervenido    & 5 \\
Caso 2 & Hda Sta Ana Autom  & 5 \\
Caso 2 & Esc La Union Automatica & 5 \\
Caso 2 & Pmo Guerrero  & 5 \\
Caso 2 & Pmo Guacheneque  & 5 \\
Caso 3 & Hda Sta Ana Autom  & 5 \\
Caso 3 & Nueva Generacion  & 5 \\
Caso 3 & Chinavita Automatica  & 5 \\
Caso 3 & Esc La Union Automatica & 5 \\
Caso 3 & Pmo Guerrero  & 5 \\
Caso 3 & Villa Teresa Automatica  & 5 \\
Caso 3 & Bosque Intervenido    & 5 \\
Caso 3 & Univ Nacional  & 5 \\
Caso 3 & Tibaitata Automatica  & 5 \\
Caso 3 & Sta Cruz De Siecha  & 5 \\
Caso 3 & Apto El Dorado  & 5 \\
Caso 3 & Pmo Rabanal Automatica   & 0 \\
Caso 3 & San Cayetano Autom   & 0 \\
Caso 3 & Pmo Chingaza  & 0 \\
Caso 3 & Ideam Bogota  & 0 \\
Caso 4 & La Capilla Autom  & 5 \\
Caso 4 & Ideam Bogota  & 0 \\
Caso 4 & Chinavita Automatica  & 5 \\
Caso 4 & Tibaitata Automatica  & 5 \\
Caso 4 & Hda Sta Ana Autom  & 5 \\
Caso 4 & Apto El Dorado  & 5 \\
Caso 4 & Sta Cruz De Siecha  & 5 \\
Caso 4 & Pmo Guerrero  & 5 \\
Caso 4 & San Cayetano Autom   & 0 \\
Caso 4 & Univ Nacional  & 5 \\
Caso 4 & Pmo Chingaza  & 1 \\
Caso 4 & Villa Teresa Automatica  & 5 \\
Caso 4 & Pmo Rabanal Automatica   & 1 \\
Caso 4 & Bosque Intervenido    & 0 \\

\end{longtable}

En algunos casos cuando se configuraba el modelo para hacer la simulación este no realizaba los cálculos, ya que para ciertas opciones de la parametrización es necesa

\begin{table}[H]
\caption{Frecuencia de ocurrencia de la combinación parametrización-opción mayor a 0.8 en $Pearson$ e inferior a 0.3 en $NMRSE$ que se encuentra entre los 5 mejores resultados.}
\centering
\label{tabla_mejores_param_opciones}
\begin{tabular}{lr}
\toprule
Combinación parametrización-opción               & Frecuencia       \\
\midrule
bl\_pbl\_physics-1     &      1 \\
bl\_pbl\_physics-5     &      7 \\
bl\_pbl\_physics-6     &      5 \\
bl\_pbl\_physics-7     &     13 \\
bl\_pbl\_physics-8     &     15 \\
bl\_pbl\_physics-9     &      5 \\
bl\_pbl\_physics-12    &      9 \\
bl\_pbl\_physics-99    &      6 \\ \hline
cu\_physics-2         &      9 \\
cu\_physics-3         &      1 \\
cu\_physics-5         &      4 \\
cu\_physics-6         &      7 \\
cu\_physics-11        &      1 \\
cu\_physics-14        &      3 \\
cu\_physics-16        &      4 \\
cu\_physics-99        &      3 \\ \hline
mp\_physics-0         &      8 \\
mp\_physics-2         &      3 \\
mp\_physics-3         &      1 \\
mp\_physics-8         &      1 \\
mp\_physics-11        &      3 \\
mp\_physics-13        &      3 \\
mp\_physics-16        &      1 \\ \hline
ra\_lw\_physics-3      &      6 \\
ra\_lw\_physics-4      &      2 \\
ra\_lw\_physics-5      &      2 \\
ra\_lw\_physics-7      &     13 \\
ra\_lw\_physics-31     &      7 \\ \hline
ra\_sw\_physics-1      &      1 \\
ra\_sw\_physics-2      &     12 \\
ra\_sw\_physics-3      &     14 \\
ra\_sw\_physics-4      &     13 \\
ra\_sw\_physics-5      &      8 \\ \hline
sf\_sfclay\_physics-1  &      2 \\
sf\_sfclay\_physics-91 &     17 \\ \hline
*sf\_surface\_physics-1 &      1 \\
IDEAM-Bogotá         &      1 \\
IDEAM-Colombia       &      4 \\
\bottomrule
\end{tabular}
\end{table}

En la Tabla \ref{tabla_mejores_param_opciones} podemos observar que para las parametrizaciones de capa límite planetaria hay diferentes resultados ya que de las 14 opciones evaluadas se obtuvieron resultados en 8. De estas 8 favorables, dos presentaron valores sobre 10, esto implica que un cambio en esta parametrización produce cambios importantes en los resultados de la temperatura.\\ 

Para la parametrización de cúmulos podemos observar que las frecuencias son inferiores a 10, esto puede implicar que esta parametrización no tenga un papel importante en el momento de la simulación de temperatura extrema.\\

En la parametrización de microfísica de nubes es importante resaltar que, en general, se presentaron números bajos y que no hay un gran efecto de la microfísica en la simulación de las temperaturas extremas pues el mejor resultado fue cuando esta parametrización está desactivada.\\

Aunque en las parametrizaciónes de radiación de onda larga y corta fueron evaluadas 9 combinaciones sólo se obtuvieron 4 combinaciones con buenos resultados respectivamente. Se puede observar que estas buenas combinaciones presentaron valores de frecuencia superiores a diez en 3 de 4 combinaciones, respectivamente. Esto implica que las modificaciones en esta parametrización mejora los resultados de los datos modelados de temperatura.\\


Es interesante que dentro de los mejores resultados aparezca la configuración del IDEAM-Colombia y que sea superior a la configuración IDEAM-Bogotá, esto implica que es una buena configuración y que en algunos casos está representando bien las temperaturas extremas, en comparación con la configuración IDEAM-Bogotá que también aparece en los resultados pero con valores inferiores.\\

Como resultado podemos observar que para la parametrización de la capa límite planetaria (bl\_pbl\_physics) la mejor parametrización es la 8, estas parametrizaciones corresponden a \textit{Bougeault–Lacarrere Scheme (BouLac)}. En la parametrización de cúmulos (cu\_physics) la mejor parametrización es la 2, estas parametrizaciones corresponden a parametrización \textit{Moisture–advection–based Trigger for Kain–Fritsch Cumulus Scheme}, \textit{Grell 3D Ensemble Scheme}. En la microfísica de nubes (mp\_physics) se obtuvo que la mejor es la parametrización 0, esto significa que se debe mantener desactivada la parametrización de microfísica. En la parametrización de la onda larga (ra\_lw\_physics) se observa que la mejor parametrización es la opción 7, la cual corresponde a \textit{Fu–Liou–Gu Shortwave and Longwave Schemes}. En la parametrización de la onda corta (ra\_sw\_physics) se observa que la mejor parametrización es la opción 3 \textit{CAM Shortwave and Longwave Schemes}. En la parametrización de superfície (sf\_sfclay\_physics) la única combinación que presentó buenos resultados fue la opción 91, la cual corresponde a \textit{MM5 Similarity Scheme}. Y para finalizar, la configuración del IDEAM con la configuración para Colombia presentó buenos resultados, esto quiere decir que fue una buena decisión haber tomado la configuración del IDEAM para Colombia y experimentar con ella, ya que presenta buenos resultados.\\

Posteriormente a la obtención de los resultados de las mejores opciones se realizaron dos nuevas configuraciones usando los resultados obtenidos. La primera configuración se realizó sobre la configuración del IDEAM-Colombia, junto con las mejores opciones de la Tabla \ref{tabla_mejores_param_opciones} creándose de la configuración IDEAM-Colombia-modificada (icm). Esta configuración fue evaluada con los resultados obtenidos, ver Tabla \ref{tabla_parametrizacion_opciones_2}.

La segunda configuración usó las mejores combinaciones de la Tabla \ref{tabla_mejores_param_opciones} pero para la parametrización de microfísica de nubes se uso la configuración del IDEAM que es la configuración 3 que corresponde a la WSM3 \textit{WRF Single–moment 3–class and 5–class Schemes}. Es importante dejar activada esta parametrización ya que esta es la que se encarga de resolver el vapor de agua y el proceso de precipitación, estos procesos no se estaban teniendo en cuenta con la opción 0, se espera que los resultados de esta investigación sean usados como insumo para la modelación de cultivos, por esta razón no se puede obviar el vapor de agua y el proceso de precipitación. Adicionalmente es importante resaltar que en las fechas de los casos se presentaron precipitaciones en la Sabana de Bogotá, como se puede observar en la Tabla \ref{tabla_lluvias}. \\

\begin{table}[H]
\caption{Resumen de la precipitación de 96 estaciones convencionales para los diferentes casos. Número de estaciones convencionales que presentaron lluvia para las fechas. Valor promedio de la precipitación de las estaciones que presentaron lluvias. Y valor máximo de la precipitación registrado.}
\label{tabla_lluvias}
\begin{tabular}{llp{4cm}ll}
Caso & Fecha & Número de estaciones con precipitación & Promedio & Valor máximo \\ \hline
	1 & 2007-02-03 & 0  & NaN & NaN \\
	2 & 2007-02-03 & 5  & 0.3 & 0.5 \\
	3 & 2007-02-03 & 23 & 2.5 & 13.5\\
	4 & 2007-02-03 & 19 & 1.1 & 3.8 \\

\end{tabular}
\end{table}

En la Tabla \ref{tabla_lluvias} podemos observar que para en el caso 1 no se presentaron precipitaciones en ninguna de las estaciones convencionales del área de estudio, en el caso 2 podemos observar que se presentó precipitación en 5 estaciones y la precipitación fue de menos de un milímetro, el caso 3 fue el evento en el que más estaciones reportaron precipitaciones, con el promedio más alto y presentó los valores más altos, y en el caso 4 también se presentaron precipitaciones que fueron mayores a 1 milímetro.

%Posterior a estos resultados se procedió a resolver los empates para el caso de la capa límite planetaria y para la parametrización de cúmulos. Para esto se tomaron las mejores opciones para las parametrizaciones que no presentaron empates como: mp\_physics-11, ra\_lw\_physics-7, ra\_sw\_physics-3 y sf\_sfclay\_physics-91; estas fueron modificadas dentro de las opciones de la \texttt{namelist.input} del IDEAM para Colombia, creándose de esta forma la IDEAM-Colombia-modificada (icm).\\


%Para buscar cuál es la mejor de los empates se procedió a hacer 6 modelaciones más con las siguientes combinaciones:

%\begin{itemize}
%    \item icm: bl\_pbl\_physics-5 y cu\_physics-0
%    \item icm\_pbl-5\_cu-5: bl\_pbl\_physics-5 y cu\_physics-5
%    \item icm\_pbl-5\_cu-14: bl\_pbl\_physics-5 y cu\_physics-14
%    \item icm\_pbl-7\_cu-0: bl\_pbl\_physics-7 y cu\_physics-0
%    \item icm\_pbl-7\_cu-5: bl\_pbl\_physics-7 y cu\_physics-5
%    \item icm\_pbl-7\_cu-14: bl\_pbl\_physics-7 y cu\_physics-14
%\end{itemize}

%Luego de la obtención de los resultados estas nuevas modelaciones fueron de nuevo evaluadas con todas las opciones de la Tabla \ref{tabla_parametrizacion_opciones} más las 6 nuevas simulaciones. En resumen se evaluaron 93 formas diferentes de configuración del modelo, ver Tabla \ref{tabla_parametrizacion_opciones_2}. Como resultado de este proceso se obtuvo la tabla \ref{tabla_parametrizacion_opciones_2}.\\

%Se usaron 9 estaciones para el caso 1, en el caso 2 se usaron 13 estaciones, en el caso 3 se usaron 15 estaciones y en el caso 4 se usaron 16 estaciones.\\ %% Las estaciones están en la carpeta Extraccion_dominios/estaciones_usadas_year_XXXX.csv

\begin{table}[H]
\caption{Parametrizaciones y opciones usadas en el WRF versión 9.1.1.}
\label{tabla_parametrizacion_opciones_2}
\begin{tabular}{lp{8cm}}
Parametrización & Opciones \\ \hline
bl\_pbl\_physics & 0, 1, 2, 3, 4, 5, 6, 7, 8, 9, 10, 11, 12, 99 \\ %14
cu\_physics      & 0, 1, 2, 3, 4, 5, 6, 7, 11, 14, 16, 93, 99 \\ %13
mp\_physics      & 0, 1, 2, 3, 5, 6, 7, 8, 9, 10, 11, 13, 14, 15, 16, 19, 30, 32, 50, 51, 52 \\%21
ra\_lw\_phyisics & 0, 1, 2, 3, 4, 5, 7, 31, 99 \\%9
ra\_sw\_phyisics & 0, 1, 2, 3, 4, 5, 7, 31, 99 \\%9 
sf\_sfclay\_physics & 0, 1, 2, 3, 4, 5, 7, 10, 91\\ %9
sf\_surface\_physics & 0, 1, 2, 3, 4, 5, 7, 8, 31, 99\\ % 10
IDEAM-Colombia & \\
IDEAM-Bogota & \\ 
icm & \\ 
icm-mp_physics_3 & \\


\end{tabular}
\end{table}

Como resultado de esta nueva evaluación se obtuvo la Tabla \ref{tabla_mejores_param_opciones_2}.\\
% la que mejores resultados tuvo fué una frecuencia de 12 pero ahora con la modificación la mejor es de 25
\begin{table}[H]
\caption{Frecuencia de ocurrencia de la combinación parametrización-opción mayor a 0.8 en Pearson e inferior a 0.3 en NMRSE que se encuentra entre los 5 mejores resultados. * Sólo se evaluó una opción, porque las demás configuraciones no generaron resultados en el WRF.}
\label{tabla_mejores_param_opciones_2}
\centering
\begin{tabular}{lr}
\toprule
Combinación parametrización-opción               & Frecuencia       \\
\midrule

bl\_pbl\_physics-5     &      3 \\
bl\_pbl\_physics-6     &      1 \\
bl\_pbl\_physics-7     &      4 \\
bl\_pbl\_physics-8     &      6 \\
bl\_pbl\_physics-9     &      4 \\
bl\_pbl\_physics-12    &      7 \\
bl\_pbl\_physics-99    &      4 \\ \hline
cu\_physics-2         &      4 \\
cu\_physics-5         &      2 \\
cu\_physics-6         &      4 \\
cu\_physics-11        &      1 \\
cu\_physics-14        &      2 \\
cu\_physics-16        &      3 \\
cu\_physics-99        &      2 \\ \hline
mp\_physics-0         &      1 \\
mp\_physics-2         &      1 \\
mp\_physics-11        &      2 \\
mp\_physics-13        &      2 \\ \hline
ra\_lw\_physics-3      &      3 \\
ra\_lw\_physics-5      &      1 \\
ra\_lw\_physics-7      &      6 \\
ra\_lw\_physics-31     &      6 \\ \hline
ra\_sw\_physics-2      &      5 \\
ra\_sw\_physics-3      &      6 \\
ra\_sw\_physics-4      &      5 \\
ra\_sw\_physics-5      &      4 \\ \hline
*sf\_sfclay\_physics-91 &      9 \\ \hline
sf\_surface\_physics-1 &      1 \\ \hline
ideam-bogota         &      1 \\
ideam-colombia       &      2 \\
ideam-icm            &     24 \\
ideam-icm\_3          &     19 \\


\bottomrule
\end{tabular}
\end{table}

%icm ideam-mejor
%icm\_pbl-5\_cu-5 ideam-cu\_5
%icm\_pbl-5\_cu-14 ideam-cu\_14\_pbl\_5
%icm\_pbl-7\_cu-0 ideam-mejor7?? 
%icm\_pbl-7\_cu-5 cu\_physics\_5\_5
%icm\_pbl-7\_cu-14 ideam-cu\_14  cu_physics_14


Como se puede observar en la Tabla \ref{tabla_mejores_param_opciones_2}, la nueva configuración usada tiene la más alta frecuencia entre todas las combinaciones parametrización-opción. Cuando se comparan los resultados de la Tabla \ref{tabla_mejores_param_opciones} con la Tabla \ref{tabla_mejores_param_opciones_2}, se puede observar que efectivamente hay una respuesta al cambio de las parametrizaciones, ya que para algunas combinaciones se disminuye la frecuencia, porque la nueva combinación es contada en los 5 mejores resultados por estación.\\

Para el caso de simulaciones de temperaturas extremas la mejor combinación de las parametrizaciones corresponde a la configuración icm cuya configuración del \textit{namelist.input} está en la Figura \ref{imag_tab_ej3}.\\

\begin{figure}[H]
\center
\texttt{
...\\
\&physics\\
mp\_physics~~~~~~~~~~~~~~~=~0,~0,\\
ra\_lw\_physics~~~~~~~~~~~~=~7,~~7,\\
ra\_sw\_physics~~~~~~~~~~~~=~3,~~3,\\
%radt~~~~~~~~~~~~~~~~~~~~~=~30,~30,\\
sf\_sfclay\_physics~~~~~~~~=~91,~91,\\
sf\_surface\_physics~~~~~~~=~1,~~1,\\
bl\_pbl\_physics~~~~~~~~~~~=~8,~~8,\\
bldt~~~~~~~~~~~~~~~~~~~~~~=~1,~~1,\\
cu\_physics~~~~~~~~~~~~~~~~=~2,~~2,\\
...\\
}
\centering
\includegraphics[draft=false, scale=0.0018]{ideam_ej_1.png}
\caption{Sección del \texttt{namelist.input} para la combinación icm.}
\label{imag_tab_ej3}


\end{figure}


\section{Resultados}
%Como conclusión podemos observar que los cambios en las parametrizaciones del WRF no resulta acumulativo, es decir que si se encuentran las mejores opciones para las diferentes parametrizaciones no garantiza que esto se vea reflejado en un mejor resultado.\\

%La configuración usada por el IDEAM para ofrece buenos resultados para la representación de las altas y bajas temperaturas.\\

%El modelo WRF tienen dificultades para representar temperaturas bajo 0\celsius, pero presenta buenos resultados cuando se trabaja con altas temperaturas. El modelo WRF con las configuraciones halladas no es capaz de representar una helada, pero si es capaz de representar un fenómeno de altas temperaturas.\\

El método usado para la selección de la mejore parametrización-opción resultó de utilidad, ya que se obtuvo la mayor frecuencia para la configuración icm. La mejor configuración evaluada para la representación de las altas y bajas temperaturas corresponde a la combinación icm.\\

Las respuestas del modelo se vieron influenciadas por el cambio de cada parametrizaciones. Es difícil determinar en qué proporción se mejoran los resultados dependiendo de la parametrización, pero sí se puede decir que para el caso de las temperaturas extremas las parametrizaciones más importantes son las de radiación, suelo y capa límite planetaria.\\


\end{itemize}
%\end{comment}

%\begin{comment}

\section{Búsqueda del mejor tiempo para iniciar las simulaciones}

El modelo WRF toma un tiempo para la calibración y los ajustes de las condiciones iniciales y frontera \citep{Jimenez2014}. \citet{Skamarock2008} reconoce que se debe dejar un tiempo de 6 horas antes de la hora objetivo, ya que este tiempo permite realizar un ajuste a la topografía. \citet{cortes2010} resalta que los modelos de mesoescala presentan un problema con el arranque inicial el cual está relacionado con el proceso de iniciar los movimientos verticales y las circulaciones divergentes del modelo, por esta razón \citet{cortes2010} descartó las primeras 12 horas de simulación. \citet{Hu2010} eliminó las primeras 12 horas de las simulaciones el cual le pertmitió obtener mejores resultados. \citet{Arango2011} establecieron que un \textit{spin-up} de 6 horas es muy bajo para la determinación de la temperatura, por esta razón proponen un \textit{spin-up} de 10 a 15 horas. \citet{Draxl2007} simularon vientos para un período de tiempo de un mes y usó un \textit{spin-up} de 48 horas. \citet{Corrales2015} no tuvo en cuenta el período de \textit{spin-up} para evaluar las temperaturas en México.\\

Con la finalidad de buscar el mejor tiempo en el cuál se deben iniciar las simulaciones antes del evento se planteó la evaluación de diferentes periodos de tiempos de inicio antes del fenómeno de estudio (ver Tabla \ref{tab_horas_inicio}). Los tiempos de finalización fueron los mismos para cada caso (ver Tabla \ref{tabla_fechas_finalizacion}).\\ 

\begin{table}[h]
\caption{Simulaciones y horas de inicio antes del fenómeno.}
\label{tab_horas_inicio}
\centering
\begin{tabular}{ll}
Simulación & Horas antes del evento \\ \hline
1 & 0 \\
2 & 6 \\
3 & 12 \\
4 & 18 \\
5 & 24 \\
6 & 30 \\
7 & 36 \\
8 & 42 \\
9 & 48 \\
10 & 54 \\
11 & 60 \\
12 & 66 \\
13 & 72 \\
14 & 78 \\
15 & 84 \\
16 & 108
\end{tabular}
\end{table}

Se tomaron los cuatro casos y fueron configurados con la combinación icm las fechas y las horas de inicio se encuentran en la Tabla \ref{tabla_fechas_inicio}. Los datos usados para proveer condiciones iniciales y de frontera corresponden a los datos de análisis del GFS los cuales están disponibles cada 6 horas. La fecha y hora de finalización no cambió, las fechas se encuentran en la Tabla \ref{tabla_fechas_finalizacion}.

\begin{table}[H]
\caption{Fechas de inicio para los 4 diferentes casos y las simulaciones realizadas}
\label{tabla_fechas_inicio}
\begin{tabular}{lllll}
 Simulación & Caso 1 & Caso 2 & Caso3 & Caso 4 \\ \hline
 1 & 2007-02-04 06:00 & 2014-08-30 06:00 & 2015-08-27  06:00 & 2015-09-08  06:00\\
 2 & 2007-02-04 00:00 & 2014-08-30 00:00 & 2015-08-27  00:00 & 2015-09-08  00:00\\
 3 & 2007-02-03 18:00 & 2014-08-29 18:00 & 2015-08-26  18:00 & 2015-09-07  18:00\\
 4 & 2007-02-03 12:00 & 2014-08-29 12:00 & 2015-08-26  12:00 & 2015-09-07  12:00\\
 5 & 2007-02-03 06:00 & 2014-08-29 06:00 & 2015-08-26  06:00 & 2015-09-07  06:00\\
 6 & 2007-02-03 00:00 & 2014-08-29 00:00 & 2015-08-26  00:00 & 2015-09-07  00:00\\
 7 & 2007-02-02 18:00 & 2014-08-28 18:00 & 2015-08-25  18:00 & 2015-09-06  18:00\\
 8 & 2007-02-02 12:00 & 2014-08-28 12:00 & 2015-08-25  12:00 & 2015-09-06  12:00\\
 9 & 2007-02-02 06:00 & 2014-08-28 06:00 & 2015-08-25  06:00 & 2015-09-06  06:00\\
10 & 2007-02-02 00:00 & 2014-08-28 00:00 & 2015-08-25  00:00 & 2015-09-06  00:00\\
11 & 2007-02-01 18:00 & 2014-08-27 18:00 & 2015-08-24  18:00 & 2015-09-05  18:00\\
12 & 2007-02-01 12:00 & 2014-08-27 12:00 & 2015-08-24  12:00 & 2015-09-05  12:00\\
13 & 2007-02-01 06:00 & 2014-08-27 06:00 & 2015-08-24  06:00 & 2015-09-05  06:00\\
14 & 2007-02-01 00:00 & 2014-08-27 00:00 & 2015-08-24  00:00 & 2015-09-05  00:00\\
15 & 2007-01-31 18:00 & 2014-08-26 18:00 & 2015-08-23  18:00 & 2015-09-04  18:00\\
16 & 2007-01-30 18:00 & 2014-08-25 18:00 & 2015-08-22  18:00 & 2015-09-03  18:00\\
\end{tabular}
\end{table}

\begin{table}[H]
\caption{Fechas de finalización para los 4 diferentes casos}
\label{tabla_fechas_finalizacion}
\begin{tabular}{lllll}
  & Caso 1 & Caso 2 & Caso3 & Caso 4 \\ \hline
 Fin & 2007-02-05 00:00 & 2014-09-01 00:00 & 2015-08-30 00:00 & 2015-09-10 00:00 \\
 
\end{tabular}
\end{table}



Como se vio en anteriores secciones (Caracterización de las heladas y altas temperaturas (Sección \ref{area_caracterizacion_heladas_extremas})), las horas más probables de ocurrencia de una helada son entre la 1 am y las 6 am y las horas más probables para que se presenten altas temperaturas son entre las 11 am y las 2 pm. Por esta razón las modelaciones tomaron como la hora 0 las 6 horas UTC del día en el que se presentó el evento, esto equivale a inicial la modelación a la 1 a.m. (hora local ), la cual corresponde a la simulación 1, ver Tabla \ref{tab_horas_inicio}. Las diferentes horas y el tiempo de cómputo usado para estos cálculos se encuentran en la Tabla \ref{tabla:tiempos_inicializacion}.\\

Para encontrar los mejores resultados se usó la misma metodología que se usó para evaluar los modelos. Por lo tanto se realizó un conteo de la cantidad de veces que la combinación simulación fecha presentó resultados superiores a 0.8 en la columna $ET$ (ver Anexo \ref{anexo:resultados_comparaciones_estadisticos_tiempos}).\\

\begin{table}[]
\centering
\caption{Simulaciones, tiempo entre el caso de estudio y el inicio de la simulación (Tiempo (h)) y tiempo usado para computar cada una de las simulaciones (Tiempo de computo horas).}
\label{tabla:tiempos_inicializacion}
\begin{tabular}{llll}
Simulación    & Tiempo (h)    & Tiempo de computo (horas) \\ \hline
  1   &  0            &     2           \\
  2   &  6            &     3.33           \\
  3   &  12           &     3           \\
  4   &  18           &     4           \\ 
  5   &  24           &     4           \\ 
  6   &  30           &     5           \\ 
  7   &  36           &     6.3          \\ 
  8  &  42           &     6.5          \\ 
 9   &  48           &     7.33          \\
 10   &  60           &     9          \\
  11  &  54           &     7          \\
   12 & 66           &     8.5          \\
   13 &  72         &     9          \\
  14  &  78         &     9.5          \\
  15  &  84           &     10           \\
  16  &  108          &     13           \\


            


 
\end{tabular}
\end{table}



Para la selección de las mejores combinaciones simulación-estación se calculó el índice $ET$ a cada uno de los cuatro casos. Fueron descartadas aquellas combinaciones que tuvieron valores inferiores a 0.8 en el índice $ET$, de las combinaciones restantes se seleccionaron las 5 que tuvieron los valores más altos en el índice $ET$ en cada estación para cada uno de los casos. La Tabla \ref{tabla:resultado_tiempo} muestra la frecuencia de ocurrencia de las combinaciones parametrización-opción que son mayores a 0.8 en el índice $ET$ y que se encuentran en las 5 mejores por estación.\\


\begin{longtable}{lll}

\caption{Conteo de las mejores nueve simulaciones para 10 estaciones. La simulación ideal tendría un valor de frecuencia igual a 10 que sería el número de combinaciones simulaciones-dominios (ver Tabla \ref{tab:estado_hydras}).}
\label{tab_estaciones_5_tiempo}\\
\hline
Caso & Estación & Valores \\

\midrule
\endhead
\midrule
\multicolumn{3}{r}{{Continúa en la siguiente página.}} \\
\midrule
\endfoot

\bottomrule
\endlastfoot


Caso 1 & La Capilla Autom  & 5\\
Caso 1 & Ciudad Bolivar  & 5\\
Caso 1 & Hda Sta Ana Autom  & 5\\
Caso 1 & La Boyera Automatica  & 5\\
Caso 1 & Chinavita Automatica  & 5\\
Caso 1 & Subia Automatica  & 5\\
Caso 1 & Tibaitata Automatica  & 5\\
Caso 1 & Sta Cruz De Siecha  & 5\\
Caso 1 & Pmo Guacheneque  & 5\\
Caso 2 & Hda Sta Ana Autom  & 5\\
Caso 2 & Esc La Union Automatica & 5\\
Caso 2 & Apto El Dorado  & 5\\
Caso 2 & La Capilla Autom  & 5\\
Caso 2 & Chinavita Automatica  & 5\\
Caso 2 & La Boyera Automatica  & 5\\
Caso 2 & Sta Cruz De Siecha  & 0\\
Caso 2 & Univ Nacional  & 5\\
Caso 2 & Bosque Intervenido    & 5\\
Caso 2 & Pmo Guerrero  & 5\\
Caso 2 & Pnn Chingaza Autom   & 5\\
Caso 2 & Pmo Guacheneque  & 5\\
Caso 2 & Ideam Bogota  & 0\\
Caso 2 & San Cayetano Autom   & 0\\
Caso 3 & Tibaitata Automatica  & 5\\
Caso 3 & Sta Cruz De Siecha  & 5\\
Caso 3 & Hda Sta Ana Autom  & 5\\
Caso 3 & La Capilla Autom  & 5\\
Caso 3 & Chinavita Automatica  & 5\\
Caso 3 & Univ Nacional  & 5\\
Caso 3 & Pmo Chingaza  & 0\\
Caso 3 & Apto El Dorado  & 5\\
Caso 3 & Nueva Generacion  & 5\\
Caso 3 & Esc La Union Automatica & 5\\
Caso 3 & Ideam Bogota  & 0\\
Caso 3 & Pmo Rabanal Automatica   & 0\\
Caso 3 & Villa Teresa Automatica  & 5\\
Caso 3 & Pmo Guerrero  & 5\\
Caso 3 & San Cayetano Autom   & 5\\
Caso 4 & La Capilla Autom  & 5\\
Caso 4 & Tibaitata Automatica  & 5\\
Caso 4 & Bosque Intervenido    & 2\\
Caso 4 & Chinavita Automatica  & 5\\
Caso 4 & Pmo Rabanal Automatica   & 5\\
Caso 4 & Sta Cruz De Siecha  & 5\\
Caso 4 & Apto El Dorado  & 5\\
Caso 4 & Ideam Bogota  & 0\\
Caso 4 & San Cayetano Autom   & 5\\
Caso 4 & Nueva Generacion  & 5\\
Caso 4 & Villa Teresa Automatica  & 5\\
Caso 4 & Pmo Chingaza  & 0\\
Caso 4 & Univ Nacional  & 5\\
Caso 4 & Pmo Guerrero  & 5\\


\end{longtable}






\begin{table}
    \centering
    \caption{Frecuencia de ocurrencia de la combinación parametrización-opción mayor a 0.8 en el índice $ET$ y que se encuentra entre los 5 mejores resultados.}
    \label{tabla:resultado_tiempo}
\begin{tabular}{rlrr}
\toprule
 Simulación &  Frecuencia \\
\midrule
Simulación 1  &     43 \\
Simulación 2  &     33 \\
Simulación 3  &     41 \\
Simulación 4  &     18 \\
Simulación 5  &     14 \\
Simulación 6  &     11 \\
Simulación 7  &     15 \\
Simulación 8  &      8 \\
Simulación 9  &      5 \\
Simulación 10 &      2 \\
Simulación 11 &     10 \\
Simulación 12 &     14 \\
Simulación 13 &      5 \\
Simulación 14 &      1 \\
Simulación 15 &      6 \\
Simulación 16 &     11 \\
\bottomrule
\end{tabular}
\end{table}
%\justify

Basado en los resultados de la Tabla \ref{tabla:resultado_tiempo} se puede observar que la simulación 1 y 3 presentaron los mejores resultados ya que tienen una frecuencia de 37. Estas simulaciones corresponden a las horas de inicio 6 UTC del día del evento y 18 UTC del día anterior al evento. Adicionalmente, se realizaron los diagramas de Taylor para la estación Tibaitatá en los diferentes casos (Figura \ref{fig:taylor_tiempo}) las demás gráficas se encuentran en el Anexo \ref{anexo:graficas_taylor_tiempos_4casos}.

\begin{figure}[H]
    \centering
    
\begin{subfigure}[normla]{0.4\textwidth}
\includegraphics[draft=false, scale=0.25]{graficas_taylor_tiempo_casos/taylor_20070221206990.png}
\caption{Estación Tibaitata Automatica  código 21206990 caso 1.}
\end{subfigure}
~
\begin{subfigure}[normla]{0.4\textwidth}
\includegraphics[draft=false, scale=0.25]{graficas_taylor_tiempo_casos/taylor_20140821206990.png}
\caption{Estación Tibaitata Automatica  código 21206990 caso 2.}
\end{subfigure}
~
\begin{subfigure}[normla]{0.4\textwidth}
\includegraphics[draft=false, scale=0.25]{graficas_taylor_tiempo_casos/taylor_20150821206990.png}
\caption{Estación Tibaitata Automatica  código 21206990 caso 3.}
\end{subfigure}
~
\begin{subfigure}[normla]{0.4\textwidth}
\includegraphics[draft=false, scale=0.25]{graficas_taylor_tiempo_casos/taylor_20150921206990.png}
\caption{Estación Tibaitata Automatica  código 21206990 caso 4.}
\end{subfigure}
    
    \caption{Diagramas de Taylor para la estación Tibaitatá para los cuatro casos evaluados.}
    \label{fig:taylor_tiempo}
\end{figure}

En la Figura \ref{fig:taylor_tiempo} se observa una buena dispersión de los datos. En estas gráficas podemos observar como los números pequeños se encuentran más cerca a los datos reales, esto quiere decir que los resultados resultan mejores si el inicio de la modelación se encuentra más cerca a la fecha de interés, ya que como lo nombra \citet{Kovacik2007} los modelos necesitan cierto tiempo para lograr un equilibrio térmico e hidrológico entre el suelo y la atmósfera. Por esta razón teniendo en cuenta la bibliografía citada el mejor tiempo para iniciar el modelo es 12 horas antes del fenómeno lo que equivale a la simulación 3.

\section{Conclusiones}

El sistema desarrollado para la búsqueda de las mejores parametrizaciones presentó buenos resultados para buscar las mejores parametrizaciones. Como resultado se obtuvo que las mejores parametrizaciones físicas corresponde a:

\begin{itemize}
    \item Capa límite planetaria: \textit{Mellor–Yamada Nakanishi Niino (MYNN) Level 2.5 and Level 3 Schemes}
    \item Cúmulos: Parametrización desactivada
    \item Parametrización de onda larga: \textit{Fu–Liou–Gu Shortwave and Longwave Schemes}
    \item Parametrización de onda corta: \textit{CAM Shortwave and Longwave Schemes}
    \item Microfísica: Parametrización desactivada
\end{itemize}

La simulación 3 que inicia 12 horas antes del evento es la mejor opción para iniciar el modelo. Se presentó un empate con 0 horas antes del inicio del fenómeno, pero la literatura apoya la idea que el modelo debe tener un tiempo en el que se debe estabilizar para posteriormente generar mejores resultados.

%\end{comment}

\section{Descripción de algunas variables en días de las altas y bajas temperaturas a nivel horario}

%Se tomó la configuración icm para cada uno de los casos y se realizó una comparación visual de los resultados de las simulaciones contra los datos reales de las estaciones. Estos datos fueron comparados a nivel horario. La comparación se realizó para los días de los casos descritos anteriormente. Para explicar el comportamiento de las estaciones se presentará la estación automática Tibaitatá. 

%A nivel horario se ve en la figura \ref{subfig:temp_total} que entre las 0 am y las 6 am se presentan las horas más bajas, %posteriormente se observa un aumento de la temperatura el cuál se encuentra relacionado con el amanecer. Durante el día podemos %observar que la más alta temperatura se encuentra entre las 12 m y las 2 pm. Posteriormente se observa un descenso de la %temperatura que inicia a las 6 pm.\\
%
%A partir de los todos los datos de temperatura de la estación Tibaitatá se seleccionaron los días en los que se presentaron %heladas, estos días fueron seleccionados y gaficados en la figura \ref{subfig:temp_min}. En esta figura se puede observar que %las temperaturas bajo 0\celsius se comienzan a presentar desde la 1 am y que las temperaturas más bajas se presentaron entre %las 6 y 6:30 am. Luego se presentan altas temperaturas, casi que en todos los casos las temperaturas son iguales o superiores a %17\celsius y en una gran proporción es superiór a 20\celsius. Luego se observa que la temperatura desciende rápidamente. Se %observan datos atípicos en la figura \ref{subfig:temp_min} entre las 2:00 pm y las 5:00 pm.\\
%
%Se extrajeron los datos de las altas temperaturas  y se graficaron horariamente esto se aprecia en la figura %\ref{subfig:temp_max}. Si comparamos esta figura con la figura de las temperaturas bajas \ref{subfig:temp_min} podemos observar %que hay una mayor dispersión de los datos cuando hay temperaturas altas. Esto se aprecia porque enntre las 0 am y las 6 am se %presentan temperaturas sobre 9\celsius. Según la figura se puede concluír que las altas temperaturas se pueden presentar desde %las 9 am hasta las 4 pm. Se observan dos valores que presentan altas temperaturas muy altas, estos datos se deben analizar %puntualmente ya que se puede tratar de un error en los sensores. Adicionalmente, se observa que los días que presentaron altas %temperaturas se pueden presentar temperaturas bajo 0\celsius.

%\begin{figure}
%    \centering
%    \begin{subfigure}[b]{0.35\textwidth}
%	\includegraphics[draft=false, scale=0.3]{grafica_var_dia/21206990_1.png}
%    \caption{Gráfica todos los datos de temperatura.}
%    \label{subfig:temp_total}
%	\end{subfigure}
%	~
%    \begin{subfigure}[b]{0.35\textwidth}
%	\includegraphics[draft=false, scale=0.3]{grafica_var_dia/21206990_3.png}
%    \caption{Gráfica de los datos en un día de helada.}
%    \label{subfig:temp_min}
%	\end{subfigure}
%		~
%    \begin{subfigure}[b]{0.35\textwidth}
%	\includegraphics[draft=false, scale=0.3]{grafica_var_dia/21206990_5.png}
%    \caption{Gráfica de los datos en un día de altas temperaturas.}
%    \label{subfig:temp_max}
%	\end{subfigure}
%	
%    \includegraphics{}
%    \caption{Gráficas de temperatura del aire de la estación Tibaitatá a nivel horario.}
%    \label{fig:temp_tibaitata}
%\end{figure}

Con la finalidad de hacer una caracterización de diferentes variables a nivel horario se usaron los datos de la estación Tibaitatá, se marcaron los días en los que se presentaron heladas y altas temperaturas. A partir de estos datos se realizaron 3 gráficas en las cuales encontramos todos los valores, los días que se presentaron heladas y los días que hubo altas temperaturas. Los datos de la variable temperatura fueron analizados anteriormente (Sección \ref{area_caracterizacion_heladas_extremas}).\\

La humedad relativa para la estación Tibaitatá presenta altos valores desde las 0 am hasta las 6 am, posteriormente se observa una ampliación de los rangos de humedad relativa con una tendencia a la disminución, figura \ref{subfig:hum_total}. Los valores más bajo de humedad relativa se encuentra a la 1 pm, luego de esta hora los valores de humedad relativa van en aumento. Cuando se presentan las heladas se puede observar que la humedad alcanza valores más bajos a las 12 m., Figura \ref{subfig:hum_min}. Entre las 4 am y las 6 am se observa que la humedad relativa es alta, luego se presenta un descenso. En los días de altas temperaturas se puede observar una mayor dispersión en los valores de humedad Figura \ref{subfig:hum_max} en comparación con los días en los que se presentaron heladas. En algunos casos la humedad relativa es menor a 20\%. Luego de las 2 pm la humedad relativa presenta un aumento continuo. En el caso de las altas temperaturas la humedad presenta un mayor rango de valores.\\


\begin{figure}[H]
    \centering
    \begin{subfigure}[b]{0.45\textwidth}
       \caption{Valores de humedad relativa para todos los días, se usaron 96939 datos.}
	\includegraphics[draft=false, scale=0.4]{grafica_var_dia/21206990_hum_2m_1.png}
     \label{subfig:hum_total}
	\end{subfigure}
	~
    \begin{subfigure}[b]{0.45\textwidth}
       \caption{Humedad relativa para días con heladas, se usaron 546 datos.}
	\includegraphics[draft=false, scale=0.4]{grafica_var_dia/21206990_hum_2m_3.png}
     \label{subfig:hum_min}
	\end{subfigure}
		~
    \begin{subfigure}[b]{0.45\textwidth}
       \caption{Humedad relativa para días con altas temperaturas, se usaron 625 datos.}
	\includegraphics[draft=false, scale=0.4]{grafica_var_dia/21206990_hum_2m_5.png}
     \label{subfig:hum_max}
	\end{subfigure}
	
%    \includegraphics{}
    \caption{Humedad relativa para la estación Tibaitatá a nivel horario, entre el año 2007 hasta el año 2018.}
    \label{fig:hr_tibaitata}
\end{figure}


La radiación desde las 0 am hasta las 5 am presenta los mas bajos valores, posterior a estas horas se presenta un aumento de los valores de radiación como se ve en la Figura \ref{subfig:rad_total}. La hora en la que se presenta la mayor radiación es a las 12 m, luego de esta hora se presenta una disminución de los valores hasta las 6 pm, este es un comportamiento normal que se encuentra condicionado por el amancer y el atardecer.\\

Los valores de radiación para los días que hubo heladas se observa que los primero valores de radiación son registrados a las 7 am, después de esta hora se presenta un rápido aumento en los valores de la radiación, figura \ref{subfig:rad_min}. Adicionalmente a las 9, 10 y 11 am no se observan valores inferiores a 350$w/m^2$ antes de las 11 a.m., esto implica que los días de heladas son días despejados lo que concuerda con lo dicho por \citet{Snyder2005}.\\

Cuando hay altas temperaturas podemos observar que los valores de radiación estuvieron más dispersos y alcanzaron valores más altos de radiación Figura \ref{subfig:rad_max} en comparación con la Figura \ref{subfig:rad_min} en la cual los máximos valores son graficados como valores fuera del diagrama de cajas y bigotes. Adicionalmente se observa en la Figura \ref{subfig:rad_min} que las 9, 10 y 11 a.m. sí se presentaron valores menores a 350$w/m^2$, comparado con la Figura \ref{subfig:rad_max} en la cual no se presentaron, esto implica que los días con altas temperaturas se puede presentar cierta cobertura nubosa.\\

Es importante notar que la cantidad de eventos de heladas es menor que los eventos de altas temperaturas. \\

\begin{figure}[H]
    \centering
    \begin{subfigure}[b]{0.45\textwidth}
    \caption{Valores de radiación para todos los días evaluados, se usaron 74333 datos.}
	\includegraphics[draft=false, scale=0.4]{grafica_var_dia/21206990_rad_1_1.png}
    \label{subfig:rad_total}
	\end{subfigure}
	~
    \begin{subfigure}[b]{0.45\textwidth}
    \caption{Radiación para días con heladas, se usaron 406 datos.}
	\includegraphics[draft=false, scale=0.4]{grafica_var_dia/21206990_rad_1_3.png}
    \label{subfig:rad_min}
	\end{subfigure}
		~
    \begin{subfigure}[b]{0.45\textwidth}
    \caption{Radiación para días con altas temperaturas, se usaron 411 datos.}
	\includegraphics[draft=false, scale=0.4]{grafica_var_dia/21206990_rad_1_5.png}
    \label{subfig:rad_max}
	\end{subfigure}
	
%    \includegraphics{}
	\caption{Radiación para la estación Tibaitatá a nivel horario, entre el año 2007 hasta el año 2018. La radiación medida en la estación automática es radiación de onda corta.}
    \label{fig:rad_tibaitata}
\end{figure}

%Hablar sobre la precipitación 

La estación automática HYDRAS reporta el valor promedio de la precipitación cada 10 minutos. Para poder realizar una comparación se sumaron todos los valores de precipitación cada hora, estos valores son horarios y no son acumulativos. La precipitación fue sometida a el proceso de control de calidad. Es importante notar que en Bogotá una de las precipitaciones más altas registradas por el Observatorio Ambiental fue de 56.4 mm en una hora, registrado en la estación El Bosque en el año 2010 \ref{OAB2011}.\\

La precipitación horaria en esta estación se presenta principalmente entre las 11 a.m. y las 6 p.m., aunque también se presentan precipitaciones con menor intensidad y frecuencia entre las 6 pm y las 11 pm, (Figura \ref{subfig:prec_total}). El día que se presentaron heladas no hubo valores de precipitación, ya que los valores registrados fueron menores a 1 mm, figura \ref{subfig:prec_min}, esto implica que para estos días no se presentaron precipitaciones. En el caso de los días con altas temperaturas no se presentan precipitaciones desde las 0 am hasta las 3 pm, pero luego de estas horas en algunos casos se presentaron precipitaciones. Estas precipitaciones pueden ser el resultado de las altas temperaturas como lo nombra \citet{Fischer2015} en su artículo.

\begin{figure}[H]
    \centering
    \begin{subfigure}[b]{0.45\textwidth}
    \caption{Valores de precipitación para todos los días, se usaron 603002 datos.}
	\includegraphics[draft=false, scale=0.4]{grafica_var_dia/21206990_precip_1_1.png}
    \label{subfig:prec_total}
	\end{subfigure}
	~
    \begin{subfigure}[b]{0.45\textwidth}
    \caption{Precipitación para días con heladas, se usaron 3342 datos.}
	\includegraphics[draft=false, scale=0.4]{grafica_var_dia/21206990_precip_1_3.png}
    \label{subfig:prec_min}
	\end{subfigure}
		~
    \begin{subfigure}[b]{0.45\textwidth}
    \caption{Precipitación para días con altas temperaturas, se usaron 4107 datos.}
	\includegraphics[draft=false, scale=0.4]{grafica_var_dia/21206990_precip_1_5.png}
    \label{subfig:prec_max}
	\end{subfigure}
	
%    \includegraphics{}
    \caption{Precipitación para la estación Tibaitatá a nivel horario, entre el año 2007 hasta el año 2018.}
%    \label{fig:temp_tibaitata}
\end{figure}



%%%%%%% Hablar sobre el bulbo húmedo

El bulbo húmedo presenta valores que oscilan entre los 10\celsius\ entre las 0 y las 6, posteriormente se presenta un aumento logrando los valores máximos entre las 13 y 14 horas, luego se evidencia una disminución de los valores, Figura \ref{subfig:wb_total}. En los días que se presentaron heladas se evidencia un rápido salto entre las 7 y las 9 am, posterior a estas horas alcanza los valores máximos entre las 13 y las 14 horas, Figura \ref{subfig:wb_min}. Cuando hay altas temperaturas se puede observar que las temperaturas de bulbo húmedo en la mañana no son tan bajas como en los días de heladas, pero se presenta un rápido cambio entre las 7 y las 8 am, Figura \ref{subfig:wb_max}. En estas gráficas podemos observar que la temperatura de bulbo húmedo para los días de helada son bajos en la mañana al igual que en la noche, para las altas temperaturas se observa que las temperaturas en la mañana. En la noche no descienden tanto en comparación con los días de helada.

\begin{figure}[H]
    \centering
    \begin{subfigure}[b]{0.45\textwidth}
    \caption{Valores de bulbo húmedo para todos los días, se usaron 69945 datos.}
	\includegraphics[draft=false, scale=0.4]{grafica_var_dia/21206990_wb_1.png}
    \label{subfig:wb_total}
	\end{subfigure}
	~
    \begin{subfigure}[b]{0.45\textwidth}
    \caption{Valores de bulbo húmedo para días con heladas, se usaron 371 datos.}
	\includegraphics[draft=false, scale=0.4]{grafica_var_dia/21206990_wb_3.png}
    \label{subfig:wb_min}
	\end{subfigure}
		~
    \begin{subfigure}[b]{0.45\textwidth}
    \caption{Valores de bulbo húmedo para días con altas temperaturas, se usaron 455 datos.}
	\includegraphics[draft=false, scale=0.4]{grafica_var_dia/21206990_wb_5.png}
    \label{subfig:wb_max}
	\end{subfigure}
	
%    \includegraphics{}
    \caption{Bulbo húmedo para la estación Tibaitatá a nivel horario, entre el año 2007 hasta el año 2018.}
%    \label{fig:temp_tibaitata}
\end{figure}

%%%% Hablar sobre el punto de rocío

La temperatura de punto de rocío tiene un comportamiento similar al presentado para el bulbo húmedo, ya que hay bajas temperaturas entre 0 y 6 horas, posteriormente se presenta un aumento hasta las 14 horas y luego se presenta un descenso de la temperatura, Figura \ref{subfig:td_min}. En los días que se presentaron heladas podemos observar que hay bajas temperaturas en las horas de la mañana y se presenta un cambio brusco en la temperatura entre las 7 y 9 horas, posterior hay un aumento en los valores de temperatura cuyos valores más altos se alcanzan entre las 14 y 15 horas, Figura \ref{subfig:td_min}. Cuando se presentaron altas temperaturas podemos observar que la media de la temperatura no fue tan baja como la que se presentó los días con heladas, \ref{subfig:td_max}.

\begin{figure}[H]
    \centering
    \begin{subfigure}[b]{0.45\textwidth}
    \caption{Punto de rocío para todos los días, se usaron 69945 datos.}
	\includegraphics[draft=false, scale=0.4]{grafica_var_dia/21206990_Td_1.png}
    \label{subfig:td_total}
	\end{subfigure}
	~
    \begin{subfigure}[b]{0.45\textwidth}
    \caption{Punto de rocío para días con heladas, se usaron 371 datos.}
	\includegraphics[draft=false, scale=0.4]{grafica_var_dia/21206990_Td_3.png}
    \label{subfig:td_min}
	\end{subfigure}
		~
    \begin{subfigure}[b]{0.45\textwidth}
    \caption{Punto de rocío para días con altas temperaturas, se usaron 455 datos.}
	\includegraphics[draft=false, scale=0.4]{grafica_var_dia/21206990_Td_5.png}
    \label{subfig:td_max}
	\end{subfigure}
	
%    \includegraphics{}
    \caption{Punto de rocío para la estación Tibaitatá a nivel horario, entre el año 2007 hasta el año 2018.}
%    \label{fig:temp_tibaitata}
\end{figure}


\section{Comparación de temperaturas simuladas y observadas}

Para evaluar el comportamiento de las nuevas configuraciones del modelo WRF se realizaron gráficas que permitan comparar los datos de una estación automática, los datos de la configuración usada por el IDEAM para Colombia llamada IDEAM-Colombia y las nuevas configuraciónes llamadas icm e icm-mp\_physics 3. La comparación se realizó para los cuatro casos descritos anteriormente, y para la estación automática Tibaitatá. Se escogió esta estación porque esta estación es una de las estaciones de referencia en la Sabana de Bogotá para el IDEAM.\\

Como resultado de las gráficas podemos observar que las simulaciones realizadas por WRF para el caso 1 (Figura \ref{caso1_tiba_wrf}) no reprodujeron éxitosamente las temperaturas las temperaturas más bajas y más altas registradas por la estación. De las tres configuraciones del modelo evaluadas, la que presentó los resultados menos lejanos a las observaciones fue la configuración icm.\\

Para el caso 2 (Figura \ref{caso2_tiba_wrf}), se puede observar que los datos modelados presentan un mejor ajuste con los datos observados, excepto en las temperaturas diarias más bajas. Para el día 31 de agosto, la temperatura mínima del modelo fue mas baja que la observada, las tres configuraciones presentaron resultados similares.\\


En el caso 3 Figura \ref{caso3_tiba_wrf} podemos observar que los datos de la estación automática presenta varios cambios repentinos en la temperatura antes de las 6 am (hora local). En cuanto a las temperaturas mínimas de las simulaciones para el 28 de agosto, la configuración IDEAM-Colombia fue la que presentó resultados menos lejanos con una diferencia de 2.5\celsius. En el día 28 de agosto la temperatura máxima registrada por la estación automática presentó un pico de temperatura, las temperaturas modeladas no reprodujeron este pico. Las temperaturas mínimas del día 29 de agosto fueron sobre estimadas en promedio 4\celsius y la configuración que presentó valores menos lejanos fue IDEAM-Colombia. El valor máximo para el día 29 tuvo una buena aproximación para la configuración icm e icm-mp\_physics 3.\\

En el caso 4 (Figura \ref{caso4_tiba_wrf}) podemos observar que los datos de temperatura de la estación automática no presentan tanta variación como en el caso 3. Se puede observar que para el día 8 de septiembre la temperatura mínima modelada de las combinaciónes icm e icm-mp\_physics 3 para el dominio 2 presentaron los valores menos lejanos a la temperatura registrada por la estación 1.75\celsius. Por otra parte, fueron las combinaciónes icm e icm-mp\_physics 3 las que presentaron la mejor aproximación para las altas temperaturas del mismo día. El día 9 de septiembre todas las simulaciones produjeron valores de temperatura más bajos que los valores de la estación en promedio 2.5\celsius y las configuraciones que presentaron los valores menos lejanos fueron IDEAM-Colombia e icm-physics 3. Las altas temperaturas para el día 9 de septiembre que presentaron los resultados menos lejanos fue la combinación icm que tuvo una diferencia de 1\celsius.\\

\begin{figure}[H]
    
\begin{subfigure}[normla]{0.4\textwidth}
\caption{Caso 1 helada del 4 de febrero de 2007.}
%"comparacion_real_wrf_200702.py"
\label{caso1_tiba_wrf}
\includegraphics[draft=false, scale=0.4]{comparacion_grafica/200702_21206990.png}
\end{subfigure}
~
\begin{subfigure}[normla]{0.4\textwidth}
\caption{Caso 2 helada del 30 de octubre de 2014.}
\label{caso2_tiba_wrf}
\includegraphics[draft=false, scale=0.4]{comparacion_grafica/201408_21206990.png}
\end{subfigure}
~
\centering
\begin{subfigure}[normla]{0.4\textwidth}
\caption{Caso 3 altas temperaturas para el 28 de octubre de 2015.}
\label{caso3_tiba_wrf}
\includegraphics[draft=false, scale=0.4]{comparacion_grafica/201508_21206990.png}
\end{subfigure}
~
\centering
\begin{subfigure}[normla]{0.4\textwidth}
\caption{Caso 4 altas temperaturas para el 8 de septiembre de 2015.}
\label{caso4_tiba_wrf}
\includegraphics[draft=false, scale=0.4]{comparacion_grafica/201509_21206990.png}
\end{subfigure}
~

    \caption{Temperatura del aire a dos metros para los cuatro casos escogidos de acuerdo a los valores registrados en la estación automática Tibaitatá y los valores simulados con las configuraciones ideam-Colombia, icm e icm-mp\_physics 3.}
    \label{fig:wrf_temp_tibaitata}

\end{figure}



La configuración icm presenta, en general valores mínimos más bajos y máximos más altos en comparación con los datos la simulación ideam-Colombia r icm-mp\_physics 3. Si comparamos la configuración icm y icm-mp\_physics 3 podemos observar que la configuración icm presentó los valores mínimos más bajos. Esto implica que si se quiere simular valores extremos de temperatura la combinación icm presenta mejores resultados.\\

Comparando los resultados de las simulaciones frente a los datos reales se puede apreciar que la configuración icm e icm-mp\_physics 3 presenta un mejor ajuste a las temperaturas altas en comparación con las temperaturas mínimas.\\


\section{Comparación de otras variables}

Además del análisis realizado para la variable de temperatura del aire se evaluó cuál fue el desempeño de la combinación icm, icm-mp\_physics 3 e IDEAM-Colombia con respecto a otras variables medidas por las estaciones automáticas de la zona de estudio tales como precipitación, humedad relativa, radiación, bulbo húmedo, punto de rocío y velocidad del viento. Se va a mostrar los datos de la estación Tibaitatá, ya que es una estación de referencia para el IDEAM, las gráficas de las demás estaciones se encuentran en el Anexo \ref{anexo:graficas_otras_variables_wrf}.\\

\subsection{Humedad relativa}


En el caso 1, la noche entre el 3 y 4 de febrero de 2007, podemos observar que los valores de humedad según la estación automática estuvieron cercanos a 100\% por un periodo cercano a 8 horas. Posteriormente, la humedad presentó un rápido descenso el día 4 de febrero (Figura \ref{caso1_tiba_wrf_hum}). Los datos modelados lograron reproducir el valor máximo de humedad relativa, pero el modelo no fue capaz de reproducir la duración del período en el que la humedad estuvo cercana al 100\%. Las tres configuraciones no fueron capaces de reproducir la humedad mínima. La configuración icm e icm-mp\_physics 3 presentaron un comportamiento similar, ya que el dominio 1 presentó valores superiores al 100\% y fue el dominio 2 el que alcanzó valores de 100\% y no excedió este valor.\\

En el caso 2 (Figura \ref{caso2_tiba_wrf_hum}) podemos observar un mejor ajuste entre el día 29 de octubre a las 12 y el 30 de octubre a las 12 (Hora local). Los datos reportados por la estación automática muestran que entre el día 29 y 30 de octubre de 2014 la humedad llegó hasta 100\%. En el siguiente día la humedad llegó hasta 90\% (Figura \ref{caso2_tiba_wrf_hum}). Las configuraciones icm-mp\_physics 3 e IDEAM-Colombia presentaron los resultados menos lejanos hasta las 4 pm del 30 de agosto de 2014, ya que posteriormente presentó una sobreestimación. La noche entre el 30 y 31 no presentó un buen ajuste ya que la humedad fue sobreestimada aproximadamente en 10\% de humedad relativa. La configuración que los resultados menos lejanos fue icm-mp\_physiscs 3 IDEAM-Colombia, de nuevo la configuración icm sobre estimó la humedad en las horas de la noche.\\

En el caso 3 podemos observar que hay un buen ajuste de los datos modelados con los datos reales de las estaciones automáticas. La estación automática presenta un pico de baja humedad el día 28 de octubre del 2015, este valor no fue representado por las simulaciones (Figura \ref{caso2_tiba_wrf_hum}). En la noche entre el 28 y 29 de octubre del 2015 la configuración IDEAM-Colombia e icm-mp\_physics 3 presentaron valores cercanos a el valor de la estación. Los valores mínimos de humedad para el día 29 de octubre de 2015 no fueron bien representados por las simulaciones, ya que las simulaciones no fueron capaces de modelar el valor mínimo de la estación automática. La configuración que mejor ajuste presenta es la IDEAM-Colombia y se observa una sobre estimación de la configuración icm cuyo valor de humedad que llega a 115\%.\\

En el caso 4 podemos observar que los datos observados y los datos simulados presentaron un buen ajuste. El día 8 de septiembre de 2015 la estación automática presentó valores de 40\% y las tres configuraciones no fueron capaces de generar estos valores, la configuración que presentó los valores más cercanos fue IDEAM-Colombia, Figura \ref{caso4_tiba_wrf_hum}. Es interesante notar cómo los valores máximos fueron muy bien representados por la configuración IDEAM-Colombia.\\

En general para esta sección podemos observar que las configuraci IDEAM-Colombia e icm-mp\_physics 3 representa mejor la humedad relativa y que la configuración icm en las horas de la noche sobre estima la humedad, ya que esta supera el 100\%.


\begin{figure}[H]

\begin{subfigure}[normla]{0.4\textwidth}
\caption{Caso 1 helada del 4 de febrero de 2007.}
\label{caso1_tiba_wrf_hum}
\includegraphics[draft=false, scale=0.4]{comparacion_graficas_otras_var/200702_21206990_humedad.png}
\end{subfigure}
~
\begin{subfigure}[normla]{0.4\textwidth}
\caption{Caso 2 helada del 30 de octubre de 2014.}
\label{caso2_tiba_wrf_hum}
\includegraphics[draft=false, scale=0.4]{comparacion_graficas_otras_var/201408_21206990_humedad.png}
\end{subfigure}
~
\centering
\begin{subfigure}[normla]{0.4\textwidth}
\caption{Caso 3 altas temperaturas para el 28 de octubre de 2015.}
\label{caso3_tiba_wrf_hum}
\includegraphics[draft=false, scale=0.4]{comparacion_graficas_otras_var/201508_21206990_humedad.png}
\end{subfigure}
~
\centering
\begin{subfigure}[normla]{0.4\textwidth}
\caption{Caso 4 altas temperaturas para el 8 de septiembre de 2015.}
\label{caso4_tiba_wrf_hum}
\includegraphics[draft=false, scale=0.4]{comparacion_graficas_otras_var/201509_21206990_humedad.png}
\end{subfigure}

    \caption{Humedad relativa para los cuatro casos escogidos de acuerdo a los valores registrados en la  estación automática Tibaitatá y los valores simulados con las configuraciones icm, icm-mp\_phyiscs 3 e IDEAM-Colombia.}
    \label{fig:wrf_hum_tibaitata}
\end{figure}


\subsection{Radiación}


La radiación es una de las variables que presenta más fallas por falta de datos observados, esto se evidencia en las gráfica \ref{fig:wrf_rad_tibaitata}. Para el caso 1 (\ref{caso1_tiba_wrf_rad}) hay disponibilidad de la mayoría de los datos observados, pero para los otros casos (Figura \ref{caso2_tiba_wrf_rad}, \ref{caso3_tiba_wrf_rad} y \ref{caso4_tiba_wrf_rad}) se observa que en las horas de la noche se está presentado fallas de registro.\\

Para los cuatro casos la radiación no fue correctamente modelada ya que los valores fueron sobreestimados por todas las configuraciones..\\

\begin{figure}[H]
    
\begin{subfigure}[normla]{0.4\textwidth}
\caption{Caso 1 helada del 4 de febrero de 2007.}
\label{caso1_tiba_wrf_rad}
\includegraphics[draft=false, scale=0.4]{comparacion_graficas_otras_var/200702_21206990_radiacion.png}
\end{subfigure}
~
\begin{subfigure}[normla]{0.4\textwidth}
\caption{Caso 2 helada del 30 de octubre de 2014.}
\label{caso2_tiba_wrf_rad}
\includegraphics[draft=false, scale=0.4]{comparacion_graficas_otras_var/201408_21206990_radiacion.png}
\end{subfigure}
~
\centering
\begin{subfigure}[normla]{0.4\textwidth}
\caption{Caso 3 altas temperaturas para el 28 de octubre de 2015.}
\label{caso3_tiba_wrf_rad}
\includegraphics[draft=false, scale=0.4]{comparacion_graficas_otras_var/201508_21206990_radiacion.png}
\end{subfigure}
~
\centering
\begin{subfigure}[normla]{0.4\textwidth}
\caption{Caso 4 altas temperaturas para el 8 de septiembre de 2015.}
\label{caso4_tiba_wrf_rad}
\includegraphics[draft=false, scale=0.4]{comparacion_graficas_otras_var/201509_21206990_radiacion.png}
\end{subfigure}

    \caption{Radiación de onda corta para los cuatro casos escogidos de acuerdo a los valores registrados en la estacón Tibaitatá y a los valores simulados con las configuraciones icm, icm-mp\_physics 3 e IDEAM-Colombia.} % La variable que se usó para la extracción del wrf fue SWDOWN = DOWNWARD SHORT WAVE FLUX AT GROUND SURFACE (W m-2)
    \label{fig:wrf_rad_tibaitata}
\end{figure}


\subsection{Precipitación}

En la Figura \ref{fig:wrf_prec_tibaitata} se puede observar que en ninguno de los días de los cuatro casos se presentaron lluvias y las dos configuraciones del modelo generaron precipitaciones en 3 de los 4 casos analizados. En el caso 1 (Figura \ref{caso1_tiba_wrf_prec}) se observa que la configuración icm e IDEAM-Colombia representó correctamente la no precipitación, mientras que la configuración icm-mp\_physics presentó equivocadamente valores de precipitación. Para el caso 2 se observa que no hubo precipitación y la configuración icm presentó una precipitación acumulada superior a los 2 mm y la configuración icm-mp\_physics 3 presentó una precipitación acumulada de 0.5 mm(Figura \ref{caso2_tiba_wrf_prec}). En el caso 3, todas las configuraciones presentaron valores de precipitación siendo mayores los de la configuración icm (Figura \ref{caso3_tiba_wrf_prec}). Y en el caso 4 sucede algo similar al caso 3 en el cual ambas configuraciones presentaron valores de precipitación y fue la configuración icm-mp_physics 3 fue la que presentó una mayor sobreestimación, ver Figura \ref{caso4_tiba_wrf_prec}.\\

%La precipitación del modelo resulta de la suma de dos variables internas de las salidas una de ellas es RAINC la cual viene del esquema de cúmulos y la otra es RAINNC la cual viene de la microfísica de nubes. Ya que los resultados mostraron que la mejor opción es mantener parametrizaciones apagadas mp\_physics y cu\_physics, como resultado no se van a obtener valores de precipitación en las salidas de la configuración icm.\\

\begin{figure}[H]
    
\begin{subfigure}[normla]{0.4\textwidth}
\caption{Caso 1 helada del 4 de febrero de 2007.}
\label{caso1_tiba_wrf_prec}
\includegraphics[draft=false, scale=0.4]{comparacion_graficas_otras_var/200702_21206990_rain.png}
\end{subfigure}
~
\begin{subfigure}[normla]{0.4\textwidth}
\caption{Caso 2 helada del 30 de octubre de 2014.}
\label{caso2_tiba_wrf_prec}
\includegraphics[draft=false, scale=0.4]{comparacion_graficas_otras_var/201408_21206990_rain.png}
\end{subfigure}
~
\centering
\begin{subfigure}[normla]{0.4\textwidth}
\caption{Caso 3 altas temperaturas para el 28 de octubre de 2015.}
\label{caso3_tiba_wrf_prec}
\includegraphics[draft=false, scale=0.4]{comparacion_graficas_otras_var/201508_21206990_rain.png}
\end{subfigure}
~
\centering
\begin{subfigure}[normla]{0.4\textwidth}
\caption{Caso 4 altas temperaturas para el 8 de septiembre de 2015.}
\label{caso4_tiba_wrf_prec}
\includegraphics[draft=false, scale=0.4]{comparacion_graficas_otras_var/201509_21206990_rain.png}
\end{subfigure}

    \caption{Figuras de comparación entre los datos de precipitación para la estación Tibaitatá en los cuatro casos escogidos. La línea continua representa los datos de la estación meteorológica automática del IDEAM, la línea gris representa los valores modelados teniendo en cuenta la configuración del IDEAM-Colombia y la línea roja representa los datos de la configuración hallada icm. Los triángulos representan los resultados de el dominio 1 y los pentágonos representan los resultados del dominio 2.}
    \label{fig:wrf_prec_tibaitata}
\end{figure}


%\subsection{Bulbo húmedo}
%
%En el caso 1 \ref{caso1_tiba_wrf_wb} los valores modelados no representaron correctamente los valores mínimos, el día 3 de febrero del 2007 la configuración icm tuvo una mejor aproximación a los valores maximos y el día 4 de febrero de 2007 la configuración IDEAM-Colombia fue la que presentó la mejor aproximación a los valore máximos reales. En el caso 2 las ambas configuraciones presentaron resultados similares a los valores generados por la estación automática, para el día 30 de octubre del 2014 los valores de las configuraciones no presentaron valores similares a los valores de la estación automática, ver Figura \ref{caso2_tiba_wrf_wb}. En el caso 3 el valor del bulbo húmedo para los modelos tuvo una buena aproximación, el valor mínimo del 28 de octubre del 2015 no fue correctamente representado, pero el valor máximo fue correctamente representado por la configuración icm, ver Figura \ref{caso3_tiba_wrf_prec}. Y en el caso 4 se observa que la configuración icm presentó un mejor ajuste a los valores de la estación automática, el día 8 de septiembre del 2015 ambas configuraciones presentaron una subestimación y fue más evidente en la configuración icm, ver Figura \ref{caso4_tiba_wrf_wb}.\\
%
%%La precipitación del modelo resulta de la suma de dos variables internas de las salidas una de ellas es RAINC la cual viene del esquema de cúmulos y la otra es RAINNC la cual viene de la microfísica de nubes. Ya que los resultados mostraron que la mejor opción es mantener parametrizaciones apagadas mp\_physics y cu\_physics, como resultado no se van a obtener valores de precipitación en las salidas de la configuración icm.\\
%
%\begin{figure}[H]
%    
%\begin{subfigure}[normla]{0.4\textwidth}
%\caption{Caso 1 helada del 4 de febrero de 2007.}
%\label{caso1_tiba_wrf_wb}
%\includegraphics[draft=false, scale=0.4]{comparacion_graficas_otras_var/200702_21206990_wetbulb.png}
%\end{subfigure}
%~
%\begin{subfigure}[normla]{0.4\textwidth}
%\caption{Caso 2 helada del 30 de octubre de 2014.}
%\label{caso2_tiba_wrf_wb}
%\includegraphics[draft=false, scale=0.4]{comparacion_graficas_otras_var/201408_21206990_wetbulb.png}
%\end{subfigure}
%~
%\centering
%\begin{subfigure}[normla]{0.4\textwidth}
%\caption{Caso 3 altas temperaturas para el 28 de octubre de 2015.}
%\label{caso3_tiba_wrf_wb}
%\includegraphics[draft=false, scale=0.4]{comparacion_graficas_otras_var/201508_21206990_wetbulb.png}
%\end{subfigure}
%~
%\centering
%\begin{subfigure}[normla]{0.4\textwidth}
%\caption{Caso 4 altas temperaturas para el 8 de septiembre de 2015.}
%\label{caso4_tiba_wrf_wb}
%\includegraphics[draft=false, scale=0.4]{comparacion_graficas_otras_var/201509_21206990_wetbulb.png}
%\end{subfigure}
%
%    \caption{Figuras de comparación entre los datos de bulbo húmedo para la estación Tibaitatá en los cuatro casos escogidos. La línea continua representa los datos de la estación meteorológica automática del IDEAM, la línea gris representa los valores modelados teniendo en cuenta la configuración del IDEAM-Colombia y la línea roja representa los datos de la configuración hallada icm. Los triángulos representan los resultados de el dominio 1 y los pentágonos representan los resultados del dominio 2.}
%    \label{fig:wrf_wb_tibaitata}
%\end{figure}


\subsection{Punto de rocío}

En el caso 1 (Figura \ref{caso1_tiba_wrf_dp}) se observa que los resultados de las simulaciones no se ajustaron a los valores de la estación automática. Para el caso 2, se observa que hubo mejor respuesta de los datos modelados en comparación con los datos del caso 1, los días 29 y 30 de agosto del 2014 los valores máximos tuvieron una mejor aproximación, ya que se presentaron los valores de punto de rocío pero con un lapso de 3 horas promedio por parte de la configuración icm e icm-mp\_physics 3, ver Figura \ref{caso2_tiba_wrf_dp}. En el caso 3, sucede algo similar al caso 2, ya que en los primeros dos días de modelación los valores máximos con la configuración icm presentan los valores de punto de rocío, pero con un lapso de retraso de 5 horas en promedio, ver Figura \ref{caso3_tiba_wrf_dp}. En el caso 4 los valores modelados no presentan similitud con los valores de la estación automática, ver Figura \ref{caso4_tiba_wrf_prec}.\\



\begin{figure}[H]
    
\begin{subfigure}[normla]{0.4\textwidth}
\caption{Caso 1 helada del 4 de febrero de 2007.}
\label{caso1_tiba_wrf_dp}
\includegraphics[draft=false, scale=0.4]{comparacion_graficas_otras_var/200702_21206990_dewpoint.png}
\end{subfigure}
~
\begin{subfigure}[normla]{0.4\textwidth}
\caption{Caso 2 helada del 30 de octubre de 2014.}
\label{caso2_tiba_wrf_dp}
\includegraphics[draft=false, scale=0.4]{comparacion_graficas_otras_var/201408_21206990_dewpoint.png}
\end{subfigure}
~
\centering
\begin{subfigure}[normla]{0.4\textwidth}
\caption{Caso 3 altas temperaturas para el 28 de octubre de 2015.}
\label{caso3_tiba_wrf_dp}
\includegraphics[draft=false, scale=0.4]{comparacion_graficas_otras_var/201508_21206990_dewpoint.png}
\end{subfigure}
~
\centering
\begin{subfigure}[normla]{0.4\textwidth}
\caption{Caso 4 altas temperaturas para el 8 de septiembre de 2015.}
\label{caso4_tiba_wrf_dp}
\includegraphics[draft=false, scale=0.4]{comparacion_graficas_otras_var/201509_21206990_dewpoint.png}
\end{subfigure}

    \caption{Temperatura de punto de rocío para los cuatro casos escogidos de acuerdo a los valores registrados en la estación automática Tibaitatá y a los valores simulados con la configuraciones icm, icm-mp\_physics 3 e IDEAM-Colombia.}
    \label{fig:wrf_dp_tibaitata}
\end{figure}


\subsection{Rapidez del viento}

Los valores de velocidad del viento para la estación Tibaitatá no pasaron el control de calidad, por esta razón se trabajó con la estación La Boyera, ya que esta se encuentra a una altura similar a la de Tibaitatá.\\

En el caso 1 se observa que las simulaciones no representaron correctamente los valores de la rapidez del viento para la estación La Boyera, solo el día 3 de febrero del 2007 la rapidez máximas alcanzaron valores similares, ver Figura \ref{caso1_tiba_wrf_vv}. Para el caso 2 se observa que hubo un mejor ajuste de los datos modelados frente a los valores de la estación automática, hay un mejor ajuste entre los valores mínimos en comparación con los valores máximos que tienen una diferencia de 2\celsius en promedio, ambas configuraciones del modelo presentan resultados similares, ver Figura \ref{caso2_tiba_wrf_vv}. En el caso 3 podemos observar que los datos modelados presentaron un mejor ajuste y esta vez la configuración IDEAM-Colombia presentó un buen ajuste frente a las rapidez máxima el día 29 de octubre del 2015, ver Figura \ref{caso3_tiba_wrf_vv}. Y en el caso 4 se puede observar que en los valores de velocidad baja los valores modelados presentan un mejor ajuste, ver Figura \ref{caso4_tiba_wrf_vv}.\\

%La precipitación del modelo resulta de la suma de dos variables internas de las salidas una de ellas es RAINC la cual viene del esquema de cúmulos y la otra es RAINNC la cual viene de la microfísica de nubes. Ya que los resultados mostraron que la mejor opción es mantener parametrizaciones apagadas mp\_physics y cu\_physics, como resultado no se van a obtener valores de precipitación en las salidas de la configuración icm.\\

\begin{figure}[H]
    
\begin{subfigure}[normla]{0.4\textwidth}
\caption{Caso 1 helada del 4 de febrero de 2007 para la estación La Boyera.}
\label{caso1_tiba_wrf_vv}
\includegraphics[draft=false, scale=0.4]{comparacion_graficas_otras_var/200702_24015110_vel_viento.png}
\end{subfigure}
~
\begin{subfigure}[normla]{0.4\textwidth}
\caption{Caso 2 helada del 30 de octubre de 2014 para la estación Tibaitatá.}
\label{caso2_tiba_wrf_vv}
\includegraphics[draft=false, scale=0.4]{comparacion_graficas_otras_var/201408_21206990_vel_viento.png}
\end{subfigure}
~
\centering
\begin{subfigure}[normla]{0.4\textwidth}
\caption{Caso 3 altas temperaturas para el 28 de octubre de 2015 para la estación Tibaitatá.}
\label{caso3_tiba_wrf_vv}
\includegraphics[draft=false, scale=0.4]{comparacion_graficas_otras_var/201508_21206990_vel_viento.png}
\end{subfigure}
~
\centering
\begin{subfigure}[normla]{0.4\textwidth}
\caption{Caso 4 altas temperaturas para el 8 de septiembre de 2015 para la estación Tibaitatá.}
\label{caso4_tiba_wrf_vv}
\includegraphics[draft=false, scale=0.4]{comparacion_graficas_otras_var/201509_21206990_vel_viento.png}
\end{subfigure}

    \caption{Rapidez del viento para los cuatro casos escogidos de acuerdo a los valores registrados en las estaciones automáticas la Boyera y Tibaitatá y los valores simulados con las configuraciones icm, icm-mp\_physics 3 e IDEAM-Colombia.}
    \label{fig:wrf_vv_tibaitata}
\end{figure}


\section{Comparación estadística entre la configuración icm, icm-mp\_physics\_3 e IDEAM-Colombia}

Para el caso de las heladas, la información sobre temperatura, humedad, y rapidez del viento en capas cercanas a la superficie juega un papel importante en la toma de decisiones para la protección de los cultivos \citep{prabha2008}. Por esta razón, se realizó una comparación estadística de las variables temperatura, humedad relativa, y rapidez del viento para los cuatro casos de los datos modelados y los datos de las estaciones automáticas.\\

%Para la extracción de estas variables se usaron las siguientes fórmulas.\\

Para saber la frecuencia horaria en cada año que se presentaron heladas y altas temperaturas se realizaron gráficas de la cantidad de horas en las que se presentaron temperaturas por encima de 0\celc\ y por debajo de 25\celc (Figura \ref{subfig:temp_horarias_ext}) usando los valores validados de las estaciones automáticas.\\

\begin{figure}[H]
    \centering
    \caption{Número de horas por año que la temperatura estuvo por debajo de 0\celc.}
    \begin{subfigure}[b]{0.45\textwidth}
	\includegraphics[draft=false, scale=0.45]{prabha/grafica2/bajas_tmp.png}
    \label{subfig:tmp_0}
	\end{subfigure}
	~
    \caption{Número de horas por año que la temperatura estuvo por encima de 25\celc.}
    \begin{subfigure}[b]{0.45\textwidth}
	\includegraphics[draft=false, scale=0.45]{prabha/grafica2/altas_tmp.png}
    \label{subfig:tmp_25}
	\end{subfigure}

\caption{Número de horas por año en las cuales la temperatura estuvo por debajo de 0\celc\ o por encima de 25\celc.}	
\label{subfig:temp_horarias_ext}	
\end{figure}

Para el caso de las temperaturas bajo 0\celc\ podemos observar que hay varios picos estos picos están asociados a años en los que se presentó eventos el Niño (2006/2007, 2009/2010 y 2014/2015/2016; \citep{NOAA-ORI}). La estación que presentó la mayor cantidad de valores de horas bajo 0\celc\ en la Figura \ref{subfig:tmp_0} es la estación Hda Santa Ana en el municipio de Nemocón a 2572 msnm. La estación que presentó la mayor cantidad de horas anuales con temperaturas por encima de los 25\celc\ es la estación La Capilla Autom en el municipio de a 1917 msnm, la estación Hda Santa Ana se encuentra en un valle entre montañas, lo que implica que las masas de aire frío se puedan depositar en este lugar, la estación La Capilla Autom se encuentra en el município de La Capilla el cual tiene una temperatura promedio de 18\celc, entonces temperaturas por encima de 25\celc\ no son extrañas en esta región.\\

Para poder realizar una comparación de los valores obtenidos con el modelo WRF y los datos observados se usó el promedio del error, Ecuación \ref{eq:mbe}. Este estadístico nos ayuda a determinar cuando el modelo presenta sobreestimación o subestimación. Este estadístico fue calculado a nivel horario.

\begin{equation}\label{eq:mbe}
MBE = \mathlarger{\frac{1}{n} \sum_{i=1}^n (x'_i - x_i)}
\end{equation}

Donde $n$ es el número de datos, $x_i$ corresponde a los registros de las estaciones automáticas, $x'_i$ corresponde a los datos modelados.\\
%%%% Voy acá
Para el cálculo del sesgo horario se usó la fórmula de la ecuación \ref{eq:mbe} y los valores se promediaron dependiendo la cantidad de días de cada caso. El análisis para todas las estaciones se realizó con la configuración icm. Se tomaron todas las estaciones y se excluyó la estación Tibaitatá, ya que solo para esta estación se analizará la configuración IDEAM-Colombia e icm a forma de ejemplo.\\

Para este análisis se tomaron los valores horarios de las variables temperatura del aire, punto de rocío, bulbo húmedo y velocidad del viento, estos valores fueron promediados y se les sacó la desviación estándar. Dentro de este grupo de datos no se incluyó la estación Tibaitatá, ya que esta fue analizada por separado.\\

Se realizó una comparación del rendimiento de la configuración icm frente a los datos de temperatura de las estaciones automáticas. Se tomó como referencia la estación Tibaitatá y sólo para esta estación se compararon los resultados de la configuración IDEAM-Colombia e icm, esto se hizo con la finalidad de observar si se presentan mejoras con la nueva configuración.\\

\subsection{Caso 1}

Los valores de velocidad del viento de la estación Tibaitatá en el caso 1 no fueron suficientes para hacer una gráfica, por esa razón solo para el caso 1 se tendrá en cuenta la estación La Boyera, ubicada en Ubaté para analizar la velocidad del viento.
	
\begin{figure}[H]
    \centering
    \begin{subfigure}[b]{0.45\textwidth}
        \caption{Tendencia para la temperatura del aire.}
	\includegraphics[draft=false, scale=0.45]{prabha/grafica4abcd_final/200702_tmp_2m.png}
    \label{subfig:tmp_0_caso1}
	\end{subfigure}
	~
	    \begin{subfigure}[b]{0.45\textwidth}
	        \caption{Tendencia para el punto de rocío.}
	\includegraphics[draft=false, scale=0.45]{prabha/grafica4abcd_final/200702_Td.png}

    \label{subfig:td_caso1}
	\end{subfigure}
	~
	    \begin{subfigure}[b]{0.45\textwidth}
	\caption{Tendencia para el bulbo húmedo.}
	\includegraphics[draft=false, scale=0.45]{prabha/grafica4abcd_final/200702_wb.png}
    \label{subfig:wb_caso1}
	\end{subfigure}
	~
	    \begin{subfigure}[b]{0.45\textwidth}
	\caption{Tendencia para la velocidad del viento.}	
	\includegraphics[draft=false, scale=0.45]{prabha/grafica4abcd_final/200702_vel_vi10.png}
    
    \label{subfig:vel_caso1}
	\end{subfigure}
	~

\caption{Tendencias horarias calculadas con la ecuación \ref{eq:mbe} para el caso1 comprendido entre el 3 y el 4 de febrero de 2007. Los símbolos representan los valores promedios y las líneas representan la desviación estándar.}	
\label{subfig:mbe_caso1}	
\end{figure}

%Cómo se calculó el punto de rocío y la tmp de bulbo húmedo y su importancia

El grupo de estaciones diferente a la estación Tibaitatá presenta un buen comportamiento a través del día, pero en las horas del día hay una tendencia a subestimar los valores de temperatura modelados llegando hasta -3\celc, en la estación Tibaitatá podemos observar que en el día los valores son subestimados llegando hasta -6\celc\ y en la noche los valores son sobre estimados alcanzando valores de 8\celc, ver figura \ref{subfig:tmp_0_caso1}. Para el punto de rocío se puede observar que para el grupo de estaciones se presentó una sobre estimación en la mayoría de las horas la cual alcanza valores de 7\celc, que se ve disminuida a las 13 horas llegando a valores de -2\celc, para la estación Tibaitatá en el día se presentó una subestimación que llegó hasta -8\celc\ y en las horas de la noche se presentó una sobre estimación que alcanzó valores de 7\celc, ver gráfica \ref{subfig:td_caso1}. El grupo de estaciones presentaron una sobre estimación generalizada excepto entre las 8 y 9 horas, para la estación Tibaitatá se presentó una sobre estimación generalizada excepto a las 13 horas y entre las 17 y 19 horas que se presentaron valores negativos que llegaron a -3\celc, ver Figura \ref{subfig:wb_caso1}. Para la velocidad del viento podemos observar que para el grupo de estaciones en las horas de la mañana hubo una sobre estimación que llegó hasta los 4\celc\ y en las horas de la noche mejoró el comportamiento, para la estación La Boyera entre las 8 y 20 horas se presentó una sobre estimación cuyo valor máximo fue de 3\celc\ las demás horas fueron subestimadas alcanzando valores de -4\celc, Figura \ref{subfig:vel_caso1}. La configuración icm presentó mejores resultados de MBE para la temperatura, punto de rocío y bulbo húmedo.\\



\subsection{Caso 2}

\begin{figure}[H]
    \centering
    \begin{subfigure}[b]{0.45\textwidth}
        \caption{Tendencia para la temperatura del aire.}
	\includegraphics[draft=false, scale=0.45]{prabha/grafica4abcd_final/201408_tmp_2m.png}
    \label{subfig:tmp_0_caso2}
	\end{subfigure}
	~
	    \begin{subfigure}[b]{0.45\textwidth}
	        \caption{Tendencia para el punto de rocío.}
	\includegraphics[draft=false, scale=0.45]{prabha/grafica4abcd_final/201408_Td.png}

    \label{subfig:td_caso2}
	\end{subfigure}
	~
	    \begin{subfigure}[b]{0.45\textwidth}
	\caption{Tendencia para el bulbo húmedo.}
	\includegraphics[draft=false, scale=0.45]{prabha/grafica4abcd_final/201408_wb.png}
    \label{subfig:wb_caso2}
	\end{subfigure}
	~
	    \begin{subfigure}[b]{0.45\textwidth}
	\caption{Tendencia para la velocidad del viento.}	
	\includegraphics[draft=false, scale=0.45]{prabha/grafica4abcd_final/201408_vel_vi10.png}
    
    \label{subfig:vel_caso2}
	\end{subfigure}
	~

\caption{Tendencias horarias calculadas con la ecuación \ref{eq:mbe} para el caso 2 comprendido entre el 29 y 31 de agosto de 2014.}	
\label{subfig:mbe_caso2}	
\end{figure}

%Cómo se calculó el punto de rocío y la tmp de bulbo húmedo y su importancia

El grupo de estaciones diferente a la estación Tibaitatá presenta una subestimación en las 24 horas siendo menor a las 13 horas con vaores promedio de 0\celc, para el caso de la estación Tibaitatá se observa los valores presentan una oscilación entre 2 y -2\celc\ alrededor del valor de 0\celc, ver figura \ref{subfig:tmp_0_caso2}. Para el punto de rocío se puede observar que para el grupo de estaciones se presentó una sobre estimación en la mayoría de las horas y para la estación Tibaitatá en el día se presentó una subestimación que llegó hasta -6\celc\ y en las horas de la noche se presentó una sobre estimación similar al que se presenta en el grupo de estaciones llegando hasta 4\celc, ver gráfica \ref{subfig:td_caso2}. Para el bulbo húmedo se presentaron valores cercanos a 0\celc\ con tendencia a la sobre estimación llegando a valores promedio de 1\celc, para la estación Tibaitatá se presentó una sobre estimación más alta, solo  horas de la mañana se presentaron valores subestimados que alcanzaron -3\celc, ver Figura \ref{subfig:wb_caso2}. Para la velocidad del viento podemos observar que para el grupo de estaciones como la estación Tibaitatá presentaron un comportamiento similar, en el cual están cerca de 0\celc\ y se presenta una sobreestimación entre las 13 y 19 horas que alcanza valores máximos promedios de 3\celc, ver Figura \ref{subfig:vel_caso2}. La configuración IDEAM-Colombia presentó mejores resultados para la temperatura, punto de rocío, velocidad del viento y bulbo húmedo.\\


\subsection{Caso 3}

\begin{figure}[H]
    \centering
    \begin{subfigure}[b]{0.45\textwidth}
        \caption{Tendencia para la temperatura del aire.}
	\includegraphics[draft=false, scale=0.45]{prabha/grafica4abcd_final/201508_tmp_2m.png}
    \label{subfig:tmp_0_caso3}
	\end{subfigure}
	~
	    \begin{subfigure}[b]{0.45\textwidth}
	        \caption{Tendencia para el punto de rocío.}
	\includegraphics[draft=false, scale=0.45]{prabha/grafica4abcd_final/201508_Td.png}

    \label{subfig:td_caso3}
	\end{subfigure}
	~
	    \begin{subfigure}[b]{0.45\textwidth}
	\caption{Tendencia para el bulbo húmedo.}
	\includegraphics[draft=false, scale=0.45]{prabha/grafica4abcd_final/201508_wb.png}
    \label{subfig:wb_caso3}
	\end{subfigure}
	~
	    \begin{subfigure}[b]{0.45\textwidth}
	\caption{Tendencia para la velocidad del viento.}	
	\includegraphics[draft=false, scale=0.45]{prabha/grafica4abcd_final/201508_vel_vi10.png}
    
    \label{subfig:vel_caso3}
	\end{subfigure}
	~

\caption{Tendencias horarias calculadas con la ecuación \ref{eq:mbe} para el caso 3 comprendido entre el 27 y 29 de octubre de 2015.}	
\label{subfig:mbe_caso3}	
\end{figure}

%Cómo se calculó el punto de rocío y la tmp de bulbo húmedo y su importancia

El grupo de estaciones diferente a la estación Tibaitatá presenta una sobre estimación entre las 8 y las 18 horas alcanzando valores de 4\celc\ y en las demás horas presenta una subestimación que llega a valores de -4\celc\ a las 2 y 6 horas, para el caso de la estación Tibaitatá se presenta la mayor subestimación a las 9 horas llegando a valores de -8\celc, en la mayoría de las horas del día se presenta una sobre estimación que llega hasta valores de 5\celc, ver figura \ref{subfig:tmp_0_caso3}. Para el punto de rocío se puede observar que para el grupo de estaciones se presentó una sobre estimación en todas las horas y para la estación Tibaitatá en el día se presentó una subestimación que llegó hasta -3\celc\ y en las horas de la noche se presentó una sobre estimación similar al que se presenta en el grupo de estaciones llegando hasta 5.5\celc, ver gráfica \ref{subfig:td_caso3}. Para el bulbo húmedo se presentaron valores cercanos a 0\celc\ con tendencia a la sobre estimación llegando a valores promedio de 1\celc, para la estación Tibaitatá se presentó una sobre estimación más alta, solo  horas de la mañana se presentaron valores subestimados que alcanzaron -3\celc, ver Figura \ref{subfig:wb_caso3}. Para la velocidad del viento podemos observar que para el grupo de estaciones como la estación Tibaitatá presentaron un comportamiento similar, en el cual están cerca de 0\celc\ y se presenta una sobreestimación entre las 13 y 19 horas que alcanza valores máximos promedios de 3\celc, ver Figura \ref{subfig:vel_caso3}. La configuración IDEAM-Colombia presentó mejores resultados para la temperatura, punto de rocío, velocidad del viento y bulbo húmedo.\\





\subsection{Caso 4}

\begin{figure}[H]
    \centering
    \begin{subfigure}[b]{0.45\textwidth}
        \caption{Tendencia para la temperatura del aire.}
	\includegraphics[draft=false, scale=0.45]{prabha/grafica4abcd_final/201509_tmp_2m.png}
    \label{subfig:tmp_0_caso4}
	\end{subfigure}
	~
	    \begin{subfigure}[b]{0.45\textwidth}
	        \caption{Tendencia para el punto de rocío.}
	\includegraphics[draft=false, scale=0.45]{prabha/grafica4abcd_final/201509_Td.png}

    \label{subfig:td_caso4}
	\end{subfigure}
	~
	    \begin{subfigure}[b]{0.45\textwidth}
	\caption{Tendencia para el bulbo húmedo.}
	\includegraphics[draft=false, scale=0.45]{prabha/grafica4abcd_final/201509_wb.png}
    \label{subfig:wb_caso4}
	\end{subfigure}
	~
	    \begin{subfigure}[b]{0.45\textwidth}
	\caption{Tendencia para la velocidad del viento.}	
	\includegraphics[draft=false, scale=0.45]{prabha/grafica4abcd_final/201509_vel_vi10.png}
    
    \label{subfig:vel_caso4}
	\end{subfigure}
	~

\caption{Tendencias horarias calculadas con la ecuación \ref{eq:mbe} para el caso 3 comprendido entre el 7 y el 9 de septiembre de 2015.}	
\label{subfig:mbe_caso4}	
\end{figure}

%Cómo se calculó el punto de rocío y la tmp de bulbo húmedo y su importancia

El grupo de estaciones diferente a la estación Tibaitatá presentaron una sobre estimación entre las 8 y 16 horas alcanzando valores de 4\celc, en las demás horas presenta una subestimación el cual alcanzó sus valores mínimos a las 15 horas, para el caso de la estación Tibaitatá se presenta la mayor sobre estimación a las 16 horas llegando a valores de 3\celc\ y el valor mínimo fue alcanzado a las 15 horas, en general se presentó una alta oscilación, ver figura \ref{subfig:tmp_0_caso2}. Para el punto de rocío se puede observar que para el grupo de estaciones se presentó una sobre estimación desde las 7 horas hasta las 23 horas y posteriormente presentó valores cercanos a 0\celc, para la estación Tibaitatá entre las 0 y 16 horas se presentó una subestimación seguido de una sobre estimación que alcanzó valores de 4\celc, ver gráfica \ref{subfig:td_caso2}. Para el bulbo húmedo en el grupo de estaciones los valores oscilaron entre -3.5 y 2.2\celc\ presentando una sobre estimación en las horas de la mañana y una subestimación en las horas de la tarde, para la estación Tibaitatá se presentó un comportamiento opuesto a el comportamiento de el grupo de estaciones, ya que hubo una subestimación en las horas del día y una sobre estimación en las horas de la noche, ver Figura \ref{subfig:wb_caso2}. Para la velocidad del viento podemos observar que para el grupo de estaciones se presentaron sobre estimaciones y las más altas sobre estimaciones son a las 18 horas tanto para el grupo de estaciones como para la estación Tibaitatá, ver Figura \ref{subfig:vel_caso2}.\\

Resulta complejo diferenciar entre los resultados de la configuración icm y la configuración IDEAM-Colombia, ya que dan resultados que en las gráficas son muy similares.\\

%Los mayores valores de MBE para la temperatura se presentaron para el caso 1 en las horas de la noche, este comportamiento se puede observar en la gráfica \ref{caso1_tiba_wrf}. En el caso 2 se evidenció que había una gran desviación estándar y que los datos en general tienden a ser subestimados. En los casos de altas temperaturas se puede observar que el conjunto de datos se comporta de una forma similar, ya que en el día se presenta una sobre estimación y en las horas de la noche hay una subestimación los valores en ambos casos oscilan entre 4 y -4\celc\ y las desviaciones estandar hacen que los valores lleguen desde -7\celc\ hasta 7\celc. En el caso 1, 3 y 4 la configuración icm presentó mejores resultados que la del IDEAM-Colombia.\\

%En general para los 4 casos de estudio la velocidad del viento fue siempre sobre estimada, solo para la estación La Boyera se puede observar una subestimación clara. Se evidencia en los 4 casos una sobre estimación entre las 16 y 17 horas en todos los casos y en las horas de la noche se observa una mejora en los resultados.\\

%La temperatura de bulbo húmedo para los casos 1 y 2 se comportó de una forma similar ya que en general se sobre estimó con algunas horas en las que fue sub estimada en las horas de la mañana. En los casos 3 y 4 se presentó una sobre estimación en el día que para el caso 3 se prolongó hasta las 23 horas. hay una gran desviación estándar y  los valores promedios oscilaron entre los 4 y los -2\celc.\\


\section{Comparación de temperaturas extremas observadas y simuladas con configuraciones icm e icm-mp\_physics\_3}

Para realizar un análisis de las temperaturas extremas se tomaron los valores de temperatura mínimo y máximo de las estaciones y se comparó con los resultados que brinda el modelo. Adicionalmente se tomaron unas horas ya que según \citet{snyder2005frost} es un insumo útil que puede ayudar a la predicción de las heladas \citep{prabha2008}, por esta razón se analizaron las temperaturas a las 18 y 22 horas, ver Figura 


\begin{figure}[H]
    \centering
    \begin{subfigure}[b]{0.45\textwidth}
        \caption{Temperaturas a las 18, 22, máximas y mínimas para el caso 1.}
	\includegraphics[draft=false, scale=0.45]{prabha/grafica7/200702_ideam_i_d01.png}
    \label{subfig:tmp_ext_caso1_d01}
	\end{subfigure}
	~
	    \begin{subfigure}[b]{0.45\textwidth}
        \caption{Temperaturas a las 18, 22, máximas y mínimas para el caso 2.}
	\includegraphics[draft=false, scale=0.45]{prabha/grafica7/201408_ideam_i_d01.png}

    \label{subfig:tmp_ext_caso2_d01}
	\end{subfigure}
	~
	    \begin{subfigure}[b]{0.45\textwidth}
        \caption{Temperaturas a las 18, 22, máximas y mínimas para el caso 3.}
	\includegraphics[draft=false, scale=0.45]{prabha/grafica7/201508_ideam_i_d01.png}
    \label{subfig:tmp_ext_cas3_d01}
	\end{subfigure}
	~
	    \begin{subfigure}[b]{0.45\textwidth}
        \caption{Temperaturas a las 18, 22, máximas y mínimas para el caso 4.}
	\includegraphics[draft=false, scale=0.45]{prabha/grafica7/201509_ideam_i_d01.png}
    
    \label{subfig:tmp_ext_caso4_d01}
	\end{subfigure}
	~

\caption{Comparación de las temperaturas a las 18, 22, máximas y mínimas entre los valores de icm frente a los valores de las estaciones automáticas.}	
\label{subfig:tmp_ext_d01}	
\end{figure}

En la Figura \ref{subfig:tmp_ext_d01} se puede observar el comportamiento de los datos modelados frente a los datos registrados de las estaciones automáticas. Se realizó una comparación de los datos modelados frente a los reales para cada uno de los casos en el caso 1 (Figura \ref{subfig:tmp_ext_caso1_d01}) se observa que la configuración icm presenta valores similares a los de la estación automática, pero para las temperaturas mínimas y máximas se puede observar que el modelo no es capaz de reproducir correctamente los valores temperatura del aire.\\

En el caso 2 se observa que en general los datos de temperatura modelados estuvieron cercanos a los datos de las estaciones, pero existe un sesgo en el que el modelo da como resultado valores inferiores a los valores reportados en las estaciones convencionales, ver Figura \ref{tmp_ext_caso2_d01}.\\

Para el caso 3 se observa que hay una proporción similar entre los datos sobre estimados y los datos subestimados, ver figura \ref{subfig:tmp_ext_caso2_d01}. En el caso 3 se puede observar unas temperaturas máximas que el modelo está sobre estimando. Para los datos extremos podemos observar que el modelo no tiene buenos resultados cuando se trata de temperaturas extremas ya que las temperaturas mínimas son sobre estimadas y las temperaturas máximas son subestimadas.

En el caso 4 se  puede observar que hay temperaturas máximas que fueron sobre estimadas por el modelo, pero en general se observa una buena respuesta del modelo frente a los datos de las estaciones automáticas. Al igual que en los anteriores casos las temperaturas más bajas y altas no fueron representadas correctamente.


\begin{figure}[H]
    \centering
    \begin{subfigure}[b]{0.45\textwidth}
        \caption{Temperaturas a las 18, 22, máximas y mínimas para el caso 1.}
	\includegraphics[draft=false, scale=0.45]{prabha/grafica7/200702_ideam_i_d02.png}
    \label{subfig:tmp_ext_caso1_d01}
	\end{subfigure}
	~
	    \begin{subfigure}[b]{0.45\textwidth}
        \caption{Temperaturas a las 18, 22, máximas y mínimas para el caso 2.}
	\includegraphics[draft=false, scale=0.45]{prabha/grafica7/201408_ideam_i_d02.png}

    \label{subfig:tmp_ext_caso2_d01}
	\end{subfigure}
	~
	    \begin{subfigure}[b]{0.45\textwidth}
        \caption{Temperaturas a las 18, 22, máximas y mínimas para el caso 3.}
	\includegraphics[draft=false, scale=0.45]{prabha/grafica7/201508_ideam_i_d02.png}
    \label{subfig:tmp_ext_cas31_d01}
	\end{subfigure}
	~
	    \begin{subfigure}[b]{0.45\textwidth}
        \caption{Temperaturas a las 18, 22, máximas y mínimas para el caso 4.}
	\includegraphics[draft=false, scale=0.45]{prabha/grafica7/201509_ideam_i_d02.png}
    
    \label{subfig:tmp_ext_caso4_d01}
	\end{subfigure}
	~

\caption{Comparación de las temperaturas a las 18, 22, máximas y mínimas entre los valores de icm frente a los valores de las estaciones automáticas.}	
\label{subfig:tmp_ext_d01}	
\end{figure}

\section{Conclusiones}

\begin{itemize}
    \item El modelo WRF no es capaz de representar cambios bruscos en las variables.
    \item Los datos de la estación automática de radiación presenta fallas en las hora de la noche.
    \item Los resultados de radiación para las configuraciones IDEAM-Colombia e icm en general tienden a sobre estimar los valores reportados por la estación automática.
    \item En general los resultados de la precipitación presentan una sobre estimación con ambas configuraciones pero la sobre estimación es más alta para la configuración icm.
    \item La configuración icm no fue capaz de reproducir satisfactoriamente las bajas temperaturas, pero si presentó un buen ajuste frente a los datos de temperaturas altas.
    \item En el caso 1 la combinación icm presentó dificultad para reproducir el comportamiento de la temperatura, bulbo húmedo, punto de rocío y velocidad del viento.
\end{itemize}
% de acá se sacó la formula de dew point dewpoint https://iridl.ldeo.columbia.edu/dochelp/QA/Basic/dewpoint.html

%
%
%
\clearpage


\begin{landscape}
\chapter{Tabla de las estaciones automáticas y convencionales usadas.}
\label{anexo:estaciones_autom_y_conv_usadas}

Cuando la estación no se encuentra suspendida se señala con un NaN. Para este proyecto se usaron dos tipos de estaciones del IDEAM estaciones automáticas (AUT) y estaciones convencionales (CON).

%%%Tabla de las ubicaciones

\newpage


\begin{longtable}{lp{4cm}lp{3cm}lrrll}
\caption{Estaciones automáticas y convencionales usadas.}
\label{tab:estaciones}\\
\toprule
 Código &                      Nombre & Tipo &              Municipio &   Latitud &   Longitud &  Fecha instalación &  Fecha suspensión \\
\midrule
\endhead
\midrule
\multicolumn{3}{r}{{Continúa en la siguiente página.}} \\
\midrule
\endfoot

\bottomrule
\endlastfoot
    21195160 &            SUBIA AUTOMATICA &  AUT &          SILVANIA &  4.476611 & -74.383889 &  28/09/2004 &         NaN \\
   21201200 &     ESC LA UNION AUTOMATICA &  AUT &       BOGOTA D.C. &  4.342944 & -74.183889 &  15/03/1985 &         NaN \\
   21201580 &        PASQUILLA AUTOMATICA &  AUT &       BOGOTA D.C. &  4.446500 & -74.154833 &  15/11/1981 &         NaN \\
   21202270 &      PLUVIOMETRO AUTOMATICO &  AUT &       BOGOTA D.C. &  4.608056 & -74.072889 &    09/05/12 &         NaN \\
   21202271 &          PLUVIOMETRO AUTOMA &  AUT &       BOGOTA D.C. &  4.608056 & -74.072889 &    11/04/13 &         NaN \\
   21205012 &               UNIV NACIONAL &  AUT &       BOGOTA D.C. &  4.638083 & -74.089083 &  19/08/2004 &         NaN \\
   21205791 &              APTO EL DORADO &  AUT &       BOGOTA D.C. &  4.705583 & -74.150667 &  31/10/2002 &         NaN \\
   21206600 &            NUEVA GENERACION &  AUT &       BOGOTA D.C. &  4.782222 & -74.094333 &  15/11/2001 &         NaN \\
   21206710 &      SAN JOAQUIN AUTOMATICA &  AUT &           LA MESA &  4.633333 & -74.516667 &    01/06/05 &         NaN \\
   21206790 &           HDA STA ANA AUTOM &  AUT &           NEMOCÓN &  5.090500 & -73.881250 &    10/06/04 &         NaN \\
   21206920 &     VILLA TERESA AUTOMATICA &  AUT &       BOGOTA D.C. &  4.350000 & -74.150000 &  19/07/2005 &         NaN \\
   21206930 &                PMO GUERRERO &  AUT &         ZIPAQUIRÁ &  5.086444 & -74.022167 &    01/12/04 &         NaN \\
   21206940 &              CIUDAD BOLIVAR &  AUT &       BOGOTA D.C. &  4.576861 & -74.176778 &  19/05/2005 &         NaN \\
   21206950 &             PMO GUACHENEQUE &  AUT &       VILLAPINZÓN &  5.236056 & -73.525083 &  19/07/2005 &         NaN \\
   21206960 &                IDEAM BOGOTA &  AUT &       BOGOTA D.C. &  4.600000 & -74.066667 &    11/04/05 &         NaN \\
   21206980 &          STA CRUZ DE SIECHA &  AUT &            GUASCA &  4.784278 & -73.870806 &    08/11/05 &         NaN \\
   21206990 &        TIBAITATA AUTOMATICA &  AUT &          MOSQUERA &  4.691417 & -74.209000 &  27/10/2004 &         NaN \\
   21209920 &            STA ROSITA AUTOM &  AUT &            SUESCA &  5.192250 & -73.779056 &  19/05/2005 &         NaN \\
   23125170 &          SAN CAYETANO AUTOM &  AUT &      SAN CAYETANO &  4.516753 & -74.088222 &    01/12/04 &         NaN \\
   24015110 &        LA BOYERA AUTOMATICA &  AUT &             UBATÉ &  5.303806 & -73.851750 &  15/03/1960 &         NaN \\
   26127010 &                EL ALAMBRADO &  AUT &            ZARZAL &  4.410250 & -74.124611 &  15/01/1953 &         NaN \\
   35025080 &          PNN CHINGAZA AUTOM &  AUT &         LA CALERA &  4.661000 & -73.827333 &    06/12/07 &         NaN \\
   35025090 &          BOSQUE INTERVENIDO &  AUT &         LA CALERA &  4.664889 & -73.846639 &    09/07/08 &         NaN \\
   35025100 &              CALOSTROS BAJO &  AUT &         LA CALERA &  4.673778 & -73.818889 &  24/05/2008 &         NaN \\
   35027001 &             PLAZA DE FERIAS &  AUT &           CÁQUEZA &  4.403389 & -73.940556 &  21/07/2012 &         NaN \\
   35027002 &         PARQUE RAFAEL NUÑEZ &  AUT &           CÁQUEZA &  4.407417 & -73.947500 &  21/07/2012 &         NaN \\
   35027510 &              CALOSTROS BAJO &  AUT &         LA CALERA &  4.673778 & -73.818889 &  24/05/2008 &         NaN \\
   35035130 &                PMO CHINGAZA &  AUT &         LA CALERA &  4.713667 & -73.803250 &  24/11/2004 &         NaN \\
   35075070 &        CHINAVITA AUTOMATICA &  AUT &         CHINAVITA &  5.219250 & -73.350389 &    06/03/05 &         NaN \\
   35075080 &      PMO RABANAL AUTOMATICA &  AUT &      VENTAQUEMADA &  5.392389 & -73.562778 &  13/03/2005 &         NaN \\
   35085080 &            LA CAPILLA AUTOM &  AUT &        LA CAPILLA &  5.099194 & -73.436000 &    08/03/05 &         NaN \\
   21200160 &          PANONIA [21200160] &  CON &          CHOCONTÁ &  5.057972 & -73.734333 &  15/03/1985 &         NaN \\
   21200620 &           PISCIS [21200620] &  CON &          CHOCONTÁ &  5.079167 & -73.696861 &  15/03/1985 &         NaN \\
   21200780 &               POTRERO LARGO &  CON &         GUATAVITA &  4.929222 & -73.780472 &  15/03/1985 &         NaN \\
   21200840 &      FLORESTA LA [21200840] &  CON &            GUASCA &  4.850000 & -73.783333 &  15/04/1962 &  15/09/1972 \\
   21201050 &          LOURDES [21201050] &  CON &        GACHANCIPÁ &  4.982889 & -73.864667 &  15/03/1985 &         NaN \\
   21201060 &           PANTANO REDONDO 1 &  CON &         ZIPAQUIRÁ &  5.043250 & -74.033389 &  15/01/1951 &         NaN \\
   21201070 &       CORAZON EL [21201070] &  CON &        FACATATIVÁ &  4.865361 & -74.289417 &  15/07/1974 &         NaN \\
   21201080 &               SAN FRANCISCO &  CON &              SOPÓ &  4.900000 & -73.950000 &  15/07/1974 &  15/12/1979 \\
   21201140 &     ESPERANZA LA [21201140] &  CON &             TENJO &  4.802167 & -74.179972 &  15/07/1974 &         NaN \\
   21201160 &             EDIFICIO SARAGA &  CON &       BOGOTA D.C. &  4.600000 & -74.083333 &  15/11/1976 &  15/09/1986 \\
   21201180 &        GUANQUICA [21201180] &  CON &             TAUSA &  5.184278 & -73.941111 &  15/03/1985 &         NaN \\
   21201190 &        LAGUNITAS [21201190] &  CON &             TAUSA &  5.214528 & -73.907250 &  15/03/1985 &         NaN \\
   21201210 &        HATO EL   [21201210] &  CON &             TENJO &  4.866389 & -74.153861 &  15/06/1978 &         NaN \\
   21201220 &          STA CRUZ DE SIECHA &  CON &            GUASCA &  4.784278 & -73.870806 &  15/09/1978 &         NaN \\
   21201230 &            ENMANUEL D ALZON &  CON &       BOGOTA D.C. &  4.701125 & -74.070306 &  15/04/1974 &         NaN \\
   21201240 &           STA MARIA DE USME &  CON &       BOGOTA D.C. &  4.481306 & -74.126278 &  15/12/1977 &         NaN \\
   21201250 &        SAN PEDRO [21201250] &  CON &              SOPÓ &  4.871639 & -73.966667 &  15/06/1980 &  15/02/2013 \\
   21201270 &         TIBAR EL [21201270] &  CON &            MADRID &  4.816667 & -74.233333 &  15/12/1980 &  15/02/2002 \\
   21201300 &        AUSTRALIA [21201300] &  CON &       BOGOTA D.C. &  4.394250 & -74.132000 &  15/03/1985 &         NaN \\
   21201310 &          PREVENTORIO INFANT &  CON &            SIBATÉ &  4.465444 & -74.267500 &  15/03/1985 &         NaN \\
   21201320 &         UNION LA [21201320] &  CON &            SIBATÉ &  4.509361 & -74.268806 &  15/03/1985 &         NaN \\
   21201550 &         ROBLE EL [21201550] &  CON &            MADRID &  4.796667 & -74.226389 &  15/07/1985 &         NaN \\
   21201570 &          ESC COL INGENIERIA &  CON &       BOGOTA D.C. &  4.783333 & -74.050000 &  15/11/1981 &  15/04/1986 \\
   21201600 &           SEDE IDEAM KRA 10 &  CON &       BOGOTA D.C. &  4.607111 & -74.072889 &  15/09/1986 &         NaN \\
   21201610 &       SAN ISIDRO [21201610] &  CON &            GUASCA &  4.850289 & -73.890722 &  15/05/1987 &         NaN \\
   21201620 &        SUESCA    [21201620] &  CON &            SUESCA &  5.109583 & -73.796972 &  15/05/1987 &         NaN \\
   21201630 &        TABIO GJA [21201630] &  CON &             TABIO &  4.933056 & -74.065611 &  15/05/1987 &         NaN \\
   21201640 &             VILLAPINZON GJA &  CON &       VILLAPINZÓN &  5.263750 & -73.590861 &  15/05/1987 &         NaN \\
   21201650 &       STA ROSITA [21201650] &  CON &            SUESCA &  5.115917 & -73.757389 &  15/08/1988 &         NaN \\
   21201920 &             ALTO SAN MIGUEL &  CON &            SIBATÉ &  4.449667 & -74.299722 &  15/05/1993 &         NaN \\
   21201930 &        GUANQUICA [21201930] &  CON &             TAUSA &  5.184278 & -73.941111 &  15/02/1996 &         NaN \\
   21202100 &           IDEAM FONTIBON HB &  CON &       BOGOTA D.C. &  4.700000 & -74.166667 &  15/06/1998 &    11/12/08 \\
   21202160 &     HIDROPARAISO [21202160] &  CON &        EL COLEGIO &  4.573167 & -74.404833 &  15/01/1999 &         NaN \\
   21202280 &    SEDE IDEAM CALLE 25D KRA &  CON &       BOGOTA D.C. &  4.684000 & -74.129000 &    12/12/16 &         NaN \\
   21205013 &              UNISALLE NORTE &  CON &       BOGOTA D.C. &  4.794444 & -74.030556 &  14/12/2009 &    07/11/12 \\
   21205230 &            OBS MET NACIONAL &  CON &       BOGOTA D.C. &  4.633333 & -74.100000 &  15/03/1941 &  15/11/1993 \\
   21205420 &        TIBAITATA [21205420] &  CON &          MOSQUERA &  4.691417 & -74.209000 &  15/03/1954 &         NaN \\
   21205520 &          ELDORADO DIDACTICA &  CON &       BOGOTA D.C. &  4.700000 & -74.150000 &  15/01/1959 &  15/03/2000 \\
   21205580 &           VENADO ORO VIVERO &  CON &       BOGOTA D.C. &  4.598361 & -74.061556 &  15/08/1965 &         NaN \\
   21205600 &          VELODROMO 1 D MAYO &  CON &       BOGOTA D.C. &  4.616667 & -74.066667 &  15/03/1986 &  15/06/1993 \\
   21205660 &     MERCEDES LAS [21205660] &  CON &          ANAPOIMA &  4.581889 & -74.526611 &  15/09/1970 &         NaN \\
   21205670 &       FLORIDA LA [21205670] &  CON &          ANOLAIMA &  4.770889 & -74.437639 &  15/11/1970 &         NaN \\
   21205700 &           GUASCA [21205700] &  CON &            GUASCA &  4.879861 & -73.868111 &  15/07/1974 &         NaN \\
   21205710 &             JARDIN BOTANICO &  CON &       BOGOTA D.C. &  4.669333 & -74.102667 &  15/09/1974 &         NaN \\
   21205720 &    SAN JORGE GJA [21205720] &  CON &            SOACHA &  4.505750 & -74.189278 &  15/04/1960 &         NaN \\
   21205730 &            CENTRO MED ANDES &  CON &       BOGOTA D.C. &  4.698167 & -74.036833 &  15/01/1985 &    10/05/11 \\
   21205740 &            SILOS [21205740] &  CON &          CHOCONTÁ &  5.117722 & -73.701417 &  15/03/1974 &         NaN \\
   21205750 &           REP LOS MUCHACHOS &  CON &             FUNZA &  4.733333 & -74.166667 &  15/05/1976 &  15/06/1985 \\
   21205760 &          CLINICA SAN RAFAEL &  CON &       BOGOTA D.C. &  4.600000 & -74.083333 &  15/01/1985 &  15/01/1998 \\
   21205770 &           BASE AEREA MADRID &  CON &            MADRID &  4.728806 & -74.272500 &  15/07/1974 &         NaN \\
   21205780 &         SENA GJA [21205780] &  CON &          MOSQUERA &  4.700000 & -74.216667 &  15/07/1973 &  15/01/1980 \\
   21205790 &              APTO EL DORADO &  CON &       BOGOTA D.C. &  4.705583 & -74.150667 &  15/02/1972 &         NaN \\
   21205800 &          BOMBEROS DEL NORTE &  CON &       BOGOTA D.C. &  4.650000 & -74.066667 &  15/11/1972 &  15/11/1983 \\
   21205810 &          CAPITOLIO NACIONAL &  CON &       BOGOTA D.C. &  4.600000 & -74.083333 &  15/11/1972 &  15/10/1979 \\
   21205820 &          LICORERA BOGOINAMA &  CON &       BOGOTA D.C. &  4.616667 & -74.100000 &  15/11/1973 &  15/11/1983 \\
   21205830 &           MUZU CENTRO SALUD &  CON &       BOGOTA D.C. &  4.600000 & -74.133333 &  15/11/1972 &  15/10/1979 \\
   21205840 &        SENA K 30 [21205840] &  CON &       BOGOTA D.C. &  4.595361 & -74.111833 &  15/04/1985 &    10/05/11 \\
   21205850 &    COLOMBIANO EL [21205850] &  CON &          SESQUILÉ &  5.033889 & -73.848194 &  15/09/1976 &         NaN \\
   21205860 &         CORZO EL [21205860] &  CON &       BOGOTA D.C. &  4.650000 & -74.200000 &  15/09/1976 &  15/02/1992 \\
   21205870 &       SALITRE EL [21205870] &  CON &            BOJACÁ &  4.738889 & -74.334278 &  15/09/1976 &         NaN \\
   21205880 &              FLORES CHIBCHA &  CON &            MADRID &  4.789722 & -74.264778 &  15/09/1976 &         NaN \\
   21205890 &          GUANATA [21205890] &  CON &              CHÍA &  4.885944 & -74.054333 &  15/09/1976 &         NaN \\
   21205900 &      INDUQUIMICA [21205900] &  CON &            SOACHA &  4.583333 & -74.233333 &  15/09/1976 &  15/12/1977 \\
   21205910 &       COSECHA LA [21205910] &  CON &         ZIPAQUIRÁ &  4.989222 & -74.001194 &  15/09/1976 &         NaN \\
   21205920 &      SUASUQUE    [21205920] &  CON &              SOPÓ &  4.820833 & -73.963889 &  15/09/1976 &         NaN \\
   21205930 &       VILLA ROSA [21205930] &  CON &              COTA &  4.833333 & -74.100000 &  15/09/1976 &  15/09/1992 \\
   21205940 &       VILLA INES [21205940] &  CON &        FACATATIVÁ &  4.834972 & -74.383972 &  15/02/1977 &         NaN \\
   21205950 &              TIBACHOQUE HDA &  CON &             FUNZA &  4.759056 & -74.205167 &  15/02/1977 &    10/12/08 \\
   21205960 &            TACHI [21205960] &  CON &        SUBACHOQUE &  4.939056 & -74.152583 &  15/02/1977 &         NaN \\
   21205970 &      STA ANA HDA [21205970] &  CON &           NEMOCÓN &  5.090500 & -73.881250 &  15/02/1977 &         NaN \\
   21205980 &             PROVIDENCIA GJA &  CON &             TENJO &  4.792389 & -74.200917 &  15/02/1977 &         NaN \\
   21205990 &         LLANO EL [21205990] &  CON &              SOPÓ &  4.927778 & -73.950000 &  15/02/1977 &         NaN \\
   21206000 &         ADPOSTAL [21206000] &  CON &       BOGOTA D.C. &  4.680750 & -74.123639 &  15/04/1985 &    10/05/11 \\
   21206010 &     LORETOKI HDA [21206010] &  CON &            SUESCA &  5.089028 & -73.802750 &  15/02/1977 &         NaN \\
   21206020 &    SANTILLANA    [21206020] &  CON &             TABIO &  4.898528 & -74.104833 &  15/02/1977 &         NaN \\
   21206030 &     SAN CAYETANO [21206030] &  CON &        SUBACHOQUE &  4.916833 & -74.181667 &  15/02/1977 &         NaN \\
   21206040 &             ESAP [21206040] &  CON &       BOGOTA D.C. &  4.646778 & -74.096361 &  15/04/1985 &    08/09/09 \\
   21206050 &          ESC COL INGENIERIA &  CON &       BOGOTA D.C. &  4.783333 & -74.050000 &  15/04/1986 &    10/12/08 \\
   21206060 &       CASABLANCA [21206060] &  CON &            MADRID &  4.717111 & -74.253333 &  15/09/1976 &         NaN \\
   21206150 &           MOLINOS DEL NORTE &  CON &       BOGOTA D.C. &  4.700000 & -74.050000 &  15/10/1983 &  15/04/1986 \\
   21206160 &     HIDROPARAISO [21206160] &  CON &        EL COLEGIO &  4.573167 & -74.404833 &  15/04/1984 &  15/12/1998 \\
   21206170 &       CLARETIANO [21206170] &  CON &       BOGOTA D.C. &  4.600000 & -74.200000 &  15/08/1984 &  15/08/1985 \\
   21206190 &             UNIV PEDAGOGICA &  CON &       BOGOTA D.C. &  4.666667 & -74.066667 &  15/11/1986 &    10/12/08 \\
   21206200 &          TUNDAMA [21206200] &  CON &          MOSQUERA &  4.733333 & -74.250000 &  15/08/1986 &  15/02/1992 \\
   21206210 &          FLORES COLOMBIANAS &  CON &             FUNZA &  4.736250 & -74.157333 &  15/07/1986 &  15/02/2013 \\
   21206220 &               UNIV NACIONAL &  CON &       BOGOTA D.C. &  4.638083 & -74.089083 &  15/05/1987 &         NaN \\
   21206230 &               VEGAS LAS HDA &  CON &       BOGOTA D.C. &  4.661667 & -74.151419 &  15/08/1987 &    10/05/11 \\
   21206240 &             CENTRO GAVIOTAS &  CON &       BOGOTA D.C. &  4.600000 & -74.066667 &  15/08/1987 &  15/09/1992 \\
   21206250 &           CORITO [21206250] &  CON &        FACATATIVÁ &  4.800000 & -74.366667 &  15/06/1988 &  15/05/1991 \\
   21206260 &           C.UNIV.AGROP-UDCA &  CON &       BOGOTA D.C. &  4.798639 & -74.049722 &  15/12/1988 &         NaN \\
   21206280 &         ACAPULCO [21206280] &  CON &            BOJACÁ &  4.653833 & -74.333056 &  15/02/1990 &         NaN \\
   21206450 &        TERMOZIPA [21206450] &  CON &         TOCANCIPÁ &  4.983333 & -73.933333 &  15/05/1992 &  15/01/1996 \\
   21206490 &        HATO ALTO [21206490] &  CON &             TENJO &  4.835083 & -74.139917 &  15/08/1993 &  31/12/2012 \\
   21206500 &          COL ABRAHAM LINCOL &  CON &       BOGOTA D.C. &  4.756639 & -74.061583 &  15/10/1994 &    03/11/09 \\
   21206510 &             CASD [21206510] &  CON &       BOGOTA D.C. &  4.600000 & -74.083333 &  15/12/1995 &  15/08/2002 \\
   21206540 &            EDIFICIO PREMIUM &  CON &       BOGOTA D.C. &  4.686944 & -74.054222 &  15/01/1997 &    10/12/08 \\
   21206550 &               SENA MOSQUERA &  CON &          MOSQUERA &  4.700000 & -74.216667 &  15/01/1998 &  31/03/2008 \\
   21206560 &     INEM KENNEDY [21206560] &  CON &       BOGOTA D.C. &  4.661111 & -74.134778 &  15/07/1998 &         NaN \\
   21206570 &       APTO CATAM [21206570] &  CON &       BOGOTA D.C. &  4.705583 & -74.150667 &  15/01/2001 &         NaN \\
   21206610 &            EFRAIN CA\#AVERAL &  CON &       BOGOTA D.C. &  4.583333 & -74.066667 &  15/11/2001 &    05/08/09 \\
   21206620 &           COL H DURAN DUSAN &  CON &       BOGOTA D.C. &  4.634611 & -74.173750 &  15/11/2001 &         NaN \\
   21206630 &          BILBAO MAXIMO POTI &  CON &       BOGOTA D.C. &  4.751139 & -74.091583 &  15/11/2001 &         NaN \\
   21206640 &         SAN JOSE [21206640] &  CON &       BOGOTA D.C. &  4.501556 & -74.119306 &  15/11/2001 &    10/05/11 \\
   21206650 &            COL SAN CAYETANO &  CON &       BOGOTA D.C. &  4.516753 & -74.088222 &  15/11/2001 &         NaN \\
   21206660 &          COL SANTIAGO PEREZ &  CON &       BOGOTA D.C. &  4.576222 & -74.130917 &  15/11/2001 &         NaN \\
   21206670 &              COL BUCKINGHAM &  CON &       BOGOTA D.C. &  4.792056 & -74.049583 &  15/11/2001 &    12/11/09 \\
   21206680 &            COL NUEVO RETIRO &  CON &       BOGOTA D.C. &  4.734111 & -74.037028 &  15/11/2001 &    05/08/09 \\
   21206690 &          COL MIGUEL A. CARO &  CON &       BOGOTA D.C. &  4.813167 & -74.031111 &  15/11/2001 &         NaN \\
   21206700 &          CEA CENT.EST.AERO. &  CON &       BOGOTA D.C. &  4.691028 & -74.134417 &  15/08/2002 &    07/09/09 \\
   21206970 &             UNISALLE CENTRO &  CON &       BOGOTA D.C. &  4.595000 & -74.070361 &  24/04/2008 &         NaN \\
   21208670 &        GUANQUICA [21208670] &  CON &             TAUSA &  5.184278 & -73.941111 &  15/12/1954 &         NaN \\
   23065100 &         SABANETA [23065100] &  CON &     SAN FRANCISCO &  4.901750 & -74.307389 &  15/08/1986 &         NaN \\
   24010070 &          LETICIA [24010070] &  CON &       LENGUAZAQUE &  5.303194 & -73.709750 &  15/02/1974 &         NaN \\
   24010140 &         CUCUNUBA [24010140] &  CON &          CUCUNUBÁ &  5.251028 & -73.770750 &  15/01/1958 &         NaN \\
   24010170 &         GUACHETA [24010170] &  CON &          GUACHETÁ &  5.385889 & -73.691056 &  15/01/1958 &         NaN \\
   24010610 &            CARMEN DE CARUPA &  CON &  CARMEN DE CARUPA &  5.351278 & -73.904472 &  15/03/1974 &  31/03/2010 \\
   24010800 &        MINAS LAS [24010800] &  CON &            SAMACÁ &  5.483333 & -73.533333 &  15/07/1979 &  15/02/2002 \\
   24011060 &             SUSA [24011060] &  CON &              SUSA &  5.462444 & -73.801556 &  15/07/1984 &         NaN \\
   24011080 &       CUCUNUBA 1 [24011080] &  CON &          CUCUNUBÁ &  5.248000 & -73.752500 &  15/05/1987 &         NaN \\
   24011090 &        UBATE GJA [24011090] &  CON &             UBATÉ &  5.327333 & -73.791444 &  15/05/1987 &  31/03/2010 \\
   24011100 &          ISLA DEL SANTUARIO &  CON &           FÚQUENE &  5.467278 & -73.734806 &  15/04/1960 &         NaN \\
   24011150 &       ALIZOS LOS [24011150] &  CON &  CARMEN DE CARUPA &  5.329389 & -73.850056 &  31/03/2010 &         NaN \\
   24015120 &          ISLA DEL SANTUARIO &  CON &           FÚQUENE &  5.467278 & -73.734806 &  15/05/1942 &         NaN \\
   24015220 &   VILLA CARMEN   [24015220] &  CON &            SAMACÁ &  5.509389 & -73.495778 &  15/02/1968 &         NaN \\
   24015290 &        GACHANECA [24015290] &  CON &            SAMACÁ &  5.433333 & -73.550000 &  15/07/1979 &  15/04/1994 \\
   24015380 &            CARMEN DE CARUPA &  CON &  CARMEN DE CARUPA &  5.347222 & -73.898333 &  31/03/2010 &         NaN \\
   24017150 &             LA BOYERA AUTOM &  CON &             UBATÉ &  5.305972 & -73.855444 &  15/01/1960 &         NaN \\
   24017240 &         ANCON EL [24017240] &  CON &            SAMACÁ &  5.466667 & -73.533333 &  15/07/1968 &  15/02/2002 \\
   24017610 &      BOQUERON    [24017610] &  CON &       LENGUAZAQUE &  5.328250 & -73.699722 &  15/02/1974 &         NaN \\
   24017630 &        GACHANECA [24017630] &  CON &            SAMACÁ &  5.450000 & -73.550000 &  15/09/1969 &  15/09/1998 \\
   24017660 &          CANAL RUCHICAL BOC &  CON &            SAMACÁ &  5.483333 & -73.516667 &  15/05/1979 &  15/12/1998 \\
   24017670 &      AMARILLO EL [24017670] &  CON &            SAMACÁ &  5.500000 & -73.533333 &  15/05/1979 &  15/12/1998 \\
   24017680 &       REBOSADERO [24017680] &  CON &            SAMACÁ &  5.450000 & -73.533333 &  15/05/1979 &  15/06/1985 \\
   24017690 &              SALIDA EMBALSE &  CON &            SAMACÁ &  5.450000 & -73.533333 &  15/05/1979 &  15/06/1985 \\
   24017700 &           PTE EL [24017700] &  CON &            SAMACÁ &  5.456167 & -73.540111 &  15/05/1979 &  28/08/2014 \\
   24017720 &               CANAL PATAGUY &  CON &            SAMACÁ &  5.450000 & -73.500000 &  15/02/1979 &  15/12/1998 \\
   24017730 &                CUATRO COPAS &  CON &            SAMACÁ &  5.466667 & -73.533333 &  15/03/1981 &  15/12/1998 \\
   35020280 &          CHOACHI [35020280] &  CON &           CHOACHÍ &  4.522917 & -73.926583 &  15/11/1981 &         NaN \\
   35020290 &          FOMEQUE [35020290] &  CON &           FÓMEQUE &  4.486528 & -73.890417 &  15/11/1981 &         NaN \\
   35020320 &      LLANO LARGO [35020320] &  CON &            UBAQUE &  4.482833 & -74.030278 &  15/11/1984 &  15/04/1986 \\
   35020330 &         BOLSA LA [35020330] &  CON &           CHOACHÍ &  4.575417 & -73.981417 &  15/09/1985 &  15/05/1987 \\
   35025050 &      LLANO LARGO [35025050] &  CON &            UBAQUE &  4.482833 & -74.030278 &  15/04/1986 &         NaN \\
   35025060 &         BOLSA LA [35025060] &  CON &           CHOACHÍ &  4.575417 & -73.981417 &  15/05/1987 &         NaN \\
   35027100 &           CARAZA [35027100] &  CON &          CHIPAQUE &  4.428639 & -74.010194 &  15/01/1970 &         NaN \\
   35027220 &      LLANO LARGO [35027220] &  CON &            UBAQUE &  4.485056 & -74.030222 &  15/11/1984 &         NaN \\
   35027500 &             QDA RINCONAUTOM &  CON &         LA CALERA &  4.664944 & -73.857389 &    08/07/08 &         NaN \\
   35030080 &        CASAS LAS [35030080] &  CON &           CÁQUEZA &  4.441167 & -73.936389 &  15/05/1986 &         NaN \\
   35060020 &            SUEVA [35060020] &  CON &             JUNÍN &  4.810000 & -73.707167 &  15/07/1972 &         NaN \\
   35060160 &       POTRERITOS [35060160] &  CON &         GUATAVITA &  4.828806 & -73.769278 &  15/08/1972 &         NaN \\
   35060200 &     AMOLADERO EL [35060200] &  CON &         GUATAVITA &  4.857972 & -73.745389 &  15/04/1972 &         NaN \\
   35070020 &  VENTAQUEMADA    [35070020] &  CON &      VENTAQUEMADA &  5.383056 & -73.602889 &  15/03/1956 &         NaN \\
   35070030 &      TURMEQUE    [35070030] &  CON &          TURMEQUÉ &  5.317750 & -73.496361 &  15/07/1957 &         NaN \\
   35070040 &           TIBANA [35070040] &  CON &            TIBANÁ &  5.315278 & -73.395944 &  15/09/1958 &         NaN \\
   35070050 &           UMBITA [35070050] &  CON &            ÚMBITA &  5.219111 & -73.444556 &  15/07/1956 &         NaN \\
   35070060 &   QUINCHOS LOS   [35070060] &  CON &         CHINAVITA &  5.219556 & -73.347917 &  15/04/1971 &         NaN \\
   35070070 &     CHINAVITA    [35070070] &  CON &         CHINAVITA &  5.164861 & -73.364250 &  15/09/1955 &         NaN \\
   35070210 &        PACHAVITA [35070210] &  CON &         PACHAVITA &  5.139250 & -73.395639 &  15/03/1976 &         NaN \\
   35070230 &           MACHETA GJA AGROP &  CON &           MACHETÁ &  5.075111 & -73.579417 &  15/12/1979 &         NaN \\
   35070370 &            BELEN [35070370] &  CON &           MACHETÁ &  5.083333 & -73.566667 &  15/05/1962 &  15/10/1971 \\
   35070380 &      ROSALES LOS [35070380] &  CON &           MACHETÁ &  5.083333 & -73.616667 &  15/07/1965 &  15/10/1971 \\
   35075010 &    NUEVO COLON   [35075010] &  CON &       NUEVO COLÓN &  5.352694 & -73.453778 &  15/08/1965 &         NaN \\
   35077100 &       SAN JOSE   [35077100] &  CON &            SAMACÁ &  5.428639 & -73.528278 &  15/05/1988 &         NaN \\
\end{longtable}



\end{landscape} % primer Segundo
%\begin{figure}
    
    
	\begin{subfigure}[normla]{0.4\textwidth}
	\includegraphics[draft=false, scale=0.4]{validacion_convencionales/21205230_1_1.png}
		\caption{Conteo de no valores anuales para la estación Obs Met Nacional código 21205230.}
		\label{subfig:a1}
		\end{subfigure}
		~
    \begin{subfigure}[normla]{0.4\textwidth}
	\includegraphics[draft=false, scale=0.4]{validacion_convencionales/21205230_1_2.png}
		\caption{Conteo de no valores mensuales para la estación Obs Met Nacional código 21205230.}
		\label{subfig:a2}
		\end{subfigure}
		
    \begin{subfigure}[normla]{0.4\textwidth}
	\includegraphics[draft=false, scale=0.4]{validacion_convencionales/21205230_1_3.png}
		\caption{Conteo de saltos anuales para la estación Obs Met Nacional código 21205230.}
		\label{subfig:a1}
		\end{subfigure}
		~
    \begin{subfigure}[normla]{0.4\textwidth}
	\includegraphics[draft=false, scale=0.4]{validacion_convencionales/21205230_1_4.png}
		\caption{Conteo de saltos mensuales para la estación Obs Met Nacional código 21205230.}
		\label{subfig:a2}
		\end{subfigure}

	
\end{figure}
           
\begin{figure}[H]\ContinuedFloat
\centering
	\begin{subfigure}[normla]{0.4\textwidth}
	\includegraphics[draft=false, scale=0.4]{validacion_convencionales/21206020_1_1.png}
		\caption{Conteo de no valores anuales para la estación Santillana    código 21206020.}
		\label{subfig:a1}
		\end{subfigure}
		~
    \begin{subfigure}[normla]{0.4\textwidth}
	\includegraphics[draft=false, scale=0.4]{validacion_convencionales/21206020_1_2.png}
		\caption{Conteo de no valores mensuales para la estación Santillana    código 21206020.}
		\label{subfig:a2}
		\end{subfigure}
		
    \begin{subfigure}[normla]{0.4\textwidth}
	\includegraphics[draft=false, scale=0.4]{validacion_convencionales/21206020_1_3.png}
		\caption{Conteo de saltos anuales para la estación Santillana    código 21206020.}
		\label{subfig:a1}
		\end{subfigure}
		~
    \begin{subfigure}[normla]{0.4\textwidth}
	\includegraphics[draft=false, scale=0.4]{validacion_convencionales/21206020_1_4.png}
		\caption{Conteo de saltos mensuales para la estación Santillana    código 21206020.}
		\label{subfig:a2}
		\end{subfigure}

	
\end{figure}
           
\begin{figure}[H]\ContinuedFloat
\centering
	\begin{subfigure}[normla]{0.4\textwidth}
	\includegraphics[draft=false, scale=0.4]{validacion_convencionales/21206030_1_1.png}
		\caption{Conteo de no valores anuales para la estación San Cayetano código 21206030.}
		\label{subfig:a1}
		\end{subfigure}
		~
    \begin{subfigure}[normla]{0.4\textwidth}
	\includegraphics[draft=false, scale=0.4]{validacion_convencionales/21206030_1_2.png}
		\caption{Conteo de no valores mensuales para la estación San Cayetano código 21206030.}
		\label{subfig:a2}
		\end{subfigure}
		
    \begin{subfigure}[normla]{0.4\textwidth}
	\includegraphics[draft=false, scale=0.4]{validacion_convencionales/21206030_1_3.png}
		\caption{Conteo de saltos anuales para la estación San Cayetano código 21206030.}
		\label{subfig:a1}
		\end{subfigure}
		~
    \begin{subfigure}[normla]{0.4\textwidth}
	\includegraphics[draft=false, scale=0.4]{validacion_convencionales/21206030_1_4.png}
		\caption{Conteo de saltos mensuales para la estación San Cayetano código 21206030.}
		\label{subfig:a2}
		\end{subfigure}

	
\end{figure}
           
\begin{figure}[H]\ContinuedFloat
\centering
	\begin{subfigure}[normla]{0.4\textwidth}
	\includegraphics[draft=false, scale=0.4]{validacion_convencionales/21206050_1_1.png}
		\caption{Conteo de no valores anuales para la estación Esc Col Ingenieria código 21206050.}
		\label{subfig:a1}
		\end{subfigure}
		~
    \begin{subfigure}[normla]{0.4\textwidth}
	\includegraphics[draft=false, scale=0.4]{validacion_convencionales/21206050_1_2.png}
		\caption{Conteo de no valores mensuales para la estación Esc Col Ingenieria código 21206050.}
		\label{subfig:a2}
		\end{subfigure}
		
    \begin{subfigure}[normla]{0.4\textwidth}
	\includegraphics[draft=false, scale=0.4]{validacion_convencionales/21206050_1_3.png}
		\caption{Conteo de saltos anuales para la estación Esc Col Ingenieria código 21206050.}
		\label{subfig:a1}
		\end{subfigure}
		~
    \begin{subfigure}[normla]{0.4\textwidth}
	\includegraphics[draft=false, scale=0.4]{validacion_convencionales/21206050_1_4.png}
		\caption{Conteo de saltos mensuales para la estación Esc Col Ingenieria código 21206050.}
		\label{subfig:a2}
		\end{subfigure}

	
\end{figure}
           
\begin{figure}[H]\ContinuedFloat
\centering
	\begin{subfigure}[normla]{0.4\textwidth}
	\includegraphics[draft=false, scale=0.4]{validacion_convencionales/21205970_1_1.png}
		\caption{Conteo de no valores anuales para la estación Sta Ana Hda código 21205970.}
		\label{subfig:a1}
		\end{subfigure}
		~
    \begin{subfigure}[normla]{0.4\textwidth}
	\includegraphics[draft=false, scale=0.4]{validacion_convencionales/21205970_1_2.png}
		\caption{Conteo de no valores mensuales para la estación Sta Ana Hda código 21205970.}
		\label{subfig:a2}
		\end{subfigure}
		
    \begin{subfigure}[normla]{0.4\textwidth}
	\includegraphics[draft=false, scale=0.4]{validacion_convencionales/21205970_1_3.png}
		\caption{Conteo de saltos anuales para la estación Sta Ana Hda código 21205970.}
		\label{subfig:a1}
		\end{subfigure}
		~
    \begin{subfigure}[normla]{0.4\textwidth}
	\includegraphics[draft=false, scale=0.4]{validacion_convencionales/21205970_1_4.png}
		\caption{Conteo de saltos mensuales para la estación Sta Ana Hda código 21205970.}
		\label{subfig:a2}
		\end{subfigure}

	
\end{figure}
           
\begin{figure}[H]\ContinuedFloat
\centering
	\begin{subfigure}[normla]{0.4\textwidth}
	\includegraphics[draft=false, scale=0.4]{validacion_convencionales/21205980_1_1.png}
		\caption{Conteo de no valores anuales para la estación Providencia Gja código 21205980.}
		\label{subfig:a1}
		\end{subfigure}
		~
    \begin{subfigure}[normla]{0.4\textwidth}
	\includegraphics[draft=false, scale=0.4]{validacion_convencionales/21205980_1_2.png}
		\caption{Conteo de no valores mensuales para la estación Providencia Gja código 21205980.}
		\label{subfig:a2}
		\end{subfigure}
		
    \begin{subfigure}[normla]{0.4\textwidth}
	\includegraphics[draft=false, scale=0.4]{validacion_convencionales/21205980_1_3.png}
		\caption{Conteo de saltos anuales para la estación Providencia Gja código 21205980.}
		\label{subfig:a1}
		\end{subfigure}
		~
    \begin{subfigure}[normla]{0.4\textwidth}
	\includegraphics[draft=false, scale=0.4]{validacion_convencionales/21205980_1_4.png}
		\caption{Conteo de saltos mensuales para la estación Providencia Gja código 21205980.}
		\label{subfig:a2}
		\end{subfigure}

	
\end{figure}
           
\begin{figure}[H]
	\centering
	\begin{subfigure}[normla]{0.4\textwidth}
	\includegraphics[draft=false, scale=0.4]{validacion_convencionales/21205910_1_1.png}
		\caption{Conteo de no valores anuales para la estación Cosecha La código 21205910.}
		\label{subfig:a1}
		\end{subfigure}
		~
    \begin{subfigure}[normla]{0.4\textwidth}
	\includegraphics[draft=false, scale=0.4]{validacion_convencionales/21205910_1_2.png}
		\caption{Conteo de no valores mensuales para la estación Cosecha La código 21205910.}
		\label{subfig:a2}
		\end{subfigure}
		
    \begin{subfigure}[normla]{0.4\textwidth}
	\includegraphics[draft=false, scale=0.4]{validacion_convencionales/21205910_1_3.png}
		\caption{Conteo de saltos anuales para la estación Cosecha La código 21205910.}
		\label{subfig:a1}
		\end{subfigure}
		~
    \begin{subfigure}[normla]{0.4\textwidth}
	\includegraphics[draft=false, scale=0.4]{validacion_convencionales/21205910_1_4.png}
		\caption{Conteo de saltos mensuales para la estación Cosecha La código 21205910.}
		\label{subfig:a2}
		\end{subfigure}

	
\end{figure}
           
\begin{figure}[H]\ContinuedFloat
\centering
	\begin{subfigure}[normla]{0.4\textwidth}
	\includegraphics[draft=false, scale=0.4]{validacion_convencionales/21205920_1_1.png}
		\caption{Conteo de no valores anuales para la estación Suasuque    código 21205920.}
		\label{subfig:a1}
		\end{subfigure}
		~
    \begin{subfigure}[normla]{0.4\textwidth}
	\includegraphics[draft=false, scale=0.4]{validacion_convencionales/21205920_1_2.png}
		\caption{Conteo de no valores mensuales para la estación Suasuque    código 21205920.}
		\label{subfig:a2}
		\end{subfigure}
		
    \begin{subfigure}[normla]{0.4\textwidth}
	\includegraphics[draft=false, scale=0.4]{validacion_convencionales/21205920_1_3.png}
		\caption{Conteo de saltos anuales para la estación Suasuque    código 21205920.}
		\label{subfig:a1}
		\end{subfigure}
		~
    \begin{subfigure}[normla]{0.4\textwidth}
	\includegraphics[draft=false, scale=0.4]{validacion_convencionales/21205920_1_4.png}
		\caption{Conteo de saltos mensuales para la estación Suasuque    código 21205920.}
		\label{subfig:a2}
		\end{subfigure}

	
\end{figure}
           
\begin{figure}[H]\ContinuedFloat
\centering
	\begin{subfigure}[normla]{0.4\textwidth}
	\includegraphics[draft=false, scale=0.4]{validacion_convencionales/21205580_1_1.png}
		\caption{Conteo de no valores anuales para la estación Venado Oro Vivero código 21205580.}
		\label{subfig:a1}
		\end{subfigure}
		~
    \begin{subfigure}[normla]{0.4\textwidth}
	\includegraphics[draft=false, scale=0.4]{validacion_convencionales/21205580_1_2.png}
		\caption{Conteo de no valores mensuales para la estación Venado Oro Vivero código 21205580.}
		\label{subfig:a2}
		\end{subfigure}
		
    \begin{subfigure}[normla]{0.4\textwidth}
	\includegraphics[draft=false, scale=0.4]{validacion_convencionales/21205580_1_3.png}
		\caption{Conteo de saltos anuales para la estación Venado Oro Vivero código 21205580.}
		\label{subfig:a1}
		\end{subfigure}
		~
    \begin{subfigure}[normla]{0.4\textwidth}
	\includegraphics[draft=false, scale=0.4]{validacion_convencionales/21205580_1_4.png}
		\caption{Conteo de saltos mensuales para la estación Venado Oro Vivero código 21205580.}
		\label{subfig:a2}
		\end{subfigure}

	
\end{figure}
           
\begin{figure}[H]\ContinuedFloat
\centering
	\begin{subfigure}[normla]{0.4\textwidth}
	\includegraphics[draft=false, scale=0.4]{validacion_convencionales/21205700_1_1.png}
		\caption{Conteo de no valores anuales para la estación Guasca código 21205700.}
		\label{subfig:a1}
		\end{subfigure}
		~
    \begin{subfigure}[normla]{0.4\textwidth}
	\includegraphics[draft=false, scale=0.4]{validacion_convencionales/21205700_1_2.png}
		\caption{Conteo de no valores mensuales para la estación Guasca código 21205700.}
		\label{subfig:a2}
		\end{subfigure}
		
    \begin{subfigure}[normla]{0.4\textwidth}
	\includegraphics[draft=false, scale=0.4]{validacion_convencionales/21205700_1_3.png}
		\caption{Conteo de saltos anuales para la estación Guasca código 21205700.}
		\label{subfig:a1}
		\end{subfigure}
		~
    \begin{subfigure}[normla]{0.4\textwidth}
	\includegraphics[draft=false, scale=0.4]{validacion_convencionales/21205700_1_4.png}
		\caption{Conteo de saltos mensuales para la estación Guasca código 21205700.}
		\label{subfig:a2}
		\end{subfigure}

	
\end{figure}
           
\begin{figure}[H]\ContinuedFloat
\centering
	\begin{subfigure}[normla]{0.4\textwidth}
	\includegraphics[draft=false, scale=0.4]{validacion_convencionales/21205740_1_1.png}
		\caption{Conteo de no valores anuales para la estación Silos código 21205740.}
		\label{subfig:a1}
		\end{subfigure}
		~
    \begin{subfigure}[normla]{0.4\textwidth}
	\includegraphics[draft=false, scale=0.4]{validacion_convencionales/21205740_1_2.png}
		\caption{Conteo de no valores mensuales para la estación Silos código 21205740.}
		\label{subfig:a2}
		\end{subfigure}
		
    \begin{subfigure}[normla]{0.4\textwidth}
	\includegraphics[draft=false, scale=0.4]{validacion_convencionales/21205740_1_3.png}
		\caption{Conteo de saltos anuales para la estación Silos código 21205740.}
		\label{subfig:a1}
		\end{subfigure}
		~
    \begin{subfigure}[normla]{0.4\textwidth}
	\includegraphics[draft=false, scale=0.4]{validacion_convencionales/21205740_1_4.png}
		\caption{Conteo de saltos mensuales para la estación Silos código 21205740.}
		\label{subfig:a2}
		\end{subfigure}

	
\end{figure}
           
\begin{figure}[H]\ContinuedFloat
\centering
	\begin{subfigure}[normla]{0.4\textwidth}
	\includegraphics[draft=false, scale=0.4]{validacion_convencionales/21205750_1_1.png}
		\caption{Conteo de no valores anuales para la estación Rep Los Muchachos código 21205750.}
		\label{subfig:a1}
		\end{subfigure}
		~
    \begin{subfigure}[normla]{0.4\textwidth}
	\includegraphics[draft=false, scale=0.4]{validacion_convencionales/21205750_1_2.png}
		\caption{Conteo de no valores mensuales para la estación Rep Los Muchachos código 21205750.}
		\label{subfig:a2}
		\end{subfigure}
		
    \begin{subfigure}[normla]{0.4\textwidth}
	\includegraphics[draft=false, scale=0.4]{validacion_convencionales/21205750_1_3.png}
		\caption{Conteo de saltos anuales para la estación Rep Los Muchachos código 21205750.}
		\label{subfig:a1}
		\end{subfigure}
		~
    \begin{subfigure}[normla]{0.4\textwidth}
	\includegraphics[draft=false, scale=0.4]{validacion_convencionales/21205750_1_4.png}
		\caption{Conteo de saltos mensuales para la estación Rep Los Muchachos código 21205750.}
		\label{subfig:a2}
		\end{subfigure}

	
\end{figure}
           
\begin{figure}[H]
	\centering
	\begin{subfigure}[normla]{0.4\textwidth}
	\includegraphics[draft=false, scale=0.4]{validacion_convencionales/21205790_1_1.png}
		\caption{Conteo de no valores anuales para la estación Apto El Dorado código 21205790.}
		\label{subfig:a1}
		\end{subfigure}
		~
    \begin{subfigure}[normla]{0.4\textwidth}
	\includegraphics[draft=false, scale=0.4]{validacion_convencionales/21205790_1_2.png}
		\caption{Conteo de no valores mensuales para la estación Apto El Dorado código 21205790.}
		\label{subfig:a2}
		\end{subfigure}
		
    \begin{subfigure}[normla]{0.4\textwidth}
	\includegraphics[draft=false, scale=0.4]{validacion_convencionales/21205790_1_3.png}
		\caption{Conteo de saltos anuales para la estación Apto El Dorado código 21205790.}
		\label{subfig:a1}
		\end{subfigure}
		~
    \begin{subfigure}[normla]{0.4\textwidth}
	\includegraphics[draft=false, scale=0.4]{validacion_convencionales/21205790_1_4.png}
		\caption{Conteo de saltos mensuales para la estación Apto El Dorado código 21205790.}
		\label{subfig:a2}
		\end{subfigure}

	
\end{figure}
           
\begin{figure}[H]\ContinuedFloat
\centering
	\begin{subfigure}[normla]{0.4\textwidth}
	\includegraphics[draft=false, scale=0.4]{validacion_convencionales/21205850_1_1.png}
		\caption{Conteo de no valores anuales para la estación Colombiano El código 21205850.}
		\label{subfig:a1}
		\end{subfigure}
		~
    \begin{subfigure}[normla]{0.4\textwidth}
	\includegraphics[draft=false, scale=0.4]{validacion_convencionales/21205850_1_2.png}
		\caption{Conteo de no valores mensuales para la estación Colombiano El código 21205850.}
		\label{subfig:a2}
		\end{subfigure}
		
    \begin{subfigure}[normla]{0.4\textwidth}
	\includegraphics[draft=false, scale=0.4]{validacion_convencionales/21205850_1_3.png}
		\caption{Conteo de saltos anuales para la estación Colombiano El código 21205850.}
		\label{subfig:a1}
		\end{subfigure}
		~
    \begin{subfigure}[normla]{0.4\textwidth}
	\includegraphics[draft=false, scale=0.4]{validacion_convencionales/21205850_1_4.png}
		\caption{Conteo de saltos mensuales para la estación Colombiano El código 21205850.}
		\label{subfig:a2}
		\end{subfigure}

	
\end{figure}
           
\begin{figure}[H]\ContinuedFloat
\centering
	\begin{subfigure}[normla]{0.4\textwidth}
	\includegraphics[draft=false, scale=0.4]{validacion_convencionales/21205870_1_1.png}
		\caption{Conteo de no valores anuales para la estación Salitre El código 21205870.}
		\label{subfig:a1}
		\end{subfigure}
		~
    \begin{subfigure}[normla]{0.4\textwidth}
	\includegraphics[draft=false, scale=0.4]{validacion_convencionales/21205870_1_2.png}
		\caption{Conteo de no valores mensuales para la estación Salitre El código 21205870.}
		\label{subfig:a2}
		\end{subfigure}
		
    \begin{subfigure}[normla]{0.4\textwidth}
	\includegraphics[draft=false, scale=0.4]{validacion_convencionales/21205870_1_3.png}
		\caption{Conteo de saltos anuales para la estación Salitre El código 21205870.}
		\label{subfig:a1}
		\end{subfigure}
		~
    \begin{subfigure}[normla]{0.4\textwidth}
	\includegraphics[draft=false, scale=0.4]{validacion_convencionales/21205870_1_4.png}
		\caption{Conteo de saltos mensuales para la estación Salitre El código 21205870.}
		\label{subfig:a2}
		\end{subfigure}

	
\end{figure}
           
\begin{figure}[H]\ContinuedFloat
\centering
	\begin{subfigure}[normla]{0.4\textwidth}
	\includegraphics[draft=false, scale=0.4]{validacion_convencionales/21205940_1_1.png}
		\caption{Conteo de no valores anuales para la estación Villa Ines código 21205940.}
		\label{subfig:a1}
		\end{subfigure}
		~
    \begin{subfigure}[normla]{0.4\textwidth}
	\includegraphics[draft=false, scale=0.4]{validacion_convencionales/21205940_1_2.png}
		\caption{Conteo de no valores mensuales para la estación Villa Ines código 21205940.}
		\label{subfig:a2}
		\end{subfigure}
		
    \begin{subfigure}[normla]{0.4\textwidth}
	\includegraphics[draft=false, scale=0.4]{validacion_convencionales/21205940_1_3.png}
		\caption{Conteo de saltos anuales para la estación Villa Ines código 21205940.}
		\label{subfig:a1}
		\end{subfigure}
		~
    \begin{subfigure}[normla]{0.4\textwidth}
	\includegraphics[draft=false, scale=0.4]{validacion_convencionales/21205940_1_4.png}
		\caption{Conteo de saltos mensuales para la estación Villa Ines código 21205940.}
		\label{subfig:a2}
		\end{subfigure}

	
\end{figure}
           
\begin{figure}[H]\ContinuedFloat
\centering
	\begin{subfigure}[normla]{0.4\textwidth}
	\includegraphics[draft=false, scale=0.4]{validacion_convencionales/21206150_1_1.png}
		\caption{Conteo de no valores anuales para la estación Molinos Del Norte código 21206150.}
		\label{subfig:a1}
		\end{subfigure}
		~
    \begin{subfigure}[normla]{0.4\textwidth}
	\includegraphics[draft=false, scale=0.4]{validacion_convencionales/21206150_1_2.png}
		\caption{Conteo de no valores mensuales para la estación Molinos Del Norte código 21206150.}
		\label{subfig:a2}
		\end{subfigure}
		
    \begin{subfigure}[normla]{0.4\textwidth}
	\includegraphics[draft=false, scale=0.4]{validacion_convencionales/21206150_1_3.png}
		\caption{Conteo de saltos anuales para la estación Molinos Del Norte código 21206150.}
		\label{subfig:a1}
		\end{subfigure}
		~
    \begin{subfigure}[normla]{0.4\textwidth}
	\includegraphics[draft=false, scale=0.4]{validacion_convencionales/21206150_1_4.png}
		\caption{Conteo de saltos mensuales para la estación Molinos Del Norte código 21206150.}
		\label{subfig:a2}
		\end{subfigure}

	
\end{figure}
           
\begin{figure}[H]\ContinuedFloat
\centering
	\begin{subfigure}[normla]{0.4\textwidth}
	\includegraphics[draft=false, scale=0.4]{validacion_convencionales/21206170_1_1.png}
		\caption{Conteo de no valores anuales para la estación Claretiano código 21206170.}
		\label{subfig:a1}
		\end{subfigure}
		~
    \begin{subfigure}[normla]{0.4\textwidth}
	\includegraphics[draft=false, scale=0.4]{validacion_convencionales/21206170_1_2.png}
		\caption{Conteo de no valores mensuales para la estación Claretiano código 21206170.}
		\label{subfig:a2}
		\end{subfigure}
		
    \begin{subfigure}[normla]{0.4\textwidth}
	\includegraphics[draft=false, scale=0.4]{validacion_convencionales/21206170_1_3.png}
		\caption{Conteo de saltos anuales para la estación Claretiano código 21206170.}
		\label{subfig:a1}
		\end{subfigure}
		~
    \begin{subfigure}[normla]{0.4\textwidth}
	\includegraphics[draft=false, scale=0.4]{validacion_convencionales/21206170_1_4.png}
		\caption{Conteo de saltos mensuales para la estación Claretiano código 21206170.}
		\label{subfig:a2}
		\end{subfigure}

	
\end{figure}
           
           
\begin{figure}[H]
	\centering
	\begin{subfigure}[normla]{0.4\textwidth}
	\includegraphics[draft=false, scale=0.4]{validacion_convencionales/21205420_1_1.png}
		\caption{Conteo de no valores anuales para la estación Tibaitata código 21205420.}
		\label{subfig:a1}
		\end{subfigure}
		~
    \begin{subfigure}[normla]{0.4\textwidth}
	\includegraphics[draft=false, scale=0.4]{validacion_convencionales/21205420_1_2.png}
		\caption{Conteo de no valores mensuales para la estación Tibaitata código 21205420.}
		\label{subfig:a2}
		\end{subfigure}
		
    \begin{subfigure}[normla]{0.4\textwidth}
	\includegraphics[draft=false, scale=0.4]{validacion_convencionales/21205420_1_3.png}
		\caption{Conteo de saltos anuales para la estación Tibaitata código 21205420.}
		\label{subfig:a1}
		\end{subfigure}
		~
    \begin{subfigure}[normla]{0.4\textwidth}
	\includegraphics[draft=false, scale=0.4]{validacion_convencionales/21205420_1_4.png}
		\caption{Conteo de saltos mensuales para la estación Tibaitata código 21205420.}
		\label{subfig:a2}
		\end{subfigure}

	
\end{figure}
           
\begin{figure}[H]\ContinuedFloat
\centering
	\begin{subfigure}[normla]{0.4\textwidth}
	\includegraphics[draft=false, scale=0.4]{validacion_convencionales/21205520_1_1.png}
		\caption{Conteo de no valores anuales para la estación Eldorado Didactica código 21205520.}
		\label{subfig:a1}
		\end{subfigure}
		~
    \begin{subfigure}[normla]{0.4\textwidth}
	\includegraphics[draft=false, scale=0.4]{validacion_convencionales/21205520_1_2.png}
		\caption{Conteo de no valores mensuales para la estación Eldorado Didactica código 21205520.}
		\label{subfig:a2}
		\end{subfigure}
		
    \begin{subfigure}[normla]{0.4\textwidth}
	\includegraphics[draft=false, scale=0.4]{validacion_convencionales/21205520_1_3.png}
		\caption{Conteo de saltos anuales para la estación Eldorado Didactica código 21205520.}
		\label{subfig:a1}
		\end{subfigure}
		~
    \begin{subfigure}[normla]{0.4\textwidth}
	\includegraphics[draft=false, scale=0.4]{validacion_convencionales/21205520_1_4.png}
		\caption{Conteo de saltos mensuales para la estación Eldorado Didactica código 21205520.}
		\label{subfig:a2}
		\end{subfigure}

	
\end{figure}
           
\begin{figure}[H]\ContinuedFloat
\centering
	\begin{subfigure}[normla]{0.4\textwidth}
	\includegraphics[draft=false, scale=0.4]{validacion_convencionales/21205710_1_1.png}
		\caption{Conteo de no valores anuales para la estación Jardin Botanico código 21205710.}
		\label{subfig:a1}
		\end{subfigure}
		~
    \begin{subfigure}[normla]{0.4\textwidth}
	\includegraphics[draft=false, scale=0.4]{validacion_convencionales/21205710_1_2.png}
		\caption{Conteo de no valores mensuales para la estación Jardin Botanico código 21205710.}
		\label{subfig:a2}
		\end{subfigure}
		
    \begin{subfigure}[normla]{0.4\textwidth}
	\includegraphics[draft=false, scale=0.4]{validacion_convencionales/21205710_1_3.png}
		\caption{Conteo de saltos anuales para la estación Jardin Botanico código 21205710.}
		\label{subfig:a1}
		\end{subfigure}
		~
    \begin{subfigure}[normla]{0.4\textwidth}
	\includegraphics[draft=false, scale=0.4]{validacion_convencionales/21205710_1_4.png}
		\caption{Conteo de saltos mensuales para la estación Jardin Botanico código 21205710.}
		\label{subfig:a2}
		\end{subfigure}

	
\end{figure}
           
\begin{figure}[H]\ContinuedFloat
\centering
	\begin{subfigure}[normla]{0.4\textwidth}
	\includegraphics[draft=false, scale=0.4]{validacion_convencionales/21205720_1_1.png}
		\caption{Conteo de no valores anuales para la estación San Jorge Gja código 21205720.}
		\label{subfig:a1}
		\end{subfigure}
		~
    \begin{subfigure}[normla]{0.4\textwidth}
	\includegraphics[draft=false, scale=0.4]{validacion_convencionales/21205720_1_2.png}
		\caption{Conteo de no valores mensuales para la estación San Jorge Gja código 21205720.}
		\label{subfig:a2}
		\end{subfigure}
		
    \begin{subfigure}[normla]{0.4\textwidth}
	\includegraphics[draft=false, scale=0.4]{validacion_convencionales/21205720_1_3.png}
		\caption{Conteo de saltos anuales para la estación San Jorge Gja código 21205720.}
		\label{subfig:a1}
		\end{subfigure}
		~
    \begin{subfigure}[normla]{0.4\textwidth}
	\includegraphics[draft=false, scale=0.4]{validacion_convencionales/21205720_1_4.png}
		\caption{Conteo de saltos mensuales para la estación San Jorge Gja código 21205720.}
		\label{subfig:a2}
		\end{subfigure}

	
\end{figure}
           
\begin{figure}[H]\ContinuedFloat
\centering
	\begin{subfigure}[normla]{0.4\textwidth}
	\includegraphics[draft=false, scale=0.4]{validacion_convencionales/21205770_1_1.png}
		\caption{Conteo de no valores anuales para la estación Base Aerea Madrid código 21205770.}
		\label{subfig:a1}
		\end{subfigure}
		~
    \begin{subfigure}[normla]{0.4\textwidth}
	\includegraphics[draft=false, scale=0.4]{validacion_convencionales/21205770_1_2.png}
		\caption{Conteo de no valores mensuales para la estación Base Aerea Madrid código 21205770.}
		\label{subfig:a2}
		\end{subfigure}
		
    \begin{subfigure}[normla]{0.4\textwidth}
	\includegraphics[draft=false, scale=0.4]{validacion_convencionales/21205770_1_3.png}
		\caption{Conteo de saltos anuales para la estación Base Aerea Madrid código 21205770.}
		\label{subfig:a1}
		\end{subfigure}
		~
    \begin{subfigure}[normla]{0.4\textwidth}
	\includegraphics[draft=false, scale=0.4]{validacion_convencionales/21205770_1_4.png}
		\caption{Conteo de saltos mensuales para la estación Base Aerea Madrid código 21205770.}
		\label{subfig:a2}
		\end{subfigure}

	
\end{figure}
           
\begin{figure}[H]\ContinuedFloat
\centering
\end{figure}
           
\begin{figure}[H]\ContinuedFloat
\centering
	\begin{subfigure}[normla]{0.4\textwidth}
	\includegraphics[draft=false, scale=0.4]{validacion_convencionales/21205840_1_1.png}
		\caption{Conteo de no valores anuales para la estación Sena K 30 código 21205840.}
		\label{subfig:a1}
		\end{subfigure}
		~
    \begin{subfigure}[normla]{0.4\textwidth}
	\includegraphics[draft=false, scale=0.4]{validacion_convencionales/21205840_1_2.png}
		\caption{Conteo de no valores mensuales para la estación Sena K 30 código 21205840.}
		\label{subfig:a2}
		\end{subfigure}
		
    \begin{subfigure}[normla]{0.4\textwidth}
	\includegraphics[draft=false, scale=0.4]{validacion_convencionales/21205840_1_3.png}
		\caption{Conteo de saltos anuales para la estación Sena K 30 código 21205840.}
		\label{subfig:a1}
		\end{subfigure}
		~
    \begin{subfigure}[normla]{0.4\textwidth}
	\includegraphics[draft=false, scale=0.4]{validacion_convencionales/21205840_1_4.png}
		\caption{Conteo de saltos mensuales para la estación Sena K 30 código 21205840.}
		\label{subfig:a2}
		\end{subfigure}

	
\end{figure}
           
\begin{figure}[H]
	\centering
	\begin{subfigure}[normla]{0.4\textwidth}
	\includegraphics[draft=false, scale=0.4]{validacion_convencionales/21205880_1_1.png}
		\caption{Conteo de no valores anuales para la estación Flores Chibcha código 21205880.}
		\label{subfig:a1}
		\end{subfigure}
		~
    \begin{subfigure}[normla]{0.4\textwidth}
	\includegraphics[draft=false, scale=0.4]{validacion_convencionales/21205880_1_2.png}
		\caption{Conteo de no valores mensuales para la estación Flores Chibcha código 21205880.}
		\label{subfig:a2}
		\end{subfigure}
		
    \begin{subfigure}[normla]{0.4\textwidth}
	\includegraphics[draft=false, scale=0.4]{validacion_convencionales/21205880_1_3.png}
		\caption{Conteo de saltos anuales para la estación Flores Chibcha código 21205880.}
		\label{subfig:a1}
		\end{subfigure}
		~
    \begin{subfigure}[normla]{0.4\textwidth}
	\includegraphics[draft=false, scale=0.4]{validacion_convencionales/21205880_1_4.png}
		\caption{Conteo de saltos mensuales para la estación Flores Chibcha código 21205880.}
		\label{subfig:a2}
		\end{subfigure}

	
\end{figure}
           
\begin{figure}[H]\ContinuedFloat
\centering
	\begin{subfigure}[normla]{0.4\textwidth}
	\includegraphics[draft=false, scale=0.4]{validacion_convencionales/21205890_1_1.png}
		\caption{Conteo de no valores anuales para la estación Guanata código 21205890.}
		\label{subfig:a1}
		\end{subfigure}
		~
    \begin{subfigure}[normla]{0.4\textwidth}
	\includegraphics[draft=false, scale=0.4]{validacion_convencionales/21205890_1_2.png}
		\caption{Conteo de no valores mensuales para la estación Guanata código 21205890.}
		\label{subfig:a2}
		\end{subfigure}
		
    \begin{subfigure}[normla]{0.4\textwidth}
	\includegraphics[draft=false, scale=0.4]{validacion_convencionales/21205890_1_3.png}
		\caption{Conteo de saltos anuales para la estación Guanata código 21205890.}
		\label{subfig:a1}
		\end{subfigure}
		~
    \begin{subfigure}[normla]{0.4\textwidth}
	\includegraphics[draft=false, scale=0.4]{validacion_convencionales/21205890_1_4.png}
		\caption{Conteo de saltos mensuales para la estación Guanata código 21205890.}
		\label{subfig:a2}
		\end{subfigure}

	
\end{figure}
           
\begin{figure}[H]\ContinuedFloat
\centering
\end{figure}
           
\begin{figure}[H]\ContinuedFloat
\centering
\end{figure}
           
\begin{figure}[H]\ContinuedFloat
\centering
	\begin{subfigure}[normla]{0.4\textwidth}
	\includegraphics[draft=false, scale=0.4]{validacion_convencionales/21205960_1_1.png}
		\caption{Conteo de no valores anuales para la estación Tachi código 21205960.}
		\label{subfig:a1}
		\end{subfigure}
		~
    \begin{subfigure}[normla]{0.4\textwidth}
	\includegraphics[draft=false, scale=0.4]{validacion_convencionales/21205960_1_2.png}
		\caption{Conteo de no valores mensuales para la estación Tachi código 21205960.}
		\label{subfig:a2}
		\end{subfigure}
		
    \begin{subfigure}[normla]{0.4\textwidth}
	\includegraphics[draft=false, scale=0.4]{validacion_convencionales/21205960_1_3.png}
		\caption{Conteo de saltos anuales para la estación Tachi código 21205960.}
		\label{subfig:a1}
		\end{subfigure}
		~
    \begin{subfigure}[normla]{0.4\textwidth}
	\includegraphics[draft=false, scale=0.4]{validacion_convencionales/21205960_1_4.png}
		\caption{Conteo de saltos mensuales para la estación Tachi código 21205960.}
		\label{subfig:a2}
		\end{subfigure}

	
\end{figure}
           
\begin{figure}[H]\ContinuedFloat
\centering
	\begin{subfigure}[normla]{0.4\textwidth}
	\includegraphics[draft=false, scale=0.4]{validacion_convencionales/21205990_1_1.png}
		\caption{Conteo de no valores anuales para la estación Llano El código 21205990.}
		\label{subfig:a1}
		\end{subfigure}
		~
    \begin{subfigure}[normla]{0.4\textwidth}
	\includegraphics[draft=false, scale=0.4]{validacion_convencionales/21205990_1_2.png}
		\caption{Conteo de no valores mensuales para la estación Llano El código 21205990.}
		\label{subfig:a2}
		\end{subfigure}
		
    \begin{subfigure}[normla]{0.4\textwidth}
	\includegraphics[draft=false, scale=0.4]{validacion_convencionales/21205990_1_3.png}
		\caption{Conteo de saltos anuales para la estación Llano El código 21205990.}
		\label{subfig:a1}
		\end{subfigure}
		~
    \begin{subfigure}[normla]{0.4\textwidth}
	\includegraphics[draft=false, scale=0.4]{validacion_convencionales/21205990_1_4.png}
		\caption{Conteo de saltos mensuales para la estación Llano El código 21205990.}
		\label{subfig:a2}
		\end{subfigure}

	
\end{figure}
           
\begin{figure}[H]\ContinuedFloat
\centering
	\begin{subfigure}[normla]{0.4\textwidth}
	\includegraphics[draft=false, scale=0.4]{validacion_convencionales/21206010_1_1.png}
		\caption{Conteo de no valores anuales para la estación Loretoki Hda código 21206010.}
		\label{subfig:a1}
		\end{subfigure}
		~
    \begin{subfigure}[normla]{0.4\textwidth}
	\includegraphics[draft=false, scale=0.4]{validacion_convencionales/21206010_1_2.png}
		\caption{Conteo de no valores mensuales para la estación Loretoki Hda código 21206010.}
		\label{subfig:a2}
		\end{subfigure}
		
    \begin{subfigure}[normla]{0.4\textwidth}
	\includegraphics[draft=false, scale=0.4]{validacion_convencionales/21206010_1_3.png}
		\caption{Conteo de saltos anuales para la estación Loretoki Hda código 21206010.}
		\label{subfig:a1}
		\end{subfigure}
		~
    \begin{subfigure}[normla]{0.4\textwidth}
	\includegraphics[draft=false, scale=0.4]{validacion_convencionales/21206010_1_4.png}
		\caption{Conteo de saltos mensuales para la estación Loretoki Hda código 21206010.}
		\label{subfig:a2}
		\end{subfigure}

	
\end{figure}
           
           
\begin{figure}[H]
	\centering
	\begin{subfigure}[normla]{0.4\textwidth}
	\includegraphics[draft=false, scale=0.4]{validacion_convencionales/35025050_1_1.png}
		\caption{Conteo de no valores anuales para la estación Llano Largo código 35025050.}
		\label{subfig:a1}
		\end{subfigure}
		~
    \begin{subfigure}[normla]{0.4\textwidth}
	\includegraphics[draft=false, scale=0.4]{validacion_convencionales/35025050_1_2.png}
		\caption{Conteo de no valores mensuales para la estación Llano Largo código 35025050.}
		\label{subfig:a2}
		\end{subfigure}
		
    \begin{subfigure}[normla]{0.4\textwidth}
	\includegraphics[draft=false, scale=0.4]{validacion_convencionales/35025050_1_3.png}
		\caption{Conteo de saltos anuales para la estación Llano Largo código 35025050.}
		\label{subfig:a1}
		\end{subfigure}
		~
    \begin{subfigure}[normla]{0.4\textwidth}
	\includegraphics[draft=false, scale=0.4]{validacion_convencionales/35025050_1_4.png}
		\caption{Conteo de saltos mensuales para la estación Llano Largo código 35025050.}
		\label{subfig:a2}
		\end{subfigure}

	
\end{figure}
           
\begin{figure}[H]\ContinuedFloat
\centering
	\begin{subfigure}[normla]{0.4\textwidth}
	\includegraphics[draft=false, scale=0.4]{validacion_convencionales/35025060_1_1.png}
		\caption{Conteo de no valores anuales para la estación Bolsa La código 35025060.}
		\label{subfig:a1}
		\end{subfigure}
		~
    \begin{subfigure}[normla]{0.4\textwidth}
	\includegraphics[draft=false, scale=0.4]{validacion_convencionales/35025060_1_2.png}
		\caption{Conteo de no valores mensuales para la estación Bolsa La código 35025060.}
		\label{subfig:a2}
		\end{subfigure}
		
    \begin{subfigure}[normla]{0.4\textwidth}
	\includegraphics[draft=false, scale=0.4]{validacion_convencionales/35025060_1_3.png}
		\caption{Conteo de saltos anuales para la estación Bolsa La código 35025060.}
		\label{subfig:a1}
		\end{subfigure}
		~
    \begin{subfigure}[normla]{0.4\textwidth}
	\includegraphics[draft=false, scale=0.4]{validacion_convencionales/35025060_1_4.png}
		\caption{Conteo de saltos mensuales para la estación Bolsa La código 35025060.}
		\label{subfig:a2}
		\end{subfigure}

\end{figure}
           
 % Segundo pero tiene problemas porque %son %muchos archivos
%%\appendix
\clearpage
%\addappheadtotoc
%\appendixpage

\chapter{Resumen del control de calidad realizado a las estaciones convencionales de la zona de estudio.}
\label{anexo:resumen_control_calidad_est_con2}


\begin{longtable}{llrrr}
\caption{Tabla resumen del control de calidad de las estaciones}
\label{tab:control_cal_conv}\\
\hline
   Código &               Tipo &  No datos (\%) &  P. Saltos (\%) &  Total de datos \\
\midrule
\endhead
\midrule
\multicolumn{3}{r}{{Continúa en la siguiente página}} \\
\midrule
\endfoot

\bottomrule
\endlastfoot
 21205013 &  Máximos absolutos &       38.5286 &         0.0000 &          1033.0 \\
 21205013 &      Mínmos medios &       34.0426 &         0.0000 &          1034.0 \\
 21205013 &     Valores medios &       34.5595 &         0.0000 &          1033.0 \\
 21205230 &  Máximos absolutos &       29.6816 &         0.0540 &          5559.0 \\
 21205230 &      Mínmos medios &       30.6835 &         0.0180 &          5560.0 \\
 21205230 &     Valores medios &       24.0691 &         0.0180 &          5559.0 \\
 21205420 &  Máximos absolutos &       20.4297 &         0.0000 &         16618.0 \\
 21205420 &      Mínmos medios &        5.6084 &         0.0000 &         16618.0 \\
 21205420 &     Valores medios &       18.4739 &         0.0000 &         16618.0 \\
 21205520 &  Máximos absolutos &       23.0316 &         0.0000 &          6782.0 \\
 21205520 &      Mínmos medios &       15.7747 &         0.0147 &          6783.0 \\
 21205520 &     Valores medios &       14.0961 &         0.0000 &          6782.0 \\
 21205580 &  Máximos absolutos &       45.2998 &         0.0140 &         14276.0 \\
 21205580 &      Mínmos medios &       27.8299 &         0.0210 &         14276.0 \\
 21206230 &     Valores medios &        9.4621 &         0.0000 &          7641.0 \\
 21206240 &  Máximos absolutos &       93.1673 &         0.0579 &          1727.0 \\
 21206240 &      Mínmos medios &       77.0833 &         0.0000 &          1728.0 \\
 21206240 &     Valores medios &       68.6161 &         0.0000 &          1727.0 \\
 21206260 &  Máximos absolutos &       22.7370 &         0.0291 &         10318.0 \\
 21206260 &      Mínmos medios &       18.2708 &         0.0194 &         10317.0 \\
 21206260 &     Valores medios &       24.1423 &         0.0000 &         10318.0 \\
 21206280 &  Máximos absolutos &       27.4412 &         0.0106 &          9391.0 \\
 21206280 &      Mínmos medios &       11.9689 &         0.0106 &          9391.0 \\
 21206280 &     Valores medios &       21.5526 &         0.0000 &          9391.0 \\
 21206450 &  Máximos absolutos &       44.1931 &         0.0000 &          1326.0 \\
 21206450 &      Mínmos medios &       40.7994 &         0.0754 &          1326.0 \\
 21206450 &     Valores medios &       54.2642 &         0.0000 &          1325.0 \\
 21206490 &      Mínmos medios &       21.6327 &         0.0563 &          7105.0 \\
 21206490 &     Valores medios &       99.3102 &         0.0000 &          7104.0 \\
 21206500 &  Máximos absolutos &       55.0272 &         0.0815 &          3680.0 \\
 21206500 &      Mínmos medios &       20.1359 &         0.0272 &          3680.0 \\
 21206500 &     Valores medios &       51.1141 &         0.0272 &          3680.0 \\
 21206510 &  Máximos absolutos &       55.5051 &         0.0000 &          1980.0 \\
 21206510 &      Mínmos medios &       47.1212 &         0.0000 &          1980.0 \\
 21206510 &     Valores medios &       64.5455 &         0.0000 &          1980.0 \\
 21206550 &  Máximos absolutos &       55.5318 &         0.0357 &          2802.0 \\
 21206550 &      Mínmos medios &       19.8073 &         0.0357 &          2802.0 \\
 21206550 &     Valores medios &       27.4090 &         0.0000 &          2802.0 \\
 21206560 &  Máximos absolutos &       39.0447 &         0.1169 &          6846.0 \\
 21206560 &      Mínmos medios &       23.8533 &         0.0292 &          6846.0 \\
 21206570 &  Máximos absolutos &       76.7801 &         0.0000 &          5112.0 \\
 21206570 &      Mínmos medios &       73.1025 &         0.0196 &          5112.0 \\
 21206570 &     Valores medios &       75.2787 &         0.0000 &          5113.0 \\
 21206610 &  Máximos absolutos &       83.1994 &         0.0000 &          1369.0 \\
 21206610 &      Mínmos medios &       81.3002 &         0.0000 &          1369.0 \\
 21206610 &     Valores medios &       88.0117 &         0.0000 &          1368.0 \\
 21206620 &  Máximos absolutos &       61.9889 &         0.0198 &          5038.0 \\
 21206620 &      Mínmos medios &       58.1977 &         0.0397 &          5038.0 \\
 21206620 &     Valores medios &       63.0409 &         0.0595 &          5038.0 \\
 21206630 &  Máximos absolutos &       34.3616 &         0.0000 &          2334.0 \\
 21206630 &      Mínmos medios &       36.0326 &         0.0428 &          2334.0 \\
 21206630 &     Valores medios &       27.8063 &         0.0428 &          2334.0 \\
 21206640 &  Máximos absolutos &       74.0975 &         0.0744 &          2687.0 \\
 21205580 &     Valores medios &       27.3676 &         0.0210 &         14276.0 \\
 21205750 &     Valores medios &       75.9911 &         0.0445 &          2245.0 \\
 21205870 &      Mínmos medios &        7.1811 &         0.0000 &         14761.0 \\
 21205940 &      Mínmos medios &       17.0594 &         0.0000 &         14725.0 \\
 21206010 &      Mínmos medios &       16.5029 &         0.0069 &         14452.0 \\
 21206170 &     Valores medios &       55.7377 &         0.0000 &           122.0 \\
 21206230 &      Mínmos medios &       10.4974 &         0.0000 &          7640.0 \\
 21206490 &  Máximos absolutos &       99.6340 &         0.0000 &          7104.0 \\
 21206560 &     Valores medios &       39.9796 &         0.0146 &          6846.0 \\
 21206640 &      Mínmos medios &       69.2594 &         0.0744 &          2687.0 \\
 21206690 &  Máximos absolutos &       46.6043 &         0.0918 &          5448.0 \\
 21206640 &     Valores medios &       74.7768 &         0.0000 &          2688.0 \\
 21206650 &  Máximos absolutos &       45.4799 &         0.0743 &          5387.0 \\
 21206650 &      Mínmos medios &       43.1966 &         0.0186 &          5387.0 \\
 21206650 &     Valores medios &       57.8244 &         0.0557 &          5387.0 \\
 21206660 &  Máximos absolutos &       48.5142 &         0.0632 &          4745.0 \\
 21206660 &      Mínmos medios &       44.4468 &         0.0211 &          4745.0 \\
 21206660 &     Valores medios &       56.9652 &         0.0211 &          4745.0 \\
 21206670 &  Máximos absolutos &       81.3092 &         0.0000 &          2429.0 \\
 21206670 &      Mínmos medios &       75.8025 &         0.0000 &          2430.0 \\
 21206670 &     Valores medios &       87.2840 &         0.0000 &          2430.0 \\
 21206680 &  Máximos absolutos &       47.8588 &         0.0000 &          1331.0 \\
 21206680 &      Mínmos medios &       50.8640 &         0.0000 &          1331.0 \\
 21206680 &     Valores medios &       53.9444 &         0.0751 &          1331.0 \\
 21205700 &  Máximos absolutos &       22.0830 &         0.0209 &         14364.0 \\
 21205700 &      Mínmos medios &       13.3171 &         0.0139 &         14365.0 \\
 21205700 &     Valores medios &       24.4205 &         0.0070 &         14365.0 \\
 21205710 &  Máximos absolutos &       27.6158 &         0.0209 &         14336.0 \\
 21205710 &      Mínmos medios &       28.5714 &         0.0209 &         14336.0 \\
 21205710 &     Valores medios &       33.2682 &         0.0000 &         14335.0 \\
 21205720 &  Máximos absolutos &       25.9030 &         0.0069 &         14396.0 \\
 21205720 &      Mínmos medios &       20.6724 &         0.0278 &         14396.0 \\
 21205720 &     Valores medios &       20.6863 &         0.0139 &         14396.0 \\
 21205740 &  Máximos absolutos &       17.7887 &         0.0000 &         14914.0 \\
 21205740 &      Mínmos medios &       17.6747 &         0.0000 &         14914.0 \\
 21205740 &     Valores medios &       22.2811 &         0.0134 &         14914.0 \\
 21205750 &  Máximos absolutos &       87.3553 &         0.0000 &          2246.0 \\
 21205750 &      Mínmos medios &       75.7684 &         0.0891 &          2245.0 \\
 21205770 &  Máximos absolutos &       53.7892 &         0.0592 &         13512.0 \\
 21205770 &      Mínmos medios &       47.2617 &         0.0222 &         13512.0 \\
 21205770 &     Valores medios &       37.8728 &         0.0148 &         13511.0 \\
 21205790 &  Máximos absolutos &       17.5318 &         0.0000 &         16376.0 \\
 21205790 &      Mínmos medios &        2.2838 &         0.0000 &         16376.0 \\
 21205790 &     Valores medios &       17.3486 &         0.0000 &         16376.0 \\
 21205820 &      Mínmos medios &        1.3072 &         0.0000 &           306.0 \\
 21205840 &  Máximos absolutos &        0.0000 &         0.0000 &            31.0 \\
 21205840 &      Mínmos medios &        0.0000 &         0.0000 &            31.0 \\
 21205840 &     Valores medios &       23.3333 &         0.0000 &            30.0 \\
 21205850 &      Mínmos medios &        8.7608 &         0.0224 &         13412.0 \\
 21205850 &     Valores medios &       86.6527 &         0.0075 &         13411.0 \\
 21205860 &      Mínmos medios &       17.9968 &         0.0202 &          4962.0 \\
 21206690 &      Mínmos medios &       40.0330 &         0.0734 &          5448.0 \\
 21206690 &     Valores medios &       49.6512 &         0.0000 &          5448.0 \\
 21206700 &  Máximos absolutos &       92.8571 &         0.0000 &           294.0 \\
 21206700 &      Mínmos medios &       57.2881 &         0.0000 &           295.0 \\
 21206700 &     Valores medios &       98.2993 &         0.0000 &           294.0 \\
 21206970 &  Máximos absolutos &       26.7013 &         0.0000 &          1734.0 \\
 21206970 &      Mínmos medios &       19.9539 &         0.0000 &          1734.0 \\
 21206970 &     Valores medios &       27.1626 &         0.0000 &          1734.0 \\
 35025050 &  Máximos absolutos &        6.0292 &         0.0000 &         10897.0 \\
 35025050 &      Mínmos medios &        4.5334 &         0.0000 &         10897.0 \\
 35025050 &     Valores medios &       10.4065 &         0.0184 &         10897.0 \\
 35025060 &  Máximos absolutos &       30.0510 &         0.0093 &         10775.0 \\
 35025060 &      Mínmos medios &       13.0302 &         0.0093 &         10775.0 \\
 35025060 &     Valores medios &       14.1810 &         0.0093 &         10775.0 \\
 21205870 &     Valores medios &       97.6423 &         0.0068 &         14760.0 \\
 21205880 &      Mínmos medios &       15.5912 &         0.0279 &         14335.0 \\
 21205880 &     Valores medios &       94.1468 &         0.0000 &         14334.0 \\
 21205890 &  Máximos absolutos &       99.9932 &         0.0000 &         14790.0 \\
 21205890 &      Mínmos medios &       11.2839 &         0.0000 &         14791.0 \\
 21205890 &     Valores medios &       86.3092 &         0.0000 &         14791.0 \\
 21205910 &  Máximos absolutos &       96.1097 &         0.0000 &         14729.0 \\
 21205910 &      Mínmos medios &        7.5686 &         0.0068 &         14732.0 \\
 21205910 &     Valores medios &       83.5427 &         0.0000 &         14729.0 \\
 21205920 &  Máximos absolutos &       86.6608 &         0.0068 &         14761.0 \\
 21205920 &      Mínmos medios &        7.7835 &         0.0000 &         14762.0 \\
 21205920 &     Valores medios &       74.3649 &         0.0000 &         14761.0 \\
 21205930 &      Mínmos medios &       19.1752 &         0.0000 &          5747.0 \\
 21205940 &  Máximos absolutos &       82.2011 &         0.0000 &         14720.0 \\
 21205940 &     Valores medios &       83.0367 &         0.0068 &         14720.0 \\
 21205950 &      Mínmos medios &       13.6800 &         0.0096 &         10424.0 \\
 21205960 &  Máximos absolutos &       84.4339 &         0.0000 &         14750.0 \\
 21205960 &      Mínmos medios &       14.4872 &         0.0000 &         14751.0 \\
 21205960 &     Valores medios &       82.2361 &         0.0068 &         14749.0 \\
 21205970 &      Mínmos medios &       10.4307 &         0.0000 &         14697.0 \\
 21205970 &     Valores medios &       91.3428 &         0.0000 &         14693.0 \\
 21205980 &  Máximos absolutos &       28.2209 &         0.0000 &         14142.0 \\
 21205980 &      Mínmos medios &        2.8283 &         0.0000 &         14143.0 \\
 21205980 &     Valores medios &       26.2198 &         0.0000 &         14142.0 \\
 21205990 &      Mínmos medios &       14.2425 &         0.0000 &         13228.0 \\
 21205990 &     Valores medios &       98.3141 &         0.0076 &         13227.0 \\
 21206010 &  Máximos absolutos &       99.9931 &         0.0000 &         14451.0 \\
 21206010 &     Valores medios &       99.0727 &         0.0000 &         14451.0 \\
 21206020 &      Mínmos medios &       16.0377 &         0.0068 &         14759.0 \\
 21206020 &     Valores medios &       98.9220 &         0.0000 &         14749.0 \\
 21206030 &  Máximos absolutos &       94.9322 &         0.0000 &         14385.0 \\
 21206030 &      Mínmos medios &       14.6869 &         0.0070 &         14387.0 \\
 21206030 &     Valores medios &       93.1730 &         0.0070 &         14384.0 \\
 21206050 &  Máximos absolutos &       10.0073 &         0.0121 &          8244.0 \\
 21206050 &      Mínmos medios &        8.5032 &         0.0000 &          8244.0 \\
 21206050 &     Valores medios &       27.6201 &         0.0000 &          8244.0 \\
 21206060 &      Mínmos medios &       15.1922 &         0.0141 &         14152.0 \\
 21206150 &  Máximos absolutos &       49.8780 &         0.0000 &           820.0 \\
 21206150 &      Mínmos medios &       48.2339 &         0.0000 &           821.0 \\
 21206150 &     Valores medios &       51.3415 &         0.0000 &           820.0 \\
 21206170 &  Máximos absolutos &       52.4590 &         0.0000 &           122.0 \\
 21206170 &      Mínmos medios &       17.0732 &         0.0000 &           123.0 \\
 21206190 &  Máximos absolutos &       64.8752 &         0.1457 &          5489.0 \\
 21206190 &      Mínmos medios &       54.0619 &         0.0546 &          5490.0 \\
 21206190 &     Valores medios &       58.4988 &         0.0182 &          5489.0 \\
 21206200 &  Máximos absolutos &       40.2075 &         0.0649 &          1542.0 \\
 21206200 &      Mínmos medios &       40.5318 &         0.0649 &          1542.0 \\
 21206200 &     Valores medios &       44.6174 &         0.0000 &          1542.0 \\
 21206210 &  Máximos absolutos &       62.5827 &         0.0342 &          8766.0 \\
 21206210 &      Mínmos medios &       31.3712 &         0.0456 &          8766.0 \\
 21206210 &     Valores medios &       79.5436 &         0.0000 &          8765.0 \\
 21206220 &  Máximos absolutos &       40.8024 &         0.0211 &          9497.0 \\
 21206220 &      Mínmos medios &       33.8949 &         0.0000 &          9497.0 \\
 21206220 &     Valores medios &       41.5184 &         0.0211 &          9497.0 \\
 21206230 &  Máximos absolutos &       18.9242 &         0.0000 &          7641.0 \\
\end{longtable}


 % Segundo
%%\appendix
%\clearpage
%\addappheadtotoc
%\appendixpage

\chapter{Series de la temperatura del aire de las estaciones automáticas, desde el 31 de enero del 2007 hasta el quinto día del mes de febrero del 2007.}
\label{anexo:series_tiempo_temperatura}



\begin{figure}[H]
\centering



\begin{subfigure}[normla]{0.4\textwidth}
\includegraphics[draft=false, scale=0.25]{automaticas_periodos/21195160.png}
\caption{Estación SUBIA AUTOMATICA código 21195160.}
\end{subfigure}
~
%\begin{subfigure}[normla]{0.4\textwidth}
%\includegraphics[draft=false, scale=0.25]{automaticas_periodos/21205012.png}
%\caption{Estación UNIV NACIONAL código 21205012.}
%\end{subfigure}
%~
\begin{subfigure}[normla]{0.4\textwidth}
\includegraphics[draft=false, scale=0.25]{automaticas_periodos/21206790.png}
\caption{Estación HDA STA ANA AUTOM código 21206790.}
\end{subfigure}
~
%\begin{subfigure}[normla]{0.4\textwidth}
%\includegraphics[draft=false, scale=0.25]{automaticas_periodos/21206920.png}
%\caption{Estación VILLA TERESA AUTOMATICA código 21206920.}
%\end{subfigure}
%~
\begin{subfigure}[normla]{0.4\textwidth}
\includegraphics[draft=false, scale=0.25]{automaticas_periodos/21206930.png}
\caption{Estación PMO GUERRERO código 21206930.}
\end{subfigure}
~
\begin{subfigure}[normla]{0.4\textwidth}
\includegraphics[draft=false, scale=0.25]{automaticas_periodos/21206940.png}
\caption{Estación CIUDAD BOLIVAR código 21206940.}
\end{subfigure}
~
\end{figure}
           
\begin{figure}[H]\ContinuedFloat
\centering 
\begin{subfigure}[normla]{0.4\textwidth}
\includegraphics[draft=false, scale=0.25]{automaticas_periodos/21206950.png}
\caption{Estación PMO GUACHENEQUE código 21206950.}
\end{subfigure}
~
\begin{subfigure}[normla]{0.4\textwidth}
\includegraphics[draft=false, scale=0.25]{automaticas_periodos/21206980.png}
\caption{Estación STA CRUZ DE SIECHA código 21206980.}
\end{subfigure}
~

\begin{subfigure}[normla]{0.4\textwidth}
\includegraphics[draft=false, scale=0.25]{automaticas_periodos/21206990.png}
\caption{Estación TIBAITATA AUTOMATICA código 21206990.}
\end{subfigure}
~
%\begin{subfigure}[normla]{0.4\textwidth}
%\includegraphics[draft=false, scale=0.25]{automaticas_periodos/23125170.png}
%\caption{Estación SAN CAYETANO AUTOM código 23125170.}
%\end{subfigure}
%~ 
\begin{subfigure}[normla]{0.4\textwidth}
\includegraphics[draft=false, scale=0.25]{automaticas_periodos/24015110.png}
\caption{Estación LA BOYERA AUTOMATICA código 24015110.}
\end{subfigure}
~
%\begin{subfigure}[normla]{0.4\textwidth}
%\includegraphics[draft=false, scale=0.25]{automaticas_periodos/35025080.png}
%\caption{Estación PNN CHINGAZA AUTOM código 35025080.}
%\end{subfigure}
%~

%\begin{subfigure}[normla]{0.4\textwidth}
%\includegraphics[draft=false, scale=0.25]{automaticas_periodos/35027510.png}
%\caption{Estación CALOSTROS BAJO código 35027510.}
%\end{subfigure}
%~
%\begin{subfigure}[normla]{0.4\textwidth}
%\includegraphics[draft=false, scale=0.25]{automaticas_periodos/35035130.png}
%\caption{Estación PMO CHINGAZA código 35035130.}
%\end{subfigure}
%~
\begin{subfigure}[normla]{0.4\textwidth}
\includegraphics[draft=false, scale=0.25]{automaticas_periodos/35075070.png}
\caption{Estación CHINAVITA AUTOMATICA código 35075070.}
\end{subfigure}
~
%\begin{subfigure}[normla]{0.4\textwidth}
%\includegraphics[draft=false, scale=0.25]{automaticas_periodos/35075080.png}
%\caption{Estación PMO RABANAL AUTOMATICA código 35075080.}
%\end{subfigure}
%~
\begin{subfigure}[normla]{0.4\textwidth}
\includegraphics[draft=false, scale=0.25]{automaticas_periodos/35085080.png}
\caption{Estación LA CAPILLA AUTOM código 35085080.}
\end{subfigure}
~




\caption{Las líneas negras representan el período de tiempo entre el 31 de enero del 2007 y el 03 de febrero del 2007.}
\end{figure}
 % Tercero
%\textit{
%%\appendix
%\clearpage
%\addappheadtotoc
%\appendixpage

\chapter{\textit{namelist} usadas para realizar el WPS y WRF en la selección de la cantidad de dominios.}
%las namelist WPS están en /home/edwin/Downloads/wps/descargas_20180713/wps/namelist
\label{anexo:namelist-input-wps}

\section{\textit{namelist.wps} usadas para la selección de los mejores dominios y subdomínios.}

%%%%% Caso 1 36-12-4



\textbf{namelist.wps  Simulación 1}


\&share\\
~wrf\_core~=~'ARW',\\
~max\_dom~=~3,\\
~start\_date~=~'2007-02-01\_00:00:00','2007-02-01\_00:00:00','2007-02-01\_00:00:00',\\
~end\_date~~~=~'2007-02-05\_00:00:00','2007-02-05\_00:00:00','2007-02-05\_00:00:00',\\
~interval\_seconds~=~21600,\\
~io\_form\_geogrid~=~2,\\
\\
~debug\_level~=~0,\\
/\\
\\
\&geogrid\\
~parent\_id~~~~~~~~~=~1,1,2,\\
~parent\_grid\_ratio~=~1,3,3,\\
~i\_parent\_start~~~~=~1,12,33,\\
~j\_parent\_start~~~~=~1,12,33,\\
~e\_we~~~~~~~~~~=~99,232,448,\\
~e\_sn~~~~~~~~~~=~94,214,385,\\
~geog\_data\_res~=~'igac\_lu\_30s+30s','igac\_lu\_30s+30s','igac\_lu\_30s+30s',\\
~dx~=~36000,\\
~dy~=~36000,\\
~map\_proj~=~~'mercator',\\
~ref\_lat~~~=~4.916,\\
~ref\_lon~~~=~-73.902,\\
~truelat1~~=~4.916,\\
~truelat2~~=~0,\\
~stand\_lon~=~-73.902,\\
~geog\_data\_path~=~'/home/edwin/wrf\_b/wrf/geog',\\
\\
~ref\_x~=~49.5,\\
~ref\_y~=~47.0,\\
/\\
\\
\&ungrib\\
~out\_format~=~'WPS',\\
~prefix~=~'FILE',\\
/\\
\\
\&metgrid\\
~fg\_name~=~'FILE',\\
~io\_form\_metgrid~=~2,\\
\\
\\
/\\
\\
\&mod\_levs\\
~press\_pa~=~201300~,~200100~,~100000~,\\
~~~~~~~~~~~~~95000~,~~90000~,\\
~~~~~~~~~~~~~85000~,~~80000~,\\
~~~~~~~~~~~~~75000~,~~70000~,\\
~~~~~~~~~~~~~65000~,~~60000~,\\
~~~~~~~~~~~~~55000~,~~50000~,\\
~~~~~~~~~~~~~45000~,~~40000~,\\
~~~~~~~~~~~~~35000~,~~30000~,\\
~~~~~~~~~~~~~25000~,~~20000~,\\
~~~~~~~~~~~~~15000~,~~10000~,\\
~~~~~~~~~~~~~~5000~,~~~1000\\
~/\\
\\
\\
\&domain\_wizard\\
\\
~grib\_vtable~=~'Vtable.GFS',\\
~dwiz\_name~~~~=2\_dominios\\
~dwiz\_desc~~~~=\\
~dwiz\_user\_rect\_x1~=2345\\
~dwiz\_user\_rect\_y1~=1875\\
~dwiz\_user\_rect\_x2~=2429\\
~dwiz\_user\_rect\_y2~=1955\\
~dwiz\_show\_political~=true\\
~dwiz\_center\_over\_gmt~=true\\
~dwiz\_latlon\_space\_in\_deg~=10\\
~dwiz\_latlon\_linecolor~=-8355712\\
~dwiz\_map\_scale\_pct~=50.0\\
~dwiz\_map\_vert\_scrollbar\_pos~=1625\\
~dwiz\_map\_horiz\_scrollbar\_pos~=1795\\
~dwiz\_gridpt\_dist\_km~=4.2\\
~dwiz\_mpi\_command~=null\\
~dwiz\_tcvitals~=null\\
~dwiz\_bigmap~=Y\\
/\\

%%%%% Caso 2 18-6-2
\textbf{namelist.wps Simulación 2}


\&share\\
~wrf\_core~=~'ARW',\\
~max\_dom~=~3,\\
~start\_date~=~'2007-01-30\_00:00:00','2007-01-30\_00:00:00','2007-01-30\_00:00:00',\\
~end\_date~~~=~'2007-02-05\_00:00:00','2007-02-05\_00:00:00','2007-02-01\_00:00:00',\\
~interval\_seconds~=~21600,\\
~io\_form\_geogrid~=~2,\\
\\
~debug\_level~=~0,\\
/\\
\\
\&geogrid\\
~parent\_id~~~~~~~~~=~1,1,2,\\
~parent\_grid\_ratio~=~1,3,3,\\
~i\_parent\_start~~~~=~1,12,33,\\
~j\_parent\_start~~~~=~1,12,33,\\
~e\_we~~~~~~~~~~=~99,232,448,\\
~e\_sn~~~~~~~~~~=~94,214,385,\\
~geog\_data\_res~=~'igac\_lu\_30s+30s','igac\_lu\_30s+30s','igac\_lu\_30s+30s',\\
~dx~=~18000,\\
~dy~=~18000,\\
~map\_proj~=~~'mercator',\\
~ref\_lat~~~=~4.916,\\
~ref\_lon~~~=~-73.902,\\
~truelat1~~=~4.916,\\
~truelat2~~=~0,\\
~stand\_lon~=~-73.902,\\
~geog\_data\_path~=~'/home/edwin/wrf\_b/wrf/geog',\\
\\
~ref\_x~=~49.5,\\
~ref\_y~=~47.0,\\
/\\
\\
\&ungrib\\
~out\_format~=~'WPS',\\
~prefix~=~'FILE',\\
/\\
\\
\&metgrid\\
~fg\_name~=~'FILE',\\
~io\_form\_metgrid~=~2,\\
\\
\\
/\\
\\
\&mod\_levs\\
~press\_pa~=~201300~,~200100~,~100000~,\\
~~~~~~~~~~~~~95000~,~~90000~,\\
~~~~~~~~~~~~~85000~,~~80000~,\\
~~~~~~~~~~~~~75000~,~~70000~,\\
~~~~~~~~~~~~~65000~,~~60000~,\\
~~~~~~~~~~~~~55000~,~~50000~,\\
~~~~~~~~~~~~~45000~,~~40000~,\\
~~~~~~~~~~~~~35000~,~~30000~,\\
~~~~~~~~~~~~~25000~,~~20000~,\\
~~~~~~~~~~~~~15000~,~~10000~,\\
~~~~~~~~~~~~~~5000~,~~~1000\\
~/\\
\\
\\
\&domain\_wizard\\
\\
~grib\_vtable~=~'Vtable.GFS',\\
~dwiz\_name~~~~=2\_dominios\\
~dwiz\_desc~~~~=\\
~dwiz\_user\_rect\_x1~=2345\\
~dwiz\_user\_rect\_y1~=1875\\
~dwiz\_user\_rect\_x2~=2429\\
~dwiz\_user\_rect\_y2~=1955\\
~dwiz\_show\_political~=true\\
~dwiz\_center\_over\_gmt~=true\\
~dwiz\_latlon\_space\_in\_deg~=10\\
~dwiz\_latlon\_linecolor~=-8355712\\
~dwiz\_map\_scale\_pct~=50.0\\
~dwiz\_map\_vert\_scrollbar\_pos~=1625\\
~dwiz\_map\_horiz\_scrollbar\_pos~=1795\\
~dwiz\_gridpt\_dist\_km~=4.2\\
~dwiz\_mpi\_command~=null\\
~dwiz\_tcvitals~=null\\
~dwiz\_bigmap~=Y\\
/\\

%%%%% Caso 3 sólo2



%%%%% Caso 3 10-3 corto

\textbf{namelist.wps Simulación 3}

\&share\\
~wrf\_core~=~'ARW',\\
~max\_dom~=~2,\\
~start\_date~=~'2007-01-30\_00:00:00','2007-01-30\_00:00:00',\\
~end\_date~~~=~'2007-02-05\_00:00:00','2007-02-05\_00:00:00',\\
~interval\_seconds~=~21600,\\
~io\_form\_geogrid~=~2,\\
\\
~debug\_level~=~0,\\
/\\
\\
\&geogrid\\
~parent\_id~~~~~~~~~=~1,1,\\
~parent\_grid\_ratio~=~1,3,\\
~i\_parent\_start~~~~=~1,25,\\
~j\_parent\_start~~~~=~1,26,\\
~e\_we~~~~~~~~~~=~100,148,\\
~e\_sn~~~~~~~~~~=~99,196,\\
~geog\_data\_res~=~'igac\_lu\_30s+30s','igac\_lu\_30s+30s',\\
~dx~=~10000,\\
~dy~=~10000,\\
~map\_proj~=~~'mercator',\\
~ref\_lat~~~=~4.542,\\
~ref\_lon~~~=~-74.144,\\
~truelat1~~=~4.542,\\
~truelat2~~=~0,\\
~stand\_lon~=~-74.144,\\
~geog\_data\_path~=~'/home/edwin/wrf\_b/wrf/geog',\\
\\
~ref\_x~=~50.0,\\
~ref\_y~=~49.5,\\
/\\
\\
\&ungrib\\
~out\_format~=~'WPS',\\
~prefix~=~'FILE',\\
/\\
\\
\&metgrid\\
~fg\_name~=~'FILE',\\
~io\_form\_metgrid~=~2,\\
\\
\\
/\\
\\
\&mod\_levs\\
~press\_pa~=~201300~,~200100~,~100000~,\\
~~~~~~~~~~~~~95000~,~~90000~,\\
~~~~~~~~~~~~~85000~,~~80000~,\\
~~~~~~~~~~~~~75000~,~~70000~,\\
~~~~~~~~~~~~~65000~,~~60000~,\\
~~~~~~~~~~~~~55000~,~~50000~,\\
~~~~~~~~~~~~~45000~,~~40000~,\\
~~~~~~~~~~~~~35000~,~~30000~,\\
~~~~~~~~~~~~~25000~,~~20000~,\\
~~~~~~~~~~~~~15000~,~~10000~,\\
~~~~~~~~~~~~~~5000~,~~~1000\\
~/\\
\\
\\
\&domain\_wizard\\
\\
~grib\_vtable~=~'Vtable.GFS',\\
~dwiz\_name~~~~=2\_dominios\\
~dwiz\_desc~~~~=\\
~dwiz\_user\_rect\_x1~=2345\\
~dwiz\_user\_rect\_y1~=1875\\
~dwiz\_user\_rect\_x2~=2429\\
~dwiz\_user\_rect\_y2~=1955\\
~dwiz\_show\_political~=true\\
~dwiz\_center\_over\_gmt~=true\\
~dwiz\_latlon\_space\_in\_deg~=10\\
~dwiz\_latlon\_linecolor~=-8355712\\
~dwiz\_map\_scale\_pct~=50.0\\
~dwiz\_map\_vert\_scrollbar\_pos~=1625\\
~dwiz\_map\_horiz\_scrollbar\_pos~=1795\\
~dwiz\_gridpt\_dist\_km~=4.2\\
~dwiz\_mpi\_command~=null\\
~dwiz\_tcvitals~=null\\
~dwiz\_bigmap~=Y\\
/\\


%%%% Caso 4 10-3n

\textbf{namelist.wps Simulación 4}

\&share\\
~wrf\_core~=~'ARW',\\
~max\_dom~=~2,\\
~start\_date~=~'2007-01-30\_00:00:00','2007-01-30\_00:00:00',\\
~end\_date~~~=~'2007-02-05\_00:00:00','2007-02-05\_00:00:00',\\
~interval\_seconds~=~21600,\\
~io\_form\_geogrid~=~2,\\
\\
~debug\_level~=~0,\\
/\\
\\
\&geogrid\\
~parent\_id~~~~~~~~~=~1,1,\\
~parent\_grid\_ratio~=~1,3,\\
~i\_parent\_start~~~~=~1,25,\\
~j\_parent\_start~~~~=~1,26,\\
~e\_we~~~~~~~~~~=~100,148,\\
~e\_sn~~~~~~~~~~=~99,196,\\
~geog\_data\_res~=~'igac\_lu\_30s+30s','igac\_lu\_30s+30s',\\
~dx~=~10000,\\
~dy~=~10000,\\
~map\_proj~=~~'mercator',\\
~ref\_lat~~~=~4.542,\\
~ref\_lon~~~=~-74.144,\\
~truelat1~~=~4.542,\\
~truelat2~~=~0,\\
~stand\_lon~=~-74.144,\\
~geog\_data\_path~=~'/home/edwin/wrf\_b/wrf/geog',\\
\\
~ref\_x~=~50.0,\\
~ref\_y~=~49.5,\\
/\\
\\
\&ungrib\\
~out\_format~=~'WPS',\\
~prefix~=~'FILE',\\
/\\
\\
\&metgrid\\
~fg\_name~=~'FILE',\\
~io\_form\_metgrid~=~2,\\
\\
\\
/\\
\\
\&mod\_levs\\
~press\_pa~=~201300~,~200100~,~100000~,\\
~~~~~~~~~~~~~95000~,~~90000~,\\
~~~~~~~~~~~~~85000~,~~80000~,\\
~~~~~~~~~~~~~75000~,~~70000~,\\
~~~~~~~~~~~~~65000~,~~60000~,\\
~~~~~~~~~~~~~55000~,~~50000~,\\
~~~~~~~~~~~~~45000~,~~40000~,\\
~~~~~~~~~~~~~35000~,~~30000~,\\
~~~~~~~~~~~~~25000~,~~20000~,\\
~~~~~~~~~~~~~15000~,~~10000~,\\
~~~~~~~~~~~~~~5000~,~~~1000\\
~/\\
\\
\\
\&domain\_wizard\\
\\
~grib\_vtable~=~'Vtable.GFS',\\
~dwiz\_name~~~~=2\_dominios\\
~dwiz\_desc~~~~=\\
~dwiz\_user\_rect\_x1~=2345\\
~dwiz\_user\_rect\_y1~=1875\\
~dwiz\_user\_rect\_x2~=2429\\
~dwiz\_user\_rect\_y2~=1955\\
~dwiz\_show\_political~=true\\
~dwiz\_center\_over\_gmt~=true\\
~dwiz\_latlon\_space\_in\_deg~=10\\
~dwiz\_latlon\_linecolor~=-8355712\\
~dwiz\_map\_scale\_pct~=50.0\\
~dwiz\_map\_vert\_scrollbar\_pos~=1625\\
~dwiz\_map\_horiz\_scrollbar\_pos~=1795\\
~dwiz\_gridpt\_dist\_km~=4.2\\
~dwiz\_mpi\_command~=null\\
~dwiz\_tcvitals~=null\\
~dwiz\_bigmap~=Y\\
/\\




%%%%%%%%
%% Simulación 5

\textbf{namelist.wps Simulación 5}

\&share\\
~wrf\_core~=~'ARW',\\
~max\_dom~=~2,\\
~start\_date~=~'2007-01-30\_00:00:00','2007-01-30\_00:00:00',\\
~end\_date~~~=~'2007-02-05\_00:00:00','2007-02-05\_00:00:00',\\
~interval\_seconds~=~21600,\\
~io\_form\_geogrid~=~2,\\
\\
~debug\_level~=~0,\\
/\\
\\
\&geogrid\\
~parent\_id~~~~~~~~~=~1,1,\\
~parent\_grid\_ratio~=~1,3,\\
~i\_parent\_start~~~~=~1,16,\\
~j\_parent\_start~~~~=~1,15,\\
~e\_we~~~~~~~~~~=~70,112,\\
~e\_sn~~~~~~~~~~=~70,118,\\
~geog\_data\_res~=~'igac\_lu\_30s+30s','igac\_lu\_30s+30s',\\
~dx~=~12000,\\
~dy~=~12000,\\
~map\_proj~=~~'mercator',\\
~ref\_lat~~~=~4.79,\\
~ref\_lon~~~=~-74.02,\\
~truelat1~~=~4.79,\\
~truelat2~~=~0,\\
~stand\_lon~=~-74.02,\\
~geog\_data\_path~=~'/home/edwin/wrf\_b/wrf/geog',\\
\\
~ref\_x~=~35.0,\\
~ref\_y~=~35.0,\\
/\\
\\
\&ungrib\\
~out\_format~=~'WPS',\\
~prefix~=~'FILE',\\
/\\
\\
\&metgrid\\
~fg\_name~=~'FILE',\\
~io\_form\_metgrid~=~2,\\
\\
\\
/\\
\\
\&mod\_levs\\
~press\_pa~=~201300~,~200100~,~100000~,\\
~~~~~~~~~~~~~95000~,~~90000~,\\
~~~~~~~~~~~~~85000~,~~80000~,\\
~~~~~~~~~~~~~75000~,~~70000~,\\
~~~~~~~~~~~~~65000~,~~60000~,\\
~~~~~~~~~~~~~55000~,~~50000~,\\
~~~~~~~~~~~~~45000~,~~40000~,\\
~~~~~~~~~~~~~35000~,~~30000~,\\
~~~~~~~~~~~~~25000~,~~20000~,\\
~~~~~~~~~~~~~15000~,~~10000~,\\
~~~~~~~~~~~~~~5000~,~~~1000\\
~/\\
\\
\\
\&domain\_wizard\\
\\
~grib\_vtable~=~'Vtable.GFS',\\
~dwiz\_name~~~~=2\_dominios\\
~dwiz\_desc~~~~=\\
~dwiz\_user\_rect\_x1~=2345\\
~dwiz\_user\_rect\_y1~=1875\\
~dwiz\_user\_rect\_x2~=2429\\
~dwiz\_user\_rect\_y2~=1955\\
~dwiz\_show\_political~=true\\
~dwiz\_center\_over\_gmt~=true\\
~dwiz\_latlon\_space\_in\_deg~=10\\
~dwiz\_latlon\_linecolor~=-8355712\\
~dwiz\_map\_scale\_pct~=50.0\\
~dwiz\_map\_vert\_scrollbar\_pos~=1625\\
~dwiz\_map\_horiz\_scrollbar\_pos~=1795\\
~dwiz\_gridpt\_dist\_km~=4.2\\
~dwiz\_mpi\_command~=null\\
~dwiz\_tcvitals~=null\\
~dwiz\_bigmap~=Y\\
/\\

%
%
%%% SImulación 6


\textbf{namelist.wps Simulación 6}

\&share\\
~wrf\_core~=~'ARW',\\
~max\_dom~=~2,\\
~start\_date~=~'2007-01-30\_00:00:00','2007-01-30\_00:00:00',\\
~end\_date~~~=~'2007-02-05\_00:00:00','2007-02-05\_00:00:00',\\
~interval\_seconds~=~21600,\\
~io\_form\_geogrid~=~2,\\
\\
~debug\_level~=~0,\\
/\\
\\
\&geogrid\\
~parent\_id~~~~~~~~~=~1,1,\\
~parent\_grid\_ratio~=~1,3,\\
~i\_parent\_start~~~~=~1,12,\\
~j\_parent\_start~~~~=~1,12,\\
~e\_we~~~~~~~~~~=~56,91,\\
~e\_sn~~~~~~~~~~=~56,94,\\
~geog\_data\_res~=~'igac\_lu\_30s+30s','igac\_lu\_30s+30s',\\
~dx~=~15000,\\
~dy~=~15000,\\
~map\_proj~=~~'mercator',\\
~ref\_lat~~~=~4.79,\\
~ref\_lon~~~=~-74.02,\\
~truelat1~~=~4.79,\\
~truelat2~~=~0,\\
~stand\_lon~=~-74.02,\\
~geog\_data\_path~=~'/home/edwin/wrf\_b/wrf/geog',\\
\\
~/\\
\\
\&ungrib\\
~out\_format~=~'WPS',\\
~prefix~=~'FILE',\\
/\\
\\
\&metgrid\\
~fg\_name~=~'FILE',\\
~io\_form\_metgrid~=~2,\\
\\
\\
/\\
\\
\&mod\_levs\\
~press\_pa~=~201300~,~200100~,~100000~,\\
~~~~~~~~~~~~~95000~,~~90000~,\\
~~~~~~~~~~~~~85000~,~~80000~,\\
~~~~~~~~~~~~~75000~,~~70000~,\\
~~~~~~~~~~~~~65000~,~~60000~,\\
~~~~~~~~~~~~~55000~,~~50000~,\\
~~~~~~~~~~~~~45000~,~~40000~,\\
~~~~~~~~~~~~~35000~,~~30000~,\\
~~~~~~~~~~~~~25000~,~~20000~,\\
~~~~~~~~~~~~~15000~,~~10000~,\\
~~~~~~~~~~~~~~5000~,~~~1000\\
~/\\
\\
\\
\&domain\_wizard\\
\\
~grib\_vtable~=~'Vtable.GFS',\\
~dwiz\_name~~~~=2\_dominios\\
~dwiz\_desc~~~~=\\
~dwiz\_user\_rect\_x1~=2345\\
~dwiz\_user\_rect\_y1~=1875\\
~dwiz\_user\_rect\_x2~=2429\\
~dwiz\_user\_rect\_y2~=1955\\
~dwiz\_show\_political~=true\\
~dwiz\_center\_over\_gmt~=true\\
~dwiz\_latlon\_space\_in\_deg~=10\\
~dwiz\_latlon\_linecolor~=-8355712\\
~dwiz\_map\_scale\_pct~=50.0\\
~dwiz\_map\_vert\_scrollbar\_pos~=1625\\
~dwiz\_map\_horiz\_scrollbar\_pos~=1795\\
~dwiz\_gridpt\_dist\_km~=4.2\\
~dwiz\_mpi\_command~=null\\
~dwiz\_tcvitals~=null\\
~dwiz\_bigmap~=Y\\
/\\



%%%%%
% Simulación 7

\textbf{namelist.wps Simulación 7}

\&share\\
~wrf\_core~=~'ARW',\\
~max\_dom~=~2,\\
~start\_date~=~'2007-01-30\_00:00:00','2007-01-30\_00:00:00',\\
~end\_date~~~=~'2007-02-05\_00:00:00','2007-02-05\_00:00:00',\\
~interval\_seconds~=~21600,\\
~io\_form\_geogrid~=~2,\\
\\
~debug\_level~=~0,\\
/\\
\\
\&geogrid\\
~parent\_id~~~~~~~~~=~1,1,\\
~parent\_grid\_ratio~=~1,3,\\
~i\_parent\_start~~~~=~1,10,\\
~j\_parent\_start~~~~=~1,9,\\
~e\_we~~~~~~~~~~=~47,76,\\
~e\_sn~~~~~~~~~~=~47,79,\\
~geog\_data\_res~=~'igac\_lu\_30s+30s','igac\_lu\_30s+30s',\\
~dx~=~18000,\\
~dy~=~18000,\\
~map\_proj~=~~'mercator',\\
~ref\_lat~~~=~4.79,\\
~ref\_lon~~~=~-74.02,\\
~truelat1~~=~4.79,\\
~truelat2~~=~0,\\
~stand\_lon~=~-74.02,\\
~geog\_data\_path~=~'/home/edwin/wrf\_b/wrf/geog',\\
\\
/\\
\\
\&ungrib\\
~out\_format~=~'WPS',\\
~prefix~=~'FILE',\\
/\\
\\
\&metgrid\\
~fg\_name~=~'FILE',\\
~io\_form\_metgrid~=~2,\\
\\
\\
/\\
\\
\&mod\_levs\\
~press\_pa~=~201300~,~200100~,~100000~,\\
~~~~~~~~~~~~~95000~,~~90000~,\\
~~~~~~~~~~~~~85000~,~~80000~,\\
~~~~~~~~~~~~~75000~,~~70000~,\\
~~~~~~~~~~~~~65000~,~~60000~,\\
~~~~~~~~~~~~~55000~,~~50000~,\\
~~~~~~~~~~~~~45000~,~~40000~,\\
~~~~~~~~~~~~~35000~,~~30000~,\\
~~~~~~~~~~~~~25000~,~~20000~,\\
~~~~~~~~~~~~~15000~,~~10000~,\\
~~~~~~~~~~~~~~5000~,~~~1000\\
~/\\
\\
\\
\&domain\_wizard\\
\\
~grib\_vtable~=~'Vtable.GFS',\\
~dwiz\_name~~~~=2\_dominios\\
~dwiz\_desc~~~~=\\
~dwiz\_user\_rect\_x1~=2345\\
~dwiz\_user\_rect\_y1~=1875\\
~dwiz\_user\_rect\_x2~=2429\\
~dwiz\_user\_rect\_y2~=1955\\
~dwiz\_show\_political~=true\\
~dwiz\_center\_over\_gmt~=true\\
~dwiz\_latlon\_space\_in\_deg~=10\\
~dwiz\_latlon\_linecolor~=-8355712\\
~dwiz\_map\_scale\_pct~=50.0\\
~dwiz\_map\_vert\_scrollbar\_pos~=1625\\
~dwiz\_map\_horiz\_scrollbar\_pos~=1795\\
~dwiz\_gridpt\_dist\_km~=4.2\\
~dwiz\_mpi\_command~=null\\
~dwiz\_tcvitals~=null\\
~dwiz\_bigmap~=Y\\
/\\


%%%%%% Simulación 8

\textbf{namelist.wps Simulación 8}

\&share\\
~wrf\_core~=~'ARW',\\
~max\_dom~=~3,\\
~start\_date~=~'2007-01-30\_00:00:00','2007-01-30\_00:00:00','2007-01-30\_00:00:00',\\
~end\_date~~~=~'2007-02-05\_00:00:00','2007-02-05\_00:00:00','2007-02-05\_00:00:00',~\\
~interval\_seconds~=~21600,\\
~io\_form\_geogrid~=~2,\\
\\
~debug\_level~=~0,\\
/\\
\\
\&geogrid\\
~parent\_id~~~~~~~~~=~1,1,2,\\
~parent\_grid\_ratio~=~1,3,3,\\
~i\_parent\_start~~~~=~1,25,48,\\
~j\_parent\_start~~~~=~1,26,58,\\
~e\_we~~~~~~~~~~=~99,148,190,\\
~e\_sn~~~~~~~~~~=~99,169,184,\\
~geog\_data\_res~=~'igac\_lu\_30s+30s','igac\_lu\_30s+30s','igac\_lu\_30s+30s',\\
~dx~=~12000,\\
~dy~=~12000,\\
~map\_proj~=~~'mercator',\\
~ref\_lat~~~=~4.542,\\
~ref\_lon~~~=~-74.144,\\
~truelat1~~=~4.542,\\
~truelat2~~=~0,\\
~stand\_lon~=~-74.144,\\
~geog\_data\_path~=~'/home/edwin/wrf\_b/wrf/geog',\\
\\
~ref\_x~=~49.5,\\
~ref\_y~=~49.5,\\
\\
\\
\\
/\\
\\
\&ungrib\\
~out\_format~=~'WPS',\\
~prefix~=~'FILE',\\
/\\
\\
\&metgrid\\
~fg\_name~=~'FILE',\\
~io\_form\_metgrid~=~2,\\
\\
\\
/\\
\\
\&mod\_levs\\
~press\_pa~=~201300~,~200100~,~100000~,\\
~~~~~~~~~~~~~95000~,~~90000~,\\
~~~~~~~~~~~~~85000~,~~80000~,\\
~~~~~~~~~~~~~75000~,~~70000~,\\
~~~~~~~~~~~~~65000~,~~60000~,\\
~~~~~~~~~~~~~55000~,~~50000~,\\
~~~~~~~~~~~~~45000~,~~40000~,\\
~~~~~~~~~~~~~35000~,~~30000~,\\
~~~~~~~~~~~~~25000~,~~20000~,\\
~~~~~~~~~~~~~15000~,~~10000~,\\
~~~~~~~~~~~~~~5000~,~~~1000\\
~/\\
\\
\\
\&domain\_wizard\\
\\
~grib\_vtable~=~'Vtable.GFS',\\
~dwiz\_name~~~~=2\_dominios\\
~dwiz\_desc~~~~=\\
~dwiz\_user\_rect\_x1~=2345\\
~dwiz\_user\_rect\_y1~=1875\\
~dwiz\_user\_rect\_x2~=2429\\
~dwiz\_user\_rect\_y2~=1955\\
~dwiz\_show\_political~=true\\
~dwiz\_center\_over\_gmt~=true\\
~dwiz\_latlon\_space\_in\_deg~=10\\
~dwiz\_latlon\_linecolor~=-8355712\\
~dwiz\_map\_scale\_pct~=50.0\\
~dwiz\_map\_vert\_scrollbar\_pos~=1625\\
~dwiz\_map\_horiz\_scrollbar\_pos~=1795\\
~dwiz\_gridpt\_dist\_km~=4.2\\
~dwiz\_mpi\_command~=null\\
~dwiz\_tcvitals~=null\\
~dwiz\_bigmap~=Y\\
/\\


%%%%%% Simulación 9

\textbf{namelist.wps Simulación 9}

\&share\\
~wrf\_core~=~'ARW',\\
~max\_dom~=~3,\\
~start\_date~=~'2007-01-30\_00:00:00','2007-01-30\_00:00:00','2007-01-30\_00:00:00',\\
~end\_date~~~=~'2007-02-05\_00:00:00','2007-02-05\_00:00:00','2007-02-05\_00:00:00',~\\
~interval\_seconds~=~21600,\\
~io\_form\_geogrid~=~2,\\
\\
~debug\_level~=~0,\\
/\\
\\
\&geogrid\\
~parent\_id~~~~~~~~~=~1,1,2,\\
~parent\_grid\_ratio~=~1,3,3,\\
~i\_parent\_start~~~~=~1,25,48,\\
~j\_parent\_start~~~~=~1,26,58,\\
~e\_we~~~~~~~~~~=~99,148,190,\\
~e\_sn~~~~~~~~~~=~99,169,184,\\
~geog\_data\_res~=~'igac\_lu\_30s+30s','igac\_lu\_30s+30s','igac\_lu\_30s+30s',\\
~dx~=~10000,\\
~dy~=~10000,\\
~map\_proj~=~~'mercator',\\
~ref\_lat~~~=~4.542,\\
~ref\_lon~~~=~-74.144,\\
~truelat1~~=~4.542,\\
~truelat2~~=~0,\\
~stand\_lon~=~-74.144,\\
~geog\_data\_path~=~'/home/edwin/wrf\_b/wrf/geog',\\
\\
~ref\_x~=~49.5,\\
~ref\_y~=~49.5,\\
\\
\\
\\
/\\
\\
\&ungrib\\
~out\_format~=~'WPS',\\
~prefix~=~'FILE',\\
/\\
\\
\&metgrid\\
~fg\_name~=~'FILE',\\
~io\_form\_metgrid~=~2,\\
\\
\\
/\\
\\
\&mod\_levs\\
~press\_pa~=~201300~,~200100~,~100000~,\\
~~~~~~~~~~~~~95000~,~~90000~,\\
~~~~~~~~~~~~~85000~,~~80000~,\\
~~~~~~~~~~~~~75000~,~~70000~,\\
~~~~~~~~~~~~~65000~,~~60000~,\\
~~~~~~~~~~~~~55000~,~~50000~,\\
~~~~~~~~~~~~~45000~,~~40000~,\\
~~~~~~~~~~~~~35000~,~~30000~,\\
~~~~~~~~~~~~~25000~,~~20000~,\\
~~~~~~~~~~~~~15000~,~~10000~,\\
~~~~~~~~~~~~~~5000~,~~~1000\\
~/\\
\\
\\
\&domain\_wizard\\
\\
~grib\_vtable~=~'Vtable.GFS',\\
~dwiz\_name~~~~=2\_dominios\\
~dwiz\_desc~~~~=\\
~dwiz\_user\_rect\_x1~=2345\\
~dwiz\_user\_rect\_y1~=1875\\
~dwiz\_user\_rect\_x2~=2429\\
~dwiz\_user\_rect\_y2~=1955\\
~dwiz\_show\_political~=true\\
~dwiz\_center\_over\_gmt~=true\\
~dwiz\_latlon\_space\_in\_deg~=10\\
~dwiz\_latlon\_linecolor~=-8355712\\
~dwiz\_map\_scale\_pct~=50.0\\
~dwiz\_map\_vert\_scrollbar\_pos~=1625\\
~dwiz\_map\_horiz\_scrollbar\_pos~=1795\\
~dwiz\_gridpt\_dist\_km~=4.2\\
~dwiz\_mpi\_command~=null\\
~dwiz\_tcvitals~=null\\
~dwiz\_bigmap~=Y\\
/\\



%%%%%%%%%%%%%%%%%%%%%%%%%%%%%%%%%%%%%%%%%%%%%%%%%%%%%%%%%%%
%%%%%%%%%%%%%%INPUT
%%%%%%%%%%%%%%%%%%%%%%%%%%%%%%%%%%%%%%%%%%%%%%%%%%%%%%%%%%%

\section{\textit{namelist.input} usadas para la selección de los mejores dominios y subdomínios.}


%%%%Caso 2 36-12-4

\textbf{namelist.input Simulación 1}

\&time\_control~~~~~~~~~~~~\\
run\_days~~~~~~~~~~~~~~~~~=~0,\\
run\_hours~~~~~~~~~~~~~~~~=~0,\\
run\_minutes~~~~~~~~~~~~~~=~0,\\
run\_seconds~~~~~~~~~~~~~~=~0,\\
start\_year~~~~~~~~~~~~~~~=~2007,~2007,~2007,\\
start\_month~~~~~~~~~~~~~~=~02,~~~02,~~~02,~~\\
start\_day~~~~~~~~~~~~~~~~=~01,~~~01,~~~01,~~\\
start\_hour~~~~~~~~~~~~~~~=~00,~~~00,~~~00,~~\\
start\_minute~~~~~~~~~~~~~=~00,~~~00,~~~00,~~\\
start\_second~~~~~~~~~~~~~=~00,~~~00,~~~00,~~\\
end\_year~~~~~~~~~~~~~~~~~=~2007,~2007,~2007,\\
end\_month~~~~~~~~~~~~~~~~=~02,~~~02,~~~02,~~\\
end\_day~~~~~~~~~~~~~~~~~~=~05,~~~05,~~~05,~~\\
end\_hour~~~~~~~~~~~~~~~~~=~00,~~~00,~~~00,~~\\
end\_minute~~~~~~~~~~~~~~~=~00,~~~00,~~~00,~~\\
end\_second~~~~~~~~~~~~~~~=~00,~~~00,~~~00,~~\\
interval\_seconds~~~~~~~~~=~21600,\\
input\_from\_file~~~~~~~~~~=~.true.,~~~.true.,~~~.true.,\\
history\_interval~~~~~~~~~=~60,~~~~~~~60,~~~~~~~60,\\
frames\_per\_outfile~~~~~~~=~1,~~~~~~~~1,~~~~~~~~1,\\
restart~~~~~~~~~~~~~~~~~~=~.false.,\\
restart\_interval~~~~~~~~~=~5000,\\
io\_form\_history~~~~~~~~~~=~2,\\
io\_form\_restart~~~~~~~~~~=~2,\\
io\_form\_input~~~~~~~~~~~~=~2,\\
io\_form\_boundary~~~~~~~~~=~2,\\
debug\_level~~~~~~~~~~~~~~=~0,\\
/\\
\\
\&domains~~~~~~~~~~~~~~~~~\\
time\_step~~~~~~~~~~~~~~~~=~60,\\
time\_step\_fract\_num~~~~~~=~0,\\
time\_step\_fract\_den~~~~~~=~1,\\
max\_dom~~~~~~~~~~~~~~~~~~=~3,\\
e\_we~~~~~~~~~~~~~~~~~~~~~=~99,~~~~~~232,~~~~~~448,\\
e\_sn~~~~~~~~~~~~~~~~~~~~~=~94,~~~~~~214,~~~~~~385,\\
e\_vert~~~~~~~~~~~~~~~~~~~=~35,~~~~~~~35,~~~~~~~35,\\
p\_top\_requested~~~~~~~~~~=~5000,\\
num\_metgrid\_levels~~~~~~~=~27,\\
num\_metgrid\_soil\_levels~~=~4,\\
dx~~~~~~~~~~~~~~~~~~~~~~~=~36000,~~~12000,~~~4000,\\
dy~~~~~~~~~~~~~~~~~~~~~~~=~36000,~~~12000,~~~4000,\\
grid\_id~~~~~~~~~~~~~~~~~~=~1,~~~~~~~~2,~~~~~~~~3,\\
parent\_id~~~~~~~~~~~~~~~~=~1,~~~~~~~~1,~~~~~~~~2,\\
i\_parent\_start~~~~~~~~~~~=~1,~~~~~~~12,~~~~~~~33,\\
j\_parent\_start~~~~~~~~~~~=~1,~~~~~~~12,~~~~~~~33,\\
parent\_grid\_ratio~~~~~~~~=~1,~~~~~~~~3,~~~~~~~~3,\\
parent\_time\_step\_ratio~~~=~1,~~~~~~~~3,~~~~~~~~3,\\
feedback~~~~~~~~~~~~~~~~~=~1,\\
smooth\_option~~~~~~~~~~~~=~0,\\
/\\
\\
\&physics~~~~~~~~~~~~~~~~~\\
mp\_physics~~~~~~~~~~~~~~~=~3,~~3,~3,~\\
ra\_lw\_physics~~~~~~~~~~~~=~1,~~1,~1,~\\
ra\_sw\_physics~~~~~~~~~~~~=~1,~~1,~1,~\\
radt~~~~~~~~~~~~~~~~~~~~~=~30,~30,30,\\
sf\_sfclay\_physics~~~~~~~~=~1,~~1,~1,~\\
sf\_surface\_physics~~~~~~~=~1,~~1,~1,~\\
bl\_pbl\_physics~~~~~~~~~~~=~1,~~1,~1,~\\
bldt~~~~~~~~~~~~~~~~~~~~~=~0,~~0,~0,~\\
cu\_physics~~~~~~~~~~~~~~~=~1,~~1,~1,~\\
cudt~~~~~~~~~~~~~~~~~~~~~=~5,~~5,~5,~\\
isfflx~~~~~~~~~~~~~~~~~~~=~1,\\
ifsnow~~~~~~~~~~~~~~~~~~~=~0,\\
icloud~~~~~~~~~~~~~~~~~~~=~1,\\
surface\_input\_source~~~~~=~1,\\
num\_soil\_layers~~~~~~~~~~=~4,\\
sf\_urban\_physics~~~~~~~~~=~0,~~\\
maxiens~~~~~~~~~~~~~~~~~~=~1,\\
maxens~~~~~~~~~~~~~~~~~~~=~3,\\
maxens2~~~~~~~~~~~~~~~~~~=~3,\\
maxens3~~~~~~~~~~~~~~~~~~=~16,\\
ensdim~~~~~~~~~~~~~~~~~~~=~144,\\
/\\
\\
\&fdda~~~~~~~~~~~~~~~~~~~~\\
/\\
\\
\&dynamics~~~~~~~~~~~~~~~~\\
w\_damping~~~~~~~~~~~~~~~~=~0,\\
diff\_opt~~~~~~~~~~~~~~~~~=~1,\\
km\_opt~~~~~~~~~~~~~~~~~~~=~4,\\
diff\_6th\_opt~~~~~~~~~~~~~=~0,~~~~~~~~\\
diff\_6th\_factor~~~~~~~~~~=~0.12,~~~~~\\
base\_temp~~~~~~~~~~~~~~~~=~290.,\\
damp\_opt~~~~~~~~~~~~~~~~~=~0,\\
zdamp~~~~~~~~~~~~~~~~~~~~=~5000.,~~~~\\
dampcoef~~~~~~~~~~~~~~~~~=~0.2,~~~~~0.2,~~~~~0.2,~~~~~~~~\\
khdif~~~~~~~~~~~~~~~~~~~~=~0,~~~~~~~0,~~~~~~~0,~~~~~~~~~~\\
kvdif~~~~~~~~~~~~~~~~~~~~=~0,~~~~~~~0,~~~~~~~0,~~~~~~~~~~\\
non\_hydrostatic~~~~~~~~~~=~.true.,~~.true.,~~.true.,~~~~~\\
moist\_adv\_opt~~~~~~~~~~~~=~1,~~~~~~~1,~~~~~~~1,~~~~~~~~~~\\
scalar\_adv\_opt~~~~~~~~~~~=~1,~~~~~~~1,~~~~~~~1,~~~~~~~~~~\\
/\\
\\
\&bdy\_control~~~~~~~~~~~~~\\
spec\_bdy\_width~~~~~~~~~~~=~5,\\
spec\_zone~~~~~~~~~~~~~~~~=~1,\\
relax\_zone~~~~~~~~~~~~~~~=~4,\\
specified~~~~~~~~~~~~~~~~=~.true.,~~.true.,~~.true.,~~\\
nested~~~~~~~~~~~~~~~~~~~=~.false.,~.false.,~.false.,~~\\
/\\
\\
\&grib2~~~~~~~~~~~~~~~~~~~\\
/\\
\\
\&namelist\_quilt~~~~~~~~~~\\
nio\_tasks\_per\_group~~~~~~=~0,\\
nio\_groups~~~~~~~~~~~~~~~=~1,\\
/\\



\textbf{namelist.input Simulación 1}

%%%%% Caso 2 18-6-2

\textbf{namelist.input Simulación 2}

\&time\_control~~~~~~~~~~~~\\
run\_days~~~~~~~~~~~~~~~~~=~0,\\
run\_hours~~~~~~~~~~~~~~~~=~0,\\
run\_minutes~~~~~~~~~~~~~~=~0,\\
run\_seconds~~~~~~~~~~~~~~=~0,\\
start\_year~~~~~~~~~~~~~~~=~2007,~2007,~2007,\\
start\_month~~~~~~~~~~~~~~=~02,~~~02,~~~02,~~\\
start\_day~~~~~~~~~~~~~~~~=~01,~~~01,~~~01,~~\\
start\_hour~~~~~~~~~~~~~~~=~00,~~~00,~~~00,~~\\
start\_minute~~~~~~~~~~~~~=~00,~~~00,~~~00,~~\\
start\_second~~~~~~~~~~~~~=~00,~~~00,~~~00,~~\\
end\_year~~~~~~~~~~~~~~~~~=~2007,~2007,~2007,\\
end\_month~~~~~~~~~~~~~~~~=~02,~~~02,~~~02,~~\\
end\_day~~~~~~~~~~~~~~~~~~=~05,~~~05,~~~05,~~\\
end\_hour~~~~~~~~~~~~~~~~~=~00,~~~00,~~~00,~~\\
end\_minute~~~~~~~~~~~~~~~=~00,~~~00,~~~00,~~\\
end\_second~~~~~~~~~~~~~~~=~00,~~~00,~~~00,~~\\
interval\_seconds~~~~~~~~~=~21600,\\
input\_from\_file~~~~~~~~~~=~.true.,~~~.true.,~~~.true.,\\
history\_interval~~~~~~~~~=~60,~~~~~~~60,~~~~~~~60,\\
frames\_per\_outfile~~~~~~~=~1,~~~~~~~~1,~~~~~~~~1,\\
restart~~~~~~~~~~~~~~~~~~=~.false.,\\
restart\_interval~~~~~~~~~=~5000,\\
io\_form\_history~~~~~~~~~~=~2,\\
io\_form\_restart~~~~~~~~~~=~2,\\
io\_form\_input~~~~~~~~~~~~=~2,\\
io\_form\_boundary~~~~~~~~~=~2,\\
debug\_level~~~~~~~~~~~~~~=~0,\\
/\\
\\
\&domains~~~~~~~~~~~~~~~~~\\
time\_step~~~~~~~~~~~~~~~~=~20,\\
time\_step\_fract\_num~~~~~~=~0,\\
time\_step\_fract\_den~~~~~~=~1,\\
max\_dom~~~~~~~~~~~~~~~~~~=~3,\\
e\_we~~~~~~~~~~~~~~~~~~~~~=~99,~~~~~~232,~~~~~~448,\\
e\_sn~~~~~~~~~~~~~~~~~~~~~=~94,~~~~~~214,~~~~~~385,\\
e\_vert~~~~~~~~~~~~~~~~~~~=~35,~~~~~~~35,~~~~~~~35,\\
p\_top\_requested~~~~~~~~~~=~5000,\\
num\_metgrid\_levels~~~~~~~=~27,\\
num\_metgrid\_soil\_levels~~=~4,\\
dx~~~~~~~~~~~~~~~~~~~~~~~=~18000,~~~6000,~~~2000,\\
dy~~~~~~~~~~~~~~~~~~~~~~~=~18000,~~~6000,~~~2000,\\
grid\_id~~~~~~~~~~~~~~~~~~=~1,~~~~~~~~2,~~~~~~~~3,\\
parent\_id~~~~~~~~~~~~~~~~=~1,~~~~~~~~1,~~~~~~~~2,\\
i\_parent\_start~~~~~~~~~~~=~1,~~~~~~~12,~~~~~~~33,\\
j\_parent\_start~~~~~~~~~~~=~1,~~~~~~~12,~~~~~~~33,\\
parent\_grid\_ratio~~~~~~~~=~1,~~~~~~~~3,~~~~~~~~3,\\
parent\_time\_step\_ratio~~~=~1,~~~~~~~~3,~~~~~~~~3,\\
feedback~~~~~~~~~~~~~~~~~=~1,\\
smooth\_option~~~~~~~~~~~~=~0,\\
/\\
\\
\&physics~~~~~~~~~~~~~~~~~\\
mp\_physics~~~~~~~~~~~~~~~=~3,~~3,~3,~\\
ra\_lw\_physics~~~~~~~~~~~~=~1,~~1,~1,~\\
ra\_sw\_physics~~~~~~~~~~~~=~1,~~1,~1,~\\
radt~~~~~~~~~~~~~~~~~~~~~=~30,~30,30,\\
sf\_sfclay\_physics~~~~~~~~=~1,~~1,~1,~\\
sf\_surface\_physics~~~~~~~=~1,~~1,~1,~\\
bl\_pbl\_physics~~~~~~~~~~~=~1,~~1,~1,~\\
bldt~~~~~~~~~~~~~~~~~~~~~=~0,~~0,~0,~\\
cu\_physics~~~~~~~~~~~~~~~=~1,~~1,~1,~\\
cudt~~~~~~~~~~~~~~~~~~~~~=~5,~~5,~5,~\\
isfflx~~~~~~~~~~~~~~~~~~~=~1,\\
ifsnow~~~~~~~~~~~~~~~~~~~=~0,\\
icloud~~~~~~~~~~~~~~~~~~~=~1,\\
surface\_input\_source~~~~~=~1,\\
num\_soil\_layers~~~~~~~~~~=~4,\\
sf\_urban\_physics~~~~~~~~~=~0,~~\\
maxiens~~~~~~~~~~~~~~~~~~=~1,\\
maxens~~~~~~~~~~~~~~~~~~~=~3,\\
maxens2~~~~~~~~~~~~~~~~~~=~3,\\
maxens3~~~~~~~~~~~~~~~~~~=~16,\\
ensdim~~~~~~~~~~~~~~~~~~~=~144,\\
/\\
\\
\&fdda~~~~~~~~~~~~~~~~~~~~\\
/\\
\\
\&dynamics~~~~~~~~~~~~~~~~\\
w\_damping~~~~~~~~~~~~~~~~=~0,\\
diff\_opt~~~~~~~~~~~~~~~~~=~1,\\
km\_opt~~~~~~~~~~~~~~~~~~~=~4,\\
diff\_6th\_opt~~~~~~~~~~~~~=~0,~~~~~~~~\\
diff\_6th\_factor~~~~~~~~~~=~0.12,~~~~~\\
base\_temp~~~~~~~~~~~~~~~~=~290.,\\
damp\_opt~~~~~~~~~~~~~~~~~=~0,\\
zdamp~~~~~~~~~~~~~~~~~~~~=~5000.,~~~~\\
dampcoef~~~~~~~~~~~~~~~~~=~0.2,~~~~~0.2,~~~~~0.2,~~~~~~~~\\
khdif~~~~~~~~~~~~~~~~~~~~=~0,~~~~~~~0,~~~~~~~0,~~~~~~~~~~\\
kvdif~~~~~~~~~~~~~~~~~~~~=~0,~~~~~~~0,~~~~~~~0,~~~~~~~~~~\\
non\_hydrostatic~~~~~~~~~~=~.true.,~~.true.,~~.true.,~~~~~\\
moist\_adv\_opt~~~~~~~~~~~~=~1,~~~~~~~1,~~~~~~~1,~~~~~~~~~~\\
scalar\_adv\_opt~~~~~~~~~~~=~1,~~~~~~~1,~~~~~~~1,~~~~~~~~~~\\
/\\
\\
\&bdy\_control~~~~~~~~~~~~~\\
spec\_bdy\_width~~~~~~~~~~~=~5,\\
spec\_zone~~~~~~~~~~~~~~~~=~1,\\
relax\_zone~~~~~~~~~~~~~~~=~4,\\
specified~~~~~~~~~~~~~~~~=~.true.,~~.true.,~~.true.,~~\\
nested~~~~~~~~~~~~~~~~~~~=~.false.,~.false.,~.false.,~~\\
/\\
\\
\&grib2~~~~~~~~~~~~~~~~~~~\\
/\\
\\
\&namelist\_quilt~~~~~~~~~~\\
nio\_tasks\_per\_group~~~~~~=~0,\\
nio\_groups~~~~~~~~~~~~~~~=~1,\\
/\\



%%Caso 3

\textbf{namelist.input Simulación 3}

\&time\_control~~~~~~~~~~~~\\
run\_days~~~~~~~~~~~~~~~~~=~0,\\
run\_hours~~~~~~~~~~~~~~~~=~0,\\
run\_minutes~~~~~~~~~~~~~~=~0,\\
run\_seconds~~~~~~~~~~~~~~=~0,\\
start\_year~~~~~~~~~~~~~~~=~2007,~2007,~\\
start\_month~~~~~~~~~~~~~~=~02,~~~02,~~~\\
start\_day~~~~~~~~~~~~~~~~=~01,~~~01,~~~\\
start\_hour~~~~~~~~~~~~~~~=~00,~~~00,~~~\\
start\_minute~~~~~~~~~~~~~=~00,~~~00,~~~\\
start\_second~~~~~~~~~~~~~=~00,~~~00,~~~\\
end\_year~~~~~~~~~~~~~~~~~=~2007,~2007,~\\
end\_month~~~~~~~~~~~~~~~~=~02,~~~02,~~~\\
end\_day~~~~~~~~~~~~~~~~~~=~05,~~~05,~~~\\
end\_hour~~~~~~~~~~~~~~~~~=~00,~~~00,~~~\\
end\_minute~~~~~~~~~~~~~~~=~00,~~~00,~~~\\
end\_second~~~~~~~~~~~~~~~=~00,~~~00,~~~\\
interval\_seconds~~~~~~~~~=~21600,\\
input\_from\_file~~~~~~~~~~=~.true.,~~~.true.,~~\\
history\_interval~~~~~~~~~=~60,~~~~~~~60,~~~~~~\\
frames\_per\_outfile~~~~~~~=~1,~~~~~~~~1,~~~~~~~\\
restart~~~~~~~~~~~~~~~~~~=~.false.,\\
restart\_interval~~~~~~~~~=~5000,\\
io\_form\_history~~~~~~~~~~=~2,\\
io\_form\_restart~~~~~~~~~~=~2,\\
io\_form\_input~~~~~~~~~~~~=~2,\\
io\_form\_boundary~~~~~~~~~=~2,\\
debug\_level~~~~~~~~~~~~~~=~0,\\
/\\
\\
\&domains~~~~~~~~~~~~~~~~~\\
time\_step~~~~~~~~~~~~~~~~=~30,\\
time\_step\_fract\_num~~~~~~=~0,\\
time\_step\_fract\_den~~~~~~=~1,\\
max\_dom~~~~~~~~~~~~~~~~~~=~2,\\
e\_we~~~~~~~~~~~~~~~~~~~~~=~99,~~~~~~232,~\\
e\_sn~~~~~~~~~~~~~~~~~~~~~=~94,~~~~~~214,~\\
e\_vert~~~~~~~~~~~~~~~~~~~=~35,~~~~~~~35,~\\
p\_top\_requested~~~~~~~~~~=~5000,\\
num\_metgrid\_levels~~~~~~~=~27,\\
num\_metgrid\_soil\_levels~~=~4,\\
dx~~~~~~~~~~~~~~~~~~~~~~~=~10000,~~~3333.334,~\\
dy~~~~~~~~~~~~~~~~~~~~~~~=~10000,~~~3333.334,~\\
grid\_id~~~~~~~~~~~~~~~~~~=~1,~~~~~~~~2,~~\\
parent\_id~~~~~~~~~~~~~~~~=~1,~~~~~~~~1,~~\\
i\_parent\_start~~~~~~~~~~~=~1,~~~~~~~12,~~\\
j\_parent\_start~~~~~~~~~~~=~1,~~~~~~~12,~~\\
parent\_grid\_ratio~~~~~~~~=~1,~~~~~~~~3,~~\\
parent\_time\_step\_ratio~~~=~1,~~~~~~~~3,~~\\
feedback~~~~~~~~~~~~~~~~~=~1,\\
smooth\_option~~~~~~~~~~~~=~0,\\
/\\
\\
\&physics~~~~~~~~~~~~~~~~~\\
mp\_physics~~~~~~~~~~~~~~~=~3,~~3,~\\
ra\_lw\_physics~~~~~~~~~~~~=~1,~~1,~\\
ra\_sw\_physics~~~~~~~~~~~~=~1,~~1,~\\
radt~~~~~~~~~~~~~~~~~~~~~=~30,~30,\\
sf\_sfclay\_physics~~~~~~~~=~1,~~1,~\\
sf\_surface\_physics~~~~~~~=~1,~~1,~\\
bl\_pbl\_physics~~~~~~~~~~~=~1,~~1,~\\
bldt~~~~~~~~~~~~~~~~~~~~~=~0,~~0,~\\
cu\_physics~~~~~~~~~~~~~~~=~1,~~1,~\\
cudt~~~~~~~~~~~~~~~~~~~~~=~5,~~5,~\\
isfflx~~~~~~~~~~~~~~~~~~~=~1,\\
ifsnow~~~~~~~~~~~~~~~~~~~=~0,\\
icloud~~~~~~~~~~~~~~~~~~~=~1,\\
surface\_input\_source~~~~~=~1,\\
num\_soil\_layers~~~~~~~~~~=~4,\\
sf\_urban\_physics~~~~~~~~~=~0,~~\\
maxiens~~~~~~~~~~~~~~~~~~=~1,\\
maxens~~~~~~~~~~~~~~~~~~~=~3,\\
maxens2~~~~~~~~~~~~~~~~~~=~3,\\
maxens3~~~~~~~~~~~~~~~~~~=~16,\\
ensdim~~~~~~~~~~~~~~~~~~~=~144,\\
/\\
\\
\&fdda~~~~~~~~~~~~~~~~~~~~\\
/\\
\\
\&dynamics~~~~~~~~~~~~~~~~\\
w\_damping~~~~~~~~~~~~~~~~=~0,\\
diff\_opt~~~~~~~~~~~~~~~~~=~1,\\
km\_opt~~~~~~~~~~~~~~~~~~~=~4,\\
diff\_6th\_opt~~~~~~~~~~~~~=~0,~~~~~~~~\\
diff\_6th\_factor~~~~~~~~~~=~0.12,~~~~~\\
base\_temp~~~~~~~~~~~~~~~~=~290.,\\
damp\_opt~~~~~~~~~~~~~~~~~=~0,\\
zdamp~~~~~~~~~~~~~~~~~~~~=~5000.,~~~~\\
dampcoef~~~~~~~~~~~~~~~~~=~0.2,~~~~~0.2,~~~~\\
khdif~~~~~~~~~~~~~~~~~~~~=~0,~~~~~~~0,~~~~~~\\
kvdif~~~~~~~~~~~~~~~~~~~~=~0,~~~~~~~0,~~~~~~\\
non\_hydrostatic~~~~~~~~~~=~.true.,~~.true.,~\\
moist\_adv\_opt~~~~~~~~~~~~=~1,~~~~~~~1,~~~~~~\\
scalar\_adv\_opt~~~~~~~~~~~=~1,~~~~~~~1,~~~~~~\\
/\\
\\
\&bdy\_control~~~~~~~~~~~~~\\
spec\_bdy\_width~~~~~~~~~~~=~5,\\
spec\_zone~~~~~~~~~~~~~~~~=~1,\\
relax\_zone~~~~~~~~~~~~~~~=~4,\\
specified~~~~~~~~~~~~~~~~=~.true.,~~.true.,~\\
nested~~~~~~~~~~~~~~~~~~~=~.false.,~.false.,\\
/\\
\\
\&grib2~~~~~~~~~~~~~~~~~~~\\
/\\
\\
\&namelist\_quilt~~~~~~~~~~\\
nio\_tasks\_per\_group~~~~~~=~0,\\
nio\_groups~~~~~~~~~~~~~~~=~1,\\
/\\



%% Caso 4

\textbf{namelist.input Simulación 4}
\&time\_control~~~~~~~~~~~~\\
run\_days~~~~~~~~~~~~~~~~~=~0,\\
run\_hours~~~~~~~~~~~~~~~~=~0,\\
run\_minutes~~~~~~~~~~~~~~=~0,\\
run\_seconds~~~~~~~~~~~~~~=~0,\\
start\_year~~~~~~~~~~~~~~~=~2007,~2007,~\\
start\_month~~~~~~~~~~~~~~=~02,~~~02,~~~\\
start\_day~~~~~~~~~~~~~~~~=~01,~~~01,~~~\\
start\_hour~~~~~~~~~~~~~~~=~00,~~~00,~~~\\
start\_minute~~~~~~~~~~~~~=~00,~~~00,~~~\\
start\_second~~~~~~~~~~~~~=~00,~~~00,~~~\\
end\_year~~~~~~~~~~~~~~~~~=~2007,~2007,~\\
end\_month~~~~~~~~~~~~~~~~=~02,~~~02,~~~\\
end\_day~~~~~~~~~~~~~~~~~~=~05,~~~05,~~~\\
end\_hour~~~~~~~~~~~~~~~~~=~00,~~~00,~~~\\
end\_minute~~~~~~~~~~~~~~~=~00,~~~00,~~~\\
end\_second~~~~~~~~~~~~~~~=~00,~~~00,~~~\\
interval\_seconds~~~~~~~~~=~21600,\\
input\_from\_file~~~~~~~~~~=~.true.,~~~.true.,~\\
history\_interval~~~~~~~~~=~60,~~~~~~~60,~~~~~\\
frames\_per\_outfile~~~~~~~=~1,~~~~~~~~1,~~~~~~\\
restart~~~~~~~~~~~~~~~~~~=~.false.,\\
restart\_interval~~~~~~~~~=~5000,\\
io\_form\_history~~~~~~~~~~=~2,\\
io\_form\_restart~~~~~~~~~~=~2,\\
io\_form\_input~~~~~~~~~~~~=~2,\\
io\_form\_boundary~~~~~~~~~=~2,\\
debug\_level~~~~~~~~~~~~~~=~0,\\
/\\
\\
\&domains~~~~~~~~~~~~~~~~~\\
time\_step~~~~~~~~~~~~~~~~=~60,\\
time\_step\_fract\_num~~~~~~=~0,\\
time\_step\_fract\_den~~~~~~=~1,\\
max\_dom~~~~~~~~~~~~~~~~~~=~2,\\
e\_we~~~~~~~~~~~~~~~~~~~~~=~100,~~~~~~148,~~~~\\
e\_sn~~~~~~~~~~~~~~~~~~~~~=~99,~~~~~~196,~~~~\\
e\_vert~~~~~~~~~~~~~~~~~~~=~35,~~~~~~~35,~~~~\\
p\_top\_requested~~~~~~~~~~=~5000,\\
num\_metgrid\_levels~~~~~~~=~27,\\
num\_metgrid\_soil\_levels~~=~4,\\
dx~~~~~~~~~~~~~~~~~~~~~~~=~10000,~~~3333.334,\\
dy~~~~~~~~~~~~~~~~~~~~~~~=~10000,~~~3333.334,\\
grid\_id~~~~~~~~~~~~~~~~~~=~1,~~~~~~~~2,~~~~~~\\
parent\_id~~~~~~~~~~~~~~~~=~1,~~~~~~~~1,~~~~~~\\
i\_parent\_start~~~~~~~~~~~=~1,~~~~~~~25,~~~~~~\\
j\_parent\_start~~~~~~~~~~~=~1,~~~~~~~26,~~~~~~\\
parent\_grid\_ratio~~~~~~~~=~1,~~~~~~~~3,~~~~~~\\
parent\_time\_step\_ratio~~~=~1,~~~~~~~~3,~~~~~~\\
feedback~~~~~~~~~~~~~~~~~=~1,\\
smooth\_option~~~~~~~~~~~~=~0,\\
/\\
\\
\&physics~~~~~~~~~~~~~~~~~\\
mp\_physics~~~~~~~~~~~~~~~=~3,~~3,~\\
ra\_lw\_physics~~~~~~~~~~~~=~1,~~1,~\\
ra\_sw\_physics~~~~~~~~~~~~=~1,~~1,~\\
radt~~~~~~~~~~~~~~~~~~~~~=~30,~30,\\
sf\_sfclay\_physics~~~~~~~~=~1,~~1,~\\
sf\_surface\_physics~~~~~~~=~1,~~1,~\\
bl\_pbl\_physics~~~~~~~~~~~=~1,~~1,~\\
bldt~~~~~~~~~~~~~~~~~~~~~=~0,~~0,~\\
cu\_physics~~~~~~~~~~~~~~~=~1,~~1,~\\
cudt~~~~~~~~~~~~~~~~~~~~~=~5,~~5,~\\
isfflx~~~~~~~~~~~~~~~~~~~=~1,\\
ifsnow~~~~~~~~~~~~~~~~~~~=~0,\\
icloud~~~~~~~~~~~~~~~~~~~=~1,\\
surface\_input\_source~~~~~=~1,\\
num\_soil\_layers~~~~~~~~~~=~4,\\
sf\_urban\_physics~~~~~~~~~=~0,~~\\
maxiens~~~~~~~~~~~~~~~~~~=~1,\\
maxens~~~~~~~~~~~~~~~~~~~=~3,\\
maxens2~~~~~~~~~~~~~~~~~~=~3,\\
maxens3~~~~~~~~~~~~~~~~~~=~16,\\
ensdim~~~~~~~~~~~~~~~~~~~=~144,\\
/\\
\\
\&fdda~~~~~~~~~~~~~~~~~~~~\\
/\\
\\
\&dynamics~~~~~~~~~~~~~~~~\\
w\_damping~~~~~~~~~~~~~~~~=~0,\\
diff\_opt~~~~~~~~~~~~~~~~~=~1,\\
km\_opt~~~~~~~~~~~~~~~~~~~=~4,\\
diff\_6th\_opt~~~~~~~~~~~~~=~0,~~~~~~~~\\
diff\_6th\_factor~~~~~~~~~~=~0.12,~~~~~\\
base\_temp~~~~~~~~~~~~~~~~=~290.,\\
damp\_opt~~~~~~~~~~~~~~~~~=~0,\\
zdamp~~~~~~~~~~~~~~~~~~~~=~5000.,~~~~\\
dampcoef~~~~~~~~~~~~~~~~~=~0.2,~~~~~0.2,~~~~~\\
khdif~~~~~~~~~~~~~~~~~~~~=~0,~~~~~~~0,~~~~~~~\\
kvdif~~~~~~~~~~~~~~~~~~~~=~0,~~~~~~~0,~~~~~~~\\
non\_hydrostatic~~~~~~~~~~=~.true.,~~.true.,~~\\
moist\_adv\_opt~~~~~~~~~~~~=~1,~~~~~~~1,~~~~~~~\\
scalar\_adv\_opt~~~~~~~~~~~=~1,~~~~~~~1,~~~~~~~\\
/\\
\\
\&bdy\_control~~~~~~~~~~~~~\\
spec\_bdy\_width~~~~~~~~~~~=~5,\\
spec\_zone~~~~~~~~~~~~~~~~=~1,\\
relax\_zone~~~~~~~~~~~~~~~=~4,\\
specified~~~~~~~~~~~~~~~~=~.true.,~~.true.,~\\
nested~~~~~~~~~~~~~~~~~~~=~.false.,~.false.,\\
/\\
\\
\&grib2~~~~~~~~~~~~~~~~~~~\\
/\\
\\
\&namelist\_quilt~~~~~~~~~~\\
nio\_tasks\_per\_group~~~~~~=~0,\\
nio\_groups~~~~~~~~~~~~~~~=~1,\\
/\\


%%% Simulación 5*

\textbf{namelist.input Simulación 5}

\&time\_control~~~~~~~~~~~~\\
run\_days~~~~~~~~~~~~~~~~~=~0,\\
run\_hours~~~~~~~~~~~~~~~~=~0,\\
run\_minutes~~~~~~~~~~~~~~=~0,\\
run\_seconds~~~~~~~~~~~~~~=~0,\\
start\_year~~~~~~~~~~~~~~~=~2007,~2007,~\\
start\_month~~~~~~~~~~~~~~=~02,~~~02,~~~\\
start\_day~~~~~~~~~~~~~~~~=~01,~~~01,~~~\\
start\_hour~~~~~~~~~~~~~~~=~00,~~~00,~~~\\
start\_minute~~~~~~~~~~~~~=~00,~~~00,~~~\\
start\_second~~~~~~~~~~~~~=~00,~~~00,~~~\\
end\_year~~~~~~~~~~~~~~~~~=~2007,~2007,~\\
end\_month~~~~~~~~~~~~~~~~=~02,~~~02,~~~\\
end\_day~~~~~~~~~~~~~~~~~~=~05,~~~05,~~~\\
end\_hour~~~~~~~~~~~~~~~~~=~00,~~~00,~~~\\
end\_minute~~~~~~~~~~~~~~~=~00,~~~00,~~~\\
end\_second~~~~~~~~~~~~~~~=~00,~~~00,~~~\\
interval\_seconds~~~~~~~~~=~21600,\\
input\_from\_file~~~~~~~~~~=~.true.,~~~.true.,~\\
history\_interval~~~~~~~~~=~60,~~~~~~~60,~~~~~\\
frames\_per\_outfile~~~~~~~=~1,~~~~~~~~1,~~~~~~\\
restart~~~~~~~~~~~~~~~~~~=~.false.,\\
restart\_interval~~~~~~~~~=~5000,\\
io\_form\_history~~~~~~~~~~=~2,\\
io\_form\_restart~~~~~~~~~~=~2,\\
io\_form\_input~~~~~~~~~~~~=~2,\\
io\_form\_boundary~~~~~~~~~=~2,\\
debug\_level~~~~~~~~~~~~~~=~0,\\
/\\
\\
\&domains~~~~~~~~~~~~~~~~~\\
time\_step~~~~~~~~~~~~~~~~=~60,\\
time\_step\_fract\_num~~~~~~=~0,\\
time\_step\_fract\_den~~~~~~=~1,\\
max\_dom~~~~~~~~~~~~~~~~~~=~2,\\
e\_we~~~~~~~~~~~~~~~~~~~~~=~100,~~~~~~148,~~~~\\
e\_sn~~~~~~~~~~~~~~~~~~~~~=~99,~~~~~~196,~~~~\\
e\_vert~~~~~~~~~~~~~~~~~~~=~35,~~~~~~~35,~~~~\\
p\_top\_requested~~~~~~~~~~=~5000,\\
num\_metgrid\_levels~~~~~~~=~27,\\
num\_metgrid\_soil\_levels~~=~4,\\
dx~~~~~~~~~~~~~~~~~~~~~~~=~12000,~~~4000,\\
dy~~~~~~~~~~~~~~~~~~~~~~~=~12000,~~~4000,\\
grid\_id~~~~~~~~~~~~~~~~~~=~1,~~~~~~~~2,~~~~~~\\
parent\_id~~~~~~~~~~~~~~~~=~1,~~~~~~~~1,~~~~~~\\
i\_parent\_start~~~~~~~~~~~=~1,~~~~~~~25,~~~~~~\\
j\_parent\_start~~~~~~~~~~~=~1,~~~~~~~26,~~~~~~\\
parent\_grid\_ratio~~~~~~~~=~1,~~~~~~~~3,~~~~~~\\
parent\_time\_step\_ratio~~~=~1,~~~~~~~~3,~~~~~~\\
feedback~~~~~~~~~~~~~~~~~=~1,\\
smooth\_option~~~~~~~~~~~~=~0,\\
/\\
\\
\&physics~~~~~~~~~~~~~~~~~\\
mp\_physics~~~~~~~~~~~~~~~=~3,~~3,~\\
ra\_lw\_physics~~~~~~~~~~~~=~1,~~1,~\\
ra\_sw\_physics~~~~~~~~~~~~=~1,~~1,~\\
radt~~~~~~~~~~~~~~~~~~~~~=~30,~30,\\
sf\_sfclay\_physics~~~~~~~~=~1,~~1,~\\
sf\_surface\_physics~~~~~~~=~1,~~1,~\\
bl\_pbl\_physics~~~~~~~~~~~=~1,~~1,~\\
bldt~~~~~~~~~~~~~~~~~~~~~=~0,~~0,~\\
cu\_physics~~~~~~~~~~~~~~~=~1,~~1,~\\
cudt~~~~~~~~~~~~~~~~~~~~~=~5,~~5,~\\
isfflx~~~~~~~~~~~~~~~~~~~=~1,\\
ifsnow~~~~~~~~~~~~~~~~~~~=~0,\\
icloud~~~~~~~~~~~~~~~~~~~=~1,\\
surface\_input\_source~~~~~=~1,\\
num\_soil\_layers~~~~~~~~~~=~4,\\
sf\_urban\_physics~~~~~~~~~=~0,~~\\
maxiens~~~~~~~~~~~~~~~~~~=~1,\\
maxens~~~~~~~~~~~~~~~~~~~=~3,\\
maxens2~~~~~~~~~~~~~~~~~~=~3,\\
maxens3~~~~~~~~~~~~~~~~~~=~16,\\
ensdim~~~~~~~~~~~~~~~~~~~=~144,\\
/\\
\\
\&fdda~~~~~~~~~~~~~~~~~~~~\\
/\\
\\
\&dynamics~~~~~~~~~~~~~~~~\\
w\_damping~~~~~~~~~~~~~~~~=~0,\\
diff\_opt~~~~~~~~~~~~~~~~~=~1,\\
km\_opt~~~~~~~~~~~~~~~~~~~=~4,\\
diff\_6th\_opt~~~~~~~~~~~~~=~0,~~~~~~~~\\
diff\_6th\_factor~~~~~~~~~~=~0.12,~~~~~\\
base\_temp~~~~~~~~~~~~~~~~=~290.,\\
damp\_opt~~~~~~~~~~~~~~~~~=~0,\\
zdamp~~~~~~~~~~~~~~~~~~~~=~5000.,~~~~\\
dampcoef~~~~~~~~~~~~~~~~~=~0.2,~~~~~0.2,~~~~~\\
khdif~~~~~~~~~~~~~~~~~~~~=~0,~~~~~~~0,~~~~~~~\\
kvdif~~~~~~~~~~~~~~~~~~~~=~0,~~~~~~~0,~~~~~~~\\
non\_hydrostatic~~~~~~~~~~=~.true.,~~.true.,~~\\
moist\_adv\_opt~~~~~~~~~~~~=~1,~~~~~~~1,~~~~~~~\\
scalar\_adv\_opt~~~~~~~~~~~=~1,~~~~~~~1,~~~~~~~\\
/\\
\\
\&bdy\_control~~~~~~~~~~~~~\\
spec\_bdy\_width~~~~~~~~~~~=~5,\\
spec\_zone~~~~~~~~~~~~~~~~=~1,\\
relax\_zone~~~~~~~~~~~~~~~=~4,\\
specified~~~~~~~~~~~~~~~~=~.true.,~~.true.,~\\
nested~~~~~~~~~~~~~~~~~~~=~.false.,~.false.,\\
/\\
\\
\&grib2~~~~~~~~~~~~~~~~~~~\\
/\\
\\
\&namelist\_quilt~~~~~~~~~~\\
nio\_tasks\_per\_group~~~~~~=~0,\\
nio\_groups~~~~~~~~~~~~~~~=~1,\\
/\\


%%%%
% Simulación 6

\textbf{namelist.input Simulación 6}

\&time\_control~~~~~~~~~~~~\\
run\_days~~~~~~~~~~~~~~~~~=~0,\\
run\_hours~~~~~~~~~~~~~~~~=~0,\\
run\_minutes~~~~~~~~~~~~~~=~0,\\
run\_seconds~~~~~~~~~~~~~~=~0,\\
start\_year~~~~~~~~~~~~~~~=~2007,~2007,~\\
start\_month~~~~~~~~~~~~~~=~02,~~~02,~~~\\
start\_day~~~~~~~~~~~~~~~~=~01,~~~01,~~~\\
start\_hour~~~~~~~~~~~~~~~=~00,~~~00,~~~\\
start\_minute~~~~~~~~~~~~~=~00,~~~00,~~~\\
start\_second~~~~~~~~~~~~~=~00,~~~00,~~~\\
end\_year~~~~~~~~~~~~~~~~~=~2007,~2007,~\\
end\_month~~~~~~~~~~~~~~~~=~02,~~~02,~~~\\
end\_day~~~~~~~~~~~~~~~~~~=~05,~~~05,~~~\\
end\_hour~~~~~~~~~~~~~~~~~=~00,~~~00,~~~\\
end\_minute~~~~~~~~~~~~~~~=~00,~~~00,~~~\\
end\_second~~~~~~~~~~~~~~~=~00,~~~00,~~~\\
interval\_seconds~~~~~~~~~=~21600,\\
input\_from\_file~~~~~~~~~~=~.true.,~~~.true.,~\\
history\_interval~~~~~~~~~=~60,~~~~~~~60,~~~~~\\
frames\_per\_outfile~~~~~~~=~1,~~~~~~~~1,~~~~~~\\
restart~~~~~~~~~~~~~~~~~~=~.false.,\\
restart\_interval~~~~~~~~~=~5000,\\
io\_form\_history~~~~~~~~~~=~2,\\
io\_form\_restart~~~~~~~~~~=~2,\\
io\_form\_input~~~~~~~~~~~~=~2,\\
io\_form\_boundary~~~~~~~~~=~2,\\
debug\_level~~~~~~~~~~~~~~=~0,\\
/\\
\\
\&domains~~~~~~~~~~~~~~~~~\\
time\_step~~~~~~~~~~~~~~~~=~60,\\
time\_step\_fract\_num~~~~~~=~0,\\
time\_step\_fract\_den~~~~~~=~1,\\
max\_dom~~~~~~~~~~~~~~~~~~=~2,\\
e\_we~~~~~~~~~~~~~~~~~~~~~=~100,~~~~~~148,~~~~\\
e\_sn~~~~~~~~~~~~~~~~~~~~~=~99,~~~~~~196,~~~~\\
e\_vert~~~~~~~~~~~~~~~~~~~=~35,~~~~~~~35,~~~~\\
p\_top\_requested~~~~~~~~~~=~5000,\\
num\_metgrid\_levels~~~~~~~=~27,\\
num\_metgrid\_soil\_levels~~=~4,\\
dx~~~~~~~~~~~~~~~~~~~~~~~=~15000,~~~5000,\\
dy~~~~~~~~~~~~~~~~~~~~~~~=~15000,~~~5000,\\
grid\_id~~~~~~~~~~~~~~~~~~=~1,~~~~~~~~2,~~~~~~\\
parent\_id~~~~~~~~~~~~~~~~=~1,~~~~~~~~1,~~~~~~\\
i\_parent\_start~~~~~~~~~~~=~1,~~~~~~~25,~~~~~~\\
j\_parent\_start~~~~~~~~~~~=~1,~~~~~~~26,~~~~~~\\
parent\_grid\_ratio~~~~~~~~=~1,~~~~~~~~3,~~~~~~\\
parent\_time\_step\_ratio~~~=~1,~~~~~~~~3,~~~~~~\\
feedback~~~~~~~~~~~~~~~~~=~1,\\
smooth\_option~~~~~~~~~~~~=~0,\\
/\\
\\
\&physics~~~~~~~~~~~~~~~~~\\
mp\_physics~~~~~~~~~~~~~~~=~3,~~3,~\\
ra\_lw\_physics~~~~~~~~~~~~=~1,~~1,~\\
ra\_sw\_physics~~~~~~~~~~~~=~1,~~1,~\\
radt~~~~~~~~~~~~~~~~~~~~~=~30,~30,\\
sf\_sfclay\_physics~~~~~~~~=~1,~~1,~\\
sf\_surface\_physics~~~~~~~=~1,~~1,~\\
bl\_pbl\_physics~~~~~~~~~~~=~1,~~1,~\\
bldt~~~~~~~~~~~~~~~~~~~~~=~0,~~0,~\\
cu\_physics~~~~~~~~~~~~~~~=~1,~~1,~\\
cudt~~~~~~~~~~~~~~~~~~~~~=~5,~~5,~\\
isfflx~~~~~~~~~~~~~~~~~~~=~1,\\
ifsnow~~~~~~~~~~~~~~~~~~~=~0,\\
icloud~~~~~~~~~~~~~~~~~~~=~1,\\
surface\_input\_source~~~~~=~1,\\
num\_soil\_layers~~~~~~~~~~=~4,\\
sf\_urban\_physics~~~~~~~~~=~0,~~\\
maxiens~~~~~~~~~~~~~~~~~~=~1,\\
maxens~~~~~~~~~~~~~~~~~~~=~3,\\
maxens2~~~~~~~~~~~~~~~~~~=~3,\\
maxens3~~~~~~~~~~~~~~~~~~=~16,\\
ensdim~~~~~~~~~~~~~~~~~~~=~144,\\
/\\
\\
\&fdda~~~~~~~~~~~~~~~~~~~~\\
/\\
\\
\&dynamics~~~~~~~~~~~~~~~~\\
w\_damping~~~~~~~~~~~~~~~~=~0,\\
diff\_opt~~~~~~~~~~~~~~~~~=~1,\\
km\_opt~~~~~~~~~~~~~~~~~~~=~4,\\
diff\_6th\_opt~~~~~~~~~~~~~=~0,~~~~~~~~\\
diff\_6th\_factor~~~~~~~~~~=~0.12,~~~~~\\
base\_temp~~~~~~~~~~~~~~~~=~290.,\\
damp\_opt~~~~~~~~~~~~~~~~~=~0,\\
zdamp~~~~~~~~~~~~~~~~~~~~=~5000.,~~~~\\
dampcoef~~~~~~~~~~~~~~~~~=~0.2,~~~~~0.2,~~~~~\\
khdif~~~~~~~~~~~~~~~~~~~~=~0,~~~~~~~0,~~~~~~~\\
kvdif~~~~~~~~~~~~~~~~~~~~=~0,~~~~~~~0,~~~~~~~\\
non\_hydrostatic~~~~~~~~~~=~.true.,~~.true.,~~\\
moist\_adv\_opt~~~~~~~~~~~~=~1,~~~~~~~1,~~~~~~~\\
scalar\_adv\_opt~~~~~~~~~~~=~1,~~~~~~~1,~~~~~~~\\
/\\
\\
\&bdy\_control~~~~~~~~~~~~~\\
spec\_bdy\_width~~~~~~~~~~~=~5,\\
spec\_zone~~~~~~~~~~~~~~~~=~1,\\
relax\_zone~~~~~~~~~~~~~~~=~4,\\
specified~~~~~~~~~~~~~~~~=~.true.,~~.true.,~\\
nested~~~~~~~~~~~~~~~~~~~=~.false.,~.false.,\\
/\\
\\
\&grib2~~~~~~~~~~~~~~~~~~~\\
/\\
\\
\&namelist\_quilt~~~~~~~~~~\\
nio\_tasks\_per\_group~~~~~~=~0,\\
nio\_groups~~~~~~~~~~~~~~~=~1,\\
/\\


%%%%
% Simulación 7

\textbf{namelist.input Simulación 7}

\&time\_control~~~~~~~~~~~~\\
run\_days~~~~~~~~~~~~~~~~~=~0,\\
run\_hours~~~~~~~~~~~~~~~~=~0,\\
run\_minutes~~~~~~~~~~~~~~=~0,\\
run\_seconds~~~~~~~~~~~~~~=~0,\\
start\_year~~~~~~~~~~~~~~~=~2007,~2007,~\\
start\_month~~~~~~~~~~~~~~=~02,~~~02,~~~\\
start\_day~~~~~~~~~~~~~~~~=~01,~~~01,~~~\\
start\_hour~~~~~~~~~~~~~~~=~00,~~~00,~~~\\
start\_minute~~~~~~~~~~~~~=~00,~~~00,~~~\\
start\_second~~~~~~~~~~~~~=~00,~~~00,~~~\\
end\_year~~~~~~~~~~~~~~~~~=~2007,~2007,~\\
end\_month~~~~~~~~~~~~~~~~=~02,~~~02,~~~\\
end\_day~~~~~~~~~~~~~~~~~~=~05,~~~05,~~~\\
end\_hour~~~~~~~~~~~~~~~~~=~00,~~~00,~~~\\
end\_minute~~~~~~~~~~~~~~~=~00,~~~00,~~~\\
end\_second~~~~~~~~~~~~~~~=~00,~~~00,~~~\\
interval\_seconds~~~~~~~~~=~21600,\\
input\_from\_file~~~~~~~~~~=~.true.,~~~.true.,~\\
history\_interval~~~~~~~~~=~60,~~~~~~~60,~~~~~\\
frames\_per\_outfile~~~~~~~=~1,~~~~~~~~1,~~~~~~\\
restart~~~~~~~~~~~~~~~~~~=~.false.,\\
restart\_interval~~~~~~~~~=~5000,\\
io\_form\_history~~~~~~~~~~=~2,\\
io\_form\_restart~~~~~~~~~~=~2,\\
io\_form\_input~~~~~~~~~~~~=~2,\\
io\_form\_boundary~~~~~~~~~=~2,\\
debug\_level~~~~~~~~~~~~~~=~0,\\
/\\
\\
\&domains~~~~~~~~~~~~~~~~~\\
time\_step~~~~~~~~~~~~~~~~=~60,\\
time\_step\_fract\_num~~~~~~=~0,\\
time\_step\_fract\_den~~~~~~=~1,\\
max\_dom~~~~~~~~~~~~~~~~~~=~2,\\
e\_we~~~~~~~~~~~~~~~~~~~~~=~47,~~~~~~76,~~~~\\
e\_sn~~~~~~~~~~~~~~~~~~~~~=~47,~~~~~~79,~~~~\\
e\_vert~~~~~~~~~~~~~~~~~~~=~35,~~~~~~~35,~~~~\\
p\_top\_requested~~~~~~~~~~=~5000,\\
num\_metgrid\_levels~~~~~~~=~27,\\
num\_metgrid\_soil\_levels~~=~4,\\
dx~~~~~~~~~~~~~~~~~~~~~~~=~18000,~~~6000,\\
dy~~~~~~~~~~~~~~~~~~~~~~~=~18000,~~~6000,\\
grid\_id~~~~~~~~~~~~~~~~~~=~1,~~~~~~~~2,~~~~~~\\
parent\_id~~~~~~~~~~~~~~~~=~1,~~~~~~~~1,~~~~~~\\
i\_parent\_start~~~~~~~~~~~=~1,~~~~~~~10,~~~~~~\\
j\_parent\_start~~~~~~~~~~~=~1,~~~~~~~9,~~~~~~\\
parent\_grid\_ratio~~~~~~~~=~1,~~~~~~~~3,~~~~~~\\
parent\_time\_step\_ratio~~~=~1,~~~~~~~~3,~~~~~~\\
feedback~~~~~~~~~~~~~~~~~=~1,\\
smooth\_option~~~~~~~~~~~~=~0,\\
/\\
\\
\&physics~~~~~~~~~~~~~~~~~\\
mp\_physics~~~~~~~~~~~~~~~=~3,~~3,~\\
ra\_lw\_physics~~~~~~~~~~~~=~1,~~1,~\\
ra\_sw\_physics~~~~~~~~~~~~=~1,~~1,~\\
radt~~~~~~~~~~~~~~~~~~~~~=~30,~30,\\
sf\_sfclay\_physics~~~~~~~~=~1,~~1,~\\
sf\_surface\_physics~~~~~~~=~1,~~1,~\\
bl\_pbl\_physics~~~~~~~~~~~=~1,~~1,~\\
bldt~~~~~~~~~~~~~~~~~~~~~=~0,~~0,~\\
cu\_physics~~~~~~~~~~~~~~~=~1,~~1,~\\
cudt~~~~~~~~~~~~~~~~~~~~~=~5,~~5,~\\
isfflx~~~~~~~~~~~~~~~~~~~=~1,\\
ifsnow~~~~~~~~~~~~~~~~~~~=~0,\\
icloud~~~~~~~~~~~~~~~~~~~=~1,\\
surface\_input\_source~~~~~=~1,\\
num\_soil\_layers~~~~~~~~~~=~4,\\
sf\_urban\_physics~~~~~~~~~=~0,~~\\
maxiens~~~~~~~~~~~~~~~~~~=~1,\\
maxens~~~~~~~~~~~~~~~~~~~=~3,\\
maxens2~~~~~~~~~~~~~~~~~~=~3,\\
maxens3~~~~~~~~~~~~~~~~~~=~16,\\
ensdim~~~~~~~~~~~~~~~~~~~=~144,\\
/\\
\\
\&fdda~~~~~~~~~~~~~~~~~~~~\\
/\\
\\
\&dynamics~~~~~~~~~~~~~~~~\\
w\_damping~~~~~~~~~~~~~~~~=~0,\\
diff\_opt~~~~~~~~~~~~~~~~~=~1,\\
km\_opt~~~~~~~~~~~~~~~~~~~=~4,\\
diff\_6th\_opt~~~~~~~~~~~~~=~0,~~~~~~~~\\
diff\_6th\_factor~~~~~~~~~~=~0.12,~~~~~\\
base\_temp~~~~~~~~~~~~~~~~=~290.,\\
damp\_opt~~~~~~~~~~~~~~~~~=~0,\\
zdamp~~~~~~~~~~~~~~~~~~~~=~5000.,~~~~\\
dampcoef~~~~~~~~~~~~~~~~~=~0.2,~~~~~0.2,~~~~~\\
khdif~~~~~~~~~~~~~~~~~~~~=~0,~~~~~~~0,~~~~~~~\\
kvdif~~~~~~~~~~~~~~~~~~~~=~0,~~~~~~~0,~~~~~~~\\
non\_hydrostatic~~~~~~~~~~=~.true.,~~.true.,~~\\
moist\_adv\_opt~~~~~~~~~~~~=~1,~~~~~~~1,~~~~~~~\\
scalar\_adv\_opt~~~~~~~~~~~=~1,~~~~~~~1,~~~~~~~\\
/\\
\\
\&bdy\_control~~~~~~~~~~~~~\\
spec\_bdy\_width~~~~~~~~~~~=~5,\\
spec\_zone~~~~~~~~~~~~~~~~=~1,\\
relax\_zone~~~~~~~~~~~~~~~=~4,\\
specified~~~~~~~~~~~~~~~~=~.true.,~~.true.,~\\
nested~~~~~~~~~~~~~~~~~~~=~.false.,~.false.,\\
/\\
\\
\&grib2~~~~~~~~~~~~~~~~~~~\\
/\\
\\
\&namelist\_quilt~~~~~~~~~~\\
nio\_tasks\_per\_group~~~~~~=~0,\\
nio\_groups~~~~~~~~~~~~~~~=~1,\\
/\\

%%%%%
% Simulación 9

\textbf{namelist.input Simulación 9}

\&time\_control~~~~~~~~~~~~\\
run\_days~~~~~~~~~~~~~~~~~=~0,\\
run\_hours~~~~~~~~~~~~~~~~=~0,\\
run\_minutes~~~~~~~~~~~~~~=~0,\\
run\_seconds~~~~~~~~~~~~~~=~0,\\
start\_year~~~~~~~~~~~~~~~=~2007,~2007,~2007,\\
start\_month~~~~~~~~~~~~~~=~02,~~~02,~~~02,~~\\
start\_day~~~~~~~~~~~~~~~~=~02,~~~02,~~~02,~~\\
start\_hour~~~~~~~~~~~~~~~=~00,~~~00,~~~00,~~\\
start\_minute~~~~~~~~~~~~~=~00,~~~00,~~~00,~~\\
start\_second~~~~~~~~~~~~~=~00,~~~00,~~~00,~~\\
end\_year~~~~~~~~~~~~~~~~~=~2007,~2007,~2007,\\
end\_month~~~~~~~~~~~~~~~~=~02,~~~02,~~~02,~~\\
end\_day~~~~~~~~~~~~~~~~~~=~05,~~~05,~~~05,~~\\
end\_hour~~~~~~~~~~~~~~~~~=~00,~~~00,~~~00,~~\\
end\_minute~~~~~~~~~~~~~~~=~00,~~~00,~~~00,~~\\
end\_second~~~~~~~~~~~~~~~=~00,~~~00,~~~00,~~\\
interval\_seconds~~~~~~~~~=~21600,\\
input\_from\_file~~~~~~~~~~=~.true.,~~~.true.,~~~.true.,\\
history\_interval~~~~~~~~~=~60,~~~~~~~60,~~~~~~~60,\\
frames\_per\_outfile~~~~~~~=~1,~~~~~~~~1,~~~~~~~~1,\\
restart~~~~~~~~~~~~~~~~~~=~.false.,\\
restart\_interval~~~~~~~~~=~5000,\\
io\_form\_history~~~~~~~~~~=~2,\\
io\_form\_restart~~~~~~~~~~=~2,\\
io\_form\_input~~~~~~~~~~~~=~2,\\
io\_form\_boundary~~~~~~~~~=~2,\\
debug\_level~~~~~~~~~~~~~~=~0,\\
/\\
\\
\&domains~~~~~~~~~~~~~~~~~\\
time\_step~~~~~~~~~~~~~~~~=~20,\\
time\_step\_fract\_num~~~~~~=~0,\\
time\_step\_fract\_den~~~~~~=~1,\\
max\_dom~~~~~~~~~~~~~~~~~~=~3,\\
e\_we~~~~~~~~~~~~~~~~~~~~~=~99,~~~~~~148,~~~~~~190,\\
e\_sn~~~~~~~~~~~~~~~~~~~~~=~99,~~~~~~169,~~~~~~184,\\
e\_vert~~~~~~~~~~~~~~~~~~~=~35,~~~~~~~35,~~~~~~~35,~\\
p\_top\_requested~~~~~~~~~~=~5000,\\
num\_metgrid\_levels~~~~~~~=~27,\\
num\_metgrid\_soil\_levels~~=~4,\\
dx~~~~~~~~~~~~~~~~~~~~~~~=~10000,~~~3333.33334,~~~1111.11112,\\
dy~~~~~~~~~~~~~~~~~~~~~~~=~10000,~~~3333.33334,~~~1111.11112,\\
grid\_id~~~~~~~~~~~~~~~~~~=~1,~~~~~~~~2,~~~~~~~~3,\\
parent\_id~~~~~~~~~~~~~~~~=~1,~~~~~~~~1,~~~~~~~~2,\\
i\_parent\_start~~~~~~~~~~~=~2,~~~~~~~25,~~~~~~~48,\\
j\_parent\_start~~~~~~~~~~~=~1,~~~~~~~26,~~~~~~~58,\\
parent\_grid\_ratio~~~~~~~~=~1,~~~~~~~~3,~~~~~~~~3,\\
parent\_time\_step\_ratio~~~=~1,~~~~~~~~3,~~~~~~~~3,\\
feedback~~~~~~~~~~~~~~~~~=~1,\\
smooth\_option~~~~~~~~~~~~=~0,\\
/\\
\\
\&physics~~~~~~~~~~~~~~~~~\\
mp\_physics~~~~~~~~~~~~~~~=~3,~~3,~3,~\\
ra\_lw\_physics~~~~~~~~~~~~=~1,~~1,~1,~\\
ra\_sw\_physics~~~~~~~~~~~~=~1,~~1,~1,~\\
radt~~~~~~~~~~~~~~~~~~~~~=~30,~30,30,\\
sf\_sfclay\_physics~~~~~~~~=~1,~~1,~1,~\\
sf\_surface\_physics~~~~~~~=~1,~~1,~1,~\\
bl\_pbl\_physics~~~~~~~~~~~=~1,~~1,~1,~\\
bldt~~~~~~~~~~~~~~~~~~~~~=~0,~~0,~0,~\\
cu\_physics~~~~~~~~~~~~~~~=~1,~~1,~1,~\\
cudt~~~~~~~~~~~~~~~~~~~~~=~5,~~5,~5,~\\
isfflx~~~~~~~~~~~~~~~~~~~=~1,\\
ifsnow~~~~~~~~~~~~~~~~~~~=~0,\\
icloud~~~~~~~~~~~~~~~~~~~=~1,\\
surface\_input\_source~~~~~=~1,\\
num\_soil\_layers~~~~~~~~~~=~4,\\
sf\_urban\_physics~~~~~~~~~=~0,~~\\
maxiens~~~~~~~~~~~~~~~~~~=~1,\\
maxens~~~~~~~~~~~~~~~~~~~=~3,\\
maxens2~~~~~~~~~~~~~~~~~~=~3,\\
maxens3~~~~~~~~~~~~~~~~~~=~16,\\
ensdim~~~~~~~~~~~~~~~~~~~=~144,\\
/\\
\\
\&fdda~~~~~~~~~~~~~~~~~~~~\\
/\\
\\
\&dynamics~~~~~~~~~~~~~~~~\\
w\_damping~~~~~~~~~~~~~~~~=~0,\\
diff\_opt~~~~~~~~~~~~~~~~~=~1,\\
km\_opt~~~~~~~~~~~~~~~~~~~=~4,\\
diff\_6th\_opt~~~~~~~~~~~~~=~0,~~~~~~~~\\
diff\_6th\_factor~~~~~~~~~~=~0.12,~~~~~\\
base\_temp~~~~~~~~~~~~~~~~=~290.,\\
damp\_opt~~~~~~~~~~~~~~~~~=~0,\\
zdamp~~~~~~~~~~~~~~~~~~~~=~5000.,~~~~\\
dampcoef~~~~~~~~~~~~~~~~~=~0.2,~~~~~0.2,~~~~~0.2,~~~~~~~~\\
khdif~~~~~~~~~~~~~~~~~~~~=~0,~~~~~~~0,~~~~~~~0,~~~~~~~~~~\\
kvdif~~~~~~~~~~~~~~~~~~~~=~0,~~~~~~~0,~~~~~~~0,~~~~~~~~~~\\
non\_hydrostatic~~~~~~~~~~=~.true.,~~.true.,~~.true.,~~~~~\\
moist\_adv\_opt~~~~~~~~~~~~=~1,~~~~~~~1,~~~~~~~1,~~~~~~~~~~\\
scalar\_adv\_opt~~~~~~~~~~~=~1,~~~~~~~1,~~~~~~~1,~~~~~~~~~~\\
/\\
\\
\&bdy\_control~~~~~~~~~~~~~\\
spec\_bdy\_width~~~~~~~~~~~=~5,\\
spec\_zone~~~~~~~~~~~~~~~~=~1,\\
relax\_zone~~~~~~~~~~~~~~~=~4,\\
specified~~~~~~~~~~~~~~~~=~.true.,~~.true.,~~.true.,~~\\
nested~~~~~~~~~~~~~~~~~~~=~.false.,~.false.,~.false.,~~\\
/\\
\\
\&grib2~~~~~~~~~~~~~~~~~~~\\
/\\
\\
\&namelist\_quilt~~~~~~~~~~\\
nio\_tasks\_per\_group~~~~~~=~0,\\
nio\_groups~~~~~~~~~~~~~~~=~1,\\
/\\
 % Tercero
%}
%%\appendix
\chapter{Resultados de las comparaciones entre los estadísticos y el modelo para la selección de los dominios.}\label{AnexoA}
\label{anexo:resultado_comparaciones_entre_estadisticos}
\begin{landscape}


%\chapter{}
%%%%%%%%%%%%%%%%%%%%%%%%%%
\begin{longtable}{p{5cm}rrrrrrrrrr}
\caption{Resultados de las diferentes simulaciones.}
\label{tab:resultados_dominios}\\
\hline
   Nombre &  Simulación & Dominio &  Pearson &    NRMSE &  $NRMSE_{esc}$ &    $Pearson_{esc}$ &      ET \\
   
\midrule
\endhead
\midrule
\multicolumn{3}{r}{{Continúa en la siguiente página.}} \\
\midrule
\endfoot

\bottomrule
\endlastfoot
     La Capilla Autom  & 1 &     d01 &   0.9640 &  0.7190 &    0.3997 &   0.3772 &    0.6945 \\
     La Capilla Autom  & 1 &     d02 &   0.9539 &  0.3956 &    0.7679 &   0.4520 &    0.7584 \\
     La Capilla Autom  & 1 &     d03 &   0.9501 &  0.2402 &    0.9448 &   0.5991 &    0.8005 \\
     La Capilla Autom  & 2 &     d01 &   0.9645 &  1.0043 &    0.0750 &   0.6056 &    0.5375 \\
     La Capilla Autom  & 2 &     d02 &   0.9546 &  0.6365 &    0.4937 &   0.2750 &    0.6287 \\
     La Capilla Autom  & 2 &     d03 &   0.9526 &  0.6716 &    0.4537 &   0.2860 &    0.5852 \\
     La Capilla Autom  & 3 &     d01 &   0.9475 &  0.2791 &    0.9005 &   0.4904 &    0.7473 \\
     La Capilla Autom  & 3 &     d02 &   0.9508 &  0.3108 &    0.8644 &   0.6158 &    0.7691 \\
     La Capilla Autom  & 4 &     d01 &   0.9565 &  0.6157 &    0.5173 &   0.3864 &    0.6634 \\
     La Capilla Autom  & 4 &     d02 &   0.9490 &  0.2493 &    0.9344 &   0.6221 &    0.7833 \\
     La Capilla Autom  & 5 &     d01 &   0.9640 &  0.9390 &    0.1494 &   0.4003 &    0.5694 \\
     La Capilla Autom  & 5 &     d02 &   0.9562 &  0.7975 &    0.3104 &   0.0000 &    0.5571 \\
     La Capilla Autom  & 6 &     d01 &   0.9612 &  0.4331 &    0.7251 &   0.7929 &    0.8240 \\
     La Capilla Autom  & 6 &     d02 &   0.9591 &  0.5443 &    0.5986 &   0.5394 &    0.7354 \\
     La Capilla Autom  & 7 &     d01 &   0.9579 &  0.3204 &    0.8535 &   0.7523 &    0.8482 \\
     La Capilla Autom  & 7 &     d02 &   0.9616 &  0.5166 &    0.6301 &   0.6652 &    0.7813 \\
     La Capilla Autom  & 8 &     d01 &   0.9644 &  0.8783 &    0.2184 &   0.3671 &    0.6089 \\
     La Capilla Autom  & 8 &     d02 &   0.9484 &  0.3191 &    0.8550 &   0.5070 &    0.7357 \\
     La Capilla Autom  & 8 &     d03 &   0.9461 &  0.2278 &    0.9589 &   0.7035 &    0.7602 \\
     La Capilla Autom  & 9 &     d01 &   0.9548 &  1.0702 &    0.0000 &   0.3691 &    0.3848 \\
     La Capilla Autom  & 9 &     d02 &   0.9257 &  0.2509 &    0.9326 &   0.8475 &    0.5041 \\
     La Capilla Autom  & 9 &     d03 &   0.9225 &  0.1917 &    1.0000 &   1.0000 &    0.5000 \\
 Chinavita Automatica  & 1 &     d01 &   0.9654 &  0.8305 &    0.1481 &   0.0000 &    0.4715 \\
 Chinavita Automatica  & 1 &     d02 &   0.9646 &  0.5980 &    0.5418 &   0.0750 &    0.6566 \\
 Chinavita Automatica  & 1 &     d03 &   0.9549 &  0.3896 &    0.8947 &   0.0610 &    0.7045 \\
 Chinavita Automatica  & 2 &     d01 &   0.9694 &  0.8115 &    0.1802 &   0.3921 &    0.5402 \\
 Chinavita Automatica  & 2 &     d02 &   0.9662 &  0.9180 &    0.0000 &   0.1558 &    0.4071 \\
 Chinavita Automatica  & 2 &     d03 &   0.9507 &  0.4956 &    0.7152 &   0.1133 &    0.5596 \\
 Chinavita Automatica  & 3 &     d01 &   0.9663 &  0.8022 &    0.1961 &   0.1150 &    0.5071 \\
 Chinavita Automatica  & 3 &     d02 &   0.9622 &  0.4598 &    0.7759 &   0.1385 &    0.7428 \\
 Chinavita Automatica  & 4 &     d01 &   0.9683 &  0.6330 &    0.4826 &   0.3824 &    0.6764 \\
 Chinavita Automatica  & 4 &     d02 &   0.9547 &  0.3947 &    0.8860 &   0.0827 &    0.6982 \\
 Chinavita Automatica  & 5 &     d01 &   0.9653 &  0.4724 &    0.7545 &   0.4941 &    0.7734 \\
 Chinavita Automatica  & 5 &     d02 &   0.9644 &  0.5354 &    0.6478 &   0.3578 &    0.7078 \\
 Chinavita Automatica  & 6 &     d01 &   0.9682 &  0.5470 &    0.6281 &   0.4729 &    0.7477 \\
 Chinavita Automatica  & 6 &     d02 &   0.9732 &  0.5014 &    0.7054 &   1.0000 &    0.8527 \\
 Chinavita Automatica  & 7 &     d01 &   0.9690 &  0.7563 &    0.2738 &   0.2117 &    0.5817 \\
 Chinavita Automatica  & 7 &     d02 &   0.9622 &  0.4988 &    0.7097 &   0.3774 &    0.7092 \\
 Chinavita Automatica  & 8 &     d01 &   0.9679 &  0.8120 &    0.1795 &   0.1130 &    0.5199 \\
 Chinavita Automatica  & 8 &     d02 &   0.9515 &  0.3401 &    0.9785 &   0.1666 &    0.7010 \\
 Chinavita Automatica  & 8 &     d03 &   0.9509 &  0.3383 &    0.9817 &   0.1820 &    0.6945 \\
 Chinavita Automatica  & 9 &     d01 &   0.9500 &  0.4243 &    0.8360 &   0.5045 &    0.6108 \\
 Chinavita Automatica  & 9 &     d02 &   0.9373 &  0.3747 &    0.9199 &   0.3887 &    0.4840 \\
 Chinavita Automatica  & 9 &     d03 &   0.9355 &  0.3274 &    1.0000 &   0.5191 &    0.5000 \\
 La Boyera Automatica  & 1 &     d01 &   0.9621 &  0.2949 &    0.2990 &   0.0606 &    0.5033 \\
 La Boyera Automatica  & 1 &     d02 &   0.9303 &  0.3220 &    0.1749 &   0.0000 &    0.0894 \\
 La Boyera Automatica  & 1 &     d03 &   0.9301 &  0.2703 &    0.4115 &   0.1970 &    0.2057 \\
 La Boyera Automatica  & 2 &     d01 &   0.9694 &  0.3254 &    0.1594 &   0.1380 &    0.5137 \\
 La Boyera Automatica  & 2 &     d02 &   0.9631 &  0.3602 &    0.0000 &   0.4973 &    0.3647 \\
 La Boyera Automatica  & 2 &     d03 &   0.9753 &  0.2621 &    0.4487 &   0.5902 &    0.7244 \\
 La Boyera Automatica  & 3 &     d01 &   0.9681 &  0.2249 &    0.6191 &   0.2460 &    0.7295 \\
 La Boyera Automatica  & 3 &     d02 &   0.9720 &  0.2075 &    0.6987 &   0.5976 &    0.8129 \\
 La Boyera Automatica  & 4 &     d01 &   0.9695 &  0.2576 &    0.4697 &   0.5225 &    0.6703 \\
 La Boyera Automatica  & 4 &     d02 &   0.9697 &  0.1555 &    0.9367 &   0.6904 &    0.9059 \\
 La Boyera Automatica  & 5 &     d01 &   0.9633 &  0.1478 &    0.9722 &   0.9092 &    0.8535 \\
 La Boyera Automatica  & 5 &     d02 &   0.9652 &  0.1615 &    0.9091 &   0.5631 &    0.8421 \\
 La Boyera Automatica  & 6 &     d01 &   0.9629 &  0.3348 &    0.1160 &   0.4900 &    0.4208 \\
 La Boyera Automatica  & 6 &     d02 &   0.9624 &  0.1583 &    0.9240 &   0.5915 &    0.8190 \\
 La Boyera Automatica  & 7 &     d01 &   0.9640 &  0.1417 &    1.0000 &   1.0000 &    0.8747 \\
 La Boyera Automatica  & 7 &     d02 &   0.9616 &  0.1688 &    0.8757 &   0.6502 &    0.7864 \\
 La Boyera Automatica  & 8 &     d01 &   0.9657 &  0.3554 &    0.0218 &   0.4596 &    0.4045 \\
 La Boyera Automatica  & 8 &     d02 &   0.9705 &  0.1807 &    0.8213 &   0.6129 &    0.8574 \\
 La Boyera Automatica  & 8 &     d03 &   0.9639 &  0.2053 &    0.7088 &   0.1568 &    0.7285 \\
 La Boyera Automatica  & 9 &     d01 &   0.9705 &  0.1469 &    0.9764 &   0.7310 &    0.9352 \\
 La Boyera Automatica  & 9 &     d02 &   0.9706 &  0.2068 &    0.7018 &   0.5605 &    0.7990 \\
 La Boyera Automatica  & 9 &     d03 &   0.9633 &  0.1861 &    0.7968 &   0.2580 &    0.7652 \\
 Tibaitata Automatica  & 1 &     d01 &   0.9502 &  0.2721 &    0.7131 &   0.5220 &    0.7133 \\
 Tibaitata Automatica  & 1 &     d02 &   0.9459 &  0.2530 &    0.7786 &   0.6363 &    0.6808 \\
 Tibaitata Automatica  & 1 &     d03 &   0.9424 &  0.3045 &    0.6014 &   0.4095 &    0.5394 \\
 Tibaitata Automatica  & 2 &     d01 &   0.9544 &  0.4789 &    0.0000 &   0.0000 &    0.4209 \\
 Tibaitata Automatica  & 2 &     d02 &   0.9587 &  0.2069 &    0.9377 &   0.7646 &    0.9548 \\
 Tibaitata Automatica  & 2 &     d03 &   0.9596 &  0.1899 &    0.9964 &   0.8591 &    0.9982 \\
 Tibaitata Automatica  & 3 &     d01 &   0.9550 &  0.2091 &    0.9302 &   0.8180 &    0.8960 \\
 Tibaitata Automatica  & 3 &     d02 &   0.9508 &  0.2054 &    0.9429 &   0.8635 &    0.8387 \\
 Tibaitata Automatica  & 4 &     d01 &   0.9515 &  0.2467 &    0.8006 &   0.6000 &    0.7778 \\
 Tibaitata Automatica  & 4 &     d02 &   0.9514 &  0.2006 &    0.9595 &   0.8837 &    0.8559 \\
 Tibaitata Automatica  & 5 &     d01 &   0.9484 &  0.2740 &    0.7065 &   0.5042 &    0.6832 \\
 Tibaitata Automatica  & 5 &     d02 &   0.9489 &  0.2050 &    0.9441 &   0.8585 &    0.8091 \\
 Tibaitata Automatica  & 6 &     d01 &   0.9536 &  0.2182 &    0.8986 &   0.7497 &    0.8582 \\
 Tibaitata Automatica  & 6 &     d02 &   0.9515 &  0.2047 &    0.9454 &   0.8428 &    0.8506 \\
 Tibaitata Automatica  & 7 &     d01 &   0.9530 &  0.3166 &    0.5595 &   0.3396 &    0.6793 \\
 Tibaitata Automatica  & 7 &     d02 &   0.9526 &  0.2066 &    0.9387 &   0.8466 &    0.8637 \\
 Tibaitata Automatica  & 8 &     d01 &   0.9467 &  0.2274 &    0.8671 &   0.6815 &    0.7380 \\
 Tibaitata Automatica  & 8 &     d02 &   0.9481 &  0.2141 &    0.9127 &   0.7520 &    0.7822 \\
 Tibaitata Automatica  & 8 &     d03 &   0.9463 &  0.1979 &    0.9686 &   0.8770 &    0.7821 \\
 Tibaitata Automatica  & 9 &     d01 &   0.9316 &  0.2033 &    0.9500 &   0.8967 &    0.5499 \\
 Tibaitata Automatica  & 9 &     d02 &   0.9266 &  0.2205 &    0.8908 &   0.7708 &    0.4454 \\
 Tibaitata Automatica  & 9 &     d03 &   0.9316 &  0.1888 &    1.0000 &   1.0000 &    0.5754 \\
   Sta Cruz De Siecha  & 1 &     d01 &   0.9226 &  0.4175 &    0.4000 &   0.4626 &    0.3206 \\
   Sta Cruz De Siecha  & 1 &     d02 &   0.9419 &  0.1181 &    0.9846 &   0.9557 &    0.8165 \\
   Sta Cruz De Siecha  & 1 &     d03 &   0.9421 &  0.1427 &    0.9366 &   0.4353 &    0.7941 \\
   Sta Cruz De Siecha  & 2 &     d01 &   0.9505 &  0.1873 &    0.8497 &   0.9572 &    0.8391 \\
   Sta Cruz De Siecha  & 2 &     d02 &   0.9586 &  0.2980 &    0.6335 &   0.4460 &    0.8167 \\
   Sta Cruz De Siecha  & 2 &     d03 &   0.9416 &  0.2701 &    0.6878 &   0.3132 &    0.6648 \\
   Sta Cruz De Siecha  & 3 &     d01 &   0.9406 &  0.1447 &    0.9328 &   0.9749 &    0.7763 \\
   Sta Cruz De Siecha  & 3 &     d02 &   0.9266 &  0.3907 &    0.4524 &   0.0000 &    0.3886 \\
   Sta Cruz De Siecha  & 4 &     d01 &   0.9345 &  0.3849 &    0.4637 &   0.4796 &    0.4775 \\
   Sta Cruz De Siecha  & 4 &     d02 &   0.9170 &  0.1587 &    0.9054 &   0.5520 &    0.5145 \\
   Sta Cruz De Siecha  & 5 &     d01 &   0.9478 &  0.1251 &    0.9710 &   0.3196 &    0.8713 \\
   Sta Cruz De Siecha  & 5 &     d02 &   0.9183 &  0.1872 &    0.8497 &   0.8399 &    0.5004 \\
   Sta Cruz De Siecha  & 6 &     d01 &   0.9478 &  0.1482 &    0.9260 &   0.8436 &    0.8493 \\
   Sta Cruz De Siecha  & 6 &     d02 &   0.9412 &  0.1377 &    0.9464 &   0.9459 &    0.7898 \\
   Sta Cruz De Siecha  & 7 &     d01 &   0.9467 &  0.1103 &    1.0000 &   0.4715 &    0.8743 \\
   Sta Cruz De Siecha  & 7 &     d02 &   0.9287 &  0.2112 &    0.8028 &   0.9041 &    0.5866 \\
   Sta Cruz De Siecha  & 8 &     d01 &   0.9414 &  0.1564 &    0.9100 &   1.0000 &    0.7732 \\
   Sta Cruz De Siecha  & 8 &     d02 &   0.9411 &  0.2816 &    0.6654 &   0.9641 &    0.6479 \\
   Sta Cruz De Siecha  & 8 &     d03 &   0.9368 &  0.1748 &    0.8739 &   0.2315 &    0.7071 \\
   Sta Cruz De Siecha  & 9 &     d01 &   0.9165 &  0.6224 &    0.0000 &   0.5510 &    0.0557 \\
   Sta Cruz De Siecha  & 9 &     d02 &   0.9206 &  0.3464 &    0.5389 &   0.5542 &    0.3687 \\
   Sta Cruz De Siecha  & 9 &     d03 &   0.9112 &  0.2485 &    0.7301 &   0.5410 &    0.3650 \\
      Pmo Guacheneque  & 1 &     d01 &   0.9223 &  0.3397 &    0.4779 &   0.5488 &    0.6637 \\
      Pmo Guacheneque  & 1 &     d02 &   0.9118 &  0.1324 &    1.0000 &   0.8976 &    0.7988 \\
      Pmo Guacheneque  & 1 &     d03 &   0.8871 &  0.2605 &    0.6774 &   0.6202 &    0.3387 \\
      Pmo Guacheneque  & 2 &     d01 &   0.9229 &  0.5196 &    0.0247 &   0.5785 &    0.4448 \\
      Pmo Guacheneque  & 2 &     d02 &   0.9184 &  0.1593 &    0.9322 &   0.5224 &    0.8446 \\
      Pmo Guacheneque  & 2 &     d03 &   0.8982 &  0.1694 &    0.9068 &   1.0000 &    0.5875 \\
      Pmo Guacheneque  & 3 &     d01 &   0.9285 &  0.3276 &    0.5082 &   0.1791 &    0.7541 \\
      Pmo Guacheneque  & 3 &     d02 &   0.9047 &  0.1576 &    0.9364 &   0.9972 &    0.6812 \\
      Pmo Guacheneque  & 4 &     d01 &   0.9188 &  0.1467 &    0.9639 &   0.5583 &    0.8646 \\
      Pmo Guacheneque  & 4 &     d02 &   0.9144 &  0.1348 &    0.9940 &   0.7773 &    0.8271 \\
      Pmo Guacheneque  & 5 &     d01 &   0.9215 &  0.2911 &    0.6002 &   0.2713 &    0.7153 \\
      Pmo Guacheneque  & 5 &     d02 &   0.9101 &  0.1528 &    0.9485 &   0.4667 &    0.7525 \\
      Pmo Guacheneque  & 6 &     d01 &   0.9240 &  0.3300 &    0.5022 &   0.5178 &    0.6970 \\
      Pmo Guacheneque  & 6 &     d02 &   0.9151 &  0.1333 &    0.9977 &   0.4565 &    0.8368 \\
      Pmo Guacheneque  & 7 &     d01 &   0.9281 &  0.5294 &    0.0000 &   0.4119 &    0.4946 \\
      Pmo Guacheneque  & 7 &     d02 &   0.9168 &  0.1972 &    0.8367 &   0.0000 &    0.7768 \\
      Pmo Guacheneque  & 8 &     d01 &   0.9198 &  0.1867 &    0.8632 &   0.2633 &    0.8266 \\
      Pmo Guacheneque  & 8 &     d02 &   0.9070 &  0.1658 &    0.9159 &   0.8085 &    0.6978 \\
      Pmo Guacheneque  & 8 &     d03 &   0.8962 &  0.3110 &    0.5500 &   0.9625 &    0.3844 \\
      Pmo Guacheneque  & 9 &     d01 &   0.9225 &  0.1370 &    0.9884 &   0.3118 &    0.9218 \\
      Pmo Guacheneque  & 9 &     d02 &   0.9090 &  0.1854 &    0.8666 &   0.6153 &    0.6971 \\
      Pmo Guacheneque  & 9 &     d03 &   0.9003 &  0.2804 &    0.6272 &   0.9016 &    0.4723 \\
       Ciudad Bolivar  & 1 &     d01 &   0.9518 &  0.1590 &    0.8211 &   0.6806 &    0.7960 \\
       Ciudad Bolivar  & 1 &     d02 &   0.9578 &  0.1150 &    0.9666 &   0.7544 &    0.9479 \\
       Ciudad Bolivar  & 1 &     d03 &   0.9345 &  0.1724 &    0.7769 &   0.7063 &    0.5451 \\
       Ciudad Bolivar  & 2 &     d01 &   0.9604 &  0.3080 &    0.3293 &   0.0978 &    0.6647 \\
       Ciudad Bolivar  & 2 &     d02 &   0.9599 &  0.1316 &    0.9117 &   0.9320 &    0.9489 \\
       Ciudad Bolivar  & 2 &     d03 &   0.9527 &  0.1307 &    0.9147 &   0.8680 &    0.8547 \\
       Ciudad Bolivar  & 3 &     d01 &   0.9515 &  0.1699 &    0.7852 &   0.1864 &    0.7748 \\
       Ciudad Bolivar  & 3 &     d02 &   0.9503 &  0.1141 &    0.9695 &   0.7402 &    0.8499 \\
       Ciudad Bolivar  & 4 &     d01 &   0.9423 &  0.1208 &    0.9472 &   1.0000 &    0.7336 \\
       Ciudad Bolivar  & 4 &     d02 &   0.9227 &  0.1330 &    0.9072 &   0.9993 &    0.4536 \\
       Ciudad Bolivar  & 5 &     d01 &   0.9477 &  0.1539 &    0.8381 &   0.7723 &    0.7499 \\
       Ciudad Bolivar  & 5 &     d02 &   0.9296 &  0.1428 &    0.8748 &   0.6481 &    0.5292 \\
       Ciudad Bolivar  & 6 &     d01 &   0.9535 &  0.3545 &    0.1761 &   0.1590 &    0.4955 \\
       Ciudad Bolivar  & 6 &     d02 &   0.9399 &  0.1366 &    0.8951 &   0.4133 &    0.6751 \\
       Ciudad Bolivar  & 7 &     d01 &   0.9584 &  0.4078 &    0.0000 &   0.0000 &    0.4729 \\
       Ciudad Bolivar  & 7 &     d02 &   0.9562 &  0.1048 &    1.0000 &   0.5259 &    0.9443 \\
       Ciudad Bolivar  & 8 &     d01 &   0.9553 &  0.1177 &    0.9574 &   0.7887 &    0.9101 \\
       Ciudad Bolivar  & 8 &     d02 &   0.9506 &  0.2002 &    0.6852 &   0.4818 &    0.7122 \\
       Ciudad Bolivar  & 8 &     d03 &   0.9537 &  0.1558 &    0.8317 &   0.9287 &    0.8264 \\
       Ciudad Bolivar  & 9 &     d01 &   0.9421 &  0.1551 &    0.8342 &   0.5230 &    0.6748 \\
       Ciudad Bolivar  & 9 &     d02 &   0.9455 &  0.1412 &    0.8801 &   0.1797 &    0.7420 \\
       Ciudad Bolivar  & 9 &     d03 &   0.9392 &  0.1790 &    0.7554 &   0.7273 &    0.5965 \\
         Pmo Guerrero  & 1 &     d01 &   0.9517 &  0.3508 &    0.7139 &   0.5945 &    0.7654 \\
         Pmo Guerrero  & 1 &     d02 &   0.9232 &  0.2987 &    0.7972 &   0.0382 &    0.3986 \\
         Pmo Guerrero  & 1 &     d03 &   0.9306 &  0.2727 &    0.8389 &   0.1750 &    0.5256 \\
         Pmo Guerrero  & 2 &     d01 &   0.9533 &  0.7970 &    0.0000 &   0.2680 &    0.4312 \\
         Pmo Guerrero  & 2 &     d02 &   0.9504 &  0.3870 &    0.6559 &   0.1793 &    0.7185 \\
         Pmo Guerrero  & 2 &     d03 &   0.9401 &  0.3217 &    0.7604 &   0.1607 &    0.6220 \\
         Pmo Guerrero  & 3 &     d01 &   0.9514 &  0.2264 &    0.9129 &   0.4476 &    0.8613 \\
         Pmo Guerrero  & 3 &     d02 &   0.9487 &  0.2326 &    0.9029 &   0.6903 &    0.8172 \\
         Pmo Guerrero  & 4 &     d01 &   0.9453 &  0.2928 &    0.8067 &   0.2151 &    0.7200 \\
         Pmo Guerrero  & 4 &     d02 &   0.9432 &  0.3589 &    0.7009 &   0.1223 &    0.6367 \\
         Pmo Guerrero  & 5 &     d01 &   0.9553 &  0.1832 &    0.9819 &   0.7863 &    0.9514 \\
         Pmo Guerrero  & 5 &     d02 &   0.9424 &  0.1857 &    0.9780 &   0.7946 &    0.7643 \\
         Pmo Guerrero  & 6 &     d01 &   0.9497 &  0.2489 &    0.8769 &   1.0000 &    0.8178 \\
         Pmo Guerrero  & 6 &     d02 &   0.9486 &  0.2203 &    0.9227 &   0.7034 &    0.8250 \\
         Pmo Guerrero  & 7 &     d01 &   0.9581 &  0.2585 &    0.8616 &   0.8136 &    0.9308 \\
         Pmo Guerrero  & 7 &     d02 &   0.9536 &  0.1720 &    1.0000 &   0.7989 &    0.9365 \\
         Pmo Guerrero  & 8 &     d01 &   0.9551 &  0.2123 &    0.9355 &   0.5764 &    0.9259 \\
         Pmo Guerrero  & 8 &     d02 &   0.9428 &  0.4480 &    0.5583 &   0.0000 &    0.5598 \\
         Pmo Guerrero  & 8 &     d03 &   0.9320 &  0.3194 &    0.7641 &   0.1099 &    0.5082 \\
         Pmo Guerrero  & 9 &     d01 &   0.9452 &  0.2123 &    0.9355 &   0.6486 &    0.7834 \\
         Pmo Guerrero  & 9 &     d02 &   0.9422 &  0.3018 &    0.7922 &   0.5239 &    0.6682 \\
         Pmo Guerrero  & 9 &     d03 &   0.9353 &  0.2640 &    0.8528 &   0.4505 &    0.6000 \\
    Hda Sta Ana Autom  & 1 &     d01 &   0.9530 &  0.4222 &    0.0000 &   0.0922 &    0.1936 \\
    Hda Sta Ana Autom  & 1 &     d02 &   0.9522 &  0.4155 &    0.0533 &   0.0980 &    0.1964 \\
    Hda Sta Ana Autom  & 1 &     d03 &   0.9529 &  0.3645 &    0.4571 &   0.5582 &    0.4217 \\
    Hda Sta Ana Autom  & 2 &     d01 &   0.9508 &  0.4173 &    0.0389 &   0.3329 &    0.1401 \\
    Hda Sta Ana Autom  & 2 &     d02 &   0.9545 &  0.3898 &    0.2563 &   0.2432 &    0.3740 \\
    Hda Sta Ana Autom  & 2 &     d03 &   0.9524 &  0.3138 &    0.8586 &   0.7335 &    0.6049 \\
    Hda Sta Ana Autom  & 3 &     d01 &   0.9517 &  0.3466 &    0.5985 &   0.5672 &    0.4519 \\
    Hda Sta Ana Autom  & 3 &     d02 &   0.9519 &  0.3262 &    0.7600 &   0.9035 &    0.5385 \\
    Hda Sta Ana Autom  & 4 &     d01 &   0.9495 &  0.3955 &    0.2114 &   0.1539 &    0.1849 \\
    Hda Sta Ana Autom  & 4 &     d02 &   0.9487 &  0.3312 &    0.7207 &   0.8141 &    0.4127 \\
    Hda Sta Ana Autom  & 5 &     d01 &   0.9521 &  0.4113 &    0.0861 &   0.0410 &    0.2077 \\
    Hda Sta Ana Autom  & 5 &     d02 &   0.9501 &  0.3994 &    0.1803 &   0.2253 &    0.1863 \\
    Hda Sta Ana Autom  & 6 &     d01 &   0.9508 &  0.3440 &    0.6191 &   0.6501 &    0.4310 \\
    Hda Sta Ana Autom  & 6 &     d02 &   0.9472 &  0.3349 &    0.6912 &   0.7042 &    0.3456 \\
    Hda Sta Ana Autom  & 7 &     d01 &   0.9540 &  0.3605 &    0.4885 &   0.4435 &    0.4735 \\
    Hda Sta Ana Autom  & 7 &     d02 &   0.9494 &  0.3107 &    0.8826 &   0.9321 &    0.5167 \\
    Hda Sta Ana Autom  & 8 &     d01 &   0.9621 &  0.4009 &    0.1688 &   0.1520 &    0.5844 \\
    Hda Sta Ana Autom  & 8 &     d02 &   0.9570 &  0.3538 &    0.5420 &   0.6371 &    0.6011 \\
    Hda Sta Ana Autom  & 8 &     d03 &   0.9591 &  0.3355 &    0.6864 &   0.8548 &    0.7413 \\
    Hda Sta Ana Autom  & 9 &     d01 &   0.9593 &  0.4116 &    0.0841 &   0.0000 &    0.4483 \\
    Hda Sta Ana Autom  & 9 &     d02 &   0.9507 &  0.3135 &    0.8606 &   0.9306 &    0.5486 \\
    Hda Sta Ana Autom  & 9 &     d03 &   0.9513 &  0.2959 &    1.0000 &   1.0000 &    0.6374 \\
     Subia Automatica  & 1 &     d01 &   0.9205 &  0.5757 &    0.0000 &   0.6702 &    0.4763 \\
     Subia Automatica  & 1 &     d02 &   0.9226 &  0.2871 &    0.6956 &   0.4800 &    0.8464 \\
     Subia Automatica  & 1 &     d03 &   0.9131 &  0.2673 &    0.7432 &   0.5014 &    0.7706 \\
     Subia Automatica  & 2 &     d01 &   0.9227 &  0.2646 &    0.7496 &   0.7592 &    0.8748 \\
     Subia Automatica  & 2 &     d02 &   0.9144 &  0.2028 &    0.8986 &   0.2660 &    0.8625 \\
     Subia Automatica  & 2 &     d03 &   0.9111 &  0.2903 &    0.6878 &   0.2054 &    0.7228 \\
     Subia Automatica  & 3 &     d01 &   0.9013 &  0.1976 &    0.9111 &   0.1984 &    0.7321 \\
     Subia Automatica  & 3 &     d02 &   0.8749 &  0.2577 &    0.7664 &   0.0818 &    0.3832 \\
     Subia Automatica  & 4 &     d01 &   0.8970 &  0.2705 &    0.7354 &   0.0000 &    0.5994 \\
     Subia Automatica  & 4 &     d02 &   0.9126 &  0.2204 &    0.8561 &   0.6783 &    0.8223 \\
     Subia Automatica  & 5 &     d01 &   0.9207 &  0.4360 &    0.3367 &   0.5884 &    0.6470 \\
     Subia Automatica  & 5 &     d02 &   0.9217 &  0.2155 &    0.8680 &   0.5303 &    0.9233 \\
     Subia Automatica  & 6 &     d01 &   0.8892 &  0.4769 &    0.2381 &   0.0451 &    0.2684 \\
     Subia Automatica  & 6 &     d02 &   0.9186 &  0.2685 &    0.7402 &   0.4723 &    0.8270 \\
     Subia Automatica  & 7 &     d01 &   0.8892 &  0.5497 &    0.0626 &   0.0913 &    0.1813 \\
     Subia Automatica  & 7 &     d02 &   0.9163 &  0.4296 &    0.3520 &   0.4515 &    0.6094 \\
     Subia Automatica  & 8 &     d01 &   0.8904 &  0.2740 &    0.7271 &   0.0754 &    0.5252 \\
     Subia Automatica  & 8 &     d02 &   0.9190 &  0.3070 &    0.6476 &   0.5255 &    0.7854 \\
     Subia Automatica  & 8 &     d03 &   0.9176 &  0.1607 &    1.0000 &   0.7802 &    0.9470 \\
     Subia Automatica  & 9 &     d01 &   0.9086 &  0.3916 &    0.4437 &   0.5459 &    0.5746 \\
     Subia Automatica  & 9 &     d02 &   0.8953 &  0.1807 &    0.9519 &   0.7369 &    0.6899 \\
     Subia Automatica  & 9 &     d03 &   0.8951 &  0.1838 &    0.9444 &   1.0000 &    0.6836 \\ 
 
\end{longtable}
\end{landscape}
 % Tercero
%%\appendix
%\clearpage
%\addappheadtotoc
%\appendixpage

\chapter{Diagramas de Taylor divididas de a grupos de a 5.}
\label{anexo:diagramas_taylor_div_estaciones}

Se crearon niveles siendo el nivel 0 el nivel con las mejores simulaciones.



\begin{figure}[H]
    \centering
    
  
\begin{subfigure}[normla]{0.4\textwidth}
\includegraphics[draft=false, scale=0.3]{graficas_taylor_dom_div/taylor_0_21206940.png}
\caption{Estación CIUDAD BOLIVAR  código 21206940 nivel 0.}
\end{subfigure}
~
\begin{subfigure}[normla]{0.4\textwidth}
\includegraphics[draft=false, scale=0.3]{graficas_taylor_dom_div/taylor_1_21206940.png}
\caption{Estación CIUDAD BOLIVAR  código 21206940 nivel 1.}
\end{subfigure}

\begin{subfigure}[normla]{0.4\textwidth}
\includegraphics[draft=false, scale=0.3]{graficas_taylor_dom_div/taylor_2_21206940.png}
\caption{Estación CIUDAD BOLIVAR  código 21206940 nivel 2.}
\end{subfigure}
~
\begin{subfigure}[normla]{0.4\textwidth}
\includegraphics[draft=false, scale=0.3]{graficas_taylor_dom_div/taylor_3_21206940.png}
\caption{Estación CIUDAD BOLIVAR  código 21206940 nivel 3.}
\end{subfigure}

\end{figure}
           
\begin{figure}[H]\ContinuedFloat
\begin{subfigure}[normla]{0.4\textwidth}
\includegraphics[draft=false, scale=0.3]{graficas_taylor_dom_div/taylor_4_21206940.png}
\caption{Estación CIUDAD BOLIVAR  código 21206940 nivel 4.}
\end{subfigure}
~
\begin{subfigure}[normla]{0.4\textwidth}
\includegraphics[draft=false, scale=0.3]{graficas_taylor_dom_div/taylor_0_21206950.png}
\caption{Estación PMO GUACHENEQUE  código 21206950 nivel 0.}
\end{subfigure}

\begin{subfigure}[normla]{0.4\textwidth}
\includegraphics[draft=false, scale=0.3]{graficas_taylor_dom_div/taylor_1_21206950.png}
\caption{Estación PMO GUACHENEQUE  código 21206950 nivel 1.}
\end{subfigure}
~
\begin{subfigure}[normla]{0.4\textwidth}
\includegraphics[draft=false, scale=0.3]{graficas_taylor_dom_div/taylor_2_21206950.png}
\caption{Estación PMO GUACHENEQUE  código 21206950 nivel 2.}
\end{subfigure}

\begin{subfigure}[normla]{0.4\textwidth}
\includegraphics[draft=false, scale=0.3]{graficas_taylor_dom_div/taylor_3_21206950.png}
\caption{Estación PMO GUACHENEQUE  código 21206950 nivel 3.}
\end{subfigure}
~
\begin{subfigure}[normla]{0.4\textwidth}
\includegraphics[draft=false, scale=0.3]{graficas_taylor_dom_div/taylor_4_21206950.png}
\caption{Estación PMO GUACHENEQUE  código 21206950 nivel 4.}
\end{subfigure}
~
\end{figure}
           
\begin{figure}[H]\ContinuedFloat
\centering
\begin{subfigure}[normla]{0.4\textwidth}
\includegraphics[draft=false, scale=0.3]{graficas_taylor_dom_div/taylor_0_21206980.png}
\caption{Estación STA CRUZ DE SIECHA  código 21206980 nivel 0.}
\end{subfigure}
~
\begin{subfigure}[normla]{0.4\textwidth}
\includegraphics[draft=false, scale=0.3]{graficas_taylor_dom_div/taylor_1_21206980.png}
\caption{Estación STA CRUZ DE SIECHA  código 21206980 nivel 1.}
\end{subfigure}
~
\begin{subfigure}[normla]{0.4\textwidth}
\includegraphics[draft=false, scale=0.3]{graficas_taylor_dom_div/taylor_2_21206980.png}
\caption{Estación STA CRUZ DE SIECHA  código 21206980 nivel 2.}
\end{subfigure}
~
\begin{subfigure}[normla]{0.4\textwidth}
\includegraphics[draft=false, scale=0.3]{graficas_taylor_dom_div/taylor_3_21206980.png}
\caption{Estación STA CRUZ DE SIECHA  código 21206980 nivel 3.}
\end{subfigure}
~
\begin{subfigure}[normla]{0.4\textwidth}
\includegraphics[draft=false, scale=0.3]{graficas_taylor_dom_div/taylor_4_21206980.png}
\caption{Estación STA CRUZ DE SIECHA  código 21206980 nivel 4.}
\end{subfigure}
~


    \caption{Estación X.}
    \label{fig:taylor_dom_dea5}
\end{figure} % Tercero
%%\appendix
%\clearpage
%\addappheadtotoc
%\appendixpage

\chapter{Correcciones de altura para cada una de las estaciones, simulaciones y dominios.}
\label{anexo:correccion_altura}

\begin{landscape}

\begin{longtable}{lrlp{2cm}p{2cm}p{3cm}p{2cm}r}

\caption{Correcciones del las salidas del modelo para cada una de las estaciones, simulaciones y dominio.}
\label{tab:correccion_alturas}\\

\hline
                  Nombre &       Código &         Simulación & Dominio &  Altura IDEAM &  Altura ALOS-PALSAR &      Altura modelo WRF &  Corrección \\
\midrule
\endhead
\midrule
\multicolumn{3}{r}{{Continúa en la siguiente página.}} \\
\midrule
\endfoot

\bottomrule
\endlastfoot
ESC LA UNION AUTOMATICA &  21201200 &   Simulación 4 &   d01 &      3320 &     3393 &  3541.293 &    -0.964 \\
 ESC LA UNION AUTOMATICA &  21201200 &   Simulación 4 &   d02 &      3320 &     3393 &  3424.449 &    -0.204 \\
 ESC LA UNION AUTOMATICA &  21201200 &   Simulación 1 &   d01 &      3320 &     3393 &  3294.583 &     0.640 \\
 ESC LA UNION AUTOMATICA &  21201200 &   Simulación 1 &   d02 &      3320 &     3393 &  3541.293 &    -0.964 \\
 ESC LA UNION AUTOMATICA &  21201200 &   Simulación 1 &   d03 &      3320 &     3393 &  3424.449 &    -0.204 \\
 ESC LA UNION AUTOMATICA &  21201200 &   Simulación 2 &   d01 &      3320 &     3393 &  2794.138 &     3.893 \\
 ESC LA UNION AUTOMATICA &  21201200 &   Simulación 2 &   d02 &      3320 &     3393 &  3353.045 &     0.260 \\
 ESC LA UNION AUTOMATICA &  21201200 &   Simulación 2 &   d03 &      3320 &     3393 &  3504.820 &    -0.727 \\
 ESC LA UNION AUTOMATICA &  21201200 &   Simulación 3 &   d01 &      3320 &     3393 &  3427.714 &    -0.226 \\
 ESC LA UNION AUTOMATICA &  21201200 &   Simulación 5 &   d01 &      3320 &     3393 &  3289.564 &     0.672 \\
 ESC LA UNION AUTOMATICA &  21201200 &   Simulación 5 &   d02 &      3320 &     3393 &  3507.704 &    -0.746 \\
 ESC LA UNION AUTOMATICA &  21201200 &   Simulación 6 &   d01 &      3320 &     3393 &  3521.569 &    -0.836 \\
 ESC LA UNION AUTOMATICA &  21201200 &   Simulación 6 &   d02 &      3320 &     3393 &  3373.582 &     0.126 \\
 ESC LA UNION AUTOMATICA &  21201200 &  Simulación 10 &   d01 &      3320 &     3393 &  3310.201 &     0.538 \\
 ESC LA UNION AUTOMATICA &  21201200 &  Simulación 10 &   d02 &      3320 &     3393 &  3539.941 &    -0.955 \\
 ESC LA UNION AUTOMATICA &  21201200 &  Simulación 10 &   d03 &      3320 &     3393 &  3373.028 &     0.130 \\
 ESC LA UNION AUTOMATICA &  21201200 &   Simulación 7 &   d01 &      3320 &     3393 &  3145.767 &     1.607 \\
 ESC LA UNION AUTOMATICA &  21201200 &   Simulación 7 &   d02 &      3320 &     3393 &  3499.643 &    -0.693 \\
 ESC LA UNION AUTOMATICA &  21201200 &   Simulación 8 &   d01 &      3320 &     3393 &  3208.613 &     1.199 \\
 ESC LA UNION AUTOMATICA &  21201200 &   Simulación 8 &   d02 &      3320 &     3393 &  3516.516 &    -0.803 \\
 ESC LA UNION AUTOMATICA &  21201200 &   Simulación 9 &   d01 &      3320 &     3393 &  2826.282 &     3.684 \\
 ESC LA UNION AUTOMATICA &  21201200 &   Simulación 9 &   d02 &      3320 &     3393 &  3496.001 &    -0.670 \\
    PASQUILLA AUTOMATICA &  21201580 &   Simulación 4 &   d01 &      3000 &     2982 &  3198.655 &    -1.408 \\
    PASQUILLA AUTOMATICA &  21201580 &   Simulación 4 &   d02 &      3000 &     2982 &  3136.289 &    -1.003 \\
    PASQUILLA AUTOMATICA &  21201580 &   Simulación 1 &   d01 &      3000 &     2982 &  2965.059 &     0.110 \\
    PASQUILLA AUTOMATICA &  21201580 &   Simulación 1 &   d02 &      3000 &     2982 &  3198.655 &    -1.408 \\
    PASQUILLA AUTOMATICA &  21201580 &   Simulación 1 &   d03 &      3000 &     2982 &  3136.303 &    -1.003 \\
    PASQUILLA AUTOMATICA &  21201580 &   Simulación 2 &   d01 &      3000 &     2982 &  2779.850 &     1.314 \\
    PASQUILLA AUTOMATICA &  21201580 &   Simulación 2 &   d02 &      3000 &     2982 &  3035.870 &    -0.350 \\
    PASQUILLA AUTOMATICA &  21201580 &   Simulación 2 &   d03 &      3000 &     2982 &  3300.655 &    -2.071 \\
    PASQUILLA AUTOMATICA &  21201580 &   Simulación 3 &   d01 &      3000 &     2982 &  3023.201 &    -0.268 \\
    PASQUILLA AUTOMATICA &  21201580 &   Simulación 5 &   d01 &      3000 &     2982 &  3063.388 &    -0.529 \\
    PASQUILLA AUTOMATICA &  21201580 &   Simulación 5 &   d02 &      3000 &     2982 &  3181.801 &    -1.299 \\
    PASQUILLA AUTOMATICA &  21201580 &   Simulación 6 &   d01 &      3000 &     2982 &  3294.258 &    -2.030 \\
    PASQUILLA AUTOMATICA &  21201580 &   Simulación 6 &   d02 &      3000 &     2982 &  3072.742 &    -0.590 \\
    PASQUILLA AUTOMATICA &  21201580 &  Simulación 10 &   d01 &      3000 &     2982 &  2994.113 &    -0.079 \\
    PASQUILLA AUTOMATICA &  21201580 &  Simulación 10 &   d02 &      3000 &     2982 &  3201.350 &    -1.426 \\
    PASQUILLA AUTOMATICA &  21201580 &  Simulación 10 &   d03 &      3000 &     2982 &  2915.148 &     0.435 \\
    PASQUILLA AUTOMATICA &  21201580 &   Simulación 7 &   d01 &      3000 &     2982 &  3092.741 &    -0.720 \\
    PASQUILLA AUTOMATICA &  21201580 &   Simulación 7 &   d02 &      3000 &     2982 &  3272.193 &    -1.886 \\
    PASQUILLA AUTOMATICA &  21201580 &   Simulación 8 &   d01 &      3000 &     2982 &  3208.613 &    -1.473 \\
    PASQUILLA AUTOMATICA &  21201580 &   Simulación 8 &   d02 &      3000 &     2982 &  3212.179 &    -1.496 \\
    PASQUILLA AUTOMATICA &  21201580 &   Simulación 9 &   d01 &      3000 &     2982 &  2957.104 &     0.162 \\
    PASQUILLA AUTOMATICA &  21201580 &   Simulación 9 &   d02 &      3000 &     2982 &  3214.341 &    -1.510 \\
  PLUVIOMETRO AUTOMATICO &  21202270 &   Simulación 4 &   d01 &      2685 &     2632 &  2722.653 &    -0.589 \\
  PLUVIOMETRO AUTOMATICO &  21202270 &   Simulación 4 &   d02 &      2685 &     2632 &  2647.986 &    -0.104 \\
  PLUVIOMETRO AUTOMATICO &  21202270 &   Simulación 1 &   d01 &      2685 &     2632 &  2721.551 &    -0.582 \\
  PLUVIOMETRO AUTOMATICO &  21202270 &   Simulación 1 &   d02 &      2685 &     2632 &  2722.653 &    -0.589 \\
  PLUVIOMETRO AUTOMATICO &  21202270 &   Simulación 1 &   d03 &      2685 &     2632 &  2647.984 &    -0.104 \\
  PLUVIOMETRO AUTOMATICO &  21202270 &   Simulación 2 &   d01 &      2685 &     2632 &  2779.850 &    -0.961 \\
  PLUVIOMETRO AUTOMATICO &  21202270 &   Simulación 2 &   d02 &      2685 &     2632 &  2800.063 &    -1.092 \\
  PLUVIOMETRO AUTOMATICO &  21202270 &   Simulación 2 &   d03 &      2685 &     2632 &  2837.535 &    -1.336 \\
  PLUVIOMETRO AUTOMATICO &  21202270 &   Simulación 3 &   d01 &      2685 &     2632 &  2703.304 &    -0.463 \\
  PLUVIOMETRO AUTOMATICO &  21202270 &   Simulación 5 &   d01 &      2685 &     2632 &  2829.348 &    -1.283 \\
  PLUVIOMETRO AUTOMATICO &  21202270 &   Simulación 5 &   d02 &      2685 &     2632 &  2733.804 &    -0.662 \\
  PLUVIOMETRO AUTOMATICO &  21202270 &   Simulación 6 &   d01 &      2685 &     2632 &  2911.771 &    -1.819 \\
  PLUVIOMETRO AUTOMATICO &  21202270 &   Simulación 6 &   d02 &      2685 &     2632 &  2615.880 &     0.105 \\
  PLUVIOMETRO AUTOMATICO &  21202270 &  Simulación 10 &   d01 &      2685 &     2632 &  2794.905 &    -1.059 \\
  PLUVIOMETRO AUTOMATICO &  21202270 &  Simulación 10 &   d02 &      2685 &     2632 &  2633.979 &    -0.013 \\
  PLUVIOMETRO AUTOMATICO &  21202270 &  Simulación 10 &   d03 &      2685 &     2632 &  2583.674 &     0.314 \\
  PLUVIOMETRO AUTOMATICO &  21202270 &   Simulación 7 &   d01 &      2685 &     2632 &  2779.924 &    -0.962 \\
  PLUVIOMETRO AUTOMATICO &  21202270 &   Simulación 7 &   d02 &      2685 &     2632 &  2854.143 &    -1.444 \\
  PLUVIOMETRO AUTOMATICO &  21202270 &   Simulación 8 &   d01 &      2685 &     2632 &  2783.742 &    -0.986 \\
  PLUVIOMETRO AUTOMATICO &  21202270 &   Simulación 8 &   d02 &      2685 &     2632 &  2795.671 &    -1.064 \\
  PLUVIOMETRO AUTOMATICO &  21202270 &   Simulación 9 &   d01 &      2685 &     2632 &  2804.042 &    -1.118 \\
  PLUVIOMETRO AUTOMATICO &  21202270 &   Simulación 9 &   d02 &      2685 &     2632 &  3038.651 &    -2.643 \\
      PLUVIOMETRO AUTOMA &  21202271 &   Simulación 4 &   d01 &      2685 &     2632 &  2722.653 &    -0.589 \\
      PLUVIOMETRO AUTOMA &  21202271 &   Simulación 4 &   d02 &      2685 &     2632 &  2647.986 &    -0.104 \\
      PLUVIOMETRO AUTOMA &  21202271 &   Simulación 1 &   d01 &      2685 &     2632 &  2721.551 &    -0.582 \\
      PLUVIOMETRO AUTOMA &  21202271 &   Simulación 1 &   d02 &      2685 &     2632 &  2722.653 &    -0.589 \\
      PLUVIOMETRO AUTOMA &  21202271 &   Simulación 1 &   d03 &      2685 &     2632 &  2647.984 &    -0.104 \\
      PLUVIOMETRO AUTOMA &  21202271 &   Simulación 2 &   d01 &      2685 &     2632 &  2779.850 &    -0.961 \\
      PLUVIOMETRO AUTOMA &  21202271 &   Simulación 2 &   d02 &      2685 &     2632 &  2800.063 &    -1.092 \\
      PLUVIOMETRO AUTOMA &  21202271 &   Simulación 2 &   d03 &      2685 &     2632 &  2837.535 &    -1.336 \\
      PLUVIOMETRO AUTOMA &  21202271 &   Simulación 3 &   d01 &      2685 &     2632 &  2703.304 &    -0.463 \\
      PLUVIOMETRO AUTOMA &  21202271 &   Simulación 5 &   d01 &      2685 &     2632 &  2829.348 &    -1.283 \\
      PLUVIOMETRO AUTOMA &  21202271 &   Simulación 5 &   d02 &      2685 &     2632 &  2733.804 &    -0.662 \\
      PLUVIOMETRO AUTOMA &  21202271 &   Simulación 6 &   d01 &      2685 &     2632 &  2911.771 &    -1.819 \\
      PLUVIOMETRO AUTOMA &  21202271 &   Simulación 6 &   d02 &      2685 &     2632 &  2615.880 &     0.105 \\
      PLUVIOMETRO AUTOMA &  21202271 &  Simulación 10 &   d01 &      2685 &     2632 &  2794.905 &    -1.059 \\
      PLUVIOMETRO AUTOMA &  21202271 &  Simulación 10 &   d02 &      2685 &     2632 &  2633.979 &    -0.013 \\
      PLUVIOMETRO AUTOMA &  21202271 &  Simulación 10 &   d03 &      2685 &     2632 &  2583.674 &     0.314 \\
      PLUVIOMETRO AUTOMA &  21202271 &   Simulación 7 &   d01 &      2685 &     2632 &  2779.924 &    -0.962 \\
      PLUVIOMETRO AUTOMA &  21202271 &   Simulación 7 &   d02 &      2685 &     2632 &  2854.143 &    -1.444 \\
      PLUVIOMETRO AUTOMA &  21202271 &   Simulación 8 &   d01 &      2685 &     2632 &  2783.742 &    -0.986 \\
      PLUVIOMETRO AUTOMA &  21202271 &   Simulación 8 &   d02 &      2685 &     2632 &  2795.671 &    -1.064 \\
      PLUVIOMETRO AUTOMA &  21202271 &   Simulación 9 &   d01 &      2685 &     2632 &  2804.042 &    -1.118 \\
      PLUVIOMETRO AUTOMA &  21202271 &   Simulación 9 &   d02 &      2685 &     2632 &  3038.651 &    -2.643 \\
           UNIV NACIONAL &  21205012 &   Simulación 4 &   d01 &      2556 &     2579 &  2602.717 &    -0.154 \\
           UNIV NACIONAL &  21205012 &   Simulación 4 &   d02 &      2556 &     2579 &  2548.208 &     0.200 \\
           UNIV NACIONAL &  21205012 &   Simulación 1 &   d01 &      2556 &     2579 &  2721.551 &    -0.927 \\
           UNIV NACIONAL &  21205012 &   Simulación 1 &   d02 &      2556 &     2579 &  2602.717 &    -0.154 \\
           UNIV NACIONAL &  21205012 &   Simulación 1 &   d03 &      2556 &     2579 &  2548.208 &     0.200 \\
           UNIV NACIONAL &  21205012 &   Simulación 2 &   d01 &      2556 &     2579 &  2779.850 &    -1.306 \\
           UNIV NACIONAL &  21205012 &   Simulación 2 &   d02 &      2556 &     2579 &  2800.063 &    -1.437 \\
           UNIV NACIONAL &  21205012 &   Simulación 2 &   d03 &      2556 &     2579 &  2544.917 &     0.222 \\
           UNIV NACIONAL &  21205012 &   Simulación 3 &   d01 &      2556 &     2579 &  2542.831 &     0.235 \\
           UNIV NACIONAL &  21205012 &   Simulación 5 &   d01 &      2556 &     2579 &  2626.328 &    -0.308 \\
           UNIV NACIONAL &  21205012 &   Simulación 5 &   d02 &      2556 &     2579 &  2526.459 &     0.342 \\
           UNIV NACIONAL &  21205012 &   Simulación 6 &   d01 &      2556 &     2579 &  2721.717 &    -0.928 \\
           UNIV NACIONAL &  21205012 &   Simulación 6 &   d02 &      2556 &     2579 &  2550.750 &     0.184 \\
           UNIV NACIONAL &  21205012 &  Simulación 10 &   d01 &      2556 &     2579 &  2794.905 &    -1.403 \\
           UNIV NACIONAL &  21205012 &  Simulación 10 &   d02 &      2556 &     2579 &  2582.112 &    -0.020 \\
           UNIV NACIONAL &  21205012 &  Simulación 10 &   d03 &      2556 &     2579 &  2553.219 &     0.168 \\
           UNIV NACIONAL &  21205012 &   Simulación 7 &   d01 &      2556 &     2579 &  2635.035 &    -0.364 \\
           UNIV NACIONAL &  21205012 &   Simulación 7 &   d02 &      2556 &     2579 &  2552.116 &     0.175 \\
           UNIV NACIONAL &  21205012 &   Simulación 8 &   d01 &      2556 &     2579 &  2682.244 &    -0.671 \\
           UNIV NACIONAL &  21205012 &   Simulación 8 &   d02 &      2556 &     2579 &  2533.702 &     0.294 \\
           UNIV NACIONAL &  21205012 &   Simulación 9 &   d01 &      2556 &     2579 &  2690.521 &    -0.725 \\
           UNIV NACIONAL &  21205012 &   Simulación 9 &   d02 &      2556 &     2579 &  2575.308 &     0.024 \\
          APTO EL DORADO &  21205791 &   Simulación 4 &   d01 &      2547 &     2567 &  2533.177 &     0.220 \\
          APTO EL DORADO &  21205791 &   Simulación 4 &   d02 &      2547 &     2567 &  2545.579 &     0.139 \\
          APTO EL DORADO &  21205791 &   Simulación 1 &   d01 &      2547 &     2567 &  2655.737 &    -0.577 \\
          APTO EL DORADO &  21205791 &   Simulación 1 &   d02 &      2547 &     2567 &  2533.177 &     0.220 \\
          APTO EL DORADO &  21205791 &   Simulación 1 &   d03 &      2547 &     2567 &  2545.579 &     0.139 \\
          APTO EL DORADO &  21205791 &   Simulación 2 &   d01 &      2547 &     2567 &  2779.850 &    -1.384 \\
          APTO EL DORADO &  21205791 &   Simulación 2 &   d02 &      2547 &     2567 &  2581.123 &    -0.092 \\
          APTO EL DORADO &  21205791 &   Simulación 2 &   d03 &      2547 &     2567 &  2546.363 &     0.134 \\
          APTO EL DORADO &  21205791 &   Simulación 3 &   d01 &      2547 &     2567 &  2545.316 &     0.141 \\
          APTO EL DORADO &  21205791 &   Simulación 5 &   d01 &      2547 &     2567 &  2556.435 &     0.069 \\
          APTO EL DORADO &  21205791 &   Simulación 5 &   d02 &      2547 &     2567 &  2545.067 &     0.143 \\
          APTO EL DORADO &  21205791 &   Simulación 6 &   d01 &      2547 &     2567 &  2565.275 &     0.011 \\
          APTO EL DORADO &  21205791 &   Simulación 6 &   d02 &      2547 &     2567 &  2545.973 &     0.137 \\
          APTO EL DORADO &  21205791 &  Simulación 10 &   d01 &      2547 &     2567 &  2563.007 &     0.026 \\
          APTO EL DORADO &  21205791 &  Simulación 10 &   d02 &      2547 &     2567 &  2543.840 &     0.151 \\
          APTO EL DORADO &  21205791 &  Simulación 10 &   d03 &      2547 &     2567 &  2546.065 &     0.136 \\
          APTO EL DORADO &  21205791 &   Simulación 7 &   d01 &      2547 &     2567 &  2635.035 &    -0.442 \\
          APTO EL DORADO &  21205791 &   Simulación 7 &   d02 &      2547 &     2567 &  2543.565 &     0.152 \\
          APTO EL DORADO &  21205791 &   Simulación 8 &   d01 &      2547 &     2567 &  2682.244 &    -0.749 \\
          APTO EL DORADO &  21205791 &   Simulación 8 &   d02 &      2547 &     2567 &  2541.497 &     0.166 \\
          APTO EL DORADO &  21205791 &   Simulación 9 &   d01 &      2547 &     2567 &  2690.521 &    -0.803 \\
          APTO EL DORADO &  21205791 &   Simulación 9 &   d02 &      2547 &     2567 &  2538.571 &     0.185 \\
        NUEVA GENERACION &  21206600 &   Simulación 4 &   d01 &      2590 &     2574 &  2587.997 &    -0.091 \\
        NUEVA GENERACION &  21206600 &   Simulación 4 &   d02 &      2590 &     2574 &  2545.151 &     0.188 \\
        NUEVA GENERACION &  21206600 &   Simulación 1 &   d01 &      2590 &     2574 &  2655.737 &    -0.531 \\
        NUEVA GENERACION &  21206600 &   Simulación 1 &   d02 &      2590 &     2574 &  2587.997 &    -0.091 \\
        NUEVA GENERACION &  21206600 &   Simulación 1 &   d03 &      2590 &     2574 &  2545.151 &     0.188 \\
        NUEVA GENERACION &  21206600 &   Simulación 2 &   d01 &      2590 &     2574 &  2722.959 &    -0.968 \\
        NUEVA GENERACION &  21206600 &   Simulación 2 &   d02 &      2590 &     2574 &  2600.809 &    -0.174 \\
        NUEVA GENERACION &  21206600 &   Simulación 2 &   d03 &      2590 &     2574 &  2555.247 &     0.122 \\
        NUEVA GENERACION &  21206600 &   Simulación 3 &   d01 &      2590 &     2574 &  2558.323 &     0.102 \\
        NUEVA GENERACION &  21206600 &   Simulación 5 &   d01 &      2590 &     2574 &  2597.661 &    -0.154 \\
        NUEVA GENERACION &  21206600 &   Simulación 5 &   d02 &      2590 &     2574 &  2580.811 &    -0.044 \\
        NUEVA GENERACION &  21206600 &   Simulación 6 &   d01 &      2590 &     2574 &  2614.594 &    -0.264 \\
        NUEVA GENERACION &  21206600 &   Simulación 6 &   d02 &      2590 &     2574 &  2548.005 &     0.169 \\
        NUEVA GENERACION &  21206600 &  Simulación 10 &   d01 &      2590 &     2574 &  2583.482 &    -0.062 \\
        NUEVA GENERACION &  21206600 &  Simulación 10 &   d02 &      2590 &     2574 &  2548.838 &     0.164 \\
        NUEVA GENERACION &  21206600 &  Simulación 10 &   d03 &      2590 &     2574 &  2552.604 &     0.139 \\
        NUEVA GENERACION &  21206600 &   Simulación 7 &   d01 &      2590 &     2574 &  2575.312 &    -0.009 \\
        NUEVA GENERACION &  21206600 &   Simulación 7 &   d02 &      2590 &     2574 &  2559.412 &     0.095 \\
        NUEVA GENERACION &  21206600 &   Simulación 8 &   d01 &      2590 &     2574 &  2624.236 &    -0.327 \\
        NUEVA GENERACION &  21206600 &   Simulación 8 &   d02 &      2590 &     2574 &  2570.761 &     0.021 \\
        NUEVA GENERACION &  21206600 &   Simulación 9 &   d01 &      2590 &     2574 &  2690.521 &    -0.757 \\
        NUEVA GENERACION &  21206600 &   Simulación 9 &   d02 &      2590 &     2574 &  2551.696 &     0.145 \\
  SAN JOAQUIN AUTOMATICA &  21206710 &   Simulación 4 &   d01 &       757 &      679 &   933.051 &    -1.651 \\
  SAN JOAQUIN AUTOMATICA &  21206710 &   Simulación 4 &   d02 &       757 &      679 &   666.025 &     0.084 \\
  SAN JOAQUIN AUTOMATICA &  21206710 &   Simulación 1 &   d01 &       757 &      679 &  1348.519 &    -4.352 \\
  SAN JOAQUIN AUTOMATICA &  21206710 &   Simulación 1 &   d02 &       757 &      679 &   933.051 &    -1.651 \\
  SAN JOAQUIN AUTOMATICA &  21206710 &   Simulación 1 &   d03 &       757 &      679 &   666.042 &     0.084 \\
  SAN JOAQUIN AUTOMATICA &  21206710 &   Simulación 2 &   d01 &       757 &      679 &  1760.607 &    -7.030 \\
  SAN JOAQUIN AUTOMATICA &  21206710 &   Simulación 2 &   d02 &       757 &      679 &   951.464 &    -1.771 \\
  SAN JOAQUIN AUTOMATICA &  21206710 &   Simulación 2 &   d03 &       757 &      679 &   786.335 &    -0.698 \\
  SAN JOAQUIN AUTOMATICA &  21206710 &   Simulación 3 &   d01 &       757 &      679 &   700.176 &    -0.138 \\
  SAN JOAQUIN AUTOMATICA &  21206710 &   Simulación 5 &   d01 &       757 &      679 &  1058.955 &    -2.470 \\
  SAN JOAQUIN AUTOMATICA &  21206710 &   Simulación 5 &   d02 &       757 &      679 &   767.243 &    -0.574 \\
  SAN JOAQUIN AUTOMATICA &  21206710 &   Simulación 6 &   d01 &       757 &      679 &  1122.110 &    -2.880 \\
  SAN JOAQUIN AUTOMATICA &  21206710 &   Simulación 6 &   d02 &       757 &      679 &   808.603 &    -0.842 \\
  SAN JOAQUIN AUTOMATICA &  21206710 &  Simulación 10 &   d01 &       757 &      679 &   860.803 &    -1.182 \\
  SAN JOAQUIN AUTOMATICA &  21206710 &  Simulación 10 &   d02 &       757 &      679 &   783.854 &    -0.682 \\
  SAN JOAQUIN AUTOMATICA &  21206710 &  Simulación 10 &   d03 &       757 &      679 &   728.894 &    -0.324 \\
  SAN JOAQUIN AUTOMATICA &  21206710 &   Simulación 7 &   d01 &       757 &      679 &  1112.754 &    -2.819 \\
  SAN JOAQUIN AUTOMATICA &  21206710 &   Simulación 7 &   d02 &       757 &      679 &   836.271 &    -1.022 \\
  SAN JOAQUIN AUTOMATICA &  21206710 &   Simulación 8 &   d01 &       757 &      679 &  1065.776 &    -2.514 \\
  SAN JOAQUIN AUTOMATICA &  21206710 &   Simulación 8 &   d02 &       757 &      679 &   900.093 &    -1.437 \\
  SAN JOAQUIN AUTOMATICA &  21206710 &   Simulación 9 &   d01 &       757 &      679 &  1085.332 &    -2.641 \\
  SAN JOAQUIN AUTOMATICA &  21206710 &   Simulación 9 &   d02 &       757 &      679 &   990.785 &    -2.027 \\
       HDA STA ANA AUTOM &  21206790 &   Simulación 4 &   d01 &      2572 &     2590 &  2623.148 &    -0.215 \\
       HDA STA ANA AUTOM &  21206790 &   Simulación 4 &   d02 &      2572 &     2590 &  2569.476 &     0.133 \\
       HDA STA ANA AUTOM &  21206790 &   Simulación 1 &   d01 &      2572 &     2590 &  2839.934 &    -1.625 \\
       HDA STA ANA AUTOM &  21206790 &   Simulación 1 &   d02 &      2572 &     2590 &  2623.148 &    -0.215 \\
       HDA STA ANA AUTOM &  21206790 &   Simulación 1 &   d03 &      2572 &     2590 &  2569.479 &     0.133 \\
       HDA STA ANA AUTOM &  21206790 &   Simulación 2 &   d01 &      2572 &     2590 &  2782.875 &    -1.254 \\
       HDA STA ANA AUTOM &  21206790 &   Simulación 2 &   d02 &      2572 &     2590 &  2809.169 &    -1.425 \\
       HDA STA ANA AUTOM &  21206790 &   Simulación 2 &   d03 &      2572 &     2590 &  2617.562 &    -0.179 \\
       HDA STA ANA AUTOM &  21206790 &   Simulación 3 &   d01 &      2572 &     2590 &  2550.597 &     0.256 \\
       HDA STA ANA AUTOM &  21206790 &   Simulación 5 &   d01 &      2572 &     2590 &  2730.188 &    -0.911 \\
       HDA STA ANA AUTOM &  21206790 &   Simulación 5 &   d02 &      2572 &     2590 &  2569.245 &     0.135 \\
       HDA STA ANA AUTOM &  21206790 &   Simulación 6 &   d01 &      2572 &     2590 &  2822.516 &    -1.511 \\
       HDA STA ANA AUTOM &  21206790 &   Simulación 6 &   d02 &      2572 &     2590 &  2594.370 &    -0.028 \\
       HDA STA ANA AUTOM &  21206790 &  Simulación 10 &   d01 &      2572 &     2590 &  2874.522 &    -1.849 \\
       HDA STA ANA AUTOM &  21206790 &  Simulación 10 &   d02 &      2572 &     2590 &  2611.096 &    -0.137 \\
       HDA STA ANA AUTOM &  21206790 &  Simulación 10 &   d03 &      2572 &     2590 &  2560.835 &     0.190 \\
       HDA STA ANA AUTOM &  21206790 &   Simulación 7 &   d01 &      2572 &     2590 &  2900.687 &    -2.019 \\
       HDA STA ANA AUTOM &  21206790 &   Simulación 7 &   d02 &      2572 &     2590 &  2644.717 &    -0.356 \\
       HDA STA ANA AUTOM &  21206790 &   Simulación 8 &   d01 &      2572 &     2590 &  2811.274 &    -1.438 \\
       HDA STA ANA AUTOM &  21206790 &   Simulación 8 &   d02 &      2572 &     2590 &  2664.807 &    -0.486 \\
       HDA STA ANA AUTOM &  21206790 &   Simulación 9 &   d01 &      2572 &     2590 &  2881.689 &    -1.896 \\
       HDA STA ANA AUTOM &  21206790 &   Simulación 9 &   d02 &      2572 &     2590 &  2636.435 &    -0.302 \\
 VILLA TERESA AUTOMATICA &  21206920 &   Simulación 4 &   d01 &      3624 &     3423 &  3361.720 &     0.398 \\
 VILLA TERESA AUTOMATICA &  21206920 &   Simulación 4 &   d02 &      3624 &     3423 &  3322.876 &     0.651 \\
 VILLA TERESA AUTOMATICA &  21206920 &   Simulación 1 &   d01 &      3624 &     3423 &  2965.059 &     2.977 \\
 VILLA TERESA AUTOMATICA &  21206920 &   Simulación 1 &   d02 &      3624 &     3423 &  3361.720 &     0.398 \\
 VILLA TERESA AUTOMATICA &  21206920 &   Simulación 1 &   d03 &      3624 &     3423 &  3322.876 &     0.651 \\
 VILLA TERESA AUTOMATICA &  21206920 &   Simulación 2 &   d01 &      3624 &     3423 &  2794.138 &     4.088 \\
 VILLA TERESA AUTOMATICA &  21206920 &   Simulación 2 &   d02 &      3624 &     3423 &  3353.045 &     0.455 \\
 VILLA TERESA AUTOMATICA &  21206920 &   Simulación 2 &   d03 &      3624 &     3423 &  3488.915 &    -0.428 \\
 VILLA TERESA AUTOMATICA &  21206920 &   Simulación 3 &   d01 &      3624 &     3423 &  3347.407 &     0.491 \\
 VILLA TERESA AUTOMATICA &  21206920 &   Simulación 5 &   d01 &      3624 &     3423 &  3353.524 &     0.452 \\
 VILLA TERESA AUTOMATICA &  21206920 &   Simulación 5 &   d02 &      3624 &     3423 &  3447.272 &    -0.158 \\
 VILLA TERESA AUTOMATICA &  21206920 &   Simulación 6 &   d01 &      3624 &     3423 &  3521.569 &    -0.641 \\
 VILLA TERESA AUTOMATICA &  21206920 &   Simulación 6 &   d02 &      3624 &     3423 &  3405.558 &     0.113 \\
 VILLA TERESA AUTOMATICA &  21206920 &  Simulación 10 &   d01 &      3624 &     3423 &  3310.201 &     0.733 \\
 VILLA TERESA AUTOMATICA &  21206920 &  Simulación 10 &   d02 &      3624 &     3423 &  3488.554 &    -0.426 \\
 VILLA TERESA AUTOMATICA &  21206920 &  Simulación 10 &   d03 &      3624 &     3423 &  3474.092 &    -0.332 \\
 VILLA TERESA AUTOMATICA &  21206920 &   Simulación 7 &   d01 &      3624 &     3423 &  3366.127 &     0.370 \\
 VILLA TERESA AUTOMATICA &  21206920 &   Simulación 7 &   d02 &      3624 &     3423 &  3396.196 &     0.174 \\
 VILLA TERESA AUTOMATICA &  21206920 &   Simulación 8 &   d01 &      3624 &     3423 &  3208.613 &     1.394 \\
 VILLA TERESA AUTOMATICA &  21206920 &   Simulación 8 &   d02 &      3624 &     3423 &  3539.942 &    -0.760 \\
 VILLA TERESA AUTOMATICA &  21206920 &   Simulación 9 &   d01 &      3624 &     3423 &  2957.104 &     3.028 \\
 VILLA TERESA AUTOMATICA &  21206920 &   Simulación 9 &   d02 &      3624 &     3423 &  3601.742 &    -1.162 \\
            PMO GUERRERO &  21206930 &   Simulación 4 &   d01 &      3257 &     3253 &  3130.215 &     0.798 \\
            PMO GUERRERO &  21206930 &   Simulación 4 &   d02 &      3257 &     3253 &  3206.798 &     0.300 \\
            PMO GUERRERO &  21206930 &   Simulación 1 &   d01 &      3257 &     3253 &  2945.157 &     2.001 \\
            PMO GUERRERO &  21206930 &   Simulación 1 &   d02 &      3257 &     3253 &  3130.215 &     0.798 \\
            PMO GUERRERO &  21206930 &   Simulación 1 &   d03 &      3257 &     3253 &  3206.814 &     0.300 \\
            PMO GUERRERO &  21206930 &   Simulación 2 &   d01 &      3257 &     3253 &  2332.186 &     5.985 \\
            PMO GUERRERO &  21206930 &   Simulación 2 &   d02 &      3257 &     3253 &  2984.162 &     1.747 \\
            PMO GUERRERO &  21206930 &   Simulación 2 &   d03 &      3257 &     3253 &  3273.513 &    -0.133 \\
            PMO GUERRERO &  21206930 &   Simulación 3 &   d01 &      3257 &     3253 &  3280.275 &    -0.177 \\
            PMO GUERRERO &  21206930 &   Simulación 5 &   d01 &      3257 &     3253 &  3070.346 &     1.187 \\
            PMO GUERRERO &  21206930 &   Simulación 5 &   d02 &      3257 &     3253 &  3249.456 &     0.023 \\
            PMO GUERRERO &  21206930 &   Simulación 6 &   d01 &      3257 &     3253 &  3051.588 &     1.309 \\
            PMO GUERRERO &  21206930 &   Simulación 6 &   d02 &      3257 &     3253 &  3366.951 &    -0.741 \\
            PMO GUERRERO &  21206930 &  Simulación 10 &   d01 &      3257 &     3253 &  3070.802 &     1.184 \\
            PMO GUERRERO &  21206930 &  Simulación 10 &   d02 &      3257 &     3253 &  3284.143 &    -0.202 \\
            PMO GUERRERO &  21206930 &  Simulación 10 &   d03 &      3257 &     3253 &  3298.917 &    -0.298 \\
            PMO GUERRERO &  21206930 &   Simulación 7 &   d01 &      3257 &     3253 &  3056.063 &     1.280 \\
            PMO GUERRERO &  21206930 &   Simulación 7 &   d02 &      3257 &     3253 &  3107.787 &     0.944 \\
            PMO GUERRERO &  21206930 &   Simulación 8 &   d01 &      3257 &     3253 &  2929.954 &     2.100 \\
            PMO GUERRERO &  21206930 &   Simulación 8 &   d02 &      3257 &     3253 &  3235.714 &     0.112 \\
            PMO GUERRERO &  21206930 &   Simulación 9 &   d01 &      3257 &     3253 &  2760.857 &     3.199 \\
            PMO GUERRERO &  21206930 &   Simulación 9 &   d02 &      3257 &     3253 &  3154.573 &     0.640 \\
          CIUDAD BOLIVAR &  21206940 &   Simulación 4 &   d01 &      2687 &     2837 &  2601.333 &     1.532 \\
          CIUDAD BOLIVAR &  21206940 &   Simulación 4 &   d02 &      2687 &     2837 &  2682.704 &     1.003 \\
          CIUDAD BOLIVAR &  21206940 &   Simulación 1 &   d01 &      2687 &     2837 &  2721.551 &     0.750 \\
          CIUDAD BOLIVAR &  21206940 &   Simulación 1 &   d02 &      2687 &     2837 &  2601.333 &     1.532 \\
          CIUDAD BOLIVAR &  21206940 &   Simulación 1 &   d03 &      2687 &     2837 &  2682.696 &     1.003 \\
          CIUDAD BOLIVAR &  21206940 &   Simulación 2 &   d01 &      2687 &     2837 &  2779.850 &     0.371 \\
          CIUDAD BOLIVAR &  21206940 &   Simulación 2 &   d02 &      2687 &     2837 &  2733.900 &     0.670 \\
          CIUDAD BOLIVAR &  21206940 &   Simulación 2 &   d03 &      2687 &     2837 &  2592.826 &     1.587 \\
          CIUDAD BOLIVAR &  21206940 &   Simulación 3 &   d01 &      2687 &     2837 &  2682.417 &     1.005 \\
          CIUDAD BOLIVAR &  21206940 &   Simulación 5 &   d01 &      2687 &     2837 &  2703.521 &     0.868 \\
          CIUDAD BOLIVAR &  21206940 &   Simulación 5 &   d02 &      2687 &     2837 &  2606.599 &     1.498 \\
          CIUDAD BOLIVAR &  21206940 &   Simulación 6 &   d01 &      2687 &     2837 &  2723.267 &     0.739 \\
          CIUDAD BOLIVAR &  21206940 &   Simulación 6 &   d02 &      2687 &     2837 &  2606.279 &     1.500 \\
          CIUDAD BOLIVAR &  21206940 &  Simulación 10 &   d01 &      2687 &     2837 &  2680.858 &     1.015 \\
          CIUDAD BOLIVAR &  21206940 &  Simulación 10 &   d02 &      2687 &     2837 &  2705.726 &     0.853 \\
          CIUDAD BOLIVAR &  21206940 &  Simulación 10 &   d03 &      2687 &     2837 &  2707.488 &     0.842 \\
          CIUDAD BOLIVAR &  21206940 &   Simulación 7 &   d01 &      2687 &     2837 &  2821.398 &     0.101 \\
          CIUDAD BOLIVAR &  21206940 &   Simulación 7 &   d02 &      2687 &     2837 &  2653.813 &     1.191 \\
          CIUDAD BOLIVAR &  21206940 &   Simulación 8 &   d01 &      2687 &     2837 &  2933.723 &    -0.629 \\
          CIUDAD BOLIVAR &  21206940 &   Simulación 8 &   d02 &      2687 &     2837 &  2649.767 &     1.217 \\
          CIUDAD BOLIVAR &  21206940 &   Simulación 9 &   d01 &      2687 &     2837 &  2804.042 &     0.214 \\
          CIUDAD BOLIVAR &  21206940 &   Simulación 9 &   d02 &      2687 &     2837 &  2551.758 &     1.854 \\
         PMO GUACHENEQUE &  21206950 &   Simulación 4 &   d01 &      2300 &     3288 &  3015.451 &     1.772 \\
         PMO GUACHENEQUE &  21206950 &   Simulación 4 &   d02 &      2300 &     3288 &  3292.294 &    -0.028 \\
         PMO GUACHENEQUE &  21206950 &   Simulación 1 &   d01 &      2300 &     3288 &  2719.605 &     3.695 \\
         PMO GUACHENEQUE &  21206950 &   Simulación 1 &   d02 &      2300 &     3288 &  3015.451 &     1.772 \\
         PMO GUACHENEQUE &  21206950 &   Simulación 1 &   d03 &      2300 &     3288 &  3292.303 &    -0.028 \\
         PMO GUACHENEQUE &  21206950 &   Simulación 2 &   d01 &      2300 &     3288 &  2576.131 &     4.627 \\
         PMO GUACHENEQUE &  21206950 &   Simulación 2 &   d02 &      2300 &     3288 &  2793.305 &     3.216 \\
         PMO GUACHENEQUE &  21206950 &   Simulación 2 &   d03 &      2300 &     3288 &  3155.789 &     0.859 \\
         PMO GUACHENEQUE &  21206950 &   Simulación 3 &   d01 &      2300 &     3288 &  3290.693 &    -0.018 \\
         PMO GUACHENEQUE &  21206950 &   Simulación 5 &   d01 &      2300 &     3288 &  2750.177 &     3.496 \\
         PMO GUACHENEQUE &  21206950 &   Simulación 5 &   d02 &      2300 &     3288 &  3220.652 &     0.438 \\
         PMO GUACHENEQUE &  21206950 &   Simulación 6 &   d01 &      2300 &     3288 &  2812.589 &     3.090 \\
         PMO GUACHENEQUE &  21206950 &   Simulación 6 &   d02 &      2300 &     3288 &  3136.449 &     0.985 \\
         PMO GUACHENEQUE &  21206950 &  Simulación 10 &   d01 &      2300 &     3288 &  2875.902 &     2.679 \\
         PMO GUACHENEQUE &  21206950 &  Simulación 10 &   d02 &      2300 &     3288 &  3092.728 &     1.269 \\
         PMO GUACHENEQUE &  21206950 &  Simulación 10 &   d03 &      2300 &     3288 &  3325.311 &    -0.243 \\
         PMO GUACHENEQUE &  21206950 &   Simulación 7 &   d01 &      2300 &     3288 &  2635.701 &     4.240 \\
         PMO GUACHENEQUE &  21206950 &   Simulación 7 &   d02 &      2300 &     3288 &  3145.616 &     0.925 \\
         PMO GUACHENEQUE &  21206950 &   Simulación 8 &   d01 &      2300 &     3288 &  2593.574 &     4.514 \\
         PMO GUACHENEQUE &  21206950 &   Simulación 8 &   d02 &      2300 &     3288 &  3075.276 &     1.383 \\
         PMO GUACHENEQUE &  21206950 &   Simulación 9 &   d01 &      2300 &     3288 &  2547.508 &     4.813 \\
         PMO GUACHENEQUE &  21206950 &   Simulación 9 &   d02 &      2300 &     3288 &  2909.620 &     2.459 \\
            IDEAM BOGOTA &  21206960 &   Simulación 4 &   d01 &      2646 &     2679 &  2722.653 &    -0.284 \\
            IDEAM BOGOTA &  21206960 &   Simulación 4 &   d02 &      2646 &     2679 &  2706.269 &    -0.177 \\
            IDEAM BOGOTA &  21206960 &   Simulación 1 &   d01 &      2646 &     2679 &  2721.551 &    -0.277 \\
            IDEAM BOGOTA &  21206960 &   Simulación 1 &   d02 &      2646 &     2679 &  2722.653 &    -0.284 \\
            IDEAM BOGOTA &  21206960 &   Simulación 1 &   d03 &      2646 &     2679 &  2706.283 &    -0.177 \\
            IDEAM BOGOTA &  21206960 &   Simulación 2 &   d01 &      2646 &     2679 &  2779.850 &    -0.656 \\
            IDEAM BOGOTA &  21206960 &   Simulación 2 &   d02 &      2646 &     2679 &  2800.063 &    -0.787 \\
            IDEAM BOGOTA &  21206960 &   Simulación 2 &   d03 &      2646 &     2679 &  2837.535 &    -1.030 \\
            IDEAM BOGOTA &  21206960 &   Simulación 3 &   d01 &      2646 &     2679 &  2773.016 &    -0.611 \\
            IDEAM BOGOTA &  21206960 &   Simulación 5 &   d01 &      2646 &     2679 &  2892.579 &    -1.388 \\
            IDEAM BOGOTA &  21206960 &   Simulación 5 &   d02 &      2646 &     2679 &  2837.607 &    -1.031 \\
            IDEAM BOGOTA &  21206960 &   Simulación 6 &   d01 &      2646 &     2679 &  2911.771 &    -1.513 \\
            IDEAM BOGOTA &  21206960 &   Simulación 6 &   d02 &      2646 &     2679 &  3056.533 &    -2.454 \\
            IDEAM BOGOTA &  21206960 &  Simulación 10 &   d01 &      2646 &     2679 &  2794.905 &    -0.753 \\
            IDEAM BOGOTA &  21206960 &  Simulación 10 &   d02 &      2646 &     2679 &  2954.639 &    -1.792 \\
            IDEAM BOGOTA &  21206960 &  Simulación 10 &   d03 &      2646 &     2679 &  2780.155 &    -0.658 \\
            IDEAM BOGOTA &  21206960 &   Simulación 7 &   d01 &      2646 &     2679 &  2779.924 &    -0.656 \\
            IDEAM BOGOTA &  21206960 &   Simulación 7 &   d02 &      2646 &     2679 &  2854.143 &    -1.138 \\
            IDEAM BOGOTA &  21206960 &   Simulación 8 &   d01 &      2646 &     2679 &  2783.742 &    -0.681 \\
            IDEAM BOGOTA &  21206960 &   Simulación 8 &   d02 &      2646 &     2679 &  2795.671 &    -0.758 \\
            IDEAM BOGOTA &  21206960 &   Simulación 9 &   d01 &      2646 &     2679 &  2804.042 &    -0.813 \\
            IDEAM BOGOTA &  21206960 &   Simulación 9 &   d02 &      2646 &     2679 &  3038.651 &    -2.338 \\
      STA CRUZ DE SIECHA &  21206980 &   Simulación 4 &   d01 &      3100 &     3194 &  3029.000 &     1.072 \\
      STA CRUZ DE SIECHA &  21206980 &   Simulación 4 &   d02 &      3100 &     3194 &  3034.146 &     1.039 \\
      STA CRUZ DE SIECHA &  21206980 &   Simulación 1 &   d01 &      3100 &     3194 &  3026.181 &     1.091 \\
      STA CRUZ DE SIECHA &  21206980 &   Simulación 1 &   d02 &      3100 &     3194 &  3029.000 &     1.072 \\
      STA CRUZ DE SIECHA &  21206980 &   Simulación 1 &   d03 &      3100 &     3194 &  3034.146 &     1.039 \\
      STA CRUZ DE SIECHA &  21206980 &   Simulación 2 &   d01 &      3100 &     3194 &  2827.854 &     2.380 \\
      STA CRUZ DE SIECHA &  21206980 &   Simulación 2 &   d02 &      3100 &     3194 &  3065.028 &     0.838 \\
      STA CRUZ DE SIECHA &  21206980 &   Simulación 2 &   d03 &      3100 &     3194 &  3218.524 &    -0.159 \\
      STA CRUZ DE SIECHA &  21206980 &   Simulación 3 &   d01 &      3100 &     3194 &  3105.415 &     0.576 \\
      STA CRUZ DE SIECHA &  21206980 &   Simulación 5 &   d01 &      3100 &     3194 &  3024.045 &     1.105 \\
      STA CRUZ DE SIECHA &  21206980 &   Simulación 5 &   d02 &      3100 &     3194 &  3217.807 &    -0.155 \\
      STA CRUZ DE SIECHA &  21206980 &   Simulación 6 &   d01 &      3100 &     3194 &  3159.301 &     0.226 \\
      STA CRUZ DE SIECHA &  21206980 &   Simulación 6 &   d02 &      3100 &     3194 &  3096.111 &     0.636 \\
      STA CRUZ DE SIECHA &  21206980 &  Simulación 10 &   d01 &      3100 &     3194 &  3021.660 &     1.120 \\
      STA CRUZ DE SIECHA &  21206980 &  Simulación 10 &   d02 &      3100 &     3194 &  3274.939 &    -0.526 \\
      STA CRUZ DE SIECHA &  21206980 &  Simulación 10 &   d03 &      3100 &     3194 &  3068.785 &     0.814 \\
      STA CRUZ DE SIECHA &  21206980 &   Simulación 7 &   d01 &      3100 &     3194 &  2959.399 &     1.525 \\
      STA CRUZ DE SIECHA &  21206980 &   Simulación 7 &   d02 &      3100 &     3194 &  3154.565 &     0.256 \\
      STA CRUZ DE SIECHA &  21206980 &   Simulación 8 &   d01 &      3100 &     3194 &  2993.287 &     1.305 \\
      STA CRUZ DE SIECHA &  21206980 &   Simulación 8 &   d02 &      3100 &     3194 &  3104.888 &     0.579 \\
      STA CRUZ DE SIECHA &  21206980 &   Simulación 9 &   d01 &      3100 &     3194 &  2910.266 &     1.844 \\
      STA CRUZ DE SIECHA &  21206980 &   Simulación 9 &   d02 &      3100 &     3194 &  3200.764 &    -0.044 \\
    TIBAITATA AUTOMATICA &  21206990 &   Simulación 4 &   d01 &      2543 &     2560 &  2542.886 &     0.111 \\
    TIBAITATA AUTOMATICA &  21206990 &   Simulación 4 &   d02 &      2543 &     2560 &  2543.264 &     0.109 \\
    TIBAITATA AUTOMATICA &  21206990 &   Simulación 1 &   d01 &      2543 &     2560 &  2655.737 &    -0.622 \\
    TIBAITATA AUTOMATICA &  21206990 &   Simulación 1 &   d02 &      2543 &     2560 &  2542.886 &     0.111 \\
    TIBAITATA AUTOMATICA &  21206990 &   Simulación 1 &   d03 &      2543 &     2560 &  2543.264 &     0.109 \\
    TIBAITATA AUTOMATICA &  21206990 &   Simulación 2 &   d01 &      2543 &     2560 &  2779.850 &    -1.429 \\
    TIBAITATA AUTOMATICA &  21206990 &   Simulación 2 &   d02 &      2543 &     2560 &  2581.123 &    -0.137 \\
    TIBAITATA AUTOMATICA &  21206990 &   Simulación 2 &   d03 &      2543 &     2560 &  2543.919 &     0.105 \\
    TIBAITATA AUTOMATICA &  21206990 &   Simulación 3 &   d01 &      2543 &     2560 &  2541.202 &     0.122 \\
    TIBAITATA AUTOMATICA &  21206990 &   Simulación 5 &   d01 &      2543 &     2560 &  2574.024 &    -0.091 \\
    TIBAITATA AUTOMATICA &  21206990 &   Simulación 5 &   d02 &      2543 &     2560 &  2543.080 &     0.110 \\
    TIBAITATA AUTOMATICA &  21206990 &   Simulación 6 &   d01 &      2543 &     2560 &  2565.673 &    -0.037 \\
    TIBAITATA AUTOMATICA &  21206990 &   Simulación 6 &   d02 &      2543 &     2560 &  2533.766 &     0.171 \\
    TIBAITATA AUTOMATICA &  21206990 &  Simulación 10 &   d01 &      2543 &     2560 &  2563.007 &    -0.020 \\
    TIBAITATA AUTOMATICA &  21206990 &  Simulación 10 &   d02 &      2543 &     2560 &  2541.515 &     0.120 \\
    TIBAITATA AUTOMATICA &  21206990 &  Simulación 10 &   d03 &      2543 &     2560 &  2541.705 &     0.119 \\
    TIBAITATA AUTOMATICA &  21206990 &   Simulación 7 &   d01 &      2543 &     2560 &  2542.112 &     0.116 \\
    TIBAITATA AUTOMATICA &  21206990 &   Simulación 7 &   d02 &      2543 &     2560 &  2537.334 &     0.147 \\
    TIBAITATA AUTOMATICA &  21206990 &   Simulación 8 &   d01 &      2543 &     2560 &  2682.244 &    -0.795 \\
    TIBAITATA AUTOMATICA &  21206990 &   Simulación 8 &   d02 &      2543 &     2560 &  2544.215 &     0.103 \\
    TIBAITATA AUTOMATICA &  21206990 &   Simulación 9 &   d01 &      2543 &     2560 &  2475.938 &     0.546 \\
    TIBAITATA AUTOMATICA &  21206990 &   Simulación 9 &   d02 &      2543 &     2560 &  2546.963 &     0.085 \\
        STA ROSITA AUTOM &  21209920 &   Simulación 4 &   d01 &      2618 &     2887 &  2906.010 &    -0.124 \\
        STA ROSITA AUTOM &  21209920 &   Simulación 4 &   d02 &      2618 &     2887 &  2897.928 &    -0.071 \\
        STA ROSITA AUTOM &  21209920 &   Simulación 1 &   d01 &      2618 &     2887 &  2943.492 &    -0.367 \\
        STA ROSITA AUTOM &  21209920 &   Simulación 1 &   d02 &      2618 &     2887 &  2906.010 &    -0.124 \\
        STA ROSITA AUTOM &  21209920 &   Simulación 1 &   d03 &      2618 &     2887 &  2897.940 &    -0.071 \\
        STA ROSITA AUTOM &  21209920 &   Simulación 2 &   d01 &      2618 &     2887 &  2782.875 &     0.677 \\
        STA ROSITA AUTOM &  21209920 &   Simulación 2 &   d02 &      2618 &     2887 &  2798.867 &     0.573 \\
        STA ROSITA AUTOM &  21209920 &   Simulación 2 &   d03 &      2618 &     2887 &  2887.856 &    -0.006 \\
        STA ROSITA AUTOM &  21209920 &   Simulación 3 &   d01 &      2618 &     2887 &  2892.493 &    -0.036 \\
        STA ROSITA AUTOM &  21209920 &   Simulación 5 &   d01 &      2618 &     2887 &  2808.339 &     0.511 \\
        STA ROSITA AUTOM &  21209920 &   Simulación 5 &   d02 &      2618 &     2887 &  2946.644 &    -0.388 \\
        STA ROSITA AUTOM &  21209920 &   Simulación 6 &   d01 &      2618 &     2887 &  2795.152 &     0.597 \\
        STA ROSITA AUTOM &  21209920 &   Simulación 6 &   d02 &      2618 &     2887 &  2938.428 &    -0.334 \\
        STA ROSITA AUTOM &  21209920 &  Simulación 10 &   d01 &      2618 &     2887 &  2795.454 &     0.595 \\
        STA ROSITA AUTOM &  21209920 &  Simulación 10 &   d02 &      2618 &     2887 &  2884.661 &     0.015 \\
        STA ROSITA AUTOM &  21209920 &  Simulación 10 &   d03 &      2618 &     2887 &  2898.550 &    -0.075 \\
        STA ROSITA AUTOM &  21209920 &   Simulación 7 &   d01 &      2618 &     2887 &  2830.168 &     0.369 \\
        STA ROSITA AUTOM &  21209920 &   Simulación 7 &   d02 &      2618 &     2887 &  2948.572 &    -0.400 \\
        STA ROSITA AUTOM &  21209920 &   Simulación 8 &   d01 &      2618 &     2887 &  2846.121 &     0.266 \\
        STA ROSITA AUTOM &  21209920 &   Simulación 8 &   d02 &      2618 &     2887 &  2910.236 &    -0.151 \\
        STA ROSITA AUTOM &  21209920 &   Simulación 9 &   d01 &      2618 &     2887 &  2940.136 &    -0.345 \\
        STA ROSITA AUTOM &  21209920 &   Simulación 9 &   d02 &      2618 &     2887 &  2886.506 &     0.003 \\
      SAN CAYETANO AUTOM &  23125170 &   Simulación 4 &   d01 &      2807 &     3103 &  2931.182 &     1.117 \\
      SAN CAYETANO AUTOM &  23125170 &   Simulación 4 &   d02 &      2807 &     3103 &  3024.846 &     0.508 \\
      SAN CAYETANO AUTOM &  23125170 &   Simulación 1 &   d01 &      2807 &     3103 &  2721.551 &     2.479 \\
      SAN CAYETANO AUTOM &  23125170 &   Simulación 1 &   d02 &      2807 &     3103 &  2931.182 &     1.117 \\
      SAN CAYETANO AUTOM &  23125170 &   Simulación 1 &   d03 &      2807 &     3103 &  3024.836 &     0.508 \\
      SAN CAYETANO AUTOM &  23125170 &   Simulación 2 &   d01 &      2807 &     3103 &  2779.850 &     2.100 \\
      SAN CAYETANO AUTOM &  23125170 &   Simulación 2 &   d02 &      2807 &     3103 &  2864.682 &     1.549 \\
      SAN CAYETANO AUTOM &  23125170 &   Simulación 2 &   d03 &      2807 &     3103 &  2877.195 &     1.468 \\
      SAN CAYETANO AUTOM &  23125170 &   Simulación 3 &   d01 &      2807 &     3103 &  3114.151 &    -0.072 \\
      SAN CAYETANO AUTOM &  23125170 &   Simulación 5 &   d01 &      2807 &     3103 &  2826.004 &     1.800 \\
      SAN CAYETANO AUTOM &  23125170 &   Simulación 5 &   d02 &      2807 &     3103 &  2868.847 &     1.522 \\
      SAN CAYETANO AUTOM &  23125170 &   Simulación 6 &   d01 &      2807 &     3103 &  2878.447 &     1.460 \\
      SAN CAYETANO AUTOM &  23125170 &   Simulación 6 &   d02 &      2807 &     3103 &  2998.833 &     0.677 \\
      SAN CAYETANO AUTOM &  23125170 &  Simulación 10 &   d01 &      2807 &     3103 &  2959.109 &     0.935 \\
      SAN CAYETANO AUTOM &  23125170 &  Simulación 10 &   d02 &      2807 &     3103 &  2952.493 &     0.978 \\
      SAN CAYETANO AUTOM &  23125170 &  Simulación 10 &   d03 &      2807 &     3103 &  3050.688 &     0.340 \\
      SAN CAYETANO AUTOM &  23125170 &   Simulación 7 &   d01 &      2807 &     3103 &  3092.741 &     0.067 \\
      SAN CAYETANO AUTOM &  23125170 &   Simulación 7 &   d02 &      2807 &     3103 &  3026.408 &     0.498 \\
      SAN CAYETANO AUTOM &  23125170 &   Simulación 8 &   d01 &      2807 &     3103 &  2933.723 &     1.100 \\
      SAN CAYETANO AUTOM &  23125170 &   Simulación 8 &   d02 &      2807 &     3103 &  2852.890 &     1.626 \\
      SAN CAYETANO AUTOM &  23125170 &   Simulación 9 &   d01 &      2807 &     3103 &  2804.042 &     1.943 \\
      SAN CAYETANO AUTOM &  23125170 &   Simulación 9 &   d02 &      2807 &     3103 &  2992.384 &     0.719 \\
    LA BOYERA AUTOMATICA &  24015110 &   Simulación 4 &   d01 &      2610 &     2640 &  2897.148 &    -1.671 \\
    LA BOYERA AUTOMATICA &  24015110 &   Simulación 4 &   d02 &      2610 &     2640 &  2853.154 &    -1.385 \\
    LA BOYERA AUTOMATICA &  24015110 &   Simulación 1 &   d01 &      2610 &     2640 &  2943.492 &    -1.973 \\
    LA BOYERA AUTOMATICA &  24015110 &   Simulación 1 &   d02 &      2610 &     2640 &  2897.148 &    -1.671 \\
    LA BOYERA AUTOMATICA &  24015110 &   Simulación 1 &   d03 &      2610 &     2640 &  2853.241 &    -1.386 \\
    LA BOYERA AUTOMATICA &  24015110 &   Simulación 2 &   d01 &      2610 &     2640 &  2782.875 &    -0.929 \\
    LA BOYERA AUTOMATICA &  24015110 &   Simulación 2 &   d02 &      2610 &     2640 &  2890.346 &    -1.627 \\
    LA BOYERA AUTOMATICA &  24015110 &   Simulación 2 &   d03 &      2610 &     2640 &  2836.159 &    -1.275 \\
    LA BOYERA AUTOMATICA &  24015110 &   Simulación 3 &   d01 &      2610 &     2640 &  2832.534 &    -1.251 \\
    LA BOYERA AUTOMATICA &  24015110 &   Simulación 5 &   d01 &      2610 &     2640 &  2951.611 &    -2.025 \\
    LA BOYERA AUTOMATICA &  24015110 &   Simulación 5 &   d02 &      2610 &     2640 &  2804.781 &    -1.071 \\
    LA BOYERA AUTOMATICA &  24015110 &   Simulación 6 &   d01 &      2610 &     2640 &  2994.189 &    -2.302 \\
    LA BOYERA AUTOMATICA &  24015110 &   Simulación 6 &   d02 &      2610 &     2640 &  2722.666 &    -0.537 \\
    LA BOYERA AUTOMATICA &  24015110 &  Simulación 10 &   d01 &      2610 &     2640 &  2908.321 &    -1.744 \\
    LA BOYERA AUTOMATICA &  24015110 &  Simulación 10 &   d02 &      2610 &     2640 &  2776.358 &    -0.886 \\
    LA BOYERA AUTOMATICA &  24015110 &  Simulación 10 &   d03 &      2610 &     2640 &  2732.615 &    -0.602 \\
    LA BOYERA AUTOMATICA &  24015110 &   Simulación 7 &   d01 &      2610 &     2640 &  2837.391 &    -1.283 \\
    LA BOYERA AUTOMATICA &  24015110 &   Simulación 7 &   d02 &      2610 &     2640 &  2717.053 &    -0.501 \\
    LA BOYERA AUTOMATICA &  24015110 &   Simulación 8 &   d01 &      2610 &     2640 &  2900.612 &    -1.694 \\
    LA BOYERA AUTOMATICA &  24015110 &   Simulación 8 &   d02 &      2610 &     2640 &  2687.858 &    -0.311 \\
    LA BOYERA AUTOMATICA &  24015110 &   Simulación 9 &   d01 &      2610 &     2640 &  2891.396 &    -1.634 \\
    LA BOYERA AUTOMATICA &  24015110 &   Simulación 9 &   d02 &      2610 &     2640 &  2718.785 &    -0.512 \\
            EL ALAMBRADO &  26127010 &   Simulación 4 &   d01 &       151 &     3240 &  3198.655 &     0.269 \\
            EL ALAMBRADO &  26127010 &   Simulación 4 &   d02 &       151 &     3240 &  3145.245 &     0.616 \\
            EL ALAMBRADO &  26127010 &   Simulación 1 &   d01 &       151 &     3240 &  2965.059 &     1.787 \\
            EL ALAMBRADO &  26127010 &   Simulación 1 &   d02 &       151 &     3240 &  3198.655 &     0.269 \\
            EL ALAMBRADO &  26127010 &   Simulación 1 &   d03 &       151 &     3240 &  3145.250 &     0.616 \\
            EL ALAMBRADO &  26127010 &   Simulación 2 &   d01 &       151 &     3240 &  2794.138 &     2.898 \\
            EL ALAMBRADO &  26127010 &   Simulación 2 &   d02 &       151 &     3240 &  3353.045 &    -0.735 \\
            EL ALAMBRADO &  26127010 &   Simulación 2 &   d03 &       151 &     3240 &  3156.968 &     0.540 \\
            EL ALAMBRADO &  26127010 &   Simulación 3 &   d01 &       151 &     3240 &  3193.222 &     0.304 \\
            EL ALAMBRADO &  26127010 &   Simulación 5 &   d01 &       151 &     3240 &  3353.524 &    -0.738 \\
            EL ALAMBRADO &  26127010 &   Simulación 5 &   d02 &       151 &     3240 &  3169.800 &     0.456 \\
            EL ALAMBRADO &  26127010 &   Simulación 6 &   d01 &       151 &     3240 &  3294.258 &    -0.353 \\
            EL ALAMBRADO &  26127010 &   Simulación 6 &   d02 &       151 &     3240 &  3148.286 &     0.596 \\
            EL ALAMBRADO &  26127010 &  Simulación 10 &   d01 &       151 &     3240 &  3151.137 &     0.578 \\
            EL ALAMBRADO &  26127010 &  Simulación 10 &   d02 &       151 &     3240 &  3140.147 &     0.649 \\
            EL ALAMBRADO &  26127010 &  Simulación 10 &   d03 &       151 &     3240 &  3192.974 &     0.306 \\
            EL ALAMBRADO &  26127010 &   Simulación 7 &   d01 &       151 &     3240 &  3366.127 &    -0.820 \\
            EL ALAMBRADO &  26127010 &   Simulación 7 &   d02 &       151 &     3240 &  3165.527 &     0.484 \\
            EL ALAMBRADO &  26127010 &   Simulación 8 &   d01 &       151 &     3240 &  3208.613 &     0.204 \\
            EL ALAMBRADO &  26127010 &   Simulación 8 &   d02 &       151 &     3240 &  3017.176 &     1.448 \\
            EL ALAMBRADO &  26127010 &   Simulación 9 &   d01 &       151 &     3240 &  2957.104 &     1.839 \\
            EL ALAMBRADO &  26127010 &   Simulación 9 &   d02 &       151 &     3240 &  3134.054 &     0.689 \\
      PNN CHINGAZA AUTOM &  35025080 &   Simulación 4 &   d01 &      3205 &     3205 &  3245.511 &    -0.263 \\
      PNN CHINGAZA AUTOM &  35025080 &   Simulación 4 &   d02 &      3205 &     3205 &  3348.540 &    -0.933 \\
      PNN CHINGAZA AUTOM &  35025080 &   Simulación 1 &   d01 &      3205 &     3205 &  2964.804 &     1.561 \\
      PNN CHINGAZA AUTOM &  35025080 &   Simulación 1 &   d02 &      3205 &     3205 &  3245.511 &    -0.263 \\
      PNN CHINGAZA AUTOM &  35025080 &   Simulación 1 &   d03 &      3205 &     3205 &  3348.554 &    -0.933 \\
      PNN CHINGAZA AUTOM &  35025080 &   Simulación 2 &   d01 &      3205 &     3205 &  2586.547 &     4.020 \\
      PNN CHINGAZA AUTOM &  35025080 &   Simulación 2 &   d02 &      3205 &     3205 &  3199.569 &     0.035 \\
      PNN CHINGAZA AUTOM &  35025080 &   Simulación 2 &   d03 &      3205 &     3205 &  3355.040 &    -0.975 \\
      PNN CHINGAZA AUTOM &  35025080 &   Simulación 3 &   d01 &      3205 &     3205 &  3147.438 &     0.374 \\
      PNN CHINGAZA AUTOM &  35025080 &   Simulación 5 &   d01 &      3205 &     3205 &  3072.658 &     0.860 \\
      PNN CHINGAZA AUTOM &  35025080 &   Simulación 5 &   d02 &      3205 &     3205 &  3152.897 &     0.339 \\
      PNN CHINGAZA AUTOM &  35025080 &   Simulación 6 &   d01 &      3205 &     3205 &  3281.018 &    -0.494 \\
      PNN CHINGAZA AUTOM &  35025080 &   Simulación 6 &   d02 &      3205 &     3205 &  3390.819 &    -1.208 \\
      PNN CHINGAZA AUTOM &  35025080 &  Simulación 10 &   d01 &      3205 &     3205 &  3142.409 &     0.407 \\
      PNN CHINGAZA AUTOM &  35025080 &  Simulación 10 &   d02 &      3205 &     3205 &  3046.121 &     1.033 \\
      PNN CHINGAZA AUTOM &  35025080 &  Simulación 10 &   d03 &      3205 &     3205 &  3221.467 &    -0.107 \\
      PNN CHINGAZA AUTOM &  35025080 &   Simulación 7 &   d01 &      3205 &     3205 &  3176.841 &     0.183 \\
      PNN CHINGAZA AUTOM &  35025080 &   Simulación 7 &   d02 &      3205 &     3205 &  2952.147 &     1.644 \\
      PNN CHINGAZA AUTOM &  35025080 &   Simulación 8 &   d01 &      3205 &     3205 &  2983.209 &     1.442 \\
      PNN CHINGAZA AUTOM &  35025080 &   Simulación 8 &   d02 &      3205 &     3205 &  3032.782 &     1.119 \\
      PNN CHINGAZA AUTOM &  35025080 &   Simulación 9 &   d01 &      3205 &     3205 &  3046.922 &     1.028 \\
      PNN CHINGAZA AUTOM &  35025080 &   Simulación 9 &   d02 &      3205 &     3205 &  3177.387 &     0.179 \\
      BOSQUE INTERVENIDO &  35025090 &   Simulación 4 &   d01 &      2944 &     2919 &  3245.511 &    -2.122 \\
      BOSQUE INTERVENIDO &  35025090 &   Simulación 4 &   d02 &      2944 &     2919 &  2991.931 &    -0.474 \\
      BOSQUE INTERVENIDO &  35025090 &   Simulación 1 &   d01 &      2944 &     2919 &  2964.804 &    -0.298 \\
      BOSQUE INTERVENIDO &  35025090 &   Simulación 1 &   d02 &      2944 &     2919 &  3245.511 &    -2.122 \\
      BOSQUE INTERVENIDO &  35025090 &   Simulación 1 &   d03 &      2944 &     2919 &  2991.958 &    -0.474 \\
      BOSQUE INTERVENIDO &  35025090 &   Simulación 2 &   d01 &      2944 &     2919 &  2586.547 &     2.161 \\
      BOSQUE INTERVENIDO &  35025090 &   Simulación 2 &   d02 &      2944 &     2919 &  3199.569 &    -1.824 \\
      BOSQUE INTERVENIDO &  35025090 &   Simulación 2 &   d03 &      2944 &     2919 &  2867.174 &     0.337 \\
      BOSQUE INTERVENIDO &  35025090 &   Simulación 3 &   d01 &      2944 &     2919 &  2737.137 &     1.182 \\
      BOSQUE INTERVENIDO &  35025090 &   Simulación 5 &   d01 &      2944 &     2919 &  3072.658 &    -0.999 \\
      BOSQUE INTERVENIDO &  35025090 &   Simulación 5 &   d02 &      2944 &     2919 &  2829.975 &     0.579 \\
      BOSQUE INTERVENIDO &  35025090 &   Simulación 6 &   d01 &      2944 &     2919 &  3105.296 &    -1.211 \\
      BOSQUE INTERVENIDO &  35025090 &   Simulación 6 &   d02 &      2944 &     2919 &  2934.097 &    -0.098 \\
      BOSQUE INTERVENIDO &  35025090 &  Simulación 10 &   d01 &      2944 &     2919 &  3142.409 &    -1.452 \\
      BOSQUE INTERVENIDO &  35025090 &  Simulación 10 &   d02 &      2944 &     2919 &  3046.121 &    -0.826 \\
      BOSQUE INTERVENIDO &  35025090 &  Simulación 10 &   d03 &      2944 &     2919 &  2820.849 &     0.638 \\
      BOSQUE INTERVENIDO &  35025090 &   Simulación 7 &   d01 &      2944 &     2919 &  3176.841 &    -1.676 \\
      BOSQUE INTERVENIDO &  35025090 &   Simulación 7 &   d02 &      2944 &     2919 &  3049.822 &    -0.850 \\
      BOSQUE INTERVENIDO &  35025090 &   Simulación 8 &   d01 &      2944 &     2919 &  2983.209 &    -0.417 \\
      BOSQUE INTERVENIDO &  35025090 &   Simulación 8 &   d02 &      2944 &     2919 &  3032.782 &    -0.740 \\
      BOSQUE INTERVENIDO &  35025090 &   Simulación 9 &   d01 &      2944 &     2919 &  3046.922 &    -0.831 \\
      BOSQUE INTERVENIDO &  35025090 &   Simulación 9 &   d02 &      2944 &     2919 &  3177.387 &    -1.680 \\
          CALOSTROS BAJO &  35025100 &   Simulación 4 &   d01 &      2943 &     3421 &  3350.205 &     0.460 \\
          CALOSTROS BAJO &  35025100 &   Simulación 4 &   d02 &      2943 &     3421 &  3437.023 &    -0.104 \\
          CALOSTROS BAJO &  35025100 &   Simulación 1 &   d01 &      2943 &     3421 &  3026.181 &     2.566 \\
          CALOSTROS BAJO &  35025100 &   Simulación 1 &   d02 &      2943 &     3421 &  3350.205 &     0.460 \\
          CALOSTROS BAJO &  35025100 &   Simulación 1 &   d03 &      2943 &     3421 &  3437.016 &    -0.104 \\
          CALOSTROS BAJO &  35025100 &   Simulación 2 &   d01 &      2943 &     3421 &  2586.547 &     5.424 \\
          CALOSTROS BAJO &  35025100 &   Simulación 2 &   d02 &      2943 &     3421 &  3199.569 &     1.439 \\
          CALOSTROS BAJO &  35025100 &   Simulación 2 &   d03 &      2943 &     3421 &  3355.040 &     0.429 \\
          CALOSTROS BAJO &  35025100 &   Simulación 3 &   d01 &      2943 &     3421 &  3535.234 &    -0.743 \\
          CALOSTROS BAJO &  35025100 &   Simulación 5 &   d01 &      2943 &     3421 &  3072.658 &     2.264 \\
          CALOSTROS BAJO &  35025100 &   Simulación 5 &   d02 &      2943 &     3421 &  3217.667 &     1.322 \\
          CALOSTROS BAJO &  35025100 &   Simulación 6 &   d01 &      2943 &     3421 &  3281.018 &     0.910 \\
          CALOSTROS BAJO &  35025100 &   Simulación 6 &   d02 &      2943 &     3421 &  3438.448 &    -0.113 \\
          CALOSTROS BAJO &  35025100 &  Simulación 10 &   d01 &      2943 &     3421 &  3137.918 &     1.840 \\
          CALOSTROS BAJO &  35025100 &  Simulación 10 &   d02 &      2943 &     3421 &  3486.659 &    -0.427 \\
          CALOSTROS BAJO &  35025100 &  Simulación 10 &   d03 &      2943 &     3421 &  3494.644 &    -0.479 \\
          CALOSTROS BAJO &  35025100 &   Simulación 7 &   d01 &      2943 &     3421 &  3176.841 &     1.587 \\
          CALOSTROS BAJO &  35025100 &   Simulación 7 &   d02 &      2943 &     3421 &  3460.696 &    -0.258 \\
          CALOSTROS BAJO &  35025100 &   Simulación 8 &   d01 &      2943 &     3421 &  2983.209 &     2.846 \\
          CALOSTROS BAJO &  35025100 &   Simulación 8 &   d02 &      2943 &     3421 &  3032.782 &     2.523 \\
          CALOSTROS BAJO &  35025100 &   Simulación 9 &   d01 &      2943 &     3421 &  3046.922 &     2.432 \\
          CALOSTROS BAJO &  35025100 &   Simulación 9 &   d02 &      2943 &     3421 &  3177.387 &     1.583 \\
         PLAZA DE FERIAS &  35027001 &   Simulación 4 &   d01 &      1670 &     1677 &  1911.562 &    -1.525 \\
         PLAZA DE FERIAS &  35027001 &   Simulación 4 &   d02 &      1670 &     1677 &  1667.538 &     0.062 \\
         PLAZA DE FERIAS &  35027001 &   Simulación 1 &   d01 &      1670 &     1677 &  2666.027 &    -6.429 \\
         PLAZA DE FERIAS &  35027001 &   Simulación 1 &   d02 &      1670 &     1677 &  1911.562 &    -1.525 \\
         PLAZA DE FERIAS &  35027001 &   Simulación 1 &   d03 &      1670 &     1677 &  1667.532 &     0.062 \\
         PLAZA DE FERIAS &  35027001 &   Simulación 2 &   d01 &      1670 &     1677 &  2794.138 &    -7.261 \\
         PLAZA DE FERIAS &  35027001 &   Simulación 2 &   d02 &      1670 &     1677 &  2299.305 &    -4.045 \\
         PLAZA DE FERIAS &  35027001 &   Simulación 2 &   d03 &      1670 &     1677 &  2007.094 &    -2.146 \\
         PLAZA DE FERIAS &  35027001 &   Simulación 3 &   d01 &      1670 &     1677 &  1946.125 &    -1.749 \\
         PLAZA DE FERIAS &  35027001 &   Simulación 5 &   d01 &      1670 &     1677 &  2192.359 &    -3.350 \\
         PLAZA DE FERIAS &  35027001 &   Simulación 5 &   d02 &      1670 &     1677 &  1810.377 &    -0.867 \\
         PLAZA DE FERIAS &  35027001 &   Simulación 6 &   d01 &      1670 &     1677 &  2245.520 &    -3.695 \\
         PLAZA DE FERIAS &  35027001 &   Simulación 6 &   d02 &      1670 &     1677 &  1747.021 &    -0.455 \\
         PLAZA DE FERIAS &  35027001 &  Simulación 10 &   d01 &      1670 &     1677 &  2398.082 &    -4.687 \\
         PLAZA DE FERIAS &  35027001 &  Simulación 10 &   d02 &      1670 &     1677 &  1887.008 &    -1.365 \\
         PLAZA DE FERIAS &  35027001 &  Simulación 10 &   d03 &      1670 &     1677 &  1718.875 &    -0.272 \\
         PLAZA DE FERIAS &  35027001 &   Simulación 7 &   d01 &      1670 &     1677 &  2199.944 &    -3.399 \\
         PLAZA DE FERIAS &  35027001 &   Simulación 7 &   d02 &      1670 &     1677 &  2070.834 &    -2.560 \\
         PLAZA DE FERIAS &  35027001 &   Simulación 8 &   d01 &      1670 &     1677 &  2266.078 &    -3.829 \\
         PLAZA DE FERIAS &  35027001 &   Simulación 8 &   d02 &      1670 &     1677 &  1948.014 &    -1.762 \\
         PLAZA DE FERIAS &  35027001 &   Simulación 9 &   d01 &      1670 &     1677 &  2486.538 &    -5.262 \\
         PLAZA DE FERIAS &  35027001 &   Simulación 9 &   d02 &      1670 &     1677 &  2039.684 &    -2.357 \\
     PARQUE RAFAEL NUÑEZ &  35027002 &   Simulación 4 &   d01 &      1671 &     1686 &  1911.562 &    -1.466 \\
     PARQUE RAFAEL NUÑEZ &  35027002 &   Simulación 4 &   d02 &      1671 &     1686 &  1667.538 &     0.120 \\
     PARQUE RAFAEL NUÑEZ &  35027002 &   Simulación 1 &   d01 &      1671 &     1686 &  2666.027 &    -6.370 \\
     PARQUE RAFAEL NUÑEZ &  35027002 &   Simulación 1 &   d02 &      1671 &     1686 &  1911.562 &    -1.466 \\
     PARQUE RAFAEL NUÑEZ &  35027002 &   Simulación 1 &   d03 &      1671 &     1686 &  1667.532 &     0.120 \\
     PARQUE RAFAEL NUÑEZ &  35027002 &   Simulación 2 &   d01 &      1671 &     1686 &  2794.138 &    -7.203 \\
     PARQUE RAFAEL NUÑEZ &  35027002 &   Simulación 2 &   d02 &      1671 &     1686 &  2299.305 &    -3.986 \\
     PARQUE RAFAEL NUÑEZ &  35027002 &   Simulación 2 &   d03 &      1671 &     1686 &  2007.094 &    -2.087 \\
     PARQUE RAFAEL NUÑEZ &  35027002 &   Simulación 3 &   d01 &      1671 &     1686 &  1649.373 &     0.238 \\
     PARQUE RAFAEL NUÑEZ &  35027002 &   Simulación 5 &   d01 &      1671 &     1686 &  2192.359 &    -3.291 \\
     PARQUE RAFAEL NUÑEZ &  35027002 &   Simulación 5 &   d02 &      1671 &     1686 &  1810.377 &    -0.808 \\
     PARQUE RAFAEL NUÑEZ &  35027002 &   Simulación 6 &   d01 &      1671 &     1686 &  2245.520 &    -3.637 \\
     PARQUE RAFAEL NUÑEZ &  35027002 &   Simulación 6 &   d02 &      1671 &     1686 &  1747.021 &    -0.397 \\
     PARQUE RAFAEL NUÑEZ &  35027002 &  Simulación 10 &   d01 &      1671 &     1686 &  2398.082 &    -4.629 \\
     PARQUE RAFAEL NUÑEZ &  35027002 &  Simulación 10 &   d02 &      1671 &     1686 &  1887.008 &    -1.307 \\
     PARQUE RAFAEL NUÑEZ &  35027002 &  Simulación 10 &   d03 &      1671 &     1686 &  1718.875 &    -0.214 \\
     PARQUE RAFAEL NUÑEZ &  35027002 &   Simulación 7 &   d01 &      1671 &     1686 &  2199.944 &    -3.341 \\
     PARQUE RAFAEL NUÑEZ &  35027002 &   Simulación 7 &   d02 &      1671 &     1686 &  2070.834 &    -2.501 \\
     PARQUE RAFAEL NUÑEZ &  35027002 &   Simulación 8 &   d01 &      1671 &     1686 &  2266.078 &    -3.771 \\
     PARQUE RAFAEL NUÑEZ &  35027002 &   Simulación 8 &   d02 &      1671 &     1686 &  1948.014 &    -1.703 \\
     PARQUE RAFAEL NUÑEZ &  35027002 &   Simulación 9 &   d01 &      1671 &     1686 &  2486.538 &    -5.203 \\
     PARQUE RAFAEL NUÑEZ &  35027002 &   Simulación 9 &   d02 &      1671 &     1686 &  2039.684 &    -2.299 \\
          CALOSTROS BAJO &  35027510 &   Simulación 4 &   d01 &      2943 &     3421 &  3350.205 &     0.460 \\
          CALOSTROS BAJO &  35027510 &   Simulación 4 &   d02 &      2943 &     3421 &  3437.023 &    -0.104 \\
          CALOSTROS BAJO &  35027510 &   Simulación 1 &   d01 &      2943 &     3421 &  3026.181 &     2.566 \\
          CALOSTROS BAJO &  35027510 &   Simulación 1 &   d02 &      2943 &     3421 &  3350.205 &     0.460 \\
          CALOSTROS BAJO &  35027510 &   Simulación 1 &   d03 &      2943 &     3421 &  3437.016 &    -0.104 \\
          CALOSTROS BAJO &  35027510 &   Simulación 2 &   d01 &      2943 &     3421 &  2586.547 &     5.424 \\
          CALOSTROS BAJO &  35027510 &   Simulación 2 &   d02 &      2943 &     3421 &  3199.569 &     1.439 \\
          CALOSTROS BAJO &  35027510 &   Simulación 2 &   d03 &      2943 &     3421 &  3355.040 &     0.429 \\
          CALOSTROS BAJO &  35027510 &   Simulación 3 &   d01 &      2943 &     3421 &  3535.234 &    -0.743 \\
          CALOSTROS BAJO &  35027510 &   Simulación 5 &   d01 &      2943 &     3421 &  3072.658 &     2.264 \\
          CALOSTROS BAJO &  35027510 &   Simulación 5 &   d02 &      2943 &     3421 &  3217.667 &     1.322 \\
          CALOSTROS BAJO &  35027510 &   Simulación 6 &   d01 &      2943 &     3421 &  3281.018 &     0.910 \\
          CALOSTROS BAJO &  35027510 &   Simulación 6 &   d02 &      2943 &     3421 &  3438.448 &    -0.113 \\
          CALOSTROS BAJO &  35027510 &  Simulación 10 &   d01 &      2943 &     3421 &  3137.918 &     1.840 \\
          CALOSTROS BAJO &  35027510 &  Simulación 10 &   d02 &      2943 &     3421 &  3486.659 &    -0.427 \\
          CALOSTROS BAJO &  35027510 &  Simulación 10 &   d03 &      2943 &     3421 &  3494.644 &    -0.479 \\
          CALOSTROS BAJO &  35027510 &   Simulación 7 &   d01 &      2943 &     3421 &  3176.841 &     1.587 \\
          CALOSTROS BAJO &  35027510 &   Simulación 7 &   d02 &      2943 &     3421 &  3460.696 &    -0.258 \\
          CALOSTROS BAJO &  35027510 &   Simulación 8 &   d01 &      2943 &     3421 &  2983.209 &     2.846 \\
          CALOSTROS BAJO &  35027510 &   Simulación 8 &   d02 &      2943 &     3421 &  3032.782 &     2.523 \\
          CALOSTROS BAJO &  35027510 &   Simulación 9 &   d01 &      2943 &     3421 &  3046.922 &     2.432 \\
          CALOSTROS BAJO &  35027510 &   Simulación 9 &   d02 &      2943 &     3421 &  3177.387 &     1.583 \\
            PMO CHINGAZA &  35035130 &   Simulación 4 &   d01 &      3863 &     3856 &  3350.205 &     3.288 \\
            PMO CHINGAZA &  35035130 &   Simulación 4 &   d02 &      3863 &     3856 &  3640.489 &     1.401 \\
            PMO CHINGAZA &  35035130 &   Simulación 1 &   d01 &      3863 &     3856 &  3026.181 &     5.394 \\
            PMO CHINGAZA &  35035130 &   Simulación 1 &   d02 &      3863 &     3856 &  3350.205 &     3.288 \\
            PMO CHINGAZA &  35035130 &   Simulación 1 &   d03 &      3863 &     3856 &  3640.490 &     1.401 \\
            PMO CHINGAZA &  35035130 &   Simulación 2 &   d01 &      3863 &     3856 &  2586.547 &     8.251 \\
            PMO CHINGAZA &  35035130 &   Simulación 2 &   d02 &      3863 &     3856 &  3199.569 &     4.267 \\
            PMO CHINGAZA &  35035130 &   Simulación 2 &   d03 &      3863 &     3856 &  3437.383 &     2.721 \\
            PMO CHINGAZA &  35035130 &   Simulación 3 &   d01 &      3863 &     3856 &  3589.144 &     1.735 \\
            PMO CHINGAZA &  35035130 &   Simulación 5 &   d01 &      3863 &     3856 &  3160.177 &     4.523 \\
            PMO CHINGAZA &  35035130 &   Simulación 5 &   d02 &      3863 &     3856 &  3570.430 &     1.856 \\
            PMO CHINGAZA &  35035130 &   Simulación 6 &   d01 &      3863 &     3856 &  3281.018 &     3.737 \\
            PMO CHINGAZA &  35035130 &   Simulación 6 &   d02 &      3863 &     3856 &  3523.152 &     2.164 \\
            PMO CHINGAZA &  35035130 &  Simulación 10 &   d01 &      3863 &     3856 &  3137.918 &     4.668 \\
            PMO CHINGAZA &  35035130 &  Simulación 10 &   d02 &      3863 &     3856 &  3527.375 &     2.136 \\
            PMO CHINGAZA &  35035130 &  Simulación 10 &   d03 &      3863 &     3856 &  3614.133 &     1.572 \\
            PMO CHINGAZA &  35035130 &   Simulación 7 &   d01 &      3863 &     3856 &  3176.841 &     4.415 \\
            PMO CHINGAZA &  35035130 &   Simulación 7 &   d02 &      3863 &     3856 &  3498.221 &     2.326 \\
            PMO CHINGAZA &  35035130 &   Simulación 8 &   d01 &      3863 &     3856 &  3048.488 &     5.249 \\
            PMO CHINGAZA &  35035130 &   Simulación 8 &   d02 &      3863 &     3856 &  3494.923 &     2.347 \\
            PMO CHINGAZA &  35035130 &   Simulación 9 &   d01 &      3863 &     3856 &  3046.922 &     5.259 \\
            PMO CHINGAZA &  35035130 &   Simulación 9 &   d02 &      3863 &     3856 &  3381.471 &     3.084 \\
    CHINAVITA AUTOMATICA &  35075070 &   Simulación 4 &   d01 &      2012 &     1938 &  2451.384 &    -3.337 \\
    CHINAVITA AUTOMATICA &  35075070 &   Simulación 4 &   d02 &      2012 &     1938 &  2045.217 &    -0.697 \\
    CHINAVITA AUTOMATICA &  35075070 &   Simulación 1 &   d01 &      2012 &     1938 &  2397.799 &    -2.989 \\
    CHINAVITA AUTOMATICA &  35075070 &   Simulación 1 &   d02 &      2012 &     1938 &  2451.384 &    -3.337 \\
    CHINAVITA AUTOMATICA &  35075070 &   Simulación 1 &   d03 &      2012 &     1938 &  2045.217 &    -0.697 \\
    CHINAVITA AUTOMATICA &  35075070 &   Simulación 2 &   d01 &      2012 &     1938 &  2576.131 &    -4.148 \\
    CHINAVITA AUTOMATICA &  35075070 &   Simulación 2 &   d02 &      2012 &     1938 &  2420.755 &    -3.138 \\
    CHINAVITA AUTOMATICA &  35075070 &   Simulación 2 &   d03 &      2012 &     1938 &  2189.530 &    -1.635 \\
    CHINAVITA AUTOMATICA &  35075070 &   Simulación 3 &   d01 &      2012 &     1938 &  2052.186 &    -0.742 \\
    CHINAVITA AUTOMATICA &  35075070 &   Simulación 5 &   d01 &      2012 &     1938 &  2456.888 &    -3.373 \\
    CHINAVITA AUTOMATICA &  35075070 &   Simulación 5 &   d02 &      2012 &     1938 &  2155.263 &    -1.412 \\
    CHINAVITA AUTOMATICA &  35075070 &   Simulación 6 &   d01 &      2012 &     1938 &  2456.307 &    -3.369 \\
    CHINAVITA AUTOMATICA &  35075070 &   Simulación 6 &   d02 &      2012 &     1938 &  2047.262 &    -0.710 \\
    CHINAVITA AUTOMATICA &  35075070 &  Simulación 10 &   d01 &      2012 &     1938 &  2438.694 &    -3.255 \\
    CHINAVITA AUTOMATICA &  35075070 &  Simulación 10 &   d02 &      2012 &     1938 &  2034.361 &    -0.626 \\
    CHINAVITA AUTOMATICA &  35075070 &  Simulación 10 &   d03 &      2012 &     1938 &  1975.035 &    -0.241 \\
    CHINAVITA AUTOMATICA &  35075070 &   Simulación 7 &   d01 &      2012 &     1938 &  2416.242 &    -3.109 \\
    CHINAVITA AUTOMATICA &  35075070 &   Simulación 7 &   d02 &      2012 &     1938 &  2325.706 &    -2.520 \\
    CHINAVITA AUTOMATICA &  35075070 &   Simulación 8 &   d01 &      2012 &     1938 &  2319.919 &    -2.482 \\
    CHINAVITA AUTOMATICA &  35075070 &   Simulación 8 &   d02 &      2012 &     1938 &  2380.736 &    -2.878 \\
    CHINAVITA AUTOMATICA &  35075070 &   Simulación 9 &   d01 &      2012 &     1938 &  2264.817 &    -2.124 \\
    CHINAVITA AUTOMATICA &  35075070 &   Simulación 9 &   d02 &      2012 &     1938 &  2259.207 &    -2.088 \\
  PMO RABANAL AUTOMATICA &  35075080 &   Simulación 4 &   d01 &      3398 &     3379 &  3180.479 &     1.290 \\
  PMO RABANAL AUTOMATICA &  35075080 &   Simulación 4 &   d02 &      3398 &     3379 &  3306.698 &     0.470 \\
  PMO RABANAL AUTOMATICA &  35075080 &   Simulación 1 &   d01 &      3398 &     3379 &  2815.553 &     3.662 \\
  PMO RABANAL AUTOMATICA &  35075080 &   Simulación 1 &   d02 &      3398 &     3379 &  3180.479 &     1.290 \\
  PMO RABANAL AUTOMATICA &  35075080 &   Simulación 1 &   d03 &      3398 &     3379 &  3306.734 &     0.470 \\
  PMO RABANAL AUTOMATICA &  35075080 &   Simulación 2 &   d01 &      3398 &     3379 &  2576.131 &     5.219 \\
  PMO RABANAL AUTOMATICA &  35075080 &   Simulación 2 &   d02 &      3398 &     3379 &  2847.845 &     3.453 \\
  PMO RABANAL AUTOMATICA &  35075080 &   Simulación 2 &   d03 &      3398 &     3379 &  3238.792 &     0.911 \\
  PMO RABANAL AUTOMATICA &  35075080 &   Simulación 3 &   d01 &      3398 &     3379 &  3246.405 &     0.862 \\
  PMO RABANAL AUTOMATICA &  35075080 &   Simulación 5 &   d01 &      3398 &     3379 &  3044.153 &     2.177 \\
  PMO RABANAL AUTOMATICA &  35075080 &   Simulación 5 &   d02 &      3398 &     3379 &  3293.091 &     0.558 \\
  PMO RABANAL AUTOMATICA &  35075080 &   Simulación 6 &   d01 &      3398 &     3379 &  3034.500 &     2.239 \\
  PMO RABANAL AUTOMATICA &  35075080 &   Simulación 6 &   d02 &      3398 &     3379 &  3387.375 &    -0.054 \\
  PMO RABANAL AUTOMATICA &  35075080 &  Simulación 10 &   d01 &      3398 &     3379 &  2922.391 &     2.968 \\
  PMO RABANAL AUTOMATICA &  35075080 &  Simulación 10 &   d02 &      3398 &     3379 &  3174.693 &     1.328 \\
  PMO RABANAL AUTOMATICA &  35075080 &  Simulación 10 &   d03 &      3398 &     3379 &  3273.300 &     0.687 \\
  PMO RABANAL AUTOMATICA &  35075080 &   Simulación 7 &   d01 &      3398 &     3379 &  2894.892 &     3.147 \\
  PMO RABANAL AUTOMATICA &  35075080 &   Simulación 7 &   d02 &      3398 &     3379 &  3257.963 &     0.787 \\
  PMO RABANAL AUTOMATICA &  35075080 &   Simulación 8 &   d01 &      3398 &     3379 &  2920.071 &     2.983 \\
  PMO RABANAL AUTOMATICA &  35075080 &   Simulación 8 &   d02 &      3398 &     3379 &  3224.170 &     1.006 \\
  PMO RABANAL AUTOMATICA &  35075080 &   Simulación 9 &   d01 &      3398 &     3379 &  2848.822 &     3.446 \\
  PMO RABANAL AUTOMATICA &  35075080 &   Simulación 9 &   d02 &      3398 &     3379 &  3240.228 &     0.902 \\
        LA CAPILLA AUTOM &  35085080 &   Simulación 4 &   d01 &      1917 &     1903 &  1994.294 &    -0.593 \\
        LA CAPILLA AUTOM &  35085080 &   Simulación 4 &   d02 &      1917 &     1903 &  1809.357 &     0.609 \\
        LA CAPILLA AUTOM &  35085080 &   Simulación 1 &   d01 &      1917 &     1903 &  2432.497 &    -3.442 \\
        LA CAPILLA AUTOM &  35085080 &   Simulación 1 &   d02 &      1917 &     1903 &  1994.294 &    -0.593 \\
        LA CAPILLA AUTOM &  35085080 &   Simulación 1 &   d03 &      1917 &     1903 &  1809.363 &     0.609 \\
        LA CAPILLA AUTOM &  35085080 &   Simulación 2 &   d01 &      1917 &     1903 &  2576.131 &    -4.375 \\
        LA CAPILLA AUTOM &  35085080 &   Simulación 2 &   d02 &      1917 &     1903 &  2186.789 &    -1.845 \\
        LA CAPILLA AUTOM &  35085080 &   Simulación 2 &   d03 &      1917 &     1903 &  2165.506 &    -1.706 \\
        LA CAPILLA AUTOM &  35085080 &   Simulación 3 &   d01 &      1917 &     1903 &  1801.648 &     0.659 \\
        LA CAPILLA AUTOM &  35085080 &   Simulación 5 &   d01 &      1917 &     1903 &  1994.528 &    -0.595 \\
        LA CAPILLA AUTOM &  35085080 &   Simulación 5 &   d02 &      1917 &     1903 &  1901.532 &     0.010 \\
        LA CAPILLA AUTOM &  35085080 &   Simulación 6 &   d01 &      1917 &     1903 &  2189.551 &    -1.863 \\
        LA CAPILLA AUTOM &  35085080 &   Simulación 6 &   d02 &      1917 &     1903 &  1821.184 &     0.532 \\
        LA CAPILLA AUTOM &  35085080 &  Simulación 10 &   d01 &      1917 &     1903 &  2308.152 &    -2.633 \\
        LA CAPILLA AUTOM &  35085080 &  Simulación 10 &   d02 &      1917 &     1903 &  2065.917 &    -1.059 \\
        LA CAPILLA AUTOM &  35085080 &  Simulación 10 &   d03 &      1917 &     1903 &  1811.766 &     0.593 \\
        LA CAPILLA AUTOM &  35085080 &   Simulación 7 &   d01 &      1917 &     1903 &  2307.351 &    -2.628 \\
        LA CAPILLA AUTOM &  35085080 &   Simulación 7 &   d02 &      1917 &     1903 &  2248.585 &    -2.246 \\
        LA CAPILLA AUTOM &  35085080 &   Simulación 8 &   d01 &      1917 &     1903 &  2278.959 &    -2.444 \\
        LA CAPILLA AUTOM &  35085080 &   Simulación 8 &   d02 &      1917 &     1903 &  2114.692 &    -1.376 \\
        LA CAPILLA AUTOM &  35085080 &   Simulación 9 &   d01 &      1917 &     1903 &  2202.658 &    -1.948 \\
        LA CAPILLA AUTOM &  35085080 &   Simulación 9 &   d02 &      1917 &     1903 &  2119.346 &    -1.406 \\
        SUBIA AUTOMATICA &  21195160 &   Simulación 4 &   d02 &      2075 &     2080 &  2181.566 &    -0.660 \\
        SUBIA AUTOMATICA &  21195160 &   Simulación 1 &   d01 &      2075 &     2080 &  2432.308 &    -2.290 \\
        SUBIA AUTOMATICA &  21195160 &   Simulación 1 &   d02 &      2075 &     2080 &  2356.274 &    -1.796 \\
        SUBIA AUTOMATICA &  21195160 &   Simulación 1 &   d03 &      2075 &     2080 &  2181.624 &    -0.661 \\
        SUBIA AUTOMATICA &  21195160 &   Simulación 2 &   d01 &      2075 &     2080 &  1760.607 &     2.076 \\
        SUBIA AUTOMATICA &  21195160 &   Simulación 2 &   d02 &      2075 &     2080 &  1869.559 &     1.368 \\
        SUBIA AUTOMATICA &  21195160 &   Simulación 2 &   d03 &      2075 &     2080 &  2197.112 &    -0.761 \\
        SUBIA AUTOMATICA &  21195160 &   Simulación 3 &   d01 &      2075 &     2080 &  2265.318 &    -1.205 \\
        SUBIA AUTOMATICA &  21195160 &   Simulación 5 &   d01 &      2075 &     2080 &  1891.935 &     1.222 \\
        SUBIA AUTOMATICA &  21195160 &   Simulación 5 &   d02 &      2075 &     2080 &  2125.953 &    -0.299 \\
        SUBIA AUTOMATICA &  21195160 &   Simulación 6 &   d01 &      2075 &     2080 &  1696.934 &     2.490 \\
        SUBIA AUTOMATICA &  21195160 &   Simulación 6 &   d02 &      2075 &     2080 &  2131.585 &    -0.335 \\
        SUBIA AUTOMATICA &  21195160 &  Simulación 10 &   d01 &      2075 &     2080 &  1670.562 &     2.661 \\
        SUBIA AUTOMATICA &  21195160 &  Simulación 10 &   d02 &      2075 &     2080 &  2228.518 &    -0.965 \\
        SUBIA AUTOMATICA &  21195160 &  Simulación 10 &   d03 &      2075 &     2080 &  2032.815 &     0.307 \\
        SUBIA AUTOMATICA &  21195160 &   Simulación 7 &   d01 &      2075 &     2080 &  2224.662 &    -0.940 \\
        SUBIA AUTOMATICA &  21195160 &   Simulación 7 &   d02 &      2075 &     2080 &  2125.321 &    -0.295 \\
        SUBIA AUTOMATICA &  21195160 &   Simulación 8 &   d01 &      2075 &     2080 &  1540.349 &     3.508 \\
        SUBIA AUTOMATICA &  21195160 &   Simulación 8 &   d02 &      2075 &     2080 &  2173.068 &    -0.605 \\
        SUBIA AUTOMATICA &  21195160 &   Simulación 9 &   d01 &      2075 &     2080 &  1565.655 &     3.343 \\
        SUBIA AUTOMATICA &  21195160 &   Simulación 9 &   d02 &      2075 &     2080 &  2323.289 &    -1.581 \\
\end{longtable}

\end{landscape}

 % Tercero
%%\appendix

\chapter{Diagramas de Taylor para cada una de las estaciones automáticas, válidas.}
\label{anexo:diag_taylor_estaciones_aut_val}

\centering
\begin{figure}[H]
\centering

\begin{subfigure}[normla]{0.4\textwidth}
\includegraphics[draft=false, scale=0.45]{taylor/taylor_21195160.png}
\caption{Estación SUBIA AUTOMATICA código 21195160.}
\end{subfigure}
~
\begin{subfigure}[normla]{0.4\textwidth}
\includegraphics[draft=false, scale=0.45]{taylor/taylor_21206790.png}
\caption{Estación HDA STA ANA AUTOM código 21206790.}
\end{subfigure}
~
\begin{subfigure}[normla]{0.4\textwidth}
\includegraphics[draft=false, scale=0.45]{taylor/taylor_21206930.png}
\caption{Estación PMO GUERRERO código 21206930.}
\end{subfigure}
~
\begin{subfigure}[normla]{0.4\textwidth}
\includegraphics[draft=false, scale=0.45]{taylor/taylor_21206940.png}
\caption{Estación CIUDAD BOLIVAR código 21206940.}
\end{subfigure}
~
\begin{subfigure}[normla]{0.4\textwidth}
\includegraphics[draft=false, scale=0.45]{taylor/taylor_21206950.png}
\caption{Estación PMO GUACHENEQUE código 21206950.}
\end{subfigure}
~
\begin{subfigure}[normla]{0.4\textwidth}
\includegraphics[draft=false, scale=0.45]{taylor/taylor_21206980.png}
\caption{Estación STA CRUZ DE SIECHA código 21206980.}
\end{subfigure}
~
\end{figure}
           
\begin{figure}[H]\ContinuedFloat
\centering
\begin{subfigure}[normla]{0.4\textwidth}
\includegraphics[draft=false, scale=0.45]{taylor/taylor_21206990.png}
\caption{Estación TIBAITATA AUTOMATICA código 21206990.}
\end{subfigure}
~
\begin{subfigure}[normla]{0.4\textwidth}
\includegraphics[draft=false, scale=0.45]{taylor/taylor_24015110.png}
\caption{Estación LA BOYERA AUTOMATICA código 24015110.}
\end{subfigure}
~
\begin{subfigure}[normla]{0.4\textwidth}
\includegraphics[draft=false, scale=0.45]{taylor/taylor_35075070.png}
\caption{Estación CHINAVITA AUTOMATICA código 35075070.}
\end{subfigure}
~
\begin{subfigure}[normla]{0.4\textwidth}
\includegraphics[draft=false, scale=0.45]{taylor/taylor_35085080.png}
\caption{Estación LA CAPILLA AUTOM código 35085080.}
\end{subfigure}
~

	
	\caption{Diagramas de Taylor de los casos evaluados.}.
	 \label{gra:papa_cund}
\end{figure}
 % Tercero
%\texttt{
%

\chapter{\textit{namelist.input} usadas para las la selección de la mejor combinación parametrización-opción.}

%las namelist WPS están en /home/edwin/Downloads/wps/descargas_20180713/wps/namelist
\label{anexo:namelist_mejor_parametrizacion}

\section{\textit{namelist.input} usadas para la selección de la mejor combinación parametrización-opción del mes de febrero año 2007.}


%%%%% bl_pbl_physics

\textbf{\textit{\textit{namelist.input}}  Combinación bl\_pbl\_physics}

\input{param_opcio_200702/bl_pbl_physics}

\textbf{\textit{namelist.input}  Combinación cu\_physics}

\input{param_opcio_200702/cu_physics}

\textbf{\textit{namelist.input}  Combinación mp\_physics}

\input{param_opcio_200702/mp_physics}

\textbf{\textit{namelist.input}  Combinación ra\_lw\_physics}

\input{param_opcio_200702/ra_lw_physics}

\textbf{\textit{namelist.input}  Combinación ra\_sw\_physics}

\input{param_opcio_200702/ra_sw_physics}

\textbf{\textit{namelist.input}  Combinación sf\_sfclay\_physics}

\input{param_opcio_200702/sf_sfclay_physics}

\textbf{\textit{namelist.input}  Combinación sf\_surface\_physics}

\input{param_opcio_200702/sf_surface_physics}

\textbf{\textit{namelist.input}  Combinación ideam\_colombia}

\input{param_opcio_200702/ideam_colombia}

\textbf{\textit{namelist.input}  Combinación ideam\_bogota}

\input{param_opcio_200702/ideam_bogota}

\textbf{\textit{namelist.input}  Combinación icm\_pbl-5\_cu-0}

\input{param_opcio_200702/mejor}

\textbf{\textit{namelist.input}  Combinación icm\_pbl-5\_cu-5}

\input{param_opcio_200702/cu_physics_5}

\textbf{\textit{namelist.input}  Combinación icm\_pbl-7\_cu-0}

\input{param_opcio_200702/mejor7}

\textbf{\textit{namelist.input}  Combinación icm\_pbl-5\_cu-14}

\input{param_opcio_200702/cu_physics_14}

\textbf{\textit{namelist.input}  Combinación icm\_pbl-5\_cu-14}

\input{param_opcio_200702/cu_14_pbl_5}

\textbf{\textit{namelist.input}  Combinación icm\_pbl-7\_cu-14}

\input{param_opcio_200702/cu_physics_14.tex}



\section{\textit{namelist.input} usadas para la búsqueda de la mejor combinación parametrización-opción del mes de agosto año 2014.}

%%%%% bl_pbl_physics

\textbf{\textit{namelist.input}  Combinación bl\_pbl\_physics}

\input{param_opcio_201408/bl_pbl_physics}

\textbf{\textit{namelist.input}  Combinación cu\_physics}

\input{param_opcio_201408/cu_physics}

\textbf{\textit{namelist.input}  Combinación mp\_physics}

\input{param_opcio_201408/mp_physics}

\textbf{\textit{namelist.input}  Combinación ra\_lw\_physics}

\input{param_opcio_201408/ra_lw_physics}

\textbf{\textit{namelist.input}  Combinación ra\_sw\_physics}

\input{param_opcio_201408/ra_sw_physics}

\textbf{\textit{namelist.input}  Combinación sf\_sfclay\_physics}

\input{param_opcio_201408/sf_sfclay_physics}

\textbf{\textit{namelist.input}  Combinación sf\_surface\_physics}

\input{param_opcio_201408/sf_surface_physics}

\textbf{\textit{namelist.input}  Combinación ideam\_colombia}

\input{param_opcio_201408/ideam_colombia}

\textbf{\textit{namelist.input}  Combinación ideam\_bogota}

\input{param_opcio_201408/ideam_bogota}

\textbf{\textit{namelist.input}  Combinación icm\_pbl-5\_cu-0}

\input{param_opcio_201408/mejor}

\textbf{\textit{namelist.input}  Combinación icm\_pbl-7\_cu-0}

\input{param_opcio_201408/mejor7}

\textbf{\textit{namelist.input}  Combinación icm\_pbl-5\_cu-5}

\input{param_opcio_201408/cu_physics_5}

\textbf{\textit{namelist.input}  Combinación icm\_pbl-5\_cu-14}

\input{param_opcio_201408/cu_physics_14}

\textbf{\textit{namelist.input}  Combinación icm\_pbl-5\_cu-14}

\input{param_opcio_201408/cu_14_pbl_5}

\textbf{\textit{namelist.input}  Combinación icm\_pbl-7\_cu-14}

\input{param_opcio_201408/cu_physics_14.tex}


\section{\textit{namelist.input} usadas para la búsqueda de la mejor combinación parametrización-opción del mes de agosto año 2015.}

%%%%% bl_pbl_physics

\textbf{\textit{namelist.input}  Combinación bl\_pbl\_physics}

\input{param_opcio_201508/bl_pbl_physics}

\textbf{\textit{namelist.input}  Combinación cu\_physics}

\input{param_opcio_201508/cu_physics}

\textbf{\textit{namelist.input}  Combinación mp\_physics}

\input{param_opcio_201508/mp_physics}

\textbf{\textit{namelist.input}  Combinación ra\_lw\_physics}

\input{param_opcio_201508/ra_lw_physics}

\textbf{\textit{namelist.input}  Combinación ra\_sw\_physics}

\input{param_opcio_201508/ra_sw_physics}

\textbf{\textit{namelist.input}  Combinación sf\_sfclay\_physics}

\input{param_opcio_201508/sf_sfclay_physics}

\textbf{\textit{namelist.input}  Combinación sf\_surface\_physics}

\input{param_opcio_201508/sf_surface_physics}

\textbf{\textit{namelist.input}  Combinación ideam\_colombia}

\input{param_opcio_201508/ideam_colombia}

\textbf{\textit{namelist.input}  Combinación ideam\_bogota}

\input{param_opcio_201508/ideam_bogota}

\textbf{\textit{namelist.input}  Combinación icm\_pbl-5\_cu-0}

\input{param_opcio_201508/mejor}

\textbf{\textit{namelist.input}  Combinación icm\_pbl-7\_cu-0}

\input{param_opcio_201508/mejor7}

\textbf{\textit{namelist.input}  Combinación icm\_pbl-5\_cu-5}

\input{param_opcio_201508/cu_physics_5}

\textbf{\textit{namelist.input}  Combinación icm\_pbl-5\_cu-14}

\input{param_opcio_201508/cu_physics_14}

\textbf{\textit{namelist.input}  Combinación icm\_pbl-5\_cu-14}

\input{param_opcio_201508/cu_14_pbl_5}

\textbf{\textit{namelist.input}  Combinación icm\_pbl-7\_cu-14}

\input{param_opcio_201508/cu_physics_14.tex}

\section{\textit{namelist.input} usadas para la búsqueda de la mejor combinación parametrización-opción del mes de septiembre año 2015.}

%%%%% bl_pbl_physics

\textbf{\textit{namelist.input}  Combinación bl\_pbl\_physics}

\input{param_opcio_201509/bl_pbl_physics}

\textbf{\textit{namelist.input}  Combinación cu\_physics}

\input{param_opcio_201509/cu_physics}

\textbf{\textit{namelist.input}  Combinación mp\_physics}

\input{param_opcio_201509/mp_physics}

\textbf{\textit{namelist.input}  Combinación ra\_lw\_physics}

\input{param_opcio_201509/ra_lw_physics}

\textbf{\textit{namelist.input}  Combinación ra\_sw\_physics}

\input{param_opcio_201509/ra_sw_physics}

\textbf{\textit{namelist.input}  Combinación sf\_sfclay\_physics}

\input{param_opcio_201509/sf_sfclay_physics}

\textbf{\textit{namelist.input}  Combinación sf\_surface\_physics}

\input{param_opcio_201509/sf_surface_physics}

\textbf{\textit{namelist.input}  Combinación ideam\_colombia}

\input{param_opcio_201509/ideam_colombia}

\textbf{\textit{namelist.input}  Combinación ideam\_bogota}

\input{param_opcio_201509/ideam_bogota}

\textbf{\textit{namelist.input}  Combinación icm\_pbl-5\_cu-0}

\input{param_opcio_201509/mejor}

\textbf{\textit{namelist.input}  Combinación icm\_pbl-7\_cu-0}

\input{param_opcio_201509/mejor7}

\textbf{\textit{namelist.input}  Combinación icm\_pbl-5\_cu-5}

\input{param_opcio_201509/cu_physics_5}

\textbf{\textit{namelist.input}  Combinación icm\_pbl-5\_cu-14}

\input{param_opcio_201509/cu_physics_14}

\textbf{\textit{namelist.input}  Combinación icm\_pbl-5\_cu-14}

\input{param_opcio_201509/cu_14_pbl_5}

\textbf{\textit{namelist.input}  Combinación icm\_pbl-7\_cu-14}

\input{param_opcio_201509/cu_physics_14.tex} % Cuarto
%}
%

\chapter{Resultados de las comparaciones entre los estadísticos y el modelo para la selección de los tiempos en los diferentes casos.}
\label{anexo:resultados_comparaciones_estadisticos_tiempos}
\begin{landscape}

\textbf{Caso 1}

\begin{longtable}{p{2cm}rrrrrrrrrr}
\caption{Resultados de las diferentes simulaciones.}
\label{tab:estaciones_tiempo_wrf}\\
\hline
   Nombre &  Simulación & Dominio &  Pearson &     RMSE &  $RMSE_{esc}$ &    $Pearson_{esc}$ &      ET \\
   
\midrule
\endhead
\midrule
\multicolumn{3}{r}{{Continúa en la siguiente página.}} \\
\midrule
\endfoot

\bottomrule
\endlastfoot
 
     La Capilla Autom  &          1 &     d01 &   0.9739 &  5.9902 &        0.3010 &           0.7122 &  0.5066 \\
     La Capilla Autom  &          1 &     d02 &   0.9855 &  1.6368 &        0.9844 &           0.9315 &  0.9579 \\
     La Capilla Autom  &          2 &     d01 &   0.9428 &  7.9076 &        0.0000 &           0.1258 &  0.0629 \\
     La Capilla Autom  &          2 &     d02 &   0.9372 &  4.2380 &        0.5761 &           0.0198 &  0.2979 \\
     La Capilla Autom  &          3 &     d01 &   0.9815 &  6.0658 &        0.2891 &           0.8550 &  0.5721 \\
     La Capilla Autom  &          3 &     d02 &   0.9892 &  1.5373 &        1.0000 &           1.0000 &  1.0000 \\
     La Capilla Autom  &          4 &     d01 &   0.9662 &  6.4159 &        0.2342 &           0.5665 &  0.4003 \\
     La Capilla Autom  &          4 &     d02 &   0.9739 &  2.0047 &        0.9266 &           0.7118 &  0.8192 \\
     La Capilla Autom  &          5 &     d01 &   0.9745 &  6.4478 &        0.2292 &           0.7233 &  0.4762 \\
     La Capilla Autom  &          5 &     d02 &   0.9798 &  2.0541 &        0.9189 &           0.8228 &  0.8709 \\
     La Capilla Autom  &          6 &     d01 &   0.9503 &  7.2073 &        0.1099 &           0.2669 &  0.1884 \\
     La Capilla Autom  &          6 &     d02 &   0.9452 &  3.1784 &        0.7424 &           0.1713 &  0.4568 \\
     La Capilla Autom  &          7 &     d01 &   0.9739 &  6.3450 &        0.2453 &           0.7130 &  0.4792 \\
     La Capilla Autom  &          7 &     d02 &   0.9827 &  1.8654 &        0.9485 &           0.8790 &  0.9138 \\
     La Capilla Autom  &          8 &     d01 &   0.9746 &  6.5480 &        0.2134 &           0.7252 &  0.4693 \\
     La Capilla Autom  &          8 &     d02 &   0.9789 &  2.0663 &        0.9170 &           0.8059 &  0.8614 \\
     La Capilla Autom  &          9 &     d01 &   0.9738 &  6.6564 &        0.1964 &           0.7099 &  0.4532 \\
     La Capilla Autom  &          9 &     d02 &   0.9775 &  2.2251 &        0.8920 &           0.7807 &  0.8364 \\
     La Capilla Autom  &         10 &     d01 &   0.9418 &  7.3678 &        0.0847 &           0.1070 &  0.0959 \\
     La Capilla Autom  &         10 &     d02 &   0.9361 &  3.3010 &        0.7231 &           0.0000 &  0.3616 \\
     La Capilla Autom  &         11 &     d01 &   0.9780 &  6.6055 &        0.2044 &           0.7904 &  0.4974 \\
     La Capilla Autom  &         11 &     d02 &   0.9824 &  2.0969 &        0.9121 &           0.8717 &  0.8919 \\
     La Capilla Autom  &         12 &     d01 &   0.9748 &  6.8227 &        0.1703 &           0.7298 &  0.4500 \\
     La Capilla Autom  &         12 &     d02 &   0.9785 &  2.3290 &        0.8757 &           0.7986 &  0.8372 \\
     La Capilla Autom  &         13 &     d01 &   0.9703 &  6.9323 &        0.1531 &           0.6445 &  0.3988 \\
     La Capilla Autom  &         13 &     d02 &   0.9734 &  2.5113 &        0.8471 &           0.7026 &  0.7749 \\
     La Capilla Autom  &         14 &     d01 &   0.9490 &  7.3355 &        0.0898 &           0.2426 &  0.1662 \\
     La Capilla Autom  &         14 &     d02 &   0.9452 &  3.1214 &        0.7513 &           0.1717 &  0.4615 \\
     La Capilla Autom  &         15 &     d01 &   0.9726 &  6.8235 &        0.1702 &           0.6871 &  0.4286 \\
     La Capilla Autom  &         15 &     d02 &   0.9782 &  2.3093 &        0.8788 &           0.7935 &  0.8361 \\
     La Capilla Autom  &         16 &     d01 &   0.9764 &  6.8503 &        0.1660 &           0.7594 &  0.4627 \\
     La Capilla Autom  &         16 &     d02 &   0.9786 &  2.3201 &        0.8771 &           0.8004 &  0.8387 \\
 Chinavita Automatica  &          1 &     d01 &   0.9647 &  6.5271 &        0.1597 &           0.3779 &  0.2688 \\
 Chinavita Automatica  &          1 &     d02 &   0.9828 &  2.9631 &        0.9463 &           0.8397 &  0.8930 \\
 Chinavita Automatica  &          2 &     d01 &   0.9852 &  7.2509 &        0.0000 &           0.9033 &  0.4517 \\
 Chinavita Automatica  &          2 &     d02 &   0.9831 &  3.0933 &        0.9176 &           0.8491 &  0.8833 \\
 Chinavita Automatica  &          3 &     d01 &   0.9678 &  6.4823 &        0.1696 &           0.4581 &  0.3139 \\
 Chinavita Automatica  &          3 &     d02 &   0.9764 &  2.7197 &        1.0000 &           0.6785 &  0.8392 \\
 Chinavita Automatica  &          4 &     d01 &   0.9535 &  6.7833 &        0.1032 &           0.0913 &  0.0972 \\
 Chinavita Automatica  &          4 &     d02 &   0.9686 &  3.3295 &        0.8654 &           0.4786 &  0.6720 \\
 Chinavita Automatica  &          5 &     d01 &   0.9773 &  6.4205 &        0.1833 &           0.6990 &  0.4411 \\
 Chinavita Automatica  &          5 &     d02 &   0.9860 &  2.9946 &        0.9393 &           0.9225 &  0.9309 \\
 Chinavita Automatica  &          6 &     d01 &   0.9796 &  6.8340 &        0.0920 &           0.7597 &  0.4259 \\
 Chinavita Automatica  &          6 &     d02 &   0.9691 &  3.1460 &        0.9059 &           0.4914 &  0.6987 \\
 Chinavita Automatica  &          7 &     d01 &   0.9890 &  6.2449 &        0.2220 &           1.0000 &  0.6110 \\
 Chinavita Automatica  &          7 &     d02 &   0.9866 &  2.7300 &        0.9977 &           0.9374 &  0.9675 \\
 Chinavita Automatica  &          8 &     d01 &   0.9801 &  6.4576 &        0.1751 &           0.7712 &  0.4731 \\
 Chinavita Automatica  &          8 &     d02 &   0.9851 &  2.9881 &        0.9408 &           0.8986 &  0.9197 \\
 Chinavita Automatica  &          9 &     d01 &   0.9812 &  6.3912 &        0.1897 &           0.7992 &  0.4944 \\
 Chinavita Automatica  &          9 &     d02 &   0.9795 &  2.9082 &        0.9584 &           0.7574 &  0.8579 \\
 Chinavita Automatica  &         10 &     d01 &   0.9639 &  6.9212 &        0.0728 &           0.3574 &  0.2151 \\
 Chinavita Automatica  &         10 &     d02 &   0.9499 &  3.2983 &        0.8723 &           0.0000 &  0.4362 \\
 Chinavita Automatica  &         11 &     d01 &   0.9818 &  6.5606 &        0.1523 &           0.8164 &  0.4844 \\
 Chinavita Automatica  &         11 &     d02 &   0.9842 &  2.9077 &        0.9585 &           0.8767 &  0.9176 \\
 Chinavita Automatica  &         12 &     d01 &   0.9847 &  6.8801 &        0.0818 &           0.8883 &  0.4851 \\
 Chinavita Automatica  &         12 &     d02 &   0.9810 &  3.2921 &        0.8737 &           0.7947 &  0.8342 \\
 Chinavita Automatica  &         13 &     d01 &   0.9797 &  6.8589 &        0.0865 &           0.7627 &  0.4246 \\
 Chinavita Automatica  &         13 &     d02 &   0.9778 &  3.2606 &        0.8806 &           0.7132 &  0.7969 \\
 Chinavita Automatica  &         14 &     d01 &   0.9658 &  7.1657 &        0.0188 &           0.4062 &  0.2125 \\
 Chinavita Automatica  &         14 &     d02 &   0.9522 &  3.5846 &        0.8091 &           0.0587 &  0.4339 \\
 Chinavita Automatica  &         15 &     d01 &   0.9815 &  6.9325 &        0.0703 &           0.8085 &  0.4394 \\
 Chinavita Automatica  &         15 &     d02 &   0.9785 &  3.1875 &        0.8968 &           0.7299 &  0.8134 \\
 Chinavita Automatica  &         16 &     d01 &   0.9771 &  6.8990 &        0.0777 &           0.6949 &  0.3863 \\
 Chinavita Automatica  &         16 &     d02 &   0.9771 &  3.0752 &        0.9215 &           0.6961 &  0.8088 \\
 La Boyera Automatica  &          1 &     d01 &   0.9815 &  3.3497 &        0.2297 &           0.5604 &  0.3951 \\
 La Boyera Automatica  &          1 &     d02 &   0.9867 &  1.9270 &        0.7876 &           0.7405 &  0.7640 \\
 La Boyera Automatica  &          2 &     d01 &   0.9943 &  3.4224 &        0.2012 &           1.0000 &  0.6006 \\
 La Boyera Automatica  &          2 &     d02 &   0.9936 &  1.5884 &        0.9203 &           0.9767 &  0.9485 \\
 La Boyera Automatica  &          3 &     d01 &   0.9909 &  3.1367 &        0.3133 &           0.8835 &  0.5984 \\
 La Boyera Automatica  &          3 &     d02 &   0.9940 &  1.5978 &        0.9166 &           0.9907 &  0.9537 \\
 La Boyera Automatica  &          4 &     d01 &   0.9873 &  3.2541 &        0.2672 &           0.7594 &  0.5133 \\
 La Boyera Automatica  &          4 &     d02 &   0.9903 &  1.6086 &        0.9124 &           0.8650 &  0.8887 \\
 La Boyera Automatica  &          5 &     d01 &   0.9883 &  3.4866 &        0.1760 &           0.7940 &  0.4850 \\
 La Boyera Automatica  &          5 &     d02 &   0.9902 &  1.6932 &        0.8792 &           0.8585 &  0.8689 \\
 La Boyera Automatica  &          6 &     d01 &   0.9834 &  3.7357 &        0.0784 &           0.6268 &  0.3526 \\
 La Boyera Automatica  &          6 &     d02 &   0.9842 &  1.9635 &        0.7733 &           0.6523 &  0.7128 \\
 La Boyera Automatica  &          7 &     d01 &   0.9902 &  3.1782 &        0.2970 &           0.8601 &  0.5785 \\
 La Boyera Automatica  &          7 &     d02 &   0.9925 &  1.3852 &        1.0000 &           0.9380 &  0.9690 \\
 La Boyera Automatica  &          8 &     d01 &   0.9828 &  3.5790 &        0.1398 &           0.6055 &  0.3727 \\
 La Boyera Automatica  &          8 &     d02 &   0.9857 &  1.8104 &        0.8333 &           0.7047 &  0.7690 \\
 La Boyera Automatica  &          9 &     d01 &   0.9863 &  3.4623 &        0.1856 &           0.7270 &  0.4563 \\
 La Boyera Automatica  &          9 &     d02 &   0.9890 &  1.7083 &        0.8733 &           0.8190 &  0.8461 \\
 La Boyera Automatica  &         10 &     d01 &   0.9858 &  3.5707 &        0.1430 &           0.7084 &  0.4257 \\
 La Boyera Automatica  &         10 &     d02 &   0.9881 &  1.7789 &        0.8456 &           0.7883 &  0.8169 \\
 La Boyera Automatica  &         11 &     d01 &   0.9860 &  3.2040 &        0.2868 &           0.7154 &  0.5011 \\
 La Boyera Automatica  &         11 &     d02 &   0.9901 &  1.4957 &        0.9567 &           0.8562 &  0.9064 \\
 La Boyera Automatica  &         12 &     d01 &   0.9856 &  3.3152 &        0.2432 &           0.7026 &  0.4729 \\
 La Boyera Automatica  &         12 &     d02 &   0.9887 &  1.5993 &        0.9161 &           0.8088 &  0.8624 \\
 La Boyera Automatica  &         13 &     d01 &   0.9815 &  3.4804 &        0.1785 &           0.5613 &  0.3699 \\
 La Boyera Automatica  &         13 &     d02 &   0.9842 &  1.7656 &        0.8509 &           0.6528 &  0.7518 \\
 La Boyera Automatica  &         14 &     d01 &   0.9652 &  3.9356 &        0.0000 &           0.0000 &  0.0000 \\
 La Boyera Automatica  &         14 &     d02 &   0.9665 &  2.3447 &        0.6238 &           0.0468 &  0.3353 \\
 La Boyera Automatica  &         15 &     d01 &   0.9757 &  3.6444 &        0.1142 &           0.3624 &  0.2383 \\
 La Boyera Automatica  &         15 &     d02 &   0.9794 &  1.9323 &        0.7855 &           0.4875 &  0.6365 \\
 La Boyera Automatica  &         16 &     d01 &   0.9742 &  3.5906 &        0.1353 &           0.3090 &  0.2221 \\
 La Boyera Automatica  &         16 &     d02 &   0.9783 &  1.8702 &        0.8099 &           0.4513 &  0.6306 \\
 Tibaitata Automatica  &          1 &     d01 &   0.9957 &  3.5734 &        0.3735 &           0.9937 &  0.6836 \\
 Tibaitata Automatica  &          1 &     d02 &   0.9960 &  2.7904 &        0.8147 &           1.0000 &  0.9073 \\
 Tibaitata Automatica  &          2 &     d01 &   0.9796 &  3.2964 &        0.5296 &           0.6701 &  0.5999 \\
 Tibaitata Automatica  &          2 &     d02 &   0.9838 &  2.4614 &        1.0000 &           0.7552 &  0.8776 \\
 Tibaitata Automatica  &          3 &     d01 &   0.9940 &  4.2365 &        0.0000 &           0.9595 &  0.4797 \\
 Tibaitata Automatica  &          3 &     d02 &   0.9920 &  3.4791 &        0.4267 &           0.9187 &  0.6727 \\
 Tibaitata Automatica  &          4 &     d01 &   0.9921 &  4.0736 &        0.0918 &           0.9207 &  0.5062 \\
 Tibaitata Automatica  &          4 &     d02 &   0.9907 &  3.3534 &        0.4975 &           0.8933 &  0.6954 \\
 Tibaitata Automatica  &          5 &     d01 &   0.9928 &  3.9113 &        0.1832 &           0.9359 &  0.5596 \\
 Tibaitata Automatica  &          5 &     d02 &   0.9928 &  3.1320 &        0.6222 &           0.9355 &  0.7788 \\
 Tibaitata Automatica  &          6 &     d01 &   0.9834 &  3.6776 &        0.3149 &           0.7475 &  0.5312 \\
 Tibaitata Automatica  &          6 &     d02 &   0.9837 &  2.8807 &        0.7638 &           0.7524 &  0.7581 \\
 Tibaitata Automatica  &          7 &     d01 &   0.9928 &  4.1595 &        0.0434 &           0.9348 &  0.4891 \\
 Tibaitata Automatica  &          7 &     d02 &   0.9901 &  3.5394 &        0.3927 &           0.8809 &  0.6368 \\
 Tibaitata Automatica  &          8 &     d01 &   0.9944 &  4.0949 &        0.0798 &           0.9667 &  0.5232 \\
 Tibaitata Automatica  &          8 &     d02 &   0.9939 &  3.4690 &        0.4324 &           0.9568 &  0.6946 \\
 Tibaitata Automatica  &          9 &     d01 &   0.9828 &  3.8259 &        0.2313 &           0.7348 &  0.4830 \\
 Tibaitata Automatica  &          9 &     d02 &   0.9855 &  3.2122 &        0.5770 &           0.7900 &  0.6835 \\
 Tibaitata Automatica  &         10 &     d01 &   0.9691 &  3.7888 &        0.2522 &           0.4602 &  0.3562 \\
 Tibaitata Automatica  &         10 &     d02 &   0.9691 &  3.1953 &        0.5865 &           0.4609 &  0.5237 \\
 Tibaitata Automatica  &         11 &     d01 &   0.9920 &  3.7588 &        0.2691 &           0.9190 &  0.5940 \\
 Tibaitata Automatica  &         11 &     d02 &   0.9902 &  3.1132 &        0.6328 &           0.8830 &  0.7579 \\
 Tibaitata Automatica  &         12 &     d01 &   0.9901 &  3.5963 &        0.3606 &           0.8809 &  0.6208 \\
 Tibaitata Automatica  &         12 &     d02 &   0.9906 &  2.9250 &        0.7388 &           0.8912 &  0.8150 \\
 Tibaitata Automatica  &         13 &     d01 &   0.9751 &  3.5475 &        0.3881 &           0.5810 &  0.4846 \\
 Tibaitata Automatica  &         13 &     d02 &   0.9770 &  2.9311 &        0.7354 &           0.6189 &  0.6771 \\
 Tibaitata Automatica  &         14 &     d01 &   0.9462 &  3.9169 &        0.1801 &           0.0019 &  0.0910 \\
 Tibaitata Automatica  &         14 &     d02 &   0.9461 &  3.4748 &        0.4291 &           0.0000 &  0.2146 \\
 Tibaitata Automatica  &         15 &     d01 &   0.9828 &  3.3875 &        0.4783 &           0.7352 &  0.6067 \\
 Tibaitata Automatica  &         15 &     d02 &   0.9826 &  2.7550 &        0.8346 &           0.7312 &  0.7829 \\
 Tibaitata Automatica  &         16 &     d01 &   0.9810 &  3.2328 &        0.5654 &           0.7000 &  0.6327 \\
 Tibaitata Automatica  &         16 &     d02 &   0.9821 &  2.6395 &        0.8997 &           0.7217 &  0.8107 \\
   Sta Cruz De Siecha  &          1 &     d01 &   0.9834 &  2.9719 &        0.3928 &           0.9305 &  0.6617 \\
   Sta Cruz De Siecha  &          1 &     d02 &   0.9856 &  1.2708 &        0.9967 &           0.9643 &  0.9805 \\
   Sta Cruz De Siecha  &          2 &     d01 &   0.9880 &  3.1934 &        0.3141 &           1.0000 &  0.6571 \\
   Sta Cruz De Siecha  &          2 &     d02 &   0.9596 &  1.6812 &        0.8510 &           0.5737 &  0.7124 \\
   Sta Cruz De Siecha  &          3 &     d01 &   0.9673 &  2.8959 &        0.4197 &           0.6893 &  0.5545 \\
   Sta Cruz De Siecha  &          3 &     d02 &   0.9629 &  1.5936 &        0.8821 &           0.6236 &  0.7529 \\
   Sta Cruz De Siecha  &          4 &     d01 &   0.9852 &  3.0712 &        0.3575 &           0.9571 &  0.6573 \\
   Sta Cruz De Siecha  &          4 &     d02 &   0.9644 &  1.2615 &        1.0000 &           0.6464 &  0.8232 \\
   Sta Cruz De Siecha  &          5 &     d01 &   0.9794 &  3.3082 &        0.2734 &           0.8705 &  0.5719 \\
   Sta Cruz De Siecha  &          5 &     d02 &   0.9570 &  1.3719 &        0.9608 &           0.5350 &  0.7479 \\
   Sta Cruz De Siecha  &          6 &     d01 &   0.9578 &  3.4075 &        0.2381 &           0.5474 &  0.3928 \\
   Sta Cruz De Siecha  &          6 &     d02 &   0.9557 &  1.2957 &        0.9879 &           0.5158 &  0.7518 \\
   Sta Cruz De Siecha  &          7 &     d01 &   0.9439 &  3.4667 &        0.2171 &           0.3379 &  0.2775 \\
   Sta Cruz De Siecha  &          7 &     d02 &   0.9307 &  1.9748 &        0.7467 &           0.1407 &  0.4437 \\
   Sta Cruz De Siecha  &          8 &     d01 &   0.9732 &  3.4327 &        0.2292 &           0.7780 &  0.5036 \\
   Sta Cruz De Siecha  &          8 &     d02 &   0.9554 &  1.5930 &        0.8823 &           0.5114 &  0.6968 \\
   Sta Cruz De Siecha  &          9 &     d01 &   0.9668 &  3.6449 &        0.1539 &           0.6821 &  0.4180 \\
   Sta Cruz De Siecha  &          9 &     d02 &   0.9556 &  1.6807 &        0.8512 &           0.5131 &  0.6821 \\
   Sta Cruz De Siecha  &         10 &     d01 &   0.9493 &  3.9569 &        0.0431 &           0.4187 &  0.2309 \\
   Sta Cruz De Siecha  &         10 &     d02 &   0.9326 &  2.2255 &        0.6577 &           0.1691 &  0.4134 \\
   Sta Cruz De Siecha  &         11 &     d01 &   0.9376 &  3.8434 &        0.0834 &           0.2438 &  0.1636 \\
   Sta Cruz De Siecha  &         11 &     d02 &   0.9267 &  2.3083 &        0.6284 &           0.0805 &  0.3544 \\
   Sta Cruz De Siecha  &         12 &     d01 &   0.9595 &  3.7956 &        0.1004 &           0.5724 &  0.3364 \\
   Sta Cruz De Siecha  &         12 &     d02 &   0.9352 &  2.2227 &        0.6587 &           0.2079 &  0.4333 \\
   Sta Cruz De Siecha  &         13 &     d01 &   0.9560 &  3.9182 &        0.0568 &           0.5197 &  0.2883 \\
   Sta Cruz De Siecha  &         13 &     d02 &   0.9382 &  2.1150 &        0.6970 &           0.2529 &  0.4749 \\
   Sta Cruz De Siecha  &         14 &     d01 &   0.9490 &  4.0783 &        0.0000 &           0.4150 &  0.2075 \\
   Sta Cruz De Siecha  &         14 &     d02 &   0.9357 &  2.2065 &        0.6645 &           0.2149 &  0.4397 \\
   Sta Cruz De Siecha  &         15 &     d01 &   0.9473 &  3.9208 &        0.0559 &           0.3888 &  0.2223 \\
   Sta Cruz De Siecha  &         15 &     d02 &   0.9353 &  2.2446 &        0.6510 &           0.2099 &  0.4304 \\
   Sta Cruz De Siecha  &         16 &     d01 &   0.9382 &  3.6342 &        0.1577 &           0.2528 &  0.2052 \\
   Sta Cruz De Siecha  &         16 &     d02 &   0.9213 &  1.8801 &        0.7804 &           0.0000 &  0.3902 \\
      Pmo Guacheneque  &          1 &     d01 &   0.8879 &  2.4858 &        0.6342 &           0.4083 &  0.5213 \\
      Pmo Guacheneque  &          1 &     d02 &   0.8601 &  2.5374 &        0.6006 &           0.0000 &  0.3003 \\
      Pmo Guacheneque  &          2 &     d01 &   0.9080 &  3.0269 &        0.2824 &           0.7020 &  0.4922 \\
      Pmo Guacheneque  &          2 &     d02 &   0.8958 &  3.4613 &        0.0000 &           0.5238 &  0.2619 \\
      Pmo Guacheneque  &          3 &     d01 &   0.9170 &  2.8238 &        0.4145 &           0.8350 &  0.6248 \\
      Pmo Guacheneque  &          3 &     d02 &   0.9059 &  2.8020 &        0.4287 &           0.6712 &  0.5499 \\
      Pmo Guacheneque  &          4 &     d01 &   0.9283 &  2.2036 &        0.8177 &           1.0000 &  0.9089 \\
      Pmo Guacheneque  &          4 &     d02 &   0.9131 &  2.0599 &        0.9111 &           0.7771 &  0.8441 \\
      Pmo Guacheneque  &          5 &     d01 &   0.9159 &  2.1792 &        0.8336 &           0.8179 &  0.8257 \\
      Pmo Guacheneque  &          5 &     d02 &   0.9018 &  2.4172 &        0.6788 &           0.6110 &  0.6449 \\
      Pmo Guacheneque  &          6 &     d01 &   0.9016 &  2.5695 &        0.5798 &           0.6092 &  0.5945 \\
      Pmo Guacheneque  &          6 &     d02 &   0.8873 &  3.0096 &        0.2937 &           0.3988 &  0.3462 \\
      Pmo Guacheneque  &          7 &     d01 &   0.8949 &  2.7721 &        0.4481 &           0.5109 &  0.4795 \\
      Pmo Guacheneque  &          7 &     d02 &   0.8794 &  2.7216 &        0.4809 &           0.2833 &  0.3821 \\
      Pmo Guacheneque  &          8 &     d01 &   0.9200 &  2.1684 &        0.8406 &           0.8785 &  0.8595 \\
      Pmo Guacheneque  &          8 &     d02 &   0.9070 &  1.9986 &        0.9509 &           0.6881 &  0.8195 \\
      Pmo Guacheneque  &          9 &     d01 &   0.9075 &  2.2636 &        0.7787 &           0.6955 &  0.7371 \\
      Pmo Guacheneque  &          9 &     d02 &   0.8951 &  2.1984 &        0.8211 &           0.5136 &  0.6673 \\
      Pmo Guacheneque  &         10 &     d01 &   0.9143 &  2.3319 &        0.7342 &           0.7944 &  0.7643 \\
      Pmo Guacheneque  &         10 &     d02 &   0.9058 &  2.4044 &        0.6872 &           0.6707 &  0.6789 \\
      Pmo Guacheneque  &         11 &     d01 &   0.9083 &  2.5563 &        0.5884 &           0.7073 &  0.6478 \\
      Pmo Guacheneque  &         11 &     d02 &   0.8962 &  2.3352 &        0.7321 &           0.5301 &  0.6311 \\
      Pmo Guacheneque  &         12 &     d01 &   0.9219 &  2.2324 &        0.7990 &           0.9061 &  0.8525 \\
      Pmo Guacheneque  &         12 &     d02 &   0.9133 &  1.9232 &        1.0000 &           0.7807 &  0.8904 \\
      Pmo Guacheneque  &         13 &     d01 &   0.9062 &  2.3201 &        0.7419 &           0.6761 &  0.7090 \\
      Pmo Guacheneque  &         13 &     d02 &   0.8917 &  2.1986 &        0.8210 &           0.4632 &  0.6421 \\
      Pmo Guacheneque  &         14 &     d01 &   0.9193 &  2.2585 &        0.7820 &           0.8686 &  0.8253 \\
      Pmo Guacheneque  &         14 &     d02 &   0.9074 &  2.2240 &        0.8044 &           0.6937 &  0.7490 \\
      Pmo Guacheneque  &         15 &     d01 &   0.9125 &  2.4420 &        0.6627 &           0.7686 &  0.7156 \\
      Pmo Guacheneque  &         15 &     d02 &   0.8967 &  2.2294 &        0.8009 &           0.5375 &  0.6692 \\
      Pmo Guacheneque  &         16 &     d01 &   0.9026 &  2.5296 &        0.6057 &           0.6229 &  0.6143 \\
      Pmo Guacheneque  &         16 &     d02 &   0.8838 &  2.2713 &        0.7737 &           0.3485 &  0.5611 \\
       Ciudad Bolivar  &          1 &     d01 &   0.9851 &  2.2247 &        0.7271 &           0.9001 &  0.8136 \\
       Ciudad Bolivar  &          1 &     d02 &   0.9859 &  2.5246 &        0.6250 &           0.9196 &  0.7723 \\
       Ciudad Bolivar  &          2 &     d01 &   0.9573 &  4.2489 &        0.0378 &           0.2448 &  0.1413 \\
       Ciudad Bolivar  &          2 &     d02 &   0.9584 &  4.3597 &        0.0000 &           0.2714 &  0.1357 \\
       Ciudad Bolivar  &          3 &     d01 &   0.9886 &  1.5354 &        0.9619 &           0.9821 &  0.9720 \\
       Ciudad Bolivar  &          3 &     d02 &   0.9894 &  1.4234 &        1.0000 &           1.0000 &  1.0000 \\
       Ciudad Bolivar  &          4 &     d01 &   0.9820 &  1.6902 &        0.9091 &           0.8271 &  0.8681 \\
       Ciudad Bolivar  &          4 &     d02 &   0.9826 &  1.7620 &        0.8847 &           0.8415 &  0.8631 \\
       Ciudad Bolivar  &          5 &     d01 &   0.9825 &  1.8010 &        0.8714 &           0.8394 &  0.8554 \\
       Ciudad Bolivar  &          5 &     d02 &   0.9850 &  1.9053 &        0.8359 &           0.8976 &  0.8667 \\
       Ciudad Bolivar  &          6 &     d01 &   0.9647 &  2.5532 &        0.6152 &           0.4194 &  0.5173 \\
       Ciudad Bolivar  &          6 &     d02 &   0.9678 &  2.7967 &        0.5323 &           0.4934 &  0.5129 \\
       Ciudad Bolivar  &          7 &     d01 &   0.9855 &  1.5796 &        0.9468 &           0.9085 &  0.9277 \\
       Ciudad Bolivar  &          7 &     d02 &   0.9848 &  1.5971 &        0.9409 &           0.8925 &  0.9167 \\
       Ciudad Bolivar  &          8 &     d01 &   0.9852 &  1.4361 &        0.9957 &           0.9023 &  0.9490 \\
       Ciudad Bolivar  &          8 &     d02 &   0.9850 &  1.5020 &        0.9732 &           0.8981 &  0.9356 \\
       Ciudad Bolivar  &          9 &     d01 &   0.9803 &  1.8276 &        0.8623 &           0.7858 &  0.8241 \\
       Ciudad Bolivar  &          9 &     d02 &   0.9727 &  1.9731 &        0.8128 &           0.6069 &  0.7098 \\
       Ciudad Bolivar  &         10 &     d01 &   0.9469 &  2.8944 &        0.4990 &           0.0000 &  0.2495 \\
       Ciudad Bolivar  &         10 &     d02 &   0.9487 &  2.9875 &        0.4673 &           0.0429 &  0.2551 \\
       Ciudad Bolivar  &         11 &     d01 &   0.9851 &  1.5420 &        0.9596 &           0.8998 &  0.9297 \\
       Ciudad Bolivar  &         11 &     d02 &   0.9855 &  1.5543 &        0.9554 &           0.9094 &  0.9324 \\
       Ciudad Bolivar  &         12 &     d01 &   0.9851 &  1.5319 &        0.9631 &           0.8997 &  0.9314 \\
       Ciudad Bolivar  &         12 &     d02 &   0.9864 &  1.6630 &        0.9184 &           0.9303 &  0.9243 \\
       Ciudad Bolivar  &         13 &     d01 &   0.9774 &  1.9009 &        0.8374 &           0.7177 &  0.7775 \\
       Ciudad Bolivar  &         13 &     d02 &   0.9797 &  2.0298 &        0.7935 &           0.7725 &  0.7830 \\
       Ciudad Bolivar  &         14 &     d01 &   0.9506 &  2.6840 &        0.5707 &           0.0863 &  0.3285 \\
       Ciudad Bolivar  &         14 &     d02 &   0.9553 &  2.7860 &        0.5359 &           0.1983 &  0.3671 \\
       Ciudad Bolivar  &         15 &     d01 &   0.9727 &  1.9248 &        0.8292 &           0.6086 &  0.7189 \\
       Ciudad Bolivar  &         15 &     d02 &   0.9773 &  1.9322 &        0.8267 &           0.7153 &  0.7710 \\
       Ciudad Bolivar  &         16 &     d01 &   0.9699 &  1.8940 &        0.8397 &           0.5426 &  0.6912 \\
       Ciudad Bolivar  &         16 &     d02 &   0.9677 &  1.9731 &        0.8128 &           0.4902 &  0.6515 \\
         Pmo Guerrero  &          1 &     d01 &   0.9772 &  3.5272 &        0.3587 &           0.7897 &  0.5742 \\
         Pmo Guerrero  &          1 &     d02 &   0.9469 &  4.3743 &        0.0000 &           0.3239 &  0.1620 \\
         Pmo Guerrero  &          2 &     d01 &   0.9910 &  2.0128 &        1.0000 &           1.0000 &  1.0000 \\
         Pmo Guerrero  &          2 &     d02 &   0.9756 &  3.5236 &        0.3602 &           0.7642 &  0.5622 \\
         Pmo Guerrero  &          3 &     d01 &   0.9541 &  3.0156 &        0.5753 &           0.4347 &  0.5050 \\
         Pmo Guerrero  &          3 &     d02 &   0.9258 &  3.6007 &        0.3276 &           0.0000 &  0.1638 \\
         Pmo Guerrero  &          4 &     d01 &   0.9791 &  3.0891 &        0.5442 &           0.8188 &  0.6815 \\
         Pmo Guerrero  &          4 &     d02 &   0.9645 &  4.0838 &        0.1230 &           0.5947 &  0.3589 \\
         Pmo Guerrero  &          5 &     d01 &   0.9809 &  3.2337 &        0.4830 &           0.8462 &  0.6646 \\
         Pmo Guerrero  &          5 &     d02 &   0.9719 &  3.8823 &        0.2083 &           0.7082 &  0.4583 \\
         Pmo Guerrero  &          6 &     d01 &   0.9676 &  2.9800 &        0.5904 &           0.6414 &  0.6159 \\
         Pmo Guerrero  &          6 &     d02 &   0.9496 &  3.7182 &        0.2778 &           0.3664 &  0.3221 \\
         Pmo Guerrero  &          7 &     d01 &   0.9608 &  3.0950 &        0.5417 &           0.5368 &  0.5393 \\
         Pmo Guerrero  &          7 &     d02 &   0.9375 &  3.4710 &        0.3825 &           0.1807 &  0.2816 \\
         Pmo Guerrero  &          8 &     d01 &   0.9810 &  3.0226 &        0.5724 &           0.8466 &  0.7095 \\
         Pmo Guerrero  &          8 &     d02 &   0.9811 &  3.7721 &        0.2550 &           0.8482 &  0.5516 \\
         Pmo Guerrero  &          9 &     d01 &   0.9650 &  2.9462 &        0.6047 &           0.6017 &  0.6032 \\
         Pmo Guerrero  &          9 &     d02 &   0.9586 &  3.8028 &        0.2420 &           0.5041 &  0.3731 \\
         Pmo Guerrero  &         10 &     d01 &   0.9505 &  2.8801 &        0.6327 &           0.3797 &  0.5062 \\
         Pmo Guerrero  &         10 &     d02 &   0.9460 &  3.7746 &        0.2539 &           0.3103 &  0.2821 \\
         Pmo Guerrero  &         11 &     d01 &   0.9539 &  2.9501 &        0.6031 &           0.4317 &  0.5174 \\
         Pmo Guerrero  &         11 &     d02 &   0.9409 &  3.5944 &        0.3302 &           0.2324 &  0.2813 \\
         Pmo Guerrero  &         12 &     d01 &   0.9655 &  2.9083 &        0.6208 &           0.6102 &  0.6155 \\
         Pmo Guerrero  &         12 &     d02 &   0.9623 &  3.7748 &        0.2539 &           0.5608 &  0.4073 \\
         Pmo Guerrero  &         13 &     d01 &   0.9681 &  2.8585 &        0.6419 &           0.6495 &  0.6457 \\
         Pmo Guerrero  &         13 &     d02 &   0.9680 &  3.7350 &        0.2707 &           0.6483 &  0.4595 \\
         Pmo Guerrero  &         14 &     d01 &   0.9568 &  2.7578 &        0.6845 &           0.4767 &  0.5806 \\
         Pmo Guerrero  &         14 &     d02 &   0.9483 &  3.7941 &        0.2457 &           0.3453 &  0.2955 \\
         Pmo Guerrero  &         15 &     d01 &   0.9525 &  2.8344 &        0.6521 &           0.4106 &  0.5314 \\
         Pmo Guerrero  &         15 &     d02 &   0.9381 &  3.7292 &        0.2732 &           0.1888 &  0.2310 \\
         Pmo Guerrero  &         16 &     d01 &   0.9463 &  2.7797 &        0.6752 &           0.3148 &  0.4950 \\
         Pmo Guerrero  &         16 &     d02 &   0.9304 &  3.5537 &        0.3475 &           0.0707 &  0.2091 \\
    Hda Sta Ana Autom  &          1 &     d01 &   0.9894 &  5.4474 &        0.0000 &           0.9029 &  0.4514 \\
    Hda Sta Ana Autom  &          1 &     d02 &   0.9925 &  4.8734 &        0.2614 &           0.9853 &  0.6233 \\
    Hda Sta Ana Autom  &          2 &     d01 &   0.9908 &  4.1195 &        0.6047 &           0.9387 &  0.7717 \\
    Hda Sta Ana Autom  &          2 &     d02 &   0.9920 &  3.2514 &        1.0000 &           0.9719 &  0.9860 \\
    Hda Sta Ana Autom  &          3 &     d01 &   0.9860 &  4.6529 &        0.3618 &           0.8111 &  0.5865 \\
    Hda Sta Ana Autom  &          3 &     d02 &   0.9931 &  4.0943 &        0.6162 &           1.0000 &  0.8081 \\
    Hda Sta Ana Autom  &          4 &     d01 &   0.9803 &  5.0486 &        0.1816 &           0.6601 &  0.4209 \\
    Hda Sta Ana Autom  &          4 &     d02 &   0.9810 &  4.5227 &        0.4211 &           0.6778 &  0.5494 \\
    Hda Sta Ana Autom  &          5 &     d01 &   0.9866 &  5.0377 &        0.1865 &           0.8262 &  0.5064 \\
    Hda Sta Ana Autom  &          5 &     d02 &   0.9862 &  4.4976 &        0.4325 &           0.8163 &  0.6244 \\
    Hda Sta Ana Autom  &          6 &     d01 &   0.9847 &  4.5587 &        0.4047 &           0.7758 &  0.5903 \\
    Hda Sta Ana Autom  &          6 &     d02 &   0.9842 &  4.0005 &        0.6589 &           0.7635 &  0.7112 \\
    Hda Sta Ana Autom  &          7 &     d01 &   0.9829 &  4.6060 &        0.3831 &           0.7285 &  0.5558 \\
    Hda Sta Ana Autom  &          7 &     d02 &   0.9876 &  4.1107 &        0.6087 &           0.8547 &  0.7317 \\
    Hda Sta Ana Autom  &          8 &     d01 &   0.9822 &  4.9536 &        0.2249 &           0.7115 &  0.4682 \\
    Hda Sta Ana Autom  &          8 &     d02 &   0.9831 &  4.4405 &        0.4585 &           0.7339 &  0.5962 \\
    Hda Sta Ana Autom  &          9 &     d01 &   0.9886 &  4.8845 &        0.2563 &           0.8801 &  0.5682 \\
    Hda Sta Ana Autom  &          9 &     d02 &   0.9885 &  4.5110 &        0.4264 &           0.8784 &  0.6524 \\
    Hda Sta Ana Autom  &         10 &     d01 &   0.9793 &  4.6036 &        0.3843 &           0.6319 &  0.5081 \\
    Hda Sta Ana Autom  &         10 &     d02 &   0.9776 &  4.1336 &        0.5983 &           0.5868 &  0.5925 \\
    Hda Sta Ana Autom  &         11 &     d01 &   0.9835 &  4.6723 &        0.3529 &           0.7439 &  0.5484 \\
    Hda Sta Ana Autom  &         11 &     d02 &   0.9870 &  4.2602 &        0.5406 &           0.8385 &  0.6895 \\
    Hda Sta Ana Autom  &         12 &     d01 &   0.9861 &  4.6840 &        0.3476 &           0.8151 &  0.5814 \\
    Hda Sta Ana Autom  &         12 &     d02 &   0.9873 &  4.1654 &        0.5838 &           0.8450 &  0.7144 \\
    Hda Sta Ana Autom  &         13 &     d01 &   0.9848 &  4.5029 &        0.4301 &           0.7805 &  0.6053 \\
    Hda Sta Ana Autom  &         13 &     d02 &   0.9851 &  3.9805 &        0.6680 &           0.7870 &  0.7275 \\
    Hda Sta Ana Autom  &         14 &     d01 &   0.9562 &  4.6166 &        0.3783 &           0.0188 &  0.1986 \\
    Hda Sta Ana Autom  &         14 &     d02 &   0.9555 &  4.0525 &        0.6352 &           0.0000 &  0.3176 \\
    Hda Sta Ana Autom  &         15 &     d01 &   0.9723 &  4.4661 &        0.4468 &           0.4458 &  0.4463 \\
    Hda Sta Ana Autom  &         15 &     d02 &   0.9739 &  3.9089 &        0.7006 &           0.4894 &  0.5950 \\
    Hda Sta Ana Autom  &         16 &     d01 &   0.9711 &  4.2447 &        0.5477 &           0.4144 &  0.4810 \\
    Hda Sta Ana Autom  &         16 &     d02 &   0.9739 &  3.7714 &        0.7632 &           0.4899 &  0.6266 \\
     Subia Automatica  &          1 &     d01 &   0.9683 &  1.6901 &        0.8872 &           0.9064 &  0.8968 \\
     Subia Automatica  &          1 &     d02 &   0.9793 &  3.4092 &        0.1562 &           1.0000 &  0.5781 \\
     Subia Automatica  &          2 &     d01 &   0.9554 &  1.7257 &        0.8721 &           0.7971 &  0.8346 \\
     Subia Automatica  &          2 &     d02 &   0.9725 &  2.8421 &        0.3974 &           0.9422 &  0.6698 \\
     Subia Automatica  &          3 &     d01 &   0.9548 &  3.6707 &        0.0450 &           0.7920 &  0.4185 \\
     Subia Automatica  &          3 &     d02 &   0.9610 &  1.5110 &        0.9634 &           0.8450 &  0.9042 \\
     Subia Automatica  &          4 &     d01 &   0.9331 &  2.5485 &        0.5222 &           0.6080 &  0.5651 \\
     Subia Automatica  &          4 &     d02 &   0.9592 &  2.2067 &        0.6676 &           0.8295 &  0.7486 \\
     Subia Automatica  &          5 &     d01 &   0.9362 &  2.5604 &        0.5172 &           0.6342 &  0.5757 \\
     Subia Automatica  &          5 &     d02 &   0.9621 &  2.2554 &        0.6468 &           0.8545 &  0.7507 \\
     Subia Automatica  &          6 &     d01 &   0.9437 &  2.7665 &        0.4295 &           0.6979 &  0.5637 \\
     Subia Automatica  &          6 &     d02 &   0.9651 &  1.9256 &        0.7871 &           0.8801 &  0.8336 \\
     Subia Automatica  &          7 &     d01 &   0.9558 &  3.7765 &        0.0000 &           0.8009 &  0.4005 \\
     Subia Automatica  &          7 &     d02 &   0.9646 &  1.4250 &        1.0000 &           0.8751 &  0.9376 \\
     Subia Automatica  &          8 &     d01 &   0.9369 &  3.1369 &        0.2720 &           0.6402 &  0.4561 \\
     Subia Automatica  &          8 &     d02 &   0.9574 &  1.8119 &        0.8355 &           0.8142 &  0.8248 \\
     Subia Automatica  &          9 &     d01 &   0.8614 &  3.1795 &        0.2539 &           0.0000 &  0.1269 \\
     Subia Automatica  &          9 &     d02 &   0.8948 &  2.7807 &        0.4235 &           0.2831 &  0.3533 \\
     Subia Automatica  &         10 &     d01 &   0.8987 &  2.8533 &        0.3926 &           0.3167 &  0.3547 \\
     Subia Automatica  &         10 &     d02 &   0.9231 &  2.5219 &        0.5335 &           0.5238 &  0.5287 \\
     Subia Automatica  &         11 &     d01 &   0.9506 &  3.2469 &        0.2252 &           0.7567 &  0.4910 \\
     Subia Automatica  &         11 &     d02 &   0.9546 &  1.6814 &        0.8910 &           0.7906 &  0.8408 \\
     Subia Automatica  &         12 &     d01 &   0.9263 &  2.8124 &        0.4100 &           0.5506 &  0.4803 \\
     Subia Automatica  &         12 &     d02 &   0.9419 &  2.1614 &        0.6868 &           0.6833 &  0.6851 \\
     Subia Automatica  &         13 &     d01 &   0.9036 &  2.8605 &        0.3895 &           0.3584 &  0.3740 \\
     Subia Automatica  &         13 &     d02 &   0.9242 &  2.4083 &        0.5818 &           0.5329 &  0.5574 \\
     Subia Automatica  &         14 &     d01 &   0.8923 &  2.8904 &        0.3768 &           0.2625 &  0.3197 \\
     Subia Automatica  &         14 &     d02 &   0.9150 &  2.5101 &        0.5386 &           0.4548 &  0.4967 \\
     Subia Automatica  &         15 &     d01 &   0.9472 &  3.0801 &        0.2961 &           0.7282 &  0.5122 \\
     Subia Automatica  &         15 &     d02 &   0.9516 &  1.7635 &        0.8561 &           0.7653 &  0.8107 \\
     Subia Automatica  &         16 &     d01 &   0.9441 &  3.0402 &        0.3131 &           0.7017 &  0.5074 \\
     Subia Automatica  &         16 &     d02 &   0.9487 &  1.8157 &        0.8339 &           0.7403 &  0.7871 \\

\label{tab:estaciones_tiempo_wrf}

\end{longtable}


\textbf{Caso 2}
\begin{longtable}{p{2cm}rrrrrrrrrr}
\caption{Resultados de las diferentes simulaciones.}
\label{tab:estaciones_tiempo_wrf}\\
\hline
   Nombre &  Simulación & Dominio &  Pearson &     RMSE &  $RMSE_{esc}$ &    $Pearson_{esc}$ &      ET \\
   
\midrule
\endhead
\midrule
\multicolumn{3}{r}{{Continúa en la siguiente página.}} \\
\midrule
\endfoot

\bottomrule
\endlastfoot

     La Capilla Autom  &          1 &     d01 &   0.9739 &  5.9902 &        0.3010 &           0.7122 &  0.5066 \\
     La Capilla Autom  &          1 &     d02 &   0.9855 &  1.6368 &        0.9844 &           0.9315 &  0.9579 \\
     La Capilla Autom  &          2 &     d01 &   0.9428 &  7.9076 &        0.0000 &           0.1258 &  0.0629 \\
     La Capilla Autom  &          2 &     d02 &   0.9372 &  4.2380 &        0.5761 &           0.0198 &  0.2979 \\
     La Capilla Autom  &          3 &     d01 &   0.9815 &  6.0658 &        0.2891 &           0.8550 &  0.5721 \\
     La Capilla Autom  &          3 &     d02 &   0.9892 &  1.5373 &        1.0000 &           1.0000 &  1.0000 \\
     La Capilla Autom  &          4 &     d01 &   0.9662 &  6.4159 &        0.2342 &           0.5665 &  0.4003 \\
     La Capilla Autom  &          4 &     d02 &   0.9739 &  2.0047 &        0.9266 &           0.7118 &  0.8192 \\
     La Capilla Autom  &          5 &     d01 &   0.9745 &  6.4478 &        0.2292 &           0.7233 &  0.4762 \\
     La Capilla Autom  &          5 &     d02 &   0.9798 &  2.0541 &        0.9189 &           0.8228 &  0.8709 \\
     La Capilla Autom  &          6 &     d01 &   0.9503 &  7.2073 &        0.1099 &           0.2669 &  0.1884 \\
     La Capilla Autom  &          6 &     d02 &   0.9452 &  3.1784 &        0.7424 &           0.1713 &  0.4568 \\
     La Capilla Autom  &          7 &     d01 &   0.9739 &  6.3450 &        0.2453 &           0.7130 &  0.4792 \\
     La Capilla Autom  &          7 &     d02 &   0.9827 &  1.8654 &        0.9485 &           0.8790 &  0.9138 \\
     La Capilla Autom  &          8 &     d01 &   0.9746 &  6.5480 &        0.2134 &           0.7252 &  0.4693 \\
     La Capilla Autom  &          8 &     d02 &   0.9789 &  2.0663 &        0.9170 &           0.8059 &  0.8614 \\
     La Capilla Autom  &          9 &     d01 &   0.9738 &  6.6564 &        0.1964 &           0.7099 &  0.4532 \\
     La Capilla Autom  &          9 &     d02 &   0.9775 &  2.2251 &        0.8920 &           0.7807 &  0.8364 \\
     La Capilla Autom  &         10 &     d01 &   0.9418 &  7.3678 &        0.0847 &           0.1070 &  0.0959 \\
     La Capilla Autom  &         10 &     d02 &   0.9361 &  3.3010 &        0.7231 &           0.0000 &  0.3616 \\
     La Capilla Autom  &         11 &     d01 &   0.9780 &  6.6055 &        0.2044 &           0.7904 &  0.4974 \\
     La Capilla Autom  &         11 &     d02 &   0.9824 &  2.0969 &        0.9121 &           0.8717 &  0.8919 \\
     La Capilla Autom  &         12 &     d01 &   0.9748 &  6.8227 &        0.1703 &           0.7298 &  0.4500 \\
     La Capilla Autom  &         12 &     d02 &   0.9785 &  2.3290 &        0.8757 &           0.7986 &  0.8372 \\
     La Capilla Autom  &         13 &     d01 &   0.9703 &  6.9323 &        0.1531 &           0.6445 &  0.3988 \\
     La Capilla Autom  &         13 &     d02 &   0.9734 &  2.5113 &        0.8471 &           0.7026 &  0.7749 \\
     La Capilla Autom  &         14 &     d01 &   0.9490 &  7.3355 &        0.0898 &           0.2426 &  0.1662 \\
     La Capilla Autom  &         14 &     d02 &   0.9452 &  3.1214 &        0.7513 &           0.1717 &  0.4615 \\
     La Capilla Autom  &         15 &     d01 &   0.9726 &  6.8235 &        0.1702 &           0.6871 &  0.4286 \\
     La Capilla Autom  &         15 &     d02 &   0.9782 &  2.3093 &        0.8788 &           0.7935 &  0.8361 \\
     La Capilla Autom  &         16 &     d01 &   0.9764 &  6.8503 &        0.1660 &           0.7594 &  0.4627 \\
     La Capilla Autom  &         16 &     d02 &   0.9786 &  2.3201 &        0.8771 &           0.8004 &  0.8387 \\
 Chinavita Automatica  &          1 &     d01 &   0.9647 &  6.5271 &        0.1597 &           0.3779 &  0.2688 \\
 Chinavita Automatica  &          1 &     d02 &   0.9828 &  2.9631 &        0.9463 &           0.8397 &  0.8930 \\
 Chinavita Automatica  &          2 &     d01 &   0.9852 &  7.2509 &        0.0000 &           0.9033 &  0.4517 \\
 Chinavita Automatica  &          2 &     d02 &   0.9831 &  3.0933 &        0.9176 &           0.8491 &  0.8833 \\
 Chinavita Automatica  &          3 &     d01 &   0.9678 &  6.4823 &        0.1696 &           0.4581 &  0.3139 \\
 Chinavita Automatica  &          3 &     d02 &   0.9764 &  2.7197 &        1.0000 &           0.6785 &  0.8392 \\
 Chinavita Automatica  &          4 &     d01 &   0.9535 &  6.7833 &        0.1032 &           0.0913 &  0.0972 \\
 Chinavita Automatica  &          4 &     d02 &   0.9686 &  3.3295 &        0.8654 &           0.4786 &  0.6720 \\
 Chinavita Automatica  &          5 &     d01 &   0.9773 &  6.4205 &        0.1833 &           0.6990 &  0.4411 \\
 Chinavita Automatica  &          5 &     d02 &   0.9860 &  2.9946 &        0.9393 &           0.9225 &  0.9309 \\
 Chinavita Automatica  &          6 &     d01 &   0.9796 &  6.8340 &        0.0920 &           0.7597 &  0.4259 \\
 Chinavita Automatica  &          6 &     d02 &   0.9691 &  3.1460 &        0.9059 &           0.4914 &  0.6987 \\
 Chinavita Automatica  &          7 &     d01 &   0.9890 &  6.2449 &        0.2220 &           1.0000 &  0.6110 \\
 Chinavita Automatica  &          7 &     d02 &   0.9866 &  2.7300 &        0.9977 &           0.9374 &  0.9675 \\
 Chinavita Automatica  &          8 &     d01 &   0.9801 &  6.4576 &        0.1751 &           0.7712 &  0.4731 \\
 Chinavita Automatica  &          8 &     d02 &   0.9851 &  2.9881 &        0.9408 &           0.8986 &  0.9197 \\
 Chinavita Automatica  &          9 &     d01 &   0.9812 &  6.3912 &        0.1897 &           0.7992 &  0.4944 \\
 Chinavita Automatica  &          9 &     d02 &   0.9795 &  2.9082 &        0.9584 &           0.7574 &  0.8579 \\
 Chinavita Automatica  &         10 &     d01 &   0.9639 &  6.9212 &        0.0728 &           0.3574 &  0.2151 \\
 Chinavita Automatica  &         10 &     d02 &   0.9499 &  3.2983 &        0.8723 &           0.0000 &  0.4362 \\
 Chinavita Automatica  &         11 &     d01 &   0.9818 &  6.5606 &        0.1523 &           0.8164 &  0.4844 \\
 Chinavita Automatica  &         11 &     d02 &   0.9842 &  2.9077 &        0.9585 &           0.8767 &  0.9176 \\
 Chinavita Automatica  &         12 &     d01 &   0.9847 &  6.8801 &        0.0818 &           0.8883 &  0.4851 \\
 Chinavita Automatica  &         12 &     d02 &   0.9810 &  3.2921 &        0.8737 &           0.7947 &  0.8342 \\
 Chinavita Automatica  &         13 &     d01 &   0.9797 &  6.8589 &        0.0865 &           0.7627 &  0.4246 \\
 Chinavita Automatica  &         13 &     d02 &   0.9778 &  3.2606 &        0.8806 &           0.7132 &  0.7969 \\
 Chinavita Automatica  &         14 &     d01 &   0.9658 &  7.1657 &        0.0188 &           0.4062 &  0.2125 \\
 Chinavita Automatica  &         14 &     d02 &   0.9522 &  3.5846 &        0.8091 &           0.0587 &  0.4339 \\
 Chinavita Automatica  &         15 &     d01 &   0.9815 &  6.9325 &        0.0703 &           0.8085 &  0.4394 \\
 Chinavita Automatica  &         15 &     d02 &   0.9785 &  3.1875 &        0.8968 &           0.7299 &  0.8134 \\
 Chinavita Automatica  &         16 &     d01 &   0.9771 &  6.8990 &        0.0777 &           0.6949 &  0.3863 \\
 Chinavita Automatica  &         16 &     d02 &   0.9771 &  3.0752 &        0.9215 &           0.6961 &  0.8088 \\
 La Boyera Automatica  &          1 &     d01 &   0.9815 &  3.3497 &        0.2297 &           0.5604 &  0.3951 \\
 La Boyera Automatica  &          1 &     d02 &   0.9867 &  1.9270 &        0.7876 &           0.7405 &  0.7640 \\
 La Boyera Automatica  &          2 &     d01 &   0.9943 &  3.4224 &        0.2012 &           1.0000 &  0.6006 \\
 La Boyera Automatica  &          2 &     d02 &   0.9936 &  1.5884 &        0.9203 &           0.9767 &  0.9485 \\
 La Boyera Automatica  &          3 &     d01 &   0.9909 &  3.1367 &        0.3133 &           0.8835 &  0.5984 \\
 La Boyera Automatica  &          3 &     d02 &   0.9940 &  1.5978 &        0.9166 &           0.9907 &  0.9537 \\
 La Boyera Automatica  &          4 &     d01 &   0.9873 &  3.2541 &        0.2672 &           0.7594 &  0.5133 \\
 La Boyera Automatica  &          4 &     d02 &   0.9903 &  1.6086 &        0.9124 &           0.8650 &  0.8887 \\
 La Boyera Automatica  &          5 &     d01 &   0.9883 &  3.4866 &        0.1760 &           0.7940 &  0.4850 \\
 La Boyera Automatica  &          5 &     d02 &   0.9902 &  1.6932 &        0.8792 &           0.8585 &  0.8689 \\
 La Boyera Automatica  &          6 &     d01 &   0.9834 &  3.7357 &        0.0784 &           0.6268 &  0.3526 \\
 La Boyera Automatica  &          6 &     d02 &   0.9842 &  1.9635 &        0.7733 &           0.6523 &  0.7128 \\
 La Boyera Automatica  &          7 &     d01 &   0.9902 &  3.1782 &        0.2970 &           0.8601 &  0.5785 \\
 La Boyera Automatica  &          7 &     d02 &   0.9925 &  1.3852 &        1.0000 &           0.9380 &  0.9690 \\
 La Boyera Automatica  &          8 &     d01 &   0.9828 &  3.5790 &        0.1398 &           0.6055 &  0.3727 \\
 La Boyera Automatica  &          8 &     d02 &   0.9857 &  1.8104 &        0.8333 &           0.7047 &  0.7690 \\
 La Boyera Automatica  &          9 &     d01 &   0.9863 &  3.4623 &        0.1856 &           0.7270 &  0.4563 \\
 La Boyera Automatica  &          9 &     d02 &   0.9890 &  1.7083 &        0.8733 &           0.8190 &  0.8461 \\
 La Boyera Automatica  &         10 &     d01 &   0.9858 &  3.5707 &        0.1430 &           0.7084 &  0.4257 \\
 La Boyera Automatica  &         10 &     d02 &   0.9881 &  1.7789 &        0.8456 &           0.7883 &  0.8169 \\
 La Boyera Automatica  &         11 &     d01 &   0.9860 &  3.2040 &        0.2868 &           0.7154 &  0.5011 \\
 La Boyera Automatica  &         11 &     d02 &   0.9901 &  1.4957 &        0.9567 &           0.8562 &  0.9064 \\
 La Boyera Automatica  &         12 &     d01 &   0.9856 &  3.3152 &        0.2432 &           0.7026 &  0.4729 \\
 La Boyera Automatica  &         12 &     d02 &   0.9887 &  1.5993 &        0.9161 &           0.8088 &  0.8624 \\
 La Boyera Automatica  &         13 &     d01 &   0.9815 &  3.4804 &        0.1785 &           0.5613 &  0.3699 \\
 La Boyera Automatica  &         13 &     d02 &   0.9842 &  1.7656 &        0.8509 &           0.6528 &  0.7518 \\
 La Boyera Automatica  &         14 &     d01 &   0.9652 &  3.9356 &        0.0000 &           0.0000 &  0.0000 \\
 La Boyera Automatica  &         14 &     d02 &   0.9665 &  2.3447 &        0.6238 &           0.0468 &  0.3353 \\
 La Boyera Automatica  &         15 &     d01 &   0.9757 &  3.6444 &        0.1142 &           0.3624 &  0.2383 \\
 La Boyera Automatica  &         15 &     d02 &   0.9794 &  1.9323 &        0.7855 &           0.4875 &  0.6365 \\
 La Boyera Automatica  &         16 &     d01 &   0.9742 &  3.5906 &        0.1353 &           0.3090 &  0.2221 \\
 La Boyera Automatica  &         16 &     d02 &   0.9783 &  1.8702 &        0.8099 &           0.4513 &  0.6306 \\
 Tibaitata Automatica  &          1 &     d01 &   0.9957 &  3.5734 &        0.3735 &           0.9937 &  0.6836 \\
 Tibaitata Automatica  &          1 &     d02 &   0.9960 &  2.7904 &        0.8147 &           1.0000 &  0.9073 \\
 Tibaitata Automatica  &          2 &     d01 &   0.9796 &  3.2964 &        0.5296 &           0.6701 &  0.5999 \\
 Tibaitata Automatica  &          2 &     d02 &   0.9838 &  2.4614 &        1.0000 &           0.7552 &  0.8776 \\
 Tibaitata Automatica  &          3 &     d01 &   0.9940 &  4.2365 &        0.0000 &           0.9595 &  0.4797 \\
 Tibaitata Automatica  &          3 &     d02 &   0.9920 &  3.4791 &        0.4267 &           0.9187 &  0.6727 \\
 Tibaitata Automatica  &          4 &     d01 &   0.9921 &  4.0736 &        0.0918 &           0.9207 &  0.5062 \\
 Tibaitata Automatica  &          4 &     d02 &   0.9907 &  3.3534 &        0.4975 &           0.8933 &  0.6954 \\
 Tibaitata Automatica  &          5 &     d01 &   0.9928 &  3.9113 &        0.1832 &           0.9359 &  0.5596 \\
 Tibaitata Automatica  &          5 &     d02 &   0.9928 &  3.1320 &        0.6222 &           0.9355 &  0.7788 \\
 Tibaitata Automatica  &          6 &     d01 &   0.9834 &  3.6776 &        0.3149 &           0.7475 &  0.5312 \\
 Tibaitata Automatica  &          6 &     d02 &   0.9837 &  2.8807 &        0.7638 &           0.7524 &  0.7581 \\
 Tibaitata Automatica  &          7 &     d01 &   0.9928 &  4.1595 &        0.0434 &           0.9348 &  0.4891 \\
 Tibaitata Automatica  &          7 &     d02 &   0.9901 &  3.5394 &        0.3927 &           0.8809 &  0.6368 \\
 Tibaitata Automatica  &          8 &     d01 &   0.9944 &  4.0949 &        0.0798 &           0.9667 &  0.5232 \\
 Tibaitata Automatica  &          8 &     d02 &   0.9939 &  3.4690 &        0.4324 &           0.9568 &  0.6946 \\
 Tibaitata Automatica  &          9 &     d01 &   0.9828 &  3.8259 &        0.2313 &           0.7348 &  0.4830 \\
 Tibaitata Automatica  &          9 &     d02 &   0.9855 &  3.2122 &        0.5770 &           0.7900 &  0.6835 \\
 Tibaitata Automatica  &         10 &     d01 &   0.9691 &  3.7888 &        0.2522 &           0.4602 &  0.3562 \\
 Tibaitata Automatica  &         10 &     d02 &   0.9691 &  3.1953 &        0.5865 &           0.4609 &  0.5237 \\
 Tibaitata Automatica  &         11 &     d01 &   0.9920 &  3.7588 &        0.2691 &           0.9190 &  0.5940 \\
 Tibaitata Automatica  &         11 &     d02 &   0.9902 &  3.1132 &        0.6328 &           0.8830 &  0.7579 \\
 Tibaitata Automatica  &         12 &     d01 &   0.9901 &  3.5963 &        0.3606 &           0.8809 &  0.6208 \\
 Tibaitata Automatica  &         12 &     d02 &   0.9906 &  2.9250 &        0.7388 &           0.8912 &  0.8150 \\
 Tibaitata Automatica  &         13 &     d01 &   0.9751 &  3.5475 &        0.3881 &           0.5810 &  0.4846 \\
 Tibaitata Automatica  &         13 &     d02 &   0.9770 &  2.9311 &        0.7354 &           0.6189 &  0.6771 \\
 Tibaitata Automatica  &         14 &     d01 &   0.9462 &  3.9169 &        0.1801 &           0.0019 &  0.0910 \\
 Tibaitata Automatica  &         14 &     d02 &   0.9461 &  3.4748 &        0.4291 &           0.0000 &  0.2146 \\
 Tibaitata Automatica  &         15 &     d01 &   0.9828 &  3.3875 &        0.4783 &           0.7352 &  0.6067 \\
 Tibaitata Automatica  &         15 &     d02 &   0.9826 &  2.7550 &        0.8346 &           0.7312 &  0.7829 \\
 Tibaitata Automatica  &         16 &     d01 &   0.9810 &  3.2328 &        0.5654 &           0.7000 &  0.6327 \\
 Tibaitata Automatica  &         16 &     d02 &   0.9821 &  2.6395 &        0.8997 &           0.7217 &  0.8107 \\
   Sta Cruz De Siecha  &          1 &     d01 &   0.9834 &  2.9719 &        0.3928 &           0.9305 &  0.6617 \\
   Sta Cruz De Siecha  &          1 &     d02 &   0.9856 &  1.2708 &        0.9967 &           0.9643 &  0.9805 \\
   Sta Cruz De Siecha  &          2 &     d01 &   0.9880 &  3.1934 &        0.3141 &           1.0000 &  0.6571 \\
   Sta Cruz De Siecha  &          2 &     d02 &   0.9596 &  1.6812 &        0.8510 &           0.5737 &  0.7124 \\
   Sta Cruz De Siecha  &          3 &     d01 &   0.9673 &  2.8959 &        0.4197 &           0.6893 &  0.5545 \\
   Sta Cruz De Siecha  &          3 &     d02 &   0.9629 &  1.5936 &        0.8821 &           0.6236 &  0.7529 \\
   Sta Cruz De Siecha  &          4 &     d01 &   0.9852 &  3.0712 &        0.3575 &           0.9571 &  0.6573 \\
   Sta Cruz De Siecha  &          4 &     d02 &   0.9644 &  1.2615 &        1.0000 &           0.6464 &  0.8232 \\
   Sta Cruz De Siecha  &          5 &     d01 &   0.9794 &  3.3082 &        0.2734 &           0.8705 &  0.5719 \\
   Sta Cruz De Siecha  &          5 &     d02 &   0.9570 &  1.3719 &        0.9608 &           0.5350 &  0.7479 \\
   Sta Cruz De Siecha  &          6 &     d01 &   0.9578 &  3.4075 &        0.2381 &           0.5474 &  0.3928 \\
   Sta Cruz De Siecha  &          6 &     d02 &   0.9557 &  1.2957 &        0.9879 &           0.5158 &  0.7518 \\
   Sta Cruz De Siecha  &          7 &     d01 &   0.9439 &  3.4667 &        0.2171 &           0.3379 &  0.2775 \\
   Sta Cruz De Siecha  &          7 &     d02 &   0.9307 &  1.9748 &        0.7467 &           0.1407 &  0.4437 \\
   Sta Cruz De Siecha  &          8 &     d01 &   0.9732 &  3.4327 &        0.2292 &           0.7780 &  0.5036 \\
   Sta Cruz De Siecha  &          8 &     d02 &   0.9554 &  1.5930 &        0.8823 &           0.5114 &  0.6968 \\
   Sta Cruz De Siecha  &          9 &     d01 &   0.9668 &  3.6449 &        0.1539 &           0.6821 &  0.4180 \\
   Sta Cruz De Siecha  &          9 &     d02 &   0.9556 &  1.6807 &        0.8512 &           0.5131 &  0.6821 \\
   Sta Cruz De Siecha  &         10 &     d01 &   0.9493 &  3.9569 &        0.0431 &           0.4187 &  0.2309 \\
   Sta Cruz De Siecha  &         10 &     d02 &   0.9326 &  2.2255 &        0.6577 &           0.1691 &  0.4134 \\
   Sta Cruz De Siecha  &         11 &     d01 &   0.9376 &  3.8434 &        0.0834 &           0.2438 &  0.1636 \\
   Sta Cruz De Siecha  &         11 &     d02 &   0.9267 &  2.3083 &        0.6284 &           0.0805 &  0.3544 \\
   Sta Cruz De Siecha  &         12 &     d01 &   0.9595 &  3.7956 &        0.1004 &           0.5724 &  0.3364 \\
   Sta Cruz De Siecha  &         12 &     d02 &   0.9352 &  2.2227 &        0.6587 &           0.2079 &  0.4333 \\
   Sta Cruz De Siecha  &         13 &     d01 &   0.9560 &  3.9182 &        0.0568 &           0.5197 &  0.2883 \\
   Sta Cruz De Siecha  &         13 &     d02 &   0.9382 &  2.1150 &        0.6970 &           0.2529 &  0.4749 \\
   Sta Cruz De Siecha  &         14 &     d01 &   0.9490 &  4.0783 &        0.0000 &           0.4150 &  0.2075 \\
   Sta Cruz De Siecha  &         14 &     d02 &   0.9357 &  2.2065 &        0.6645 &           0.2149 &  0.4397 \\
   Sta Cruz De Siecha  &         15 &     d01 &   0.9473 &  3.9208 &        0.0559 &           0.3888 &  0.2223 \\
   Sta Cruz De Siecha  &         15 &     d02 &   0.9353 &  2.2446 &        0.6510 &           0.2099 &  0.4304 \\
   Sta Cruz De Siecha  &         16 &     d01 &   0.9382 &  3.6342 &        0.1577 &           0.2528 &  0.2052 \\
   Sta Cruz De Siecha  &         16 &     d02 &   0.9213 &  1.8801 &        0.7804 &           0.0000 &  0.3902 \\
      Pmo Guacheneque  &          1 &     d01 &   0.8879 &  2.4858 &        0.6342 &           0.4083 &  0.5213 \\
      Pmo Guacheneque  &          1 &     d02 &   0.8601 &  2.5374 &        0.6006 &           0.0000 &  0.3003 \\
      Pmo Guacheneque  &          2 &     d01 &   0.9080 &  3.0269 &        0.2824 &           0.7020 &  0.4922 \\
      Pmo Guacheneque  &          2 &     d02 &   0.8958 &  3.4613 &        0.0000 &           0.5238 &  0.2619 \\
      Pmo Guacheneque  &          3 &     d01 &   0.9170 &  2.8238 &        0.4145 &           0.8350 &  0.6248 \\
      Pmo Guacheneque  &          3 &     d02 &   0.9059 &  2.8020 &        0.4287 &           0.6712 &  0.5499 \\
      Pmo Guacheneque  &          4 &     d01 &   0.9283 &  2.2036 &        0.8177 &           1.0000 &  0.9089 \\
      Pmo Guacheneque  &          4 &     d02 &   0.9131 &  2.0599 &        0.9111 &           0.7771 &  0.8441 \\
      Pmo Guacheneque  &          5 &     d01 &   0.9159 &  2.1792 &        0.8336 &           0.8179 &  0.8257 \\
      Pmo Guacheneque  &          5 &     d02 &   0.9018 &  2.4172 &        0.6788 &           0.6110 &  0.6449 \\
      Pmo Guacheneque  &          6 &     d01 &   0.9016 &  2.5695 &        0.5798 &           0.6092 &  0.5945 \\
      Pmo Guacheneque  &          6 &     d02 &   0.8873 &  3.0096 &        0.2937 &           0.3988 &  0.3462 \\
      Pmo Guacheneque  &          7 &     d01 &   0.8949 &  2.7721 &        0.4481 &           0.5109 &  0.4795 \\
      Pmo Guacheneque  &          7 &     d02 &   0.8794 &  2.7216 &        0.4809 &           0.2833 &  0.3821 \\
      Pmo Guacheneque  &          8 &     d01 &   0.9200 &  2.1684 &        0.8406 &           0.8785 &  0.8595 \\
      Pmo Guacheneque  &          8 &     d02 &   0.9070 &  1.9986 &        0.9509 &           0.6881 &  0.8195 \\
      Pmo Guacheneque  &          9 &     d01 &   0.9075 &  2.2636 &        0.7787 &           0.6955 &  0.7371 \\
      Pmo Guacheneque  &          9 &     d02 &   0.8951 &  2.1984 &        0.8211 &           0.5136 &  0.6673 \\
      Pmo Guacheneque  &         10 &     d01 &   0.9143 &  2.3319 &        0.7342 &           0.7944 &  0.7643 \\
      Pmo Guacheneque  &         10 &     d02 &   0.9058 &  2.4044 &        0.6872 &           0.6707 &  0.6789 \\
      Pmo Guacheneque  &         11 &     d01 &   0.9083 &  2.5563 &        0.5884 &           0.7073 &  0.6478 \\
      Pmo Guacheneque  &         11 &     d02 &   0.8962 &  2.3352 &        0.7321 &           0.5301 &  0.6311 \\
      Pmo Guacheneque  &         12 &     d01 &   0.9219 &  2.2324 &        0.7990 &           0.9061 &  0.8525 \\
      Pmo Guacheneque  &         12 &     d02 &   0.9133 &  1.9232 &        1.0000 &           0.7807 &  0.8904 \\
      Pmo Guacheneque  &         13 &     d01 &   0.9062 &  2.3201 &        0.7419 &           0.6761 &  0.7090 \\
      Pmo Guacheneque  &         13 &     d02 &   0.8917 &  2.1986 &        0.8210 &           0.4632 &  0.6421 \\
      Pmo Guacheneque  &         14 &     d01 &   0.9193 &  2.2585 &        0.7820 &           0.8686 &  0.8253 \\
      Pmo Guacheneque  &         14 &     d02 &   0.9074 &  2.2240 &        0.8044 &           0.6937 &  0.7490 \\
      Pmo Guacheneque  &         15 &     d01 &   0.9125 &  2.4420 &        0.6627 &           0.7686 &  0.7156 \\
      Pmo Guacheneque  &         15 &     d02 &   0.8967 &  2.2294 &        0.8009 &           0.5375 &  0.6692 \\
      Pmo Guacheneque  &         16 &     d01 &   0.9026 &  2.5296 &        0.6057 &           0.6229 &  0.6143 \\
      Pmo Guacheneque  &         16 &     d02 &   0.8838 &  2.2713 &        0.7737 &           0.3485 &  0.5611 \\
       Ciudad Bolivar  &          1 &     d01 &   0.9851 &  2.2247 &        0.7271 &           0.9001 &  0.8136 \\
       Ciudad Bolivar  &          1 &     d02 &   0.9859 &  2.5246 &        0.6250 &           0.9196 &  0.7723 \\
       Ciudad Bolivar  &          2 &     d01 &   0.9573 &  4.2489 &        0.0378 &           0.2448 &  0.1413 \\
       Ciudad Bolivar  &          2 &     d02 &   0.9584 &  4.3597 &        0.0000 &           0.2714 &  0.1357 \\
       Ciudad Bolivar  &          3 &     d01 &   0.9886 &  1.5354 &        0.9619 &           0.9821 &  0.9720 \\
       Ciudad Bolivar  &          3 &     d02 &   0.9894 &  1.4234 &        1.0000 &           1.0000 &  1.0000 \\
       Ciudad Bolivar  &          4 &     d01 &   0.9820 &  1.6902 &        0.9091 &           0.8271 &  0.8681 \\
       Ciudad Bolivar  &          4 &     d02 &   0.9826 &  1.7620 &        0.8847 &           0.8415 &  0.8631 \\
       Ciudad Bolivar  &          5 &     d01 &   0.9825 &  1.8010 &        0.8714 &           0.8394 &  0.8554 \\
       Ciudad Bolivar  &          5 &     d02 &   0.9850 &  1.9053 &        0.8359 &           0.8976 &  0.8667 \\
       Ciudad Bolivar  &          6 &     d01 &   0.9647 &  2.5532 &        0.6152 &           0.4194 &  0.5173 \\
       Ciudad Bolivar  &          6 &     d02 &   0.9678 &  2.7967 &        0.5323 &           0.4934 &  0.5129 \\
       Ciudad Bolivar  &          7 &     d01 &   0.9855 &  1.5796 &        0.9468 &           0.9085 &  0.9277 \\
       Ciudad Bolivar  &          7 &     d02 &   0.9848 &  1.5971 &        0.9409 &           0.8925 &  0.9167 \\
       Ciudad Bolivar  &          8 &     d01 &   0.9852 &  1.4361 &        0.9957 &           0.9023 &  0.9490 \\
       Ciudad Bolivar  &          8 &     d02 &   0.9850 &  1.5020 &        0.9732 &           0.8981 &  0.9356 \\
       Ciudad Bolivar  &          9 &     d01 &   0.9803 &  1.8276 &        0.8623 &           0.7858 &  0.8241 \\
       Ciudad Bolivar  &          9 &     d02 &   0.9727 &  1.9731 &        0.8128 &           0.6069 &  0.7098 \\
       Ciudad Bolivar  &         10 &     d01 &   0.9469 &  2.8944 &        0.4990 &           0.0000 &  0.2495 \\
       Ciudad Bolivar  &         10 &     d02 &   0.9487 &  2.9875 &        0.4673 &           0.0429 &  0.2551 \\
       Ciudad Bolivar  &         11 &     d01 &   0.9851 &  1.5420 &        0.9596 &           0.8998 &  0.9297 \\
       Ciudad Bolivar  &         11 &     d02 &   0.9855 &  1.5543 &        0.9554 &           0.9094 &  0.9324 \\
       Ciudad Bolivar  &         12 &     d01 &   0.9851 &  1.5319 &        0.9631 &           0.8997 &  0.9314 \\
       Ciudad Bolivar  &         12 &     d02 &   0.9864 &  1.6630 &        0.9184 &           0.9303 &  0.9243 \\
       Ciudad Bolivar  &         13 &     d01 &   0.9774 &  1.9009 &        0.8374 &           0.7177 &  0.7775 \\
       Ciudad Bolivar  &         13 &     d02 &   0.9797 &  2.0298 &        0.7935 &           0.7725 &  0.7830 \\
       Ciudad Bolivar  &         14 &     d01 &   0.9506 &  2.6840 &        0.5707 &           0.0863 &  0.3285 \\
       Ciudad Bolivar  &         14 &     d02 &   0.9553 &  2.7860 &        0.5359 &           0.1983 &  0.3671 \\
       Ciudad Bolivar  &         15 &     d01 &   0.9727 &  1.9248 &        0.8292 &           0.6086 &  0.7189 \\
       Ciudad Bolivar  &         15 &     d02 &   0.9773 &  1.9322 &        0.8267 &           0.7153 &  0.7710 \\
       Ciudad Bolivar  &         16 &     d01 &   0.9699 &  1.8940 &        0.8397 &           0.5426 &  0.6912 \\
       Ciudad Bolivar  &         16 &     d02 &   0.9677 &  1.9731 &        0.8128 &           0.4902 &  0.6515 \\
         Pmo Guerrero  &          1 &     d01 &   0.9772 &  3.5272 &        0.3587 &           0.7897 &  0.5742 \\
         Pmo Guerrero  &          1 &     d02 &   0.9469 &  4.3743 &        0.0000 &           0.3239 &  0.1620 \\
         Pmo Guerrero  &          2 &     d01 &   0.9910 &  2.0128 &        1.0000 &           1.0000 &  1.0000 \\
         Pmo Guerrero  &          2 &     d02 &   0.9756 &  3.5236 &        0.3602 &           0.7642 &  0.5622 \\
         Pmo Guerrero  &          3 &     d01 &   0.9541 &  3.0156 &        0.5753 &           0.4347 &  0.5050 \\
         Pmo Guerrero  &          3 &     d02 &   0.9258 &  3.6007 &        0.3276 &           0.0000 &  0.1638 \\
         Pmo Guerrero  &          4 &     d01 &   0.9791 &  3.0891 &        0.5442 &           0.8188 &  0.6815 \\
         Pmo Guerrero  &          4 &     d02 &   0.9645 &  4.0838 &        0.1230 &           0.5947 &  0.3589 \\
         Pmo Guerrero  &          5 &     d01 &   0.9809 &  3.2337 &        0.4830 &           0.8462 &  0.6646 \\
         Pmo Guerrero  &          5 &     d02 &   0.9719 &  3.8823 &        0.2083 &           0.7082 &  0.4583 \\
         Pmo Guerrero  &          6 &     d01 &   0.9676 &  2.9800 &        0.5904 &           0.6414 &  0.6159 \\
         Pmo Guerrero  &          6 &     d02 &   0.9496 &  3.7182 &        0.2778 &           0.3664 &  0.3221 \\
         Pmo Guerrero  &          7 &     d01 &   0.9608 &  3.0950 &        0.5417 &           0.5368 &  0.5393 \\
         Pmo Guerrero  &          7 &     d02 &   0.9375 &  3.4710 &        0.3825 &           0.1807 &  0.2816 \\
         Pmo Guerrero  &          8 &     d01 &   0.9810 &  3.0226 &        0.5724 &           0.8466 &  0.7095 \\
         Pmo Guerrero  &          8 &     d02 &   0.9811 &  3.7721 &        0.2550 &           0.8482 &  0.5516 \\
         Pmo Guerrero  &          9 &     d01 &   0.9650 &  2.9462 &        0.6047 &           0.6017 &  0.6032 \\
         Pmo Guerrero  &          9 &     d02 &   0.9586 &  3.8028 &        0.2420 &           0.5041 &  0.3731 \\
         Pmo Guerrero  &         10 &     d01 &   0.9505 &  2.8801 &        0.6327 &           0.3797 &  0.5062 \\
         Pmo Guerrero  &         10 &     d02 &   0.9460 &  3.7746 &        0.2539 &           0.3103 &  0.2821 \\
         Pmo Guerrero  &         11 &     d01 &   0.9539 &  2.9501 &        0.6031 &           0.4317 &  0.5174 \\
         Pmo Guerrero  &         11 &     d02 &   0.9409 &  3.5944 &        0.3302 &           0.2324 &  0.2813 \\
         Pmo Guerrero  &         12 &     d01 &   0.9655 &  2.9083 &        0.6208 &           0.6102 &  0.6155 \\
         Pmo Guerrero  &         12 &     d02 &   0.9623 &  3.7748 &        0.2539 &           0.5608 &  0.4073 \\
         Pmo Guerrero  &         13 &     d01 &   0.9681 &  2.8585 &        0.6419 &           0.6495 &  0.6457 \\
         Pmo Guerrero  &         13 &     d02 &   0.9680 &  3.7350 &        0.2707 &           0.6483 &  0.4595 \\
         Pmo Guerrero  &         14 &     d01 &   0.9568 &  2.7578 &        0.6845 &           0.4767 &  0.5806 \\
         Pmo Guerrero  &         14 &     d02 &   0.9483 &  3.7941 &        0.2457 &           0.3453 &  0.2955 \\
         Pmo Guerrero  &         15 &     d01 &   0.9525 &  2.8344 &        0.6521 &           0.4106 &  0.5314 \\
         Pmo Guerrero  &         15 &     d02 &   0.9381 &  3.7292 &        0.2732 &           0.1888 &  0.2310 \\
         Pmo Guerrero  &         16 &     d01 &   0.9463 &  2.7797 &        0.6752 &           0.3148 &  0.4950 \\
         Pmo Guerrero  &         16 &     d02 &   0.9304 &  3.5537 &        0.3475 &           0.0707 &  0.2091 \\
    Hda Sta Ana Autom  &          1 &     d01 &   0.9894 &  5.4474 &        0.0000 &           0.9029 &  0.4514 \\
    Hda Sta Ana Autom  &          1 &     d02 &   0.9925 &  4.8734 &        0.2614 &           0.9853 &  0.6233 \\
    Hda Sta Ana Autom  &          2 &     d01 &   0.9908 &  4.1195 &        0.6047 &           0.9387 &  0.7717 \\
    Hda Sta Ana Autom  &          2 &     d02 &   0.9920 &  3.2514 &        1.0000 &           0.9719 &  0.9860 \\
    Hda Sta Ana Autom  &          3 &     d01 &   0.9860 &  4.6529 &        0.3618 &           0.8111 &  0.5865 \\
    Hda Sta Ana Autom  &          3 &     d02 &   0.9931 &  4.0943 &        0.6162 &           1.0000 &  0.8081 \\
    Hda Sta Ana Autom  &          4 &     d01 &   0.9803 &  5.0486 &        0.1816 &           0.6601 &  0.4209 \\
    Hda Sta Ana Autom  &          4 &     d02 &   0.9810 &  4.5227 &        0.4211 &           0.6778 &  0.5494 \\
    Hda Sta Ana Autom  &          5 &     d01 &   0.9866 &  5.0377 &        0.1865 &           0.8262 &  0.5064 \\
    Hda Sta Ana Autom  &          5 &     d02 &   0.9862 &  4.4976 &        0.4325 &           0.8163 &  0.6244 \\
    Hda Sta Ana Autom  &          6 &     d01 &   0.9847 &  4.5587 &        0.4047 &           0.7758 &  0.5903 \\
    Hda Sta Ana Autom  &          6 &     d02 &   0.9842 &  4.0005 &        0.6589 &           0.7635 &  0.7112 \\
    Hda Sta Ana Autom  &          7 &     d01 &   0.9829 &  4.6060 &        0.3831 &           0.7285 &  0.5558 \\
    Hda Sta Ana Autom  &          7 &     d02 &   0.9876 &  4.1107 &        0.6087 &           0.8547 &  0.7317 \\
    Hda Sta Ana Autom  &          8 &     d01 &   0.9822 &  4.9536 &        0.2249 &           0.7115 &  0.4682 \\
    Hda Sta Ana Autom  &          8 &     d02 &   0.9831 &  4.4405 &        0.4585 &           0.7339 &  0.5962 \\
    Hda Sta Ana Autom  &          9 &     d01 &   0.9886 &  4.8845 &        0.2563 &           0.8801 &  0.5682 \\
    Hda Sta Ana Autom  &          9 &     d02 &   0.9885 &  4.5110 &        0.4264 &           0.8784 &  0.6524 \\
    Hda Sta Ana Autom  &         10 &     d01 &   0.9793 &  4.6036 &        0.3843 &           0.6319 &  0.5081 \\
    Hda Sta Ana Autom  &         10 &     d02 &   0.9776 &  4.1336 &        0.5983 &           0.5868 &  0.5925 \\
    Hda Sta Ana Autom  &         11 &     d01 &   0.9835 &  4.6723 &        0.3529 &           0.7439 &  0.5484 \\
    Hda Sta Ana Autom  &         11 &     d02 &   0.9870 &  4.2602 &        0.5406 &           0.8385 &  0.6895 \\
    Hda Sta Ana Autom  &         12 &     d01 &   0.9861 &  4.6840 &        0.3476 &           0.8151 &  0.5814 \\
    Hda Sta Ana Autom  &         12 &     d02 &   0.9873 &  4.1654 &        0.5838 &           0.8450 &  0.7144 \\
    Hda Sta Ana Autom  &         13 &     d01 &   0.9848 &  4.5029 &        0.4301 &           0.7805 &  0.6053 \\
    Hda Sta Ana Autom  &         13 &     d02 &   0.9851 &  3.9805 &        0.6680 &           0.7870 &  0.7275 \\
    Hda Sta Ana Autom  &         14 &     d01 &   0.9562 &  4.6166 &        0.3783 &           0.0188 &  0.1986 \\
    Hda Sta Ana Autom  &         14 &     d02 &   0.9555 &  4.0525 &        0.6352 &           0.0000 &  0.3176 \\
    Hda Sta Ana Autom  &         15 &     d01 &   0.9723 &  4.4661 &        0.4468 &           0.4458 &  0.4463 \\
    Hda Sta Ana Autom  &         15 &     d02 &   0.9739 &  3.9089 &        0.7006 &           0.4894 &  0.5950 \\
    Hda Sta Ana Autom  &         16 &     d01 &   0.9711 &  4.2447 &        0.5477 &           0.4144 &  0.4810 \\
    Hda Sta Ana Autom  &         16 &     d02 &   0.9739 &  3.7714 &        0.7632 &           0.4899 &  0.6266 \\
     Subia Automatica  &          1 &     d01 &   0.9683 &  1.6901 &        0.8872 &           0.9064 &  0.8968 \\
     Subia Automatica  &          1 &     d02 &   0.9793 &  3.4092 &        0.1562 &           1.0000 &  0.5781 \\
     Subia Automatica  &          2 &     d01 &   0.9554 &  1.7257 &        0.8721 &           0.7971 &  0.8346 \\
     Subia Automatica  &          2 &     d02 &   0.9725 &  2.8421 &        0.3974 &           0.9422 &  0.6698 \\
     Subia Automatica  &          3 &     d01 &   0.9548 &  3.6707 &        0.0450 &           0.7920 &  0.4185 \\
     Subia Automatica  &          3 &     d02 &   0.9610 &  1.5110 &        0.9634 &           0.8450 &  0.9042 \\
     Subia Automatica  &          4 &     d01 &   0.9331 &  2.5485 &        0.5222 &           0.6080 &  0.5651 \\
     Subia Automatica  &          4 &     d02 &   0.9592 &  2.2067 &        0.6676 &           0.8295 &  0.7486 \\
     Subia Automatica  &          5 &     d01 &   0.9362 &  2.5604 &        0.5172 &           0.6342 &  0.5757 \\
     Subia Automatica  &          5 &     d02 &   0.9621 &  2.2554 &        0.6468 &           0.8545 &  0.7507 \\
     Subia Automatica  &          6 &     d01 &   0.9437 &  2.7665 &        0.4295 &           0.6979 &  0.5637 \\
     Subia Automatica  &          6 &     d02 &   0.9651 &  1.9256 &        0.7871 &           0.8801 &  0.8336 \\
     Subia Automatica  &          7 &     d01 &   0.9558 &  3.7765 &        0.0000 &           0.8009 &  0.4005 \\
     Subia Automatica  &          7 &     d02 &   0.9646 &  1.4250 &        1.0000 &           0.8751 &  0.9376 \\
     Subia Automatica  &          8 &     d01 &   0.9369 &  3.1369 &        0.2720 &           0.6402 &  0.4561 \\
     Subia Automatica  &          8 &     d02 &   0.9574 &  1.8119 &        0.8355 &           0.8142 &  0.8248 \\
     Subia Automatica  &          9 &     d01 &   0.8614 &  3.1795 &        0.2539 &           0.0000 &  0.1269 \\
     Subia Automatica  &          9 &     d02 &   0.8948 &  2.7807 &        0.4235 &           0.2831 &  0.3533 \\
     Subia Automatica  &         10 &     d01 &   0.8987 &  2.8533 &        0.3926 &           0.3167 &  0.3547 \\
     Subia Automatica  &         10 &     d02 &   0.9231 &  2.5219 &        0.5335 &           0.5238 &  0.5287 \\
     Subia Automatica  &         11 &     d01 &   0.9506 &  3.2469 &        0.2252 &           0.7567 &  0.4910 \\
     Subia Automatica  &         11 &     d02 &   0.9546 &  1.6814 &        0.8910 &           0.7906 &  0.8408 \\
     Subia Automatica  &         12 &     d01 &   0.9263 &  2.8124 &        0.4100 &           0.5506 &  0.4803 \\
     Subia Automatica  &         12 &     d02 &   0.9419 &  2.1614 &        0.6868 &           0.6833 &  0.6851 \\
     Subia Automatica  &         13 &     d01 &   0.9036 &  2.8605 &        0.3895 &           0.3584 &  0.3740 \\
     Subia Automatica  &         13 &     d02 &   0.9242 &  2.4083 &        0.5818 &           0.5329 &  0.5574 \\
     Subia Automatica  &         14 &     d01 &   0.8923 &  2.8904 &        0.3768 &           0.2625 &  0.3197 \\
     Subia Automatica  &         14 &     d02 &   0.9150 &  2.5101 &        0.5386 &           0.4548 &  0.4967 \\
     Subia Automatica  &         15 &     d01 &   0.9472 &  3.0801 &        0.2961 &           0.7282 &  0.5122 \\
     Subia Automatica  &         15 &     d02 &   0.9516 &  1.7635 &        0.8561 &           0.7653 &  0.8107 \\
     Subia Automatica  &         16 &     d01 &   0.9441 &  3.0402 &        0.3131 &           0.7017 &  0.5074 \\
     Subia Automatica  &         16 &     d02 &   0.9487 &  1.8157 &        0.8339 &           0.7403 &  0.7871 \\

\label{tab:estaciones_tiempo_caso2}


\end{longtable}

\textbf{Caso 3}

\begin{longtable}{p{2cm}rrrrrrrrrr}
\caption{Resultados de las diferentes simulaciones.}
\label{tab:estaciones_tiempo_wrf}\\
\hline
   Nombre &  Simulación & Dominio &  Pearson &     RMSE &  $RMSE_{esc}$ &    $Pearson_{esc}$ &      ET \\
   
\midrule
\endhead
\midrule
\multicolumn{3}{r}{{Continúa en la siguiente página.}} \\
\midrule
\endfoot

\bottomrule
\endlastfoot

        La Capilla Autom  &          1 &     d01 &   0.9233 &  4.7745 &        0.0929 &           1.0000 &  0.5464 \\
        La Capilla Autom  &          1 &     d02 &   0.9181 &  1.7123 &        1.0000 &           0.9457 &  0.9728 \\
        La Capilla Autom  &          2 &     d01 &   0.9044 &  5.0881 &        0.0000 &           0.8044 &  0.4022 \\
        La Capilla Autom  &          2 &     d02 &   0.8916 &  2.0806 &        0.8909 &           0.6725 &  0.7817 \\
        La Capilla Autom  &          3 &     d01 &   0.8995 &  4.0522 &        0.3069 &           0.7540 &  0.5304 \\
        La Capilla Autom  &          3 &     d02 &   0.9011 &  1.9860 &        0.9189 &           0.7700 &  0.8445 \\
        La Capilla Autom  &          4 &     d01 &   0.8788 &  3.5272 &        0.4624 &           0.5407 &  0.5015 \\
        La Capilla Autom  &          4 &     d02 &   0.8792 &  2.4068 &        0.7943 &           0.5446 &  0.6694 \\
        La Capilla Autom  &          5 &     d01 &   0.8640 &  4.0563 &        0.3056 &           0.3874 &  0.3465 \\
        La Capilla Autom  &          5 &     d02 &   0.8655 &  2.4196 &        0.7905 &           0.4035 &  0.5970 \\
        La Capilla Autom  &          6 &     d01 &   0.8320 &  4.5587 &        0.1568 &           0.0574 &  0.1071 \\
        La Capilla Autom  &          6 &     d02 &   0.8264 &  2.6386 &        0.7256 &           0.0000 &  0.3628 \\
        La Capilla Autom  &          7 &     d01 &   0.8718 &  3.9274 &        0.3438 &           0.4680 &  0.4059 \\
        La Capilla Autom  &          7 &     d02 &   0.8717 &  2.2270 &        0.8475 &           0.4668 &  0.6571 \\
        La Capilla Autom  &          8 &     d01 &   0.8696 &  3.8333 &        0.3717 &           0.4452 &  0.4085 \\
        La Capilla Autom  &          8 &     d02 &   0.8782 &  2.2360 &        0.8448 &           0.5342 &  0.6895 \\
        La Capilla Autom  &          9 &     d01 &   0.8656 &  4.2152 &        0.2586 &           0.4040 &  0.3313 \\
        La Capilla Autom  &          9 &     d02 &   0.8694 &  2.2983 &        0.8264 &           0.4437 &  0.6351 \\
        La Capilla Autom  &         10 &     d01 &   0.8409 &  4.5241 &        0.1671 &           0.1497 &  0.1584 \\
        La Capilla Autom  &         10 &     d02 &   0.8414 &  2.5031 &        0.7657 &           0.1546 &  0.4602 \\
        La Capilla Autom  &         11 &     d01 &   0.8836 &  3.9502 &        0.3371 &           0.5902 &  0.4636 \\
        La Capilla Autom  &         11 &     d02 &   0.8815 &  2.2063 &        0.8536 &           0.5681 &  0.7109 \\
        La Capilla Autom  &         12 &     d01 &   0.8783 &  3.8357 &        0.3710 &           0.5350 &  0.4530 \\
        La Capilla Autom  &         12 &     d02 &   0.8862 &  2.2381 &        0.8442 &           0.6166 &  0.7304 \\
        La Capilla Autom  &         13 &     d01 &   0.8663 &  4.2637 &        0.2442 &           0.4114 &  0.3278 \\
        La Capilla Autom  &         13 &     d02 &   0.8685 &  2.3067 &        0.8239 &           0.4342 &  0.6290 \\
        La Capilla Autom  &         14 &     d01 &   0.8575 &  4.3822 &        0.2091 &           0.3206 &  0.2649 \\
        La Capilla Autom  &         14 &     d02 &   0.8620 &  2.3569 &        0.8091 &           0.3665 &  0.5878 \\
        La Capilla Autom  &         15 &     d01 &   0.8682 &  4.0032 &        0.3214 &           0.4314 &  0.3764 \\
        La Capilla Autom  &         15 &     d02 &   0.8693 &  2.3068 &        0.8239 &           0.4424 &  0.6332 \\
        La Capilla Autom  &         16 &     d01 &   0.8677 &  3.8852 &        0.3563 &           0.4258 &  0.3911 \\
        La Capilla Autom  &         16 &     d02 &   0.8666 &  2.3566 &        0.8091 &           0.4142 &  0.6117 \\
 Pmo Rabanal Automatica   &          1 &     d01 &   0.6715 &  6.5348 &        0.1992 &           1.0000 &  0.5996 \\
 Pmo Rabanal Automatica   &          1 &     d02 &   0.5905 &  4.8832 &        0.9199 &           0.5483 &  0.7341 \\
 Pmo Rabanal Automatica   &          2 &     d01 &   0.5872 &  6.2553 &        0.3212 &           0.5297 &  0.4254 \\
 Pmo Rabanal Automatica   &          2 &     d02 &   0.4923 &  4.7050 &        0.9976 &           0.0000 &  0.4988 \\
 Pmo Rabanal Automatica   &          3 &     d01 &   0.6551 &  6.6361 &        0.1551 &           0.9087 &  0.5319 \\
 Pmo Rabanal Automatica   &          3 &     d02 &   0.6021 &  4.9055 &        0.9101 &           0.6129 &  0.7615 \\
 Pmo Rabanal Automatica   &          4 &     d01 &   0.6517 &  6.9908 &        0.0003 &           0.8899 &  0.4451 \\
 Pmo Rabanal Automatica   &          4 &     d02 &   0.6163 &  5.2052 &        0.7794 &           0.6924 &  0.7359 \\
 Pmo Rabanal Automatica   &          5 &     d01 &   0.6244 &  6.5959 &        0.1726 &           0.7374 &  0.4550 \\
 Pmo Rabanal Automatica   &          5 &     d02 &   0.6018 &  4.8805 &        0.9210 &           0.6113 &  0.7662 \\
 Pmo Rabanal Automatica   &          6 &     d01 &   0.5087 &  6.2976 &        0.3028 &           0.0918 &  0.1973 \\
 Pmo Rabanal Automatica   &          6 &     d02 &   0.5089 &  4.6996 &        1.0000 &           0.0925 &  0.5463 \\
 Pmo Rabanal Automatica   &          7 &     d01 &   0.6134 &  6.7258 &        0.1159 &           0.6759 &  0.3959 \\
 Pmo Rabanal Automatica   &          7 &     d02 &   0.6113 &  5.0074 &        0.8657 &           0.6642 &  0.7649 \\
 Pmo Rabanal Automatica   &          8 &     d01 &   0.5884 &  6.9915 &        0.0000 &           0.5366 &  0.2683 \\
 Pmo Rabanal Automatica   &          8 &     d02 &   0.5728 &  5.2516 &        0.7591 &           0.4493 &  0.6042 \\
 Pmo Rabanal Automatica   &          9 &     d01 &   0.6120 &  6.6943 &        0.1297 &           0.6683 &  0.3990 \\
 Pmo Rabanal Automatica   &          9 &     d02 &   0.5962 &  5.0045 &        0.8669 &           0.5799 &  0.7234 \\
 Pmo Rabanal Automatica   &         10 &     d01 &   0.6094 &  6.5439 &        0.1953 &           0.6534 &  0.4244 \\
 Pmo Rabanal Automatica   &         10 &     d02 &   0.5957 &  4.9017 &        0.9118 &           0.5770 &  0.7444 \\
 Pmo Rabanal Automatica   &         11 &     d01 &   0.6699 &  6.7923 &        0.0869 &           0.9915 &  0.5392 \\
 Pmo Rabanal Automatica   &         11 &     d02 &   0.6549 &  4.9726 &        0.8809 &           0.9078 &  0.8943 \\
 Pmo Rabanal Automatica   &         12 &     d01 &   0.6634 &  6.8981 &        0.0407 &           0.9550 &  0.4978 \\
 Pmo Rabanal Automatica   &         12 &     d02 &   0.6476 &  5.1142 &        0.8191 &           0.8668 &  0.8430 \\
 Pmo Rabanal Automatica   &         13 &     d01 &   0.6379 &  6.5447 &        0.1949 &           0.8127 &  0.5038 \\
 Pmo Rabanal Automatica   &         13 &     d02 &   0.6119 &  4.8527 &        0.9332 &           0.6678 &  0.8005 \\
 Pmo Rabanal Automatica   &         14 &     d01 &   0.6279 &  6.4880 &        0.2197 &           0.7568 &  0.4883 \\
 Pmo Rabanal Automatica   &         14 &     d02 &   0.5890 &  4.8438 &        0.9371 &           0.5400 &  0.7385 \\
 Pmo Rabanal Automatica   &         15 &     d01 &   0.6528 &  6.6731 &        0.1389 &           0.8958 &  0.5173 \\
 Pmo Rabanal Automatica   &         15 &     d02 &   0.6370 &  4.9271 &        0.9007 &           0.8075 &  0.8541 \\
 Pmo Rabanal Automatica   &         16 &     d01 &   0.6487 &  6.6179 &        0.1630 &           0.8730 &  0.5180 \\
 Pmo Rabanal Automatica   &         16 &     d02 &   0.6412 &  4.8523 &        0.9334 &           0.8313 &  0.8823 \\
    Chinavita Automatica  &          1 &     d01 &   0.8614 &  4.7336 &        0.1054 &           0.9735 &  0.5395 \\
    Chinavita Automatica  &          1 &     d02 &   0.8641 &  2.0653 &        1.0000 &           1.0000 &  1.0000 \\
    Chinavita Automatica  &          2 &     d01 &   0.8488 &  5.0481 &        0.0000 &           0.8468 &  0.4234 \\
    Chinavita Automatica  &          2 &     d02 &   0.8499 &  2.2428 &        0.9405 &           0.8579 &  0.8992 \\
    Chinavita Automatica  &          3 &     d01 &   0.8473 &  4.1390 &        0.3048 &           0.8324 &  0.5686 \\
    Chinavita Automatica  &          3 &     d02 &   0.8461 &  2.1011 &        0.9880 &           0.8206 &  0.9043 \\
    Chinavita Automatica  &          4 &     d01 &   0.8262 &  3.8011 &        0.4181 &           0.6213 &  0.5197 \\
    Chinavita Automatica  &          4 &     d02 &   0.8052 &  2.4080 &        0.8851 &           0.4116 &  0.6484 \\
    Chinavita Automatica  &          5 &     d01 &   0.8231 &  4.2368 &        0.2720 &           0.5900 &  0.4310 \\
    Chinavita Automatica  &          5 &     d02 &   0.8107 &  2.4406 &        0.8742 &           0.4664 &  0.6703 \\
    Chinavita Automatica  &          6 &     d01 &   0.8084 &  4.7434 &        0.1022 &           0.4434 &  0.2728 \\
    Chinavita Automatica  &          6 &     d02 &   0.7934 &  2.5539 &        0.8362 &           0.2931 &  0.5647 \\
    Chinavita Automatica  &          7 &     d01 &   0.8432 &  3.9922 &        0.3540 &           0.7914 &  0.5727 \\
    Chinavita Automatica  &          7 &     d02 &   0.8378 &  2.2942 &        0.9233 &           0.7372 &  0.8302 \\
    Chinavita Automatica  &          8 &     d01 &   0.8350 &  3.9613 &        0.3644 &           0.7089 &  0.5366 \\
    Chinavita Automatica  &          8 &     d02 &   0.8335 &  2.3487 &        0.9050 &           0.6940 &  0.7995 \\
    Chinavita Automatica  &          9 &     d01 &   0.8215 &  4.4006 &        0.2171 &           0.5743 &  0.3957 \\
    Chinavita Automatica  &          9 &     d02 &   0.8098 &  2.4751 &        0.8626 &           0.4570 &  0.6598 \\
    Chinavita Automatica  &         10 &     d01 &   0.7803 &  4.7964 &        0.0844 &           0.1618 &  0.1231 \\
    Chinavita Automatica  &         10 &     d02 &   0.7641 &  2.6853 &        0.7921 &           0.0000 &  0.3961 \\
    Chinavita Automatica  &         11 &     d01 &   0.8234 &  4.2483 &        0.2682 &           0.5936 &  0.4309 \\
    Chinavita Automatica  &         11 &     d02 &   0.8125 &  2.3585 &        0.9017 &           0.4842 &  0.6929 \\
    Chinavita Automatica  &         12 &     d01 &   0.8292 &  4.0981 &        0.3185 &           0.6507 &  0.4846 \\
    Chinavita Automatica  &         12 &     d02 &   0.8242 &  2.3629 &        0.9002 &           0.6014 &  0.7508 \\
    Chinavita Automatica  &         13 &     d01 &   0.8127 &  4.6545 &        0.1320 &           0.4864 &  0.3092 \\
    Chinavita Automatica  &         13 &     d02 &   0.8024 &  2.5144 &        0.8494 &           0.3836 &  0.6165 \\
    Chinavita Automatica  &         14 &     d01 &   0.8083 &  4.7998 &        0.0833 &           0.4417 &  0.2625 \\
    Chinavita Automatica  &         14 &     d02 &   0.7995 &  2.5433 &        0.8398 &           0.3547 &  0.5972 \\
    Chinavita Automatica  &         15 &     d01 &   0.8258 &  4.4159 &        0.2120 &           0.6168 &  0.4144 \\
    Chinavita Automatica  &         15 &     d02 &   0.8137 &  2.3953 &        0.8894 &           0.4964 &  0.6929 \\
    Chinavita Automatica  &         16 &     d01 &   0.8425 &  4.3021 &        0.2501 &           0.7841 &  0.5171 \\
    Chinavita Automatica  &         16 &     d02 &   0.8318 &  2.3698 &        0.8979 &           0.6773 &  0.7876 \\
            Pmo Chingaza  &          1 &     d01 &   0.6820 &  5.7452 &        0.6084 &           0.7162 &  0.6623 \\
            Pmo Chingaza  &          1 &     d02 &   0.7363 &  5.2828 &        0.8812 &           1.0000 &  0.9406 \\
            Pmo Chingaza  &          2 &     d01 &   0.6095 &  5.5391 &        0.7300 &           0.3376 &  0.5338 \\
            Pmo Chingaza  &          2 &     d02 &   0.6432 &  5.0815 &        1.0000 &           0.5132 &  0.7566 \\
            Pmo Chingaza  &          3 &     d01 &   0.6174 &  6.5264 &        0.1475 &           0.3785 &  0.2630 \\
            Pmo Chingaza  &          3 &     d02 &   0.6187 &  6.0188 &        0.4470 &           0.3854 &  0.4162 \\
            Pmo Chingaza  &          4 &     d01 &   0.6335 &  6.6819 &        0.0557 &           0.4625 &  0.2591 \\
            Pmo Chingaza  &          4 &     d02 &   0.6408 &  6.0471 &        0.4303 &           0.5008 &  0.4655 \\
            Pmo Chingaza  &          5 &     d01 &   0.5829 &  6.2964 &        0.2832 &           0.1982 &  0.2407 \\
            Pmo Chingaza  &          5 &     d02 &   0.5737 &  5.7059 &        0.6316 &           0.1502 &  0.3909 \\
            Pmo Chingaza  &          6 &     d01 &   0.5535 &  6.1145 &        0.3905 &           0.0447 &  0.2176 \\
            Pmo Chingaza  &          6 &     d02 &   0.5449 &  5.5653 &        0.7145 &           0.0000 &  0.3573 \\
            Pmo Chingaza  &          7 &     d01 &   0.6157 &  6.6429 &        0.0787 &           0.3695 &  0.2241 \\
            Pmo Chingaza  &          7 &     d02 &   0.6029 &  6.0771 &        0.4126 &           0.3028 &  0.3577 \\
            Pmo Chingaza  &          8 &     d01 &   0.6403 &  6.6269 &        0.0882 &           0.4983 &  0.2932 \\
            Pmo Chingaza  &          8 &     d02 &   0.6251 &  5.9780 &        0.4710 &           0.4189 &  0.4449 \\
            Pmo Chingaza  &          9 &     d01 &   0.6294 &  6.4669 &        0.1826 &           0.4412 &  0.3119 \\
            Pmo Chingaza  &          9 &     d02 &   0.6113 &  5.8250 &        0.5613 &           0.3466 &  0.4540 \\
            Pmo Chingaza  &         10 &     d01 &   0.6335 &  6.4092 &        0.2166 &           0.4629 &  0.3398 \\
            Pmo Chingaza  &         10 &     d02 &   0.6173 &  5.7898 &        0.5821 &           0.3779 &  0.4800 \\
            Pmo Chingaza  &         11 &     d01 &   0.6818 &  6.7620 &        0.0084 &           0.7150 &  0.3617 \\
            Pmo Chingaza  &         11 &     d02 &   0.6597 &  6.1461 &        0.3719 &           0.5998 &  0.4858 \\
            Pmo Chingaza  &         12 &     d01 &   0.6901 &  6.7763 &        0.0000 &           0.7588 &  0.3794 \\
            Pmo Chingaza  &         12 &     d02 &   0.6732 &  6.1310 &        0.3808 &           0.6701 &  0.5254 \\
            Pmo Chingaza  &         13 &     d01 &   0.6728 &  6.4898 &        0.1691 &           0.6679 &  0.4185 \\
            Pmo Chingaza  &         13 &     d02 &   0.6553 &  5.8085 &        0.5710 &           0.5765 &  0.5738 \\
            Pmo Chingaza  &         14 &     d01 &   0.6287 &  6.3818 &        0.2328 &           0.4378 &  0.3353 \\
            Pmo Chingaza  &         14 &     d02 &   0.6102 &  5.7305 &        0.6171 &           0.3410 &  0.4791 \\
            Pmo Chingaza  &         15 &     d01 &   0.6698 &  6.5651 &        0.1246 &           0.6525 &  0.3886 \\
            Pmo Chingaza  &         15 &     d02 &   0.6543 &  5.9304 &        0.4991 &           0.5713 &  0.5352 \\
            Pmo Chingaza  &         16 &     d01 &   0.6475 &  6.5095 &        0.1574 &           0.5357 &  0.3465 \\
            Pmo Chingaza  &         16 &     d02 &   0.6220 &  5.8722 &        0.5334 &           0.4029 &  0.4682 \\
    Bosque Intervenido    &          1 &     d01 &   0.9008 &  1.8275 &        1.0000 &           1.0000 &  1.0000 \\
    Bosque Intervenido    &          1 &     d02 &   0.8962 &  2.0702 &        0.8070 &           0.9598 &  0.8834 \\
    Bosque Intervenido    &          2 &     d01 &   0.8846 &  1.9655 &        0.8903 &           0.8581 &  0.8742 \\
    Bosque Intervenido    &          2 &     d02 &   0.8933 &  2.2591 &        0.6567 &           0.9347 &  0.7957 \\
    Bosque Intervenido    &          3 &     d01 &   0.8615 &  2.6102 &        0.3774 &           0.6552 &  0.5163 \\
    Bosque Intervenido    &          3 &     d02 &   0.8775 &  2.2549 &        0.6601 &           0.7960 &  0.7280 \\
    Bosque Intervenido    &          4 &     d01 &   0.8641 &  2.7066 &        0.3007 &           0.6784 &  0.4895 \\
    Bosque Intervenido    &          4 &     d02 &   0.8718 &  2.2624 &        0.6541 &           0.7459 &  0.7000 \\
    Bosque Intervenido    &          5 &     d01 &   0.8514 &  2.5362 &        0.4363 &           0.5665 &  0.5014 \\
    Bosque Intervenido    &          5 &     d02 &   0.8697 &  2.2952 &        0.6280 &           0.7276 &  0.6778 \\
    Bosque Intervenido    &          6 &     d01 &   0.8454 &  2.5399 &        0.4334 &           0.5140 &  0.4737 \\
    Bosque Intervenido    &          6 &     d02 &   0.8647 &  2.4149 &        0.5328 &           0.6833 &  0.6081 \\
    Bosque Intervenido    &          7 &     d01 &   0.8459 &  2.9052 &        0.1428 &           0.5185 &  0.3307 \\
    Bosque Intervenido    &          7 &     d02 &   0.8586 &  2.4742 &        0.4856 &           0.6294 &  0.5575 \\
    Bosque Intervenido    &          8 &     d01 &   0.8387 &  2.8232 &        0.2080 &           0.4551 &  0.3315 \\
    Bosque Intervenido    &          8 &     d02 &   0.8421 &  2.3448 &        0.5885 &           0.4844 &  0.5365 \\
    Bosque Intervenido    &          9 &     d01 &   0.8351 &  2.7607 &        0.2577 &           0.4230 &  0.3404 \\
    Bosque Intervenido    &          9 &     d02 &   0.8468 &  2.3960 &        0.5478 &           0.5264 &  0.5371 \\
    Bosque Intervenido    &         10 &     d01 &   0.8298 &  2.8342 &        0.1992 &           0.3767 &  0.2880 \\
    Bosque Intervenido    &         10 &     d02 &   0.8419 &  2.5161 &        0.4523 &           0.4826 &  0.4675 \\
    Bosque Intervenido    &         11 &     d01 &   0.8281 &  3.0847 &        0.0000 &           0.3622 &  0.1811 \\
    Bosque Intervenido    &         11 &     d02 &   0.8399 &  2.5356 &        0.4368 &           0.4652 &  0.4510 \\
    Bosque Intervenido    &         12 &     d01 &   0.8388 &  3.0600 &        0.0196 &           0.4555 &  0.2376 \\
    Bosque Intervenido    &         12 &     d02 &   0.8477 &  2.4423 &        0.5110 &           0.5337 &  0.5224 \\
    Bosque Intervenido    &         13 &     d01 &   0.8312 &  2.9185 &        0.1322 &           0.3893 &  0.2607 \\
    Bosque Intervenido    &         13 &     d02 &   0.8455 &  2.4068 &        0.5392 &           0.5144 &  0.5268 \\
    Bosque Intervenido    &         14 &     d01 &   0.8081 &  2.9751 &        0.0872 &           0.1858 &  0.1365 \\
    Bosque Intervenido    &         14 &     d02 &   0.8244 &  2.5454 &        0.4290 &           0.3292 &  0.3791 \\
    Bosque Intervenido    &         15 &     d01 &   0.8217 &  3.0107 &        0.0588 &           0.3058 &  0.1823 \\
    Bosque Intervenido    &         15 &     d02 &   0.8353 &  2.4611 &        0.4961 &           0.4252 &  0.4606 \\
    Bosque Intervenido    &         16 &     d01 &   0.7869 &  3.0601 &        0.0195 &           0.0000 &  0.0098 \\
    Bosque Intervenido    &         16 &     d02 &   0.8010 &  2.5443 &        0.4298 &           0.1240 &  0.2769 \\
     San Cayetano Autom   &          1 &     d01 &   0.9305 &  5.4108 &        0.0340 &           0.8970 &  0.4655 \\
     San Cayetano Autom   &          1 &     d02 &   0.9427 &  3.6840 &        0.5226 &           1.0000 &  0.7613 \\
     San Cayetano Autom   &          2 &     d01 &   0.9111 &  5.5310 &        0.0000 &           0.7344 &  0.3672 \\
     San Cayetano Autom   &          2 &     d02 &   0.9322 &  3.6073 &        0.5443 &           0.9116 &  0.7279 \\
     San Cayetano Autom   &          3 &     d01 &   0.8835 &  4.5362 &        0.2815 &           0.5022 &  0.3918 \\
     San Cayetano Autom   &          3 &     d02 &   0.8963 &  2.8211 &        0.7667 &           0.6098 &  0.6883 \\
     San Cayetano Autom   &          4 &     d01 &   0.8284 &  4.3416 &        0.3365 &           0.0387 &  0.1876 \\
     San Cayetano Autom   &          4 &     d02 &   0.8238 &  2.8161 &        0.7682 &           0.0000 &  0.3841 \\
     San Cayetano Autom   &          5 &     d01 &   0.9095 &  4.5891 &        0.2665 &           0.7208 &  0.4936 \\
     San Cayetano Autom   &          5 &     d02 &   0.9023 &  3.1892 &        0.6626 &           0.6596 &  0.6611 \\
     San Cayetano Autom   &          6 &     d01 &   0.9168 &  4.8554 &        0.1911 &           0.7817 &  0.4864 \\
     San Cayetano Autom   &          6 &     d02 &   0.9016 &  3.5696 &        0.5550 &           0.6544 &  0.6047 \\
     San Cayetano Autom   &          7 &     d01 &   0.8982 &  3.8943 &        0.4631 &           0.6258 &  0.5444 \\
     San Cayetano Autom   &          7 &     d02 &   0.8996 &  2.6188 &        0.8240 &           0.6373 &  0.7306 \\
     San Cayetano Autom   &          8 &     d01 &   0.8959 &  3.7609 &        0.5008 &           0.6062 &  0.5535 \\
     San Cayetano Autom   &          8 &     d02 &   0.8869 &  2.3120 &        0.9108 &           0.5302 &  0.7205 \\
     San Cayetano Autom   &          9 &     d01 &   0.9253 &  4.0419 &        0.4213 &           0.8538 &  0.6375 \\
     San Cayetano Autom   &          9 &     d02 &   0.9117 &  2.7211 &        0.7950 &           0.7392 &  0.7671 \\
     San Cayetano Autom   &         10 &     d01 &   0.9019 &  4.0830 &        0.4097 &           0.6570 &  0.5333 \\
     San Cayetano Autom   &         10 &     d02 &   0.8978 &  2.6334 &        0.8199 &           0.6220 &  0.7209 \\
     San Cayetano Autom   &         11 &     d01 &   0.9014 &  3.5750 &        0.5534 &           0.6524 &  0.6029 \\
     San Cayetano Autom   &         11 &     d02 &   0.8773 &  2.0961 &        0.9719 &           0.4499 &  0.7109 \\
     San Cayetano Autom   &         12 &     d01 &   0.8863 &  3.5053 &        0.5732 &           0.5255 &  0.5493 \\
     San Cayetano Autom   &         12 &     d02 &   0.8414 &  2.0333 &        0.9896 &           0.1481 &  0.5689 \\
     San Cayetano Autom   &         13 &     d01 &   0.8917 &  3.8909 &        0.4640 &           0.5705 &  0.5173 \\
     San Cayetano Autom   &         13 &     d02 &   0.8657 &  2.2236 &        0.9358 &           0.3521 &  0.6440 \\
     San Cayetano Autom   &         14 &     d01 &   0.8667 &  3.9887 &        0.4364 &           0.3605 &  0.3984 \\
     San Cayetano Autom   &         14 &     d02 &   0.8389 &  2.2386 &        0.9316 &           0.1266 &  0.5291 \\
     San Cayetano Autom   &         15 &     d01 &   0.8832 &  3.5684 &        0.5553 &           0.4994 &  0.5273 \\
     San Cayetano Autom   &         15 &     d02 &   0.8452 &  1.9967 &        1.0000 &           0.1794 &  0.5897 \\
     San Cayetano Autom   &         16 &     d01 &   0.8710 &  3.6476 &        0.5329 &           0.3971 &  0.4650 \\
     San Cayetano Autom   &         16 &     d02 &   0.8297 &  2.1921 &        0.9447 &           0.0491 &  0.4969 \\
    Tibaitata Automatica  &          1 &     d01 &   0.9100 &  2.6795 &        0.7815 &           0.9006 &  0.8411 \\
    Tibaitata Automatica  &          1 &     d02 &   0.9198 &  2.4021 &        1.0000 &           1.0000 &  1.0000 \\
    Tibaitata Automatica  &          2 &     d01 &   0.9065 &  2.6863 &        0.7761 &           0.8652 &  0.8207 \\
    Tibaitata Automatica  &          2 &     d02 &   0.9177 &  2.4940 &        0.9276 &           0.9784 &  0.9530 \\
    Tibaitata Automatica  &          3 &     d01 &   0.8360 &  3.2789 &        0.3092 &           0.1494 &  0.2293 \\
    Tibaitata Automatica  &          3 &     d02 &   0.8334 &  3.5148 &        0.1234 &           0.1233 &  0.1233 \\
    Tibaitata Automatica  &          4 &     d01 &   0.8487 &  3.2456 &        0.3354 &           0.2780 &  0.3067 \\
    Tibaitata Automatica  &          4 &     d02 &   0.8428 &  3.4216 &        0.1968 &           0.2189 &  0.2078 \\
    Tibaitata Automatica  &          5 &     d01 &   0.8410 &  3.3070 &        0.2871 &           0.2001 &  0.2436 \\
    Tibaitata Automatica  &          5 &     d02 &   0.8361 &  3.5615 &        0.0866 &           0.1503 &  0.1184 \\
    Tibaitata Automatica  &          6 &     d01 &   0.8260 &  3.4042 &        0.2105 &           0.0480 &  0.1293 \\
    Tibaitata Automatica  &          6 &     d02 &   0.8213 &  3.6714 &        0.0000 &           0.0000 &  0.0000 \\
    Tibaitata Automatica  &          7 &     d01 &   0.8563 &  3.1940 &        0.3761 &           0.3557 &  0.3659 \\
    Tibaitata Automatica  &          7 &     d02 &   0.8494 &  3.4593 &        0.1670 &           0.2858 &  0.2264 \\
    Tibaitata Automatica  &          8 &     d01 &   0.8651 &  3.1294 &        0.4270 &           0.4444 &  0.4357 \\
    Tibaitata Automatica  &          8 &     d02 &   0.8605 &  3.3610 &        0.2446 &           0.3985 &  0.3215 \\
    Tibaitata Automatica  &          9 &     d01 &   0.8749 &  2.9650 &        0.5565 &           0.5446 &  0.5506 \\
    Tibaitata Automatica  &          9 &     d02 &   0.8711 &  3.2086 &        0.3646 &           0.5052 &  0.4349 \\
    Tibaitata Automatica  &         10 &     d01 &   0.8535 &  3.0791 &        0.4666 &           0.3272 &  0.3969 \\
    Tibaitata Automatica  &         10 &     d02 &   0.8514 &  3.3149 &        0.2808 &           0.3053 &  0.2931 \\
    Tibaitata Automatica  &         11 &     d01 &   0.8914 &  3.0050 &        0.5250 &           0.7117 &  0.6184 \\
    Tibaitata Automatica  &         11 &     d02 &   0.8858 &  3.2745 &        0.3127 &           0.6547 &  0.4837 \\
    Tibaitata Automatica  &         12 &     d01 &   0.8929 &  2.9582 &        0.5619 &           0.7267 &  0.6443 \\
    Tibaitata Automatica  &         12 &     d02 &   0.8864 &  3.2418 &        0.3385 &           0.6612 &  0.4999 \\
    Tibaitata Automatica  &         13 &     d01 &   0.8862 &  2.8775 &        0.6254 &           0.6584 &  0.6419 \\
    Tibaitata Automatica  &         13 &     d02 &   0.8824 &  3.1245 &        0.4309 &           0.6198 &  0.5254 \\
    Tibaitata Automatica  &         14 &     d01 &   0.8734 &  2.9110 &        0.5991 &           0.5286 &  0.5638 \\
    Tibaitata Automatica  &         14 &     d02 &   0.8701 &  3.1820 &        0.3855 &           0.4950 &  0.4403 \\
    Tibaitata Automatica  &         15 &     d01 &   0.9023 &  2.8135 &        0.6759 &           0.8227 &  0.7493 \\
    Tibaitata Automatica  &         15 &     d02 &   0.8967 &  3.0770 &        0.4683 &           0.7649 &  0.6166 \\
    Tibaitata Automatica  &         16 &     d01 &   0.8824 &  2.7765 &        0.7050 &           0.6205 &  0.6627 \\
    Tibaitata Automatica  &         16 &     d02 &   0.8793 &  3.0351 &        0.5013 &           0.5889 &  0.5451 \\
      Sta Cruz De Siecha  &          1 &     d01 &   0.9240 &  1.8608 &        0.9918 &           0.9709 &  0.9814 \\
      Sta Cruz De Siecha  &          1 &     d02 &   0.9285 &  2.1741 &        0.8095 &           1.0000 &  0.9048 \\
      Sta Cruz De Siecha  &          2 &     d01 &   0.8690 &  2.3106 &        0.7301 &           0.6169 &  0.6735 \\
      Sta Cruz De Siecha  &          2 &     d02 &   0.8749 &  2.2068 &        0.7906 &           0.6553 &  0.7229 \\
      Sta Cruz De Siecha  &          3 &     d01 &   0.7884 &  2.1622 &        0.8165 &           0.0986 &  0.4575 \\
      Sta Cruz De Siecha  &          3 &     d02 &   0.8191 &  3.1117 &        0.2643 &           0.2961 &  0.2802 \\
      Sta Cruz De Siecha  &          4 &     d01 &   0.8280 &  1.8467 &        1.0000 &           0.3536 &  0.6768 \\
      Sta Cruz De Siecha  &          4 &     d02 &   0.8288 &  3.1005 &        0.2707 &           0.3584 &  0.3146 \\
      Sta Cruz De Siecha  &          5 &     d01 &   0.8148 &  2.2981 &        0.7374 &           0.2687 &  0.5031 \\
      Sta Cruz De Siecha  &          5 &     d02 &   0.8104 &  3.0100 &        0.3234 &           0.2403 &  0.2818 \\
      Sta Cruz De Siecha  &          6 &     d01 &   0.7840 &  2.6972 &        0.5053 &           0.0701 &  0.2877 \\
      Sta Cruz De Siecha  &          6 &     d02 &   0.7731 &  3.0408 &        0.3055 &           0.0000 &  0.1527 \\
      Sta Cruz De Siecha  &          7 &     d01 &   0.7843 &  2.3858 &        0.6864 &           0.0723 &  0.3794 \\
      Sta Cruz De Siecha  &          7 &     d02 &   0.7746 &  3.3782 &        0.1092 &           0.0100 &  0.0596 \\
      Sta Cruz De Siecha  &          8 &     d01 &   0.7979 &  2.1997 &        0.7947 &           0.1595 &  0.4771 \\
      Sta Cruz De Siecha  &          8 &     d02 &   0.7866 &  3.4342 &        0.0767 &           0.0869 &  0.0818 \\
      Sta Cruz De Siecha  &          9 &     d01 &   0.8096 &  2.4697 &        0.6377 &           0.2350 &  0.4363 \\
      Sta Cruz De Siecha  &          9 &     d02 &   0.7868 &  3.3761 &        0.1105 &           0.0881 &  0.0993 \\
      Sta Cruz De Siecha  &         10 &     d01 &   0.8074 &  2.5742 &        0.5769 &           0.2211 &  0.3990 \\
      Sta Cruz De Siecha  &         10 &     d02 &   0.7847 &  3.3410 &        0.1309 &           0.0746 &  0.1027 \\
      Sta Cruz De Siecha  &         11 &     d01 &   0.8168 &  2.3284 &        0.7198 &           0.2813 &  0.5006 \\
      Sta Cruz De Siecha  &         11 &     d02 &   0.8027 &  3.5660 &        0.0000 &           0.1904 &  0.0952 \\
      Sta Cruz De Siecha  &         12 &     d01 &   0.8248 &  2.1429 &        0.8277 &           0.3328 &  0.5802 \\
      Sta Cruz De Siecha  &         12 &     d02 &   0.8145 &  3.5478 &        0.0106 &           0.2662 &  0.1384 \\
      Sta Cruz De Siecha  &         13 &     d01 &   0.8185 &  2.3332 &        0.7170 &           0.2920 &  0.5045 \\
      Sta Cruz De Siecha  &         13 &     d02 &   0.8009 &  3.4003 &        0.0964 &           0.1790 &  0.1377 \\
      Sta Cruz De Siecha  &         14 &     d01 &   0.8130 &  2.3724 &        0.6942 &           0.2566 &  0.4754 \\
      Sta Cruz De Siecha  &         14 &     d02 &   0.7887 &  3.4130 &        0.0890 &           0.1009 &  0.0949 \\
      Sta Cruz De Siecha  &         15 &     d01 &   0.8234 &  2.1712 &        0.8112 &           0.3240 &  0.5676 \\
      Sta Cruz De Siecha  &         15 &     d02 &   0.8114 &  3.4705 &        0.0556 &           0.2466 &  0.1511 \\
      Sta Cruz De Siecha  &         16 &     d01 &   0.8058 &  2.3197 &        0.7248 &           0.2109 &  0.4679 \\
      Sta Cruz De Siecha  &         16 &     d02 &   0.7945 &  3.5082 &        0.0336 &           0.1379 &  0.0857 \\
            Ideam Bogota  &          1 &     d01 &   0.9416 &  6.7015 &        0.2185 &           1.0000 &  0.6093 \\
            Ideam Bogota  &          1 &     d02 &   0.9259 &  6.9662 &        0.0637 &           0.8621 &  0.4629 \\
            Ideam Bogota  &          2 &     d01 &   0.9259 &  6.9681 &        0.0626 &           0.8617 &  0.4621 \\
            Ideam Bogota  &          2 &     d02 &   0.9098 &  7.0750 &        0.0000 &           0.7198 &  0.3599 \\
            Ideam Bogota  &          3 &     d01 &   0.8745 &  5.6021 &        0.8619 &           0.4087 &  0.6353 \\
            Ideam Bogota  &          3 &     d02 &   0.8648 &  5.9252 &        0.6728 &           0.3232 &  0.4980 \\
            Ideam Bogota  &          4 &     d01 &   0.9076 &  5.4747 &        0.9364 &           0.7003 &  0.8184 \\
            Ideam Bogota  &          4 &     d02 &   0.9026 &  5.8690 &        0.7057 &           0.6561 &  0.6809 \\
            Ideam Bogota  &          5 &     d01 &   0.8865 &  6.1286 &        0.5538 &           0.5150 &  0.5344 \\
            Ideam Bogota  &          5 &     d02 &   0.8769 &  6.4596 &        0.3601 &           0.4302 &  0.3951 \\
            Ideam Bogota  &          6 &     d01 &   0.8424 &  6.8063 &        0.1572 &           0.1259 &  0.1416 \\
            Ideam Bogota  &          6 &     d02 &   0.8281 &  7.0746 &        0.0002 &           0.0000 &  0.0001 \\
            Ideam Bogota  &          7 &     d01 &   0.8932 &  5.7190 &        0.7935 &           0.5741 &  0.6838 \\
            Ideam Bogota  &          7 &     d02 &   0.8834 &  6.1057 &        0.5672 &           0.4879 &  0.5275 \\
            Ideam Bogota  &          8 &     d01 &   0.8994 &  5.3660 &        1.0000 &           0.6280 &  0.8140 \\
            Ideam Bogota  &          8 &     d02 &   0.8916 &  5.7341 &        0.7846 &           0.5594 &  0.6720 \\
            Ideam Bogota  &          9 &     d01 &   0.9111 &  5.9557 &        0.6549 &           0.7316 &  0.6933 \\
            Ideam Bogota  &          9 &     d02 &   0.9012 &  6.2758 &        0.4677 &           0.6440 &  0.5558 \\
            Ideam Bogota  &         10 &     d01 &   0.8589 &  6.2961 &        0.4558 &           0.2716 &  0.3637 \\
            Ideam Bogota  &         10 &     d02 &   0.8520 &  6.5252 &        0.3217 &           0.2113 &  0.2665 \\
            Ideam Bogota  &         11 &     d01 &   0.8995 &  5.5107 &        0.9153 &           0.6295 &  0.7724 \\
            Ideam Bogota  &         11 &     d02 &   0.8932 &  5.8093 &        0.7406 &           0.5742 &  0.6574 \\
            Ideam Bogota  &         12 &     d01 &   0.8775 &  5.4195 &        0.9687 &           0.4351 &  0.7019 \\
            Ideam Bogota  &         12 &     d02 &   0.8542 &  5.6382 &        0.8407 &           0.2301 &  0.5354 \\
            Ideam Bogota  &         13 &     d01 &   0.8900 &  5.8839 &        0.6969 &           0.5456 &  0.6213 \\
            Ideam Bogota  &         13 &     d02 &   0.8804 &  6.1030 &        0.5687 &           0.4614 &  0.5151 \\
            Ideam Bogota  &         14 &     d01 &   0.8669 &  6.0113 &        0.6224 &           0.3424 &  0.4824 \\
            Ideam Bogota  &         14 &     d02 &   0.8605 &  6.0354 &        0.6083 &           0.2862 &  0.4473 \\
            Ideam Bogota  &         15 &     d01 &   0.8889 &  5.5055 &        0.9183 &           0.5361 &  0.7272 \\
            Ideam Bogota  &         15 &     d02 &   0.8710 &  5.6263 &        0.8477 &           0.3784 &  0.6130 \\
            Ideam Bogota  &         16 &     d01 &   0.8579 &  5.6412 &        0.8389 &           0.2626 &  0.5508 \\
            Ideam Bogota  &         16 &     d02 &   0.8308 &  5.8106 &        0.7398 &           0.0238 &  0.3818 \\
            Pmo Guerrero  &          1 &     d01 &   0.8614 &  3.6094 &        0.3538 &           1.0000 &  0.6769 \\
            Pmo Guerrero  &          1 &     d02 &   0.8444 &  2.0666 &        0.9703 &           0.8482 &  0.9092 \\
            Pmo Guerrero  &          2 &     d01 &   0.8306 &  3.5574 &        0.3745 &           0.7258 &  0.5501 \\
            Pmo Guerrero  &          2 &     d02 &   0.7820 &  2.4570 &        0.8143 &           0.2933 &  0.5538 \\
            Pmo Guerrero  &          3 &     d01 &   0.8358 &  4.3027 &        0.0767 &           0.7722 &  0.4245 \\
            Pmo Guerrero  &          3 &     d02 &   0.8219 &  2.2440 &        0.8993 &           0.6486 &  0.7740 \\
            Pmo Guerrero  &          4 &     d01 &   0.8006 &  4.3560 &        0.0554 &           0.4588 &  0.2571 \\
            Pmo Guerrero  &          4 &     d02 &   0.8124 &  1.9921 &        1.0000 &           0.5641 &  0.7821 \\
            Pmo Guerrero  &          5 &     d01 &   0.7954 &  4.0817 &        0.1650 &           0.4127 &  0.2888 \\
            Pmo Guerrero  &          5 &     d02 &   0.8095 &  2.3359 &        0.8626 &           0.5379 &  0.7003 \\
            Pmo Guerrero  &          6 &     d01 &   0.7572 &  4.0050 &        0.1957 &           0.0731 &  0.1344 \\
            Pmo Guerrero  &          6 &     d02 &   0.7733 &  2.6714 &        0.7285 &           0.2159 &  0.4722 \\
            Pmo Guerrero  &          7 &     d01 &   0.7788 &  4.4143 &        0.0321 &           0.2653 &  0.1487 \\
            Pmo Guerrero  &          7 &     d02 &   0.7861 &  2.3530 &        0.8558 &           0.3301 &  0.5930 \\
            Pmo Guerrero  &          8 &     d01 &   0.7803 &  4.4771 &        0.0070 &           0.2779 &  0.1425 \\
            Pmo Guerrero  &          8 &     d02 &   0.7875 &  2.3784 &        0.8456 &           0.3425 &  0.5941 \\
            Pmo Guerrero  &          9 &     d01 &   0.7818 &  4.2712 &        0.0893 &           0.2917 &  0.1905 \\
            Pmo Guerrero  &          9 &     d02 &   0.7954 &  2.5198 &        0.7891 &           0.4124 &  0.6008 \\
            Pmo Guerrero  &         10 &     d01 &   0.7636 &  4.2416 &        0.1011 &           0.1297 &  0.1154 \\
            Pmo Guerrero  &         10 &     d02 &   0.7786 &  2.7206 &        0.7089 &           0.2633 &  0.4861 \\
            Pmo Guerrero  &         11 &     d01 &   0.7926 &  4.4697 &        0.0100 &           0.3874 &  0.1987 \\
            Pmo Guerrero  &         11 &     d02 &   0.8018 &  2.5083 &        0.7937 &           0.4699 &  0.6318 \\
            Pmo Guerrero  &         12 &     d01 &   0.7953 &  4.4947 &        0.0000 &           0.4121 &  0.2060 \\
            Pmo Guerrero  &         12 &     d02 &   0.7959 &  2.4270 &        0.8262 &           0.4174 &  0.6218 \\
            Pmo Guerrero  &         13 &     d01 &   0.7795 &  4.2530 &        0.0966 &           0.2714 &  0.1840 \\
            Pmo Guerrero  &         13 &     d02 &   0.7805 &  2.5815 &        0.7645 &           0.2798 &  0.5222 \\
            Pmo Guerrero  &         14 &     d01 &   0.7559 &  4.1995 &        0.1180 &           0.0614 &  0.0897 \\
            Pmo Guerrero  &         14 &     d02 &   0.7490 &  2.7530 &        0.6960 &           0.0000 &  0.3480 \\
            Pmo Guerrero  &         15 &     d01 &   0.7775 &  4.3293 &        0.0661 &           0.2536 &  0.1598 \\
            Pmo Guerrero  &         15 &     d02 &   0.7781 &  2.5005 &        0.7969 &           0.2586 &  0.5277 \\
            Pmo Guerrero  &         16 &     d01 &   0.7635 &  4.2303 &        0.1057 &           0.1288 &  0.1172 \\
            Pmo Guerrero  &         16 &     d02 &   0.7814 &  2.4974 &        0.7981 &           0.2879 &  0.5430 \\
 Villa Teresa Automatica  &          1 &     d01 &   0.8656 &  2.3394 &        1.0000 &           0.9127 &  0.9564 \\
 Villa Teresa Automatica  &          1 &     d02 &   0.8912 &  3.1262 &        0.5850 &           1.0000 &  0.7925 \\
 Villa Teresa Automatica  &          2 &     d01 &   0.8013 &  2.9803 &        0.6620 &           0.6928 &  0.6774 \\
 Villa Teresa Automatica  &          2 &     d02 &   0.8013 &  3.2424 &        0.5238 &           0.6930 &  0.6084 \\
 Villa Teresa Automatica  &          3 &     d01 &   0.7717 &  3.0026 &        0.6502 &           0.5917 &  0.6210 \\
 Villa Teresa Automatica  &          3 &     d02 &   0.7743 &  4.1085 &        0.0670 &           0.6006 &  0.3338 \\
 Villa Teresa Automatica  &          4 &     d01 &   0.7838 &  2.6240 &        0.8499 &           0.6332 &  0.7415 \\
 Villa Teresa Automatica  &          4 &     d02 &   0.7832 &  3.9602 &        0.1452 &           0.6312 &  0.3882 \\
 Villa Teresa Automatica  &          5 &     d01 &   0.6998 &  3.2179 &        0.5367 &           0.3463 &  0.4415 \\
 Villa Teresa Automatica  &          5 &     d02 &   0.6934 &  3.9701 &        0.1400 &           0.3242 &  0.2321 \\
 Villa Teresa Automatica  &          6 &     d01 &   0.6173 &  3.7512 &        0.2554 &           0.0645 &  0.1600 \\
 Villa Teresa Automatica  &          6 &     d02 &   0.5985 &  4.1345 &        0.0533 &           0.0000 &  0.0266 \\
 Villa Teresa Automatica  &          7 &     d01 &   0.6936 &  3.2987 &        0.4941 &           0.3251 &  0.4096 \\
 Villa Teresa Automatica  &          7 &     d02 &   0.6709 &  4.1641 &        0.0377 &           0.2473 &  0.1425 \\
 Villa Teresa Automatica  &          8 &     d01 &   0.6928 &  3.0765 &        0.6113 &           0.3222 &  0.4667 \\
 Villa Teresa Automatica  &          8 &     d02 &   0.6690 &  4.1338 &        0.0536 &           0.2409 &  0.1473 \\
 Villa Teresa Automatica  &          9 &     d01 &   0.6856 &  3.5260 &        0.3742 &           0.2978 &  0.3360 \\
 Villa Teresa Automatica  &          9 &     d02 &   0.6585 &  4.1467 &        0.0468 &           0.2050 &  0.1259 \\
 Villa Teresa Automatica  &         10 &     d01 &   0.6963 &  3.6043 &        0.3329 &           0.3342 &  0.3336 \\
 Villa Teresa Automatica  &         10 &     d02 &   0.6706 &  4.1214 &        0.0602 &           0.2466 &  0.1534 \\
 Villa Teresa Automatica  &         11 &     d01 &   0.7238 &  3.3149 &        0.4855 &           0.4281 &  0.4568 \\
 Villa Teresa Automatica  &         11 &     d02 &   0.7068 &  4.2355 &        0.0000 &           0.3702 &  0.1851 \\
 Villa Teresa Automatica  &         12 &     d01 &   0.7398 &  3.0712 &        0.6141 &           0.4827 &  0.5484 \\
 Villa Teresa Automatica  &         12 &     d02 &   0.7246 &  4.1100 &        0.0662 &           0.4309 &  0.2485 \\
 Villa Teresa Automatica  &         13 &     d01 &   0.7408 &  3.2475 &        0.5211 &           0.4863 &  0.5037 \\
 Villa Teresa Automatica  &         13 &     d02 &   0.7269 &  3.9591 &        0.1458 &           0.4388 &  0.2923 \\
 Villa Teresa Automatica  &         14 &     d01 &   0.7268 &  3.3943 &        0.4437 &           0.4384 &  0.4410 \\
 Villa Teresa Automatica  &         14 &     d02 &   0.7205 &  3.9245 &        0.1640 &           0.4168 &  0.2904 \\
 Villa Teresa Automatica  &         15 &     d01 &   0.7678 &  3.1922 &        0.5502 &           0.5785 &  0.5644 \\
 Villa Teresa Automatica  &         15 &     d02 &   0.7506 &  4.0793 &        0.0824 &           0.5197 &  0.3010 \\
 Villa Teresa Automatica  &         16 &     d01 &   0.7495 &  3.1515 &        0.5717 &           0.5162 &  0.5439 \\
 Villa Teresa Automatica  &         16 &     d02 &   0.7361 &  4.0367 &        0.1048 &           0.4703 &  0.2876 \\
       Hda Sta Ana Autom  &          1 &     d01 &   0.8998 &  2.3569 &        0.4534 &           0.5640 &  0.5087 \\
       Hda Sta Ana Autom  &          1 &     d02 &   0.8942 &  2.2890 &        0.5188 &           0.4450 &  0.4819 \\
       Hda Sta Ana Autom  &          2 &     d01 &   0.8921 &  2.5978 &        0.2213 &           0.4008 &  0.3111 \\
       Hda Sta Ana Autom  &          2 &     d02 &   0.8812 &  2.4254 &        0.3875 &           0.1692 &  0.2783 \\
       Hda Sta Ana Autom  &          3 &     d01 &   0.9025 &  1.8031 &        0.9871 &           0.6194 &  0.8032 \\
       Hda Sta Ana Autom  &          3 &     d02 &   0.9039 &  2.3572 &        0.4531 &           0.6500 &  0.5516 \\
       Hda Sta Ana Autom  &          4 &     d01 &   0.8834 &  1.9025 &        0.8913 &           0.2155 &  0.5534 \\
       Hda Sta Ana Autom  &          4 &     d02 &   0.8883 &  2.6498 &        0.1712 &           0.3185 &  0.2449 \\
       Hda Sta Ana Autom  &          5 &     d01 &   0.9061 &  2.0620 &        0.7376 &           0.6966 &  0.7171 \\
       Hda Sta Ana Autom  &          5 &     d02 &   0.9069 &  2.5383 &        0.2786 &           0.7130 &  0.4958 \\
       Hda Sta Ana Autom  &          6 &     d01 &   0.8774 &  2.5425 &        0.2746 &           0.0885 &  0.1815 \\
       Hda Sta Ana Autom  &          6 &     d02 &   0.8732 &  2.8101 &        0.0168 &           0.0000 &  0.0084 \\
       Hda Sta Ana Autom  &          7 &     d01 &   0.9092 &  1.8367 &        0.9546 &           0.7628 &  0.8587 \\
       Hda Sta Ana Autom  &          7 &     d02 &   0.9152 &  2.5429 &        0.2742 &           0.8896 &  0.5819 \\
       Hda Sta Ana Autom  &          8 &     d01 &   0.9056 &  1.8438 &        0.9479 &           0.6867 &  0.8173 \\
       Hda Sta Ana Autom  &          8 &     d02 &   0.9107 &  2.8063 &        0.0204 &           0.7938 &  0.4071 \\
       Hda Sta Ana Autom  &          9 &     d01 &   0.9043 &  2.0941 &        0.7066 &           0.6582 &  0.6824 \\
       Hda Sta Ana Autom  &          9 &     d02 &   0.9046 &  2.7732 &        0.0523 &           0.6654 &  0.3588 \\
       Hda Sta Ana Autom  &         10 &     d01 &   0.9017 &  2.1967 &        0.6078 &           0.6040 &  0.6059 \\
       Hda Sta Ana Autom  &         10 &     d02 &   0.8980 &  2.8154 &        0.0117 &           0.5247 &  0.2682 \\
       Hda Sta Ana Autom  &         11 &     d01 &   0.9152 &  1.8299 &        0.9612 &           0.8889 &  0.9251 \\
       Hda Sta Ana Autom  &         11 &     d02 &   0.9122 &  2.7572 &        0.0677 &           0.8260 &  0.4469 \\
       Hda Sta Ana Autom  &         12 &     d01 &   0.9180 &  1.7896 &        1.0000 &           0.9481 &  0.9741 \\
       Hda Sta Ana Autom  &         12 &     d02 &   0.9204 &  2.7367 &        0.0875 &           1.0000 &  0.5437 \\
       Hda Sta Ana Autom  &         13 &     d01 &   0.9014 &  2.1373 &        0.6651 &           0.5960 &  0.6306 \\
       Hda Sta Ana Autom  &         13 &     d02 &   0.9005 &  2.7405 &        0.0838 &           0.5778 &  0.3308 \\
       Hda Sta Ana Autom  &         14 &     d01 &   0.8847 &  2.3516 &        0.4586 &           0.2422 &  0.3504 \\
       Hda Sta Ana Autom  &         14 &     d02 &   0.8848 &  2.8275 &        0.0000 &           0.2451 &  0.1225 \\
       Hda Sta Ana Autom  &         15 &     d01 &   0.9056 &  1.9557 &        0.8400 &           0.6853 &  0.7626 \\
       Hda Sta Ana Autom  &         15 &     d02 &   0.9060 &  2.6658 &        0.1558 &           0.6939 &  0.4248 \\
       Hda Sta Ana Autom  &         16 &     d01 &   0.8897 &  2.1482 &        0.6545 &           0.3496 &  0.5021 \\
       Hda Sta Ana Autom  &         16 &     d02 &   0.8886 &  2.7164 &        0.1071 &           0.3268 &  0.2169 \\
        Nueva Generacion  &          1 &     d01 &   0.9384 &  2.0121 &        0.9802 &           1.0000 &  0.9901 \\
        Nueva Generacion  &          1 &     d02 &   0.9368 &  2.1941 &        0.8591 &           0.9832 &  0.9212 \\
        Nueva Generacion  &          2 &     d01 &   0.9287 &  2.3548 &        0.7521 &           0.8980 &  0.8250 \\
        Nueva Generacion  &          2 &     d02 &   0.9308 &  2.4644 &        0.6791 &           0.9196 &  0.7993 \\
        Nueva Generacion  &          3 &     d01 &   0.9145 &  1.9824 &        1.0000 &           0.7487 &  0.8743 \\
        Nueva Generacion  &          3 &     d02 &   0.9192 &  2.4738 &        0.6728 &           0.7982 &  0.7355 \\
        Nueva Generacion  &          4 &     d01 &   0.9180 &  2.0929 &        0.9264 &           0.7854 &  0.8559 \\
        Nueva Generacion  &          4 &     d02 &   0.9153 &  2.6840 &        0.5328 &           0.7569 &  0.6449 \\
        Nueva Generacion  &          5 &     d01 &   0.9027 &  2.4026 &        0.7202 &           0.6242 &  0.6722 \\
        Nueva Generacion  &          5 &     d02 &   0.8991 &  2.8545 &        0.4193 &           0.5855 &  0.5024 \\
        Nueva Generacion  &          6 &     d01 &   0.8731 &  2.9105 &        0.3820 &           0.3117 &  0.3468 \\
        Nueva Generacion  &          6 &     d02 &   0.8724 &  3.2140 &        0.1799 &           0.3049 &  0.2424 \\
        Nueva Generacion  &          7 &     d01 &   0.9138 &  2.2710 &        0.8078 &           0.7404 &  0.7741 \\
        Nueva Generacion  &          7 &     d02 &   0.9113 &  2.8173 &        0.4440 &           0.7145 &  0.5793 \\
        Nueva Generacion  &          8 &     d01 &   0.9249 &  2.2692 &        0.8090 &           0.8576 &  0.8333 \\
        Nueva Generacion  &          8 &     d02 &   0.9205 &  2.8925 &        0.3940 &           0.8113 &  0.6026 \\
        Nueva Generacion  &          9 &     d01 &   0.8916 &  2.7211 &        0.5081 &           0.5064 &  0.5072 \\
        Nueva Generacion  &          9 &     d02 &   0.8885 &  3.1638 &        0.2133 &           0.4740 &  0.3437 \\
        Nueva Generacion  &         10 &     d01 &   0.8435 &  3.1409 &        0.2286 &           0.0000 &  0.1143 \\
        Nueva Generacion  &         10 &     d02 &   0.8447 &  3.4841 &        0.0000 &           0.0127 &  0.0064 \\
        Nueva Generacion  &         11 &     d01 &   0.8926 &  2.5415 &        0.6277 &           0.5172 &  0.5724 \\
        Nueva Generacion  &         11 &     d02 &   0.8922 &  3.0226 &        0.3073 &           0.5134 &  0.4104 \\
        Nueva Generacion  &         12 &     d01 &   0.8997 &  2.4736 &        0.6729 &           0.5922 &  0.6326 \\
        Nueva Generacion  &         12 &     d02 &   0.9058 &  2.8988 &        0.3898 &           0.6565 &  0.5231 \\
        Nueva Generacion  &         13 &     d01 &   0.8852 &  2.7049 &        0.5189 &           0.4395 &  0.4792 \\
        Nueva Generacion  &         13 &     d02 &   0.8900 &  3.0262 &        0.3049 &           0.4896 &  0.3972 \\
        Nueva Generacion  &         14 &     d01 &   0.8602 &  2.9847 &        0.3326 &           0.1764 &  0.2545 \\
        Nueva Generacion  &         14 &     d02 &   0.8655 &  3.2276 &        0.1708 &           0.2321 &  0.2014 \\
        Nueva Generacion  &         15 &     d01 &   0.8938 &  2.4684 &        0.6764 &           0.5297 &  0.6031 \\
        Nueva Generacion  &         15 &     d02 &   0.8954 &  2.8882 &        0.3968 &           0.5471 &  0.4720 \\
        Nueva Generacion  &         16 &     d01 &   0.8819 &  2.5865 &        0.5977 &           0.4045 &  0.5011 \\
        Nueva Generacion  &         16 &     d02 &   0.8887 &  2.9028 &        0.3871 &           0.4766 &  0.4319 \\
          Apto El Dorado  &          1 &     d01 &   0.9496 &  3.4253 &        0.6702 &           1.0000 &  0.8351 \\
          Apto El Dorado  &          1 &     d02 &   0.9382 &  3.4137 &        0.6784 &           0.8312 &  0.7548 \\
          Apto El Dorado  &          2 &     d01 &   0.9415 &  3.7923 &        0.4097 &           0.8801 &  0.6449 \\
          Apto El Dorado  &          2 &     d02 &   0.9405 &  3.8268 &        0.3852 &           0.8648 &  0.6250 \\
          Apto El Dorado  &          3 &     d01 &   0.9280 &  2.9605 &        1.0000 &           0.6796 &  0.8398 \\
          Apto El Dorado  &          3 &     d02 &   0.9289 &  3.1559 &        0.8613 &           0.6940 &  0.7777 \\
          Apto El Dorado  &          4 &     d01 &   0.9079 &  3.2287 &        0.8097 &           0.3828 &  0.5963 \\
          Apto El Dorado  &          4 &     d02 &   0.9020 &  3.5518 &        0.5804 &           0.2946 &  0.4375 \\
          Apto El Dorado  &          5 &     d01 &   0.9275 &  3.6109 &        0.5385 &           0.6724 &  0.6054 \\
          Apto El Dorado  &          5 &     d02 &   0.9270 &  3.8145 &        0.3940 &           0.6649 &  0.5294 \\
          Apto El Dorado  &          6 &     d01 &   0.8875 &  4.2067 &        0.1156 &           0.0801 &  0.0979 \\
          Apto El Dorado  &          6 &     d02 &   0.8927 &  4.3696 &        0.0000 &           0.1579 &  0.0789 \\
          Apto El Dorado  &          7 &     d01 &   0.9185 &  3.2607 &        0.7870 &           0.5395 &  0.6632 \\
          Apto El Dorado  &          7 &     d02 &   0.9162 &  3.5403 &        0.5885 &           0.5057 &  0.5471 \\
          Apto El Dorado  &          8 &     d01 &   0.9269 &  3.2232 &        0.8136 &           0.6630 &  0.7383 \\
          Apto El Dorado  &          8 &     d02 &   0.9249 &  3.4234 &        0.6715 &           0.6340 &  0.6528 \\
          Apto El Dorado  &          9 &     d01 &   0.9348 &  3.5255 &        0.5991 &           0.7803 &  0.6897 \\
          Apto El Dorado  &          9 &     d02 &   0.9347 &  3.7378 &        0.4484 &           0.7795 &  0.6139 \\
          Apto El Dorado  &         10 &     d01 &   0.8821 &  3.8931 &        0.3382 &           0.0000 &  0.1691 \\
          Apto El Dorado  &         10 &     d02 &   0.8876 &  4.0435 &        0.2315 &           0.0820 &  0.1568 \\
          Apto El Dorado  &         11 &     d01 &   0.9273 &  3.1090 &        0.8946 &           0.6690 &  0.7818 \\
          Apto El Dorado  &         11 &     d02 &   0.9261 &  3.3323 &        0.7362 &           0.6515 &  0.6938 \\
          Apto El Dorado  &         12 &     d01 &   0.9266 &  3.1632 &        0.8561 &           0.6594 &  0.7578 \\
          Apto El Dorado  &         12 &     d02 &   0.9272 &  3.3614 &        0.7155 &           0.6679 &  0.6917 \\
          Apto El Dorado  &         13 &     d01 &   0.9264 &  3.3800 &        0.7023 &           0.6569 &  0.6796 \\
          Apto El Dorado  &         13 &     d02 &   0.9279 &  3.5479 &        0.5831 &           0.6792 &  0.6312 \\
          Apto El Dorado  &         14 &     d01 &   0.8963 &  3.6803 &        0.4892 &           0.2103 &  0.3498 \\
          Apto El Dorado  &         14 &     d02 &   0.8986 &  3.8373 &        0.3778 &           0.2450 &  0.3114 \\
          Apto El Dorado  &         15 &     d01 &   0.9259 &  3.1799 &        0.8443 &           0.6494 &  0.7469 \\
          Apto El Dorado  &         15 &     d02 &   0.9251 &  3.3661 &        0.7121 &           0.6374 &  0.6748 \\
          Apto El Dorado  &         16 &     d01 &   0.9120 &  3.2829 &        0.7713 &           0.4428 &  0.6070 \\
          Apto El Dorado  &         16 &     d02 &   0.9096 &  3.4587 &        0.6464 &           0.4082 &  0.5273 \\
           Univ Nacional  &          1 &     d01 &   0.9597 &  3.5140 &        0.4445 &           0.9619 &  0.7032 \\
           Univ Nacional  &          1 &     d02 &   0.9537 &  3.4515 &        0.4769 &           0.8973 &  0.6871 \\
           Univ Nacional  &          2 &     d01 &   0.9632 &  3.7448 &        0.3252 &           1.0000 &  0.6626 \\
           Univ Nacional  &          2 &     d02 &   0.9618 &  3.6123 &        0.3937 &           0.9850 &  0.6894 \\
           Univ Nacional  &          3 &     d01 &   0.8978 &  2.5772 &        0.9290 &           0.2961 &  0.6125 \\
           Univ Nacional  &          3 &     d02 &   0.8913 &  3.0936 &        0.6619 &           0.2267 &  0.4443 \\
           Univ Nacional  &          4 &     d01 &   0.9214 &  2.4398 &        1.0000 &           0.5506 &  0.7753 \\
           Univ Nacional  &          4 &     d02 &   0.9225 &  3.1541 &        0.6306 &           0.5623 &  0.5965 \\
           Univ Nacional  &          5 &     d01 &   0.9191 &  3.2919 &        0.5594 &           0.5255 &  0.5424 \\
           Univ Nacional  &          5 &     d02 &   0.9261 &  3.6788 &        0.3593 &           0.6007 &  0.4800 \\
           Univ Nacional  &          6 &     d01 &   0.8702 &  4.1819 &        0.0992 &           0.0000 &  0.0496 \\
           Univ Nacional  &          6 &     d02 &   0.8846 &  4.3737 &        0.0000 &           0.1551 &  0.0775 \\
           Univ Nacional  &          7 &     d01 &   0.9113 &  2.8103 &        0.8084 &           0.4419 &  0.6252 \\
           Univ Nacional  &          7 &     d02 &   0.9121 &  3.2640 &        0.5738 &           0.4507 &  0.5122 \\
           Univ Nacional  &          8 &     d01 &   0.9183 &  2.6149 &        0.9095 &           0.5170 &  0.7132 \\
           Univ Nacional  &          8 &     d02 &   0.9180 &  3.3861 &        0.5107 &           0.5134 &  0.5120 \\
           Univ Nacional  &          9 &     d01 &   0.9210 &  3.2077 &        0.6029 &           0.5456 &  0.5743 \\
           Univ Nacional  &          9 &     d02 &   0.9255 &  3.6972 &        0.3498 &           0.5947 &  0.4723 \\
           Univ Nacional  &         10 &     d01 &   0.8739 &  3.7176 &        0.3392 &           0.0400 &  0.1896 \\
           Univ Nacional  &         10 &     d02 &   0.8784 &  4.1093 &        0.1367 &           0.0881 &  0.1124 \\
           Univ Nacional  &         11 &     d01 &   0.9074 &  2.8027 &        0.8123 &           0.4003 &  0.6063 \\
           Univ Nacional  &         11 &     d02 &   0.9137 &  3.3547 &        0.5269 &           0.4678 &  0.4974 \\
           Univ Nacional  &         12 &     d01 &   0.9099 &  2.7854 &        0.8213 &           0.4265 &  0.6239 \\
           Univ Nacional  &         12 &     d02 &   0.9155 &  3.3799 &        0.5139 &           0.4866 &  0.5003 \\
           Univ Nacional  &         13 &     d01 &   0.9054 &  3.2975 &        0.5565 &           0.3786 &  0.4676 \\
           Univ Nacional  &         13 &     d02 &   0.9097 &  3.7766 &        0.3088 &           0.4244 &  0.3666 \\
           Univ Nacional  &         14 &     d01 &   0.8775 &  3.6645 &        0.3667 &           0.0779 &  0.2223 \\
           Univ Nacional  &         14 &     d02 &   0.8845 &  4.0408 &        0.1721 &           0.1533 &  0.1627 \\
           Univ Nacional  &         15 &     d01 &   0.9064 &  2.9971 &        0.7118 &           0.3894 &  0.5506 \\
           Univ Nacional  &         15 &     d02 &   0.9133 &  3.4714 &        0.4666 &           0.4631 &  0.4648 \\
           Univ Nacional  &         16 &     d01 &   0.8894 &  3.1578 &        0.6287 &           0.2066 &  0.4177 \\
           Univ Nacional  &         16 &     d02 &   0.9054 &  3.5605 &        0.4205 &           0.3781 &  0.3993 \\
  Esc La Union Automatica &          1 &     d01 &   0.9240 &  3.7811 &        0.5051 &           0.9969 &  0.7510 \\
  Esc La Union Automatica &          1 &     d02 &   0.9244 &  2.4737 &        1.0000 &           1.0000 &  1.0000 \\
  Esc La Union Automatica &          2 &     d01 &   0.8868 &  4.6411 &        0.1796 &           0.7240 &  0.4518 \\
  Esc La Union Automatica &          2 &     d02 &   0.8729 &  3.2064 &        0.7226 &           0.6223 &  0.6725 \\
  Esc La Union Automatica &          3 &     d01 &   0.8417 &  3.9185 &        0.4531 &           0.3934 &  0.4232 \\
  Esc La Union Automatica &          3 &     d02 &   0.8585 &  3.4354 &        0.6360 &           0.5168 &  0.5764 \\
  Esc La Union Automatica &          4 &     d01 &   0.8082 &  3.6119 &        0.5692 &           0.1479 &  0.3585 \\
  Esc La Union Automatica &          4 &     d02 &   0.7948 &  2.9523 &        0.8188 &           0.0497 &  0.4343 \\
  Esc La Union Automatica &          5 &     d01 &   0.8199 &  4.3150 &        0.3030 &           0.2335 &  0.2683 \\
  Esc La Union Automatica &          5 &     d02 &   0.8139 &  3.3022 &        0.6864 &           0.1898 &  0.4381 \\
  Esc La Union Automatica &          6 &     d01 &   0.7968 &  5.1155 &        0.0000 &           0.0640 &  0.0320 \\
  Esc La Union Automatica &          6 &     d02 &   0.7880 &  3.8560 &        0.4768 &           0.0000 &  0.2384 \\
  Esc La Union Automatica &          7 &     d01 &   0.8237 &  4.2690 &        0.3204 &           0.2615 &  0.2910 \\
  Esc La Union Automatica &          7 &     d02 &   0.8110 &  3.3796 &        0.6571 &           0.1680 &  0.4125 \\
  Esc La Union Automatica &          8 &     d01 &   0.8354 &  4.0200 &        0.4147 &           0.3473 &  0.3810 \\
  Esc La Union Automatica &          8 &     d02 &   0.8091 &  3.1440 &        0.7463 &           0.1547 &  0.4505 \\
  Esc La Union Automatica &          9 &     d01 &   0.8206 &  4.6039 &        0.1937 &           0.2386 &  0.2161 \\
  Esc La Union Automatica &          9 &     d02 &   0.8014 &  3.5414 &        0.5959 &           0.0980 &  0.3469 \\
  Esc La Union Automatica &         10 &     d01 &   0.8317 &  4.8618 &        0.0960 &           0.3203 &  0.2082 \\
  Esc La Union Automatica &         10 &     d02 &   0.8079 &  3.7035 &        0.5345 &           0.1454 &  0.3400 \\
  Esc La Union Automatica &         11 &     d01 &   0.8336 &  4.1547 &        0.3637 &           0.3342 &  0.3489 \\
  Esc La Union Automatica &         11 &     d02 &   0.8156 &  3.2702 &        0.6985 &           0.2021 &  0.4503 \\
  Esc La Union Automatica &         12 &     d01 &   0.8450 &  3.9854 &        0.4278 &           0.4174 &  0.4226 \\
  Esc La Union Automatica &         12 &     d02 &   0.8308 &  3.0797 &        0.7706 &           0.3133 &  0.5420 \\
  Esc La Union Automatica &         13 &     d01 &   0.8458 &  4.4919 &        0.2361 &           0.4235 &  0.3298 \\
  Esc La Union Automatica &         13 &     d02 &   0.8285 &  3.2494 &        0.7064 &           0.2964 &  0.5014 \\
  Esc La Union Automatica &         14 &     d01 &   0.8166 &  4.7292 &        0.1462 &           0.2095 &  0.1779 \\
  Esc La Union Automatica &         14 &     d02 &   0.7956 &  3.5134 &        0.6065 &           0.0551 &  0.3308 \\
  Esc La Union Automatica &         15 &     d01 &   0.8394 &  4.3183 &        0.3018 &           0.3768 &  0.3393 \\
  Esc La Union Automatica &         15 &     d02 &   0.8203 &  3.3513 &        0.6678 &           0.2366 &  0.4522 \\
  Esc La Union Automatica &         16 &     d01 &   0.8442 &  4.2760 &        0.3178 &           0.4115 &  0.3646 \\
  Esc La Union Automatica &         16 &     d02 &   0.8264 &  3.2349 &        0.7119 &           0.2813 &  0.4966 \\

\label{tab:estaciones_tiempo_wrf}

\end{longtable}








\textbf{Caso 4}

\begin{longtable}{p{2cm}rrrrrrrrrr}
\caption{Resultados de las diferentes simulaciones.}
\label{tab:estaciones_tiempo_wrf}\\
\hline
   Nombre &  Simulación & Dominio &  Pearson &     RMSE &  $RMSE_{esc}$ &    $Pearson_{esc}$ &      ET \\
   
\midrule
\endhead
\midrule
\multicolumn{3}{r}{{Continúa en la siguiente página.}} \\
\midrule
\endfoot

\bottomrule
\endlastfoot

        La Capilla Autom  &          1 &     d01 &   0.9233 &  4.7745 &        0.0929 &           1.0000 &  0.5464 \\
        La Capilla Autom  &          1 &     d02 &   0.9181 &  1.7123 &        1.0000 &           0.9457 &  0.9728 \\
        La Capilla Autom  &          2 &     d01 &   0.9044 &  5.0881 &        0.0000 &           0.8044 &  0.4022 \\
        La Capilla Autom  &          2 &     d02 &   0.8916 &  2.0806 &        0.8909 &           0.6725 &  0.7817 \\
        La Capilla Autom  &          3 &     d01 &   0.8995 &  4.0522 &        0.3069 &           0.7540 &  0.5304 \\
        La Capilla Autom  &          3 &     d02 &   0.9011 &  1.9860 &        0.9189 &           0.7700 &  0.8445 \\
        La Capilla Autom  &          4 &     d01 &   0.8788 &  3.5272 &        0.4624 &           0.5407 &  0.5015 \\
        La Capilla Autom  &          4 &     d02 &   0.8792 &  2.4068 &        0.7943 &           0.5446 &  0.6694 \\
        La Capilla Autom  &          5 &     d01 &   0.8640 &  4.0563 &        0.3056 &           0.3874 &  0.3465 \\
        La Capilla Autom  &          5 &     d02 &   0.8655 &  2.4196 &        0.7905 &           0.4035 &  0.5970 \\
        La Capilla Autom  &          6 &     d01 &   0.8320 &  4.5587 &        0.1568 &           0.0574 &  0.1071 \\
        La Capilla Autom  &          6 &     d02 &   0.8264 &  2.6386 &        0.7256 &           0.0000 &  0.3628 \\
        La Capilla Autom  &          7 &     d01 &   0.8718 &  3.9274 &        0.3438 &           0.4680 &  0.4059 \\
        La Capilla Autom  &          7 &     d02 &   0.8717 &  2.2270 &        0.8475 &           0.4668 &  0.6571 \\
        La Capilla Autom  &          8 &     d01 &   0.8696 &  3.8333 &        0.3717 &           0.4452 &  0.4085 \\
        La Capilla Autom  &          8 &     d02 &   0.8782 &  2.2360 &        0.8448 &           0.5342 &  0.6895 \\
        La Capilla Autom  &          9 &     d01 &   0.8656 &  4.2152 &        0.2586 &           0.4040 &  0.3313 \\
        La Capilla Autom  &          9 &     d02 &   0.8694 &  2.2983 &        0.8264 &           0.4437 &  0.6351 \\
        La Capilla Autom  &         10 &     d01 &   0.8409 &  4.5241 &        0.1671 &           0.1497 &  0.1584 \\
        La Capilla Autom  &         10 &     d02 &   0.8414 &  2.5031 &        0.7657 &           0.1546 &  0.4602 \\
        La Capilla Autom  &         11 &     d01 &   0.8836 &  3.9502 &        0.3371 &           0.5902 &  0.4636 \\
        La Capilla Autom  &         11 &     d02 &   0.8815 &  2.2063 &        0.8536 &           0.5681 &  0.7109 \\
        La Capilla Autom  &         12 &     d01 &   0.8783 &  3.8357 &        0.3710 &           0.5350 &  0.4530 \\
        La Capilla Autom  &         12 &     d02 &   0.8862 &  2.2381 &        0.8442 &           0.6166 &  0.7304 \\
        La Capilla Autom  &         13 &     d01 &   0.8663 &  4.2637 &        0.2442 &           0.4114 &  0.3278 \\
        La Capilla Autom  &         13 &     d02 &   0.8685 &  2.3067 &        0.8239 &           0.4342 &  0.6290 \\
        La Capilla Autom  &         14 &     d01 &   0.8575 &  4.3822 &        0.2091 &           0.3206 &  0.2649 \\
        La Capilla Autom  &         14 &     d02 &   0.8620 &  2.3569 &        0.8091 &           0.3665 &  0.5878 \\
        La Capilla Autom  &         15 &     d01 &   0.8682 &  4.0032 &        0.3214 &           0.4314 &  0.3764 \\
        La Capilla Autom  &         15 &     d02 &   0.8693 &  2.3068 &        0.8239 &           0.4424 &  0.6332 \\
        La Capilla Autom  &         16 &     d01 &   0.8677 &  3.8852 &        0.3563 &           0.4258 &  0.3911 \\
        La Capilla Autom  &         16 &     d02 &   0.8666 &  2.3566 &        0.8091 &           0.4142 &  0.6117 \\
 Pmo Rabanal Automatica   &          1 &     d01 &   0.6715 &  6.5348 &        0.1992 &           1.0000 &  0.5996 \\
 Pmo Rabanal Automatica   &          1 &     d02 &   0.5905 &  4.8832 &        0.9199 &           0.5483 &  0.7341 \\
 Pmo Rabanal Automatica   &          2 &     d01 &   0.5872 &  6.2553 &        0.3212 &           0.5297 &  0.4254 \\
 Pmo Rabanal Automatica   &          2 &     d02 &   0.4923 &  4.7050 &        0.9976 &           0.0000 &  0.4988 \\
 Pmo Rabanal Automatica   &          3 &     d01 &   0.6551 &  6.6361 &        0.1551 &           0.9087 &  0.5319 \\
 Pmo Rabanal Automatica   &          3 &     d02 &   0.6021 &  4.9055 &        0.9101 &           0.6129 &  0.7615 \\
 Pmo Rabanal Automatica   &          4 &     d01 &   0.6517 &  6.9908 &        0.0003 &           0.8899 &  0.4451 \\
 Pmo Rabanal Automatica   &          4 &     d02 &   0.6163 &  5.2052 &        0.7794 &           0.6924 &  0.7359 \\
 Pmo Rabanal Automatica   &          5 &     d01 &   0.6244 &  6.5959 &        0.1726 &           0.7374 &  0.4550 \\
 Pmo Rabanal Automatica   &          5 &     d02 &   0.6018 &  4.8805 &        0.9210 &           0.6113 &  0.7662 \\
 Pmo Rabanal Automatica   &          6 &     d01 &   0.5087 &  6.2976 &        0.3028 &           0.0918 &  0.1973 \\
 Pmo Rabanal Automatica   &          6 &     d02 &   0.5089 &  4.6996 &        1.0000 &           0.0925 &  0.5463 \\
 Pmo Rabanal Automatica   &          7 &     d01 &   0.6134 &  6.7258 &        0.1159 &           0.6759 &  0.3959 \\
 Pmo Rabanal Automatica   &          7 &     d02 &   0.6113 &  5.0074 &        0.8657 &           0.6642 &  0.7649 \\
 Pmo Rabanal Automatica   &          8 &     d01 &   0.5884 &  6.9915 &        0.0000 &           0.5366 &  0.2683 \\
 Pmo Rabanal Automatica   &          8 &     d02 &   0.5728 &  5.2516 &        0.7591 &           0.4493 &  0.6042 \\
 Pmo Rabanal Automatica   &          9 &     d01 &   0.6120 &  6.6943 &        0.1297 &           0.6683 &  0.3990 \\
 Pmo Rabanal Automatica   &          9 &     d02 &   0.5962 &  5.0045 &        0.8669 &           0.5799 &  0.7234 \\
 Pmo Rabanal Automatica   &         10 &     d01 &   0.6094 &  6.5439 &        0.1953 &           0.6534 &  0.4244 \\
 Pmo Rabanal Automatica   &         10 &     d02 &   0.5957 &  4.9017 &        0.9118 &           0.5770 &  0.7444 \\
 Pmo Rabanal Automatica   &         11 &     d01 &   0.6699 &  6.7923 &        0.0869 &           0.9915 &  0.5392 \\
 Pmo Rabanal Automatica   &         11 &     d02 &   0.6549 &  4.9726 &        0.8809 &           0.9078 &  0.8943 \\
 Pmo Rabanal Automatica   &         12 &     d01 &   0.6634 &  6.8981 &        0.0407 &           0.9550 &  0.4978 \\
 Pmo Rabanal Automatica   &         12 &     d02 &   0.6476 &  5.1142 &        0.8191 &           0.8668 &  0.8430 \\
 Pmo Rabanal Automatica   &         13 &     d01 &   0.6379 &  6.5447 &        0.1949 &           0.8127 &  0.5038 \\
 Pmo Rabanal Automatica   &         13 &     d02 &   0.6119 &  4.8527 &        0.9332 &           0.6678 &  0.8005 \\
 Pmo Rabanal Automatica   &         14 &     d01 &   0.6279 &  6.4880 &        0.2197 &           0.7568 &  0.4883 \\
 Pmo Rabanal Automatica   &         14 &     d02 &   0.5890 &  4.8438 &        0.9371 &           0.5400 &  0.7385 \\
 Pmo Rabanal Automatica   &         15 &     d01 &   0.6528 &  6.6731 &        0.1389 &           0.8958 &  0.5173 \\
 Pmo Rabanal Automatica   &         15 &     d02 &   0.6370 &  4.9271 &        0.9007 &           0.8075 &  0.8541 \\
 Pmo Rabanal Automatica   &         16 &     d01 &   0.6487 &  6.6179 &        0.1630 &           0.8730 &  0.5180 \\
 Pmo Rabanal Automatica   &         16 &     d02 &   0.6412 &  4.8523 &        0.9334 &           0.8313 &  0.8823 \\
    Chinavita Automatica  &          1 &     d01 &   0.8614 &  4.7336 &        0.1054 &           0.9735 &  0.5395 \\
    Chinavita Automatica  &          1 &     d02 &   0.8641 &  2.0653 &        1.0000 &           1.0000 &  1.0000 \\
    Chinavita Automatica  &          2 &     d01 &   0.8488 &  5.0481 &        0.0000 &           0.8468 &  0.4234 \\
    Chinavita Automatica  &          2 &     d02 &   0.8499 &  2.2428 &        0.9405 &           0.8579 &  0.8992 \\
    Chinavita Automatica  &          3 &     d01 &   0.8473 &  4.1390 &        0.3048 &           0.8324 &  0.5686 \\
    Chinavita Automatica  &          3 &     d02 &   0.8461 &  2.1011 &        0.9880 &           0.8206 &  0.9043 \\
    Chinavita Automatica  &          4 &     d01 &   0.8262 &  3.8011 &        0.4181 &           0.6213 &  0.5197 \\
    Chinavita Automatica  &          4 &     d02 &   0.8052 &  2.4080 &        0.8851 &           0.4116 &  0.6484 \\
    Chinavita Automatica  &          5 &     d01 &   0.8231 &  4.2368 &        0.2720 &           0.5900 &  0.4310 \\
    Chinavita Automatica  &          5 &     d02 &   0.8107 &  2.4406 &        0.8742 &           0.4664 &  0.6703 \\
    Chinavita Automatica  &          6 &     d01 &   0.8084 &  4.7434 &        0.1022 &           0.4434 &  0.2728 \\
    Chinavita Automatica  &          6 &     d02 &   0.7934 &  2.5539 &        0.8362 &           0.2931 &  0.5647 \\
    Chinavita Automatica  &          7 &     d01 &   0.8432 &  3.9922 &        0.3540 &           0.7914 &  0.5727 \\
    Chinavita Automatica  &          7 &     d02 &   0.8378 &  2.2942 &        0.9233 &           0.7372 &  0.8302 \\
    Chinavita Automatica  &          8 &     d01 &   0.8350 &  3.9613 &        0.3644 &           0.7089 &  0.5366 \\
    Chinavita Automatica  &          8 &     d02 &   0.8335 &  2.3487 &        0.9050 &           0.6940 &  0.7995 \\
    Chinavita Automatica  &          9 &     d01 &   0.8215 &  4.4006 &        0.2171 &           0.5743 &  0.3957 \\
    Chinavita Automatica  &          9 &     d02 &   0.8098 &  2.4751 &        0.8626 &           0.4570 &  0.6598 \\
    Chinavita Automatica  &         10 &     d01 &   0.7803 &  4.7964 &        0.0844 &           0.1618 &  0.1231 \\
    Chinavita Automatica  &         10 &     d02 &   0.7641 &  2.6853 &        0.7921 &           0.0000 &  0.3961 \\
    Chinavita Automatica  &         11 &     d01 &   0.8234 &  4.2483 &        0.2682 &           0.5936 &  0.4309 \\
    Chinavita Automatica  &         11 &     d02 &   0.8125 &  2.3585 &        0.9017 &           0.4842 &  0.6929 \\
    Chinavita Automatica  &         12 &     d01 &   0.8292 &  4.0981 &        0.3185 &           0.6507 &  0.4846 \\
    Chinavita Automatica  &         12 &     d02 &   0.8242 &  2.3629 &        0.9002 &           0.6014 &  0.7508 \\
    Chinavita Automatica  &         13 &     d01 &   0.8127 &  4.6545 &        0.1320 &           0.4864 &  0.3092 \\
    Chinavita Automatica  &         13 &     d02 &   0.8024 &  2.5144 &        0.8494 &           0.3836 &  0.6165 \\
    Chinavita Automatica  &         14 &     d01 &   0.8083 &  4.7998 &        0.0833 &           0.4417 &  0.2625 \\
    Chinavita Automatica  &         14 &     d02 &   0.7995 &  2.5433 &        0.8398 &           0.3547 &  0.5972 \\
    Chinavita Automatica  &         15 &     d01 &   0.8258 &  4.4159 &        0.2120 &           0.6168 &  0.4144 \\
    Chinavita Automatica  &         15 &     d02 &   0.8137 &  2.3953 &        0.8894 &           0.4964 &  0.6929 \\
    Chinavita Automatica  &         16 &     d01 &   0.8425 &  4.3021 &        0.2501 &           0.7841 &  0.5171 \\
    Chinavita Automatica  &         16 &     d02 &   0.8318 &  2.3698 &        0.8979 &           0.6773 &  0.7876 \\
            Pmo Chingaza  &          1 &     d01 &   0.6820 &  5.7452 &        0.6084 &           0.7162 &  0.6623 \\
            Pmo Chingaza  &          1 &     d02 &   0.7363 &  5.2828 &        0.8812 &           1.0000 &  0.9406 \\
            Pmo Chingaza  &          2 &     d01 &   0.6095 &  5.5391 &        0.7300 &           0.3376 &  0.5338 \\
            Pmo Chingaza  &          2 &     d02 &   0.6432 &  5.0815 &        1.0000 &           0.5132 &  0.7566 \\
            Pmo Chingaza  &          3 &     d01 &   0.6174 &  6.5264 &        0.1475 &           0.3785 &  0.2630 \\
            Pmo Chingaza  &          3 &     d02 &   0.6187 &  6.0188 &        0.4470 &           0.3854 &  0.4162 \\
            Pmo Chingaza  &          4 &     d01 &   0.6335 &  6.6819 &        0.0557 &           0.4625 &  0.2591 \\
            Pmo Chingaza  &          4 &     d02 &   0.6408 &  6.0471 &        0.4303 &           0.5008 &  0.4655 \\
            Pmo Chingaza  &          5 &     d01 &   0.5829 &  6.2964 &        0.2832 &           0.1982 &  0.2407 \\
            Pmo Chingaza  &          5 &     d02 &   0.5737 &  5.7059 &        0.6316 &           0.1502 &  0.3909 \\
            Pmo Chingaza  &          6 &     d01 &   0.5535 &  6.1145 &        0.3905 &           0.0447 &  0.2176 \\
            Pmo Chingaza  &          6 &     d02 &   0.5449 &  5.5653 &        0.7145 &           0.0000 &  0.3573 \\
            Pmo Chingaza  &          7 &     d01 &   0.6157 &  6.6429 &        0.0787 &           0.3695 &  0.2241 \\
            Pmo Chingaza  &          7 &     d02 &   0.6029 &  6.0771 &        0.4126 &           0.3028 &  0.3577 \\
            Pmo Chingaza  &          8 &     d01 &   0.6403 &  6.6269 &        0.0882 &           0.4983 &  0.2932 \\
            Pmo Chingaza  &          8 &     d02 &   0.6251 &  5.9780 &        0.4710 &           0.4189 &  0.4449 \\
            Pmo Chingaza  &          9 &     d01 &   0.6294 &  6.4669 &        0.1826 &           0.4412 &  0.3119 \\
            Pmo Chingaza  &          9 &     d02 &   0.6113 &  5.8250 &        0.5613 &           0.3466 &  0.4540 \\
            Pmo Chingaza  &         10 &     d01 &   0.6335 &  6.4092 &        0.2166 &           0.4629 &  0.3398 \\
            Pmo Chingaza  &         10 &     d02 &   0.6173 &  5.7898 &        0.5821 &           0.3779 &  0.4800 \\
            Pmo Chingaza  &         11 &     d01 &   0.6818 &  6.7620 &        0.0084 &           0.7150 &  0.3617 \\
            Pmo Chingaza  &         11 &     d02 &   0.6597 &  6.1461 &        0.3719 &           0.5998 &  0.4858 \\
            Pmo Chingaza  &         12 &     d01 &   0.6901 &  6.7763 &        0.0000 &           0.7588 &  0.3794 \\
            Pmo Chingaza  &         12 &     d02 &   0.6732 &  6.1310 &        0.3808 &           0.6701 &  0.5254 \\
            Pmo Chingaza  &         13 &     d01 &   0.6728 &  6.4898 &        0.1691 &           0.6679 &  0.4185 \\
            Pmo Chingaza  &         13 &     d02 &   0.6553 &  5.8085 &        0.5710 &           0.5765 &  0.5738 \\
            Pmo Chingaza  &         14 &     d01 &   0.6287 &  6.3818 &        0.2328 &           0.4378 &  0.3353 \\
            Pmo Chingaza  &         14 &     d02 &   0.6102 &  5.7305 &        0.6171 &           0.3410 &  0.4791 \\
            Pmo Chingaza  &         15 &     d01 &   0.6698 &  6.5651 &        0.1246 &           0.6525 &  0.3886 \\
            Pmo Chingaza  &         15 &     d02 &   0.6543 &  5.9304 &        0.4991 &           0.5713 &  0.5352 \\
            Pmo Chingaza  &         16 &     d01 &   0.6475 &  6.5095 &        0.1574 &           0.5357 &  0.3465 \\
            Pmo Chingaza  &         16 &     d02 &   0.6220 &  5.8722 &        0.5334 &           0.4029 &  0.4682 \\
    Bosque Intervenido    &          1 &     d01 &   0.9008 &  1.8275 &        1.0000 &           1.0000 &  1.0000 \\
    Bosque Intervenido    &          1 &     d02 &   0.8962 &  2.0702 &        0.8070 &           0.9598 &  0.8834 \\
    Bosque Intervenido    &          2 &     d01 &   0.8846 &  1.9655 &        0.8903 &           0.8581 &  0.8742 \\
    Bosque Intervenido    &          2 &     d02 &   0.8933 &  2.2591 &        0.6567 &           0.9347 &  0.7957 \\
    Bosque Intervenido    &          3 &     d01 &   0.8615 &  2.6102 &        0.3774 &           0.6552 &  0.5163 \\
    Bosque Intervenido    &          3 &     d02 &   0.8775 &  2.2549 &        0.6601 &           0.7960 &  0.7280 \\
    Bosque Intervenido    &          4 &     d01 &   0.8641 &  2.7066 &        0.3007 &           0.6784 &  0.4895 \\
    Bosque Intervenido    &          4 &     d02 &   0.8718 &  2.2624 &        0.6541 &           0.7459 &  0.7000 \\
    Bosque Intervenido    &          5 &     d01 &   0.8514 &  2.5362 &        0.4363 &           0.5665 &  0.5014 \\
    Bosque Intervenido    &          5 &     d02 &   0.8697 &  2.2952 &        0.6280 &           0.7276 &  0.6778 \\
    Bosque Intervenido    &          6 &     d01 &   0.8454 &  2.5399 &        0.4334 &           0.5140 &  0.4737 \\
    Bosque Intervenido    &          6 &     d02 &   0.8647 &  2.4149 &        0.5328 &           0.6833 &  0.6081 \\
    Bosque Intervenido    &          7 &     d01 &   0.8459 &  2.9052 &        0.1428 &           0.5185 &  0.3307 \\
    Bosque Intervenido    &          7 &     d02 &   0.8586 &  2.4742 &        0.4856 &           0.6294 &  0.5575 \\
    Bosque Intervenido    &          8 &     d01 &   0.8387 &  2.8232 &        0.2080 &           0.4551 &  0.3315 \\
    Bosque Intervenido    &          8 &     d02 &   0.8421 &  2.3448 &        0.5885 &           0.4844 &  0.5365 \\
    Bosque Intervenido    &          9 &     d01 &   0.8351 &  2.7607 &        0.2577 &           0.4230 &  0.3404 \\
    Bosque Intervenido    &          9 &     d02 &   0.8468 &  2.3960 &        0.5478 &           0.5264 &  0.5371 \\
    Bosque Intervenido    &         10 &     d01 &   0.8298 &  2.8342 &        0.1992 &           0.3767 &  0.2880 \\
    Bosque Intervenido    &         10 &     d02 &   0.8419 &  2.5161 &        0.4523 &           0.4826 &  0.4675 \\
    Bosque Intervenido    &         11 &     d01 &   0.8281 &  3.0847 &        0.0000 &           0.3622 &  0.1811 \\
    Bosque Intervenido    &         11 &     d02 &   0.8399 &  2.5356 &        0.4368 &           0.4652 &  0.4510 \\
    Bosque Intervenido    &         12 &     d01 &   0.8388 &  3.0600 &        0.0196 &           0.4555 &  0.2376 \\
    Bosque Intervenido    &         12 &     d02 &   0.8477 &  2.4423 &        0.5110 &           0.5337 &  0.5224 \\
    Bosque Intervenido    &         13 &     d01 &   0.8312 &  2.9185 &        0.1322 &           0.3893 &  0.2607 \\
    Bosque Intervenido    &         13 &     d02 &   0.8455 &  2.4068 &        0.5392 &           0.5144 &  0.5268 \\
    Bosque Intervenido    &         14 &     d01 &   0.8081 &  2.9751 &        0.0872 &           0.1858 &  0.1365 \\
    Bosque Intervenido    &         14 &     d02 &   0.8244 &  2.5454 &        0.4290 &           0.3292 &  0.3791 \\
    Bosque Intervenido    &         15 &     d01 &   0.8217 &  3.0107 &        0.0588 &           0.3058 &  0.1823 \\
    Bosque Intervenido    &         15 &     d02 &   0.8353 &  2.4611 &        0.4961 &           0.4252 &  0.4606 \\
    Bosque Intervenido    &         16 &     d01 &   0.7869 &  3.0601 &        0.0195 &           0.0000 &  0.0098 \\
    Bosque Intervenido    &         16 &     d02 &   0.8010 &  2.5443 &        0.4298 &           0.1240 &  0.2769 \\
     San Cayetano Autom   &          1 &     d01 &   0.9305 &  5.4108 &        0.0340 &           0.8970 &  0.4655 \\
     San Cayetano Autom   &          1 &     d02 &   0.9427 &  3.6840 &        0.5226 &           1.0000 &  0.7613 \\
     San Cayetano Autom   &          2 &     d01 &   0.9111 &  5.5310 &        0.0000 &           0.7344 &  0.3672 \\
     San Cayetano Autom   &          2 &     d02 &   0.9322 &  3.6073 &        0.5443 &           0.9116 &  0.7279 \\
     San Cayetano Autom   &          3 &     d01 &   0.8835 &  4.5362 &        0.2815 &           0.5022 &  0.3918 \\
     San Cayetano Autom   &          3 &     d02 &   0.8963 &  2.8211 &        0.7667 &           0.6098 &  0.6883 \\
     San Cayetano Autom   &          4 &     d01 &   0.8284 &  4.3416 &        0.3365 &           0.0387 &  0.1876 \\
     San Cayetano Autom   &          4 &     d02 &   0.8238 &  2.8161 &        0.7682 &           0.0000 &  0.3841 \\
     San Cayetano Autom   &          5 &     d01 &   0.9095 &  4.5891 &        0.2665 &           0.7208 &  0.4936 \\
     San Cayetano Autom   &          5 &     d02 &   0.9023 &  3.1892 &        0.6626 &           0.6596 &  0.6611 \\
     San Cayetano Autom   &          6 &     d01 &   0.9168 &  4.8554 &        0.1911 &           0.7817 &  0.4864 \\
     San Cayetano Autom   &          6 &     d02 &   0.9016 &  3.5696 &        0.5550 &           0.6544 &  0.6047 \\
     San Cayetano Autom   &          7 &     d01 &   0.8982 &  3.8943 &        0.4631 &           0.6258 &  0.5444 \\
     San Cayetano Autom   &          7 &     d02 &   0.8996 &  2.6188 &        0.8240 &           0.6373 &  0.7306 \\
     San Cayetano Autom   &          8 &     d01 &   0.8959 &  3.7609 &        0.5008 &           0.6062 &  0.5535 \\
     San Cayetano Autom   &          8 &     d02 &   0.8869 &  2.3120 &        0.9108 &           0.5302 &  0.7205 \\
     San Cayetano Autom   &          9 &     d01 &   0.9253 &  4.0419 &        0.4213 &           0.8538 &  0.6375 \\
     San Cayetano Autom   &          9 &     d02 &   0.9117 &  2.7211 &        0.7950 &           0.7392 &  0.7671 \\
     San Cayetano Autom   &         10 &     d01 &   0.9019 &  4.0830 &        0.4097 &           0.6570 &  0.5333 \\
     San Cayetano Autom   &         10 &     d02 &   0.8978 &  2.6334 &        0.8199 &           0.6220 &  0.7209 \\
     San Cayetano Autom   &         11 &     d01 &   0.9014 &  3.5750 &        0.5534 &           0.6524 &  0.6029 \\
     San Cayetano Autom   &         11 &     d02 &   0.8773 &  2.0961 &        0.9719 &           0.4499 &  0.7109 \\
     San Cayetano Autom   &         12 &     d01 &   0.8863 &  3.5053 &        0.5732 &           0.5255 &  0.5493 \\
     San Cayetano Autom   &         12 &     d02 &   0.8414 &  2.0333 &        0.9896 &           0.1481 &  0.5689 \\
     San Cayetano Autom   &         13 &     d01 &   0.8917 &  3.8909 &        0.4640 &           0.5705 &  0.5173 \\
     San Cayetano Autom   &         13 &     d02 &   0.8657 &  2.2236 &        0.9358 &           0.3521 &  0.6440 \\
     San Cayetano Autom   &         14 &     d01 &   0.8667 &  3.9887 &        0.4364 &           0.3605 &  0.3984 \\
     San Cayetano Autom   &         14 &     d02 &   0.8389 &  2.2386 &        0.9316 &           0.1266 &  0.5291 \\
     San Cayetano Autom   &         15 &     d01 &   0.8832 &  3.5684 &        0.5553 &           0.4994 &  0.5273 \\
     San Cayetano Autom   &         15 &     d02 &   0.8452 &  1.9967 &        1.0000 &           0.1794 &  0.5897 \\
     San Cayetano Autom   &         16 &     d01 &   0.8710 &  3.6476 &        0.5329 &           0.3971 &  0.4650 \\
     San Cayetano Autom   &         16 &     d02 &   0.8297 &  2.1921 &        0.9447 &           0.0491 &  0.4969 \\
    Tibaitata Automatica  &          1 &     d01 &   0.9100 &  2.6795 &        0.7815 &           0.9006 &  0.8411 \\
    Tibaitata Automatica  &          1 &     d02 &   0.9198 &  2.4021 &        1.0000 &           1.0000 &  1.0000 \\
    Tibaitata Automatica  &          2 &     d01 &   0.9065 &  2.6863 &        0.7761 &           0.8652 &  0.8207 \\
    Tibaitata Automatica  &          2 &     d02 &   0.9177 &  2.4940 &        0.9276 &           0.9784 &  0.9530 \\
    Tibaitata Automatica  &          3 &     d01 &   0.8360 &  3.2789 &        0.3092 &           0.1494 &  0.2293 \\
    Tibaitata Automatica  &          3 &     d02 &   0.8334 &  3.5148 &        0.1234 &           0.1233 &  0.1233 \\
    Tibaitata Automatica  &          4 &     d01 &   0.8487 &  3.2456 &        0.3354 &           0.2780 &  0.3067 \\
    Tibaitata Automatica  &          4 &     d02 &   0.8428 &  3.4216 &        0.1968 &           0.2189 &  0.2078 \\
    Tibaitata Automatica  &          5 &     d01 &   0.8410 &  3.3070 &        0.2871 &           0.2001 &  0.2436 \\
    Tibaitata Automatica  &          5 &     d02 &   0.8361 &  3.5615 &        0.0866 &           0.1503 &  0.1184 \\
    Tibaitata Automatica  &          6 &     d01 &   0.8260 &  3.4042 &        0.2105 &           0.0480 &  0.1293 \\
    Tibaitata Automatica  &          6 &     d02 &   0.8213 &  3.6714 &        0.0000 &           0.0000 &  0.0000 \\
    Tibaitata Automatica  &          7 &     d01 &   0.8563 &  3.1940 &        0.3761 &           0.3557 &  0.3659 \\
    Tibaitata Automatica  &          7 &     d02 &   0.8494 &  3.4593 &        0.1670 &           0.2858 &  0.2264 \\
    Tibaitata Automatica  &          8 &     d01 &   0.8651 &  3.1294 &        0.4270 &           0.4444 &  0.4357 \\
    Tibaitata Automatica  &          8 &     d02 &   0.8605 &  3.3610 &        0.2446 &           0.3985 &  0.3215 \\
    Tibaitata Automatica  &          9 &     d01 &   0.8749 &  2.9650 &        0.5565 &           0.5446 &  0.5506 \\
    Tibaitata Automatica  &          9 &     d02 &   0.8711 &  3.2086 &        0.3646 &           0.5052 &  0.4349 \\
    Tibaitata Automatica  &         10 &     d01 &   0.8535 &  3.0791 &        0.4666 &           0.3272 &  0.3969 \\
    Tibaitata Automatica  &         10 &     d02 &   0.8514 &  3.3149 &        0.2808 &           0.3053 &  0.2931 \\
    Tibaitata Automatica  &         11 &     d01 &   0.8914 &  3.0050 &        0.5250 &           0.7117 &  0.6184 \\
    Tibaitata Automatica  &         11 &     d02 &   0.8858 &  3.2745 &        0.3127 &           0.6547 &  0.4837 \\
    Tibaitata Automatica  &         12 &     d01 &   0.8929 &  2.9582 &        0.5619 &           0.7267 &  0.6443 \\
    Tibaitata Automatica  &         12 &     d02 &   0.8864 &  3.2418 &        0.3385 &           0.6612 &  0.4999 \\
    Tibaitata Automatica  &         13 &     d01 &   0.8862 &  2.8775 &        0.6254 &           0.6584 &  0.6419 \\
    Tibaitata Automatica  &         13 &     d02 &   0.8824 &  3.1245 &        0.4309 &           0.6198 &  0.5254 \\
    Tibaitata Automatica  &         14 &     d01 &   0.8734 &  2.9110 &        0.5991 &           0.5286 &  0.5638 \\
    Tibaitata Automatica  &         14 &     d02 &   0.8701 &  3.1820 &        0.3855 &           0.4950 &  0.4403 \\
    Tibaitata Automatica  &         15 &     d01 &   0.9023 &  2.8135 &        0.6759 &           0.8227 &  0.7493 \\
    Tibaitata Automatica  &         15 &     d02 &   0.8967 &  3.0770 &        0.4683 &           0.7649 &  0.6166 \\
    Tibaitata Automatica  &         16 &     d01 &   0.8824 &  2.7765 &        0.7050 &           0.6205 &  0.6627 \\
    Tibaitata Automatica  &         16 &     d02 &   0.8793 &  3.0351 &        0.5013 &           0.5889 &  0.5451 \\
      Sta Cruz De Siecha  &          1 &     d01 &   0.9240 &  1.8608 &        0.9918 &           0.9709 &  0.9814 \\
      Sta Cruz De Siecha  &          1 &     d02 &   0.9285 &  2.1741 &        0.8095 &           1.0000 &  0.9048 \\
      Sta Cruz De Siecha  &          2 &     d01 &   0.8690 &  2.3106 &        0.7301 &           0.6169 &  0.6735 \\
      Sta Cruz De Siecha  &          2 &     d02 &   0.8749 &  2.2068 &        0.7906 &           0.6553 &  0.7229 \\
      Sta Cruz De Siecha  &          3 &     d01 &   0.7884 &  2.1622 &        0.8165 &           0.0986 &  0.4575 \\
      Sta Cruz De Siecha  &          3 &     d02 &   0.8191 &  3.1117 &        0.2643 &           0.2961 &  0.2802 \\
      Sta Cruz De Siecha  &          4 &     d01 &   0.8280 &  1.8467 &        1.0000 &           0.3536 &  0.6768 \\
      Sta Cruz De Siecha  &          4 &     d02 &   0.8288 &  3.1005 &        0.2707 &           0.3584 &  0.3146 \\
      Sta Cruz De Siecha  &          5 &     d01 &   0.8148 &  2.2981 &        0.7374 &           0.2687 &  0.5031 \\
      Sta Cruz De Siecha  &          5 &     d02 &   0.8104 &  3.0100 &        0.3234 &           0.2403 &  0.2818 \\
      Sta Cruz De Siecha  &          6 &     d01 &   0.7840 &  2.6972 &        0.5053 &           0.0701 &  0.2877 \\
      Sta Cruz De Siecha  &          6 &     d02 &   0.7731 &  3.0408 &        0.3055 &           0.0000 &  0.1527 \\
      Sta Cruz De Siecha  &          7 &     d01 &   0.7843 &  2.3858 &        0.6864 &           0.0723 &  0.3794 \\
      Sta Cruz De Siecha  &          7 &     d02 &   0.7746 &  3.3782 &        0.1092 &           0.0100 &  0.0596 \\
      Sta Cruz De Siecha  &          8 &     d01 &   0.7979 &  2.1997 &        0.7947 &           0.1595 &  0.4771 \\
      Sta Cruz De Siecha  &          8 &     d02 &   0.7866 &  3.4342 &        0.0767 &           0.0869 &  0.0818 \\
      Sta Cruz De Siecha  &          9 &     d01 &   0.8096 &  2.4697 &        0.6377 &           0.2350 &  0.4363 \\
      Sta Cruz De Siecha  &          9 &     d02 &   0.7868 &  3.3761 &        0.1105 &           0.0881 &  0.0993 \\
      Sta Cruz De Siecha  &         10 &     d01 &   0.8074 &  2.5742 &        0.5769 &           0.2211 &  0.3990 \\
      Sta Cruz De Siecha  &         10 &     d02 &   0.7847 &  3.3410 &        0.1309 &           0.0746 &  0.1027 \\
      Sta Cruz De Siecha  &         11 &     d01 &   0.8168 &  2.3284 &        0.7198 &           0.2813 &  0.5006 \\
      Sta Cruz De Siecha  &         11 &     d02 &   0.8027 &  3.5660 &        0.0000 &           0.1904 &  0.0952 \\
      Sta Cruz De Siecha  &         12 &     d01 &   0.8248 &  2.1429 &        0.8277 &           0.3328 &  0.5802 \\
      Sta Cruz De Siecha  &         12 &     d02 &   0.8145 &  3.5478 &        0.0106 &           0.2662 &  0.1384 \\
      Sta Cruz De Siecha  &         13 &     d01 &   0.8185 &  2.3332 &        0.7170 &           0.2920 &  0.5045 \\
      Sta Cruz De Siecha  &         13 &     d02 &   0.8009 &  3.4003 &        0.0964 &           0.1790 &  0.1377 \\
      Sta Cruz De Siecha  &         14 &     d01 &   0.8130 &  2.3724 &        0.6942 &           0.2566 &  0.4754 \\
      Sta Cruz De Siecha  &         14 &     d02 &   0.7887 &  3.4130 &        0.0890 &           0.1009 &  0.0949 \\
      Sta Cruz De Siecha  &         15 &     d01 &   0.8234 &  2.1712 &        0.8112 &           0.3240 &  0.5676 \\
      Sta Cruz De Siecha  &         15 &     d02 &   0.8114 &  3.4705 &        0.0556 &           0.2466 &  0.1511 \\
      Sta Cruz De Siecha  &         16 &     d01 &   0.8058 &  2.3197 &        0.7248 &           0.2109 &  0.4679 \\
      Sta Cruz De Siecha  &         16 &     d02 &   0.7945 &  3.5082 &        0.0336 &           0.1379 &  0.0857 \\
            Ideam Bogota  &          1 &     d01 &   0.9416 &  6.7015 &        0.2185 &           1.0000 &  0.6093 \\
            Ideam Bogota  &          1 &     d02 &   0.9259 &  6.9662 &        0.0637 &           0.8621 &  0.4629 \\
            Ideam Bogota  &          2 &     d01 &   0.9259 &  6.9681 &        0.0626 &           0.8617 &  0.4621 \\
            Ideam Bogota  &          2 &     d02 &   0.9098 &  7.0750 &        0.0000 &           0.7198 &  0.3599 \\
            Ideam Bogota  &          3 &     d01 &   0.8745 &  5.6021 &        0.8619 &           0.4087 &  0.6353 \\
            Ideam Bogota  &          3 &     d02 &   0.8648 &  5.9252 &        0.6728 &           0.3232 &  0.4980 \\
            Ideam Bogota  &          4 &     d01 &   0.9076 &  5.4747 &        0.9364 &           0.7003 &  0.8184 \\
            Ideam Bogota  &          4 &     d02 &   0.9026 &  5.8690 &        0.7057 &           0.6561 &  0.6809 \\
            Ideam Bogota  &          5 &     d01 &   0.8865 &  6.1286 &        0.5538 &           0.5150 &  0.5344 \\
            Ideam Bogota  &          5 &     d02 &   0.8769 &  6.4596 &        0.3601 &           0.4302 &  0.3951 \\
            Ideam Bogota  &          6 &     d01 &   0.8424 &  6.8063 &        0.1572 &           0.1259 &  0.1416 \\
            Ideam Bogota  &          6 &     d02 &   0.8281 &  7.0746 &        0.0002 &           0.0000 &  0.0001 \\
            Ideam Bogota  &          7 &     d01 &   0.8932 &  5.7190 &        0.7935 &           0.5741 &  0.6838 \\
            Ideam Bogota  &          7 &     d02 &   0.8834 &  6.1057 &        0.5672 &           0.4879 &  0.5275 \\
            Ideam Bogota  &          8 &     d01 &   0.8994 &  5.3660 &        1.0000 &           0.6280 &  0.8140 \\
            Ideam Bogota  &          8 &     d02 &   0.8916 &  5.7341 &        0.7846 &           0.5594 &  0.6720 \\
            Ideam Bogota  &          9 &     d01 &   0.9111 &  5.9557 &        0.6549 &           0.7316 &  0.6933 \\
            Ideam Bogota  &          9 &     d02 &   0.9012 &  6.2758 &        0.4677 &           0.6440 &  0.5558 \\
            Ideam Bogota  &         10 &     d01 &   0.8589 &  6.2961 &        0.4558 &           0.2716 &  0.3637 \\
            Ideam Bogota  &         10 &     d02 &   0.8520 &  6.5252 &        0.3217 &           0.2113 &  0.2665 \\
            Ideam Bogota  &         11 &     d01 &   0.8995 &  5.5107 &        0.9153 &           0.6295 &  0.7724 \\
            Ideam Bogota  &         11 &     d02 &   0.8932 &  5.8093 &        0.7406 &           0.5742 &  0.6574 \\
            Ideam Bogota  &         12 &     d01 &   0.8775 &  5.4195 &        0.9687 &           0.4351 &  0.7019 \\
            Ideam Bogota  &         12 &     d02 &   0.8542 &  5.6382 &        0.8407 &           0.2301 &  0.5354 \\
            Ideam Bogota  &         13 &     d01 &   0.8900 &  5.8839 &        0.6969 &           0.5456 &  0.6213 \\
            Ideam Bogota  &         13 &     d02 &   0.8804 &  6.1030 &        0.5687 &           0.4614 &  0.5151 \\
            Ideam Bogota  &         14 &     d01 &   0.8669 &  6.0113 &        0.6224 &           0.3424 &  0.4824 \\
            Ideam Bogota  &         14 &     d02 &   0.8605 &  6.0354 &        0.6083 &           0.2862 &  0.4473 \\
            Ideam Bogota  &         15 &     d01 &   0.8889 &  5.5055 &        0.9183 &           0.5361 &  0.7272 \\
            Ideam Bogota  &         15 &     d02 &   0.8710 &  5.6263 &        0.8477 &           0.3784 &  0.6130 \\
            Ideam Bogota  &         16 &     d01 &   0.8579 &  5.6412 &        0.8389 &           0.2626 &  0.5508 \\
            Ideam Bogota  &         16 &     d02 &   0.8308 &  5.8106 &        0.7398 &           0.0238 &  0.3818 \\
            Pmo Guerrero  &          1 &     d01 &   0.8614 &  3.6094 &        0.3538 &           1.0000 &  0.6769 \\
            Pmo Guerrero  &          1 &     d02 &   0.8444 &  2.0666 &        0.9703 &           0.8482 &  0.9092 \\
            Pmo Guerrero  &          2 &     d01 &   0.8306 &  3.5574 &        0.3745 &           0.7258 &  0.5501 \\
            Pmo Guerrero  &          2 &     d02 &   0.7820 &  2.4570 &        0.8143 &           0.2933 &  0.5538 \\
            Pmo Guerrero  &          3 &     d01 &   0.8358 &  4.3027 &        0.0767 &           0.7722 &  0.4245 \\
            Pmo Guerrero  &          3 &     d02 &   0.8219 &  2.2440 &        0.8993 &           0.6486 &  0.7740 \\
            Pmo Guerrero  &          4 &     d01 &   0.8006 &  4.3560 &        0.0554 &           0.4588 &  0.2571 \\
            Pmo Guerrero  &          4 &     d02 &   0.8124 &  1.9921 &        1.0000 &           0.5641 &  0.7821 \\
            Pmo Guerrero  &          5 &     d01 &   0.7954 &  4.0817 &        0.1650 &           0.4127 &  0.2888 \\
            Pmo Guerrero  &          5 &     d02 &   0.8095 &  2.3359 &        0.8626 &           0.5379 &  0.7003 \\
            Pmo Guerrero  &          6 &     d01 &   0.7572 &  4.0050 &        0.1957 &           0.0731 &  0.1344 \\
            Pmo Guerrero  &          6 &     d02 &   0.7733 &  2.6714 &        0.7285 &           0.2159 &  0.4722 \\
            Pmo Guerrero  &          7 &     d01 &   0.7788 &  4.4143 &        0.0321 &           0.2653 &  0.1487 \\
            Pmo Guerrero  &          7 &     d02 &   0.7861 &  2.3530 &        0.8558 &           0.3301 &  0.5930 \\
            Pmo Guerrero  &          8 &     d01 &   0.7803 &  4.4771 &        0.0070 &           0.2779 &  0.1425 \\
            Pmo Guerrero  &          8 &     d02 &   0.7875 &  2.3784 &        0.8456 &           0.3425 &  0.5941 \\
            Pmo Guerrero  &          9 &     d01 &   0.7818 &  4.2712 &        0.0893 &           0.2917 &  0.1905 \\
            Pmo Guerrero  &          9 &     d02 &   0.7954 &  2.5198 &        0.7891 &           0.4124 &  0.6008 \\
            Pmo Guerrero  &         10 &     d01 &   0.7636 &  4.2416 &        0.1011 &           0.1297 &  0.1154 \\
            Pmo Guerrero  &         10 &     d02 &   0.7786 &  2.7206 &        0.7089 &           0.2633 &  0.4861 \\
            Pmo Guerrero  &         11 &     d01 &   0.7926 &  4.4697 &        0.0100 &           0.3874 &  0.1987 \\
            Pmo Guerrero  &         11 &     d02 &   0.8018 &  2.5083 &        0.7937 &           0.4699 &  0.6318 \\
            Pmo Guerrero  &         12 &     d01 &   0.7953 &  4.4947 &        0.0000 &           0.4121 &  0.2060 \\
            Pmo Guerrero  &         12 &     d02 &   0.7959 &  2.4270 &        0.8262 &           0.4174 &  0.6218 \\
            Pmo Guerrero  &         13 &     d01 &   0.7795 &  4.2530 &        0.0966 &           0.2714 &  0.1840 \\
            Pmo Guerrero  &         13 &     d02 &   0.7805 &  2.5815 &        0.7645 &           0.2798 &  0.5222 \\
            Pmo Guerrero  &         14 &     d01 &   0.7559 &  4.1995 &        0.1180 &           0.0614 &  0.0897 \\
            Pmo Guerrero  &         14 &     d02 &   0.7490 &  2.7530 &        0.6960 &           0.0000 &  0.3480 \\
            Pmo Guerrero  &         15 &     d01 &   0.7775 &  4.3293 &        0.0661 &           0.2536 &  0.1598 \\
            Pmo Guerrero  &         15 &     d02 &   0.7781 &  2.5005 &        0.7969 &           0.2586 &  0.5277 \\
            Pmo Guerrero  &         16 &     d01 &   0.7635 &  4.2303 &        0.1057 &           0.1288 &  0.1172 \\
            Pmo Guerrero  &         16 &     d02 &   0.7814 &  2.4974 &        0.7981 &           0.2879 &  0.5430 \\
 Villa Teresa Automatica  &          1 &     d01 &   0.8656 &  2.3394 &        1.0000 &           0.9127 &  0.9564 \\
 Villa Teresa Automatica  &          1 &     d02 &   0.8912 &  3.1262 &        0.5850 &           1.0000 &  0.7925 \\
 Villa Teresa Automatica  &          2 &     d01 &   0.8013 &  2.9803 &        0.6620 &           0.6928 &  0.6774 \\
 Villa Teresa Automatica  &          2 &     d02 &   0.8013 &  3.2424 &        0.5238 &           0.6930 &  0.6084 \\
 Villa Teresa Automatica  &          3 &     d01 &   0.7717 &  3.0026 &        0.6502 &           0.5917 &  0.6210 \\
 Villa Teresa Automatica  &          3 &     d02 &   0.7743 &  4.1085 &        0.0670 &           0.6006 &  0.3338 \\
 Villa Teresa Automatica  &          4 &     d01 &   0.7838 &  2.6240 &        0.8499 &           0.6332 &  0.7415 \\
 Villa Teresa Automatica  &          4 &     d02 &   0.7832 &  3.9602 &        0.1452 &           0.6312 &  0.3882 \\
 Villa Teresa Automatica  &          5 &     d01 &   0.6998 &  3.2179 &        0.5367 &           0.3463 &  0.4415 \\
 Villa Teresa Automatica  &          5 &     d02 &   0.6934 &  3.9701 &        0.1400 &           0.3242 &  0.2321 \\
 Villa Teresa Automatica  &          6 &     d01 &   0.6173 &  3.7512 &        0.2554 &           0.0645 &  0.1600 \\
 Villa Teresa Automatica  &          6 &     d02 &   0.5985 &  4.1345 &        0.0533 &           0.0000 &  0.0266 \\
 Villa Teresa Automatica  &          7 &     d01 &   0.6936 &  3.2987 &        0.4941 &           0.3251 &  0.4096 \\
 Villa Teresa Automatica  &          7 &     d02 &   0.6709 &  4.1641 &        0.0377 &           0.2473 &  0.1425 \\
 Villa Teresa Automatica  &          8 &     d01 &   0.6928 &  3.0765 &        0.6113 &           0.3222 &  0.4667 \\
 Villa Teresa Automatica  &          8 &     d02 &   0.6690 &  4.1338 &        0.0536 &           0.2409 &  0.1473 \\
 Villa Teresa Automatica  &          9 &     d01 &   0.6856 &  3.5260 &        0.3742 &           0.2978 &  0.3360 \\
 Villa Teresa Automatica  &          9 &     d02 &   0.6585 &  4.1467 &        0.0468 &           0.2050 &  0.1259 \\
 Villa Teresa Automatica  &         10 &     d01 &   0.6963 &  3.6043 &        0.3329 &           0.3342 &  0.3336 \\
 Villa Teresa Automatica  &         10 &     d02 &   0.6706 &  4.1214 &        0.0602 &           0.2466 &  0.1534 \\
 Villa Teresa Automatica  &         11 &     d01 &   0.7238 &  3.3149 &        0.4855 &           0.4281 &  0.4568 \\
 Villa Teresa Automatica  &         11 &     d02 &   0.7068 &  4.2355 &        0.0000 &           0.3702 &  0.1851 \\
 Villa Teresa Automatica  &         12 &     d01 &   0.7398 &  3.0712 &        0.6141 &           0.4827 &  0.5484 \\
 Villa Teresa Automatica  &         12 &     d02 &   0.7246 &  4.1100 &        0.0662 &           0.4309 &  0.2485 \\
 Villa Teresa Automatica  &         13 &     d01 &   0.7408 &  3.2475 &        0.5211 &           0.4863 &  0.5037 \\
 Villa Teresa Automatica  &         13 &     d02 &   0.7269 &  3.9591 &        0.1458 &           0.4388 &  0.2923 \\
 Villa Teresa Automatica  &         14 &     d01 &   0.7268 &  3.3943 &        0.4437 &           0.4384 &  0.4410 \\
 Villa Teresa Automatica  &         14 &     d02 &   0.7205 &  3.9245 &        0.1640 &           0.4168 &  0.2904 \\
 Villa Teresa Automatica  &         15 &     d01 &   0.7678 &  3.1922 &        0.5502 &           0.5785 &  0.5644 \\
 Villa Teresa Automatica  &         15 &     d02 &   0.7506 &  4.0793 &        0.0824 &           0.5197 &  0.3010 \\
 Villa Teresa Automatica  &         16 &     d01 &   0.7495 &  3.1515 &        0.5717 &           0.5162 &  0.5439 \\
 Villa Teresa Automatica  &         16 &     d02 &   0.7361 &  4.0367 &        0.1048 &           0.4703 &  0.2876 \\
       Hda Sta Ana Autom  &          1 &     d01 &   0.8998 &  2.3569 &        0.4534 &           0.5640 &  0.5087 \\
       Hda Sta Ana Autom  &          1 &     d02 &   0.8942 &  2.2890 &        0.5188 &           0.4450 &  0.4819 \\
       Hda Sta Ana Autom  &          2 &     d01 &   0.8921 &  2.5978 &        0.2213 &           0.4008 &  0.3111 \\
       Hda Sta Ana Autom  &          2 &     d02 &   0.8812 &  2.4254 &        0.3875 &           0.1692 &  0.2783 \\
       Hda Sta Ana Autom  &          3 &     d01 &   0.9025 &  1.8031 &        0.9871 &           0.6194 &  0.8032 \\
       Hda Sta Ana Autom  &          3 &     d02 &   0.9039 &  2.3572 &        0.4531 &           0.6500 &  0.5516 \\
       Hda Sta Ana Autom  &          4 &     d01 &   0.8834 &  1.9025 &        0.8913 &           0.2155 &  0.5534 \\
       Hda Sta Ana Autom  &          4 &     d02 &   0.8883 &  2.6498 &        0.1712 &           0.3185 &  0.2449 \\
       Hda Sta Ana Autom  &          5 &     d01 &   0.9061 &  2.0620 &        0.7376 &           0.6966 &  0.7171 \\
       Hda Sta Ana Autom  &          5 &     d02 &   0.9069 &  2.5383 &        0.2786 &           0.7130 &  0.4958 \\
       Hda Sta Ana Autom  &          6 &     d01 &   0.8774 &  2.5425 &        0.2746 &           0.0885 &  0.1815 \\
       Hda Sta Ana Autom  &          6 &     d02 &   0.8732 &  2.8101 &        0.0168 &           0.0000 &  0.0084 \\
       Hda Sta Ana Autom  &          7 &     d01 &   0.9092 &  1.8367 &        0.9546 &           0.7628 &  0.8587 \\
       Hda Sta Ana Autom  &          7 &     d02 &   0.9152 &  2.5429 &        0.2742 &           0.8896 &  0.5819 \\
       Hda Sta Ana Autom  &          8 &     d01 &   0.9056 &  1.8438 &        0.9479 &           0.6867 &  0.8173 \\
       Hda Sta Ana Autom  &          8 &     d02 &   0.9107 &  2.8063 &        0.0204 &           0.7938 &  0.4071 \\
       Hda Sta Ana Autom  &          9 &     d01 &   0.9043 &  2.0941 &        0.7066 &           0.6582 &  0.6824 \\
       Hda Sta Ana Autom  &          9 &     d02 &   0.9046 &  2.7732 &        0.0523 &           0.6654 &  0.3588 \\
       Hda Sta Ana Autom  &         10 &     d01 &   0.9017 &  2.1967 &        0.6078 &           0.6040 &  0.6059 \\
       Hda Sta Ana Autom  &         10 &     d02 &   0.8980 &  2.8154 &        0.0117 &           0.5247 &  0.2682 \\
       Hda Sta Ana Autom  &         11 &     d01 &   0.9152 &  1.8299 &        0.9612 &           0.8889 &  0.9251 \\
       Hda Sta Ana Autom  &         11 &     d02 &   0.9122 &  2.7572 &        0.0677 &           0.8260 &  0.4469 \\
       Hda Sta Ana Autom  &         12 &     d01 &   0.9180 &  1.7896 &        1.0000 &           0.9481 &  0.9741 \\
       Hda Sta Ana Autom  &         12 &     d02 &   0.9204 &  2.7367 &        0.0875 &           1.0000 &  0.5437 \\
       Hda Sta Ana Autom  &         13 &     d01 &   0.9014 &  2.1373 &        0.6651 &           0.5960 &  0.6306 \\
       Hda Sta Ana Autom  &         13 &     d02 &   0.9005 &  2.7405 &        0.0838 &           0.5778 &  0.3308 \\
       Hda Sta Ana Autom  &         14 &     d01 &   0.8847 &  2.3516 &        0.4586 &           0.2422 &  0.3504 \\
       Hda Sta Ana Autom  &         14 &     d02 &   0.8848 &  2.8275 &        0.0000 &           0.2451 &  0.1225 \\
       Hda Sta Ana Autom  &         15 &     d01 &   0.9056 &  1.9557 &        0.8400 &           0.6853 &  0.7626 \\
       Hda Sta Ana Autom  &         15 &     d02 &   0.9060 &  2.6658 &        0.1558 &           0.6939 &  0.4248 \\
       Hda Sta Ana Autom  &         16 &     d01 &   0.8897 &  2.1482 &        0.6545 &           0.3496 &  0.5021 \\
       Hda Sta Ana Autom  &         16 &     d02 &   0.8886 &  2.7164 &        0.1071 &           0.3268 &  0.2169 \\
        Nueva Generacion  &          1 &     d01 &   0.9384 &  2.0121 &        0.9802 &           1.0000 &  0.9901 \\
        Nueva Generacion  &          1 &     d02 &   0.9368 &  2.1941 &        0.8591 &           0.9832 &  0.9212 \\
        Nueva Generacion  &          2 &     d01 &   0.9287 &  2.3548 &        0.7521 &           0.8980 &  0.8250 \\
        Nueva Generacion  &          2 &     d02 &   0.9308 &  2.4644 &        0.6791 &           0.9196 &  0.7993 \\
        Nueva Generacion  &          3 &     d01 &   0.9145 &  1.9824 &        1.0000 &           0.7487 &  0.8743 \\
        Nueva Generacion  &          3 &     d02 &   0.9192 &  2.4738 &        0.6728 &           0.7982 &  0.7355 \\
        Nueva Generacion  &          4 &     d01 &   0.9180 &  2.0929 &        0.9264 &           0.7854 &  0.8559 \\
        Nueva Generacion  &          4 &     d02 &   0.9153 &  2.6840 &        0.5328 &           0.7569 &  0.6449 \\
        Nueva Generacion  &          5 &     d01 &   0.9027 &  2.4026 &        0.7202 &           0.6242 &  0.6722 \\
        Nueva Generacion  &          5 &     d02 &   0.8991 &  2.8545 &        0.4193 &           0.5855 &  0.5024 \\
        Nueva Generacion  &          6 &     d01 &   0.8731 &  2.9105 &        0.3820 &           0.3117 &  0.3468 \\
        Nueva Generacion  &          6 &     d02 &   0.8724 &  3.2140 &        0.1799 &           0.3049 &  0.2424 \\
        Nueva Generacion  &          7 &     d01 &   0.9138 &  2.2710 &        0.8078 &           0.7404 &  0.7741 \\
        Nueva Generacion  &          7 &     d02 &   0.9113 &  2.8173 &        0.4440 &           0.7145 &  0.5793 \\
        Nueva Generacion  &          8 &     d01 &   0.9249 &  2.2692 &        0.8090 &           0.8576 &  0.8333 \\
        Nueva Generacion  &          8 &     d02 &   0.9205 &  2.8925 &        0.3940 &           0.8113 &  0.6026 \\
        Nueva Generacion  &          9 &     d01 &   0.8916 &  2.7211 &        0.5081 &           0.5064 &  0.5072 \\
        Nueva Generacion  &          9 &     d02 &   0.8885 &  3.1638 &        0.2133 &           0.4740 &  0.3437 \\
        Nueva Generacion  &         10 &     d01 &   0.8435 &  3.1409 &        0.2286 &           0.0000 &  0.1143 \\
        Nueva Generacion  &         10 &     d02 &   0.8447 &  3.4841 &        0.0000 &           0.0127 &  0.0064 \\
        Nueva Generacion  &         11 &     d01 &   0.8926 &  2.5415 &        0.6277 &           0.5172 &  0.5724 \\
        Nueva Generacion  &         11 &     d02 &   0.8922 &  3.0226 &        0.3073 &           0.5134 &  0.4104 \\
        Nueva Generacion  &         12 &     d01 &   0.8997 &  2.4736 &        0.6729 &           0.5922 &  0.6326 \\
        Nueva Generacion  &         12 &     d02 &   0.9058 &  2.8988 &        0.3898 &           0.6565 &  0.5231 \\
        Nueva Generacion  &         13 &     d01 &   0.8852 &  2.7049 &        0.5189 &           0.4395 &  0.4792 \\
        Nueva Generacion  &         13 &     d02 &   0.8900 &  3.0262 &        0.3049 &           0.4896 &  0.3972 \\
        Nueva Generacion  &         14 &     d01 &   0.8602 &  2.9847 &        0.3326 &           0.1764 &  0.2545 \\
        Nueva Generacion  &         14 &     d02 &   0.8655 &  3.2276 &        0.1708 &           0.2321 &  0.2014 \\
        Nueva Generacion  &         15 &     d01 &   0.8938 &  2.4684 &        0.6764 &           0.5297 &  0.6031 \\
        Nueva Generacion  &         15 &     d02 &   0.8954 &  2.8882 &        0.3968 &           0.5471 &  0.4720 \\
        Nueva Generacion  &         16 &     d01 &   0.8819 &  2.5865 &        0.5977 &           0.4045 &  0.5011 \\
        Nueva Generacion  &         16 &     d02 &   0.8887 &  2.9028 &        0.3871 &           0.4766 &  0.4319 \\
          Apto El Dorado  &          1 &     d01 &   0.9496 &  3.4253 &        0.6702 &           1.0000 &  0.8351 \\
          Apto El Dorado  &          1 &     d02 &   0.9382 &  3.4137 &        0.6784 &           0.8312 &  0.7548 \\
          Apto El Dorado  &          2 &     d01 &   0.9415 &  3.7923 &        0.4097 &           0.8801 &  0.6449 \\
          Apto El Dorado  &          2 &     d02 &   0.9405 &  3.8268 &        0.3852 &           0.8648 &  0.6250 \\
          Apto El Dorado  &          3 &     d01 &   0.9280 &  2.9605 &        1.0000 &           0.6796 &  0.8398 \\
          Apto El Dorado  &          3 &     d02 &   0.9289 &  3.1559 &        0.8613 &           0.6940 &  0.7777 \\
          Apto El Dorado  &          4 &     d01 &   0.9079 &  3.2287 &        0.8097 &           0.3828 &  0.5963 \\
          Apto El Dorado  &          4 &     d02 &   0.9020 &  3.5518 &        0.5804 &           0.2946 &  0.4375 \\
          Apto El Dorado  &          5 &     d01 &   0.9275 &  3.6109 &        0.5385 &           0.6724 &  0.6054 \\
          Apto El Dorado  &          5 &     d02 &   0.9270 &  3.8145 &        0.3940 &           0.6649 &  0.5294 \\
          Apto El Dorado  &          6 &     d01 &   0.8875 &  4.2067 &        0.1156 &           0.0801 &  0.0979 \\
          Apto El Dorado  &          6 &     d02 &   0.8927 &  4.3696 &        0.0000 &           0.1579 &  0.0789 \\
          Apto El Dorado  &          7 &     d01 &   0.9185 &  3.2607 &        0.7870 &           0.5395 &  0.6632 \\
          Apto El Dorado  &          7 &     d02 &   0.9162 &  3.5403 &        0.5885 &           0.5057 &  0.5471 \\
          Apto El Dorado  &          8 &     d01 &   0.9269 &  3.2232 &        0.8136 &           0.6630 &  0.7383 \\
          Apto El Dorado  &          8 &     d02 &   0.9249 &  3.4234 &        0.6715 &           0.6340 &  0.6528 \\
          Apto El Dorado  &          9 &     d01 &   0.9348 &  3.5255 &        0.5991 &           0.7803 &  0.6897 \\
          Apto El Dorado  &          9 &     d02 &   0.9347 &  3.7378 &        0.4484 &           0.7795 &  0.6139 \\
          Apto El Dorado  &         10 &     d01 &   0.8821 &  3.8931 &        0.3382 &           0.0000 &  0.1691 \\
          Apto El Dorado  &         10 &     d02 &   0.8876 &  4.0435 &        0.2315 &           0.0820 &  0.1568 \\
          Apto El Dorado  &         11 &     d01 &   0.9273 &  3.1090 &        0.8946 &           0.6690 &  0.7818 \\
          Apto El Dorado  &         11 &     d02 &   0.9261 &  3.3323 &        0.7362 &           0.6515 &  0.6938 \\
          Apto El Dorado  &         12 &     d01 &   0.9266 &  3.1632 &        0.8561 &           0.6594 &  0.7578 \\
          Apto El Dorado  &         12 &     d02 &   0.9272 &  3.3614 &        0.7155 &           0.6679 &  0.6917 \\
          Apto El Dorado  &         13 &     d01 &   0.9264 &  3.3800 &        0.7023 &           0.6569 &  0.6796 \\
          Apto El Dorado  &         13 &     d02 &   0.9279 &  3.5479 &        0.5831 &           0.6792 &  0.6312 \\
          Apto El Dorado  &         14 &     d01 &   0.8963 &  3.6803 &        0.4892 &           0.2103 &  0.3498 \\
          Apto El Dorado  &         14 &     d02 &   0.8986 &  3.8373 &        0.3778 &           0.2450 &  0.3114 \\
          Apto El Dorado  &         15 &     d01 &   0.9259 &  3.1799 &        0.8443 &           0.6494 &  0.7469 \\
          Apto El Dorado  &         15 &     d02 &   0.9251 &  3.3661 &        0.7121 &           0.6374 &  0.6748 \\
          Apto El Dorado  &         16 &     d01 &   0.9120 &  3.2829 &        0.7713 &           0.4428 &  0.6070 \\
          Apto El Dorado  &         16 &     d02 &   0.9096 &  3.4587 &        0.6464 &           0.4082 &  0.5273 \\
           Univ Nacional  &          1 &     d01 &   0.9597 &  3.5140 &        0.4445 &           0.9619 &  0.7032 \\
           Univ Nacional  &          1 &     d02 &   0.9537 &  3.4515 &        0.4769 &           0.8973 &  0.6871 \\
           Univ Nacional  &          2 &     d01 &   0.9632 &  3.7448 &        0.3252 &           1.0000 &  0.6626 \\
           Univ Nacional  &          2 &     d02 &   0.9618 &  3.6123 &        0.3937 &           0.9850 &  0.6894 \\
           Univ Nacional  &          3 &     d01 &   0.8978 &  2.5772 &        0.9290 &           0.2961 &  0.6125 \\
           Univ Nacional  &          3 &     d02 &   0.8913 &  3.0936 &        0.6619 &           0.2267 &  0.4443 \\
           Univ Nacional  &          4 &     d01 &   0.9214 &  2.4398 &        1.0000 &           0.5506 &  0.7753 \\
           Univ Nacional  &          4 &     d02 &   0.9225 &  3.1541 &        0.6306 &           0.5623 &  0.5965 \\
           Univ Nacional  &          5 &     d01 &   0.9191 &  3.2919 &        0.5594 &           0.5255 &  0.5424 \\
           Univ Nacional  &          5 &     d02 &   0.9261 &  3.6788 &        0.3593 &           0.6007 &  0.4800 \\
           Univ Nacional  &          6 &     d01 &   0.8702 &  4.1819 &        0.0992 &           0.0000 &  0.0496 \\
           Univ Nacional  &          6 &     d02 &   0.8846 &  4.3737 &        0.0000 &           0.1551 &  0.0775 \\
           Univ Nacional  &          7 &     d01 &   0.9113 &  2.8103 &        0.8084 &           0.4419 &  0.6252 \\
           Univ Nacional  &          7 &     d02 &   0.9121 &  3.2640 &        0.5738 &           0.4507 &  0.5122 \\
           Univ Nacional  &          8 &     d01 &   0.9183 &  2.6149 &        0.9095 &           0.5170 &  0.7132 \\
           Univ Nacional  &          8 &     d02 &   0.9180 &  3.3861 &        0.5107 &           0.5134 &  0.5120 \\
           Univ Nacional  &          9 &     d01 &   0.9210 &  3.2077 &        0.6029 &           0.5456 &  0.5743 \\
           Univ Nacional  &          9 &     d02 &   0.9255 &  3.6972 &        0.3498 &           0.5947 &  0.4723 \\
           Univ Nacional  &         10 &     d01 &   0.8739 &  3.7176 &        0.3392 &           0.0400 &  0.1896 \\
           Univ Nacional  &         10 &     d02 &   0.8784 &  4.1093 &        0.1367 &           0.0881 &  0.1124 \\
           Univ Nacional  &         11 &     d01 &   0.9074 &  2.8027 &        0.8123 &           0.4003 &  0.6063 \\
           Univ Nacional  &         11 &     d02 &   0.9137 &  3.3547 &        0.5269 &           0.4678 &  0.4974 \\
           Univ Nacional  &         12 &     d01 &   0.9099 &  2.7854 &        0.8213 &           0.4265 &  0.6239 \\
           Univ Nacional  &         12 &     d02 &   0.9155 &  3.3799 &        0.5139 &           0.4866 &  0.5003 \\
           Univ Nacional  &         13 &     d01 &   0.9054 &  3.2975 &        0.5565 &           0.3786 &  0.4676 \\
           Univ Nacional  &         13 &     d02 &   0.9097 &  3.7766 &        0.3088 &           0.4244 &  0.3666 \\
           Univ Nacional  &         14 &     d01 &   0.8775 &  3.6645 &        0.3667 &           0.0779 &  0.2223 \\
           Univ Nacional  &         14 &     d02 &   0.8845 &  4.0408 &        0.1721 &           0.1533 &  0.1627 \\
           Univ Nacional  &         15 &     d01 &   0.9064 &  2.9971 &        0.7118 &           0.3894 &  0.5506 \\
           Univ Nacional  &         15 &     d02 &   0.9133 &  3.4714 &        0.4666 &           0.4631 &  0.4648 \\
           Univ Nacional  &         16 &     d01 &   0.8894 &  3.1578 &        0.6287 &           0.2066 &  0.4177 \\
           Univ Nacional  &         16 &     d02 &   0.9054 &  3.5605 &        0.4205 &           0.3781 &  0.3993 \\
  Esc La Union Automatica &          1 &     d01 &   0.9240 &  3.7811 &        0.5051 &           0.9969 &  0.7510 \\
  Esc La Union Automatica &          1 &     d02 &   0.9244 &  2.4737 &        1.0000 &           1.0000 &  1.0000 \\
  Esc La Union Automatica &          2 &     d01 &   0.8868 &  4.6411 &        0.1796 &           0.7240 &  0.4518 \\
  Esc La Union Automatica &          2 &     d02 &   0.8729 &  3.2064 &        0.7226 &           0.6223 &  0.6725 \\
  Esc La Union Automatica &          3 &     d01 &   0.8417 &  3.9185 &        0.4531 &           0.3934 &  0.4232 \\
  Esc La Union Automatica &          3 &     d02 &   0.8585 &  3.4354 &        0.6360 &           0.5168 &  0.5764 \\
  Esc La Union Automatica &          4 &     d01 &   0.8082 &  3.6119 &        0.5692 &           0.1479 &  0.3585 \\
  Esc La Union Automatica &          4 &     d02 &   0.7948 &  2.9523 &        0.8188 &           0.0497 &  0.4343 \\
  Esc La Union Automatica &          5 &     d01 &   0.8199 &  4.3150 &        0.3030 &           0.2335 &  0.2683 \\
  Esc La Union Automatica &          5 &     d02 &   0.8139 &  3.3022 &        0.6864 &           0.1898 &  0.4381 \\
  Esc La Union Automatica &          6 &     d01 &   0.7968 &  5.1155 &        0.0000 &           0.0640 &  0.0320 \\
  Esc La Union Automatica &          6 &     d02 &   0.7880 &  3.8560 &        0.4768 &           0.0000 &  0.2384 \\
  Esc La Union Automatica &          7 &     d01 &   0.8237 &  4.2690 &        0.3204 &           0.2615 &  0.2910 \\
  Esc La Union Automatica &          7 &     d02 &   0.8110 &  3.3796 &        0.6571 &           0.1680 &  0.4125 \\
  Esc La Union Automatica &          8 &     d01 &   0.8354 &  4.0200 &        0.4147 &           0.3473 &  0.3810 \\
  Esc La Union Automatica &          8 &     d02 &   0.8091 &  3.1440 &        0.7463 &           0.1547 &  0.4505 \\
  Esc La Union Automatica &          9 &     d01 &   0.8206 &  4.6039 &        0.1937 &           0.2386 &  0.2161 \\
  Esc La Union Automatica &          9 &     d02 &   0.8014 &  3.5414 &        0.5959 &           0.0980 &  0.3469 \\
  Esc La Union Automatica &         10 &     d01 &   0.8317 &  4.8618 &        0.0960 &           0.3203 &  0.2082 \\
  Esc La Union Automatica &         10 &     d02 &   0.8079 &  3.7035 &        0.5345 &           0.1454 &  0.3400 \\
  Esc La Union Automatica &         11 &     d01 &   0.8336 &  4.1547 &        0.3637 &           0.3342 &  0.3489 \\
  Esc La Union Automatica &         11 &     d02 &   0.8156 &  3.2702 &        0.6985 &           0.2021 &  0.4503 \\
  Esc La Union Automatica &         12 &     d01 &   0.8450 &  3.9854 &        0.4278 &           0.4174 &  0.4226 \\
  Esc La Union Automatica &         12 &     d02 &   0.8308 &  3.0797 &        0.7706 &           0.3133 &  0.5420 \\
  Esc La Union Automatica &         13 &     d01 &   0.8458 &  4.4919 &        0.2361 &           0.4235 &  0.3298 \\
  Esc La Union Automatica &         13 &     d02 &   0.8285 &  3.2494 &        0.7064 &           0.2964 &  0.5014 \\
  Esc La Union Automatica &         14 &     d01 &   0.8166 &  4.7292 &        0.1462 &           0.2095 &  0.1779 \\
  Esc La Union Automatica &         14 &     d02 &   0.7956 &  3.5134 &        0.6065 &           0.0551 &  0.3308 \\
  Esc La Union Automatica &         15 &     d01 &   0.8394 &  4.3183 &        0.3018 &           0.3768 &  0.3393 \\
  Esc La Union Automatica &         15 &     d02 &   0.8203 &  3.3513 &        0.6678 &           0.2366 &  0.4522 \\
  Esc La Union Automatica &         16 &     d01 &   0.8442 &  4.2760 &        0.3178 &           0.4115 &  0.3646 \\
  Esc La Union Automatica &         16 &     d02 &   0.8264 &  3.2349 &        0.7119 &           0.2813 &  0.4966 \\


\label{tab:estaciones_tiempo_wrf}

\end{longtable}

\end{landscape} % Cuarto
%% Anexo de los diagramas de Taylor para todos los casos con las mejores configuraciones.
\chapter{Gráficas de los diagramas de Taylor de las evaluaciones de los diferentes tiempos para los 4 casos.}
\label{anexo:graficas_taylor_tiempos_4casos}

\begin{figure}[H]
 
 
 
\begin{subfigure}[normla]{0.4\textwidth}
\includegraphics[draft=false, scale=0.25]{../taylor_simulaciones_mejor/total/taylor_20070221195160.png}
\caption{Estación Subia Atutomática código 21195160 caso 1.}
\end{subfigure}
~
\begin{subfigure}[normla]{0.4\textwidth}
\includegraphics[draft=false, scale=0.25]{../taylor_simulaciones_mejor/total/taylor_20070221206790.png}
\caption{Estación Hda Sta Ana Autom código 21206790 caso 1.}
\end{subfigure}
~
\begin{subfigure}[normla]{0.4\textwidth}
\includegraphics[draft=false, scale=0.25]{../taylor_simulaciones_mejor/total/taylor_20070221206930.png}
\caption{Estación Pmo Guerrero código 21206930 caso 1.}
\end{subfigure}
~
\begin{subfigure}[normla]{0.4\textwidth}
\includegraphics[draft=false, scale=0.25]{../taylor_simulaciones_mejor/total/taylor_20070221206940.png}
\caption{Estación Ciudad Bolivar código 21206940 caso 1.}
\end{subfigure}
~
\begin{subfigure}[normla]{0.4\textwidth}
\includegraphics[draft=false, scale=0.25]{../taylor_simulaciones_mejor/total/taylor_20070221206950.png}
\caption{Estación Pmo Guacheneque código 21206950 caso 1.}
\end{subfigure}
~
\begin{subfigure}[normla]{0.4\textwidth}
\includegraphics[draft=false, scale=0.25]{../taylor_simulaciones_mejor/total/taylor_20070221206980.png}
\caption{Estación Sta Cruz De Siecha código 21206980 caso 1.}
\end{subfigure}
~
\end{figure}
 
\begin{figure}[H]\ContinuedFloat
\centering
\begin{subfigure}[normla]{0.4\textwidth}
\includegraphics[draft=false, scale=0.25]{../taylor_simulaciones_mejor/total/taylor_20070221206990.png}
\caption{Estación Tibaitata Atutomática código 21206990 caso 1.}
\end{subfigure}
~
\begin{subfigure}[normla]{0.4\textwidth}
\includegraphics[draft=false, scale=0.25]{../taylor_simulaciones_mejor/total/taylor_20070224015110.png}
\caption{Estación La Boyera Atutomática código 24015110 caso 1.}
\end{subfigure}
~
\begin{subfigure}[normla]{0.4\textwidth}
\includegraphics[draft=false, scale=0.25]{../taylor_simulaciones_mejor/total/taylor_20070235075070.png}
\caption{Estación Chinavita Atutomática código 35075070 caso 1.}
\end{subfigure}
~
\begin{subfigure}[normla]{0.4\textwidth}
\includegraphics[draft=false, scale=0.25]{../taylor_simulaciones_mejor/total/taylor_20070235085080.png}
\caption{Estación La Capilla Autom código 35085080 caso 1.}
\end{subfigure}
~
\begin{subfigure}[normla]{0.4\textwidth}
\includegraphics[draft=false, scale=0.25]{../taylor_simulaciones_mejor/total/taylor_20140821201200.png}
\caption{Estación Esc La Union Atutomática código 21201200 caso 2.}
\end{subfigure}
~
\begin{subfigure}[normla]{0.4\textwidth}
\includegraphics[draft=false, scale=0.25]{../taylor_simulaciones_mejor/total/taylor_20140821205012.png}
\caption{Estación Univ Nacional código 21205012 caso 2.}
\end{subfigure}
~
\end{figure}
 
\begin{figure}[H]\ContinuedFloat
\centering
\begin{subfigure}[normla]{0.4\textwidth}
\includegraphics[draft=false, scale=0.25]{../taylor_simulaciones_mejor/total/taylor_20140821205791.png}
\caption{Estación Apto El Dorado código 21205791 caso 2.}
\end{subfigure}
~
\begin{subfigure}[normla]{0.4\textwidth}
\includegraphics[draft=false, scale=0.25]{../taylor_simulaciones_mejor/total/taylor_20140821206790.png}
\caption{Estación Hda Sta Ana Autom código 21206790 caso 2.}
\end{subfigure}
~
\begin{subfigure}[normla]{0.4\textwidth}
\includegraphics[draft=false, scale=0.25]{../taylor_simulaciones_mejor/total/taylor_20140821206930.png}
\caption{Estación Pmo Guerrero código 21206930 caso 2.}
\end{subfigure}
~
\begin{subfigure}[normla]{0.4\textwidth}
\includegraphics[draft=false, scale=0.25]{../taylor_simulaciones_mejor/total/taylor_20140821206950.png}
\caption{Estación Pmo Guacheneque código 21206950 caso 2.}
\end{subfigure}
~
\begin{subfigure}[normla]{0.4\textwidth}
\includegraphics[draft=false, scale=0.25]{../taylor_simulaciones_mejor/total/taylor_20140821206960.png}
\caption{Estación Ideam Bogota código 21206960 caso 2.}
\end{subfigure}
~
%\begin{subfigure}[normla]{0.4\textwidth}
%\includegraphics[draft=false, scale=0.25]{../taylor_simulaciones_mejor/total/taylor_20140821206980.png}
%\caption{Estación Sta Cruz De Siecha código 21206980 caso 2.}
%\end{subfigure}
~
\end{figure}
 
\begin{figure}[H]\ContinuedFloat
\centering
\begin{subfigure}[normla]{0.4\textwidth}
\includegraphics[draft=false, scale=0.25]{../taylor_simulaciones_mejor/total/taylor_20140821206990.png}
\caption{Estación Tibaitata Atutomática código 21206990 caso 2.}
\end{subfigure}
~
\begin{subfigure}[normla]{0.4\textwidth}
\includegraphics[draft=false, scale=0.25]{../taylor_simulaciones_mejor/total/taylor_20140823125170.png}
\caption{Estación San Cayetano Autom código 23125170 caso 2.}
\end{subfigure}
~
\begin{subfigure}[normla]{0.4\textwidth}
\includegraphics[draft=false, scale=0.25]{../taylor_simulaciones_mejor/total/taylor_20140824015110.png}
\caption{Estación La Boyera Atutomática código 24015110 caso 2.}
\end{subfigure}
~
\begin{subfigure}[normla]{0.4\textwidth}
\includegraphics[draft=false, scale=0.25]{../taylor_simulaciones_mejor/total/taylor_20140835025080.png}
\caption{Estación Pnn Chingaza Autom código 35025080 caso 2.}
\end{subfigure}
~
\begin{subfigure}[normla]{0.4\textwidth}
\includegraphics[draft=false, scale=0.25]{../taylor_simulaciones_mejor/total/taylor_20140835025090.png}
\caption{Estación Bosque Intervenido código 35025090 caso 2.}
\end{subfigure}
~
\begin{subfigure}[normla]{0.4\textwidth}
\includegraphics[draft=false, scale=0.25]{../taylor_simulaciones_mejor/total/taylor_20140835075070.png}
\caption{Estación Chinavita Atutomática código 35075070 caso 2.}
\end{subfigure}
~
\end{figure}
 
\begin{figure}[H]
\centering
\begin{subfigure}[normla]{0.4\textwidth}
\includegraphics[draft=false, scale=0.25]{../taylor_simulaciones_mejor/total/taylor_20140835085080.png}
\caption{Estación La Capilla Autom código 35085080 caso 2.}
\end{subfigure}
~
\begin{subfigure}[normla]{0.4\textwidth}
\includegraphics[draft=false, scale=0.25]{../taylor_simulaciones_mejor/total/taylor_20150821201200.png}
\caption{Estación Esc La Union Atutomática código 21201200 caso 3.}
\end{subfigure}
~
\begin{subfigure}[normla]{0.4\textwidth}
\includegraphics[draft=false, scale=0.25]{../taylor_simulaciones_mejor/total/taylor_20150821205012.png}
\caption{Estación Univ Nacional código 21205012 caso 3.}
\end{subfigure}
~
\begin{subfigure}[normla]{0.4\textwidth}
\includegraphics[draft=false, scale=0.25]{../taylor_simulaciones_mejor/total/taylor_20150821205791.png}
\caption{Estación Apto El Dorado código 21205791 caso 3.}
\end{subfigure}
~
\begin{subfigure}[normla]{0.4\textwidth}
\includegraphics[draft=false, scale=0.25]{../taylor_simulaciones_mejor/total/taylor_20150821206600.png}
\caption{Estación Nueva Generacion código 21206600 caso 3.}
\end{subfigure}
~
\begin{subfigure}[normla]{0.4\textwidth}
\includegraphics[draft=false, scale=0.25]{../taylor_simulaciones_mejor/total/taylor_20150821206790.png}
\caption{Estación Hda Sta Ana Autom código 21206790 caso 3.}
\end{subfigure}
~
\end{figure}
 
\begin{figure}[H]\ContinuedFloat
\centering
\begin{subfigure}[normla]{0.4\textwidth}
\includegraphics[draft=false, scale=0.25]{../taylor_simulaciones_mejor/total/taylor_20150821206920.png}
\caption{Estación Villa Teresa Atutomática código 21206920 caso 3.}
\end{subfigure}
~
\begin{subfigure}[normla]{0.4\textwidth}
\includegraphics[draft=false, scale=0.25]{../taylor_simulaciones_mejor/total/taylor_20150821206930.png}
\caption{Estación Pmo Guerrero código 21206930 caso 3.}
\end{subfigure}
~
\begin{subfigure}[normla]{0.4\textwidth}
\includegraphics[draft=false, scale=0.25]{../taylor_simulaciones_mejor/total/taylor_20150821206960.png}
\caption{Estación Ideam Bogota código 21206960 caso 3.}
\end{subfigure}
~
\begin{subfigure}[normla]{0.4\textwidth}
\includegraphics[draft=false, scale=0.25]{../taylor_simulaciones_mejor/total/taylor_20150821206980.png}
\caption{Estación Sta Cruz De Siecha código 21206980 caso 3.}
\end{subfigure}
~
\begin{subfigure}[normla]{0.4\textwidth}
\includegraphics[draft=false, scale=0.25]{../taylor_simulaciones_mejor/total/taylor_20150821206990.png}
\caption{Estación Tibaitata Atutomática código 21206990 caso 3.}
\end{subfigure}
~
\begin{subfigure}[normla]{0.4\textwidth}
\includegraphics[draft=false, scale=0.25]{../taylor_simulaciones_mejor/total/taylor_20150823125170.png}
\caption{Estación San Cayetano Autom código 23125170 caso 3.}
\end{subfigure}
~
\end{figure}
 
\begin{figure}[H]\ContinuedFloat
\centering
\begin{subfigure}[normla]{0.4\textwidth}
\includegraphics[draft=false, scale=0.25]{../taylor_simulaciones_mejor/total/taylor_20150835025090.png}
\caption{Estación Bosque Intervenido código 35025090 caso 3.}
\end{subfigure}
~
\begin{subfigure}[normla]{0.4\textwidth}
\includegraphics[draft=false, scale=0.25]{../taylor_simulaciones_mejor/total/taylor_20150835035130.png}
\caption{Estación Pmo Chingaza código 35035130 caso 3.}
\end{subfigure}
~
\begin{subfigure}[normla]{0.4\textwidth}
\includegraphics[draft=false, scale=0.25]{../taylor_simulaciones_mejor/total/taylor_20150835075070.png}
\caption{Estación Chinavita Atutomática código 35075070 caso 3.}
\end{subfigure}
~
\begin{subfigure}[normla]{0.4\textwidth}
\includegraphics[draft=false, scale=0.25]{../taylor_simulaciones_mejor/total/taylor_20150835075080.png}
\caption{Estación Pmo Rabanal Atutomática código 35075080 caso 3.}
\end{subfigure}
~
\begin{subfigure}[normla]{0.4\textwidth}
\includegraphics[draft=false, scale=0.25]{../taylor_simulaciones_mejor/total/taylor_20150835085080.png}
\caption{Estación La Capilla Autom código 35085080 caso 3.}
\end{subfigure}
~
\begin{subfigure}[normla]{0.4\textwidth}
\includegraphics[draft=false, scale=0.25]{../taylor_simulaciones_mejor/total/taylor_20150921205012.png}
\caption{Estación Univ Nacional código 21205012 caso 4.}
\end{subfigure}
~
\end{figure}
 
\begin{figure}[H]\ContinuedFloat
\centering
\begin{subfigure}[normla]{0.4\textwidth}
\includegraphics[draft=false, scale=0.25]{../taylor_simulaciones_mejor/total/taylor_20150921205791.png}
\caption{Estación Apto El Dorado código 21205791 caso 4.}
\end{subfigure}
~
\begin{subfigure}[normla]{0.4\textwidth}
\includegraphics[draft=false, scale=0.25]{../taylor_simulaciones_mejor/total/taylor_20150921206600.png}
\caption{Estación Nueva Generacion código 21206600 caso 4.}
\end{subfigure}
~
\begin{subfigure}[normla]{0.4\textwidth}
\includegraphics[draft=false, scale=0.25]{../taylor_simulaciones_mejor/total/taylor_20150921206790.png}
\caption{Estación Hda Sta Ana Autom código 21206790 caso 4.}
\end{subfigure}
~
\begin{subfigure}[normla]{0.4\textwidth}
\includegraphics[draft=false, scale=0.25]{../taylor_simulaciones_mejor/total/taylor_20150921206920.png}
\caption{Estación Villa Teresa Atutomática código 21206920 caso 4.}
\end{subfigure}
~
\begin{subfigure}[normla]{0.4\textwidth}
\includegraphics[draft=false, scale=0.25]{../taylor_simulaciones_mejor/total/taylor_20150921206930.png}
\caption{Estación Pmo Guerrero código 21206930 caso 4.}
\end{subfigure}
~
\begin{subfigure}[normla]{0.4\textwidth}
\includegraphics[draft=false, scale=0.25]{../taylor_simulaciones_mejor/total/taylor_20150921206960.png}
\caption{Estación Ideam Bogota código 21206960 caso 4.}
\end{subfigure}
~
\end{figure}
 
\begin{figure}[H]
\centering
\begin{subfigure}[normla]{0.4\textwidth}
\includegraphics[draft=false, scale=0.25]{../taylor_simulaciones_mejor/total/taylor_20150921206980.png}
\caption{Estación Sta Cruz De Siecha código 21206980 caso 4.}
\end{subfigure}
~
\begin{subfigure}[normla]{0.4\textwidth}
\includegraphics[draft=false, scale=0.25]{../taylor_simulaciones_mejor/total/taylor_20150921206990.png}
\caption{Estación Tibaitata Atutomática código 21206990 caso 4.}
\end{subfigure}
~
\begin{subfigure}[normla]{0.4\textwidth}
\includegraphics[draft=false, scale=0.25]{../taylor_simulaciones_mejor/total/taylor_20150923125170.png}
\caption{Estación San Cayetano Autom código 23125170 caso 4.}
\end{subfigure}
~
\begin{subfigure}[normla]{0.4\textwidth}
\includegraphics[draft=false, scale=0.25]{../taylor_simulaciones_mejor/total/taylor_20150935025090.png}
\caption{Estación Bosque Intervenido código 35025090 caso 4.}
\end{subfigure}
~
\begin{subfigure}[normla]{0.4\textwidth}
\includegraphics[draft=false, scale=0.25]{../taylor_simulaciones_mejor/total/taylor_20150935035130.png}
\caption{Estación Pmo Chingaza código 35035130 caso 4.}
\end{subfigure}
~
\begin{subfigure}[normla]{0.4\textwidth}
\includegraphics[draft=false, scale=0.25]{../taylor_simulaciones_mejor/total/taylor_20150935075070.png}
\caption{Estación Chinavita Atutomática código 35075070 caso 4.}
\end{subfigure}
~
\end{figure}
 
\begin{figure}[H]\ContinuedFloat
\centering
\begin{subfigure}[normla]{0.4\textwidth}
\includegraphics[draft=false, scale=0.25]{../taylor_simulaciones_mejor/total/taylor_20150935075080.png}
\caption{Estación Pmo Rabanal Atutomática código 35075080 caso 4.}
\end{subfigure}
~
\begin{subfigure}[normla]{0.4\textwidth}
\includegraphics[draft=false, scale=0.25]{../taylor_simulaciones_mejor/total/taylor_20150935085080.png}
\caption{Estación La Capilla Autom código 35085080 caso 4.}
\end{subfigure}
~

 
 \caption{Caption}
 \label{fig:my_label}
\end{figure}
 % Cuarto


%\chapter{Gráficas de la comparación de los resultados del WRF con respecto a las estaciones automáticas para los cuatro casos}



\label{anexo:graficas_temperatura}
\newpage

\begin{figure}[H]
    
\begin{subfigure}[normla]{0.4\textwidth}
\caption{Estación Subia Automatica código 21195160 caso 1.}
\includegraphics[draft=false, scale=0.3]{../comparacion_grafica/200702_21195160.png}
\end{subfigure}
~
\begin{subfigure}[normla]{0.4\textwidth}
\caption{Estación Esc La Union Automaticacódigo 21201200 caso 2.}
\includegraphics[draft=false, scale=0.3]{../comparacion_grafica/201408_21201200.png}
\end{subfigure}
~
\begin{subfigure}[normla]{0.4\textwidth}
\caption{Estación Esc La Union Automaticacódigo 21201200 caso 3.}
\includegraphics[draft=false, scale=0.3]{../comparacion_grafica/201508_21201200.png}
\end{subfigure}
~
\begin{subfigure}[normla]{0.4\textwidth}
\caption{Estación Univ Nacional código 21205012 caso 2.}
\includegraphics[draft=false, scale=0.3]{../comparacion_grafica/201408_21205012.png}
\end{subfigure}
~
\begin{subfigure}[normla]{0.4\textwidth}
\caption{Estación Univ Nacional código 21205012 caso 3.}
\includegraphics[draft=false, scale=0.3]{../comparacion_grafica/201508_21205012.png}
\end{subfigure}
~
\begin{subfigure}[normla]{0.4\textwidth}
\caption{Estación Univ Nacional código 21205012 caso 4.}
\includegraphics[draft=false, scale=0.3]{../comparacion_grafica/201509_21205012.png}
\end{subfigure}
~
\end{figure}
           
\begin{figure}[H]\ContinuedFloat
\centering
\begin{subfigure}[normla]{0.4\textwidth}
\caption{Estación Apto El Dorado código 21205791 caso 2.}
\includegraphics[draft=false, scale=0.3]{../comparacion_grafica/201408_21205791.png}
\end{subfigure}
~
\begin{subfigure}[normla]{0.4\textwidth}
\caption{Estación Apto El Dorado código 21205791 caso 3.}
\includegraphics[draft=false, scale=0.3]{../comparacion_grafica/201508_21205791.png}
\end{subfigure}
~
\begin{subfigure}[normla]{0.4\textwidth}
\caption{Estación Apto El Dorado código 21205791 caso 4.}
\includegraphics[draft=false, scale=0.3]{../comparacion_grafica/201509_21205791.png}
\end{subfigure}
~
\begin{subfigure}[normla]{0.4\textwidth}
\caption{Estación Nueva Generacion código 21206600 caso 3.}
\includegraphics[draft=false, scale=0.3]{../comparacion_grafica/201508_21206600.png}
\end{subfigure}
~
\begin{subfigure}[normla]{0.4\textwidth}
\caption{Estación Nueva Generacion código 21206600 caso 4.}
\includegraphics[draft=false, scale=0.3]{../comparacion_grafica/201509_21206600.png}
\end{subfigure}
~
\begin{subfigure}[normla]{0.4\textwidth}
\caption{Estación Hda Sta Ana Autom código 21206790 caso 1.}
\includegraphics[draft=false, scale=0.3]{../comparacion_grafica/200702_21206790.png}
\end{subfigure}
~
\end{figure}
           
\begin{figure}[H]\ContinuedFloat
\centering
\begin{subfigure}[normla]{0.4\textwidth}
\caption{Estación Hda Sta Ana Autom código 21206790 caso 2.}
\includegraphics[draft=false, scale=0.3]{../comparacion_grafica/201408_21206790.png}
\end{subfigure}
~
\begin{subfigure}[normla]{0.4\textwidth}
\caption{Estación Hda Sta Ana Autom código 21206790 caso 3.}
\includegraphics[draft=false, scale=0.3]{../comparacion_grafica/201508_21206790.png}
\end{subfigure}
~
\begin{subfigure}[normla]{0.4\textwidth}
\caption{Estación Hda Sta Ana Autom código 21206790 caso 4.}
\includegraphics[draft=false, scale=0.3]{../comparacion_grafica/201509_21206790.png}
\end{subfigure}
~
\begin{subfigure}[normla]{0.4\textwidth}
\caption{Estación Villa Teresa Automatica código 21206920 caso 3.}
\includegraphics[draft=false, scale=0.3]{../comparacion_grafica/201508_21206920.png}
\end{subfigure}
~
\begin{subfigure}[normla]{0.4\textwidth}
\caption{Estación Villa Teresa Automatica código 21206920 caso 4.}
\includegraphics[draft=false, scale=0.3]{../comparacion_grafica/201509_21206920.png}
\end{subfigure}
~
\begin{subfigure}[normla]{0.4\textwidth}
\caption{Estación Pmo Guerrero código 21206930 caso 1.}
\includegraphics[draft=false, scale=0.3]{../comparacion_grafica/200702_21206930.png}
\end{subfigure}
~
\end{figure}
           
\begin{figure}[H]\ContinuedFloat
\centering
\begin{subfigure}[normla]{0.4\textwidth}
\caption{Estación Pmo Guerrero código 21206930 caso 2.}
\includegraphics[draft=false, scale=0.3]{../comparacion_grafica/201408_21206930.png}
\end{subfigure}
~
\begin{subfigure}[normla]{0.4\textwidth}
\caption{Estación Pmo Guerrero código 21206930 caso 3.}
\includegraphics[draft=false, scale=0.3]{../comparacion_grafica/201508_21206930.png}
\end{subfigure}
~
\begin{subfigure}[normla]{0.4\textwidth}
\caption{Estación Pmo Guerrero código 21206930 caso 4.}
\includegraphics[draft=false, scale=0.3]{../comparacion_grafica/201509_21206930.png}
\end{subfigure}
~
\begin{subfigure}[normla]{0.4\textwidth}
\caption{Estación Ciudad Bolivar código 21206940 caso 1.}
\includegraphics[draft=false, scale=0.3]{../comparacion_grafica/200702_21206940.png}
\end{subfigure}
~
\begin{subfigure}[normla]{0.4\textwidth}
\caption{Estación Pmo Guacheneque código 21206950 caso 1.}
\includegraphics[draft=false, scale=0.3]{../comparacion_grafica/200702_21206950.png}
\end{subfigure}
~
\begin{subfigure}[normla]{0.4\textwidth}
\caption{Estación Pmo Guacheneque código 21206950 caso 2.}
\includegraphics[draft=false, scale=0.3]{../comparacion_grafica/201408_21206950.png}
\end{subfigure}
~
\end{figure}
           
\begin{figure}[H]
\centering
\begin{subfigure}[normla]{0.4\textwidth}
\caption{Estación Ideam Bogota código 21206960 caso 2.}
\includegraphics[draft=false, scale=0.3]{../comparacion_grafica/201408_21206960.png}
\end{subfigure}
~
\begin{subfigure}[normla]{0.4\textwidth}
\caption{Estación Ideam Bogota código 21206960 caso 3.}
\includegraphics[draft=false, scale=0.3]{../comparacion_grafica/201508_21206960.png}
\end{subfigure}
~
\begin{subfigure}[normla]{0.4\textwidth}
\caption{Estación Ideam Bogota código 21206960 caso 4.}
\includegraphics[draft=false, scale=0.3]{../comparacion_grafica/201509_21206960.png}
\end{subfigure}
~
\begin{subfigure}[normla]{0.4\textwidth}
\caption{Estación Sta Cruz De Siecha código 21206980 caso 1.}
\includegraphics[draft=false, scale=0.3]{../comparacion_grafica/200702_21206980.png}
\end{subfigure}
~
%\begin{subfigure}[normla]{0.4\textwidth}
%\caption{Estación Sta Cruz De Siecha código 21206980 caso 2.}
%\includegraphics[draft=false, scale=0.3]{../comparacion_grafica/201408_21206980.png}
%\end{subfigure}
~
\begin{subfigure}[normla]{0.4\textwidth}
\caption{Estación Sta Cruz De Siecha código 21206980 caso 3.}
\includegraphics[draft=false, scale=0.3]{../comparacion_grafica/201508_21206980.png}
\end{subfigure}
~
\end{figure}
           
\begin{figure}[H]\ContinuedFloat
\centering
\begin{subfigure}[normla]{0.4\textwidth}
\caption{Estación Sta Cruz De Siecha código 21206980 caso 4.}
\includegraphics[draft=false, scale=0.3]{../comparacion_grafica/201509_21206980.png}
\end{subfigure}
~
\begin{subfigure}[normla]{0.4\textwidth}
\caption{Estación Tibaitata Automatica código 21206990 caso 1.}
\includegraphics[draft=false, scale=0.3]{../comparacion_grafica/200702_21206990.png}
\end{subfigure}
~
\begin{subfigure}[normla]{0.4\textwidth}
\caption{Estación Tibaitata Automatica código 21206990 caso 2.}
\includegraphics[draft=false, scale=0.3]{../comparacion_grafica/201408_21206990.png}
\end{subfigure}
~
\begin{subfigure}[normla]{0.4\textwidth}
\caption{Estación Tibaitata Automatica código 21206990 caso 3.}
\includegraphics[draft=false, scale=0.3]{../comparacion_grafica/201508_21206990.png}
\end{subfigure}
~
\begin{subfigure}[normla]{0.4\textwidth}
\caption{Estación Tibaitata Automatica código 21206990 caso 4.}
\includegraphics[draft=false, scale=0.3]{../comparacion_grafica/201509_21206990.png}
\end{subfigure}
~
\begin{subfigure}[normla]{0.4\textwidth}
\caption{Estación San Cayetano Autom  código 23125170 caso 2.}
\includegraphics[draft=false, scale=0.3]{../comparacion_grafica/201408_23125170.png}
\end{subfigure}
~
\end{figure}
           
\begin{figure}[H]
\centering
\begin{subfigure}[normla]{0.4\textwidth}
\caption{Estación San Cayetano Autom  código 23125170 caso 3.}
\includegraphics[draft=false, scale=0.3]{../comparacion_grafica/201508_23125170.png}
\end{subfigure}
~
\begin{subfigure}[normla]{0.4\textwidth}
\caption{Estación San Cayetano Autom  código 23125170 caso 4.}
\includegraphics[draft=false, scale=0.3]{../comparacion_grafica/201509_23125170.png}
\end{subfigure}
~
\begin{subfigure}[normla]{0.4\textwidth}
\caption{Estación La Boyera Automatica código 24015110 caso 1.}
\includegraphics[draft=false, scale=0.3]{../comparacion_grafica/200702_24015110.png}
\end{subfigure}
~
\begin{subfigure}[normla]{0.4\textwidth}
\caption{Estación La Boyera Automatica código 24015110 caso 2.}
\includegraphics[draft=false, scale=0.3]{../comparacion_grafica/201408_24015110.png}
\end{subfigure}
~
\begin{subfigure}[normla]{0.4\textwidth}
\caption{Estación Pnn Chingaza Autom  código 35025080 caso 2.}
\includegraphics[draft=false, scale=0.3]{../comparacion_grafica/201408_35025080.png}
\end{subfigure}
~
\begin{subfigure}[normla]{0.4\textwidth}
\caption{Estación Bosque Intervenido   código 35025090 caso 2.}
\includegraphics[draft=false, scale=0.3]{../comparacion_grafica/201408_35025090.png}
\end{subfigure}
~
\end{figure}
           
\begin{figure}[H]\ContinuedFloat
\centering
\begin{subfigure}[normla]{0.4\textwidth}
\caption{Estación Bosque Intervenido   código 35025090 caso 3.}
\includegraphics[draft=false, scale=0.3]{../comparacion_grafica/201508_35025090.png}
\end{subfigure}
~
\begin{subfigure}[normla]{0.4\textwidth}
\caption{Estación Bosque Intervenido   código 35025090 caso 4.}
\includegraphics[draft=false, scale=0.3]{../comparacion_grafica/201509_35025090.png}
\end{subfigure}
~
\begin{subfigure}[normla]{0.4\textwidth}
\caption{Estación Pmo Chingaza código 35035130 caso 3.}
\includegraphics[draft=false, scale=0.3]{../comparacion_grafica/201508_35035130.png}
\end{subfigure}
~
\begin{subfigure}[normla]{0.4\textwidth}
\caption{Estación Pmo Chingaza código 35035130 caso 4.}
\includegraphics[draft=false, scale=0.3]{../comparacion_grafica/201509_35035130.png}
\end{subfigure}
~
\begin{subfigure}[normla]{0.4\textwidth}
\caption{Estación Chinavita Automatica código 35075070 caso 1.}
\includegraphics[draft=false, scale=0.3]{../comparacion_grafica/200702_35075070.png}
\end{subfigure}
~
\begin{subfigure}[normla]{0.4\textwidth}
\caption{Estación Chinavita Automatica código 35075070 caso 2.}
\includegraphics[draft=false, scale=0.3]{../comparacion_grafica/201408_35075070.png}
\end{subfigure}
~
\end{figure}
           
\begin{figure}[H]
\centering
\begin{subfigure}[normla]{0.4\textwidth}
\caption{Estación Chinavita Automatica código 35075070 caso 3.}
\includegraphics[draft=false, scale=0.3]{../comparacion_grafica/201508_35075070.png}
\end{subfigure}
~
\begin{subfigure}[normla]{0.4\textwidth}
\caption{Estación Chinavita Automatica código 35075070 caso 4.}
\includegraphics[draft=false, scale=0.3]{../comparacion_grafica/201509_35075070.png}
\end{subfigure}
~
\begin{subfigure}[normla]{0.4\textwidth}
\caption{Estación Pmo Rabanal Automatica  código 35075080 caso 3.}
\includegraphics[draft=false, scale=0.3]{../comparacion_grafica/201508_35075080.png}
\end{subfigure}
~
\begin{subfigure}[normla]{0.4\textwidth}
\caption{Estación Pmo Rabanal Automatica  código 35075080 caso 4.}
\includegraphics[draft=false, scale=0.3]{../comparacion_grafica/201509_35075080.png}
\end{subfigure}
~
\begin{subfigure}[normla]{0.4\textwidth}
\caption{Estación La Capilla Autom código 35085080 caso 1.}
\includegraphics[draft=false, scale=0.3]{../comparacion_grafica/200702_35085080.png}
\end{subfigure}
~
\begin{subfigure}[normla]{0.4\textwidth}
\caption{Estación La Capilla Autom código 35085080 caso 2.}
\includegraphics[draft=false, scale=0.3]{../comparacion_grafica/201408_35085080.png}
\end{subfigure}
~
\end{figure}
           
\begin{figure}[H]\ContinuedFloat
\centering
\begin{subfigure}[normla]{0.4\textwidth}
\caption{Estación La Capilla Autom código 35085080 caso 3.}
\includegraphics[draft=false, scale=0.3]{../comparacion_grafica/201508_35085080.png}
\end{subfigure}
~
\begin{subfigure}[normla]{0.4\textwidth}
\caption{Estación La Capilla Autom código 35085080 caso 4.}
\includegraphics[draft=false, scale=0.3]{../comparacion_grafica/201509_35085080.png}
\end{subfigure}

    
    \caption{Gráficas de la comparación de los resultados del WRF con respecto a las estaciones automáticas para los cuatro casos}
    \label{fig:my_label}
\end{figure}
 %%Quinto
%\chapter{Gráficas de la comparación de los resultados del WRF con respecto a las estaciones automáticas para los cuatro casos}



\label{anexo:graficas_otras_variables_wrf}


\begin{figure}
\begin{subfigure}[normla]{0.4\textwidth}
\caption{Estación Esc La Union Automaticacódigo 21201200 caso 2 variable Punto de rocío.}
\includegraphics[draft=false, scale=0.3]{comparacion_graficas_otras_var/201408_21201200_dewpoint.png}
\end{subfigure}
~
\begin{subfigure}[normla]{0.4\textwidth}
\caption{Estación Esc La Union Automaticacódigo 21201200 caso 2 variable Humedad.}
\includegraphics[draft=false, scale=0.3]{comparacion_graficas_otras_var/201408_21201200_humedad.png}
\end{subfigure}
~
\begin{subfigure}[normla]{0.4\textwidth}
\caption{Estación Esc La Union Automaticacódigo 21201200 caso 2 variable Precipitación.}
\includegraphics[draft=false, scale=0.3]{comparacion_graficas_otras_var/201408_21201200_rain.png}
\end{subfigure}
~
\begin{subfigure}[normla]{0.4\textwidth}
\caption{Estación Esc La Union Automaticacódigo 21201200 caso 2 variable Bulbo húmedo.}
\includegraphics[draft=false, scale=0.3]{comparacion_graficas_otras_var/201408_21201200_wetbulb.png}
\end{subfigure}
~
\begin{subfigure}[normla]{0.4\textwidth}
\caption{Estación Esc La Union Automaticacódigo 21201200 caso 3 variable Punto de rocío.}
\includegraphics[draft=false, scale=0.3]{comparacion_graficas_otras_var/201508_21201200_dewpoint.png}
\end{subfigure}
~
\begin{subfigure}[normla]{0.4\textwidth}
\caption{Estación Esc La Union Automaticacódigo 21201200 caso 3 variable Humedad.}
\includegraphics[draft=false, scale=0.3]{comparacion_graficas_otras_var/201508_21201200_humedad.png}
\end{subfigure}
~
\end{figure}
           
\begin{figure}[H]
\centering
\begin{subfigure}[normla]{0.4\textwidth}
\caption{Estación Esc La Union Automaticacódigo 21201200 caso 3 variable Precipitación.}
\includegraphics[draft=false, scale=0.3]{comparacion_graficas_otras_var/201508_21201200_rain.png}
\end{subfigure}
~
\begin{subfigure}[normla]{0.4\textwidth}
\caption{Estación Esc La Union Automaticacódigo 21201200 caso 3 variable Bulbo húmedo.}
\includegraphics[draft=false, scale=0.3]{comparacion_graficas_otras_var/201508_21201200_wetbulb.png}
\end{subfigure}
~
\begin{subfigure}[normla]{0.4\textwidth}
\caption{Estación Pasquilla Automatica código 21201580 caso 3 variable Precipitación.}
\includegraphics[draft=false, scale=0.3]{comparacion_graficas_otras_var/201508_21201580_rain.png}
\end{subfigure}
~
\begin{subfigure}[normla]{0.4\textwidth}
\caption{Estación Univ Nacional código 21205012 caso 2 variable Punto de rocío.}
\includegraphics[draft=false, scale=0.3]{comparacion_graficas_otras_var/201408_21205012_dewpoint.png}
\end{subfigure}
~
\begin{subfigure}[normla]{0.4\textwidth}
\caption{Estación Univ Nacional código 21205012 caso 2 variable Humedad.}
\includegraphics[draft=false, scale=0.3]{comparacion_graficas_otras_var/201408_21205012_humedad.png}
\end{subfigure}
~
\begin{subfigure}[normla]{0.4\textwidth}
\caption{Estación Univ Nacional código 21205012 caso 2 variable Radiación.}
\includegraphics[draft=false, scale=0.3]{comparacion_graficas_otras_var/201408_21205012_radiacion.png}
\end{subfigure}
~
\end{figure}
           
\begin{figure}[H]
\centering
\begin{subfigure}[normla]{0.4\textwidth}
\caption{Estación Univ Nacional código 21205012 caso 2 variable Precipitación.}
\includegraphics[draft=false, scale=0.3]{comparacion_graficas_otras_var/201408_21205012_rain.png}
\end{subfigure}
~
\begin{subfigure}[normla]{0.4\textwidth}
\caption{Estación Univ Nacional código 21205012 caso 2 variable Velocidad del Viento.}
\includegraphics[draft=false, scale=0.3]{comparacion_graficas_otras_var/201408_21205012_vel_viento.png}
\end{subfigure}
~
\begin{subfigure}[normla]{0.4\textwidth}
\caption{Estación Univ Nacional código 21205012 caso 2 variable Bulbo húmedo.}
\includegraphics[draft=false, scale=0.3]{comparacion_graficas_otras_var/201408_21205012_wetbulb.png}
\end{subfigure}
~
\begin{subfigure}[normla]{0.4\textwidth}
\caption{Estación Univ Nacional código 21205012 caso 3 variable Punto de rocío.}
\includegraphics[draft=false, scale=0.3]{comparacion_graficas_otras_var/201508_21205012_dewpoint.png}
\end{subfigure}
~
\begin{subfigure}[normla]{0.4\textwidth}
\caption{Estación Univ Nacional código 21205012 caso 3 variable Humedad.}
\includegraphics[draft=false, scale=0.3]{comparacion_graficas_otras_var/201508_21205012_humedad.png}
\end{subfigure}
~
\begin{subfigure}[normla]{0.4\textwidth}
\caption{Estación Univ Nacional código 21205012 caso 3 variable Radiación.}
\includegraphics[draft=false, scale=0.3]{comparacion_graficas_otras_var/201508_21205012_radiacion.png}
\end{subfigure}
~
\end{figure}
           
\begin{figure}[H]
\centering
\begin{subfigure}[normla]{0.4\textwidth}
\caption{Estación Univ Nacional código 21205012 caso 3 variable Precipitación.}
\includegraphics[draft=false, scale=0.3]{comparacion_graficas_otras_var/201508_21205012_rain.png}
\end{subfigure}
~
\begin{subfigure}[normla]{0.4\textwidth}
\caption{Estación Univ Nacional código 21205012 caso 3 variable Velocidad del Viento.}
\includegraphics[draft=false, scale=0.3]{comparacion_graficas_otras_var/201508_21205012_vel_viento.png}
\end{subfigure}
~
\begin{subfigure}[normla]{0.4\textwidth}
\caption{Estación Univ Nacional código 21205012 caso 3 variable Bulbo húmedo.}
\includegraphics[draft=false, scale=0.3]{comparacion_graficas_otras_var/201508_21205012_wetbulb.png}
\end{subfigure}
~
\begin{subfigure}[normla]{0.4\textwidth}
\caption{Estación Univ Nacional código 21205012 caso 4 variable Punto de rocío.}
\includegraphics[draft=false, scale=0.3]{comparacion_graficas_otras_var/201509_21205012_dewpoint.png}
\end{subfigure}
~
\begin{subfigure}[normla]{0.4\textwidth}
\caption{Estación Univ Nacional código 21205012 caso 4 variable Humedad.}
\includegraphics[draft=false, scale=0.3]{comparacion_graficas_otras_var/201509_21205012_humedad.png}
\end{subfigure}
~
\begin{subfigure}[normla]{0.4\textwidth}
\caption{Estación Univ Nacional código 21205012 caso 4 variable Radiación.}
\includegraphics[draft=false, scale=0.3]{comparacion_graficas_otras_var/201509_21205012_radiacion.png}
\end{subfigure}
~
\end{figure}
           
\begin{figure}[H]
\centering
\begin{subfigure}[normla]{0.4\textwidth}
\caption{Estación Univ Nacional código 21205012 caso 4 variable Precipitación.}
\includegraphics[draft=false, scale=0.3]{comparacion_graficas_otras_var/201509_21205012_rain.png}
\end{subfigure}
~
\begin{subfigure}[normla]{0.4\textwidth}
\caption{Estación Univ Nacional código 21205012 caso 4 variable Velocidad del Viento.}
\includegraphics[draft=false, scale=0.3]{comparacion_graficas_otras_var/201509_21205012_vel_viento.png}
\end{subfigure}
~
\begin{subfigure}[normla]{0.4\textwidth}
\caption{Estación Univ Nacional código 21205012 caso 4 variable Bulbo húmedo.}
\includegraphics[draft=false, scale=0.3]{comparacion_graficas_otras_var/201509_21205012_wetbulb.png}
\end{subfigure}
~
\begin{subfigure}[normla]{0.4\textwidth}
\caption{Estación Apto El Dorado código 21205791 caso 2 variable Punto de rocío.}
\includegraphics[draft=false, scale=0.3]{comparacion_graficas_otras_var/201408_21205791_dewpoint.png}
\end{subfigure}
~
\begin{subfigure}[normla]{0.4\textwidth}
\caption{Estación Apto El Dorado código 21205791 caso 2 variable Humedad.}
\includegraphics[draft=false, scale=0.3]{comparacion_graficas_otras_var/201408_21205791_humedad.png}
\end{subfigure}
~
\begin{subfigure}[normla]{0.4\textwidth}
\caption{Estación Apto El Dorado código 21205791 caso 2 variable Radiación.}
\includegraphics[draft=false, scale=0.3]{comparacion_graficas_otras_var/201408_21205791_radiacion.png}
\end{subfigure}
~
\end{figure}
           
\begin{figure}[H]
\centering
\begin{subfigure}[normla]{0.4\textwidth}
\caption{Estación Apto El Dorado código 21205791 caso 2 variable Precipitación.}
\includegraphics[draft=false, scale=0.3]{comparacion_graficas_otras_var/201408_21205791_rain.png}
\end{subfigure}
~
\begin{subfigure}[normla]{0.4\textwidth}
\caption{Estación Apto El Dorado código 21205791 caso 2 variable Velocidad del Viento.}
\includegraphics[draft=false, scale=0.3]{comparacion_graficas_otras_var/201408_21205791_vel_viento.png}
\end{subfigure}
~
\begin{subfigure}[normla]{0.4\textwidth}
\caption{Estación Apto El Dorado código 21205791 caso 2 variable Bulbo húmedo.}
\includegraphics[draft=false, scale=0.3]{comparacion_graficas_otras_var/201408_21205791_wetbulb.png}
\end{subfigure}
~
\begin{subfigure}[normla]{0.4\textwidth}
\caption{Estación Apto El Dorado código 21205791 caso 3 variable Radiación.}
\includegraphics[draft=false, scale=0.3]{comparacion_graficas_otras_var/201508_21205791_radiacion.png}
\end{subfigure}
~
\begin{subfigure}[normla]{0.4\textwidth}
\caption{Estación Apto El Dorado código 21205791 caso 3 variable Punto de rocío.}
\includegraphics[draft=false, scale=0.3]{comparacion_graficas_otras_var/201508_21205791_dewpoint.png}
\end{subfigure}
~
\begin{subfigure}[normla]{0.4\textwidth}
\caption{Estación Apto El Dorado código 21205791 caso 3 variable Humedad.}
\includegraphics[draft=false, scale=0.3]{comparacion_graficas_otras_var/201508_21205791_humedad.png}
\end{subfigure}
~
\end{figure}
           
\begin{figure}[H]
\centering
\begin{subfigure}[normla]{0.4\textwidth}
\caption{Estación Apto El Dorado código 21205791 caso 3 variable Precipitación.}
\includegraphics[draft=false, scale=0.3]{comparacion_graficas_otras_var/201508_21205791_rain.png}
\end{subfigure}
~
\begin{subfigure}[normla]{0.4\textwidth}
\caption{Estación Apto El Dorado código 21205791 caso 3 variable Velocidad del Viento.}
\includegraphics[draft=false, scale=0.3]{comparacion_graficas_otras_var/201508_21205791_vel_viento.png}
\end{subfigure}
~
\begin{subfigure}[normla]{0.4\textwidth}
\caption{Estación Apto El Dorado código 21205791 caso 3 variable Bulbo húmedo.}
\includegraphics[draft=false, scale=0.3]{comparacion_graficas_otras_var/201508_21205791_wetbulb.png}
\end{subfigure}
~
\begin{subfigure}[normla]{0.4\textwidth}
\caption{Estación Apto El Dorado código 21205791 caso 4 variable Punto de rocío.}
\includegraphics[draft=false, scale=0.3]{comparacion_graficas_otras_var/201509_21205791_dewpoint.png}
\end{subfigure}
~
\begin{subfigure}[normla]{0.4\textwidth}
\caption{Estación Apto El Dorado código 21205791 caso 4 variable Humedad.}
\includegraphics[draft=false, scale=0.3]{comparacion_graficas_otras_var/201509_21205791_humedad.png}
\end{subfigure}
~
\begin{subfigure}[normla]{0.4\textwidth}
\caption{Estación Apto El Dorado código 21205791 caso 4 variable Radiación.}
\includegraphics[draft=false, scale=0.3]{comparacion_graficas_otras_var/201509_21205791_radiacion.png}
\end{subfigure}
~
\end{figure}
           
\begin{figure}[H]
\centering
\begin{subfigure}[normla]{0.4\textwidth}
\caption{Estación Apto El Dorado código 21205791 caso 4 variable Precipitación.}
\includegraphics[draft=false, scale=0.3]{comparacion_graficas_otras_var/201509_21205791_rain.png}
\end{subfigure}
~
\begin{subfigure}[normla]{0.4\textwidth}
\caption{Estación Apto El Dorado código 21205791 caso 4 variable Velocidad del Viento.}
\includegraphics[draft=false, scale=0.3]{comparacion_graficas_otras_var/201509_21205791_vel_viento.png}
\end{subfigure}
~
\begin{subfigure}[normla]{0.4\textwidth}
\caption{Estación Apto El Dorado código 21205791 caso 4 variable Bulbo húmedo.}
\includegraphics[draft=false, scale=0.3]{comparacion_graficas_otras_var/201509_21205791_wetbulb.png}
\end{subfigure}
~
\begin{subfigure}[normla]{0.4\textwidth}
\caption{Estación Nueva Generacion código 21206600 caso 3 variable Punto de rocío.}
\includegraphics[draft=false, scale=0.3]{comparacion_graficas_otras_var/201508_21206600_dewpoint.png}
\end{subfigure}
~
\begin{subfigure}[normla]{0.4\textwidth}
\caption{Estación Nueva Generacion código 21206600 caso 3 variable Humedad.}
\includegraphics[draft=false, scale=0.3]{comparacion_graficas_otras_var/201508_21206600_humedad.png}
\end{subfigure}
~
\begin{subfigure}[normla]{0.4\textwidth}
\caption{Estación Nueva Generacion código 21206600 caso 3 variable Radiación.}
\includegraphics[draft=false, scale=0.3]{comparacion_graficas_otras_var/201508_21206600_radiacion.png}
\end{subfigure}
~
\end{figure}
           
\begin{figure}[H]
\centering
\begin{subfigure}[normla]{0.4\textwidth}
\caption{Estación Nueva Generacion código 21206600 caso 3 variable Precipitación.}
\includegraphics[draft=false, scale=0.3]{comparacion_graficas_otras_var/201508_21206600_rain.png}
\end{subfigure}
~
\begin{subfigure}[normla]{0.4\textwidth}
\caption{Estación Nueva Generacion código 21206600 caso 3 variable Velocidad del Viento.}
\includegraphics[draft=false, scale=0.3]{comparacion_graficas_otras_var/201508_21206600_vel_viento.png}
\end{subfigure}
~
\begin{subfigure}[normla]{0.4\textwidth}
\caption{Estación Nueva Generacion código 21206600 caso 3 variable Bulbo húmedo.}
\includegraphics[draft=false, scale=0.3]{comparacion_graficas_otras_var/201508_21206600_wetbulb.png}
\end{subfigure}
~
\begin{subfigure}[normla]{0.4\textwidth}
\caption{Estación Nueva Generacion código 21206600 caso 4 variable Punto de rocío.}
\includegraphics[draft=false, scale=0.3]{comparacion_graficas_otras_var/201509_21206600_dewpoint.png}
\end{subfigure}
~
\begin{subfigure}[normla]{0.4\textwidth}
\caption{Estación Nueva Generacion código 21206600 caso 4 variable Humedad.}
\includegraphics[draft=false, scale=0.3]{comparacion_graficas_otras_var/201509_21206600_humedad.png}
\end{subfigure}
~
\begin{subfigure}[normla]{0.4\textwidth}
\caption{Estación Nueva Generacion código 21206600 caso 4 variable Radiación.}
\includegraphics[draft=false, scale=0.3]{comparacion_graficas_otras_var/201509_21206600_radiacion.png}
\end{subfigure}
~
\end{figure}
           
\begin{figure}[H]
\centering
\begin{subfigure}[normla]{0.4\textwidth}
\caption{Estación Nueva Generacion código 21206600 caso 4 variable Precipitación.}
\includegraphics[draft=false, scale=0.3]{comparacion_graficas_otras_var/201509_21206600_rain.png}
\end{subfigure}
~
\begin{subfigure}[normla]{0.4\textwidth}
\caption{Estación Nueva Generacion código 21206600 caso 4 variable Velocidad del Viento.}
\includegraphics[draft=false, scale=0.3]{comparacion_graficas_otras_var/201509_21206600_vel_viento.png}
\end{subfigure}
~
\begin{subfigure}[normla]{0.4\textwidth}
\caption{Estación Nueva Generacion código 21206600 caso 4 variable Bulbo húmedo.}
\includegraphics[draft=false, scale=0.3]{comparacion_graficas_otras_var/201509_21206600_wetbulb.png}
\end{subfigure}
~
\begin{subfigure}[normla]{0.4\textwidth}
\caption{Estación Hda Sta Ana Autom código 21206790 caso 1 variable Punto de rocío.}
\includegraphics[draft=false, scale=0.3]{comparacion_graficas_otras_var/200702_21206790_dewpoint.png}
\end{subfigure}
~
\begin{subfigure}[normla]{0.4\textwidth}
\caption{Estación Hda Sta Ana Autom código 21206790 caso 1 variable Humedad.}
\includegraphics[draft=false, scale=0.3]{comparacion_graficas_otras_var/200702_21206790_humedad.png}
\end{subfigure}
~
\begin{subfigure}[normla]{0.4\textwidth}
\caption{Estación Hda Sta Ana Autom código 21206790 caso 1 variable Radiación.}
\includegraphics[draft=false, scale=0.3]{comparacion_graficas_otras_var/200702_21206790_radiacion.png}
\end{subfigure}
~
\end{figure}
           
\begin{figure}[H]
\centering
\begin{subfigure}[normla]{0.4\textwidth}
\caption{Estación Hda Sta Ana Autom código 21206790 caso 1 variable Precipitación.}
\includegraphics[draft=false, scale=0.3]{comparacion_graficas_otras_var/200702_21206790_rain.png}
\end{subfigure}
~
\begin{subfigure}[normla]{0.4\textwidth}
\caption{Estación Hda Sta Ana Autom código 21206790 caso 1 variable Velocidad del Viento.}
\includegraphics[draft=false, scale=0.3]{comparacion_graficas_otras_var/200702_21206790_vel_viento.png}
\end{subfigure}
~
\begin{subfigure}[normla]{0.4\textwidth}
\caption{Estación Hda Sta Ana Autom código 21206790 caso 1 variable Bulbo húmedo.}
\includegraphics[draft=false, scale=0.3]{comparacion_graficas_otras_var/200702_21206790_wetbulb.png}
\end{subfigure}
~
\begin{subfigure}[normla]{0.4\textwidth}
\caption{Estación Hda Sta Ana Autom código 21206790 caso 2 variable Humedad.}
\includegraphics[draft=false, scale=0.3]{comparacion_graficas_otras_var/201408_21206790_humedad.png}
\end{subfigure}
~
\begin{subfigure}[normla]{0.4\textwidth}
\caption{Estación Hda Sta Ana Autom código 21206790 caso 2 variable Punto de rocío.}
\includegraphics[draft=false, scale=0.3]{comparacion_graficas_otras_var/201408_21206790_dewpoint.png}
\end{subfigure}
~
\begin{subfigure}[normla]{0.4\textwidth}
\caption{Estación Hda Sta Ana Autom código 21206790 caso 2 variable Radiación.}
\includegraphics[draft=false, scale=0.3]{comparacion_graficas_otras_var/201408_21206790_radiacion.png}
\end{subfigure}
~
\end{figure}
           
\begin{figure}[H]
\centering
\begin{subfigure}[normla]{0.4\textwidth}
\caption{Estación Hda Sta Ana Autom código 21206790 caso 2 variable Precipitación.}
\includegraphics[draft=false, scale=0.3]{comparacion_graficas_otras_var/201408_21206790_rain.png}
\end{subfigure}
~
\begin{subfigure}[normla]{0.4\textwidth}
\caption{Estación Hda Sta Ana Autom código 21206790 caso 2 variable Velocidad del Viento.}
\includegraphics[draft=false, scale=0.3]{comparacion_graficas_otras_var/201408_21206790_vel_viento.png}
\end{subfigure}
~
\begin{subfigure}[normla]{0.4\textwidth}
\caption{Estación Hda Sta Ana Autom código 21206790 caso 2 variable Bulbo húmedo.}
\includegraphics[draft=false, scale=0.3]{comparacion_graficas_otras_var/201408_21206790_wetbulb.png}
\end{subfigure}
~
\begin{subfigure}[normla]{0.4\textwidth}
\caption{Estación Hda Sta Ana Autom código 21206790 caso 3 variable Punto de rocío.}
\includegraphics[draft=false, scale=0.3]{comparacion_graficas_otras_var/201508_21206790_dewpoint.png}
\end{subfigure}
~
\begin{subfigure}[normla]{0.4\textwidth}
\caption{Estación Hda Sta Ana Autom código 21206790 caso 3 variable Humedad.}
\includegraphics[draft=false, scale=0.3]{comparacion_graficas_otras_var/201508_21206790_humedad.png}
\end{subfigure}
~
\begin{subfigure}[normla]{0.4\textwidth}
\caption{Estación Hda Sta Ana Autom código 21206790 caso 3 variable Radiación.}
\includegraphics[draft=false, scale=0.3]{comparacion_graficas_otras_var/201508_21206790_radiacion.png}
\end{subfigure}
~
\end{figure}
           
\begin{figure}[H]
\centering
\begin{subfigure}[normla]{0.4\textwidth}
\caption{Estación Hda Sta Ana Autom código 21206790 caso 3 variable Precipitación.}
\includegraphics[draft=false, scale=0.3]{comparacion_graficas_otras_var/201508_21206790_rain.png}
\end{subfigure}
~
\begin{subfigure}[normla]{0.4\textwidth}
\caption{Estación Hda Sta Ana Autom código 21206790 caso 3 variable Velocidad del Viento.}
\includegraphics[draft=false, scale=0.3]{comparacion_graficas_otras_var/201508_21206790_vel_viento.png}
\end{subfigure}
~
\begin{subfigure}[normla]{0.4\textwidth}
\caption{Estación Hda Sta Ana Autom código 21206790 caso 3 variable Bulbo húmedo.}
\includegraphics[draft=false, scale=0.3]{comparacion_graficas_otras_var/201508_21206790_wetbulb.png}
\end{subfigure}
~
\begin{subfigure}[normla]{0.4\textwidth}
\caption{Estación Hda Sta Ana Autom código 21206790 caso 4 variable Humedad.}
\includegraphics[draft=false, scale=0.3]{comparacion_graficas_otras_var/201509_21206790_humedad.png}
\end{subfigure}
~
\begin{subfigure}[normla]{0.4\textwidth}
\caption{Estación Hda Sta Ana Autom código 21206790 caso 4 variable Punto de rocío.}
\includegraphics[draft=false, scale=0.3]{comparacion_graficas_otras_var/201509_21206790_dewpoint.png}
\end{subfigure}
~
\begin{subfigure}[normla]{0.4\textwidth}
\caption{Estación Hda Sta Ana Autom código 21206790 caso 4 variable Radiación.}
\includegraphics[draft=false, scale=0.3]{comparacion_graficas_otras_var/201509_21206790_radiacion.png}
\end{subfigure}
~
\end{figure}
           
\begin{figure}[H]
\centering
\begin{subfigure}[normla]{0.4\textwidth}
\caption{Estación Hda Sta Ana Autom código 21206790 caso 4 variable Precipitación.}
\includegraphics[draft=false, scale=0.3]{comparacion_graficas_otras_var/201509_21206790_rain.png}
\end{subfigure}
~
\begin{subfigure}[normla]{0.4\textwidth}
\caption{Estación Hda Sta Ana Autom código 21206790 caso 4 variable Velocidad del Viento.}
\includegraphics[draft=false, scale=0.3]{comparacion_graficas_otras_var/201509_21206790_vel_viento.png}
\end{subfigure}
~
\begin{subfigure}[normla]{0.4\textwidth}
\caption{Estación Hda Sta Ana Autom código 21206790 caso 4 variable Bulbo húmedo.}
\includegraphics[draft=false, scale=0.3]{comparacion_graficas_otras_var/201509_21206790_wetbulb.png}
\end{subfigure}
~
\begin{subfigure}[normla]{0.4\textwidth}
\caption{Estación Villa Teresa Automatica código 21206920 caso 3 variable Punto de rocío.}
\includegraphics[draft=false, scale=0.3]{comparacion_graficas_otras_var/201508_21206920_dewpoint.png}
\end{subfigure}
~
\begin{subfigure}[normla]{0.4\textwidth}
\caption{Estación Villa Teresa Automatica código 21206920 caso 3 variable Humedad.}
\includegraphics[draft=false, scale=0.3]{comparacion_graficas_otras_var/201508_21206920_humedad.png}
\end{subfigure}
~
\begin{subfigure}[normla]{0.4\textwidth}
\caption{Estación Villa Teresa Automatica código 21206920 caso 3 variable Radiación.}
\includegraphics[draft=false, scale=0.3]{comparacion_graficas_otras_var/201508_21206920_radiacion.png}
\end{subfigure}
~
\end{figure}
           
\begin{figure}[H]
\centering
\begin{subfigure}[normla]{0.4\textwidth}
\caption{Estación Villa Teresa Automatica código 21206920 caso 3 variable Precipitación.}
\includegraphics[draft=false, scale=0.3]{comparacion_graficas_otras_var/201508_21206920_rain.png}
\end{subfigure}
~
\begin{subfigure}[normla]{0.4\textwidth}
\caption{Estación Villa Teresa Automatica código 21206920 caso 3 variable Velocidad del Viento.}
\includegraphics[draft=false, scale=0.3]{comparacion_graficas_otras_var/201508_21206920_vel_viento.png}
\end{subfigure}
~
\begin{subfigure}[normla]{0.4\textwidth}
\caption{Estación Villa Teresa Automatica código 21206920 caso 3 variable Bulbo húmedo.}
\includegraphics[draft=false, scale=0.3]{comparacion_graficas_otras_var/201508_21206920_wetbulb.png}
\end{subfigure}
~
\begin{subfigure}[normla]{0.4\textwidth}
\caption{Estación Villa Teresa Automatica código 21206920 caso 4 variable Punto de rocío.}
\includegraphics[draft=false, scale=0.3]{comparacion_graficas_otras_var/201509_21206920_dewpoint.png}
\end{subfigure}
~
\begin{subfigure}[normla]{0.4\textwidth}
\caption{Estación Villa Teresa Automatica código 21206920 caso 4 variable Humedad.}
\includegraphics[draft=false, scale=0.3]{comparacion_graficas_otras_var/201509_21206920_humedad.png}
\end{subfigure}
~
\begin{subfigure}[normla]{0.4\textwidth}
\caption{Estación Villa Teresa Automatica código 21206920 caso 4 variable Radiación.}
\includegraphics[draft=false, scale=0.3]{comparacion_graficas_otras_var/201509_21206920_radiacion.png}
\end{subfigure}
~
\end{figure}
           
\begin{figure}[H]
\centering
\begin{subfigure}[normla]{0.4\textwidth}
\caption{Estación Villa Teresa Automatica código 21206920 caso 4 variable Precipitación.}
\includegraphics[draft=false, scale=0.3]{comparacion_graficas_otras_var/201509_21206920_rain.png}
\end{subfigure}
~
\begin{subfigure}[normla]{0.4\textwidth}
\caption{Estación Villa Teresa Automatica código 21206920 caso 4 variable Velocidad del Viento.}
\includegraphics[draft=false, scale=0.3]{comparacion_graficas_otras_var/201509_21206920_vel_viento.png}
\end{subfigure}
~
\begin{subfigure}[normla]{0.4\textwidth}
\caption{Estación Villa Teresa Automatica código 21206920 caso 4 variable Bulbo húmedo.}
\includegraphics[draft=false, scale=0.3]{comparacion_graficas_otras_var/201509_21206920_wetbulb.png}
\end{subfigure}
~
\begin{subfigure}[normla]{0.4\textwidth}
\caption{Estación Pmo Guerrero código 21206930 caso 1 variable Punto de rocío.}
\includegraphics[draft=false, scale=0.3]{comparacion_graficas_otras_var/200702_21206930_dewpoint.png}
\end{subfigure}
~
\begin{subfigure}[normla]{0.4\textwidth}
\caption{Estación Pmo Guerrero código 21206930 caso 1 variable Humedad.}
\includegraphics[draft=false, scale=0.3]{comparacion_graficas_otras_var/200702_21206930_humedad.png}
\end{subfigure}
~
\begin{subfigure}[normla]{0.4\textwidth}
\caption{Estación Pmo Guerrero código 21206930 caso 1 variable Precipitación.}
\includegraphics[draft=false, scale=0.3]{comparacion_graficas_otras_var/200702_21206930_rain.png}
\end{subfigure}
~
\end{figure}
           
\begin{figure}[H]
\centering
\begin{subfigure}[normla]{0.4\textwidth}
\caption{Estación Pmo Guerrero código 21206930 caso 1 variable Velocidad del Viento.}
\includegraphics[draft=false, scale=0.3]{comparacion_graficas_otras_var/200702_21206930_vel_viento.png}
\end{subfigure}
~
\begin{subfigure}[normla]{0.4\textwidth}
\caption{Estación Pmo Guerrero código 21206930 caso 1 variable Bulbo húmedo.}
\includegraphics[draft=false, scale=0.3]{comparacion_graficas_otras_var/200702_21206930_wetbulb.png}
\end{subfigure}
~
\begin{subfigure}[normla]{0.4\textwidth}
\caption{Estación Pmo Guerrero código 21206930 caso 2 variable Punto de rocío.}
\includegraphics[draft=false, scale=0.3]{comparacion_graficas_otras_var/201408_21206930_dewpoint.png}
\end{subfigure}
~
\begin{subfigure}[normla]{0.4\textwidth}
\caption{Estación Pmo Guerrero código 21206930 caso 2 variable Humedad.}
\includegraphics[draft=false, scale=0.3]{comparacion_graficas_otras_var/201408_21206930_humedad.png}
\end{subfigure}
~
\begin{subfigure}[normla]{0.4\textwidth}
\caption{Estación Pmo Guerrero código 21206930 caso 2 variable Precipitación.}
\includegraphics[draft=false, scale=0.3]{comparacion_graficas_otras_var/201408_21206930_rain.png}
\end{subfigure}
~
\begin{subfigure}[normla]{0.4\textwidth}
\caption{Estación Pmo Guerrero código 21206930 caso 2 variable Velocidad del Viento.}
\includegraphics[draft=false, scale=0.3]{comparacion_graficas_otras_var/201408_21206930_vel_viento.png}
\end{subfigure}
~
\end{figure}
           
\begin{figure}[H]
\centering
\begin{subfigure}[normla]{0.4\textwidth}
\caption{Estación Pmo Guerrero código 21206930 caso 2 variable Bulbo húmedo.}
\includegraphics[draft=false, scale=0.3]{comparacion_graficas_otras_var/201408_21206930_wetbulb.png}
\end{subfigure}
~
\begin{subfigure}[normla]{0.4\textwidth}
\caption{Estación Pmo Guerrero código 21206930 caso 3 variable Humedad.}
\includegraphics[draft=false, scale=0.3]{comparacion_graficas_otras_var/201508_21206930_humedad.png}
\end{subfigure}
~
\begin{subfigure}[normla]{0.4\textwidth}
\caption{Estación Pmo Guerrero código 21206930 caso 3 variable Precipitación.}
\includegraphics[draft=false, scale=0.3]{comparacion_graficas_otras_var/201508_21206930_rain.png}
\end{subfigure}
~
\begin{subfigure}[normla]{0.4\textwidth}
\caption{Estación Pmo Guerrero código 21206930 caso 3 variable Velocidad del Viento.}
\includegraphics[draft=false, scale=0.3]{comparacion_graficas_otras_var/201508_21206930_vel_viento.png}
\end{subfigure}
~
\begin{subfigure}[normla]{0.4\textwidth}
\caption{Estación Pmo Guerrero código 21206930 caso 3 variable Bulbo húmedo.}
\includegraphics[draft=false, scale=0.3]{comparacion_graficas_otras_var/201508_21206930_wetbulb.png}
\end{subfigure}
~
\begin{subfigure}[normla]{0.4\textwidth}
\caption{Estación Pmo Guerrero código 21206930 caso 3 variable Punto de rocío.}
\includegraphics[draft=false, scale=0.3]{comparacion_graficas_otras_var/201508_21206930_dewpoint.png}
\end{subfigure}
~
\end{figure}
           
\begin{figure}[H]
\centering
\begin{subfigure}[normla]{0.4\textwidth}
\caption{Estación Pmo Guerrero código 21206930 caso 4 variable Punto de rocío.}
\includegraphics[draft=false, scale=0.3]{comparacion_graficas_otras_var/201509_21206930_dewpoint.png}
\end{subfigure}
~
\begin{subfigure}[normla]{0.4\textwidth}
\caption{Estación Pmo Guerrero código 21206930 caso 4 variable Humedad.}
\includegraphics[draft=false, scale=0.3]{comparacion_graficas_otras_var/201509_21206930_humedad.png}
\end{subfigure}
~
\begin{subfigure}[normla]{0.4\textwidth}
\caption{Estación Pmo Guerrero código 21206930 caso 4 variable Precipitación.}
\includegraphics[draft=false, scale=0.3]{comparacion_graficas_otras_var/201509_21206930_rain.png}
\end{subfigure}
~
\begin{subfigure}[normla]{0.4\textwidth}
\caption{Estación Pmo Guerrero código 21206930 caso 4 variable Velocidad del Viento.}
\includegraphics[draft=false, scale=0.3]{comparacion_graficas_otras_var/201509_21206930_vel_viento.png}
\end{subfigure}
~
\begin{subfigure}[normla]{0.4\textwidth}
\caption{Estación Pmo Guerrero código 21206930 caso 4 variable Bulbo húmedo.}
\includegraphics[draft=false, scale=0.3]{comparacion_graficas_otras_var/201509_21206930_wetbulb.png}
\end{subfigure}
~
\begin{subfigure}[normla]{0.4\textwidth}
\caption{Estación Ciudad Bolivar código 21206940 caso 1 variable Radiación.}
\includegraphics[draft=false, scale=0.3]{comparacion_graficas_otras_var/200702_21206940_radiacion.png}
\end{subfigure}
~
\end{figure}
           
\begin{figure}[H]
\centering
\begin{subfigure}[normla]{0.4\textwidth}
\caption{Estación Ciudad Bolivar código 21206940 caso 1 variable Precipitación.}
\includegraphics[draft=false, scale=0.3]{comparacion_graficas_otras_var/200702_21206940_rain.png}
\end{subfigure}
~
\begin{subfigure}[normla]{0.4\textwidth}
\caption{Estación Pmo Guacheneque código 21206950 caso 1 variable Humedad.}
\includegraphics[draft=false, scale=0.3]{comparacion_graficas_otras_var/200702_21206950_humedad.png}
\end{subfigure}
~
\begin{subfigure}[normla]{0.4\textwidth}
\caption{Estación Pmo Guacheneque código 21206950 caso 1 variable Precipitación.}
\includegraphics[draft=false, scale=0.3]{comparacion_graficas_otras_var/200702_21206950_rain.png}
\end{subfigure}
~
\begin{subfigure}[normla]{0.4\textwidth}
\caption{Estación Pmo Guacheneque código 21206950 caso 1 variable Velocidad del Viento.}
\includegraphics[draft=false, scale=0.3]{comparacion_graficas_otras_var/200702_21206950_vel_viento.png}
\end{subfigure}
~
\begin{subfigure}[normla]{0.4\textwidth}
\caption{Estación Pmo Guacheneque código 21206950 caso 1 variable Bulbo húmedo.}
\includegraphics[draft=false, scale=0.3]{comparacion_graficas_otras_var/200702_21206950_wetbulb.png}
\end{subfigure}
~
\begin{subfigure}[normla]{0.4\textwidth}
\caption{Estación Pmo Guacheneque código 21206950 caso 1 variable Punto de rocío.}
\includegraphics[draft=false, scale=0.3]{comparacion_graficas_otras_var/200702_21206950_dewpoint.png}
\end{subfigure}
~
\end{figure}
           
\begin{figure}[H]
\centering
\begin{subfigure}[normla]{0.4\textwidth}
\caption{Estación Pmo Guacheneque código 21206950 caso 2 variable Punto de rocío.}
\includegraphics[draft=false, scale=0.3]{comparacion_graficas_otras_var/201408_21206950_dewpoint.png}
\end{subfigure}
~
\begin{subfigure}[normla]{0.4\textwidth}
\caption{Estación Pmo Guacheneque código 21206950 caso 2 variable Humedad.}
\includegraphics[draft=false, scale=0.3]{comparacion_graficas_otras_var/201408_21206950_humedad.png}
\end{subfigure}
~
\begin{subfigure}[normla]{0.4\textwidth}
\caption{Estación Pmo Guacheneque código 21206950 caso 2 variable Precipitación.}
\includegraphics[draft=false, scale=0.3]{comparacion_graficas_otras_var/201408_21206950_rain.png}
\end{subfigure}
~
\begin{subfigure}[normla]{0.4\textwidth}
\caption{Estación Pmo Guacheneque código 21206950 caso 2 variable Velocidad del Viento.}
\includegraphics[draft=false, scale=0.3]{comparacion_graficas_otras_var/201408_21206950_vel_viento.png}
\end{subfigure}
~
\begin{subfigure}[normla]{0.4\textwidth}
\caption{Estación Pmo Guacheneque código 21206950 caso 2 variable Bulbo húmedo.}
\includegraphics[draft=false, scale=0.3]{comparacion_graficas_otras_var/201408_21206950_wetbulb.png}
\end{subfigure}
~
\begin{subfigure}[normla]{0.4\textwidth}
\caption{Estación Ideam Bogota código 21206960 caso 2 variable Punto de rocío.}
\includegraphics[draft=false, scale=0.3]{comparacion_graficas_otras_var/201408_21206960_dewpoint.png}
\end{subfigure}
~
\end{figure}
           
\begin{figure}[H]
\centering
\begin{subfigure}[normla]{0.4\textwidth}
\caption{Estación Ideam Bogota código 21206960 caso 2 variable Humedad.}
\includegraphics[draft=false, scale=0.3]{comparacion_graficas_otras_var/201408_21206960_humedad.png}
\end{subfigure}
~
\begin{subfigure}[normla]{0.4\textwidth}
\caption{Estación Ideam Bogota código 21206960 caso 2 variable Radiación.}
\includegraphics[draft=false, scale=0.3]{comparacion_graficas_otras_var/201408_21206960_radiacion.png}
\end{subfigure}
~
\begin{subfigure}[normla]{0.4\textwidth}
\caption{Estación Ideam Bogota código 21206960 caso 2 variable Precipitación.}
\includegraphics[draft=false, scale=0.3]{comparacion_graficas_otras_var/201408_21206960_rain.png}
\end{subfigure}
~
\begin{subfigure}[normla]{0.4\textwidth}
\caption{Estación Ideam Bogota código 21206960 caso 2 variable Velocidad del Viento.}
\includegraphics[draft=false, scale=0.3]{comparacion_graficas_otras_var/201408_21206960_vel_viento.png}
\end{subfigure}
~
\begin{subfigure}[normla]{0.4\textwidth}
\caption{Estación Ideam Bogota código 21206960 caso 2 variable Bulbo húmedo.}
\includegraphics[draft=false, scale=0.3]{comparacion_graficas_otras_var/201408_21206960_wetbulb.png}
\end{subfigure}
~
\begin{subfigure}[normla]{0.4\textwidth}
\caption{Estación Ideam Bogota código 21206960 caso 3 variable Punto de rocío.}
\includegraphics[draft=false, scale=0.3]{comparacion_graficas_otras_var/201508_21206960_dewpoint.png}
\end{subfigure}
~
\end{figure}
           
\begin{figure}[H]
\centering
\begin{subfigure}[normla]{0.4\textwidth}
\caption{Estación Ideam Bogota código 21206960 caso 3 variable Humedad.}
\includegraphics[draft=false, scale=0.3]{comparacion_graficas_otras_var/201508_21206960_humedad.png}
\end{subfigure}
~
\begin{subfigure}[normla]{0.4\textwidth}
\caption{Estación Ideam Bogota código 21206960 caso 3 variable Radiación.}
\includegraphics[draft=false, scale=0.3]{comparacion_graficas_otras_var/201508_21206960_radiacion.png}
\end{subfigure}
~
\begin{subfigure}[normla]{0.4\textwidth}
\caption{Estación Ideam Bogota código 21206960 caso 3 variable Precipitación.}
\includegraphics[draft=false, scale=0.3]{comparacion_graficas_otras_var/201508_21206960_rain.png}
\end{subfigure}
~
\begin{subfigure}[normla]{0.4\textwidth}
\caption{Estación Ideam Bogota código 21206960 caso 3 variable Velocidad del Viento.}
\includegraphics[draft=false, scale=0.3]{comparacion_graficas_otras_var/201508_21206960_vel_viento.png}
\end{subfigure}
~
\begin{subfigure}[normla]{0.4\textwidth}
\caption{Estación Ideam Bogota código 21206960 caso 3 variable Bulbo húmedo.}
\includegraphics[draft=false, scale=0.3]{comparacion_graficas_otras_var/201508_21206960_wetbulb.png}
\end{subfigure}
~
\begin{subfigure}[normla]{0.4\textwidth}
\caption{Estación Ideam Bogota código 21206960 caso 4 variable Punto de rocío.}
\includegraphics[draft=false, scale=0.3]{comparacion_graficas_otras_var/201509_21206960_dewpoint.png}
\end{subfigure}
~
\end{figure}
           
\begin{figure}[H]
\centering
\begin{subfigure}[normla]{0.4\textwidth}
\caption{Estación Ideam Bogota código 21206960 caso 4 variable Humedad.}
\includegraphics[draft=false, scale=0.3]{comparacion_graficas_otras_var/201509_21206960_humedad.png}
\end{subfigure}
~
\begin{subfigure}[normla]{0.4\textwidth}
\caption{Estación Ideam Bogota código 21206960 caso 4 variable Radiación.}
\includegraphics[draft=false, scale=0.3]{comparacion_graficas_otras_var/201509_21206960_radiacion.png}
\end{subfigure}
~
\begin{subfigure}[normla]{0.4\textwidth}
\caption{Estación Ideam Bogota código 21206960 caso 4 variable Precipitación.}
\includegraphics[draft=false, scale=0.3]{comparacion_graficas_otras_var/201509_21206960_rain.png}
\end{subfigure}
~
\begin{subfigure}[normla]{0.4\textwidth}
\caption{Estación Ideam Bogota código 21206960 caso 4 variable Velocidad del Viento.}
\includegraphics[draft=false, scale=0.3]{comparacion_graficas_otras_var/201509_21206960_vel_viento.png}
\end{subfigure}
~
\begin{subfigure}[normla]{0.4\textwidth}
\caption{Estación Ideam Bogota código 21206960 caso 4 variable Bulbo húmedo.}
\includegraphics[draft=false, scale=0.3]{comparacion_graficas_otras_var/201509_21206960_wetbulb.png}
\end{subfigure}
~
\begin{subfigure}[normla]{0.4\textwidth}
\caption{Estación Sta Cruz De Siecha código 21206980 caso 1 variable Punto de rocío.}
\includegraphics[draft=false, scale=0.3]{comparacion_graficas_otras_var/200702_21206980_dewpoint.png}
\end{subfigure}
~
\end{figure}
           
\begin{figure}[H]
\centering
\begin{subfigure}[normla]{0.4\textwidth}
\caption{Estación Sta Cruz De Siecha código 21206980 caso 1 variable Humedad.}
\includegraphics[draft=false, scale=0.3]{comparacion_graficas_otras_var/200702_21206980_humedad.png}
\end{subfigure}
~
\begin{subfigure}[normla]{0.4\textwidth}
\caption{Estación Sta Cruz De Siecha código 21206980 caso 1 variable Radiación.}
\includegraphics[draft=false, scale=0.3]{comparacion_graficas_otras_var/200702_21206980_radiacion.png}
\end{subfigure}
~
\begin{subfigure}[normla]{0.4\textwidth}
\caption{Estación Sta Cruz De Siecha código 21206980 caso 1 variable Precipitación.}
\includegraphics[draft=false, scale=0.3]{comparacion_graficas_otras_var/200702_21206980_rain.png}
\end{subfigure}
~
\begin{subfigure}[normla]{0.4\textwidth}
\caption{Estación Sta Cruz De Siecha código 21206980 caso 1 variable Velocidad del Viento.}
\includegraphics[draft=false, scale=0.3]{comparacion_graficas_otras_var/200702_21206980_vel_viento.png}
\end{subfigure}
~
\begin{subfigure}[normla]{0.4\textwidth}
\caption{Estación Sta Cruz De Siecha código 21206980 caso 1 variable Bulbo húmedo.}
\includegraphics[draft=false, scale=0.3]{comparacion_graficas_otras_var/200702_21206980_wetbulb.png}
\end{subfigure}
~
\begin{subfigure}[normla]{0.4\textwidth}
\caption{Estación Sta Cruz De Siecha código 21206980 caso 2 variable Punto de rocío.}
\includegraphics[draft=false, scale=0.3]{comparacion_graficas_otras_var/201408_21206980_dewpoint.png}
\end{subfigure}
~
\end{figure}
           
\begin{figure}[H]
\centering
\begin{subfigure}[normla]{0.4\textwidth}
\caption{Estación Sta Cruz De Siecha código 21206980 caso 2 variable Humedad.}
\includegraphics[draft=false, scale=0.3]{comparacion_graficas_otras_var/201408_21206980_humedad.png}
\end{subfigure}
~
\begin{subfigure}[normla]{0.4\textwidth}
\caption{Estación Sta Cruz De Siecha código 21206980 caso 2 variable Radiación.}
\includegraphics[draft=false, scale=0.3]{comparacion_graficas_otras_var/201408_21206980_radiacion.png}
\end{subfigure}
~
\begin{subfigure}[normla]{0.4\textwidth}
\caption{Estación Sta Cruz De Siecha código 21206980 caso 2 variable Precipitación.}
\includegraphics[draft=false, scale=0.3]{comparacion_graficas_otras_var/201408_21206980_rain.png}
\end{subfigure}
~
\begin{subfigure}[normla]{0.4\textwidth}
\caption{Estación Sta Cruz De Siecha código 21206980 caso 2 variable Velocidad del Viento.}
\includegraphics[draft=false, scale=0.3]{comparacion_graficas_otras_var/201408_21206980_vel_viento.png}
\end{subfigure}
~
\begin{subfigure}[normla]{0.4\textwidth}
\caption{Estación Sta Cruz De Siecha código 21206980 caso 2 variable Bulbo húmedo.}
\includegraphics[draft=false, scale=0.3]{comparacion_graficas_otras_var/201408_21206980_wetbulb.png}
\end{subfigure}
~
\begin{subfigure}[normla]{0.4\textwidth}
\caption{Estación Sta Cruz De Siecha código 21206980 caso 3 variable Bulbo húmedo.}
\includegraphics[draft=false, scale=0.3]{comparacion_graficas_otras_var/201508_21206980_wetbulb.png}
\end{subfigure}
~
\end{figure}
           
\begin{figure}[H]
\centering
\begin{subfigure}[normla]{0.4\textwidth}
\caption{Estación Sta Cruz De Siecha código 21206980 caso 3 variable Punto de rocío.}
\includegraphics[draft=false, scale=0.3]{comparacion_graficas_otras_var/201508_21206980_dewpoint.png}
\end{subfigure}
~
\begin{subfigure}[normla]{0.4\textwidth}
\caption{Estación Sta Cruz De Siecha código 21206980 caso 3 variable Humedad.}
\includegraphics[draft=false, scale=0.3]{comparacion_graficas_otras_var/201508_21206980_humedad.png}
\end{subfigure}
~
\begin{subfigure}[normla]{0.4\textwidth}
\caption{Estación Sta Cruz De Siecha código 21206980 caso 3 variable Radiación.}
\includegraphics[draft=false, scale=0.3]{comparacion_graficas_otras_var/201508_21206980_radiacion.png}
\end{subfigure}
~
\begin{subfigure}[normla]{0.4\textwidth}
\caption{Estación Sta Cruz De Siecha código 21206980 caso 3 variable Precipitación.}
\includegraphics[draft=false, scale=0.3]{comparacion_graficas_otras_var/201508_21206980_rain.png}
\end{subfigure}
~
\begin{subfigure}[normla]{0.4\textwidth}
\caption{Estación Sta Cruz De Siecha código 21206980 caso 3 variable Velocidad del Viento.}
\includegraphics[draft=false, scale=0.3]{comparacion_graficas_otras_var/201508_21206980_vel_viento.png}
\end{subfigure}
~
\begin{subfigure}[normla]{0.4\textwidth}
\caption{Estación Sta Cruz De Siecha código 21206980 caso 4 variable Humedad.}
\includegraphics[draft=false, scale=0.3]{comparacion_graficas_otras_var/201509_21206980_humedad.png}
\end{subfigure}
~
\end{figure}
           
\begin{figure}[H]
\centering
\begin{subfigure}[normla]{0.4\textwidth}
\caption{Estación Sta Cruz De Siecha código 21206980 caso 4 variable Radiación.}
\includegraphics[draft=false, scale=0.3]{comparacion_graficas_otras_var/201509_21206980_radiacion.png}
\end{subfigure}
~
\begin{subfigure}[normla]{0.4\textwidth}
\caption{Estación Sta Cruz De Siecha código 21206980 caso 4 variable Precipitación.}
\includegraphics[draft=false, scale=0.3]{comparacion_graficas_otras_var/201509_21206980_rain.png}
\end{subfigure}
~
\begin{subfigure}[normla]{0.4\textwidth}
\caption{Estación Sta Cruz De Siecha código 21206980 caso 4 variable Velocidad del Viento.}
\includegraphics[draft=false, scale=0.3]{comparacion_graficas_otras_var/201509_21206980_vel_viento.png}
\end{subfigure}
~
\begin{subfigure}[normla]{0.4\textwidth}
\caption{Estación Sta Cruz De Siecha código 21206980 caso 4 variable Bulbo húmedo.}
\includegraphics[draft=false, scale=0.3]{comparacion_graficas_otras_var/201509_21206980_wetbulb.png}
\end{subfigure}
~
\begin{subfigure}[normla]{0.4\textwidth}
\caption{Estación Sta Cruz De Siecha código 21206980 caso 4 variable Punto de rocío.}
\includegraphics[draft=false, scale=0.3]{comparacion_graficas_otras_var/201509_21206980_dewpoint.png}
\end{subfigure}
~
\begin{subfigure}[normla]{0.4\textwidth}
\caption{Estación Tibaitata Automatica código 21206990 caso 1 variable Punto de rocío.}
\includegraphics[draft=false, scale=0.3]{comparacion_graficas_otras_var/200702_21206990_dewpoint.png}
\end{subfigure}
~
\end{figure}
           
\begin{figure}[H]
\centering
\begin{subfigure}[normla]{0.4\textwidth}
\caption{Estación Tibaitata Automatica código 21206990 caso 1 variable Humedad.}
\includegraphics[draft=false, scale=0.3]{comparacion_graficas_otras_var/200702_21206990_humedad.png}
\end{subfigure}
~
\begin{subfigure}[normla]{0.4\textwidth}
\caption{Estación Tibaitata Automatica código 21206990 caso 1 variable Radiación.}
\includegraphics[draft=false, scale=0.3]{comparacion_graficas_otras_var/200702_21206990_radiacion.png}
\end{subfigure}
~
\begin{subfigure}[normla]{0.4\textwidth}
\caption{Estación Tibaitata Automatica código 21206990 caso 1 variable Precipitación.}
\includegraphics[draft=false, scale=0.3]{comparacion_graficas_otras_var/200702_21206990_rain.png}
\end{subfigure}
~
\begin{subfigure}[normla]{0.4\textwidth}
\caption{Estación Tibaitata Automatica código 21206990 caso 1 variable Velocidad del Viento.}
\includegraphics[draft=false, scale=0.3]{comparacion_graficas_otras_var/200702_21206990_vel_viento.png}
\end{subfigure}
~
\begin{subfigure}[normla]{0.4\textwidth}
\caption{Estación Tibaitata Automatica código 21206990 caso 1 variable Bulbo húmedo.}
\includegraphics[draft=false, scale=0.3]{comparacion_graficas_otras_var/200702_21206990_wetbulb.png}
\end{subfigure}
~
\begin{subfigure}[normla]{0.4\textwidth}
\caption{Estación Tibaitata Automatica código 21206990 caso 2 variable Humedad.}
\includegraphics[draft=false, scale=0.3]{comparacion_graficas_otras_var/201408_21206990_humedad.png}
\end{subfigure}
~
\end{figure}
           
\begin{figure}[H]
\centering
\begin{subfigure}[normla]{0.4\textwidth}
\caption{Estación Tibaitata Automatica código 21206990 caso 2 variable Radiación.}
\includegraphics[draft=false, scale=0.3]{comparacion_graficas_otras_var/201408_21206990_radiacion.png}
\end{subfigure}
~
\begin{subfigure}[normla]{0.4\textwidth}
\caption{Estación Tibaitata Automatica código 21206990 caso 2 variable Precipitación.}
\includegraphics[draft=false, scale=0.3]{comparacion_graficas_otras_var/201408_21206990_rain.png}
\end{subfigure}
~
\begin{subfigure}[normla]{0.4\textwidth}
\caption{Estación Tibaitata Automatica código 21206990 caso 2 variable Velocidad del Viento.}
\includegraphics[draft=false, scale=0.3]{comparacion_graficas_otras_var/201408_21206990_vel_viento.png}
\end{subfigure}
~
\begin{subfigure}[normla]{0.4\textwidth}
\caption{Estación Tibaitata Automatica código 21206990 caso 2 variable Bulbo húmedo.}
\includegraphics[draft=false, scale=0.3]{comparacion_graficas_otras_var/201408_21206990_wetbulb.png}
\end{subfigure}
~
\begin{subfigure}[normla]{0.4\textwidth}
\caption{Estación Tibaitata Automatica código 21206990 caso 2 variable Punto de rocío.}
\includegraphics[draft=false, scale=0.3]{comparacion_graficas_otras_var/201408_21206990_dewpoint.png}
\end{subfigure}
~
\begin{subfigure}[normla]{0.4\textwidth}
\caption{Estación Tibaitata Automatica código 21206990 caso 3 variable Punto de rocío.}
\includegraphics[draft=false, scale=0.3]{comparacion_graficas_otras_var/201508_21206990_dewpoint.png}
\end{subfigure}
~
\end{figure}
           
\begin{figure}[H]
\centering
\begin{subfigure}[normla]{0.4\textwidth}
\caption{Estación Tibaitata Automatica código 21206990 caso 3 variable Humedad.}
\includegraphics[draft=false, scale=0.3]{comparacion_graficas_otras_var/201508_21206990_humedad.png}
\end{subfigure}
~
\begin{subfigure}[normla]{0.4\textwidth}
\caption{Estación Tibaitata Automatica código 21206990 caso 3 variable Radiación.}
\includegraphics[draft=false, scale=0.3]{comparacion_graficas_otras_var/201508_21206990_radiacion.png}
\end{subfigure}
~
\begin{subfigure}[normla]{0.4\textwidth}
\caption{Estación Tibaitata Automatica código 21206990 caso 3 variable Precipitación.}
\includegraphics[draft=false, scale=0.3]{comparacion_graficas_otras_var/201508_21206990_rain.png}
\end{subfigure}
~
\begin{subfigure}[normla]{0.4\textwidth}
\caption{Estación Tibaitata Automatica código 21206990 caso 3 variable Velocidad del Viento.}
\includegraphics[draft=false, scale=0.3]{comparacion_graficas_otras_var/201508_21206990_vel_viento.png}
\end{subfigure}
~
\begin{subfigure}[normla]{0.4\textwidth}
\caption{Estación Tibaitata Automatica código 21206990 caso 3 variable Bulbo húmedo.}
\includegraphics[draft=false, scale=0.3]{comparacion_graficas_otras_var/201508_21206990_wetbulb.png}
\end{subfigure}
~
\begin{subfigure}[normla]{0.4\textwidth}
\caption{Estación Tibaitata Automatica código 21206990 caso 4 variable Punto de rocío.}
\includegraphics[draft=false, scale=0.3]{comparacion_graficas_otras_var/201509_21206990_dewpoint.png}
\end{subfigure}
~
\end{figure}
           
\begin{figure}[H]
\centering
\begin{subfigure}[normla]{0.4\textwidth}
\caption{Estación Tibaitata Automatica código 21206990 caso 4 variable Humedad.}
\includegraphics[draft=false, scale=0.3]{comparacion_graficas_otras_var/201509_21206990_humedad.png}
\end{subfigure}
~
\begin{subfigure}[normla]{0.4\textwidth}
\caption{Estación Tibaitata Automatica código 21206990 caso 4 variable Radiación.}
\includegraphics[draft=false, scale=0.3]{comparacion_graficas_otras_var/201509_21206990_radiacion.png}
\end{subfigure}
~
\begin{subfigure}[normla]{0.4\textwidth}
\caption{Estación Tibaitata Automatica código 21206990 caso 4 variable Precipitación.}
\includegraphics[draft=false, scale=0.3]{comparacion_graficas_otras_var/201509_21206990_rain.png}
\end{subfigure}
~
\begin{subfigure}[normla]{0.4\textwidth}
\caption{Estación Tibaitata Automatica código 21206990 caso 4 variable Velocidad del Viento.}
\includegraphics[draft=false, scale=0.3]{comparacion_graficas_otras_var/201509_21206990_vel_viento.png}
\end{subfigure}
~
\begin{subfigure}[normla]{0.4\textwidth}
\caption{Estación Tibaitata Automatica código 21206990 caso 4 variable Bulbo húmedo.}
\includegraphics[draft=false, scale=0.3]{comparacion_graficas_otras_var/201509_21206990_wetbulb.png}
\end{subfigure}
~
\begin{subfigure}[normla]{0.4\textwidth}
\caption{Estación San Cayetano Autom  código 23125170 caso 2 variable Punto de rocío.}
\includegraphics[draft=false, scale=0.3]{comparacion_graficas_otras_var/201408_23125170_dewpoint.png}
\end{subfigure}
~
\end{figure}
           
\begin{figure}[H]
\centering
\begin{subfigure}[normla]{0.4\textwidth}
\caption{Estación San Cayetano Autom  código 23125170 caso 2 variable Humedad.}
\includegraphics[draft=false, scale=0.3]{comparacion_graficas_otras_var/201408_23125170_humedad.png}
\end{subfigure}
~
\begin{subfigure}[normla]{0.4\textwidth}
\caption{Estación San Cayetano Autom  código 23125170 caso 2 variable Radiación.}
\includegraphics[draft=false, scale=0.3]{comparacion_graficas_otras_var/201408_23125170_radiacion.png}
\end{subfigure}
~
\begin{subfigure}[normla]{0.4\textwidth}
\caption{Estación San Cayetano Autom  código 23125170 caso 2 variable Precipitación.}
\includegraphics[draft=false, scale=0.3]{comparacion_graficas_otras_var/201408_23125170_rain.png}
\end{subfigure}
~
\begin{subfigure}[normla]{0.4\textwidth}
\caption{Estación San Cayetano Autom  código 23125170 caso 2 variable Velocidad del Viento.}
\includegraphics[draft=false, scale=0.3]{comparacion_graficas_otras_var/201408_23125170_vel_viento.png}
\end{subfigure}
~
\begin{subfigure}[normla]{0.4\textwidth}
\caption{Estación San Cayetano Autom  código 23125170 caso 2 variable Bulbo húmedo.}
\includegraphics[draft=false, scale=0.3]{comparacion_graficas_otras_var/201408_23125170_wetbulb.png}
\end{subfigure}
~
\begin{subfigure}[normla]{0.4\textwidth}
\caption{Estación San Cayetano Autom  código 23125170 caso 3 variable Punto de rocío.}
\includegraphics[draft=false, scale=0.3]{comparacion_graficas_otras_var/201508_23125170_dewpoint.png}
\end{subfigure}
~
\end{figure}
           
\begin{figure}[H]
\centering
\begin{subfigure}[normla]{0.4\textwidth}
\caption{Estación San Cayetano Autom  código 23125170 caso 3 variable Humedad.}
\includegraphics[draft=false, scale=0.3]{comparacion_graficas_otras_var/201508_23125170_humedad.png}
\end{subfigure}
~
\begin{subfigure}[normla]{0.4\textwidth}
\caption{Estación San Cayetano Autom  código 23125170 caso 3 variable Radiación.}
\includegraphics[draft=false, scale=0.3]{comparacion_graficas_otras_var/201508_23125170_radiacion.png}
\end{subfigure}
~
\begin{subfigure}[normla]{0.4\textwidth}
\caption{Estación San Cayetano Autom  código 23125170 caso 3 variable Precipitación.}
\includegraphics[draft=false, scale=0.3]{comparacion_graficas_otras_var/201508_23125170_rain.png}
\end{subfigure}
~
\begin{subfigure}[normla]{0.4\textwidth}
\caption{Estación San Cayetano Autom  código 23125170 caso 3 variable Velocidad del Viento.}
\includegraphics[draft=false, scale=0.3]{comparacion_graficas_otras_var/201508_23125170_vel_viento.png}
\end{subfigure}
~
\begin{subfigure}[normla]{0.4\textwidth}
\caption{Estación San Cayetano Autom  código 23125170 caso 3 variable Bulbo húmedo.}
\includegraphics[draft=false, scale=0.3]{comparacion_graficas_otras_var/201508_23125170_wetbulb.png}
\end{subfigure}
~
\begin{subfigure}[normla]{0.4\textwidth}
\caption{Estación San Cayetano Autom  código 23125170 caso 4 variable Punto de rocío.}
\includegraphics[draft=false, scale=0.3]{comparacion_graficas_otras_var/201509_23125170_dewpoint.png}
\end{subfigure}
~
\end{figure}
           
\begin{figure}[H]
\centering
\begin{subfigure}[normla]{0.4\textwidth}
\caption{Estación San Cayetano Autom  código 23125170 caso 4 variable Humedad.}
\includegraphics[draft=false, scale=0.3]{comparacion_graficas_otras_var/201509_23125170_humedad.png}
\end{subfigure}
~
\begin{subfigure}[normla]{0.4\textwidth}
\caption{Estación San Cayetano Autom  código 23125170 caso 4 variable Radiación.}
\includegraphics[draft=false, scale=0.3]{comparacion_graficas_otras_var/201509_23125170_radiacion.png}
\end{subfigure}
~
\begin{subfigure}[normla]{0.4\textwidth}
\caption{Estación San Cayetano Autom  código 23125170 caso 4 variable Precipitación.}
\includegraphics[draft=false, scale=0.3]{comparacion_graficas_otras_var/201509_23125170_rain.png}
\end{subfigure}
~
\begin{subfigure}[normla]{0.4\textwidth}
\caption{Estación San Cayetano Autom  código 23125170 caso 4 variable Velocidad del Viento.}
\includegraphics[draft=false, scale=0.3]{comparacion_graficas_otras_var/201509_23125170_vel_viento.png}
\end{subfigure}
~
\begin{subfigure}[normla]{0.4\textwidth}
\caption{Estación San Cayetano Autom  código 23125170 caso 4 variable Bulbo húmedo.}
\includegraphics[draft=false, scale=0.3]{comparacion_graficas_otras_var/201509_23125170_wetbulb.png}
\end{subfigure}
~
\begin{subfigure}[normla]{0.4\textwidth}
\caption{Estación La Boyera Automatica código 24015110 caso 1 variable Humedad.}
\includegraphics[draft=false, scale=0.3]{comparacion_graficas_otras_var/200702_24015110_humedad.png}
\end{subfigure}
~
\end{figure}
           
\begin{figure}[H]
\centering
\begin{subfigure}[normla]{0.4\textwidth}
\caption{Estación La Boyera Automatica código 24015110 caso 1 variable Radiación.}
\includegraphics[draft=false, scale=0.3]{comparacion_graficas_otras_var/200702_24015110_radiacion.png}
\end{subfigure}
~
\begin{subfigure}[normla]{0.4\textwidth}
\caption{Estación La Boyera Automatica código 24015110 caso 1 variable Precipitación.}
\includegraphics[draft=false, scale=0.3]{comparacion_graficas_otras_var/200702_24015110_rain.png}
\end{subfigure}
~
\begin{subfigure}[normla]{0.4\textwidth}
\caption{Estación La Boyera Automatica código 24015110 caso 1 variable Velocidad del Viento.}
\includegraphics[draft=false, scale=0.3]{comparacion_graficas_otras_var/200702_24015110_vel_viento.png}
\end{subfigure}
~
\begin{subfigure}[normla]{0.4\textwidth}
\caption{Estación La Boyera Automatica código 24015110 caso 1 variable Bulbo húmedo.}
\includegraphics[draft=false, scale=0.3]{comparacion_graficas_otras_var/200702_24015110_wetbulb.png}
\end{subfigure}
~
\begin{subfigure}[normla]{0.4\textwidth}
\caption{Estación La Boyera Automatica código 24015110 caso 1 variable Punto de rocío.}
\includegraphics[draft=false, scale=0.3]{comparacion_graficas_otras_var/200702_24015110_dewpoint.png}
\end{subfigure}
~
\begin{subfigure}[normla]{0.4\textwidth}
\caption{Estación La Boyera Automatica código 24015110 caso 2 variable Velocidad del Viento.}
\includegraphics[draft=false, scale=0.3]{comparacion_graficas_otras_var/201408_24015110_vel_viento.png}
\end{subfigure}
~
\end{figure}
           
\begin{figure}[H]
\centering
\begin{subfigure}[normla]{0.4\textwidth}
\caption{Estación La Boyera Automatica código 24015110 caso 2 variable Punto de rocío.}
\includegraphics[draft=false, scale=0.3]{comparacion_graficas_otras_var/201408_24015110_dewpoint.png}
\end{subfigure}
~
\begin{subfigure}[normla]{0.4\textwidth}
\caption{Estación La Boyera Automatica código 24015110 caso 2 variable Humedad.}
\includegraphics[draft=false, scale=0.3]{comparacion_graficas_otras_var/201408_24015110_humedad.png}
\end{subfigure}
~
\begin{subfigure}[normla]{0.4\textwidth}
\caption{Estación La Boyera Automatica código 24015110 caso 2 variable Radiación.}
\includegraphics[draft=false, scale=0.3]{comparacion_graficas_otras_var/201408_24015110_radiacion.png}
\end{subfigure}
~
\begin{subfigure}[normla]{0.4\textwidth}
\caption{Estación La Boyera Automatica código 24015110 caso 2 variable Precipitación.}
\includegraphics[draft=false, scale=0.3]{comparacion_graficas_otras_var/201408_24015110_rain.png}
\end{subfigure}
~
\begin{subfigure}[normla]{0.4\textwidth}
\caption{Estación La Boyera Automatica código 24015110 caso 2 variable Bulbo húmedo.}
\includegraphics[draft=false, scale=0.3]{comparacion_graficas_otras_var/201408_24015110_wetbulb.png}
\end{subfigure}
~
\begin{subfigure}[normla]{0.4\textwidth}
\caption{Estación Pnn Chingaza Autom  código 35025080 caso 2 variable Punto de rocío.}
\includegraphics[draft=false, scale=0.3]{comparacion_graficas_otras_var/201408_35025080_dewpoint.png}
\end{subfigure}
~
\end{figure}
           
\begin{figure}[H]
\centering
\begin{subfigure}[normla]{0.4\textwidth}
\caption{Estación Pnn Chingaza Autom  código 35025080 caso 2 variable Humedad.}
\includegraphics[draft=false, scale=0.3]{comparacion_graficas_otras_var/201408_35025080_humedad.png}
\end{subfigure}
~
\begin{subfigure}[normla]{0.4\textwidth}
\caption{Estación Pnn Chingaza Autom  código 35025080 caso 2 variable Precipitación.}
\includegraphics[draft=false, scale=0.3]{comparacion_graficas_otras_var/201408_35025080_rain.png}
\end{subfigure}
~
\begin{subfigure}[normla]{0.4\textwidth}
\caption{Estación Pnn Chingaza Autom  código 35025080 caso 2 variable Velocidad del Viento.}
\includegraphics[draft=false, scale=0.3]{comparacion_graficas_otras_var/201408_35025080_vel_viento.png}
\end{subfigure}
~
\begin{subfigure}[normla]{0.4\textwidth}
\caption{Estación Pnn Chingaza Autom  código 35025080 caso 2 variable Bulbo húmedo.}
\includegraphics[draft=false, scale=0.3]{comparacion_graficas_otras_var/201408_35025080_wetbulb.png}
\end{subfigure}
~
\begin{subfigure}[normla]{0.4\textwidth}
\caption{Estación Bosque Intervenido   código 35025090 caso 2 variable Punto de rocío.}
\includegraphics[draft=false, scale=0.3]{comparacion_graficas_otras_var/201408_35025090_dewpoint.png}
\end{subfigure}
~
\begin{subfigure}[normla]{0.4\textwidth}
\caption{Estación Bosque Intervenido   código 35025090 caso 2 variable Humedad.}
\includegraphics[draft=false, scale=0.3]{comparacion_graficas_otras_var/201408_35025090_humedad.png}
\end{subfigure}
~
\end{figure}
           
\begin{figure}[H]
\centering
\begin{subfigure}[normla]{0.4\textwidth}
\caption{Estación Bosque Intervenido   código 35025090 caso 2 variable Precipitación.}
\includegraphics[draft=false, scale=0.3]{comparacion_graficas_otras_var/201408_35025090_rain.png}
\end{subfigure}
~
\begin{subfigure}[normla]{0.4\textwidth}
\caption{Estación Bosque Intervenido   código 35025090 caso 2 variable Velocidad del Viento.}
\includegraphics[draft=false, scale=0.3]{comparacion_graficas_otras_var/201408_35025090_vel_viento.png}
\end{subfigure}
~
\begin{subfigure}[normla]{0.4\textwidth}
\caption{Estación Bosque Intervenido   código 35025090 caso 2 variable Bulbo húmedo.}
\includegraphics[draft=false, scale=0.3]{comparacion_graficas_otras_var/201408_35025090_wetbulb.png}
\end{subfigure}
~
\begin{subfigure}[normla]{0.4\textwidth}
\caption{Estación Bosque Intervenido   código 35025090 caso 3 variable Precipitación.}
\includegraphics[draft=false, scale=0.3]{comparacion_graficas_otras_var/201508_35025090_rain.png}
\end{subfigure}
~
\begin{subfigure}[normla]{0.4\textwidth}
\caption{Estación Bosque Intervenido   código 35025090 caso 4 variable Precipitación.}
\includegraphics[draft=false, scale=0.3]{comparacion_graficas_otras_var/201509_35025090_rain.png}
\end{subfigure}
~
\begin{subfigure}[normla]{0.4\textwidth}
\caption{Estación Calostros Bajo   código 35027510 caso 2 variable Precipitación.}
\includegraphics[draft=false, scale=0.3]{comparacion_graficas_otras_var/201408_35027510_rain.png}
\end{subfigure}
~
\end{figure}
           
\begin{figure}[H]
\centering
\begin{subfigure}[normla]{0.4\textwidth}
\caption{Estación Calostros Bajo   código 35027510 caso 3 variable Precipitación.}
\includegraphics[draft=false, scale=0.3]{comparacion_graficas_otras_var/201508_35027510_rain.png}
\end{subfigure}
~
\begin{subfigure}[normla]{0.4\textwidth}
\caption{Estación Calostros Bajo   código 35027510 caso 4 variable Precipitación.}
\includegraphics[draft=false, scale=0.3]{comparacion_graficas_otras_var/201509_35027510_rain.png}
\end{subfigure}
~
\begin{subfigure}[normla]{0.4\textwidth}
\caption{Estación Pmo Chingaza código 35035130 caso 1 variable Precipitación.}
\includegraphics[draft=false, scale=0.3]{comparacion_graficas_otras_var/200702_35035130_rain.png}
\end{subfigure}
~
\begin{subfigure}[normla]{0.4\textwidth}
\caption{Estación Pmo Chingaza código 35035130 caso 2 variable Precipitación.}
\includegraphics[draft=false, scale=0.3]{comparacion_graficas_otras_var/201408_35035130_rain.png}
\end{subfigure}
~
\begin{subfigure}[normla]{0.4\textwidth}
\caption{Estación Pmo Chingaza código 35035130 caso 2 variable Velocidad del Viento.}
\includegraphics[draft=false, scale=0.3]{comparacion_graficas_otras_var/201408_35035130_vel_viento.png}
\end{subfigure}
~
\begin{subfigure}[normla]{0.4\textwidth}
\caption{Estación Pmo Chingaza código 35035130 caso 2 variable Bulbo húmedo.}
\includegraphics[draft=false, scale=0.3]{comparacion_graficas_otras_var/201408_35035130_wetbulb.png}
\end{subfigure}
~
\end{figure}
           
\begin{figure}[H]
\centering
\begin{subfigure}[normla]{0.4\textwidth}
\caption{Estación Pmo Chingaza código 35035130 caso 2 variable Punto de rocío.}
\includegraphics[draft=false, scale=0.3]{comparacion_graficas_otras_var/201408_35035130_dewpoint.png}
\end{subfigure}
~
\begin{subfigure}[normla]{0.4\textwidth}
\caption{Estación Pmo Chingaza código 35035130 caso 2 variable Humedad.}
\includegraphics[draft=false, scale=0.3]{comparacion_graficas_otras_var/201408_35035130_humedad.png}
\end{subfigure}
~
\begin{subfigure}[normla]{0.4\textwidth}
\caption{Estación Pmo Chingaza código 35035130 caso 3 variable Humedad.}
\includegraphics[draft=false, scale=0.3]{comparacion_graficas_otras_var/201508_35035130_humedad.png}
\end{subfigure}
~
\begin{subfigure}[normla]{0.4\textwidth}
\caption{Estación Pmo Chingaza código 35035130 caso 3 variable Precipitación.}
\includegraphics[draft=false, scale=0.3]{comparacion_graficas_otras_var/201508_35035130_rain.png}
\end{subfigure}
~
\begin{subfigure}[normla]{0.4\textwidth}
\caption{Estación Pmo Chingaza código 35035130 caso 3 variable Velocidad del Viento.}
\includegraphics[draft=false, scale=0.3]{comparacion_graficas_otras_var/201508_35035130_vel_viento.png}
\end{subfigure}
~
\begin{subfigure}[normla]{0.4\textwidth}
\caption{Estación Pmo Chingaza código 35035130 caso 3 variable Bulbo húmedo.}
\includegraphics[draft=false, scale=0.3]{comparacion_graficas_otras_var/201508_35035130_wetbulb.png}
\end{subfigure}
~
\end{figure}
           
\begin{figure}[H]
\centering
\begin{subfigure}[normla]{0.4\textwidth}
\caption{Estación Pmo Chingaza código 35035130 caso 3 variable Punto de rocío.}
\includegraphics[draft=false, scale=0.3]{comparacion_graficas_otras_var/201508_35035130_dewpoint.png}
\end{subfigure}
~
\begin{subfigure}[normla]{0.4\textwidth}
\caption{Estación Pmo Chingaza código 35035130 caso 4 variable Precipitación.}
\includegraphics[draft=false, scale=0.3]{comparacion_graficas_otras_var/201509_35035130_rain.png}
\end{subfigure}
~
\begin{subfigure}[normla]{0.4\textwidth}
\caption{Estación Pmo Chingaza código 35035130 caso 4 variable Punto de rocío.}
\includegraphics[draft=false, scale=0.3]{comparacion_graficas_otras_var/201509_35035130_dewpoint.png}
\end{subfigure}
~
\begin{subfigure}[normla]{0.4\textwidth}
\caption{Estación Pmo Chingaza código 35035130 caso 4 variable Humedad.}
\includegraphics[draft=false, scale=0.3]{comparacion_graficas_otras_var/201509_35035130_humedad.png}
\end{subfigure}
~
\begin{subfigure}[normla]{0.4\textwidth}
\caption{Estación Pmo Chingaza código 35035130 caso 4 variable Velocidad del Viento.}
\includegraphics[draft=false, scale=0.3]{comparacion_graficas_otras_var/201509_35035130_vel_viento.png}
\end{subfigure}
~
\begin{subfigure}[normla]{0.4\textwidth}
\caption{Estación Pmo Chingaza código 35035130 caso 4 variable Bulbo húmedo.}
\includegraphics[draft=false, scale=0.3]{comparacion_graficas_otras_var/201509_35035130_wetbulb.png}
\end{subfigure}
~
\end{figure}
           
\begin{figure}[H]
\centering
\begin{subfigure}[normla]{0.4\textwidth}
\caption{Estación Chinavita Automatica código 35075070 caso 1 variable Punto de rocío.}
\includegraphics[draft=false, scale=0.3]{comparacion_graficas_otras_var/200702_35075070_dewpoint.png}
\end{subfigure}
~
\begin{subfigure}[normla]{0.4\textwidth}
\caption{Estación Chinavita Automatica código 35075070 caso 1 variable Humedad.}
\includegraphics[draft=false, scale=0.3]{comparacion_graficas_otras_var/200702_35075070_humedad.png}
\end{subfigure}
~
\begin{subfigure}[normla]{0.4\textwidth}
\caption{Estación Chinavita Automatica código 35075070 caso 1 variable Radiación.}
\includegraphics[draft=false, scale=0.3]{comparacion_graficas_otras_var/200702_35075070_radiacion.png}
\end{subfigure}
~
\begin{subfigure}[normla]{0.4\textwidth}
\caption{Estación Chinavita Automatica código 35075070 caso 1 variable Precipitación.}
\includegraphics[draft=false, scale=0.3]{comparacion_graficas_otras_var/200702_35075070_rain.png}
\end{subfigure}
~
\begin{subfigure}[normla]{0.4\textwidth}
\caption{Estación Chinavita Automatica código 35075070 caso 1 variable Velocidad del Viento.}
\includegraphics[draft=false, scale=0.3]{comparacion_graficas_otras_var/200702_35075070_vel_viento.png}
\end{subfigure}
~
\begin{subfigure}[normla]{0.4\textwidth}
\caption{Estación Chinavita Automatica código 35075070 caso 1 variable Bulbo húmedo.}
\includegraphics[draft=false, scale=0.3]{comparacion_graficas_otras_var/200702_35075070_wetbulb.png}
\end{subfigure}
~
\end{figure}
           
\begin{figure}[H]
\centering
\begin{subfigure}[normla]{0.4\textwidth}
\caption{Estación Chinavita Automatica código 35075070 caso 2 variable Punto de rocío.}
\includegraphics[draft=false, scale=0.3]{comparacion_graficas_otras_var/201408_35075070_dewpoint.png}
\end{subfigure}
~
\begin{subfigure}[normla]{0.4\textwidth}
\caption{Estación Chinavita Automatica código 35075070 caso 2 variable Humedad.}
\includegraphics[draft=false, scale=0.3]{comparacion_graficas_otras_var/201408_35075070_humedad.png}
\end{subfigure}
~
\begin{subfigure}[normla]{0.4\textwidth}
\caption{Estación Chinavita Automatica código 35075070 caso 2 variable Radiación.}
\includegraphics[draft=false, scale=0.3]{comparacion_graficas_otras_var/201408_35075070_radiacion.png}
\end{subfigure}
~
\begin{subfigure}[normla]{0.4\textwidth}
\caption{Estación Chinavita Automatica código 35075070 caso 2 variable Precipitación.}
\includegraphics[draft=false, scale=0.3]{comparacion_graficas_otras_var/201408_35075070_rain.png}
\end{subfigure}
~
\begin{subfigure}[normla]{0.4\textwidth}
\caption{Estación Chinavita Automatica código 35075070 caso 2 variable Velocidad del Viento.}
\includegraphics[draft=false, scale=0.3]{comparacion_graficas_otras_var/201408_35075070_vel_viento.png}
\end{subfigure}
~
\begin{subfigure}[normla]{0.4\textwidth}
\caption{Estación Chinavita Automatica código 35075070 caso 2 variable Bulbo húmedo.}
\includegraphics[draft=false, scale=0.3]{comparacion_graficas_otras_var/201408_35075070_wetbulb.png}
\end{subfigure}
~
\end{figure}
           
\begin{figure}[H]
\centering
\begin{subfigure}[normla]{0.4\textwidth}
\caption{Estación Chinavita Automatica código 35075070 caso 3 variable Punto de rocío.}
\includegraphics[draft=false, scale=0.3]{comparacion_graficas_otras_var/201508_35075070_dewpoint.png}
\end{subfigure}
~
\begin{subfigure}[normla]{0.4\textwidth}
\caption{Estación Chinavita Automatica código 35075070 caso 3 variable Humedad.}
\includegraphics[draft=false, scale=0.3]{comparacion_graficas_otras_var/201508_35075070_humedad.png}
\end{subfigure}
~
\begin{subfigure}[normla]{0.4\textwidth}
\caption{Estación Chinavita Automatica código 35075070 caso 3 variable Radiación.}
\includegraphics[draft=false, scale=0.3]{comparacion_graficas_otras_var/201508_35075070_radiacion.png}
\end{subfigure}
~
\begin{subfigure}[normla]{0.4\textwidth}
\caption{Estación Chinavita Automatica código 35075070 caso 3 variable Precipitación.}
\includegraphics[draft=false, scale=0.3]{comparacion_graficas_otras_var/201508_35075070_rain.png}
\end{subfigure}
~
\begin{subfigure}[normla]{0.4\textwidth}
\caption{Estación Chinavita Automatica código 35075070 caso 3 variable Velocidad del Viento.}
\includegraphics[draft=false, scale=0.3]{comparacion_graficas_otras_var/201508_35075070_vel_viento.png}
\end{subfigure}
~
\begin{subfigure}[normla]{0.4\textwidth}
\caption{Estación Chinavita Automatica código 35075070 caso 3 variable Bulbo húmedo.}
\includegraphics[draft=false, scale=0.3]{comparacion_graficas_otras_var/201508_35075070_wetbulb.png}
\end{subfigure}
~
\end{figure}
           
\begin{figure}[H]
\centering
\begin{subfigure}[normla]{0.4\textwidth}
\caption{Estación Chinavita Automatica código 35075070 caso 4 variable Punto de rocío.}
\includegraphics[draft=false, scale=0.3]{comparacion_graficas_otras_var/201509_35075070_dewpoint.png}
\end{subfigure}
~
\begin{subfigure}[normla]{0.4\textwidth}
\caption{Estación Chinavita Automatica código 35075070 caso 4 variable Humedad.}
\includegraphics[draft=false, scale=0.3]{comparacion_graficas_otras_var/201509_35075070_humedad.png}
\end{subfigure}
~
\begin{subfigure}[normla]{0.4\textwidth}
\caption{Estación Chinavita Automatica código 35075070 caso 4 variable Radiación.}
\includegraphics[draft=false, scale=0.3]{comparacion_graficas_otras_var/201509_35075070_radiacion.png}
\end{subfigure}
~
\begin{subfigure}[normla]{0.4\textwidth}
\caption{Estación Chinavita Automatica código 35075070 caso 4 variable Precipitación.}
\includegraphics[draft=false, scale=0.3]{comparacion_graficas_otras_var/201509_35075070_rain.png}
\end{subfigure}
~
\begin{subfigure}[normla]{0.4\textwidth}
\caption{Estación Chinavita Automatica código 35075070 caso 4 variable Velocidad del Viento.}
\includegraphics[draft=false, scale=0.3]{comparacion_graficas_otras_var/201509_35075070_vel_viento.png}
\end{subfigure}
~
\begin{subfigure}[normla]{0.4\textwidth}
\caption{Estación Chinavita Automatica código 35075070 caso 4 variable Bulbo húmedo.}
\includegraphics[draft=false, scale=0.3]{comparacion_graficas_otras_var/201509_35075070_wetbulb.png}
\end{subfigure}
~
\end{figure}
           
\begin{figure}[H]
\centering
\begin{subfigure}[normla]{0.4\textwidth}
\caption{Estación Pmo Rabanal Automatica  código 35075080 caso 2 variable Punto de rocío.}
\includegraphics[draft=false, scale=0.3]{comparacion_graficas_otras_var/201408_35075080_dewpoint.png}
\end{subfigure}
~
\begin{subfigure}[normla]{0.4\textwidth}
\caption{Estación Pmo Rabanal Automatica  código 35075080 caso 2 variable Humedad.}
\includegraphics[draft=false, scale=0.3]{comparacion_graficas_otras_var/201408_35075080_humedad.png}
\end{subfigure}
~
\begin{subfigure}[normla]{0.4\textwidth}
\caption{Estación Pmo Rabanal Automatica  código 35075080 caso 2 variable Precipitación.}
\includegraphics[draft=false, scale=0.3]{comparacion_graficas_otras_var/201408_35075080_rain.png}
\end{subfigure}
~
\begin{subfigure}[normla]{0.4\textwidth}
\caption{Estación Pmo Rabanal Automatica  código 35075080 caso 2 variable Velocidad del Viento.}
\includegraphics[draft=false, scale=0.3]{comparacion_graficas_otras_var/201408_35075080_vel_viento.png}
\end{subfigure}
~
\begin{subfigure}[normla]{0.4\textwidth}
\caption{Estación Pmo Rabanal Automatica  código 35075080 caso 2 variable Bulbo húmedo.}
\includegraphics[draft=false, scale=0.3]{comparacion_graficas_otras_var/201408_35075080_wetbulb.png}
\end{subfigure}
~
\begin{subfigure}[normla]{0.4\textwidth}
\caption{Estación Pmo Rabanal Automatica  código 35075080 caso 3 variable Punto de rocío.}
\includegraphics[draft=false, scale=0.3]{comparacion_graficas_otras_var/201508_35075080_dewpoint.png}
\end{subfigure}
~
\end{figure}
           
\begin{figure}[H]
\centering
\begin{subfigure}[normla]{0.4\textwidth}
\caption{Estación Pmo Rabanal Automatica  código 35075080 caso 3 variable Humedad.}
\includegraphics[draft=false, scale=0.3]{comparacion_graficas_otras_var/201508_35075080_humedad.png}
\end{subfigure}
~
\begin{subfigure}[normla]{0.4\textwidth}
\caption{Estación Pmo Rabanal Automatica  código 35075080 caso 3 variable Precipitación.}
\includegraphics[draft=false, scale=0.3]{comparacion_graficas_otras_var/201508_35075080_rain.png}
\end{subfigure}
~
\begin{subfigure}[normla]{0.4\textwidth}
\caption{Estación Pmo Rabanal Automatica  código 35075080 caso 3 variable Velocidad del Viento.}
\includegraphics[draft=false, scale=0.3]{comparacion_graficas_otras_var/201508_35075080_vel_viento.png}
\end{subfigure}
~
\begin{subfigure}[normla]{0.4\textwidth}
\caption{Estación Pmo Rabanal Automatica  código 35075080 caso 3 variable Bulbo húmedo.}
\includegraphics[draft=false, scale=0.3]{comparacion_graficas_otras_var/201508_35075080_wetbulb.png}
\end{subfigure}
~
\begin{subfigure}[normla]{0.4\textwidth}
\caption{Estación Pmo Rabanal Automatica  código 35075080 caso 4 variable Precipitación.}
\includegraphics[draft=false, scale=0.3]{comparacion_graficas_otras_var/201509_35075080_rain.png}
\end{subfigure}
~
\begin{subfigure}[normla]{0.4\textwidth}
\caption{Estación La Capilla Autom código 35085080 caso 1 variable Precipitación.}
\includegraphics[draft=false, scale=0.3]{comparacion_graficas_otras_var/200702_35085080_rain.png}
\end{subfigure}
~
\end{figure}
           
\begin{figure}[H]
\centering
\begin{subfigure}[normla]{0.4\textwidth}
\caption{Estación La Capilla Autom código 35085080 caso 1 variable Punto de rocío.}
\includegraphics[draft=false, scale=0.3]{comparacion_graficas_otras_var/200702_35085080_dewpoint.png}
\end{subfigure}
~
\begin{subfigure}[normla]{0.4\textwidth}
\caption{Estación La Capilla Autom código 35085080 caso 1 variable Humedad.}
\includegraphics[draft=false, scale=0.3]{comparacion_graficas_otras_var/200702_35085080_humedad.png}
\end{subfigure}
~
\begin{subfigure}[normla]{0.4\textwidth}
\caption{Estación La Capilla Autom código 35085080 caso 1 variable Radiación.}
\includegraphics[draft=false, scale=0.3]{comparacion_graficas_otras_var/200702_35085080_radiacion.png}
\end{subfigure}
~
\begin{subfigure}[normla]{0.4\textwidth}
\caption{Estación La Capilla Autom código 35085080 caso 1 variable Velocidad del Viento.}
\includegraphics[draft=false, scale=0.3]{comparacion_graficas_otras_var/200702_35085080_vel_viento.png}
\end{subfigure}
~
\begin{subfigure}[normla]{0.4\textwidth}
\caption{Estación La Capilla Autom código 35085080 caso 1 variable Bulbo húmedo.}
\includegraphics[draft=false, scale=0.3]{comparacion_graficas_otras_var/200702_35085080_wetbulb.png}
\end{subfigure}
~
\begin{subfigure}[normla]{0.4\textwidth}
\caption{Estación La Capilla Autom código 35085080 caso 2 variable Humedad.}
\includegraphics[draft=false, scale=0.3]{comparacion_graficas_otras_var/201408_35085080_humedad.png}
\end{subfigure}
~
\end{figure}
           
\begin{figure}[H]
\centering
\begin{subfigure}[normla]{0.4\textwidth}
\caption{Estación La Capilla Autom código 35085080 caso 2 variable Punto de rocío.}
\includegraphics[draft=false, scale=0.3]{comparacion_graficas_otras_var/201408_35085080_dewpoint.png}
\end{subfigure}
~
\begin{subfigure}[normla]{0.4\textwidth}
\caption{Estación La Capilla Autom código 35085080 caso 2 variable Radiación.}
\includegraphics[draft=false, scale=0.3]{comparacion_graficas_otras_var/201408_35085080_radiacion.png}
\end{subfigure}
~
\begin{subfigure}[normla]{0.4\textwidth}
\caption{Estación La Capilla Autom código 35085080 caso 2 variable Precipitación.}
\includegraphics[draft=false, scale=0.3]{comparacion_graficas_otras_var/201408_35085080_rain.png}
\end{subfigure}
~
\begin{subfigure}[normla]{0.4\textwidth}
\caption{Estación La Capilla Autom código 35085080 caso 2 variable Velocidad del Viento.}
\includegraphics[draft=false, scale=0.3]{comparacion_graficas_otras_var/201408_35085080_vel_viento.png}
\end{subfigure}
~
\begin{subfigure}[normla]{0.4\textwidth}
\caption{Estación La Capilla Autom código 35085080 caso 2 variable Bulbo húmedo.}
\includegraphics[draft=false, scale=0.3]{comparacion_graficas_otras_var/201408_35085080_wetbulb.png}
\end{subfigure}
~
\begin{subfigure}[normla]{0.4\textwidth}
\caption{Estación La Capilla Autom código 35085080 caso 3 variable Radiación.}
\includegraphics[draft=false, scale=0.3]{comparacion_graficas_otras_var/201508_35085080_radiacion.png}
\end{subfigure}
~
\end{figure}
           
\begin{figure}[H]
\centering
\begin{subfigure}[normla]{0.4\textwidth}
\caption{Estación La Capilla Autom código 35085080 caso 3 variable Precipitación.}
\includegraphics[draft=false, scale=0.3]{comparacion_graficas_otras_var/201508_35085080_rain.png}
\end{subfigure}
~
\begin{subfigure}[normla]{0.4\textwidth}
\caption{Estación La Capilla Autom código 35085080 caso 4 variable Radiación.}
\includegraphics[draft=false, scale=0.3]{comparacion_graficas_otras_var/201509_35085080_radiacion.png}
\end{subfigure}
~
\begin{subfigure}[normla]{0.4\textwidth}
\caption{Estación La Capilla Autom código 35085080 caso 4 variable Precipitación.}
\includegraphics[draft=false, scale=0.3]{comparacion_graficas_otras_var/201509_35085080_rain.png}
\end{subfigure}
    
    \caption{Gráficas de la comparación de los resultados del WRF con respecto a las estaciones automáticas para los cuatro casos}
    \label{fig:my_label}
\end{figure}
 %%Quinto


%\chapter{Gráficas de los diagramas de Taylor de las diferentes combinaciones para el 2007.}
\label{anexograficas_taylor_200702}

\newpage
\begin{figure}[H]
    
\begin{subfigure}[normla]{0.5\textwidth}
\caption{Diagrama de Taylor para la Estación Subia Automatica código 21195160.}
\includegraphics[draft=false, scale=0.5]{../taylor_simulaciones/200702_m/taylor_21195160.png}
\end{subfigure}
~
\begin{subfigure}[normla]{0.5\textwidth}
\caption{Diagrama de Taylor para la Estación Hda Sta Ana Autom código 21206790.}
\includegraphics[draft=false, scale=0.5]{../taylor_simulaciones/200702_m/taylor_21206790.png}
\end{subfigure}
~
\begin{subfigure}[normla]{0.5\textwidth}
\caption{Diagrama de Taylor para la Estación Pmo Guerrero código 21206930.}
\includegraphics[draft=false, scale=0.5]{../taylor_simulaciones/200702_m/taylor_21206930.png}
\end{subfigure}
~
\begin{subfigure}[normla]{0.5\textwidth}
\caption{Diagrama de Taylor para la Estación Ciudad Bolivar código 21206940.}
\includegraphics[draft=false, scale=0.5]{../taylor_simulaciones/200702_m/taylor_21206940.png}
\end{subfigure}
~
\begin{subfigure}[normla]{0.5\textwidth}
\caption{Diagrama de Taylor para la Estación Pmo Guacheneque código 21206950.}
\includegraphics[draft=false, scale=0.5]{../taylor_simulaciones/200702_m/taylor_21206950.png}
\end{subfigure}
~
\begin{subfigure}[normla]{0.5\textwidth}
\caption{Diagrama de Taylor para la Estación Sta Cruz De Siecha código 21206980.}
\includegraphics[draft=false, scale=0.5]{../taylor_simulaciones/200702_m/taylor_21206980.png}
\end{subfigure}
~
\end{figure}
           
\begin{figure}[H]\ContinuedFloat
\begin{subfigure}[normla]{0.5\textwidth}
\caption{Diagrama de Taylor para la Estación Tibaitata Automatica código 21206990.}
\includegraphics[draft=false, scale=0.5]{../taylor_simulaciones/200702_m/taylor_21206990.png}
\end{subfigure}
~
\begin{subfigure}[normla]{0.5\textwidth}
\caption{Diagrama de Taylor para la Estación La Boyera Automatica código 24015110.}
\includegraphics[draft=false, scale=0.5]{../taylor_simulaciones/200702_m/taylor_24015110.png}
\end{subfigure}
~
\begin{subfigure}[normla]{0.5\textwidth}
\caption{Diagrama de Taylor para la Estación Chinavita Automatica código 35075070.}
\includegraphics[draft=false, scale=0.5]{../taylor_simulaciones/200702_m/taylor_35075070.png}
\end{subfigure}
~
\begin{subfigure}[normla]{0.5\textwidth}
\caption{Diagrama de Taylor para la Estación La Capilla Autom código 35085080.}
\includegraphics[draft=false, scale=0.5]{../taylor_simulaciones/200702_m/taylor_35085080.png}
\end{subfigure}
~

    
    \caption{Caso 1 }
    \label{fig:my_label}
\end{figure}


    

%%%%%%%%%%%%%%%%%%%%%%%%%%%%%%%%%%%%%%%%%%%%%%%%%%%%%%%%%
%%%%%%%%%%%%%%%%%%%%%%%%%%%%%%%%%%%%%%%%%%%%%%%%%%%%%%%%%
%%%%%%%%%%%%%%%%%%%%%%%%%%%%%%%%%%%%%%%%%%%%%%%%%%%%%%%%%
%%%%%%%%%%%%%%%%%%%%%%%%%%%%%%%%%%%%%%%%%%%%%%%%%%%%%%%%%\\



\begin{figure}[H]
\begin{subfigure}[normla]{0.5\textwidth}
\caption{Diagrama de Taylor para la Estación Esc La Union Automaticacódigo 21201200.}
\includegraphics[draft=false, scale=0.5]{../taylor_simulaciones/201408_m/taylor_21201200.png}
\end{subfigure}
~
\begin{subfigure}[normla]{0.5\textwidth}
\caption{Diagrama de Taylor para la Estación Univ Nacional código 21205012.}
\includegraphics[draft=false, scale=0.5]{../taylor_simulaciones/201408_m/taylor_21205012.png}
\end{subfigure}
~
\begin{subfigure}[normla]{0.5\textwidth}
\caption{Diagrama de Taylor para la Estación Apto El Dorado código 21205791.}
\includegraphics[draft=false, scale=0.5]{../taylor_simulaciones/201408_m/taylor_21205791.png}
\end{subfigure}
~
\begin{subfigure}[normla]{0.5\textwidth}
\caption{Diagrama de Taylor para la Estación Hda Sta Ana Autom código 21206790.}
\includegraphics[draft=false, scale=0.5]{../taylor_simulaciones/201408_m/taylor_21206790.png}
\end{subfigure}
~
\begin{subfigure}[normla]{0.5\textwidth}
\caption{Diagrama de Taylor para la Estación Pmo Guerrero código 21206930.}
\includegraphics[draft=false, scale=0.5]{../taylor_simulaciones/201408_m/taylor_21206930.png}
\end{subfigure}
~
\begin{subfigure}[normla]{0.5\textwidth}
\caption{Diagrama de Taylor para la Estación Pmo Guacheneque código 21206950.}
\includegraphics[draft=false, scale=0.5]{../taylor_simulaciones/201408_m/taylor_21206950.png}
\end{subfigure}
~
\end{figure}
           
\begin{figure}[H]\ContinuedFloat
\begin{subfigure}[normla]{0.5\textwidth}
\caption{Diagrama de Taylor para la Estación Ideam Bogota código 21206960.}
\includegraphics[draft=false, scale=0.5]{../taylor_simulaciones/201408_m/taylor_21206960.png}
\end{subfigure}
~
\begin{subfigure}[normla]{0.5\textwidth}
\caption{Diagrama de Taylor para la Estación Sta Cruz De Siecha código 21206980.}
\includegraphics[draft=false, scale=0.5]{../taylor_simulaciones/201408_m/taylor_21206980.png}
\end{subfigure}
~
\begin{subfigure}[normla]{0.5\textwidth}
\caption{Diagrama de Taylor para la Estación Tibaitata Automatica código 21206990.}
\includegraphics[draft=false, scale=0.5]{../taylor_simulaciones/201408_m/taylor_21206990.png}
\end{subfigure}
~
\begin{subfigure}[normla]{0.5\textwidth}
\caption{Diagrama de Taylor para la Estación San Cayetano Autom  código 23125170.}
\includegraphics[draft=false, scale=0.5]{../taylor_simulaciones/201408_m/taylor_23125170.png}
\end{subfigure}
~
\begin{subfigure}[normla]{0.5\textwidth}
\caption{Diagrama de Taylor para la Estación La Boyera Automatica código 24015110.}
\includegraphics[draft=false, scale=0.5]{../taylor_simulaciones/201408_m/taylor_24015110.png}
\end{subfigure}
~
\begin{subfigure}[normla]{0.5\textwidth}
\caption{Diagrama de Taylor para la Estación Pnn Chingaza Autom  código 35025080.}
\includegraphics[draft=false, scale=0.5]{../taylor_simulaciones/201408_m/taylor_35025080.png}
\end{subfigure}
~
\end{figure}
           
\begin{figure}[H]\ContinuedFloat
\begin{subfigure}[normla]{0.5\textwidth}
\caption{Diagrama de Taylor para la Estación Bosque Intervenido   código 35025090.}
\includegraphics[draft=false, scale=0.5]{../taylor_simulaciones/201408_m/taylor_35025090.png}
\end{subfigure}
~
\begin{subfigure}[normla]{0.5\textwidth}
\caption{Diagrama de Taylor para la Estación Chinavita Automatica código 35075070.}
\includegraphics[draft=false, scale=0.5]{../taylor_simulaciones/201408_m/taylor_35075070.png}
\end{subfigure}
~
\begin{subfigure}[normla]{0.5\textwidth}
\caption{Diagrama de Taylor para la Estación La Capilla Autom código 35085080.}
\includegraphics[draft=false, scale=0.5]{../taylor_simulaciones/201408_m/taylor_35085080.png}
\end{subfigure}


         
\caption{Diagramas de Taylor para el Caso 2}
\label{caso2}
\end{figure}
Para la figura de Sta Cruz de Siecha 21206980 presentó valores negativos de R2. Debido a que quedaron pocos valores de la estación automática luego de la corrección.





%
%
%



\begin{figure}
\begin{subfigure}[normla]{0.5\textwidth}
\caption{Diagrama de Taylor para la Estación Esc La Union Automaticacódigo 21201200.}
\includegraphics[draft=false, scale=0.5]{../taylor_simulaciones/201508_m/taylor_21201200.png}
\end{subfigure}
~
\begin{subfigure}[normla]{0.5\textwidth}
\caption{Diagrama de Taylor para la Estación Univ Nacional código 21205012.}
\includegraphics[draft=false, scale=0.5]{../taylor_simulaciones/201508_m/taylor_21205012.png}
\end{subfigure}
~
\begin{subfigure}[normla]{0.5\textwidth}
\caption{Diagrama de Taylor para la Estación Apto El Dorado código 21205791.}
\includegraphics[draft=false, scale=0.5]{../taylor_simulaciones/201508_m/taylor_21205791.png}
\end{subfigure}
~
\begin{subfigure}[normla]{0.5\textwidth}
\caption{Diagrama de Taylor para la Estación Nueva Generacion código 21206600.}
\includegraphics[draft=false, scale=0.5]{../taylor_simulaciones/201508_m/taylor_21206600.png}
\end{subfigure}
~
\begin{subfigure}[normla]{0.5\textwidth}
\caption{Diagrama de Taylor para la Estación Hda Sta Ana Autom código 21206790.}
\includegraphics[draft=false, scale=0.5]{../taylor_simulaciones/201508_m/taylor_21206790.png}
\end{subfigure}
~
\begin{subfigure}[normla]{0.5\textwidth}
\caption{Diagrama de Taylor para la Estación Villa Teresa Automatica código 21206920.}
\includegraphics[draft=false, scale=0.5]{../taylor_simulaciones/201508_m/taylor_21206920.png}
\end{subfigure}
~
\end{figure}
           
\begin{figure}[H]\ContinuedFloat
\begin{subfigure}[normla]{0.5\textwidth}
\caption{Diagrama de Taylor para la Estación Pmo Guerrero código 21206930.}
\includegraphics[draft=false, scale=0.5]{../taylor_simulaciones/201508_m/taylor_21206930.png}
\end{subfigure}
~
\begin{subfigure}[normla]{0.5\textwidth}
\caption{Diagrama de Taylor para la Estación Ideam Bogota código 21206960.}
\includegraphics[draft=false, scale=0.5]{../taylor_simulaciones/201508_m/taylor_21206960.png}
\end{subfigure}
~
\begin{subfigure}[normla]{0.5\textwidth}
\caption{Diagrama de Taylor para la Estación Sta Cruz De Siecha código 21206980.}
\includegraphics[draft=false, scale=0.5]{../taylor_simulaciones/201508_m/taylor_21206980.png}
\end{subfigure}
~
\begin{subfigure}[normla]{0.5\textwidth}
\caption{Diagrama de Taylor para la Estación Tibaitata Automatica código 21206990.}
\includegraphics[draft=false, scale=0.5]{../taylor_simulaciones/201508_m/taylor_21206990.png}
\end{subfigure}
~
\begin{subfigure}[normla]{0.5\textwidth}
\caption{Diagrama de Taylor para la Estación San Cayetano Autom  código 23125170.}
\includegraphics[draft=false, scale=0.5]{../taylor_simulaciones/201508_m/taylor_23125170.png}
\end{subfigure}
~
\begin{subfigure}[normla]{0.5\textwidth}
\caption{Diagrama de Taylor para la Estación Bosque Intervenido   código 35025090.}
\includegraphics[draft=false, scale=0.5]{../taylor_simulaciones/201508_m/taylor_35025090.png}
\end{subfigure}
~
\end{figure}
           
\begin{figure}[H]\ContinuedFloat
\begin{subfigure}[normla]{0.5\textwidth}
\caption{Diagrama de Taylor para la Estación Pmo Chingaza código 35035130.}
\includegraphics[draft=false, scale=0.5]{../taylor_simulaciones/201508_m/taylor_35035130.png}
\end{subfigure}
~
\begin{subfigure}[normla]{0.5\textwidth}
\caption{Diagrama de Taylor para la Estación Chinavita Automatica código 35075070.}
\includegraphics[draft=false, scale=0.5]{../taylor_simulaciones/201508_m/taylor_35075070.png}
\end{subfigure}
~
\begin{subfigure}[normla]{0.5\textwidth}
\caption{Diagrama de Taylor para la Estación Pmo Rabanal Automatica  código 35075080.}
\includegraphics[draft=false, scale=0.5]{../taylor_simulaciones/201508_m/taylor_35075080.png}
\end{subfigure}
~
\begin{subfigure}[normla]{0.5\textwidth}
\caption{Diagrama de Taylor para la Estación La Capilla Autom código 35085080.}
\includegraphics[draft=false, scale=0.5]{../taylor_simulaciones/201508_m/taylor_35085080.png}
\end{subfigure}
~


\caption{Diagramas de Taylor para el Caso 3}
\label{caso3}
\end{figure}

%%%
En el diagrama de la estación Esc La Unión 21201200, se observa que la desviación estándar de los datos de la estación no son altos. Esto sugiere que los datos de la estación no son buenos. Los datos de la estación presentan poca variación \\

Los datos modelados de la estación el Dorado 21205791 presentaron valores altos de desviación estándar y los datos de la estación meteorológica automática presentó valores bajos. Los datos de la estación tienen una desviación estándar más baja

La estación Univ Nacional 21205012 no se puede observar ninguna razón para sospechar.\\

La estación Villa Teresa 21206920 y Chingaza 35035130 en el caso 3 se puede observar que presenta una baja desviación estándar. Observando la gráfica donde se compara la estación se puede observa que tiene problemas. La estación Villa Teresa Y Chingaza no fue graficada.



\begin{figure}[H]
\begin{subfigure}[normla]{0.5\textwidth}
\caption{Diagrama de Taylor para la Estación Esc La Unión 21201200.}
\includegraphics[draft=false, scale=0.5]{../comparacion_grafica/201508_21201200.png}
\end{subfigure}
~
\begin{subfigure}[normla]{0.5\textwidth}
\caption{Diagrama de Taylor para la estación el Dorado 21205791.}
\includegraphics[draft=false, scale=0.5]{../comparacion_grafica/201508_21205791.png}
\end{subfigure}
~
\begin{subfigure}[normla]{0.5\textwidth}
\caption{Diagrama de Taylor para la Estación Univ Nacional 21205012.}
\includegraphics[draft=false, scale=0.5]{../comparacion_grafica/201508_21205012.png}
\end{subfigure}
~
\begin{subfigure}[normla]{0.5\textwidth}
\caption{Diagrama de Taylor para la estación Villa Teresa 21206920.}
\includegraphics[draft=false, scale=0.5]{../comparacion_grafica/201508_21206920.png}
\end{subfigure}
\end{figure}

\begin{figure}[H]
\begin{subfigure}[normla]{0.5\textwidth}
\caption{Diagrama de Taylor para la Estación Chingaza 35035130.}
\includegraphics[draft=false, scale=0.5]{../comparacion_grafica/201508_35035130.png}
\end{subfigure}


\end{figure}

%%%
\newpage

\begin{figure}[H]


\begin{subfigure}[normla]{0.5\textwidth}
\caption{Diagrama de Taylor para la Estación Univ Nacional código 21205012.}
\includegraphics[draft=false, scale=0.5]{../taylor_simulaciones/201509_m/taylor_21205012.png}
\end{subfigure}
~
\begin{subfigure}[normla]{0.5\textwidth}
\caption{Diagrama de Taylor para la Estación Apto El Dorado código 21205791.}
\includegraphics[draft=false, scale=0.5]{../taylor_simulaciones/201509_m/taylor_21205791.png}
\end{subfigure}
~
\begin{subfigure}[normla]{0.5\textwidth}
\caption{Diagrama de Taylor para la Estación Nueva Generacion código 21206600.}
\includegraphics[draft=false, scale=0.5]{../taylor_simulaciones/201509_m/taylor_21206600.png}
\end{subfigure}
~
\begin{subfigure}[normla]{0.5\textwidth}
\caption{Diagrama de Taylor para la Estación Hda Sta Ana Autom código 21206790.}
\includegraphics[draft=false, scale=0.5]{../taylor_simulaciones/201509_m/taylor_21206790.png}
\end{subfigure}
~
\begin{subfigure}[normla]{0.5\textwidth}
\caption{Diagrama de Taylor para la Estación Villa Teresa Automatica código 21206920.}
\includegraphics[draft=false, scale=0.5]{../taylor_simulaciones/201509_m/taylor_21206920.png}
\end{subfigure}
~
\begin{subfigure}[normla]{0.5\textwidth}
\caption{Diagrama de Taylor para la Estación Pmo Guerrero código 21206930.}
\includegraphics[draft=false, scale=0.5]{../taylor_simulaciones/201509_m/taylor_21206930.png}
\end{subfigure}
~
\end{figure}
           
\begin{figure}[H]\ContinuedFloat
\begin{subfigure}[normla]{0.5\textwidth}
\caption{Diagrama de Taylor para la Estación Ideam Bogota código 21206960.}
\includegraphics[draft=false, scale=0.5]{../taylor_simulaciones/201509_m/taylor_21206960.png}
\end{subfigure}
~
\begin{subfigure}[normla]{0.5\textwidth}
\caption{Diagrama de Taylor para la Estación Sta Cruz De Siecha código 21206980.}
\includegraphics[draft=false, scale=0.5]{../taylor_simulaciones/201509_m/taylor_21206980.png}
\end{subfigure}
~
\begin{subfigure}[normla]{0.5\textwidth}
\caption{Diagrama de Taylor para la Estación Tibaitata Automatica código 21206990.}
\includegraphics[draft=false, scale=0.5]{../taylor_simulaciones/201509_m/taylor_21206990.png}
\end{subfigure}
~
\begin{subfigure}[normla]{0.5\textwidth}
\caption{Diagrama de Taylor para la Estación San Cayetano Autom  código 23125170.}
\includegraphics[draft=false, scale=0.5]{../taylor_simulaciones/201509_m/taylor_23125170.png}
\end{subfigure}
~
\begin{subfigure}[normla]{0.5\textwidth}
\caption{Diagrama de Taylor para la Estación Bosque Intervenido   código 35025090.}
\includegraphics[draft=false, scale=0.5]{../taylor_simulaciones/201509_m/taylor_35025090.png}
\end{subfigure}
~
\begin{subfigure}[normla]{0.5\textwidth}
\caption{Diagrama de Taylor para la Estación Pmo Chingaza código 35035130.}
\includegraphics[draft=false, scale=0.5]{../taylor_simulaciones/201509_m/taylor_35035130.png}
\end{subfigure}
~
\end{figure}
           
\begin{figure}[H]\ContinuedFloat
\begin{subfigure}[normla]{0.5\textwidth}
\caption{Diagrama de Taylor para la Estación Chinavita Automatica código 35075070.}
\includegraphics[draft=false, scale=0.5]{../taylor_simulaciones/201509_m/taylor_35075070.png}
\end{subfigure}
~
\begin{subfigure}[normla]{0.5\textwidth}
\caption{Diagrama de Taylor para la Estación Pmo Rabanal Automatica  código 35075080.}
\includegraphics[draft=false, scale=0.5]{../taylor_simulaciones/201509_m/taylor_35075080.png}
\end{subfigure}
~
\begin{subfigure}[normla]{0.5\textwidth}
\caption{Diagrama de Taylor para la Estación La Capilla Autom código 35085080.}
\includegraphics[draft=false, scale=0.5]{../taylor_simulaciones/201509_m/taylor_35085080.png}
\end{subfigure}



\caption{Diagramas de Taylor para el Caso 4}
\label{caso4}
\end{figure}

Para la estación Villa Teresa 21206920 la desviación estándar de la estación automática es mucho menor que los valores modelados.\\

Para la estación Sta Cruz de Siecha 21206980 creo que no está bien que sea descartada.\\

Para la estación Chingaza 35035130 se observa que los datos de la estación tiene una baja desviación estándar.\\



\begin{figure}[H]
\begin{subfigure}[normla]{0.5\textwidth}
\caption{Diagrama de Taylor para la Estación Villa Teresa 21206920.}
\includegraphics[draft=false, scale=0.5]{../comparacion_grafica/201509_21206920.png}
\end{subfigure}
~
\begin{subfigure}[normla]{0.5\textwidth}
\caption{Diagrama de Taylor para la estación Sta Cruz de Siecha 21206980.}
\includegraphics[draft=false, scale=0.5]{../comparacion_grafica/201509_21206980.png}
\end{subfigure}
~
\begin{subfigure}[normla]{0.5\textwidth}
\caption{Diagrama de Taylor para la Estación Chingaza 35035130.}
\includegraphics[draft=false, scale=0.5]{../comparacion_grafica/201509_35035130.png}
\end{subfigure}

\end{figure}




Profe estas son las gráficas de todas las simulaciones la verdad yo sí quería usarlas todas. Segú mi criterio las estaciones que se deben sacar son:

Para el primer caso: ninguna\\
Para el segundo caso: 21206980 Sta Cruz de Siecha\\
Para el tercer caso: 21201200 Esc La Unión,  21205791 El Dorado, 21205012 Univ Nacional, 21206920 Villa Teresa, 35035130 Pmo Chingaza.\\
Para el cuarto caso: 21206920 Villa teresa, 2120698 Sta Cruz de Siecha, 35035130 Chingaza.

%Profe si sacamos esta estaciones la tabla da de esta forma.
%
%
%\begin{table}[H]
%    
%\begin{tabular}{lr}
%\toprule
%
%Combinación simulación-opción               &   Frecuencia     \\
%\midrule
%bl\_pbl\_physics-0     &      2 \\
%bl\_pbl\_physics-5     &     10 \\
%bl\_pbl\_physics-6     &      8 \\
%bl\_pbl\_physics-7     &     14 \\
%bl\_pbl\_physics-8     &     12 \\
%bl\_pbl\_physics-9     &      2 \\
%bl\_pbl\_physics-12    &     11 \\
%bl\_pbl\_physics-99    &      7 \\
%cu\_physics-0         &      6 \\
%cu\_physics-2         &      2 \\
%cu\_physics-5         &      6 \\
%cu\_physics-6         &      5 \\
%cu\_physics-11        &      4 \\
%cu\_physics-14        &      6 \\
%cu\_physics-16        &      4 \\
%cu\_physics-93        &      1 \\
%cu\_physics-99        &      4 \\
%ideam-colombia       &      6 \\
%mp\_physics-0         &     10 \\
%mp\_physics-1         &      1 \\
%mp\_physics-2         &      2 \\
%mp\_physics-3         &      2 \\
%mp\_physics-8         &      1 \\
%mp\_physics-11        &      5 \\
%mp\_physics-13        &      4 \\
%mp\_physics-32        &      8 \\
%ra\_lw\_physics-3      &      4 \\
%ra\_lw\_physics-5      &      4 \\
%ra\_lw\_physics-7      &     10 \\
%ra\_lw\_physics-31     &      3 \\
%ra\_sw\_physics-2      &     17 \\
%ra\_sw\_physics-3      &     17 \\
%ra\_sw\_physics-4      &     15 \\
%ra\_sw\_physics-5      &     14 \\
%sf\_sfclay\_physics-91 &      8 \\
%\bottomrule
%\end{tabular}
%
%    
%\end{table}
%
%Esta tabla es la tabla resumen de la frecuencia de los resultados sin tener en cuenta las estaciones anteriormente mencionadas. Estos resultados son más parecidos a los que ya se habían obtenido.\\
%
%Dado que sea así queda de la siguiente forma:\\
%
%bl\_pbl = 14 (Estaba)\\
%cu\_physics = 0, 5, 14 (Estaban todas)\\
%ra\_lw = 7 (Estaba)\\
%ra\_sw = 2 y 3 (Nuevo empate)\\


Me tocaría hacer unas nuevas corridas pero no hay lío con ello.
\begin{figure}
    \centering
    \includegraphics[draft=false, scale=0.3]{graph/lebel_3.png}
    \caption{Etiquetas para cada una de las gráficas}
    \label{fig:my_label}
\end{figure} % Gráficas de todas las estaciones
%\begin{figure}[H]

\begin{subfigure}[normla]{0.4\textwidth}
\caption{Estación Sta Cruz De Siecha  código 21206980 nivel 0.}
\includegraphics[draft=false, scale=0.3]{../taylor_simulaciones/201408_5_m/taylor_0_21206980.png}
\end{subfigure}

\caption{Estaciones del caso 2.}
\label{estaciones_caso1}
\end{figure}

\newpage


%%%%%%%%%%
%%%%%%%%%%
%%%%%%%%%%
%%%%%%%%%%\\
%%%%%%%%%%
%%%%%%%%%%
%%%%%%%%%%
%%%%%%%%%%
%%%%%%%%%%
%%%%%%%%%%\\
%%%%%%%%%%
%%%%%%%%%%
%%%%%%%%%%
%%%%%%%%%%
%%%%%%%%%%


\begin{figure}[h]

\begin{subfigure}[normla]{0.4\textwidth}
\caption{Estación Univ Nacional  código 21205012 nivel 0.}
\includegraphics[draft=false, scale=0.3]{../taylor_simulaciones/201508_5_m/taylor_0_21205012.png}
\end{subfigure}
~
\begin{subfigure}[normla]{0.4\textwidth}
\caption{Estación Univ Nacional  código 21205012 nivel 1.}
\includegraphics[draft=false, scale=0.3]{../taylor_simulaciones/201508_5_m/taylor_1_21205012.png}
\end{subfigure}
~
\begin{subfigure}[normla]{0.4\textwidth}
\caption{Estación Univ Nacional  código 21205012 nivel 2.}
\includegraphics[draft=false, scale=0.3]{../taylor_simulaciones/201508_5_m/taylor_2_21205012.png}
\end{subfigure}
~
\begin{subfigure}[normla]{0.4\textwidth}
\caption{Estación Univ Nacional  código 21205012 nivel 3.}
\includegraphics[draft=false, scale=0.3]{../taylor_simulaciones/201508_5_m/taylor_3_21205012.png}
\end{subfigure}
~
\begin{subfigure}[normla]{0.4\textwidth}
\caption{Estación Univ Nacional  código 21205012 nivel 4.}
\includegraphics[draft=false, scale=0.3]{../taylor_simulaciones/201508_5_m/taylor_4_21205012.png}
\end{subfigure}
~
\begin{subfigure}[normla]{0.4\textwidth}
\caption{Estación Univ Nacional  código 21205012 nivel 5.}
\includegraphics[draft=false, scale=0.3]{../taylor_simulaciones/201508_5_m/taylor_5_21205012.png}
\end{subfigure}
~
\end{figure}
           
\begin{figure}[H]\ContinuedFloat
\centering
\begin{subfigure}[normla]{0.4\textwidth}
\caption{Estación Univ Nacional  código 21205012 nivel 6.}
\includegraphics[draft=false, scale=0.3]{../taylor_simulaciones/201508_5_m/taylor_6_21205012.png}
\end{subfigure}
~
\begin{subfigure}[normla]{0.4\textwidth}
\caption{Estación Univ Nacional  código 21205012 nivel 7.}
\includegraphics[draft=false, scale=0.3]{../taylor_simulaciones/201508_5_m/taylor_7_21205012.png}
\end{subfigure}
~
\begin{subfigure}[normla]{0.4\textwidth}
\caption{Estación Univ Nacional  código 21205012 nivel 8.}
\includegraphics[draft=false, scale=0.3]{../taylor_simulaciones/201508_5_m/taylor_8_21205012.png}
\end{subfigure}
~
\begin{subfigure}[normla]{0.4\textwidth}
\caption{Estación Univ Nacional  código 21205012 nivel 9.}
\includegraphics[draft=false, scale=0.3]{../taylor_simulaciones/201508_5_m/taylor_9_21205012.png}
\end{subfigure}
~
\begin{subfigure}[normla]{0.4\textwidth}
\caption{Estación Univ Nacional  código 21205012 nivel 10.}
\includegraphics[draft=false, scale=0.3]{../taylor_simulaciones/201508_5_m/taylor_10_21205012.png}
\end{subfigure}
~
\begin{subfigure}[normla]{0.4\textwidth}
\caption{Estación Univ Nacional  código 21205012 nivel 11.}
\includegraphics[draft=false, scale=0.3]{../taylor_simulaciones/201508_5_m/taylor_11_21205012.png}
\end{subfigure}
~
\end{figure}
           
\begin{figure}[H]\ContinuedFloat
\centering
\begin{subfigure}[normla]{0.4\textwidth}
\caption{Estación Apto El Dorado  código 21205791 nivel 0.}
\includegraphics[draft=false, scale=0.3]{../taylor_simulaciones/201508_5_m/taylor_0_21205791.png}
\end{subfigure}
~
\begin{subfigure}[normla]{0.4\textwidth}
\caption{Estación Apto El Dorado  código 21205791 nivel 1.}
\includegraphics[draft=false, scale=0.3]{../taylor_simulaciones/201508_5_m/taylor_1_21205791.png}
\end{subfigure}
~
\begin{subfigure}[normla]{0.4\textwidth}
\caption{Estación Apto El Dorado  código 21205791 nivel 2.}
\includegraphics[draft=false, scale=0.3]{../taylor_simulaciones/201508_5_m/taylor_2_21205791.png}
\end{subfigure}
~
\begin{subfigure}[normla]{0.4\textwidth}
\caption{Estación Apto El Dorado  código 21205791 nivel 3.}
\includegraphics[draft=false, scale=0.3]{../taylor_simulaciones/201508_5_m/taylor_3_21205791.png}
\end{subfigure}
~
\begin{subfigure}[normla]{0.4\textwidth}
\caption{Estación Apto El Dorado  código 21205791 nivel 4.}
\includegraphics[draft=false, scale=0.3]{../taylor_simulaciones/201508_5_m/taylor_4_21205791.png}
\end{subfigure}
~
\begin{subfigure}[normla]{0.4\textwidth}
\caption{Estación Apto El Dorado  código 21205791 nivel 5.}
\includegraphics[draft=false, scale=0.3]{../taylor_simulaciones/201508_5_m/taylor_5_21205791.png}
\end{subfigure}
~
\end{figure}
           
\begin{figure}[H]\ContinuedFloat
\centering
\begin{subfigure}[normla]{0.4\textwidth}
\caption{Estación Apto El Dorado  código 21205791 nivel 6.}
\includegraphics[draft=false, scale=0.3]{../taylor_simulaciones/201508_5_m/taylor_6_21205791.png}
\end{subfigure}
~
\begin{subfigure}[normla]{0.4\textwidth}
\caption{Estación Apto El Dorado  código 21205791 nivel 7.}
\includegraphics[draft=false, scale=0.3]{../taylor_simulaciones/201508_5_m/taylor_7_21205791.png}
\end{subfigure}
~
\begin{subfigure}[normla]{0.4\textwidth}
\caption{Estación Apto El Dorado  código 21205791 nivel 8.}
\includegraphics[draft=false, scale=0.3]{../taylor_simulaciones/201508_5_m/taylor_8_21205791.png}
\end{subfigure}
~
\begin{subfigure}[normla]{0.4\textwidth}
\caption{Estación Apto El Dorado  código 21205791 nivel 9.}
\includegraphics[draft=false, scale=0.3]{../taylor_simulaciones/201508_5_m/taylor_9_21205791.png}
\end{subfigure}
~
\begin{subfigure}[normla]{0.4\textwidth}
\caption{Estación Apto El Dorado  código 21205791 nivel 10.}
\includegraphics[draft=false, scale=0.3]{../taylor_simulaciones/201508_5_m/taylor_10_21205791.png}
\end{subfigure}
~
\begin{subfigure}[normla]{0.4\textwidth}
\caption{Estación Apto El Dorado  código 21205791 nivel 11.}
\includegraphics[draft=false, scale=0.3]{../taylor_simulaciones/201508_5_m/taylor_11_21205791.png}
\end{subfigure}
~
\end{figure}
           
\begin{figure}[H]
\centering
\begin{subfigure}[normla]{0.4\textwidth}
\caption{Estación Villa Teresa Automatica  código 21206920 nivel 0.}
\includegraphics[draft=false, scale=0.3]{../taylor_simulaciones/201508_5_m/taylor_0_21206920.png}
\end{subfigure}
~
\begin{subfigure}[normla]{0.4\textwidth}
\caption{Estación Villa Teresa Automatica  código 21206920 nivel 1.}
\includegraphics[draft=false, scale=0.3]{../taylor_simulaciones/201508_5_m/taylor_1_21206920.png}
\end{subfigure}
~
\begin{subfigure}[normla]{0.4\textwidth}
\caption{Estación Villa Teresa Automatica  código 21206920 nivel 2.}
\includegraphics[draft=false, scale=0.3]{../taylor_simulaciones/201508_5_m/taylor_2_21206920.png}
\end{subfigure}
~
\begin{subfigure}[normla]{0.4\textwidth}
\caption{Estación Villa Teresa Automatica  código 21206920 nivel 3.}
\includegraphics[draft=false, scale=0.3]{../taylor_simulaciones/201508_5_m/taylor_3_21206920.png}
\end{subfigure}
~
\begin{subfigure}[normla]{0.4\textwidth}
\caption{Estación Villa Teresa Automatica  código 21206920 nivel 4.}
\includegraphics[draft=false, scale=0.3]{../taylor_simulaciones/201508_5_m/taylor_4_21206920.png}
\end{subfigure}
~
\begin{subfigure}[normla]{0.4\textwidth}
\caption{Estación Villa Teresa Automatica  código 21206920 nivel 5.}
\includegraphics[draft=false, scale=0.3]{../taylor_simulaciones/201508_5_m/taylor_5_21206920.png}
\end{subfigure}
~
\end{figure}
           
\begin{figure}[H]\ContinuedFloat
\centering
\begin{subfigure}[normla]{0.4\textwidth}
\caption{Estación Villa Teresa Automatica  código 21206920 nivel 6.}
\includegraphics[draft=false, scale=0.3]{../taylor_simulaciones/201508_5_m/taylor_6_21206920.png}
\end{subfigure}
~
\begin{subfigure}[normla]{0.4\textwidth}
\caption{Estación Villa Teresa Automatica  código 21206920 nivel 7.}
\includegraphics[draft=false, scale=0.3]{../taylor_simulaciones/201508_5_m/taylor_7_21206920.png}
\end{subfigure}
~
\begin{subfigure}[normla]{0.4\textwidth}
\caption{Estación Villa Teresa Automatica  código 21206920 nivel 8.}
\includegraphics[draft=false, scale=0.3]{../taylor_simulaciones/201508_5_m/taylor_8_21206920.png}
\end{subfigure}
~
\begin{subfigure}[normla]{0.4\textwidth}
\caption{Estación Villa Teresa Automatica  código 21206920 nivel 9.}
\includegraphics[draft=false, scale=0.3]{../taylor_simulaciones/201508_5_m/taylor_9_21206920.png}
\end{subfigure}
~
\begin{subfigure}[normla]{0.4\textwidth}
\caption{Estación Villa Teresa Automatica  código 21206920 nivel 10.}
\includegraphics[draft=false, scale=0.3]{../taylor_simulaciones/201508_5_m/taylor_10_21206920.png}
\end{subfigure}
~
\begin{subfigure}[normla]{0.4\textwidth}
\caption{Estación Villa Teresa Automatica  código 21206920 nivel 11.}
\includegraphics[draft=false, scale=0.3]{../taylor_simulaciones/201508_5_m/taylor_11_21206920.png}
\end{subfigure}
~
\end{figure}
           
\begin{figure}[H]\ContinuedFloat
\centering
\begin{subfigure}[normla]{0.4\textwidth}
\caption{Estación Pmo Chingaza  código 35035130 nivel 0.}
\includegraphics[draft=false, scale=0.3]{../taylor_simulaciones/201508_5_m/taylor_0_35035130.png}
\end{subfigure}
~
\begin{subfigure}[normla]{0.4\textwidth}
\caption{Estación Pmo Chingaza  código 35035130 nivel 1.}
\includegraphics[draft=false, scale=0.3]{../taylor_simulaciones/201508_5_m/taylor_1_35035130.png}
\end{subfigure}
~
\begin{subfigure}[normla]{0.4\textwidth}
\caption{Estación Pmo Chingaza  código 35035130 nivel 2.}
\includegraphics[draft=false, scale=0.3]{../taylor_simulaciones/201508_5_m/taylor_2_35035130.png}
\end{subfigure}
~
\begin{subfigure}[normla]{0.4\textwidth}
\caption{Estación Pmo Chingaza  código 35035130 nivel 3.}
\includegraphics[draft=false, scale=0.3]{../taylor_simulaciones/201508_5_m/taylor_3_35035130.png}
\end{subfigure}
~
\begin{subfigure}[normla]{0.4\textwidth}
\caption{Estación Pmo Chingaza  código 35035130 nivel 4.}
\includegraphics[draft=false, scale=0.3]{../taylor_simulaciones/201508_5_m/taylor_4_35035130.png}
\end{subfigure}
~
\begin{subfigure}[normla]{0.4\textwidth}
\caption{Estación Pmo Chingaza  código 35035130 nivel 5.}
\includegraphics[draft=false, scale=0.3]{../taylor_simulaciones/201508_5_m/taylor_5_35035130.png}
\end{subfigure}
~
\end{figure}
           
\begin{figure}[H]\ContinuedFloat
\centering
\begin{subfigure}[normla]{0.4\textwidth}
\caption{Estación Pmo Chingaza  código 35035130 nivel 6.}
\includegraphics[draft=false, scale=0.3]{../taylor_simulaciones/201508_5_m/taylor_6_35035130.png}
\end{subfigure}
~
\begin{subfigure}[normla]{0.4\textwidth}
\caption{Estación Pmo Chingaza  código 35035130 nivel 7.}
\includegraphics[draft=false, scale=0.3]{../taylor_simulaciones/201508_5_m/taylor_7_35035130.png}
\end{subfigure}
~
\begin{subfigure}[normla]{0.4\textwidth}
\caption{Estación Pmo Chingaza  código 35035130 nivel 8.}
\includegraphics[draft=false, scale=0.3]{../taylor_simulaciones/201508_5_m/taylor_8_35035130.png}
\end{subfigure}
~
\begin{subfigure}[normla]{0.4\textwidth}
\caption{Estación Pmo Chingaza  código 35035130 nivel 9.}
\includegraphics[draft=false, scale=0.3]{../taylor_simulaciones/201508_5_m/taylor_9_35035130.png}
\end{subfigure}
~
\begin{subfigure}[normla]{0.4\textwidth}
\caption{Estación Pmo Chingaza  código 35035130 nivel 10.}
\includegraphics[draft=false, scale=0.3]{../taylor_simulaciones/201508_5_m/taylor_10_35035130.png}
\end{subfigure}
~
\begin{subfigure}[normla]{0.4\textwidth}
\caption{Estación Pmo Chingaza  código 35035130 nivel 11.}
\includegraphics[draft=false, scale=0.3]{../taylor_simulaciones/201508_5_m/taylor_11_35035130.png}
\end{subfigure}

\caption{Estaciones del caso 3.}
\label{estaciones_caso1}
\end{figure}


%##############################################
%##############################################
%##############################################


\begin{figure}

\begin{subfigure}[normla]{0.4\textwidth}
\caption{Estación Villa Teresa Automatica  código 21206920 nivel 0.}
\includegraphics[draft=false, scale=0.3]{../taylor_simulaciones/201509_5_m/taylor_0_21206920.png}
\end{subfigure}
~
\begin{subfigure}[normla]{0.4\textwidth}
\caption{Estación Villa Teresa Automatica  código 21206920 nivel 1.}
\includegraphics[draft=false, scale=0.3]{../taylor_simulaciones/201509_5_m/taylor_1_21206920.png}
\end{subfigure}
~
\begin{subfigure}[normla]{0.4\textwidth}
\caption{Estación Villa Teresa Automatica  código 21206920 nivel 2.}
\includegraphics[draft=false, scale=0.3]{../taylor_simulaciones/201509_5_m/taylor_2_21206920.png}
\end{subfigure}
~
\begin{subfigure}[normla]{0.4\textwidth}
\caption{Estación Villa Teresa Automatica  código 21206920 nivel 3.}
\includegraphics[draft=false, scale=0.3]{../taylor_simulaciones/201509_5_m/taylor_3_21206920.png}
\end{subfigure}
~
\begin{subfigure}[normla]{0.4\textwidth}
\caption{Estación Villa Teresa Automatica  código 21206920 nivel 4.}
\includegraphics[draft=false, scale=0.3]{../taylor_simulaciones/201509_5_m/taylor_4_21206920.png}
\end{subfigure}
~
\begin{subfigure}[normla]{0.4\textwidth}
\caption{Estación Villa Teresa Automatica  código 21206920 nivel 5.}
\includegraphics[draft=false, scale=0.3]{../taylor_simulaciones/201509_5_m/taylor_5_21206920.png}
\end{subfigure}
~
\end{figure}
           
\begin{figure}[H]\ContinuedFloat
\centering
\begin{subfigure}[normla]{0.4\textwidth}
\caption{Estación Villa Teresa Automatica  código 21206920 nivel 6.}
\includegraphics[draft=false, scale=0.3]{../taylor_simulaciones/201509_5_m/taylor_6_21206920.png}
\end{subfigure}
~
\begin{subfigure}[normla]{0.4\textwidth}
\caption{Estación Villa Teresa Automatica  código 21206920 nivel 7.}
\includegraphics[draft=false, scale=0.3]{../taylor_simulaciones/201509_5_m/taylor_7_21206920.png}
\end{subfigure}
~
\begin{subfigure}[normla]{0.4\textwidth}
\caption{Estación Villa Teresa Automatica  código 21206920 nivel 8.}
\includegraphics[draft=false, scale=0.3]{../taylor_simulaciones/201509_5_m/taylor_8_21206920.png}
\end{subfigure}
~
\begin{subfigure}[normla]{0.4\textwidth}
\caption{Estación Villa Teresa Automatica  código 21206920 nivel 9.}
\includegraphics[draft=false, scale=0.3]{../taylor_simulaciones/201509_5_m/taylor_9_21206920.png}
\end{subfigure}
~
\begin{subfigure}[normla]{0.4\textwidth}
\caption{Estación Villa Teresa Automatica  código 21206920 nivel 10.}
\includegraphics[draft=false, scale=0.3]{../taylor_simulaciones/201509_5_m/taylor_10_21206920.png}
\end{subfigure}
~
\begin{subfigure}[normla]{0.4\textwidth}
\caption{Estación Villa Teresa Automatica  código 21206920 nivel 11.}
\includegraphics[draft=false, scale=0.3]{../taylor_simulaciones/201509_5_m/taylor_11_21206920.png}
\end{subfigure}
~
\end{figure}
           
\begin{figure}[H]\ContinuedFloat
\centering
\begin{subfigure}[normla]{0.4\textwidth}
\caption{Estación Villa Teresa Automatica  código 21206920 nivel 12.}
\includegraphics[draft=false, scale=0.3]{../taylor_simulaciones/201509_5_m/taylor_12_21206920.png}
\end{subfigure}
~
\begin{subfigure}[normla]{0.4\textwidth}
\caption{Estación Villa Teresa Automatica  código 21206920 nivel 13.}
\includegraphics[draft=false, scale=0.3]{../taylor_simulaciones/201509_5_m/taylor_13_21206920.png}
\end{subfigure}
~
\begin{subfigure}[normla]{0.4\textwidth}
\caption{Estación Villa Teresa Automatica  código 21206920 nivel 14.}
\includegraphics[draft=false, scale=0.3]{../taylor_simulaciones/201509_5_m/taylor_14_21206920.png}
\end{subfigure}
~
\begin{subfigure}[normla]{0.4\textwidth}
\caption{Estación Villa Teresa Automatica  código 21206920 nivel 15.}
\includegraphics[draft=false, scale=0.3]{../taylor_simulaciones/201509_5_m/taylor_15_21206920.png}
\end{subfigure}
~
\begin{subfigure}[normla]{0.4\textwidth}
\caption{Estación Villa Teresa Automatica  código 21206920 nivel 16.}
\includegraphics[draft=false, scale=0.3]{../taylor_simulaciones/201509_5_m/taylor_16_21206920.png}
\end{subfigure}
~
\begin{subfigure}[normla]{0.4\textwidth}
\caption{Estación Villa Teresa Automatica  código 21206920 nivel 17.}
\includegraphics[draft=false, scale=0.3]{../taylor_simulaciones/201509_5_m/taylor_17_21206920.png}
\end{subfigure}
~
\end{figure}
           
\begin{figure}[H]\ContinuedFloat
\centering
\begin{subfigure}[normla]{0.4\textwidth}
\caption{Estación Villa Teresa Automatica  código 21206920 nivel 18.}
\includegraphics[draft=false, scale=0.3]{../taylor_simulaciones/201509_5_m/taylor_18_21206920.png}
\end{subfigure}
~
\begin{subfigure}[normla]{0.4\textwidth}
\caption{Estación Villa Teresa Automatica  código 21206920 nivel 19.}
\includegraphics[draft=false, scale=0.3]{../taylor_simulaciones/201509_5_m/taylor_19_21206920.png}
\end{subfigure}
~
\begin{subfigure}[normla]{0.4\textwidth}
\caption{Estación Villa Teresa Automatica  código 21206920 nivel 20.}
\includegraphics[draft=false, scale=0.3]{../taylor_simulaciones/201509_5_m/taylor_20_21206920.png}
\end{subfigure}
~
\begin{subfigure}[normla]{0.4\textwidth}
\caption{Estación Sta Cruz De Siecha  código 21206980 nivel 0.}
\includegraphics[draft=false, scale=0.3]{../taylor_simulaciones/201509_5_m/taylor_0_21206980.png}
\end{subfigure}
~
\begin{subfigure}[normla]{0.4\textwidth}
\caption{Estación Sta Cruz De Siecha  código 21206980 nivel 1.}
\includegraphics[draft=false, scale=0.3]{../taylor_simulaciones/201509_5_m/taylor_1_21206980.png}
\end{subfigure}
~
\begin{subfigure}[normla]{0.4\textwidth}
\caption{Estación Sta Cruz De Siecha  código 21206980 nivel 2.}
\includegraphics[draft=false, scale=0.3]{../taylor_simulaciones/201509_5_m/taylor_2_21206980.png}
\end{subfigure}
~
\end{figure}
           
\begin{figure}[H]
\centering
\begin{subfigure}[normla]{0.4\textwidth}
\caption{Estación Sta Cruz De Siecha  código 21206980 nivel 3.}
\includegraphics[draft=false, scale=0.3]{../taylor_simulaciones/201509_5_m/taylor_3_21206980.png}
\end{subfigure}
~
\begin{subfigure}[normla]{0.4\textwidth}
\caption{Estación Sta Cruz De Siecha  código 21206980 nivel 4.}
\includegraphics[draft=false, scale=0.3]{../taylor_simulaciones/201509_5_m/taylor_4_21206980.png}
\end{subfigure}
~
\begin{subfigure}[normla]{0.4\textwidth}
\caption{Estación Sta Cruz De Siecha  código 21206980 nivel 5.}
\includegraphics[draft=false, scale=0.3]{../taylor_simulaciones/201509_5_m/taylor_5_21206980.png}
\end{subfigure}
~
\begin{subfigure}[normla]{0.4\textwidth}
\caption{Estación Sta Cruz De Siecha  código 21206980 nivel 6.}
\includegraphics[draft=false, scale=0.3]{../taylor_simulaciones/201509_5_m/taylor_6_21206980.png}
\end{subfigure}
~
\begin{subfigure}[normla]{0.4\textwidth}
\caption{Estación Sta Cruz De Siecha  código 21206980 nivel 7.}
\includegraphics[draft=false, scale=0.3]{../taylor_simulaciones/201509_5_m/taylor_7_21206980.png}
\end{subfigure}
~
\begin{subfigure}[normla]{0.4\textwidth}
\caption{Estación Sta Cruz De Siecha  código 21206980 nivel 8.}
\includegraphics[draft=false, scale=0.3]{../taylor_simulaciones/201509_5_m/taylor_8_21206980.png}
\end{subfigure}
~
\end{figure}
           
\begin{figure}[H]\ContinuedFloat
\centering
\begin{subfigure}[normla]{0.4\textwidth}
\caption{Estación Sta Cruz De Siecha  código 21206980 nivel 9.}
\includegraphics[draft=false, scale=0.3]{../taylor_simulaciones/201509_5_m/taylor_9_21206980.png}
\end{subfigure}
~
\begin{subfigure}[normla]{0.4\textwidth}
\caption{Estación Sta Cruz De Siecha  código 21206980 nivel 10.}
\includegraphics[draft=false, scale=0.3]{../taylor_simulaciones/201509_5_m/taylor_10_21206980.png}
\end{subfigure}
~
\begin{subfigure}[normla]{0.4\textwidth}
\caption{Estación Sta Cruz De Siecha  código 21206980 nivel 11.}
\includegraphics[draft=false, scale=0.3]{../taylor_simulaciones/201509_5_m/taylor_11_21206980.png}
\end{subfigure}
~
\begin{subfigure}[normla]{0.4\textwidth}
\caption{Estación Sta Cruz De Siecha  código 21206980 nivel 12.}
\includegraphics[draft=false, scale=0.3]{../taylor_simulaciones/201509_5_m/taylor_12_21206980.png}
\end{subfigure}
~
\begin{subfigure}[normla]{0.4\textwidth}
\caption{Estación Sta Cruz De Siecha  código 21206980 nivel 13.}
\includegraphics[draft=false, scale=0.3]{../taylor_simulaciones/201509_5_m/taylor_13_21206980.png}
\end{subfigure}
~
\begin{subfigure}[normla]{0.4\textwidth}
\caption{Estación Sta Cruz De Siecha  código 21206980 nivel 14.}
\includegraphics[draft=false, scale=0.3]{../taylor_simulaciones/201509_5_m/taylor_14_21206980.png}
\end{subfigure}
~
\end{figure}
           
\begin{figure}[H]
\centering
\begin{subfigure}[normla]{0.4\textwidth}
\caption{Estación Sta Cruz De Siecha  código 21206980 nivel 15.}
\includegraphics[draft=false, scale=0.3]{../taylor_simulaciones/201509_5_m/taylor_15_21206980.png}
\end{subfigure}
~
\begin{subfigure}[normla]{0.4\textwidth}
\caption{Estación Sta Cruz De Siecha  código 21206980 nivel 16.}
\includegraphics[draft=false, scale=0.3]{../taylor_simulaciones/201509_5_m/taylor_16_21206980.png}
\end{subfigure}
~
\begin{subfigure}[normla]{0.4\textwidth}
\caption{Estación Sta Cruz De Siecha  código 21206980 nivel 17.}
\includegraphics[draft=false, scale=0.3]{../taylor_simulaciones/201509_5_m/taylor_17_21206980.png}
\end{subfigure}
~
\begin{subfigure}[normla]{0.4\textwidth}
\caption{Estación Sta Cruz De Siecha  código 21206980 nivel 18.}
\includegraphics[draft=false, scale=0.3]{../taylor_simulaciones/201509_5_m/taylor_18_21206980.png}
\end{subfigure}
~
\begin{subfigure}[normla]{0.4\textwidth}
\caption{Estación Sta Cruz De Siecha  código 21206980 nivel 19.}
\includegraphics[draft=false, scale=0.3]{../taylor_simulaciones/201509_5_m/taylor_19_21206980.png}
\end{subfigure}
~
\begin{subfigure}[normla]{0.4\textwidth}
\caption{Estación Sta Cruz De Siecha  código 21206980 nivel 20.}
\includegraphics[draft=false, scale=0.3]{../taylor_simulaciones/201509_5_m/taylor_20_21206980.png}
\end{subfigure}
~
\end{figure}
           
\begin{figure}[H]
\centering
\begin{subfigure}[normla]{0.4\textwidth}
\caption{Estación Pmo Chingaza  código 35035130 nivel 0.}
\includegraphics[draft=false, scale=0.3]{../taylor_simulaciones/201509_5_m/taylor_0_35035130.png}
\end{subfigure}
~
\begin{subfigure}[normla]{0.4\textwidth}
\caption{Estación Pmo Chingaza  código 35035130 nivel 1.}
\includegraphics[draft=false, scale=0.3]{../taylor_simulaciones/201509_5_m/taylor_1_35035130.png}
\end{subfigure}
~
\begin{subfigure}[normla]{0.4\textwidth}
\caption{Estación Pmo Chingaza  código 35035130 nivel 2.}
\includegraphics[draft=false, scale=0.3]{../taylor_simulaciones/201509_5_m/taylor_2_35035130.png}
\end{subfigure}
~
\begin{subfigure}[normla]{0.4\textwidth}
\caption{Estación Pmo Chingaza  código 35035130 nivel 3.}
\includegraphics[draft=false, scale=0.3]{../taylor_simulaciones/201509_5_m/taylor_3_35035130.png}
\end{subfigure}
~
\begin{subfigure}[normla]{0.4\textwidth}
\caption{Estación Pmo Chingaza  código 35035130 nivel 4.}
\includegraphics[draft=false, scale=0.3]{../taylor_simulaciones/201509_5_m/taylor_4_35035130.png}
\end{subfigure}
~
\begin{subfigure}[normla]{0.4\textwidth}
\caption{Estación Pmo Chingaza  código 35035130 nivel 5.}
\includegraphics[draft=false, scale=0.3]{../taylor_simulaciones/201509_5_m/taylor_5_35035130.png}
\end{subfigure}
~
\end{figure}
           
\begin{figure}[H]\ContinuedFloat
\centering
\begin{subfigure}[normla]{0.4\textwidth}
\caption{Estación Pmo Chingaza  código 35035130 nivel 6.}
\includegraphics[draft=false, scale=0.3]{../taylor_simulaciones/201509_5_m/taylor_6_35035130.png}
\end{subfigure}
~
\begin{subfigure}[normla]{0.4\textwidth}
\caption{Estación Pmo Chingaza  código 35035130 nivel 7.}
\includegraphics[draft=false, scale=0.3]{../taylor_simulaciones/201509_5_m/taylor_7_35035130.png}
\end{subfigure}
~
\begin{subfigure}[normla]{0.4\textwidth}
\caption{Estación Pmo Chingaza  código 35035130 nivel 8.}
\includegraphics[draft=false, scale=0.3]{../taylor_simulaciones/201509_5_m/taylor_8_35035130.png}
\end{subfigure}
~
\begin{subfigure}[normla]{0.4\textwidth}
\caption{Estación Pmo Chingaza  código 35035130 nivel 9.}
\includegraphics[draft=false, scale=0.3]{../taylor_simulaciones/201509_5_m/taylor_9_35035130.png}
\end{subfigure}
~
\begin{subfigure}[normla]{0.4\textwidth}
\caption{Estación Pmo Chingaza  código 35035130 nivel 10.}
\includegraphics[draft=false, scale=0.3]{../taylor_simulaciones/201509_5_m/taylor_10_35035130.png}
\end{subfigure}
~
\begin{subfigure}[normla]{0.4\textwidth}
\caption{Estación Pmo Chingaza  código 35035130 nivel 11.}
\includegraphics[draft=false, scale=0.3]{../taylor_simulaciones/201509_5_m/taylor_11_35035130.png}
\end{subfigure}
~
\end{figure}
           
\begin{figure}[H]
\centering
\begin{subfigure}[normla]{0.4\textwidth}
\caption{Estación Pmo Chingaza  código 35035130 nivel 12.}
\includegraphics[draft=false, scale=0.3]{../taylor_simulaciones/201509_5_m/taylor_12_35035130.png}
\end{subfigure}
~
\begin{subfigure}[normla]{0.4\textwidth}
\caption{Estación Pmo Chingaza  código 35035130 nivel 13.}
\includegraphics[draft=false, scale=0.3]{../taylor_simulaciones/201509_5_m/taylor_13_35035130.png}
\end{subfigure}
~
\begin{subfigure}[normla]{0.4\textwidth}
\caption{Estación Pmo Chingaza  código 35035130 nivel 14.}
\includegraphics[draft=false, scale=0.3]{../taylor_simulaciones/201509_5_m/taylor_14_35035130.png}
\end{subfigure}
~
\begin{subfigure}[normla]{0.4\textwidth}
\caption{Estación Pmo Chingaza  código 35035130 nivel 15.}
\includegraphics[draft=false, scale=0.3]{../taylor_simulaciones/201509_5_m/taylor_15_35035130.png}
\end{subfigure}
~
\begin{subfigure}[normla]{0.4\textwidth}
\caption{Estación Pmo Chingaza  código 35035130 nivel 16.}
\includegraphics[draft=false, scale=0.3]{../taylor_simulaciones/201509_5_m/taylor_16_35035130.png}
\end{subfigure}
~
\begin{subfigure}[normla]{0.4\textwidth}
\caption{Estación Pmo Chingaza  código 35035130 nivel 17.}
\includegraphics[draft=false, scale=0.3]{../taylor_simulaciones/201509_5_m/taylor_17_35035130.png}
\end{subfigure}
~
\end{figure}
           
\begin{figure}[H]\ContinuedFloat
\centering
\begin{subfigure}[normla]{0.4\textwidth}
\caption{Estación Pmo Chingaza  código 35035130 nivel 18.}
\includegraphics[draft=false, scale=0.3]{../taylor_simulaciones/201509_5_m/taylor_18_35035130.png}
\end{subfigure}
~
\begin{subfigure}[normla]{0.4\textwidth}
\caption{Estación Pmo Chingaza  código 35035130 nivel 19.}
\includegraphics[draft=false, scale=0.3]{../taylor_simulaciones/201509_5_m/taylor_19_35035130.png}
\end{subfigure}
~
\begin{subfigure}[normla]{0.4\textwidth}
\caption{Estación Pmo Chingaza  código 35035130 nivel 20.}
\includegraphics[draft=false, scale=0.3]{../taylor_simulaciones/201509_5_m/taylor_20_35035130.png}
\end{subfigure}	

\caption{Estaciones del caso 4.}
\label{estaciones_caso1}
\end{figure} % gráficas hechas para la profe Astrid
%\include{anexo_casos_total} % Graficas de todos los diagramas de Taylor 
%% Anexo de las gráficas de las diferentes temperaturas
\chapter{Gráficas de los diagramas de Taylor de las evaluaciones de los diferentes tiempos para los 4 casos.}
\label{anexo:graficas_taylor_tiempos_4casos}
\newpage
\begin{figure}[H]
 
 
 
\begin{subfigure}[normla]{0.4\textwidth}
\includegraphics[draft=false, scale=0.050]{../tablas_estaciones_dominios/graficas_200702/21195160.png}
\caption{Estación Subia Atutomática código 21195160 caso 1.}
\end{subfigure}
~
\begin{subfigure}[normla]{0.4\textwidth}
\includegraphics[draft=false, scale=0.050]{../tablas_estaciones_dominios/graficas_200702/21206790.png}
\caption{Estación Hda Sta Ana Autom código 21206790 caso 1.}
\end{subfigure}
~
\begin{subfigure}[normla]{0.4\textwidth}
\includegraphics[draft=false, scale=0.050]{../tablas_estaciones_dominios/graficas_200702/21206930.png}
\caption{Estación Pmo Guerrero código 21206930 caso 1.}
\end{subfigure}
~
\begin{subfigure}[normla]{0.4\textwidth}
\includegraphics[draft=false, scale=0.050]{../tablas_estaciones_dominios/graficas_200702/21206940.png}
\caption{Estación Ciudad Bolivar código 21206940 caso 1.}
\end{subfigure}
~
\begin{subfigure}[normla]{0.4\textwidth}
\includegraphics[draft=false, scale=0.050]{../tablas_estaciones_dominios/graficas_200702/21206950.png}
\caption{Estación Pmo Guacheneque código 21206950 caso 1.}
\end{subfigure}
~
\begin{subfigure}[normla]{0.4\textwidth}
\includegraphics[draft=false, scale=0.050]{../tablas_estaciones_dominios/graficas_200702/21206980.png}
\caption{Estación Sta Cruz De Siecha código 21206980 caso 1.}
\end{subfigure}
~
\end{figure}
 
\begin{figure}[H]\ContinuedFloat
\centering
\begin{subfigure}[normla]{0.4\textwidth}
\includegraphics[draft=false, scale=0.050]{../tablas_estaciones_dominios/graficas_200702/21206990.png}
\caption{Estación Tibaitata Atutomática código 21206990 caso 1.}
\end{subfigure}
~
\begin{subfigure}[normla]{0.4\textwidth}
\includegraphics[draft=false, scale=0.050]{../tablas_estaciones_dominios/graficas_200702/24015110.png}
\caption{Estación La Boyera Atutomática código 24015110 caso 1.}
\end{subfigure}
~
\begin{subfigure}[normla]{0.4\textwidth}
\includegraphics[draft=false, scale=0.050]{../tablas_estaciones_dominios/graficas_200702/35075070.png}
\caption{Estación Chinavita Atutomática código 35075070 caso 1.}
\end{subfigure}
~
\begin{subfigure}[normla]{0.4\textwidth}
\includegraphics[draft=false, scale=0.050]{../tablas_estaciones_dominios/graficas_200702/35085080.png}
\caption{Estación La Capilla Autom código 35085080 caso 1.}
\end{subfigure}
~
\begin{subfigure}[normla]{0.4\textwidth}
\includegraphics[draft=false, scale=0.050]{../tablas_estaciones_dominios/graficas_201408/21201200.png}
\caption{Estación Esc La Union Atutomática código 21201200 caso 2.}
\end{subfigure}
~
\begin{subfigure}[normla]{0.4\textwidth}
\includegraphics[draft=false, scale=0.050]{../tablas_estaciones_dominios/graficas_201408/21205012.png}
\caption{Estación Univ Nacional código 21205012 caso 2.}
\end{subfigure}
~
\end{figure}
 
\begin{figure}[H]\ContinuedFloat
\centering
\begin{subfigure}[normla]{0.4\textwidth}
\includegraphics[draft=false, scale=0.050]{../tablas_estaciones_dominios/graficas_201408/21205791.png}
\caption{Estación Apto El Dorado código 21205791 caso 2.}
\end{subfigure}
~
\begin{subfigure}[normla]{0.4\textwidth}
\includegraphics[draft=false, scale=0.050]{../tablas_estaciones_dominios/graficas_201408/21206790.png}
\caption{Estación Hda Sta Ana Autom código 21206790 caso 2.}
\end{subfigure}
~
\begin{subfigure}[normla]{0.4\textwidth}
\includegraphics[draft=false, scale=0.050]{../tablas_estaciones_dominios/graficas_201408/21206930.png}
\caption{Estación Pmo Guerrero código 21206930 caso 2.}
\end{subfigure}
~
\begin{subfigure}[normla]{0.4\textwidth}
\includegraphics[draft=false, scale=0.050]{../tablas_estaciones_dominios/graficas_201408/21206950.png}
\caption{Estación Pmo Guacheneque código 21206950 caso 2.}
\end{subfigure}
~
\begin{subfigure}[normla]{0.4\textwidth}
\includegraphics[draft=false, scale=0.050]{../tablas_estaciones_dominios/graficas_201408/21206960.png}
\caption{Estación Ideam Bogota código 21206960 caso 2.}
\end{subfigure}
~
%\begin{subfigure}[normla]{0.4\textwidth}
%\includegraphics[draft=false, scale=0.050]{../tablas_estaciones_dominios/graficas_201408/21206980.png}
%\caption{Estación Sta Cruz De Siecha código 21206980 caso 2.}
%\end{subfigure}
~
\end{figure}
 
\begin{figure}[H]\ContinuedFloat
\centering
\begin{subfigure}[normla]{0.4\textwidth}
\includegraphics[draft=false, scale=0.050]{../tablas_estaciones_dominios/graficas_201408/21206990.png}
\caption{Estación Tibaitata Atutomática código 21206990 caso 2.}
\end{subfigure}
~
\begin{subfigure}[normla]{0.4\textwidth}
\includegraphics[draft=false, scale=0.050]{../tablas_estaciones_dominios/graficas_201408/23125170.png}
\caption{Estación San Cayetano Autom código 23125170 caso 2.}
\end{subfigure}
~
\begin{subfigure}[normla]{0.4\textwidth}
\includegraphics[draft=false, scale=0.050]{../tablas_estaciones_dominios/graficas_201408/24015110.png}
\caption{Estación La Boyera Atutomática código 24015110 caso 2.}
\end{subfigure}
~
\begin{subfigure}[normla]{0.4\textwidth}
\includegraphics[draft=false, scale=0.050]{../tablas_estaciones_dominios/graficas_201408/35025080.png}
\caption{Estación Pnn Chingaza Autom código 35025080 caso 2.}
\end{subfigure}
~
\begin{subfigure}[normla]{0.4\textwidth}
\includegraphics[draft=false, scale=0.050]{../tablas_estaciones_dominios/graficas_201408/35025090.png}
\caption{Estación Bosque Intervenido código 35025090 caso 2.}
\end{subfigure}
~
\begin{subfigure}[normla]{0.4\textwidth}
\includegraphics[draft=false, scale=0.050]{../tablas_estaciones_dominios/graficas_201408/35075070.png}
\caption{Estación Chinavita Atutomática código 35075070 caso 2.}
\end{subfigure}
~
\end{figure}
 
\begin{figure}[H]
\centering
\begin{subfigure}[normla]{0.4\textwidth}
\includegraphics[draft=false, scale=0.050]{../tablas_estaciones_dominios/graficas_201408/35085080.png}
\caption{Estación La Capilla Autom código 35085080 caso 2.}
\end{subfigure}
~
\begin{subfigure}[normla]{0.4\textwidth}
\includegraphics[draft=false, scale=0.050]{../tablas_estaciones_dominios/graficas_201508/21201200.png}
\caption{Estación Esc La Union Atutomática código 21201200 caso 3.}
\end{subfigure}
~
\begin{subfigure}[normla]{0.4\textwidth}
\includegraphics[draft=false, scale=0.050]{../tablas_estaciones_dominios/graficas_201508/21205012.png}
\caption{Estación Univ Nacional código 21205012 caso 3.}
\end{subfigure}
~
\begin{subfigure}[normla]{0.4\textwidth}
\includegraphics[draft=false, scale=0.050]{../tablas_estaciones_dominios/graficas_201508/21205791.png}
\caption{Estación Apto El Dorado código 21205791 caso 3.}
\end{subfigure}
~
\begin{subfigure}[normla]{0.4\textwidth}
\includegraphics[draft=false, scale=0.050]{../tablas_estaciones_dominios/graficas_201508/21206600.png}
\caption{Estación Nueva Generacion código 21206600 caso 3.}
\end{subfigure}
~
\begin{subfigure}[normla]{0.4\textwidth}
\includegraphics[draft=false, scale=0.050]{../tablas_estaciones_dominios/graficas_201508/21206790.png}
\caption{Estación Hda Sta Ana Autom código 21206790 caso 3.}
\end{subfigure}
~
\end{figure}
 
\begin{figure}[H]\ContinuedFloat
\centering
\begin{subfigure}[normla]{0.4\textwidth}
\includegraphics[draft=false, scale=0.050]{../tablas_estaciones_dominios/graficas_201508/21206920.png}
\caption{Estación Villa Teresa Atutomática código 21206920 caso 3.}
\end{subfigure}
~
\begin{subfigure}[normla]{0.4\textwidth}
\includegraphics[draft=false, scale=0.050]{../tablas_estaciones_dominios/graficas_201508/21206930.png}
\caption{Estación Pmo Guerrero código 21206930 caso 3.}
\end{subfigure}
~
\begin{subfigure}[normla]{0.4\textwidth}
\includegraphics[draft=false, scale=0.050]{../tablas_estaciones_dominios/graficas_201508/21206960.png}
\caption{Estación Ideam Bogota código 21206960 caso 3.}
\end{subfigure}
~
\begin{subfigure}[normla]{0.4\textwidth}
\includegraphics[draft=false, scale=0.050]{../tablas_estaciones_dominios/graficas_201508/21206980.png}
\caption{Estación Sta Cruz De Siecha código 21206980 caso 3.}
\end{subfigure}
~
\begin{subfigure}[normla]{0.4\textwidth}
\includegraphics[draft=false, scale=0.050]{../tablas_estaciones_dominios/graficas_201508/21206990.png}
\caption{Estación Tibaitata Atutomática código 21206990 caso 3.}
\end{subfigure}
~
\begin{subfigure}[normla]{0.4\textwidth}
\includegraphics[draft=false, scale=0.050]{../tablas_estaciones_dominios/graficas_201508/23125170.png}
\caption{Estación San Cayetano Autom código 23125170 caso 3.}
\end{subfigure}
~
\end{figure}
 
\begin{figure}[H]\ContinuedFloat
\centering
\begin{subfigure}[normla]{0.4\textwidth}
\includegraphics[draft=false, scale=0.050]{../tablas_estaciones_dominios/graficas_201508/35025090.png}
\caption{Estación Bosque Intervenido código 35025090 caso 3.}
\end{subfigure}
~
\begin{subfigure}[normla]{0.4\textwidth}
\includegraphics[draft=false, scale=0.050]{../tablas_estaciones_dominios/graficas_201508/35035130.png}
\caption{Estación Pmo Chingaza código 35035130 caso 3.}
\end{subfigure}
~
\begin{subfigure}[normla]{0.4\textwidth}
\includegraphics[draft=false, scale=0.050]{../tablas_estaciones_dominios/graficas_201508/35075070.png}
\caption{Estación Chinavita Atutomática código 35075070 caso 3.}
\end{subfigure}
~
\begin{subfigure}[normla]{0.4\textwidth}
\includegraphics[draft=false, scale=0.050]{../tablas_estaciones_dominios/graficas_201508/35075080.png}
\caption{Estación Pmo Rabanal Atutomática código 35075080 caso 3.}
\end{subfigure}
~
\begin{subfigure}[normla]{0.4\textwidth}
\includegraphics[draft=false, scale=0.050]{../tablas_estaciones_dominios/graficas_201508/35085080.png}
\caption{Estación La Capilla Autom código 35085080 caso 3.}
\end{subfigure}
~
\begin{subfigure}[normla]{0.4\textwidth}
\includegraphics[draft=false, scale=0.050]{../tablas_estaciones_dominios/graficas_201509/21205012.png}
\caption{Estación Univ Nacional código 21205012 caso 4.}
\end{subfigure}
~
\end{figure}
 
\begin{figure}[H]\ContinuedFloat
\centering
\begin{subfigure}[normla]{0.4\textwidth}
\includegraphics[draft=false, scale=0.050]{../tablas_estaciones_dominios/graficas_201509/21205791.png}
\caption{Estación Apto El Dorado código 21205791 caso 4.}
\end{subfigure}
~
\begin{subfigure}[normla]{0.4\textwidth}
\includegraphics[draft=false, scale=0.050]{../tablas_estaciones_dominios/graficas_201509/21206600.png}
\caption{Estación Nueva Generacion código 21206600 caso 4.}
\end{subfigure}
~
\begin{subfigure}[normla]{0.4\textwidth}
\includegraphics[draft=false, scale=0.050]{../tablas_estaciones_dominios/graficas_201509/21206790.png}
\caption{Estación Hda Sta Ana Autom código 21206790 caso 4.}
\end{subfigure}
~
\begin{subfigure}[normla]{0.4\textwidth}
\includegraphics[draft=false, scale=0.050]{../tablas_estaciones_dominios/graficas_201509/21206920.png}
\caption{Estación Villa Teresa Atutomática código 21206920 caso 4.}
\end{subfigure}
~
\begin{subfigure}[normla]{0.4\textwidth}
\includegraphics[draft=false, scale=0.050]{../tablas_estaciones_dominios/graficas_201509/21206930.png}
\caption{Estación Pmo Guerrero código 21206930 caso 4.}
\end{subfigure}
~
\begin{subfigure}[normla]{0.4\textwidth}
\includegraphics[draft=false, scale=0.050]{../tablas_estaciones_dominios/graficas_201509/21206960.png}
\caption{Estación Ideam Bogota código 21206960 caso 4.}
\end{subfigure}
~
\end{figure}
 
\begin{figure}[H]
\centering
\begin{subfigure}[normla]{0.4\textwidth}
\includegraphics[draft=false, scale=0.050]{../tablas_estaciones_dominios/graficas_201509/21206980.png}
\caption{Estación Sta Cruz De Siecha código 21206980 caso 4.}
\end{subfigure}
~
\begin{subfigure}[normla]{0.4\textwidth}
\includegraphics[draft=false, scale=0.050]{../tablas_estaciones_dominios/graficas_201509/21206990.png}
\caption{Estación Tibaitata Atutomática código 21206990 caso 4.}
\end{subfigure}
~
\begin{subfigure}[normla]{0.4\textwidth}
\includegraphics[draft=false, scale=0.050]{../tablas_estaciones_dominios/graficas_201509/23125170.png}
\caption{Estación San Cayetano Autom código 23125170 caso 4.}
\end{subfigure}
~
\begin{subfigure}[normla]{0.4\textwidth}
\includegraphics[draft=false, scale=0.050]{../tablas_estaciones_dominios/graficas_201509/35025090.png}
\caption{Estación Bosque Intervenido código 35025090 caso 4.}
\end{subfigure}
~
\begin{subfigure}[normla]{0.4\textwidth}
\includegraphics[draft=false, scale=0.050]{../tablas_estaciones_dominios/graficas_201509/35035130.png}
\caption{Estación Pmo Chingaza código 35035130 caso 4.}
\end{subfigure}
~
\begin{subfigure}[normla]{0.4\textwidth}
\includegraphics[draft=false, scale=0.050]{../tablas_estaciones_dominios/graficas_201509/35075070.png}
\caption{Estación Chinavita Atutomática código 35075070 caso 4.}
\end{subfigure}
~
\end{figure}
 
\begin{figure}[H]\ContinuedFloat
\centering
\begin{subfigure}[normla]{0.4\textwidth}
\includegraphics[draft=false, scale=0.050]{../tablas_estaciones_dominios/graficas_201509/35075080.png}
\caption{Estación Pmo Rabanal Atutomática código 35075080 caso 4.}
\end{subfigure}
~
\begin{subfigure}[normla]{0.4\textwidth}
\includegraphics[draft=false, scale=0.050]{../tablas_estaciones_dominios/graficas_201509/35085080.png}
\caption{Estación La Capilla Autom código 35085080 caso 4.}
\end{subfigure}
~

 
 \caption{Caption}
 \label{fig:my_label}
\end{figure}
 % Anexo de las graficas de todas las simulaciones vs la temperatura de referencia




%
%%%%%%%%%%%%%%%%%%%%%%%%%%%%%%%%%%%%%%%%%%%%%%%%%%%%%%%%%%%%%%
%%%%%%%%%%%%%%%%%%%%%%%%%%%%%%%%%%%%%%%%%%%%%%%%%%%%%%%%%%%%%%
\section{Comparación entre datos de una estación automática y una convencional en el evento de helada presentado en Febrero del 2007}

%%%acá voy

Las estaciones meteorológicas automáticas están tomando el lugar de las observaciones con las estaciones convencionales. Pero es claro que las estaciones meteorológicas automáticas necesitan de un periodo de observaciones en paralelo y todos los datos debe recibir un control de calidad para asegurar una homogeneidad en los datos \citep{Ying2004}, ya que en los análisis futuros la información provendrá solo de las estaciones automáticas.

Las estaciones automáticas Los sensores permanecen en el campo hasta que se reportan problemas o el sensor cumple su tiempo útil. El tiempo de vida útil se establece en base de la experiencia de la red \citep{Shafer2000}.

El promedio y la desviación estándar no debe superar ciertos límites de lo contrario los datos serán tomadon como sospechosos \citep{Shafer2000}.

La rutina de persistencia consiste en evaluar los datos cada 24 horas, si la diferencia excede los límites entonces será marcada como sospechosa \citep{Shafer2000}. Este test es útil para encontrar los momentos cuando el sensor no varía (se queda pegado).

Las inhomogeneidades en las series climáticas pueden ser causadas por cambios en la instrumentación, prácticas de observación, el método para realizar el cálculo la temperatura media de la estación y condiciones medioambientales que rodean la estación de observación \citep{Menne2001}.

Los errores pueden ser producidos por un ruido electrónico en las medidas, comunicaciones defectuosas, degradación del sensor, golpe de un rayo o cambios de los registros en los días \citep{Menne2001}.

Uno de los errores que se presentan es el de la transposición de los números (ejemplo: 12 es registrado por 21) o la transposición de máximos y mínimos.

A nivel espacial cuando las mediciones exceden 2 desviaciones estándar serán marcadas como sospechosas y si el dato excede tres desviaciones estándar será marcada como un dato erróneo.

Se realizaron dos gráficas de las temperaturas del aire medidas a 2 metros y a 10 centímetros. La primera gráfica nos muestra los valores correspondientes a los datos registrados en 6 variables de HYDRAS. Los valores máximos de temperatura a 2 metros corresponden al color rojo, los valores de temperatura máxima corresponden al color azul y los valores de temperatura mínima corresponden al color verde. Para el caso de las temperaturas a 10 centímetros del suelo los valores de temperatura tienen color fucsia, los valores de temperatura máxima tienen color morado y los valores de temperatura mínima tienen un color café.

La primera gráfica \ref{subfig:b9} se graficó primero los valores de promedio, seguido de valores máximos y mínimos. Lo primero que se nota es que los valores de temperatura a 2 metros las tres variables coinciden en los mismos puntos, esto quiere decir que se están tomando con un mismo sensor pero se están guardando en variables diferentes. Además notamos que los valores de temperatura máxima en algunos casos no coincide con el valor máximo y lo mismo sucede con los valores mínimos. Y este comportamiento también se ve en la temperatura del suelo.

La segunda gráfica \ref{subfig:b10} se graficó cambiando el orden de las variables en este caso se graficó primero los valores máximos, mínimos y de la totalidad de la temperatura. Esto implica que la última línea corresponda a los valores de la totalidad de la temperatura, pero se pueden ver algunos puntos, esto implica que la variable de la totalidad de la temperatura tiene espacios vacíos cuyos datos son guardados en otras variables.

Los valores mínimos de las estaciones no se reportan cada hora, pero los valores de la variable llamada promedio se reportan cada hora.

\begin{figure}[H]
	\centering
			\begin{subfigure}[b]{0.3\textwidth}

			%%\includegraphics[draft=true, scale=0.1]{temp_hydr_2.png}
			\caption{Gráfica de las temperaturas de HYDRAS detalle}
			\label{subfig:b9}

			\end{subfigure}
			%% % Simbolo usado para poner las gráficas una frente a la ootra
			\begin{subfigure}[b]{0.3\textwidth}

			%%\includegraphics[draft=true, scale=0.1]{temp_hydr_1.png}
		\caption{Gráfica de las temperaturas de HYDRAS datos faltantes}
			\label{subfig:b10}

			\end{subfigure}			

		
		\caption{Gráfico de la comparación entre variables de HYDRAS}
		\label{gra:conv_vali}
	
\end{figure}



La helada más fuerte que se ha presentado en los últimos 20 años en la Sabana de Bogotá, ha sido la helada presentada entre el 29 de febrero del 2007 y el 8 de febrero del 2007. Se tomó esta fecha para realizar una comparación entre una estación automática y una estación convencional para estas fechas. Como resultado se obtuvo la figura \ref{subfig:b1}. En esta figura podemos encontrar la estación automática está reportando tres tipos de valores: máximos (tmp\_2m\_max), mínimos (tmp\_2m\_min) y promedio (tmp\_2m). Pero los valores más altos de temperatura no están siendo reportados en la variable de valores máximos, ya que esta está siendo reportada en la variable de temperaturas promedio. \textbf{Los datos de temperatura se almacenan en dos variables, pero el criterio de almacenamiento en una u otra variable depende más del tiempo que de ser valores máximos o mínimos.}\\


La estación automática HYDRAS reporta varios valores de temperatura tales como: temperatura a 2 metros, temperatura máxima a dos metros, temperatura mínima a dos metros, temperatura a 10 cm, temperatura máxima a 10 cm, temperatura mínima a 10 cm, temperatura a 10 cm bajo el suelo, temperatura a 30 cm bajo el suelo y temperatura a 50 centímetros bajo el suelo.\\

Las condiciones ambientales afectan el perfil de temperatura del suelo. La distribución de temperatura en el suelo se ve afectada por la estructura y las condiciones físicas del suelo, cobertura del suelo, la interacción con el clima, determinado por la temperatura del aire, viento, radiación solar, humedad del aire y precipitación. Las variaciones sobre el suelo pueden afectar las oscilaciones de temperatura hasta aproximadamente un metro. La temperatura por debajo de  un metro de profundidad usualmente no se ve afectada por cambios de los ciclos diurnos de la temperatura del aire y la radiación solar \citep{florides2005}.

Todas suelos poseen una capacidad diferente de transferencia de calor por ejemplo \citet{florides2005} dice que las rocas que son ricas en cuarzo poseen una graa conductividad térmica, pero en comparación las suelos que son ricos en arcilla y materia orgánica poseen menor capacidad de conductividad térmica. \textcolor{blue}{ \href{http://ces.iisc.ernet.in/hpg/envis/Remote/section994.htm}{CEP}} \citep{CES2000}.

La razón física para el retraso de las ondas de la temperatura es debido a que a cierta profundidad cierta cantidad de temperatura es absorbida o libreada a través de la propagación de la temperatura a través del camino de la propagación del la temperatura como lo muestra \citet{hillel2013} en su libro. Donde muestra una gráfica idealizada de la variación idealizada de la temperatura a través del perfil 

\begin{figure}[H]
	\begin{center}
\includegraphics[draft=true, scale=.5]{temp_prof.png}
	\caption{Variación idealizada de la temperatura del suelo en diferentes tiempo para diferentes profundidades}
	 \label{gra:tmp_soil}
	\end{center}
\end{figure}

realizar una comparación de los valores de temperatura reportado por los sensores a diferentes alturas, podemos observar que hay diferencias marcadas. Los sensores que reportan las más bajas temperaturas ordenados de menor a mayor temperatura son el sensor ubicado a 2 metros, 10 cm, 30 cm y 50 cm. Los sensores que reportan las más altas temperaturas ordenadas de mayor a menor temperatura son: sensor a 10 cm, 2 m, 30 cm y 50 cm.

Uno de los aspectos a resaltar de esta comparación es que se realizó una comparación con las estaciones convencionales. Y podemos ver que los valores de temperaturas mínimas del termómetro de mínimas de la estación convencional coinciden con los puntos más bajos de la estación automática y nunca el valor de la estación automática es menor que el valor de la estación convencional. Pero para el caso de las temperaturas altas podemos ver que en la mayoría de los casos los valores de la estación automática exceden los valores de la estación convencional.

 

\begin{figure}[H]
	\begin{center}
	\includegraphics[draft=true, scale=0.3]{conv_hyd_2007.png}
		\caption{Comparación de las temperaturas registradas por la estación convencional y la estación automática}
		\label{subfig:b1}
	\end{center}
\end{figure}

En CORPOICA Tibaitatá se encuentra una estación convencional del IDEAM que tiene la capacidad de registrar los valores de temperatura a diferentes niveles de altura como lo podemos observar en la gráfica \ref{grafica_dif_alt_suelo} donde en los años 2007 y 2012 gráficas \ref{suelo_2007} y \ref{suelo_2012}, respectivamente se observa una diferencia entre los valores de la estación automática HYDRAS y la estación convencional.

El promedio de las temperaturas mínimas para esta estación convencional se encuentra en la tabla \ref{tabla_minimas_convencional}. Donde en el conjunto de gráficas \ref{grafica_dif_alt_suelo} podemos ver que en la mayoría de los casos se presenta un orden de menor a mayor de la siguiente forma: 5 centímetros, 10 centímetros, 50 centímetros, 1 metro y 2 metros (Mínima convencional). Si calculamos la diferencia entre los promedios de las temperaturas a 2 metros y 5 centímetros podemos concluir que hay una diferencia de 2.8 \celc. Pero se nota que hay una diferencia con respecto al año 2017 ya que el los ordenes tienden a cambiar, y se presenta un mejor ajuste entre las temperaturas mínimas a 10 cm de ambas estaciones gráfica \ref{suelo_2017}.



\begin{figure}[H]
	
			
			\begin{subfigure}[b]{\textwidth}

			%%\includegraphics[draft=true, scale=0.2]{comparacion_tmp_del_suelo_2007.png}
			\caption{Detalle de la comparación de la estación automática HYDRAS contra la estación convencional en diferentes alturas para el año 2007}
			\label{suelo_2007}

			\end{subfigure}
	

			%% % Simbolo usado para poner las gráficas una frente a la ootra
			
			\begin{subfigure}[b]{\textwidth}

			%%\includegraphics[draft=true, scale=0.2]{comparacion_tmp_del_suelo_2012.png}
		\caption{Detalle de la comparación de la estación automática HYDRAS contra la estación convencional en diferentes alturas para el año 2012}
			\label{suelo_2012}

			\end{subfigure}		
			
			\begin{subfigure}[b]{\textwidth}

			%%\includegraphics[draft=true, scale=0.2]{comparacion_tmp_del_suelo_2017.png}
		\caption{Detalle de la comparación de la estación automática HYDRAS contra la estación convencional en diferentes alturas para el año 2017}
			\label{suelo_2017}

			\end{subfigure}			

		
		\caption{Gráfico de diferentes años donde se comparan la estación automática de la red HYDRAS y la estación convencional}
		\label{grafica_dif_alt_suelo}
	
\end{figure}




%%% Tabla de los valores mínimos

\begin{table}[]
\centering

\begin{tabular}{l|l}
\textbf{Promedio de la variable} & \textbf{\celc} \\ \hline
Temperatura a 5 cm               & 4.4         \\
Temperatura a 10 cm              & 5.1         \\
Temperatura a 50 cm              & 5.9         \\
Temperatura a 100 cm             & 6.7         \\
Temperatura a 200 cm             & 7.2        
\end{tabular}
\caption{Tabla resumen de las temperaturas mínimas reportados por la estación convencional}
\label{tabla_minimas_convencional}
\end{table}



%%%%%%%%%%%%%%%%%%%%%%%%%%%%%%%%%%%%%%%%%%%%%%%%%%%%%%%%%%%%%%%%%%%%%%%%%%%%%%%%%
%%%%%%%%%%%%%%%%%%%%%%%%%%%%%%%%%%%%%%%%%%%%%%%%%%%%%%%%%%%%%%%%%%%%%%%%%%%%%%%%%


\subsection{útiles para la comparación entre una estación convencional y una automática}

El funcionamiento de las estaciones automáticas y convencionales es diferente. Por esta razón se han realizado estudios, donde el objetivo es determinar si hay una diferencia entre las estaciones automáticas y las estaciones convencionales \citep{auchmann2012physics}. Por ejemplo \citet{augter2013vergleich} concluye que el cambio en las observaciones solo provoca pequeños cambios para la presión atmosférica y la temperatura y no se presentaron inhomogeneidades, la precipitación es ligeramente diferente, pero la mayor incertidumbre es reflejada en los sensores de lectura manual tales como la medición de la humedad y el brillo solar \citep{augter2013vergleich}. Pero otros autores como \citet{kamtz1831lehrbuch} han reportado diferencias en los valores de las mediciones, al igual que \citet{lacombe2010results} y \citep{kaspar2016climate} quienes aseveran que en ciertos casos la temperatura registrada por la estación convencional registra valores más altos.\\

Una de las ventajas de las estaciones automáticas es que la recolección de los datos mediante estos sensores permite un seguimiento más rápido de las condiciones atmosféricas para la toma de decisiones de importancia agrícola \citep{Soares2017}.

La temperatura promedio es calculada con la siguiente fórmula:

\begin{equation}\label{fx:promedio}
T_{promedio} = \frac{T_{I} + T_{II} +T_{II}}{4}
\end{equation}

donde $T_{I}$ es la temperatura observada a las 6:00, $T_{II}$ temperatura observada a las 12:00 y $T_{III}$ temperatura observada a las 18:00. Esta es una adaptación de la fórmula propuesta por \citet{kamtz1831lehrbuch}.

Los valores reportados por las estaciones convencionales son registrados basado en dos tipos de termómetros, para las temperaturas medias se usa un termómetro de mercurio y para las temperaturas mínimas se usa un termómetro de alcohol \citep{augter2013vergleich}. 

Los abrigos meteorológicos de las estaciones convencionales son basados en las modificaciones hechas por la Real Sociedad Meteorológica en 1884 y los sensores de las estaciones automáticas son operadas en Abrigos Laminares del tipo "LAM 630".

Las estaciones automáticas en comparación con las estaciones convencionales, en general midieron valores por debajo de lo normal \citet{kamtz1831lehrbuch}.

En caso de alta radiación y baja velocidad del viento se ha demostrado que el abrigo meteorológico LAM 630 registra valores de temperatura más alto en comparación con la garita de termómetro tipo Stevenson \citep{lacombe2010results, kaspar2016climate}.

El aumento de la temperatura es en parte generado por el efecto de la radiación y la posición del sensor dentro del abrigo. De acuerdo con \citet{kaspar2016climate} el sensor se debe ubicar en una posición que evite la radiación directa en el atardecer, ya que en el estudio realizado se ve que no hay sesgo de la temperatura en el verano.

Como conclusiones \citet{kaspar2016climate} dice que el cambio en la tecnología no introduce un incremento artificial de la temperatura media. Pero el efecto de las temperaturas extremas diarias se vio afectado por el uso del abrigo LAM 630.

Existe la necesidad de actualizar los sensores ya que desde 1989 que no se hace una actualización \citep{lacombe2010results}. El lugar de la comparación se realizó en un lugar que presentó temperaturas de 45 \celc. 

La ventilación de los sensores dan resultados erróneos \citep{lacombe2010results}

Algunas de las estaciones evaluadas por \citet{Soares2017} presentaron valores de temperatura máxima superiores en las estaciones automáticas en comparación de las estaciones convencionales. Los mejores ajustes de datos de las estaciones automáticas y convencionales se presentaron para la variable de precipitación, un ajuste intermedio se presentó para la humedad relativa y un bajo ajuste se presentó para la velocidad del viento.

El proceso final de la limpieza y calidad de los datos es el juzgamiento de los datos por parte de un experto \citep{Estevez2008}

\cite{Estevez2008} nombra varios pasos para la realización de la validación:

\begin{enumerate}
\item Validación de la estructura, se comprueba que todos los datos estén en la misma forma y dado el caso que no coincidan las fechas estos datos deben ser eliminados

\item Validación según los límites, esta parte de la evaluación tiene en cuenta ciertos rangos para los cuales los datos no deben exceder.

\item Validación de límites flexibles

\item Validación de la coherencia temporal del dato, usada para chequear el exceso de variación de los datos. Donde se compara entreo dos o más observaciones consecutivas.

\item Validación de la coherencia interna, es usado para la verificación de la coherencia entre variables. También es usado para comparar los valores máximos de alguna variable con respecto a los valores del mismo día.

\item Validación de la coherencia temporal de la serie, se usa un periodo de 24 horas y se evalúa el promedio y la desviación estándar, si dichos valores son inferiores a un valor, entonces todos los datos serán reportados como sospechosos. Ya que se espera que los datos estén "pegados".

\item Validación de la coherencia espacial, esta prueba hace referencia a que los valores de estaciones cercanas deben reportar valores similares de lo contrario serán marcados como sospechosos.

\item Inspección visual, para hacer esta revisión es muy útil realizar una representación temporal de las distintas variables en varios niveles de agregación. Para la precipitación y la irradiación resulta muy útil el método de doble masa.

\end{enumerate}

Algunos datos fueron registrados como erróneos y esto se debe a el mantenimiento periódico que se les realiza a las estaciones \citep{Estevez2008}.

La validación de estos datos permitió tomar decisiones tales como la sustitución de sensores o la depuración de registros fuera de rango antes de ser empleados \citep{Estevez2008}.

\citep{Graybeal2004b} propone que para el caso de la humedad la diferencia entre dos valores consecutivos no debe superar los 20\celc. $|T_0 - T_{-1}| > 20^{\circ}C$
 
Existen muchas formas para la determinación de los outliers, pero la determinación de los límites a los cuales se maneja es un concepto subjetivo \citet{Graybeal2004b}.

La temperatura del aire más extrema que se halla presentado fue en Libya en septiembre de 1922 y fue de 57.8\celc \citep{Kubecka2001}

Debido a los datos meteorológicos obtenidos de las estaciones meteorológicas automáticas, los modelos que usan estos datos han presentado resultados cuestionables \citep{Meek1994}.

No se pudo usar el la prueba de la consistencia en la humedad relativa, ya que en la humedad relativa es muy común que la humedad alcance el 100\% y se mantenga en estos valores.

Los valores extremos de precipitación en Colombia según \citet{Arango2014} se han presentado los mas bajos valores en la Guajira con 500 mm/anuales o menos y los más altos valores en la Región Pacífica con más de 9000 mm/año.

Debido a las condiciones orográficas las precipitaciones en el país varían demasiado \citep{Arango2014}. La precipitación posee en general un régimen bimodal, para la mayor parte de la región Andina y de la región Caribe.

La temperatura en la región andina presenta una distribución bimodal

Existen varios límites para la precipitación, \citet{Estevez2011} propone 120 mm/h, \citet{Feng2004} propone 1812.8 mm/día.\\

Entrevista con Jaime Andrés Villareal Rojas el día 20180618 funcionario encargado de la automatización de las estaciones Hydras:

\begin{enumerate}
\item Ellos han notado la diferencia entre las temperaturas de una estación automática y una estación convencional.
\item Existen varios tipos de sensores, ya que ellos no pueden comprar sensores a una sola marca porque eso no es legal.
\item Hasta ahora se van a comenzar las labores de calibración a los sensores porque hasta el momento se instalan los sensores y sólo cuando el sensor presenta datos extraños se procede a realizar el cambio de los mismos, ahora se va a hacer mantenimiento cada 6 meses y se hará calibración de los sensores cada año.
\item Yo tenía dudas con respecto a la forma como se toman los datos, pero me aclararon que los sensores son capaces de tomar la información de todas las variables al mismo tiempo.
\item Las variables acumuladas como por ejemplo la precipitación se toma de 7 am hasta el siguiente día a las 7 am.
\item Es importante para el análisis de la temperatura incluir las 3 temperaturas máximas, mínimas y horarias; ya que las máximas y mínimas registran valores que en algunos casos no se encuentran en la variable horaria.

\end{enumerate}

Buscar la justificación para los límites de la temperatura ya que usé -20\celc y 40\celc. La posible justificación es que las temperaturas mínimas registradas por las estaciones bajan hasta -10\celc entonces le damos una tolerancia de 10\celc más y lo mismo para las temperaturas máximas.

A la velocidad del viento no se le puede hacer el control para los valores pegados ya que hay muchos momentos que la velocidad no cambia.

En el modelo WRF la variable SWDOWN es igual a GHI. Global Horizontal Irradiance is the total solar radiation incident on a horizontal surface. It is the sum of Direct Normal Irradiance (DNI), Diffuse Horizontal Irradiance (DHI), and ground-reflected radiation.

La radiación reportada por los sensores de las estaciones HYDRAS corresponde al rango visible, por esta razón se usará la variable (SWDOWN DOWNWARD SHORT WAVE FLUX AT GROUND SURFACE W m-2) que corresponde a GHI.

Los parámetros para la validación de la dirección del viento se tomaron de \citep{Shafer2000a, DeGaetano1996} donde los rangos para la dirección fueron 0-360 y la diferencia de los datos no debía exceder 360?

Si la velocidad es 0 puede registrar una dirección diferente de 0 dirección. Esto se debe a que el sensor de rapidez del viento puede captar la velocidad del viento, pero si la velocidad del viento no supera cierto humbral entonces la velocidad no se registrará en la plataforma. Como ejemplo si la rapidez fue de 0.000001 el sensor de velocidad en la plataforma registrará 0, pero si la velocidad es de 0.9, entonces el sensor registrará 1.

\subsection{Comparación de los resultados de los resultados obtenidos con diferentes dominios y resoluciones}

Se crearon 3 dominios con las coordenadas de la tabla \ref{tabla_coordenadas_dominios}. Para cada uno de los dominios se cambiaron las resoluciones, teniendo en cuenta que cada nuevo dominio debe tener una tercera parte de la resolución del dominio que lo contiene. Se probaron 

\begin{center}

\begin{table}[H]
\begin{tabular}{lll}
Punto & Latitud & Longitud \\ \hline
1A    & 3.19    & -75.72   \\
1B    & 6.72    & -72.01   \\
2A    & 3.61    & -75.32   \\
2B    & 6.28    & -72.37   \\
3A    & 4.01    & -74.92   \\
3B    & 5.61    & -73.01  
\end{tabular}
\caption{Coordenadas de los dominios.}
\label{tabla_coordenadas_dominios}
\end{table}
\end{center}


\section{Determinación de las fechas de estudio}%Código busqueda_fechas.py

Los pixeles usados son de 108.8 km de ancho

Se usó la estación automática de Tibaitatá de la red HYDRAS. La fecha más antigua usada para el análisis depende de la disponibilidad de datos de la estación automática, esta fecha corresponde al 2007 ya que los datos anteriores no presentaron buenos resultados en la validación de los mismos. Esta estación reporta la temperatura a 2 metros en tres variables temperatura máxima, temperatura mínima y temperatura cada hora. El conjunto de datos de las temperaturas máximas y mínimas se encuentra en su mayoría representado por las temperaturas horarias. Solo en algunos casos las temperaturas máximas y temperaturas mínimas no están dentro de las temperaturas totales. Por esta razón los datos de las temperaturas máximas y mínimas que no se encuentran dentro de las temperaturas horarias fueron adicionados y se creó una nueva variable de temperatura que reúne las temperaturas horarias más los valores adicionales de temperatura máxima y mínima.\\

En el periodo de estudio no se presentaron casos de altas temperaturas en momentos asociados con el fenómeno el niño.\\

Se determinaron las fechas y horas en las cuales la temperatura estuvo bajo 0\celc y sobre 25\celc en la estación automática de Tibaitatá, adicionalmente se buscaron los periodos de El Niño, la Niña y los periodos más largos en horas; como resultado se obtuvo las siguientes fechas, en todos los casos se encuentra reportada la estación convencional de Tibaitatá ya para poder comparar los datos obtenidos con la estación automática:

\begin{itemize}
\item{Caso 1}%Este caso corresponde al mas largo, fuerte y está en el último mes de un periodo el niño ggweather.com/enso/oni.html

El primer caso de estudio se encuentra entre las fechas 31 de enero del 2007 hasta el 5 de febrero del 2007 que corresponde a los meses más frecuentes de heladas, en temporada reportada bajo la influencia del El Niño. Se seleccionó este caso porque presentó una de las temperaturas más bajas para el día 4 de febrero del 2007 ver tabla \ref{table:caso1}. Según la estación automática la helada tuvo una duración de 5 horas comenzando a las 2 a.m. y finalizando a las 7 a.m., esto la convierte en la helada más larga para nuestro periodo de estudio.

En el periodo estudiado 2007-2016 no se presentaron heladas en fechas asociadas con el fenómeno La Niña.

\begin{table}[H]
\centering

\begin{tabular}{llll}
Temperatura \celc & Código   & Nombre de la estación & Municipio \\ \hline
-8.8           & 21205980 & PROVIDENCIA GJA       & Tenjo     \\
-7.4           & 21205920 & SUASUQUE              & Sopó      \\
-7.0           & 21205880 & FLORES CHIBCHA        & Madrid    \\
-4.6           & 21205420 & TIBAITATÁ             & Mosquera  \\
-4.7           & 21205420 & TIBAITATÁ [Automática]& Mosquera
\end{tabular}
\caption{Tabla de las temperaturas más bajas para el caso 1 del día 4 de febrero del 2007}
\label{table:caso1}
\end{table}



\item{Caso 2}%Mes no común en un niño (el periodo del 2015 fue niño todo)

La segunda fecha seleccionada corresponde a una helada presentada en meses atípicos del año, en temporada reportada bajo la influencia del El Niño. La cuál se encuentra en el periodo correspondiente al 29 de agosto del 2014 hasta el 2 de septiembre del 2014. Para esta fecha seleccionada la helada sólo fue registrada por la estación convencional de Tibaitatá ver tabla \ref{table:caso2}. La helada se presentó el día 30 de agosto del 2017.

\begin{table}[H]
\centering

\begin{tabular}{llll}
Temperatura \celc & Código   & Nombre de la estación & Municipio \\ \hline
-1           & 21205420 & TIBAITATÁ             & Mosquera \\
-1.0         & 21205420 & TIBAITATÁ[Automática] & Mosquera
\end{tabular}
\caption{Tabla de las temperaturas más bajas para el caso 2 del día 30 de agosto del 2014}
\label{table:caso2}
\end{table}

\textit{En el periodo de estudio no se presentaron altas temperaturas en fenómenos de la NIÑA}

\item{Caso 3}

\textit{Se creó una carpeta en agrometeo llamada /home/agrometeo/wrf/resultados/zona\_201508, para procesar el wps y los archivos usados se descargaron en Downloads}

La cuarta fecha seleccionada corresponde a una alta temperatura presentada en el mes de agosto en una temporada de fenómeno El Niño. Se seleccionó este caso porque se presentó en un mes poco habitual. La cuál se encuentra en el periodo correspondiente al 24 de agosto del 2015 hasta el 28 de agosto del 2015. Según las estaciones convencionales para estas fechas en la zona de estudio no se presentaron temperaturas sobre 25\celc, razón por la cuál se usaron como nuevo límite para las convencionales un valor de 20\celc y se obtuvo la siguiente tabla \ref{table:caso3}. El valor de la estación convencional no superó los 20\celc. El periodo de duración de este evento fue de una hora, iniciando a las 9 am y finalizando a las 10 am.

\begin{table}[H]
\centering

\begin{tabular}{llll}
Temperatura \celc & Código   & Nombre de la estación & Municipio \\ \hline
22.0           & 21206620 & COL H DURAN DUSAN   & Bogotá \\
20.1           & 21205980 & PROVIDENCIA GJA   & Tenjo \\
19.6           & 21205420 & TIBAITATÁ   & Mosquera \\
26.0         & 21205420 & TIBAITATÁ[Automática] & Mosquera
\end{tabular}
\caption{Tabla de las temperaturas más altas para el caso del día 27 de agosto del 2015}
\label{table:caso3}


\end{table}



\item{Caso 4}


\textit{Se creó una carpeta en agrometeo llamada /home/agrometeo/wrf/resultados/zona\_201509, para procesar el wps y los archivos usados se descargaron en Downloads}

La tercera fecha seleccionada corresponde a una alta temperatura presentada en el mes de septiembre en una temporada de fenómeno El Niño. Se seleccionó este caso ya que fue uno de los que presentó mas horas sobre 25\celc, según la estación automática el tiempo sobre 25\celc fue de . La cuál se encuentra en el periodo correspondiente al 06 de septiembre del 2015 hasta el 09 de septiembre del 2015. Según las estaciones convencionales para estas fechas en la zona de estudio no se presentaron temperaturas sobre 25\celc, razón por la cuál se usaron como nuevo límite para las convencionales un valor de 20\celc y se obtuvo la siguiente tabla \ref{table:caso4}. Las temperaturas sobre 25\celc comenzó a las 10:33 y finalizó a las 15:21, duró casi 5 horas.\\

El día 20150907 no estaba disponible en los datos del GFS, por esta razón no se usaron estos datos. Pero si estaban los datos del día analizado que corresponde al 20150908.
%ftp://nomads.ncdc.noaa.gov/GFS/analysis_only/201509/20150907/

\begin{table}[H]
\centering

\begin{tabular}{llll}
Temperatura \celc & Código   & Nombre de la estación & Municipio \\ \hline
21.6           & 21206260 & C.UNIV-AGROP-UDCA   & Bogotá \\
21.6           & 21205980 & PROVIDENCIA GJA   & Tenjo \\
20.8           & 21206560 & INEM KENNEDY     & Bogotá \\
20.8           & 21205420 & TIBAITATÁ   & Mosquera \\
26.7         & 21205420 & TIBAITATÁ[Automática] & Mosquera
\end{tabular}
\caption{Tabla de las temperaturas más altas para el caso 4 del día 08 de septiembre del 2015}
\label{table:caso4}



\end{table}


\end{itemize}
%%%%%%%%%%%%%%%%%%%%%%%%%%%%%%%%%%%%%%%%%%%%%%%%%%%%%%%%%%%%%%%%%%%%%%%%%%%%%%%%%
%%%%%%%%%%%%%%%%%%%%%%%%%%%%%%%%%%%%%%%%%%%%%%%%%%%%%%%%%%%%%%%%%%%%%%%%%%%%%%%%%





%%%%%%%%%%%%%%%%%%%%%%%%%%%%%%%%%%%%%%%%%%%%%%%%%%%%%%%%%%%%%%%%%%%
%%%%%%%%%%%%%%%%%%%%%%%%%%%%%%%%%%%%%%%%%%%%%%%%%%%%%%%%%%%%%%%%%%%
\section{Lecturas del páramo}

La inicialización de los modelos juega un papel importante en el pronóstico de los eventos, por ejemplo \citet{Uribe2012} encontró que para un horizonte de pronóstico de 36 horas se deben descartar las primeras 12 horas para evitar el efecto de \textit{spin-up}.

Según \citep{Uribe2012} el anidamiento no mejora la influencia de la predicción, lo cual traduce en una perdida de tiempo para la predicción de lluvias en Colombia.

Según \citep{Uribe2012} la parametrización por el método de Kain-Fritschc con una resolución de 20 km de grilla presentó la mayor subestimación para la precipitación.

Al aumentar el horizonte de pronóstico de 36 a 48 horas se logra una mejor simulación de los valores de precipitación \citep{Uribe2012}.



\subsection{Prioridades de restauración ecológica}




Según el enfoque de \citep{Castro-Romero2014} cuando un suelo en una zona pierde atributos como el contenido de materia orgánica y la diversidad de especies de artropofauna, el suelo se degrada. Y esta degradación es considerada como una perdida paulatina de capital natural, que produce que sus habitantes perciban menos beneficios por los servicios ecosistémicos y una disminución en la calidad de vida de sus pobladores. Esto hace que las actividades agropecuarias sean más difíciles y costosas.
Estas teorías se basan en estudios previos de \citet{daily1997nature} y \citet{westman1977much}.

Las comunidades que producen algún tipo de degradación en los territorios no son conscientes de los costos del daño producido \citep{westman1977much}. Y estos daños se ven relfejados en la disminución de los ingresos netos de las cosechas \citep{Castro-Romero2014}.

\citet{Castro-Romero2014} estableció que la calidad del suelo se debe tener en cuenta los atributos químicos, físicos y biológicos.

El índice de los procesos denudativos del suelo se consolidan como la suma de la erosión, solifluxión, deslizamientos y desplomes \citep{Castro-Romero2014}. %Solifluxión: se produce cuando un material sólido fluye como si fuera un líquido viscoso.



Una de las conclusiones de \citep{Castro-Romero2014} es que los suelos con la menor valoración económica son aquellas como de mayor prioridad para la restauración. Del mismo modo \citep{Castro-Romero2014} encontró que las prioridades más bajas de restauración se presentan en aquellas unidades que mostraron las mejores condiciones de conservación y calidad del suelo.

\subsubsection{Emergía}

Uno de los resultados más importantes del estudio de \citep{Ernesto} es que el Páramo es usado como una mina del que se extrae carbón, materia orgánica de los suelos y biomasa de sus páramos. Pero adicionalmente este autor cuantificó la tasa a la que se extrae y la estimó en $2.49e-20 \frac{seJ}{year}$.

La emergía es la cantidad de energía requerida para producir algo, teniendo en cuenta la conservación y pérdida de energía que resulta de las leyes de la termodinámica. Entre más trabajo es realizado para producir algo, más energía debe ser transformada para la realización de ese trabajo y una alta emergía será almacenada en el producto. Emergía es una medida del trabajo medioambiental que es necesario para la producción de un bien o un servicio. Esta medida es sin embargo la suma de los flujos de energía requeridos para la producción de algo, expresada en base a las unidades de energía solar Joules \citep{lei2014ecological}.

"La emergía solar de un trozo de madera corresponde a la energía disponible que utilizó un ecosistema para producirlo, osea, la cantidad de Julios solares que se utilizaron para su elaboración" Frase ejemplo tomada de \citep{Castro-Romero2014}.

La transformicidad solar es la emergía solar requerida para conformar una unidad de energía de un servicio o producto, es decir, la relación entre la emergía de un servicio o un producto, es decir, la relación entre la emergía y una cantidad libre de un producto o servicio, por ello su unidad es $\frac{sej}{J}$.

\citep{Castro-Romero2014} realizó los cálculos de la emergía para la minería y la agricultura, teniendo en cuenta la cantidad de combustibles y los insumos usados en cada proceso. Tomaron los mapas de suelo del IGAC realizados en el 2000.

Realizaron un cálculo de la evapotranspiración a partir de Thornwaite realizado en 1966.

Realizaron una estimación de la escorrentía. 

En el estudio se muestra que se requiere $2.44e5 sej$

La emergía libre está dada en términos de la lluvia y el suelo.

La agricultura y la ganadería usa la energía gratuita del sol y la lluvia para producir ingresos. Pero en la ganadería se enfoca en un organismo consumidor.

La ganadería en su mayor parte depende en un 42\% de fuentes no renovables, frente a lo cual \citet{Castro-Romero2014} afirma que cuestiona la sostenibilidad en periodos de tiempo en los cuales la pérdida de materia orgánica del suelo impida su desarrollo.

La leche producida y vendida contiene más emergía que el dinero que recibe.

El sistema de ceba es el menos provechoso de los 3 sistemas ganaderos evaluados.

El páramo es movido mayormente por fuentes externas a este como lo son los combustibles e insumos materiales

El suelo es la mayor fuente de emergía
%%%%%%%%%%%%%%%%%%%%%%%%%%%%%%%%%%%%%%%%%%%%%%%%%%%%%%%%%%%%%%%%%%%%%%%%%%%%%%%%%
%%%%%%%%%%%%%%%%%%%%%%%%%%%%%%%%%%%%%%%%%%%%%%%%%%%%%%%%%%%%%%%%%%%%%%%%%%%%%%%%%








%%%%%%%%%%%%%%%%%%%%%%%%%%%%%%%%%%%%%%%%%%%%%%%%%%%%%%%%%%%%%%%%%%%%%%%%%%%%%%%%%
%%%%%%%%%%%%%%%%%%%%%%%%%%%%%%%%%%%%%%%%%%%%%%%%%%%%%%%%%%%%%%%%%%%%%%%%%%%%%%%%%



%%%%%%%%%%%%%%%%%%%%%%%%%%%%%%%%%%%%%%%%%%%%%%%%%%%%%%%%%%%%%%%%%%%%%%%%%%%%%%%%%
%%%%%%%%%%%%%%%%%%%%%%%%%%%%%%%%%%%%%%%%%%%%%%%%%%%%%%%%%%%%%%%%%%%%%%%%%%%%%%%%%

\section{Avances búsqueda parametrizaciones WRF}


Se creó una carpeta llamada \texttt{zona\_est\_20180320} en esta carpeta se va a ejecutar el WPS que será usado para probar las combinaciones se ejecutará desde un día antes y un día después de las fechas que se usarán en el WRF, ya que, en el WRF se usarán las fechas \texttt{20070201-20070204}. El centro del dominio será en el município de Tocancipá estas serán las coordenadas \texttt{-73.96740787376062087 4.9704810067620171} en las variables correspondientes a \texttt{ref\_lat, ref\_lon, truelat1, stand\_lon}. Y se tomarán los datos proporcionados por el IDEAM para la Vtable. Sólo se usó un dominio para llegar a una resolución de 2 kilómetros con el WRF, una copia quedó dentro de la carpeta llamada resultados.\\

Al final se crearon los archivos netCDF para las fechas seleccionadas. 

Para la comparación de los mejores modelos se usó el diagrama de Taylor a partír de un código obtenido del repositorio de \textcolor{blue}{ \href{https://gist.github.com/ycopin/3342888}{ycopin}}. En este código se usó lo propuesto por \citet{barnston1992} quien calcula el cuadrado medio del error a partir del la desviación estándar y el coeficiente de correlación.

\begin{equation}\label{eq:rms}
	     RMS = \sqrt{1-r^{2}}SD_{y}
\end{equation}

Para poder compara las temperaturas a 2 metros se unieron todas las temperaturas (máximas, mínimas y promedio) conla finalidad de solo usar un valor. Se priorizaron las variables de la variable llamada promedio sobre las demás esto se encuentra en el código llamado \texttt{Procesamiento\_wrf.py}

\begin{enumerate}

\item El IDEAM facilitó las namelist.* y la versión que están usando
\item Se instaló el WRFV3.9.1.1 en la máquina agrometeo
\item Se realizó una corrida con los datos del GFS para un periodo de 2016-12-28 00:00 - 2017-01-01 18:00 (4 días). Esta corrida tomó 12 horas y 30 minutos y se usaron 152 GB. Se realizaron 64 simulaciones. (265 días).

\end{enumerate}

\subsection{Resultado de las modelaciones con el WRF}

Se probaron cada una de las parametrizaciones físicas del modelo, (sin combinatoria). Este proceso comenzó el 20180404 y terminó el 20180412, se demoró un total de 8 días.
\subsection{Literatura sobre las parametrizaciones}

La comprensión de los procesos físicos que rigen las masas de aire frío y su adecuada representación en un modelo de predicción numérico (por sus siglas en inglés \textit{numerical weather prediction} (NWP)) son necesarios para una predicción adecauada de las temperaturas de la superficie y los escenarios de heladas \citep{prabha2008}.\\



Le balance de energía y la capa límite (bpl) se ha demostrado que pronostica erroneamente las tasas de enfriamiento y las altas temperaturas en la presencia de heladas advectivas \citep{heinemann1988}. El pronóstico de las heladas con la ayuda de moedlos de predicción enfrenta un reto debido la interacción no linear de los componetnes del balance hídrico, el cual puede causar pronósticos imprecisos \citep{gutowski2003}.\\

El estado del arte del WRF se definió como una colección de varios modelos de predicción numericos en la arquitectura de un solo software con dos modelos de pronóstico dos formas de pronóstico de solucionar que son el \textit{Advanced Reseach WRF} por sus siglas en inglés (ARW) y el \textit{Nonhydrostatic Mesoscale Model} por sus siglas en inglés (NMM). El ARW ha sido desarrollado y administrado por \textit{NCAR's Mesoscale and Microscalo Meteorology Laboratory}. El núcleo del NMM fue desarrollado por \textit{National Centers for Envoronmental Prediction} y es actualmente usado en el sistema \textit{Hurricane WRF} (HWRF).

La correcta parametrizaión va a depender de varios factores como lo nombra \citet{prabha2008evaluation} tales como: seleccionar correctamente la resolución, el poder computacional, condiciones iniciales, resolución del terreno, datos del uso del suelo y las parametrizaciones físicas a usar.

Para la verificación de la predicción de estos modelos es necesario comparar con las estaciones para de esta manera verificar los resultados (Validar??).

Uno de los primeros pasos que se deben desarrollar en esta metodología es la caracterización climática de la zona. \citet{prabha2008evaluation} realizaron una caracterización para cada una de las estaciones por 13 años.

\citet{prabha2008evaluation} define el índice de congelamiento como la integral de la temperatura del aire cuando esta se encuentra bajo 0\celc.\\

\citet{prabha2008evaluation} Usan dos índices para evaluar

El índice de congelamiento es definido como el numero de grados día por un periodo específico de tiempo \citep{frauenfeld2007northern}. El índice que propone \citet{frauenfeld2007northern} es definido como la suma de los valores de temperatura bajo 0\celc como:

\begin{equation}\label{eq:frost_index}
    \int_{t_0}^{t_n} T dt,\quad Para T \leq 0^{\circ}C
\end{equation}

donde $T$ corresponde al valore de la temperatura, integrado desde $t_0$ hasta el tiempo $t_n$. son el comienzo y final de los diferentes tiempos, $T$ es el valor de la temperatura y $dt$ es el cambio en el tiempo.

Existen dos tipos de heladas la radiativa y la advectiva \citep{snyder2005frost} grandes incusiones de aire frío, vientos atmósferas que se encuentran a temperaturas bajo 0\celc 

Caracterización de una helada radiativa cielos despejados, vientos en calma, inversión de temperatura, bajas temperaturas de punto de rocío y temperaturas que caen por debajo de 0\celc.

El tope de la inversión es variable ya que depende de la topografía y las condiciones climáticas, pero generalmente tiene un rango de 9 a 60 m \citep{snyder2005frost}.

En las noches despejadas más calor es irradiado fuera de la superficie en comparación con el que fue recibido durante el día. La temperatura cae rápidamente cerca de la superficie de radiación causando una inversión.

Según la gráfica de inversión de temperatura la inversión se nota a una altura de 3 o 5 metros. 

Para la protección de las heladas radiativas son más efectivos los métodos "Energy-intensive"

Dentro de las heladas radiativas hay dos categorías:

La helada blanca (hoar frost): Sucede cuando el vapor de agua depositado en la superficie forma una cobertura blanquecina de hielo y es comúnmente llamada "escarcha".

Helada negra (black): Ocurre cuando la temperatura cae bajo 0\celc y no hay formación de hielo en la superficie. Si la humedad es suficientemente baja, entonces la superficie no alcanzará el punto de congelamiento y no se formará hielo. Cuando la humedad es alta, hay una más alta probabilidad de una helada blanca. Se produce calor cuando se pasa de líquido a sólido, por esta razón la helada blanca es menos lesiva en comparación un una helada negra.

\begin{figure}[H]
	\begin{center}
	\includegraphics[draft=true, scale=0.6]{latent.png}
		\caption{Gráfica de calor latente}
		\label{subfig:cal_lat}	
	\end{center}

\end{figure}

Una clara seña de que estamos presenciando una helada cae en unas pocas horas antes de del amaneces, cuando la energía neta de radiación de la superficie cambia rápidamente de positiva a negativa. Este cambio se da principalmente porque la radiación solar decrece desde su valor más alto al medio día a 0 en el atardecer.

La densidad del flujo de calor en el suelo está determinada por 

\begin{equation}\label{eq:frost_index}
    G = -\lambda(\frac{\partial T}{\partial z})
\end{equation}

tomado de \citep{sauer2002soil}

Helada por advección\\
Masas de aire frío llegan a una área a remplazar el aire cálido que estaba presente antes de que el tiempo cambiara.

\begin{enumerate}
\item Condiciones nubladas
\item Vientos desde moderados hasta fuertes según \citet{prabha2008evaluation} son vientos con velocidades superiores a 2 $m/s$.
\item Sin inversión térmica
\item Baja humedad
\item Las temperaturas caen bajo 0\celc y permanecen de esta misma forma todo el día.
\item La mayoría de estos eventos ocurren en climas Mediterráneos y tienden a ser más comunes en las costas.
\item La mayoría de los eventos
\end{enumerate}
Este tipo de heladas son difíciles de combatir, es por esto que es la mayoría de los métodos de protección funcionan mejor en la presencia de una inversión. En algunos casos la helada por advección puede generar una helada radiativa.\\

Clasificación de métodos de protección

Existen varias clasificaciones como los métodos pasivos que son más preventivos como métodos biológicos, o técnicas ecológicas incluyendo prácticas como un alistamiento previo a la presencia de las heladas.

Los métodos activos son métodos que son basados en el uso intensivo de energía. Algunos ejemplos son calentadores, riego, máquinas de viento.\\

Entre las latitudes entre los trópicos de Cancer y Capricornio existen grandes áreas con temperaturas bajo cero. Pero aún en estas zonas a veces se presentan daños en lugares con grandes alturas.\\

Es menos probable que ocurra una helada cuando:
\begin{enumerate}
\item El terreno se encuentra en un área donde el viento sopla
\item El terreno se encuentra cerca a una masa de agua
\end{enumerate}

El humo no ofrece ningún tipo de protección a los cultivos \citep{itier1987influence}\\

Los daños causados en las plantas son debido al congelamiento del agua extra celular dentro de las plantas 

Las ubicaciones bajas presentan problemas de bajas temperaturas. Pero en algunas ocasiones se pueden presentar daños en algunos sitios, esto es debido al tipo de suelo.\\

Un suelo seco arenoso transfiere mejor el calor que un suelo seco y arcilloso y ambos transfieren y almacenan mejor la temperatura que un suelo orgánico.

La transferencia de calor del agua es tres veces mayor que la del suelo. Los suelos húmedos tienen mayor capacidad de conductividad en comparación con un suelo seco.

Obstáculos que inhiban el drenaje de las masas de aire frío.\\

La fertilización puede ser una forma para evitar ele efecto de las heladas en el cultivo

El riego es uno de los mecanismos más económicos para el control de las heladas.\\

La inundación de los suelos puede proveer protección de los cultivos entre 2 a 3\celc. Existe una relación entre la lamina de agua que se debe aplicar para evitar la helada de esa noche y la temperatura máxima registrada en el día.\\

Una de las formas de protección es realizar un pronóstico de las heladas

\subsubsection{Parámetros usado por Prabha en lso dos artículos \citep{prabha2008} y \citep{prabha2008evaluation}}


En el artículo \citet{prabha2008} podemos destacar varios aspectos:

\begin{enumerate}
\item Lo primero que se estudió fueron las condiciones iniciales
\item Capa límite planetaria
\item Superficie del terreno
\end{enumerate}

Se realizó una comparación contra la red de estaciones automáticas de la zona. En este estudio se manejaron 40 niveles de presión y se acomodaron 15 niveles en la PBL. (Grid resolution en el artículo)


La configuración de modelo fue la siguiente:

\begin{itemize}
\item WSM3 en microfísica \texttt{mp\_physics = 3: "WRF Single-Moment (WSM) 3-class simple ice scheme: A simple efficient scheme with ice and snow processes suitable for mesoscale grid sizes."}
\item La parametrización de cúmulos es la Kain-Fritsch \texttt{cu\_physics = 1: "Kain-Fritsch (new Eta) scheme: deep and shallow sub-grid scheme using a mass flux approach with downdrafts and CAPE removal time scale"}

\item La parametrización usada para la radiación de onda larga corresponde a RRTM \texttt{ra\_lw\_physics = 1: "RRTM scheme: Rapid Radiative Transfer Model. An accurate scheme using look-up tables for efficiency. Accounts for multiple bands, trace gases, and microphysics species. This scheme has been preliminarily tested for WRF-NMM."}

\item La parametrización usada para la radiación de onda corta corresponde a RRTM \texttt{•}
\end{itemize}


El experimento consistió en probar diferentes, capas límites planetarias, condiciones iniciales y diferes esquemas de suelo.:

Condiciones iniciales
\begin{itemize}
\item NAM = Grilla de 12 km. North American Regional Reanalysis
\item NARR grilla de 36 km. North American Mesosclase
\end{itemize}

Parameterizaciones de capa límite \texttt{bl\_pbl\_physics}
\begin{itemize}
\item MYJ = 2. \texttt{Mellor-Yamada-Janjic (Eta) TKE scheme}
\item YSU = 1. \texttt{YSU scheme}
\end{itemize}

Parametrizaciones del terreno \texttt{sf\_surface\_physics}
\begin{itemize}

\item SLAB = 1 \texttt{"Thermal Diffusion scheme: soil temperature only scheme, using five layers."}

\item Noah = 2. \texttt{"Noah Land-Surface Model: Unified NCEP/NCAR/AFWA scheme with soil temperature and moisture in four layers, fractional snow cover and frozen soil physics. This scheme has been preliminarily tested for WRF-NMM."}


\item RUC = 3. \texttt{"RUC Land-Surface Model: Rapid Update Cycle operational scheme with soil temperature and moisture in six layers, multi-layer snow and frozen soil physics. This scheme has been preliminarily tested for WRF-NMM."}



\end{itemize}



\subsubsection{Revisión de literatura relacionada}


Existen diferencias micrometeorológicas, como por ejemplo el echo que existan diferencias de 1\celc en 100 metros.

La máxima temperatura que puede soportar un tuberculo de papa es -0.8\celc\\


La capa de aire que toca el suelo se llama vientos catabático o vientos de drenaje. Son causados por el enfriamiento del aire, adyacente al suelo y descienden gracias a la influencia de la gravedad \citep{Stull1988pbl}.

El estudio de la capa límite planetaria contiene el estudio de la micrometeorología \citep{Stull1988pbl}.

Para el estudio de la micrometeorología \citet{Stull1988pbl} los clasifica en 3:

\begin{enumerate}
\item Métodos estocásticos
\item Teoría de similitud
\item Clasificación fenomenológica
\end{enumerate}




\subsection{Mecanismos de transferencia de energía}

Cuando la energía pasa a través del suelo por conducción se llama densidad de flujo de calor.\\
La protección de una helada consiste en tratar de reducir o remplazar la pérdida del contenido de calor sensible del aire y de las plantas.



Cuando el agua se congela la mayoría de las moléculas hacen enlaces con el nitrógeno para formar una estructura cristalina. 

\begin{table}[H]
\centering

\label{tabla_conductividad}
\begin{tabular}{@{}llll@{}}
\multicolumn{4}{l}{Conductividad termica de los suelos $W m^{-1}\cdot^{\circ}C^{-1} $} \\ \midrule
\multicolumn{1}{l|}{}         & Orgánicos  & Arcillosos & Arenosos \\ \midrule
\multicolumn{1}{l|}{Secos}    & 0.1        & 0.25       & 0.3      \\
\multicolumn{1}{l|}{Húmedos}  & 0.5        & 1.6        & 2.4     

\end{tabular}
\caption{Tabla de conductividad de suelos}
\end{table}


%%%Pagina de las tablas
\newpage
\begin{landscape}
\section{Tabla resumen de los códigos}

\begin{table}[ht]
\centering

\resizebox{\textwidth}{!}{\begin{tabular}{lll}
Nombre del código             & Función                                                                                                                                                      & Fecha    \\ \hline
pandas                        & Base de entrenamiento en pandas                                                                                                                              & 20180130 \\
pre\_procesamieto\_ideam.py   & Función que toma los datos en tr5 y cambia las columnas (pivot)                                                                                              & 20180131 \\
matplot\_lib.py               & Base de entrenamiento en matplotlib                                                                                                                          & 20180214 \\
pre-procesamiento-hydras.py   & Función para unir las bases de datos con todas sus variables con un mismo código                                                                             & 20180215 \\
manejo\_ideam\_diarias.py     & Base para manejar los datos en tr5 con la finalidad de determinar la cantidad de datos faltantes de una sola estación, Tibaitatá principalmente              & 20180216 \\
missin\_data.py               & Código para realizar los plots de las estaciones convecionales cuando se reportan NA oinconsistencias son guardados como gráficas resúmen                    & 20180217 \\
vickers\_mahrt.py             & Base de entrenamiento para las validaciones de los datos de las estaciones automáticas                                                                       & 20180222 \\
descarga\_IDEAM\_hydras.py    & Código para la descarga de los datos de la red HYDRAS                                                                                                        & 20180223 \\
bases\_dane\_papa.py          & Código usado para procesar la información del DANE. Usado para la creación de las tablas de estadística usados en el documento y en las gráficas municipales & 20180305 \\
nuevo\_preprocesam\_hydras.py & Código creado para 1 eliminar las bases por variable y 2 eliminar las bases por años. Poner la información de cada estación en una sola forma.               & 20180307 \\
automatizacion\_graph.py      & Grafica donde se comparan los valores de la estación convencional y la estación automática. Es la versión mejorada y automatizada                            & 20180313 \\
automatizacion\_graph.py      & Código usado para la automatización de las gráficas donde se compara la estación automática y la estación convencional.                                      & 20180313 \\
extraccion.py                 & Código usado para la extracción de los datos de la modelación del WRF, primera versión. No está muy bien                                                     & 20180321 \\
extraccion\_2.py              & Código usado para la extracción de datos del WRF versión mejorada y modificada                                                                               & 20180304 \\
validacion\_tmps.py           & Validación de los datos de temperatura de las estaciones HYDRAS, con sus gráficas                                                                            & 20180331 \\
validacion\_20180522.py           & Modificación del código antiguo de validación, se mejoró SPIKES y sólo se procesa información a 2m, que son unidas en una sola variable & 20180522 \\
Procesamiento\_wrf.py         & Código para crear una comparación de los datos de la estación HYDRAS y las diferentes parametrizaciones del modelo WRF. Se crea una sola columna para la temperatura                                       & 20180404 \\
mapas\_matplot\_lib.py        & Código exploratorio para buscar una forma de realizar el plot de las variables producidas por el WRF. Con matplotlib                                         & 20180405 \\
plot.py                       & Código creado con cartophy para plotear mapas provenientes del WRF. Buenos resultados                                                                        & 20180421 \\
validacion\_hydras\_manual.py & Gráficas de la comparación de termómetro a diferentes alturas y validación de las estaciones automáticas contra las convencionales diferentes altura                                  & 20180423 \\
busqueda\_fechas.py & Búsqueda de las fechas de las heladas y altas temperaturas Y extracción de las gráficas resumen de la zona. Gráficas de las estaciones con limpieza de datos & 20180606\\
validacióntemp.py & Búsqueda de las fechas de las heladas y altas temperaturas Y extracción de las gráficas resumen de la zona. Gráficas de las estaciones con limpieza de datos & 20180606
\end{tabular}}
\caption{Tabla resumen de los códigos}
\label{tabla_resumen}
\end{table}



\end{landscape}
\begin{longtable}{rllllrr}
\toprule
   Código &                      Nombre & Tipo &     Departamento &         Município &   Latitud &   Longitud \\
\midrule
\endhead
\midrule
\multicolumn{3}{r}{{Continued on next page}} \\
\midrule
\endfoot

\bottomrule
\endlastfoot
 21195160 &            SUBIA AUTOMATICA &  AUT &     CUNDINAMARCA &          SILVANIA &  4.476611 & -74.383889 \\
 21201200 &     ESC LA UNION AUTOMATICA &  AUT &      BOGOTA D.C. &       BOGOTA D.C. &  4.342944 & -74.183889 \\
 21201580 &        PASQUILLA AUTOMATICA &  AUT &      BOGOTA D.C. &       BOGOTA D.C. &  4.446500 & -74.154833 \\
 21202270 &      PLUVIOMETRO AUTOMATICO &  AUT &      BOGOTA D.C. &       BOGOTA D.C. &  4.608056 & -74.072889 \\
 21202271 &          PLUVIOMETRO AUTOMA &  AUT &      BOGOTA D.C. &       BOGOTA D.C. &  4.608056 & -74.072889 \\
 21205012 &               UNIV NACIONAL &  AUT &      BOGOTA D.C. &       BOGOTA D.C. &  4.638083 & -74.089083 \\
 21205791 &              APTO EL DORADO &  AUT &      BOGOTA D.C. &       BOGOTA D.C. &  4.705583 & -74.150667 \\
 21206600 &            NUEVA GENERACION &  AUT &      BOGOTA D.C. &       BOGOTA D.C. &  4.782222 & -74.094333 \\
 21206710 &      SAN JOAQUIN AUTOMATICA &  AUT &     CUNDINAMARCA &           LA MESA &  4.633333 & -74.516667 \\
 21206790 &           HDA STA ANA AUTOM &  AUT &     CUNDINAMARCA &           NEMOCÓN &  5.090500 & -73.881250 \\
 21206920 &     VILLA TERESA AUTOMATICA &  AUT &      BOGOTA D.C. &       BOGOTA D.C. &  4.350000 & -74.150000 \\
 21206930 &                PMO GUERRERO &  AUT &     CUNDINAMARCA &         ZIPAQUIRÁ &  5.086444 & -74.022167 \\
 21206940 &              CIUDAD BOLIVAR &  AUT &      BOGOTA D.C. &       BOGOTA D.C. &  4.576861 & -74.176778 \\
 21206950 &             PMO GUACHENEQUE &  AUT &     CUNDINAMARCA &       VILLAPINZÓN &  5.236056 & -73.525083 \\
 21206960 &                IDEAM BOGOTA &  AUT &      BOGOTA D.C. &       BOGOTA D.C. &  4.600000 & -74.066667 \\
 21206980 &          STA CRUZ DE SIECHA &  AUT &     CUNDINAMARCA &            GUASCA &  4.784278 & -73.870806 \\
 21206990 &        TIBAITATA AUTOMATICA &  AUT &     CUNDINAMARCA &          MOSQUERA &  4.691417 & -74.209000 \\
 21209920 &            STA ROSITA AUTOM &  AUT &     CUNDINAMARCA &            SUESCA &  5.192250 & -73.779056 \\
 23125170 &          SAN CAYETANO AUTOM &  AUT &     CUNDINAMARCA &      SAN CAYETANO &  4.516753 & -74.088222 \\
 24015110 &        LA BOYERA AUTOMATICA &  AUT &     CUNDINAMARCA &             UBATÉ &  5.303806 & -73.851750 \\
 26127010 &                EL ALAMBRADO &  AUT &  VALLE DEL CAUCA &            ZARZAL &  4.410250 & -74.124611 \\
 35025080 &          PNN CHINGAZA AUTOM &  AUT &     CUNDINAMARCA &         LA CALERA &  4.661000 & -73.827333 \\
 35025090 &          BOSQUE INTERVENIDO &  AUT &     CUNDINAMARCA &         LA CALERA &  4.664889 & -73.846639 \\
 35025100 &              CALOSTROS BAJO &  AUT &     CUNDINAMARCA &         LA CALERA &  4.673778 & -73.818889 \\
 35027001 &             PLAZA DE FERIAS &  AUT &     CUNDINAMARCA &           CÁQUEZA &  4.403389 & -73.940556 \\
 35027002 &         PARQUE RAFAEL NUÑEZ &  AUT &     CUNDINAMARCA &           CÁQUEZA &  4.407417 & -73.947500 \\
 35027510 &              CALOSTROS BAJO &  AUT &     CUNDINAMARCA &         LA CALERA &  4.673778 & -73.818889 \\
 35035130 &                PMO CHINGAZA &  AUT &     CUNDINAMARCA &         LA CALERA &  4.713667 & -73.803250 \\
 35075070 &        CHINAVITA AUTOMATICA &  AUT &           BOYACÁ &         CHINAVITA &  5.219250 & -73.350389 \\
 35075080 &      PMO RABANAL AUTOMATICA &  AUT &           BOYACÁ &      VENTAQUEMADA &  5.392389 & -73.562778 \\
 35085080 &            LA CAPILLA AUTOM &  AUT &           BOYACÁ &        LA CAPILLA &  5.099194 & -73.436000 \\
 21200160 &          PANONIA [21200160] &  CON &     CUNDINAMARCA &          CHOCONTÁ &  5.057972 & -73.734333 \\
 21200620 &           PISCIS [21200620] &  CON &     CUNDINAMARCA &          CHOCONTÁ &  5.079167 & -73.696861 \\
 21200780 &               POTRERO LARGO &  CON &     CUNDINAMARCA &         GUATAVITA &  4.929222 & -73.780472 \\
 21200840 &      FLORESTA LA [21200840] &  CON &     CUNDINAMARCA &            GUASCA &  4.850000 & -73.783333 \\
 21201050 &          LOURDES [21201050] &  CON &     CUNDINAMARCA &        GACHANCIPÁ &  4.982889 & -73.864667 \\
 21201060 &           PANTANO REDONDO 1 &  CON &     CUNDINAMARCA &         ZIPAQUIRÁ &  5.043250 & -74.033389 \\
 21201070 &       CORAZON EL [21201070] &  CON &     CUNDINAMARCA &        FACATATIVÁ &  4.865361 & -74.289417 \\
 21201080 &               SAN FRANCISCO &  CON &     CUNDINAMARCA &              SOPÓ &  4.900000 & -73.950000 \\
 21201140 &     ESPERANZA LA [21201140] &  CON &     CUNDINAMARCA &             TENJO &  4.802167 & -74.179972 \\
 21201160 &             EDIFICIO SARAGA &  CON &      BOGOTA D.C. &       BOGOTA D.C. &  4.600000 & -74.083333 \\
 21201180 &        GUANQUICA [21201180] &  CON &     CUNDINAMARCA &             TAUSA &  5.184278 & -73.941111 \\
 21201190 &        LAGUNITAS [21201190] &  CON &     CUNDINAMARCA &             TAUSA &  5.214528 & -73.907250 \\
 21201210 &        HATO EL   [21201210] &  CON &     CUNDINAMARCA &             TENJO &  4.866389 & -74.153861 \\
 21201220 &          STA CRUZ DE SIECHA &  CON &     CUNDINAMARCA &            GUASCA &  4.784278 & -73.870806 \\
 21201230 &            ENMANUEL D ALZON &  CON &      BOGOTA D.C. &       BOGOTA D.C. &  4.701125 & -74.070306 \\
 21201240 &           STA MARIA DE USME &  CON &      BOGOTA D.C. &       BOGOTA D.C. &  4.481306 & -74.126278 \\
 21201250 &        SAN PEDRO [21201250] &  CON &     CUNDINAMARCA &              SOPÓ &  4.871639 & -73.966667 \\
 21201270 &         TIBAR EL [21201270] &  CON &     CUNDINAMARCA &            MADRID &  4.816667 & -74.233333 \\
 21201300 &        AUSTRALIA [21201300] &  CON &      BOGOTA D.C. &       BOGOTA D.C. &  4.394250 & -74.132000 \\
 21201310 &          PREVENTORIO INFANT &  CON &     CUNDINAMARCA &            SIBATÉ &  4.465444 & -74.267500 \\
 21201320 &         UNION LA [21201320] &  CON &     CUNDINAMARCA &            SIBATÉ &  4.509361 & -74.268806 \\
 21201550 &         ROBLE EL [21201550] &  CON &     CUNDINAMARCA &            MADRID &  4.796667 & -74.226389 \\
 21201570 &          ESC COL INGENIERIA &  CON &      BOGOTA D.C. &       BOGOTA D.C. &  4.783333 & -74.050000 \\
 21201600 &           SEDE IDEAM KRA 10 &  CON &      BOGOTA D.C. &       BOGOTA D.C. &  4.607111 & -74.072889 \\
 21201610 &       SAN ISIDRO [21201610] &  CON &     CUNDINAMARCA &            GUASCA &  4.850289 & -73.890722 \\
 21201620 &        SUESCA    [21201620] &  CON &     CUNDINAMARCA &            SUESCA &  5.109583 & -73.796972 \\
 21201630 &        TABIO GJA [21201630] &  CON &     CUNDINAMARCA &             TABIO &  4.933056 & -74.065611 \\
 21201640 &             VILLAPINZON GJA &  CON &     CUNDINAMARCA &       VILLAPINZÓN &  5.263750 & -73.590861 \\
 21201650 &       STA ROSITA [21201650] &  CON &     CUNDINAMARCA &            SUESCA &  5.115917 & -73.757389 \\
 21201920 &             ALTO SAN MIGUEL &  CON &     CUNDINAMARCA &            SIBATÉ &  4.449667 & -74.299722 \\
 21201930 &        GUANQUICA [21201930] &  CON &     CUNDINAMARCA &             TAUSA &  5.184278 & -73.941111 \\
 21202100 &           IDEAM FONTIBON HB &  CON &      BOGOTA D.C. &       BOGOTA D.C. &  4.700000 & -74.166667 \\
 21202160 &     HIDROPARAISO [21202160] &  CON &     CUNDINAMARCA &        EL COLEGIO &  4.573167 & -74.404833 \\
 21202280 &    SEDE IDEAM CALLE 25D KRA &  CON &      BOGOTA D.C. &       BOGOTA D.C. &  4.684000 & -74.129000 \\
 21205013 &              UNISALLE NORTE &  CON &      BOGOTA D.C. &       BOGOTA D.C. &  4.794444 & -74.030556 \\
 21205230 &            OBS MET NACIONAL &  CON &      BOGOTA D.C. &       BOGOTA D.C. &  4.633333 & -74.100000 \\
 21205420 &        TIBAITATA [21205420] &  CON &     CUNDINAMARCA &          MOSQUERA &  4.691417 & -74.209000 \\
 21205520 &          ELDORADO DIDACTICA &  CON &      BOGOTA D.C. &       BOGOTA D.C. &  4.700000 & -74.150000 \\
 21205580 &           VENADO ORO VIVERO &  CON &      BOGOTA D.C. &       BOGOTA D.C. &  4.598361 & -74.061556 \\
 21205600 &          VELODROMO 1 D MAYO &  CON &      BOGOTA D.C. &       BOGOTA D.C. &  4.616667 & -74.066667 \\
 21205660 &     MERCEDES LAS [21205660] &  CON &     CUNDINAMARCA &          ANAPOIMA &  4.581889 & -74.526611 \\
 21205670 &       FLORIDA LA [21205670] &  CON &     CUNDINAMARCA &          ANOLAIMA &  4.770889 & -74.437639 \\
 21205700 &           GUASCA [21205700] &  CON &     CUNDINAMARCA &            GUASCA &  4.879861 & -73.868111 \\
 21205710 &             JARDIN BOTANICO &  CON &      BOGOTA D.C. &       BOGOTA D.C. &  4.669333 & -74.102667 \\
 21205720 &    SAN JORGE GJA [21205720] &  CON &     CUNDINAMARCA &            SOACHA &  4.505750 & -74.189278 \\
 21205730 &            CENTRO MED ANDES &  CON &      BOGOTA D.C. &       BOGOTA D.C. &  4.698167 & -74.036833 \\
 21205740 &            SILOS [21205740] &  CON &     CUNDINAMARCA &          CHOCONTÁ &  5.117722 & -73.701417 \\
 21205750 &           REP LOS MUCHACHOS &  CON &     CUNDINAMARCA &             FUNZA &  4.733333 & -74.166667 \\
 21205760 &          CLINICA SAN RAFAEL &  CON &      BOGOTA D.C. &       BOGOTA D.C. &  4.600000 & -74.083333 \\
 21205770 &           BASE AEREA MADRID &  CON &     CUNDINAMARCA &            MADRID &  4.728806 & -74.272500 \\
 21205780 &         SENA GJA [21205780] &  CON &     CUNDINAMARCA &          MOSQUERA &  4.700000 & -74.216667 \\
 21205790 &              APTO EL DORADO &  CON &      BOGOTA D.C. &       BOGOTA D.C. &  4.705583 & -74.150667 \\
 21205800 &          BOMBEROS DEL NORTE &  CON &      BOGOTA D.C. &       BOGOTA D.C. &  4.650000 & -74.066667 \\
 21205810 &          CAPITOLIO NACIONAL &  CON &      BOGOTA D.C. &       BOGOTA D.C. &  4.600000 & -74.083333 \\
 21205820 &          LICORERA BOGOINAMA &  CON &      BOGOTA D.C. &       BOGOTA D.C. &  4.616667 & -74.100000 \\
 21205830 &           MUZU CENTRO SALUD &  CON &      BOGOTA D.C. &       BOGOTA D.C. &  4.600000 & -74.133333 \\
 21205840 &        SENA K 30 [21205840] &  CON &      BOGOTA D.C. &       BOGOTA D.C. &  4.595361 & -74.111833 \\
 21205850 &    COLOMBIANO EL [21205850] &  CON &     CUNDINAMARCA &          SESQUILÉ &  5.033889 & -73.848194 \\
 21205860 &         CORZO EL [21205860] &  CON &      BOGOTA D.C. &       BOGOTA D.C. &  4.650000 & -74.200000 \\
 21205870 &       SALITRE EL [21205870] &  CON &     CUNDINAMARCA &            BOJACÁ &  4.738889 & -74.334278 \\
 21205880 &              FLORES CHIBCHA &  CON &     CUNDINAMARCA &            MADRID &  4.789722 & -74.264778 \\
 21205890 &          GUANATA [21205890] &  CON &     CUNDINAMARCA &              CHÍA &  4.885944 & -74.054333 \\
 21205900 &      INDUQUIMICA [21205900] &  CON &     CUNDINAMARCA &            SOACHA &  4.583333 & -74.233333 \\
 21205910 &       COSECHA LA [21205910] &  CON &     CUNDINAMARCA &         ZIPAQUIRÁ &  4.989222 & -74.001194 \\
 21205920 &      SUASUQUE    [21205920] &  CON &     CUNDINAMARCA &              SOPÓ &  4.820833 & -73.963889 \\
 21205930 &       VILLA ROSA [21205930] &  CON &     CUNDINAMARCA &              COTA &  4.833333 & -74.100000 \\
 21205940 &       VILLA INES [21205940] &  CON &     CUNDINAMARCA &        FACATATIVÁ &  4.834972 & -74.383972 \\
 21205950 &              TIBACHOQUE HDA &  CON &     CUNDINAMARCA &             FUNZA &  4.759056 & -74.205167 \\
 21205960 &            TACHI [21205960] &  CON &     CUNDINAMARCA &        SUBACHOQUE &  4.939056 & -74.152583 \\
 21205970 &      STA ANA HDA [21205970] &  CON &     CUNDINAMARCA &           NEMOCÓN &  5.090500 & -73.881250 \\
 21205980 &             PROVIDENCIA GJA &  CON &     CUNDINAMARCA &             TENJO &  4.792389 & -74.200917 \\
 21205990 &         LLANO EL [21205990] &  CON &     CUNDINAMARCA &              SOPÓ &  4.927778 & -73.950000 \\
 21206000 &         ADPOSTAL [21206000] &  CON &      BOGOTA D.C. &       BOGOTA D.C. &  4.680750 & -74.123639 \\
 21206010 &     LORETOKI HDA [21206010] &  CON &     CUNDINAMARCA &            SUESCA &  5.089028 & -73.802750 \\
 21206020 &    SANTILLANA    [21206020] &  CON &     CUNDINAMARCA &             TABIO &  4.898528 & -74.104833 \\
 21206030 &     SAN CAYETANO [21206030] &  CON &     CUNDINAMARCA &        SUBACHOQUE &  4.916833 & -74.181667 \\
 21206040 &             ESAP [21206040] &  CON &      BOGOTA D.C. &       BOGOTA D.C. &  4.646778 & -74.096361 \\
 21206050 &          ESC COL INGENIERIA &  CON &      BOGOTA D.C. &       BOGOTA D.C. &  4.783333 & -74.050000 \\
 21206060 &       CASABLANCA [21206060] &  CON &     CUNDINAMARCA &            MADRID &  4.717111 & -74.253333 \\
 21206150 &           MOLINOS DEL NORTE &  CON &      BOGOTA D.C. &       BOGOTA D.C. &  4.700000 & -74.050000 \\
 21206160 &     HIDROPARAISO [21206160] &  CON &     CUNDINAMARCA &        EL COLEGIO &  4.573167 & -74.404833 \\
 21206170 &       CLARETIANO [21206170] &  CON &      BOGOTA D.C. &       BOGOTA D.C. &  4.600000 & -74.200000 \\
 21206190 &             UNIV PEDAGOGICA &  CON &      BOGOTA D.C. &       BOGOTA D.C. &  4.666667 & -74.066667 \\
 21206200 &          TUNDAMA [21206200] &  CON &     CUNDINAMARCA &          MOSQUERA &  4.733333 & -74.250000 \\
 21206210 &          FLORES COLOMBIANAS &  CON &     CUNDINAMARCA &             FUNZA &  4.736250 & -74.157333 \\
 21206220 &               UNIV NACIONAL &  CON &      BOGOTA D.C. &       BOGOTA D.C. &  4.638083 & -74.089083 \\
 21206230 &               VEGAS LAS HDA &  CON &      BOGOTA D.C. &       BOGOTA D.C. &  4.661667 & -74.151419 \\
 21206240 &             CENTRO GAVIOTAS &  CON &      BOGOTA D.C. &       BOGOTA D.C. &  4.600000 & -74.066667 \\
 21206250 &           CORITO [21206250] &  CON &     CUNDINAMARCA &        FACATATIVÁ &  4.800000 & -74.366667 \\
 21206260 &           C.UNIV.AGROP-UDCA &  CON &      BOGOTA D.C. &       BOGOTA D.C. &  4.798639 & -74.049722 \\
 21206280 &         ACAPULCO [21206280] &  CON &     CUNDINAMARCA &            BOJACÁ &  4.653833 & -74.333056 \\
 21206450 &        TERMOZIPA [21206450] &  CON &     CUNDINAMARCA &         TOCANCIPÁ &  4.983333 & -73.933333 \\
 21206490 &        HATO ALTO [21206490] &  CON &     CUNDINAMARCA &             TENJO &  4.835083 & -74.139917 \\
 21206500 &          COL ABRAHAM LINCOL &  CON &      BOGOTA D.C. &       BOGOTA D.C. &  4.756639 & -74.061583 \\
 21206510 &             CASD [21206510] &  CON &      BOGOTA D.C. &       BOGOTA D.C. &  4.600000 & -74.083333 \\
 21206540 &            EDIFICIO PREMIUM &  CON &      BOGOTA D.C. &       BOGOTA D.C. &  4.686944 & -74.054222 \\
 21206550 &               SENA MOSQUERA &  CON &     CUNDINAMARCA &          MOSQUERA &  4.700000 & -74.216667 \\
 21206560 &     INEM KENNEDY [21206560] &  CON &      BOGOTA D.C. &       BOGOTA D.C. &  4.661111 & -74.134778 \\
 21206570 &       APTO CATAM [21206570] &  CON &      BOGOTA D.C. &       BOGOTA D.C. &  4.705583 & -74.150667 \\
 21206610 &            EFRAIN CA\#AVERAL &  CON &      BOGOTA D.C. &       BOGOTA D.C. &  4.583333 & -74.066667 \\
 21206620 &           COL H DURAN DUSAN &  CON &      BOGOTA D.C. &       BOGOTA D.C. &  4.634611 & -74.173750 \\
 21206630 &          BILBAO MAXIMO POTI &  CON &      BOGOTA D.C. &       BOGOTA D.C. &  4.751139 & -74.091583 \\
 21206640 &         SAN JOSE [21206640] &  CON &      BOGOTA D.C. &       BOGOTA D.C. &  4.501556 & -74.119306 \\
 21206650 &            COL SAN CAYETANO &  CON &      BOGOTA D.C. &       BOGOTA D.C. &  4.516753 & -74.088222 \\
 21206660 &          COL SANTIAGO PEREZ &  CON &      BOGOTA D.C. &       BOGOTA D.C. &  4.576222 & -74.130917 \\
 21206670 &              COL BUCKINGHAM &  CON &      BOGOTA D.C. &       BOGOTA D.C. &  4.792056 & -74.049583 \\
 21206680 &            COL NUEVO RETIRO &  CON &      BOGOTA D.C. &       BOGOTA D.C. &  4.734111 & -74.037028 \\
 21206690 &          COL MIGUEL A. CARO &  CON &      BOGOTA D.C. &       BOGOTA D.C. &  4.813167 & -74.031111 \\
 21206700 &          CEA CENT.EST.AERO. &  CON &      BOGOTA D.C. &       BOGOTA D.C. &  4.691028 & -74.134417 \\
 21206970 &             UNISALLE CENTRO &  CON &      BOGOTA D.C. &       BOGOTA D.C. &  4.595000 & -74.070361 \\
 21208670 &        GUANQUICA [21208670] &  CON &     CUNDINAMARCA &             TAUSA &  5.184278 & -73.941111 \\
 23065100 &         SABANETA [23065100] &  CON &     CUNDINAMARCA &     SAN FRANCISCO &  4.901750 & -74.307389 \\
 24010070 &          LETICIA [24010070] &  CON &     CUNDINAMARCA &       LENGUAZAQUE &  5.303194 & -73.709750 \\
 24010140 &         CUCUNUBA [24010140] &  CON &     CUNDINAMARCA &          CUCUNUBÁ &  5.251028 & -73.770750 \\
 24010170 &         GUACHETA [24010170] &  CON &     CUNDINAMARCA &          GUACHETÁ &  5.385889 & -73.691056 \\
 24010610 &            CARMEN DE CARUPA &  CON &     CUNDINAMARCA &  CARMEN DE CARUPA &  5.351278 & -73.904472 \\
 24010800 &        MINAS LAS [24010800] &  CON &           BOYACÁ &            SAMACÁ &  5.483333 & -73.533333 \\
 24011060 &             SUSA [24011060] &  CON &     CUNDINAMARCA &              SUSA &  5.462444 & -73.801556 \\
 24011080 &       CUCUNUBA 1 [24011080] &  CON &     CUNDINAMARCA &          CUCUNUBÁ &  5.248000 & -73.752500 \\
 24011090 &        UBATE GJA [24011090] &  CON &     CUNDINAMARCA &             UBATÉ &  5.327333 & -73.791444 \\
 24011100 &          ISLA DEL SANTUARIO &  CON &     CUNDINAMARCA &           FÚQUENE &  5.467278 & -73.734806 \\
 24011150 &       ALIZOS LOS [24011150] &  CON &     CUNDINAMARCA &  CARMEN DE CARUPA &  5.329389 & -73.850056 \\
 24015120 &          ISLA DEL SANTUARIO &  CON &     CUNDINAMARCA &           FÚQUENE &  5.467278 & -73.734806 \\
 24015220 &   VILLA CARMEN   [24015220] &  CON &           BOYACÁ &            SAMACÁ &  5.509389 & -73.495778 \\
 24015290 &        GACHANECA [24015290] &  CON &           BOYACÁ &            SAMACÁ &  5.433333 & -73.550000 \\
 24015380 &            CARMEN DE CARUPA &  CON &     CUNDINAMARCA &  CARMEN DE CARUPA &  5.347222 & -73.898333 \\
 24017150 &             LA BOYERA AUTOM &  CON &     CUNDINAMARCA &             UBATÉ &  5.305972 & -73.855444 \\
 24017240 &         ANCON EL [24017240] &  CON &           BOYACÁ &            SAMACÁ &  5.466667 & -73.533333 \\
 24017610 &      BOQUERON    [24017610] &  CON &     CUNDINAMARCA &       LENGUAZAQUE &  5.328250 & -73.699722 \\
 24017630 &        GACHANECA [24017630] &  CON &           BOYACÁ &            SAMACÁ &  5.450000 & -73.550000 \\
 24017660 &          CANAL RUCHICAL BOC &  CON &           BOYACÁ &            SAMACÁ &  5.483333 & -73.516667 \\
 24017670 &      AMARILLO EL [24017670] &  CON &           BOYACÁ &            SAMACÁ &  5.500000 & -73.533333 \\
 24017680 &       REBOSADERO [24017680] &  CON &           BOYACÁ &            SAMACÁ &  5.450000 & -73.533333 \\
 24017690 &              SALIDA EMBALSE &  CON &           BOYACÁ &            SAMACÁ &  5.450000 & -73.533333 \\
 24017700 &           PTE EL [24017700] &  CON &           BOYACÁ &            SAMACÁ &  5.456167 & -73.540111 \\
 24017720 &               CANAL PATAGUY &  CON &           BOYACÁ &            SAMACÁ &  5.450000 & -73.500000 \\
 24017730 &                CUATRO COPAS &  CON &           BOYACÁ &            SAMACÁ &  5.466667 & -73.533333 \\
 35020280 &          CHOACHI [35020280] &  CON &     CUNDINAMARCA &           CHOACHÍ &  4.522917 & -73.926583 \\
 35020290 &          FOMEQUE [35020290] &  CON &     CUNDINAMARCA &           FÓMEQUE &  4.486528 & -73.890417 \\
 35020320 &      LLANO LARGO [35020320] &  CON &     CUNDINAMARCA &            UBAQUE &  4.482833 & -74.030278 \\
 35020330 &         BOLSA LA [35020330] &  CON &     CUNDINAMARCA &           CHOACHÍ &  4.575417 & -73.981417 \\
 35025050 &      LLANO LARGO [35025050] &  CON &     CUNDINAMARCA &            UBAQUE &  4.482833 & -74.030278 \\
 35025060 &         BOLSA LA [35025060] &  CON &     CUNDINAMARCA &           CHOACHÍ &  4.575417 & -73.981417 \\
 35027100 &           CARAZA [35027100] &  CON &     CUNDINAMARCA &          CHIPAQUE &  4.428639 & -74.010194 \\
 35027220 &      LLANO LARGO [35027220] &  CON &     CUNDINAMARCA &            UBAQUE &  4.485056 & -74.030222 \\
 35027500 &             QDA RINCONAUTOM &  CON &     CUNDINAMARCA &         LA CALERA &  4.664944 & -73.857389 \\
 35030080 &        CASAS LAS [35030080] &  CON &     CUNDINAMARCA &           CÁQUEZA &  4.441167 & -73.936389 \\
 35060020 &            SUEVA [35060020] &  CON &     CUNDINAMARCA &             JUNÍN &  4.810000 & -73.707167 \\
 35060160 &       POTRERITOS [35060160] &  CON &     CUNDINAMARCA &         GUATAVITA &  4.828806 & -73.769278 \\
 35060200 &     AMOLADERO EL [35060200] &  CON &     CUNDINAMARCA &         GUATAVITA &  4.857972 & -73.745389 \\
 35070020 &  VENTAQUEMADA    [35070020] &  CON &           BOYACÁ &      VENTAQUEMADA &  5.383056 & -73.602889 \\
 35070030 &      TURMEQUE    [35070030] &  CON &           BOYACÁ &          TURMEQUÉ &  5.317750 & -73.496361 \\
 35070040 &           TIBANA [35070040] &  CON &           BOYACÁ &            TIBANÁ &  5.315278 & -73.395944 \\
 35070050 &           UMBITA [35070050] &  CON &           BOYACÁ &            ÚMBITA &  5.219111 & -73.444556 \\
 35070060 &   QUINCHOS LOS   [35070060] &  CON &           BOYACÁ &         CHINAVITA &  5.219556 & -73.347917 \\
 35070070 &     CHINAVITA    [35070070] &  CON &           BOYACÁ &         CHINAVITA &  5.164861 & -73.364250 \\
 35070210 &        PACHAVITA [35070210] &  CON &           BOYACÁ &         PACHAVITA &  5.139250 & -73.395639 \\
 35070230 &           MACHETA GJA AGROP &  CON &     CUNDINAMARCA &           MACHETÁ &  5.075111 & -73.579417 \\
 35070370 &            BELEN [35070370] &  CON &     CUNDINAMARCA &           MACHETÁ &  5.083333 & -73.566667 \\
 35070380 &      ROSALES LOS [35070380] &  CON &     CUNDINAMARCA &           MACHETÁ &  5.083333 & -73.616667 \\
 35075010 &    NUEVO COLON   [35075010] &  CON &           BOYACÁ &       NUEVO COLÓN &  5.352694 & -73.453778 \\
 35077100 &       SAN JOSE   [35077100] &  CON &           BOYACÁ &            SAMACÁ &  5.428639 & -73.528278 \\
\end{longtable}

\section{Cosas que sobran}


La zona de estudio es la Sabana de Bogotá. En la Sabana de Bogotá las variaciones de temperatura afectan los cultivos tales como la papa, pastos, maíz, hortalizas entre otros. Los principales limitantes de caracter agometeorológico en la producción de papa son principalmente eventos de heladas y estrés por déficit hídrico, los cuales pueden producir pérdidas de hasta el 75\% \citep{DANE2002}.\\

En las últimas décadas se documentó un incremento de las temperaturas medias a escala global de aproximadamente 0,15\celc por década \citep{jones2001} 

\section{Comandos para la corrección del documento}
%% Cómo hacer comentarios

%% En el caso que quieran quitar algo y quiera tacharlo
Esta es una palabra repetida \sout{repetida}, por ejemplo. \\

%% En el caso que se quiera resaltar algo
Esto es \emph{muy importante} por favor tener en cuenta.

%% En el caso que se quiera hacer una nota al margen
\todo{De esta forma se hacen notas en el margen}

%% En el caso que quiera adicionar algo al texto. Puede seleccionar un color y adicionar los comentarios

\textcolor{red}{Esta es una nueva idea}









%######### Esto fue tomado del capítlo 1 y 3 
% * <etorresm@unal.edu.co> 2018-09-24T23:44:37.716Z:
%
% > %######### Esto fue tomado del capítlo 1 y 3 
% > \subsection{El modelo Weather Research and Forecasting Model (WRF)}
% > El modelo WRF es un modelo muy usado al rededor del mundo gracias a los buenos resultados obtenidos, como lo reporta \citet{Jimenez2014}. El modelo WRF es un sistema de cálculo numérico para simulación atmosférica que fue diseñado para cumplir objetivos de investigación y pronóstico, este modelo sirve en un amplio rango de escalas espaciales, desde decenas de metros hasta miles de kilómetros. Los usuarios de este modelo pueden producir simulaciones basadas en las condiciones atmosféricas actuales o condiciones idealizadas \citep{Pielke2002}. WRF es capaz de realizar una reducción de escala de un modelo de pronóstico global como GFS. La reducción de escala toma las condiciones del modelo global y le aumenta la resolución teniendo en cuenta las características de la zona de estudio \citep{Sene2010}.\\
% > El modelo WRF tiene una aproximación no-hidrostática, esto quiere decir que tiene en cuenta el momento en en la dirección vertical $(w)$, en comparación con el modelo hidrostático el cual no tiene en cuenta los cambios en el momento en la dirección vertical $(w)$. Los modelos no-hidrostáticos son usado para el pronóstico de fenómenos de mesoescala o escalas menores \citep{Pielke2002, Sene2010}.\\
% > %. El modelo hidrostático supone una homogeneidad en la columna de aire y está dado por la densidad y la gravedad \citep{Pielke2002, Sene2010}.
% > En el ámbito internacional el modelo WRF se ha usado en varios países como es el caso de Perú, \citet{Saavedra2016} realizó un trabajo de modelación de fenómenos meteorológicos de mesoescala. Como resultado se obtuvo que la modelación reproduce de buena forma el ciclo diario de las temperaturas del aire a dos metros, pero las temperaturas mínimas del aire a dos metros fueron sobrestimadas en las partes altas de la zona de estudio, y subestimó la tasa de enfriamiento en el fondo de los valles, generando una mayor temperatura modelada, respecto a los valores reportados por estaciones ubicadas en las laderas.\\
% > \citet{Fernandez2011} realizaron un estudio en Argentina para la cuidad de Mendoza con tres dominios espaciales de 36, 12 y 4 km donde se tuvieron en cuenta las condiciones orográficas para la delimitación de los mismos. Se usó el esquema de capa límite planetaria de \textit{Yonsei University}. Como resultado se encontró que los valores máximos tanto de temperatura como de humedad son predichos correctamente.\\
% > \citet{Corrales2015a} usó el modelo WRF para realizar un pronóstico de las temperaturas en México usando un único dominio y con la parametrización \textit{Mellor-Yamada-Janic} para la capa límite planetaria. Y como resultado obtuvo que hay algunas zonas donde el modelo es confiable para la predicción de las temperatura, lo cual puede prevenir daños por heladas en un horizonte pronóstico de 5 días.\\
% > \citet{Hu2010} usaron el modelo WRF para modelar las temperaturas en el centro de Estados Unidos, en los estados de Texas, Lousiana y parte de Arkansas. Ellos probaron las diferentes parametrizaciones de capa límite (PLB) y, encontraron que algunas parametrizaciones producen temperaturas más altas de lo esperado.\\
% > \citet{Parra2012} realizó un estudio para Ecuador donde la finalidad era simular la meteorología de un año completo de todo el país. Para esto usaron el modelo WRF con 2 dominios y con la parametrización Mellor Yamada Jajic (MYJ), y se obtuvo como resultado que las temperaturas simuladas son coherentes con los fenómenos observados en las estaciones en tierra.\\
% > Basado en estas investigaciones se evidencia la necesidad de probar varias parametrizaciones e intentar lograr la mejor combinación de ellas. Para el caso de Colombia \citet{Uribe2012} escogió 10 eventos de duración de un día. Se probaron dos parametrizaciones diferentes , con dos diferentes resoluciones espaciales y dos diferentes horizontes de pronóstico con la finalidad de encontrar la mejor combinación. Como resultado encontró que la parametrización por el método de Kain-Fritschc con una resolución de 20 km de grilla presentó la mayor subestimación para la precipitación, mientras que la parametrización de Morrison presenta los mejores resultados. Y al aumentar el horizonte de pronóstico de 36 a 48 horas se logra una mejor simulación de los valores de precipitación \citep{Uribe2012}.\\
% > En Colombia, el Instituto de Hidrología, Meteorología y Estudios Ambientales (IDEAM) ha implementado el modero WRF desde el año 2007 para la predicción del tiempo atmosférico convirtiéndose en una herramienta muy importante para esta entidad \citep{Arango2011}. Se han realizado validaciones del modelo WRF en la Sabana de Bogotá para la variable precipitación mediante una comparación con las estaciones convencionales, como la realizada por \citet{Mejia2012}. El objetivo fue identificar y establecer cuál de los modelos operacionales del IDEAM lograba identificar de manera aceptable los patrones de comportamiento de las variables de precipitación. Se encontró que el modelo WRF alimentado con los datos del modelo \textit{Global Forecast System} (GFS) presentó los mejores resultados. Este estudio presenta una metodología útil para la validación de los modelos y la determinación del mejor modelo.\\
% > %Estos estudios fueron realizados en diferentes cultivos como \citet{prabha2008} que usó el cultivo de durazno, \citep{Gomez2014} quien realizó su estudio en pasturas, \citep{Saavedra2016} quien estudio en una zona productora de papa y maíz.\\
% > %Es importante resaltar que el modelo toma como referencia datos del suelo. Pero según \citet{Castro-Romero2014} en la Región Andina se presentan cambios por el uso del paisaje que dejan como resultado tan solo el 31\% de los bosques naturales y una degradación del 53\% en arbustales secos. Se estima que para el año 1998 el 69\% de los bosques andinos habían sido talados. Uno de estos sucede en Suesca - Cundinamarca \citet{Castro-Romero2014} reporta que debido al uso agropecuario intensivo que se le ha dado a los suelos es posible observar zonas desprovistas de cobertura vegetal y de suelo con estados muy avanzados de degradación, lo cual imposibilita su posible recuperación en años próximos. Ya que los ecosistemas se usan de una forma extractiva como,lo menciona \citet{Ernesto}, uno de los ecosistemas más importantes de Colombia el Páramo es usado como una mina del que se extrae carbón, materia orgánica de los suelos y biomasa de sus páramos. Pero adicionalmente \citep{Ernesto} cuantificó la tasa a la que se extrae y la estimó en $2.49e-20 \frac{seJ}{year}$. Estos cambios nos hacen reflexionar acerca de la importancia del mantenimiento de los suelos y que se deben hacer ajustes periódicos a los modelos ya que con estos estudios se demuestra que el suelo es dinámico.
% > %Un modelo de mesoescala es un modelo numérico de predicción del tiempo atmosférico, que se usa para hacer una predicción a escala de kilómetros y horas, basado en la dinámica atmosférica \citep{Uribe2012}.\\
% > %\section{Conclusiones}
% > %El cultivo de papa es un cultivo de gran importancia en el país. El consumo interno del país es abastecido casi en su mayoría por la producción interna. Este cultivo es seriamente afectado por muchos factores agroclimáticos y uno de los más importantes es la temperatura extrema. Cundinamarca es el Departamento con mayor producción en el país y la Sabana de Bogotá es la zona de Cundinamarca que más influencia tiene en la producción de papa, por sus condiciones topográficas de sabana se convierte en una altamente suceptible a heladas. Los meses con mayor probabilidad de heladas son diciembre, enero y febrero; y la hora en la que más se presentan bajas temperaturas es a las 5 am. Las temperaturas más altas se presentan en la mayoría de los meses pero principalmente entre los meses de diciembre y abril y las horas en las que más se presentan es a las 12 m.\\
% > %Basado en los registros de las estaciones convencionales desde 1971 hasta 2016, se evidenció un aumento de la temperatura en 1.34\celc. Adicionalmente se evidencian cambios en la frecuencia de las bajas y las altas temperaturas, ya que la frecuencia de las bajas temperaturas ha venido disminuyendo y las altas temperaturas han presentado un aumento en los valores registrados.
% > %De los modelos de pronóstico revisados el modelo WRF un modelo de regional de pronóstico es el que presenta las mejores características para realizar el estudio de las temperaturas, ya que es un modelo que tiene en cuenta aspectos físicos, dinámicos y su evolución, además es un modelo que ha sido probado y es usado actualmente en Colombia.
% > %%%%%%%%%%%%%%%%%%%%%%%%%%%%%%%%%%%%%%%%%%%%%%%%%%%%%%%%%%%%%%%%%%%%%%%%%%%%%%%%%
% > %%%%%%%%%%%%%%%%%%%%%%%%%%%%%%%%%%%%%%%%%%%%%%%%%%%%%%%%%%%%%%%%%%%%%%%%%%%%%%%%%
% > Se realizó una revisión bibliográfica en los previos capítulos con la finalidad de determinar el mejor modelo para realizar el pronóstico de las temperaturas extremas. Para facilitar las fortalezas y las limitaciones se realizó una comparación , ver tabla \ref{tab:fort_deb_mod}.
% > \begin{table}[H]
% > \resizebox{\textwidth}{!}{\begin{tabular}{p{5cm}| p{5cm} p{5cm} p{5cm}}
% > Modelo                                        & Fortaleza                                                  & Limitación                            &Fuente                                                   \\ \hline
% > \multirow{2}{*}{Modelos empíricos}            & Fácil aplicación                                                               & No posee alta precisión                                   & \citep{Gomez2014, Allen1957, Snyder2010}                   \\
% >                                               & Se puede hacer modificaciones al modelo de manera sencilla                     & Los modelos son creados para determinadas condiciones     &                                                                              \\ \hline
% > \multirow{2}{*}{Balance de energía del suelo} & Fácil aplicación siempre y cuando se tengan todas las variables                & Alta incertidumbre                                        & \citep{evett2011water,Rosenzweig2014, Rossi2002}          \\
% >                                               &                                                                                & Las variables no son fáciles de calcular                  &                                                                              \\ \hline
% > Sistemas de información geográfico            & Existe una clara relación entre los valles y las temperaturas extremas         & No  tiene en cuenta los flujos de radiación de onda larga & \citep{evett2011water, Halley2003, Blennow1998}              \\ \hline
% > Redes neuronales                              & Buenas predicciones predicciones en un corto horizonte de pronóstico (6 horas) & Pronóstico a un horizonte muy corto                       & \citep{Smith2007}                                           \\ \hline
% > Modelo \textit{Weather Research and Forecasting Model} (WRF)                   & Buenas predicciones en un plazo de 2 días                                      & Alto gasto computacional                                  & \citep{prabha2008evaluation, Arango2011, Mejia2012, Ruiz2014} \\
% >                                               &                                                                                & Es necesario poseer una alta capacidad de almacenamiento  &                                                                              \\ \hline
% > \end{tabular}}
% > \caption{Tabla resumen de las ventajas y desventajas de los modelos}
% > \label{tab:fort_deb_mod}
% > \end{table}
% > Basado en la información de la tabla \ref{tab:fort_deb_mod} y teniendo en cuenta las características de cada uno de los modelos se decidió usar el Modelo WRF, ya que es un modelo que cumple con los requerimientos para este estudio y en el país se ha usado este modelo con resultados satisfactorios como los obtenidos por \citep{Mejia2012,Arango2011,Arango2014,Ruiz2014,Uribe2012,RojasA2011,ArmentaPorras2013}.\\
% > Los pasos a seguir para realizar una modelación con el modelo WRF consiste en:
% > \begin{enumerate}
% > \item Compilar el WPS
% > \item Compilar el WRF
% > \item Determinar el área de estudio
% > \item Selección de los dominios del modelo WRF para la inicializacion del modelo
% > \item Selección de las parametrizaciones que se van a emplear
% > \item Descargar los datos que van a alimentar el modelo
% > \item Realizar el pre-proceso con el WPS
% > \item Realizar el proceso de modelación con el WRF
% > \item Realización del pos-proceso con python3.6
% > \end{enumerate}
%
% ^.


\subsection{El modelo Weather Research and Forecasting Model (WRF)}

El modelo WRF es un modelo muy usado al rededor del mundo gracias a los buenos resultados obtenidos, como lo reporta \citet{Jimenez2014}. El modelo WRF es un sistema de cálculo numérico para simulación atmosférica que fue diseñado para cumplir objetivos de investigación y pronóstico, este modelo sirve en un amplio rango de escalas espaciales, desde decenas de metros hasta miles de kilómetros. Los usuarios de este modelo pueden producir simulaciones basadas en las condiciones atmosféricas actuales o condiciones idealizadas \citep{Pielke2002}. WRF es capaz de realizar una reducción de escala de un modelo de pronóstico global como GFS. La reducción de escala toma las condiciones del modelo global y le aumenta la resolución teniendo en cuenta las características de la zona de estudio \citep{Sene2010}.\\

El modelo WRF tiene una aproximación no-hidrostática, esto quiere decir que tiene en cuenta el momento en en la dirección vertical $(w)$, en comparación con el modelo hidrostático el cual no tiene en cuenta los cambios en el momento en la dirección vertical $(w)$. Los modelos no-hidrostáticos son usado para el pronóstico de fenómenos de mesoescala o escalas menores \citep{Pielke2002, Sene2010}.\\
%. El modelo hidrostático supone una homogeneidad en la columna de aire y está dado por la densidad y la gravedad \citep{Pielke2002, Sene2010}.

En el ámbito internacional el modelo WRF se ha usado en varios países como es el caso de Perú, \citet{Saavedra2016} realizó un trabajo de modelación de fenómenos meteorológicos de mesoescala. Como resultado se obtuvo que la modelación reproduce de buena forma el ciclo diario de las temperaturas del aire a dos metros, pero las temperaturas mínimas del aire a dos metros fueron sobrestimadas en las partes altas de la zona de estudio, y subestimó la tasa de enfriamiento en el fondo de los valles, generando una mayor temperatura modelada, respecto a los valores reportados por estaciones ubicadas en las laderas.\\

\citet{Fernandez2011} realizaron un estudio en Argentina para la cuidad de Mendoza con tres dominios espaciales de 36, 12 y 4 km donde se tuvieron en cuenta las condiciones orográficas para la delimitación de los mismos. Se usó el esquema de capa límite planetaria de \textit{Yonsei University}. Como resultado se encontró que los valores máximos tanto de temperatura como de humedad son predichos correctamente.\\

\citet{Corrales2015a} usó el modelo WRF para realizar un pronóstico de las temperaturas en México usando un único dominio y con la parametrización \textit{Mellor-Yamada-Janic} para la capa límite planetaria. Y como resultado obtuvo que hay algunas zonas donde el modelo es confiable para la predicción de las temperatura, lo cual puede prevenir daños por heladas en un horizonte pronóstico de 5 días.\\

\citet{Hu2010} usaron el modelo WRF para modelar las temperaturas en el centro de Estados Unidos, en los estados de Texas, Lousiana y parte de Arkansas. Ellos probaron las diferentes parametrizaciones de capa límite (PLB) y, encontraron que algunas parametrizaciones producen temperaturas más altas de lo esperado.\\

\citet{Parra2012} realizó un estudio para Ecuador donde la finalidad era simular la meteorología de un año completo de todo el país. Para esto usaron el modelo WRF con 2 dominios y con la parametrización Mellor Yamada Jajic (MYJ), y se obtuvo como resultado que las temperaturas simuladas son coherentes con los fenómenos observados en las estaciones en tierra.\\


Basado en estas investigaciones se evidencia la necesidad de probar varias parametrizaciones e intentar lograr la mejor combinación de ellas. Para el caso de Colombia \citet{Uribe2012} escogió 10 eventos de duración de un día. Se probaron dos parametrizaciones diferentes , con dos diferentes resoluciones espaciales y dos diferentes horizontes de pronóstico con la finalidad de encontrar la mejor combinación. Como resultado encontró que la parametrización por el método de Kain-Fritschc con una resolución de 20 km de grilla presentó la mayor subestimación para la precipitación, mientras que la parametrización de Morrison presenta los mejores resultados. Y al aumentar el horizonte de pronóstico de 36 a 48 horas se logra una mejor simulación de los valores de precipitación \citep{Uribe2012}.\\

En Colombia, el Instituto de Hidrología, Meteorología y Estudios Ambientales (IDEAM) ha implementado el modero WRF desde el año 2007 para la predicción del tiempo atmosférico convirtiéndose en una herramienta muy importante para esta entidad \citep{Arango2011}. Se han realizado validaciones del modelo WRF en la Sabana de Bogotá para la variable precipitación mediante una comparación con las estaciones convencionales, como la realizada por \citet{Mejia2012}. El objetivo fue identificar y establecer cuál de los modelos operacionales del IDEAM lograba identificar de manera aceptable los patrones de comportamiento de las variables de precipitación. Se encontró que el modelo WRF alimentado con los datos del modelo \textit{Global Forecast System} (GFS) presentó los mejores resultados. Este estudio presenta una metodología útil para la validación de los modelos y la determinación del mejor modelo.\\



%Estos estudios fueron realizados en diferentes cultivos como \citet{prabha2008} que usó el cultivo de durazno, \citep{Gomez2014} quien realizó su estudio en pasturas, \citep{Saavedra2016} quien estudio en una zona productora de papa y maíz.\\





%Es importante resaltar que el modelo toma como referencia datos del suelo. Pero según \citet{Castro-Romero2014} en la Región Andina se presentan cambios por el uso del paisaje que dejan como resultado tan solo el 31\% de los bosques naturales y una degradación del 53\% en arbustales secos. Se estima que para el año 1998 el 69\% de los bosques andinos habían sido talados. Uno de estos sucede en Suesca - Cundinamarca \citet{Castro-Romero2014} reporta que debido al uso agropecuario intensivo que se le ha dado a los suelos es posible observar zonas desprovistas de cobertura vegetal y de suelo con estados muy avanzados de degradación, lo cual imposibilita su posible recuperación en años próximos. Ya que los ecosistemas se usan de una forma extractiva como,lo menciona \citet{Ernesto}, uno de los ecosistemas más importantes de Colombia el Páramo es usado como una mina del que se extrae carbón, materia orgánica de los suelos y biomasa de sus páramos. Pero adicionalmente \citep{Ernesto} cuantificó la tasa a la que se extrae y la estimó en $2.49e-20 \frac{seJ}{year}$. Estos cambios nos hacen reflexionar acerca de la importancia del mantenimiento de los suelos y que se deben hacer ajustes periódicos a los modelos ya que con estos estudios se demuestra que el suelo es dinámico.

%Un modelo de mesoescala es un modelo numérico de predicción del tiempo atmosférico, que se usa para hacer una predicción a escala de kilómetros y horas, basado en la dinámica atmosférica \citep{Uribe2012}.\\


%\section{Conclusiones}

%El cultivo de papa es un cultivo de gran importancia en el país. El consumo interno del país es abastecido casi en su mayoría por la producción interna. Este cultivo es seriamente afectado por muchos factores agroclimáticos y uno de los más importantes es la temperatura extrema. Cundinamarca es el Departamento con mayor producción en el país y la Sabana de Bogotá es la zona de Cundinamarca que más influencia tiene en la producción de papa, por sus condiciones topográficas de sabana se convierte en una altamente suceptible a heladas. Los meses con mayor probabilidad de heladas son diciembre, enero y febrero; y la hora en la que más se presentan bajas temperaturas es a las 5 am. Las temperaturas más altas se presentan en la mayoría de los meses pero principalmente entre los meses de diciembre y abril y las horas en las que más se presentan es a las 12 m.\\

%Basado en los registros de las estaciones convencionales desde 1971 hasta 2016, se evidenció un aumento de la temperatura en 1.34\celc. Adicionalmente se evidencian cambios en la frecuencia de las bajas y las altas temperaturas, ya que la frecuencia de las bajas temperaturas ha venido disminuyendo y las altas temperaturas han presentado un aumento en los valores registrados.

%De los modelos de pronóstico revisados el modelo WRF un modelo de regional de pronóstico es el que presenta las mejores características para realizar el estudio de las temperaturas, ya que es un modelo que tiene en cuenta aspectos físicos, dinámicos y su evolución, además es un modelo que ha sido probado y es usado actualmente en Colombia.
%%%%%%%%%%%%%%%%%%%%%%%%%%%%%%%%%%%%%%%%%%%%%%%%%%%%%%%%%%%%%%%%%%%%%%%%%%%%%%%%%
%%%%%%%%%%%%%%%%%%%%%%%%%%%%%%%%%%%%%%%%%%%%%%%%%%%%%%%%%%%%%%%%%%%%%%%%%%%%%%%%%




Se realizó una revisión bibliográfica en los previos capítulos con la finalidad de determinar el mejor modelo para realizar el pronóstico de las temperaturas extremas. Para facilitar las fortalezas y las limitaciones se realizó una comparación , ver tabla \ref{tab:fort_deb_mod}.



\begin{table}[H]
\resizebox{\textwidth}{!}{\begin{tabular}{p{5cm}| p{5cm} p{5cm} p{5cm}}
Modelo                                        & Fortaleza                                                  & Limitación                            &Fuente                                                   \\ \hline
\multirow{2}{*}{Modelos empíricos}            & Fácil aplicación                                                               & No posee alta precisión                                   & \citep{Gomez2014, Allen1957, Snyder2010}                   \\
                                              & Se puede hacer modificaciones al modelo de manera sencilla                     & Los modelos son creados para determinadas condiciones     &                                                                              \\ \hline
\multirow{2}{*}{Balance de energía del suelo} & Fácil aplicación siempre y cuando se tengan todas las variables                & Alta incertidumbre                                        & \citep{evett2011water,Rosenzweig2014, Rossi2002}          \\
                                              &                                                                                & Las variables no son fáciles de calcular                  &                                                                              \\ \hline
Sistemas de información geográfico            & Existe una clara relación entre los valles y las temperaturas extremas         & No  tiene en cuenta los flujos de radiación de onda larga & \citep{evett2011water, Halley2003, Blennow1998}              \\ \hline
Redes neuronales                              & Buenas predicciones predicciones en un corto horizonte de pronóstico (6 horas) & Pronóstico a un horizonte muy corto                       & \citep{Smith2007}                                           \\ \hline
Modelo \textit{Weather Research and Forecasting Model} (WRF)                   & Buenas predicciones en un plazo de 2 días                                      & Alto gasto computacional                                  & \citep{prabha2008evaluation, Arango2011, Mejia2012, Ruiz2014} \\
                                              &                                                                                & Es necesario poseer una alta capacidad de almacenamiento  &                                                                              \\ \hline


\end{tabular}}
\caption{Tabla resumen de las ventajas y desventajas de los modelos}
\label{tab:fort_deb_mod}
\end{table}

Basado en la información de la tabla \ref{tab:fort_deb_mod} y teniendo en cuenta las características de cada uno de los modelos se decidió usar el Modelo WRF, ya que es un modelo que cumple con los requerimientos para este estudio y en el país se ha usado este modelo con resultados satisfactorios como los obtenidos por \citep{Mejia2012,Arango2011,Arango2014,Ruiz2014,Uribe2012,RojasA2011,ArmentaPorras2013}.\\

Los pasos a seguir para realizar una modelación con el modelo WRF consiste en:
\begin{enumerate}
\item Compilar el WPS
\item Compilar el WRF
\item Determinar el área de estudio
\item Selección de los dominios del modelo WRF para la inicializacion del modelo
\item Selección de las parametrizaciones que se van a emplear
\item Descargar los datos que van a alimentar el modelo
\item Realizar el pre-proceso con el WPS
\item Realizar el proceso de modelación con el WRF
\item Realización del pos-proceso con python3.6


\end{enumerate}




%% Tabla del porcentaje de coincidencias entre la precipitación horaria y diária


\begin{table}[H]
\begin{center}
\caption{Porcentaje de coincidencia de los valores de precipitación horaria y el valor de la suma de la precipitación instantánea cada 10 minutos de un día.}
\begin{tabular}{llr}
\toprule
Estación &  Porcentaje de coincidencia (\%) \\
\midrule
0  &  21195160 &     24.85 \\
1  &  21201200 &     15.17 \\
2  &  21201580 &     33.71 \\
3  &  21202270 &     25.42 \\
4  &  21205012 &     18.12 \\
5  &  21205791 &     39.62 \\
6  &  21206600 &     39.13 \\
7  &  21206790 &     20.23 \\
8  &  21206920 &     13.25 \\
9  &  21206930 &     13.99 \\
10 &  21206940 &     20.32 \\
11 &  21206950 &     13.75 \\
12 &  21206960 &     47.23 \\
13 &  21206980 &     27.37 \\
14 &  21206990 &     61.43 \\
15 &  21209920 &     43.06 \\
16 &  23125170 &     27.50 \\
17 &  24015110 &     34.88 \\
18 &  26127010 &     27.35 \\
19 &  35025080 &     14.69 \\
20 &  35025090 &     20.11 \\
21 &  35027001 &     34.07 \\
22 &  35027002 &      0.00 \\
\bottomrule
\end{tabular}

\label{table:compar-porcentaje}
\end{center}
\end{table}

\begin{figure}[H]
    \centering
    \includegraphics[draft=false, scale=0.1]{altas_bajas_2007/350850800Simulacion_10d03.png}
    \caption{Caption}
    \label{fig:my_label}
\end{figure}

\begin{figure}[H]
    \centering
    \includegraphics[draft=false, scale=0.1]{altas_bajas_2007/350850800Simulacion_10d03.png}
    \caption{Caption}
    \label{fig:my_label}
\end{figure}



\begin{comment}


%%%%%%%%%%%%%%%%%%%%%%%%%%%%%%%%%%%%%%%%%%%%%%%%%%%%%%%%%%%%%%%%%%%%%%%%%%%%%%%%%
%%%%%%%%%%%%%%%%%%%%%%%%%%%%%%%%%%%%%%%%%%%%%%%%%%%%%%%%%%%%%%%%%%%%%%%%%%%%%%%%%




%%%%%%%%%%%%%%%%%%%%%%%%%%%%%%%%%%%%%%%%%%%%%%%%%%%%%%%%%%%%%%%%%%%
%%%%%%%%%%%%%%%%%%%%%%%%%%%%%%%%%%%%%%%%%%%%%%%%%%%%%%%%%%%%%%%%%%%
\section{Lecturas del páramo}

La inicialización de los modelos juega un papel importante en el pronóstico de los eventos, por ejemplo \citet{Uribe2012} encontró que para un horizonte de pronóstico de 36 horas se deben descartar las primeras 12 horas para evitar el efecto de \textit{spin-up}.

Según \citep{Uribe2012} el anidamiento no mejora la influencia de la predicción, lo cual traduce en una perdida de tiempo para la predicción de lluvias en Colombia.

Según \citep{Uribe2012} la parametrización por el método de Kain-Fritschc con una resolución de 20 km de grilla presentó la mayor subestimación para la precipitación.

Al aumentar el horizonte de pronóstico de 36 a 48 horas se logra una mejor simulación de los valores de precipitación \citep{Uribe2012}.



\subsection{Prioridades de restauración ecológica}




Según el enfoque de \citep{Castro-Romero2014} cuando un suelo en una zona pierde atributos como el contenido de materia orgánica y la diversidad de especies de artropofauna, el suelo se degrada. Y esta degradación es considerada como una perdida paulatina de capital natural, que produce que sus habitantes perciban menos beneficios por los servicios ecosistémicos y una disminución en la calidad de vida de sus pobladores. Esto hace que las actividades agropecuarias sean más difíciles y costosas.
Estas teorías se basan en estudios previos de \citet{daily1997nature} y \citet{westman1977much}.

Las comunidades que producen algún tipo de degradación en los territorios no son conscientes de los costos del daño producido \citep{westman1977much}. Y estos daños se ven relfejados en la disminución de los ingresos netos de las cosechas \citep{Castro-Romero2014}.

\citet{Castro-Romero2014} estableció que la calidad del suelo se debe tener en cuenta los atributos químicos, físicos y biológicos.

El índice de los procesos denudativos del suelo se consolidan como la suma de la erosión, solifluxión, deslizamientos y desplomes \citep{Castro-Romero2014}. %Solifluxión: se produce cuando un material sólido fluye como si fuera un líquido viscoso.



Una de las conclusiones de \citep{Castro-Romero2014} es que los suelos con la menor valoración económica son aquellas como de mayor prioridad para la restauración. Del mismo modo \citep{Castro-Romero2014} encontró que las prioridades más bajas de restauración se presentan en aquellas unidades que mostraron las mejores condiciones de conservación y calidad del suelo.

\subsubsection{Emergía}

Uno de los resultados más importantes del estudio de \citep{Ernesto} es que el Páramo es usado como una mina del que se extrae carbón, materia orgánica de los suelos y biomasa de sus páramos. Pero adicionalmente este autor cuantificó la tasa a la que se extrae y la estimó en $2.49e-20 \frac{seJ}{year}$.

La emergía es la cantidad de energía requerida para producir algo, teniendo en cuenta la conservación y pérdida de energía que resulta de las leyes de la termodinámica. Entre más trabajo es realizado para producir algo, más energía debe ser transformada para la realización de ese trabajo y una alta emergía será almacenada en el producto. Emergía es una medida del trabajo medioambiental que es necesario para la producción de un bien o un servicio. Esta medida es sin embargo la suma de los flujos de energía requeridos para la producción de algo, expresada en base a las unidades de energía solar Joules \citep{lei2014ecological}.

"La emergía solar de un trozo de madera corresponde a la energía disponible que utilizó un ecosistema para producirlo, osea, la cantidad de Julios solares que se utilizaron para su elaboración" Frase ejemplo tomada de \citep{Castro-Romero2014}.

La transformicidad solar es la emergía solar requerida para conformar una unidad de energía de un servicio o producto, es decir, la relación entre la emergía de un servicio o un producto, es decir, la relación entre la emergía y una cantidad libre de un producto o servicio, por ello su unidad es $\frac{sej}{J}$.

\citep{Castro-Romero2014} realizó los cálculos de la emergía para la minería y la agricultura, teniendo en cuenta la cantidad de combustibles y los insumos usados en cada proceso. Tomaron los mapas de suelo del IGAC realizados en el 2000.

Realizaron un cálculo de la evapotranspiración a partir de Thornwaite realizado en 1966.

Realizaron una estimación de la escorrentía. 

En el estudio se muestra que se requiere $2.44e5 sej$

La emergía libre está dada en términos de la lluvia y el suelo.

La agricultura y la ganadería usa la energía gratuita del sol y la lluvia para producir ingresos. Pero en la ganadería se enfoca en un organismo consumidor.

La ganadería en su mayor parte depende en un 42\% de fuentes no renovables, frente a lo cual \citet{Castro-Romero2014} afirma que cuestiona la sostenibilidad en periodos de tiempo en los cuales la pérdida de materia orgánica del suelo impida su desarrollo.

La leche producida y vendida contiene más emergía que el dinero que recibe.

El sistema de ceba es el menos provechoso de los 3 sistemas ganaderos evaluados.

El páramo es movido mayormente por fuentes externas a este como lo son los combustibles e insumos materiales

El suelo es la mayor fuente de emergía
%%%%%%%%%%%%%%%%%%%%%%%%%%%%%%%%%%%%%%%%%%%%%%%%%%%%%%%%%%%%%%%%%%%%%%%%%%%%%%%%%
%%%%%%%%%%%%%%%%%%%%%%%%%%%%%%%%%%%%%%%%%%%%%%%%%%%%%%%%%%%%%%%%%%%%%%%%%%%%%%%%%








%%%%%%%%%%%%%%%%%%%%%%%%%%%%%%%%%%%%%%%%%%%%%%%%%%%%%%%%%%%%%%%%%%%%%%%%%%%%%%%%%
%%%%%%%%%%%%%%%%%%%%%%%%%%%%%%%%%%%%%%%%%%%%%%%%%%%%%%%%%%%%%%%%%%%%%%%%%%%%%%%%%



%%%%%%%%%%%%%%%%%%%%%%%%%%%%%%%%%%%%%%%%%%%%%%%%%%%%%%%%%%%%%%%%%%%%%%%%%%%%%%%%%
%%%%%%%%%%%%%%%%%%%%%%%%%%%%%%%%%%%%%%%%%%%%%%%%%%%%%%%%%%%%%%%%%%%%%%%%%%%%%%%%%

\section{Avances búsqueda parametrizaciones WRF}


Se creó una carpeta llamada \texttt{zona\_est\_20180320} en esta carpeta se va a ejecutar el WPS que será usado para probar las combinaciones se ejecutará desde un día antes y un día después de las fechas que se usarán en el WRF, ya que, en el WRF se usarán las fechas \texttt{20070201-20070204}. El centro del dominio será en el município de Tocancipá estas serán las coordenadas \texttt{-73.96740787376062087 4.9704810067620171} en las variables correspondientes a \texttt{ref\_lat, ref\_lon, truelat1, stand\_lon}. Y se tomarán los datos proporcionados por el IDEAM para la Vtable. Sólo se usó un dominio para llegar a una resolución de 2 kilómetros con el WRF, una copia quedó dentro de la carpeta llamada resultados.\\

Al final se crearon los archivos netCDF para las fechas seleccionadas. 

Para la comparación de los mejores modelos se usó el diagrama de Taylor a partír de un código obtenido del repositorio de \textcolor{blue}{ \href{https://gist.github.com/ycopin/3342888}{ycopin}}. En este código se usó lo propuesto por \citet{barnston1992} quien calcula el cuadrado medio del error a partir del la desviación estándar y el coeficiente de correlación.

\begin{equation}\label{eq:rms}
	     RMS = \sqrt{1-r^{2}}SD_{y}
\end{equation}

Para poder compara las temperaturas a 2 metros se unieron todas las temperaturas (máximas, mínimas y promedio) conla finalidad de solo usar un valor. Se priorizaron las variables de la variable llamada promedio sobre las demás esto se encuentra en el código llamado \texttt{Procesamiento\_wrf.py}

\begin{enumerate}

\item El IDEAM facilitó las namelist.* y la versión que están usando
\item Se instaló el WRFV3.9.1.1 en la máquina agrometeo
\item Se realizó una corrida con los datos del GFS para un periodo de 2016-12-28 00:00 - 2017-01-01 18:00 (4 días). Esta corrida tomó 12 horas y 30 minutos y se usaron 152 GB. Se realizaron 64 simulaciones. (265 días).

\end{enumerate}

\subsection{Resultado de las modelaciones con el WRF}

Se probaron cada una de las parametrizaciones físicas del modelo, (sin combinatoria). Este proceso comenzó el 20180404 y terminó el 20180412, se demoró un total de 8 días.
\subsection{Literatura sobre las parametrizaciones}

La comprensión de los procesos físicos que rigen las masas de aire frío y su adecuada representación en un modelo de predicción numérico (por sus siglas en inglés \textit{numerical weather prediction} (NWP)) son necesarios para una predicción adecauada de las temperaturas de la superficie y los escenarios de heladas \citep{prabha2008}.\\



Le balance de energía y la capa límite (bpl) se ha demostrado que pronostica erroneamente las tasas de enfriamiento y las altas temperaturas en la presencia de heladas advectivas \citep{heinemann1988}. El pronóstico de las heladas con la ayuda de moedlos de predicción enfrenta un reto debido la interacción no linear de los componetnes del balance hídrico, el cual puede causar pronósticos imprecisos \citep{gutowski2003}.\\

El estado del arte del WRF se definió como una colección de varios modelos de predicción numericos en la arquitectura de un solo software con dos modelos de pronóstico dos formas de pronóstico de solucionar que son el \textit{Advanced Reseach WRF} por sus siglas en inglés (ARW) y el \textit{Nonhydrostatic Mesoscale Model} por sus siglas en inglés (NMM). El ARW ha sido desarrollado y administrado por \textit{NCAR's Mesoscale and Microscalo Meteorology Laboratory}. El núcleo del NMM fue desarrollado por \textit{National Centers for Envoronmental Prediction} y es actualmente usado en el sistema \textit{Hurricane WRF} (HWRF).

La correcta parametrizaión va a depender de varios factores como lo nombra \citet{prabha2008evaluation} tales como: seleccionar correctamente la resolución, el poder computacional, condiciones iniciales, resolución del terreno, datos del uso del suelo y las parametrizaciones físicas a usar.

Para la verificación de la predicción de estos modelos es necesario comparar con las estaciones para de esta manera verificar los resultados (Validar??).

Uno de los primeros pasos que se deben desarrollar en esta metodología es la caracterización climática de la zona. \citet{prabha2008evaluation} realizaron una caracterización para cada una de las estaciones por 13 años.

\citet{prabha2008evaluation} define el índice de congelamiento como la integral de la temperatura del aire cuando esta se encuentra bajo 0\celc.\\

\citet{prabha2008evaluation} Usan dos índices para evaluar

El índice de congelamiento es definido como el numero de grados día por un periodo específico de tiempo \citep{frauenfeld2007northern}. El índice que propone \citet{frauenfeld2007northern} es definido como la suma de los valores de temperatura bajo 0\celc como:

\begin{equation}\label{eq:frost_index}
    \int_{t_0}^{t_n} T dt,\quad Para T \leq 0^{\circ}C
\end{equation}

donde $T$ corresponde al valore de la temperatura, integrado desde $t_0$ hasta el tiempo $t_n$. son el comienzo y final de los diferentes tiempos, $T$ es el valor de la temperatura y $dt$ es el cambio en el tiempo.

Existen dos tipos de heladas la radiativa y la advectiva \citep{snyder2005frost} grandes incusiones de aire frío, vientos atmósferas que se encuentran a temperaturas bajo 0\celc 

Caracterización de una helada radiativa cielos despejados, vientos en calma, inversión de temperatura, bajas temperaturas de punto de rocío y temperaturas que caen por debajo de 0\celc.

El tope de la inversión es variable ya que depende de la topografía y las condiciones climáticas, pero generalmente tiene un rango de 9 a 60 m \citep{snyder2005frost}.

En las noches despejadas más calor es irradiado fuera de la superficie en comparación con el que fue recibido durante el día. La temperatura cae rápidamente cerca de la superficie de radiación causando una inversión.

Según la gráfica de inversión de temperatura la inversión se nota a una altura de 3 o 5 metros. 

Para la protección de las heladas radiativas son más efectivos los métodos "Energy-intensive"

Dentro de las heladas radiativas hay dos categorías:

La helada blanca (hoar frost): Sucede cuando el vapor de agua depositado en la superficie forma una cobertura blanquecina de hielo y es comúnmente llamada "escarcha".

Helada negra (black): Ocurre cuando la temperatura cae bajo 0\celc y no hay formación de hielo en la superficie. Si la humedad es suficientemente baja, entonces la superficie no alcanzará el punto de congelamiento y no se formará hielo. Cuando la humedad es alta, hay una más alta probabilidad de una helada blanca. Se produce calor cuando se pasa de líquido a sólido, por esta razón la helada blanca es menos lesiva en comparación un una helada negra.

\begin{figure}[H]
	\begin{center}
	\includegraphics[draft=true, scale=0.6]{latent.png}
		\caption{Gráfica de calor latente}
		\label{subfig:cal_lat}	
	\end{center}

\end{figure}

Una clara seña de que estamos presenciando una helada cae en unas pocas horas antes de del amaneces, cuando la energía neta de radiación de la superficie cambia rápidamente de positiva a negativa. Este cambio se da principalmente porque la radiación solar decrece desde su valor más alto al medio día a 0 en el atardecer.

La densidad del flujo de calor en el suelo está determinada por 

\begin{equation}\label{eq:frost_index}
    G = -\lambda(\frac{\partial T}{\partial z})
\end{equation}

tomado de \citep{sauer2002soil}

Helada por advección\\
Masas de aire frío llegan a una área a remplazar el aire cálido que estaba presente antes de que el tiempo cambiara.

\begin{enumerate}
\item Condiciones nubladas
\item Vientos desde moderados hasta fuertes según \citet{prabha2008evaluation} son vientos con velocidades superiores a 2 $m/s$.
\item Sin inversión térmica
\item Baja humedad
\item Las temperaturas caen bajo 0\celc y permanecen de esta misma forma todo el día.
\item La mayoría de estos eventos ocurren en climas Mediterráneos y tienden a ser más comunes en las costas.
\item La mayoría de los eventos
\end{enumerate}
Este tipo de heladas son difíciles de combatir, es por esto que es la mayoría de los métodos de protección funcionan mejor en la presencia de una inversión. En algunos casos la helada por advección puede generar una helada radiativa.\\

Clasificación de métodos de protección

Existen varias clasificaciones como los métodos pasivos que son más preventivos como métodos biológicos, o técnicas ecológicas incluyendo prácticas como un alistamiento previo a la presencia de las heladas.

Los métodos activos son métodos que son basados en el uso intensivo de energía. Algunos ejemplos son calentadores, riego, máquinas de viento.\\

Entre las latitudes entre los trópicos de Cancer y Capricornio existen grandes áreas con temperaturas bajo cero. Pero aún en estas zonas a veces se presentan daños en lugares con grandes alturas.\\

Es menos probable que ocurra una helada cuando:
\begin{enumerate}
\item El terreno se encuentra en un área donde el viento sopla
\item El terreno se encuentra cerca a una masa de agua
\end{enumerate}

El humo no ofrece ningún tipo de protección a los cultivos \citep{itier1987influence}\\

Los daños causados en las plantas son debido al congelamiento del agua extra celular dentro de las plantas 

Las ubicaciones bajas presentan problemas de bajas temperaturas. Pero en algunas ocasiones se pueden presentar daños en algunos sitios, esto es debido al tipo de suelo.\\

Un suelo seco arenoso transfiere mejor el calor que un suelo seco y arcilloso y ambos transfieren y almacenan mejor la temperatura que un suelo orgánico.

La transferencia de calor del agua es tres veces mayor que la del suelo. Los suelos húmedos tienen mayor capacidad de conductividad en comparación con un suelo seco.

Obstáculos que inhiban el drenaje de las masas de aire frío.\\

La fertilización puede ser una forma para evitar ele efecto de las heladas en el cultivo

El riego es uno de los mecanismos más económicos para el control de las heladas.\\

La inundación de los suelos puede proveer protección de los cultivos entre 2 a 3\celc. Existe una relación entre la lamina de agua que se debe aplicar para evitar la helada de esa noche y la temperatura máxima registrada en el día.\\

Una de las formas de protección es realizar un pronóstico de las heladas

\subsubsection{Parámetros usado por Prabha en lso dos artículos \citep{prabha2008} y \citep{prabha2008evaluation}}


En el artículo \citet{prabha2008} podemos destacar varios aspectos:

\begin{enumerate}
\item Lo primero que se estudió fueron las condiciones iniciales
\item Capa límite planetaria
\item Superficie del terreno
\end{enumerate}

Se realizó una comparación contra la red de estaciones automáticas de la zona. En este estudio se manejaron 40 niveles de presión y se acomodaron 15 niveles en la PBL. (Grid resolution en el artículo)


La configuración de modelo fue la siguiente:

\begin{itemize}
\item WSM3 en microfísica \texttt{mp\_physics = 3: "WRF Single-Moment (WSM) 3-class simple ice scheme: A simple efficient scheme with ice and snow processes suitable for mesoscale grid sizes."}
\item La parametrización de cúmulos es la Kain-Fritsch \texttt{cu\_physics = 1: "Kain-Fritsch (new Eta) scheme: deep and shallow sub-grid scheme using a mass flux approach with downdrafts and CAPE removal time scale"}

\item La parametrización usada para la radiación de onda larga corresponde a RRTM \texttt{ra\_lw\_physics = 1: "RRTM scheme: Rapid Radiative Transfer Model. An accurate scheme using look-up tables for efficiency. Accounts for multiple bands, trace gases, and microphysics species. This scheme has been preliminarily tested for WRF-NMM."}

\item La parametrización usada para la radiación de onda corta corresponde a RRTM \texttt{•}
\end{itemize}


El experimento consistió en probar diferentes, capas límites planetarias, condiciones iniciales y diferes esquemas de suelo.:

Condiciones iniciales
\begin{itemize}
\item NAM = Grilla de 12 km. North American Regional Reanalysis
\item NARR grilla de 36 km. North American Mesosclase
\end{itemize}

Parameterizaciones de capa límite \texttt{bl\_pbl\_physics}
\begin{itemize}
\item MYJ = 2. \texttt{Mellor-Yamada-Janjic (Eta) TKE scheme}
\item YSU = 1. \texttt{YSU scheme}
\end{itemize}

Parametrizaciones del terreno \texttt{sf\_surface\_physics}
\begin{itemize}

\item SLAB = 1 \texttt{"Thermal Diffusion scheme: soil temperature only scheme, using five layers."}

\item Noah = 2. \texttt{"Noah Land-Surface Model: Unified NCEP/NCAR/AFWA scheme with soil temperature and moisture in four layers, fractional snow cover and frozen soil physics. This scheme has been preliminarily tested for WRF-NMM."}


\item RUC = 3. \texttt{"RUC Land-Surface Model: Rapid Update Cycle operational scheme with soil temperature and moisture in six layers, multi-layer snow and frozen soil physics. This scheme has been preliminarily tested for WRF-NMM."}



\end{itemize}



\subsubsection{Revisión de literatura relacionada}


Existen diferencias micrometeorológicas, como por ejemplo el echo que existan diferencias de 1\celc en 100 metros.

La máxima temperatura que puede soportar un tuberculo de papa es -0.8\celc\\


La capa de aire que toca el suelo se llama vientos catabático o vientos de drenaje. Son causados por el enfriamiento del aire, adyacente al suelo y descienden gracias a la influencia de la gravedad \citep{Stull1988pbl}.

El estudio de la capa límite planetaria contiene el estudio de la micrometeorología \citep{Stull1988pbl}.

Para el estudio de la micrometeorología \citet{Stull1988pbl} los clasifica en 3:

\begin{enumerate}
\item Métodos estocásticos
\item Teoría de similitud
\item Clasificación fenomenológica
\end{enumerate}




\subsection{Mecanismos de transferencia de energía}

Cuando la energía pasa a través del suelo por conducción se llama densidad de flujo de calor.\\
La protección de una helada consiste en tratar de reducir o remplazar la pérdida del contenido de calor sensible del aire y de las plantas.



Cuando el agua se congela la mayoría de las moléculas hacen enlaces con el nitrógeno para formar una estructura cristalina. 

\begin{table}[H]
\centering

\label{tabla_conductividad}
\begin{tabular}{@{}llll@{}}
\multicolumn{4}{l}{Conductividad termica de los suelos $W m^{-1}\cdot^{\circ}C^{-1} $} \\ \midrule
\multicolumn{1}{l|}{}         & Orgánicos  & Arcillosos & Arenosos \\ \midrule
\multicolumn{1}{l|}{Secos}    & 0.1        & 0.25       & 0.3      \\
\multicolumn{1}{l|}{Húmedos}  & 0.5        & 1.6        & 2.4     

\end{tabular}
\caption{Tabla de conductividad de suelos}
\end{table}


\section{Cosas que sobran}


La zona de estudio es la Sabana de Bogotá. En la Sabana de Bogotá las variaciones de temperatura afectan los cultivos tales como la papa, pastos, maíz, hortalizas entre otros. Los principales limitantes de caracter agometeorológico en la producción de papa son principalmente eventos de heladas y estrés por déficit hídrico, los cuales pueden producir pérdidas de hasta el 75\% \citep{DANE2002}.\\

En las últimas décadas se documentó un incremento de las temperaturas medias a escala global de aproximadamente 0,15\celc por década \citep{jones2001} 

\section{Comandos para la corrección del documento}
%% Cómo hacer comentarios

%% En el caso que quieran quitar algo y quiera tacharlo
Esta es una palabra repetida \sout{repetida}, por ejemplo. \\

%% En el caso que se quiera resaltar algo
Esto es \emph{muy importante} por favor tener en cuenta.

%% En el caso que se quiera hacer una nota al margen
\todo{De esta forma se hacen notas en el margen}

%% En el caso que quiera adicionar algo al texto. Puede seleccionar un color y adicionar los comentarios

\textcolor{red}{Esta es una nueva idea}









%######### Esto fue tomado del capítlo 1 y 3 
% * <etorresm@unal.edu.co> 2018-09-24T23:44:37.716Z:
%
% > %######### Esto fue tomado del capítlo 1 y 3 
% > \subsection{El modelo Weather Research and Forecasting Model (WRF)}
% > El modelo WRF es un modelo muy usado al rededor del mundo gracias a los buenos resultados obtenidos, como lo reporta \citet{Jimenez2014}. El modelo WRF es un sistema de cálculo numérico para simulación atmosférica que fue diseñado para cumplir objetivos de investigación y pronóstico, este modelo sirve en un amplio rango de escalas espaciales, desde decenas de metros hasta miles de kilómetros. Los usuarios de este modelo pueden producir simulaciones basadas en las condiciones atmosféricas actuales o condiciones idealizadas \citep{Pielke2002}. WRF es capaz de realizar una reducción de escala de un modelo de pronóstico global como GFS. La reducción de escala toma las condiciones del modelo global y le aumenta la resolución teniendo en cuenta las características de la zona de estudio \citep{Sene2010}.\\
% > El modelo WRF tiene una aproximación no-hidrostática, esto quiere decir que tiene en cuenta el momento en en la dirección vertical $(w)$, en comparación con el modelo hidrostático el cual no tiene en cuenta los cambios en el momento en la dirección vertical $(w)$. Los modelos no-hidrostáticos son usado para el pronóstico de fenómenos de mesoescala o escalas menores \citep{Pielke2002, Sene2010}.\\
% > %. El modelo hidrostático supone una homogeneidad en la columna de aire y está dado por la densidad y la gravedad \citep{Pielke2002, Sene2010}.
% > En el ámbito internacional el modelo WRF se ha usado en varios países como es el caso de Perú, \citet{Saavedra2016} realizó un trabajo de modelación de fenómenos meteorológicos de mesoescala. Como resultado se obtuvo que la modelación reproduce de buena forma el ciclo diario de las temperaturas del aire a dos metros, pero las temperaturas mínimas del aire a dos metros fueron sobrestimadas en las partes altas de la zona de estudio, y subestimó la tasa de enfriamiento en el fondo de los valles, generando una mayor temperatura modelada, respecto a los valores reportados por estaciones ubicadas en las laderas.\\
% > \citet{Fernandez2011} realizaron un estudio en Argentina para la cuidad de Mendoza con tres dominios espaciales de 36, 12 y 4 km donde se tuvieron en cuenta las condiciones orográficas para la delimitación de los mismos. Se usó el esquema de capa límite planetaria de \textit{Yonsei University}. Como resultado se encontró que los valores máximos tanto de temperatura como de humedad son predichos correctamente.\\
% > \citet{Corrales2015a} usó el modelo WRF para realizar un pronóstico de las temperaturas en México usando un único dominio y con la parametrización \textit{Mellor-Yamada-Janic} para la capa límite planetaria. Y como resultado obtuvo que hay algunas zonas donde el modelo es confiable para la predicción de las temperatura, lo cual puede prevenir daños por heladas en un horizonte pronóstico de 5 días.\\
% > \citet{Hu2010} usaron el modelo WRF para modelar las temperaturas en el centro de Estados Unidos, en los estados de Texas, Lousiana y parte de Arkansas. Ellos probaron las diferentes parametrizaciones de capa límite (PLB) y, encontraron que algunas parametrizaciones producen temperaturas más altas de lo esperado.\\
% > \citet{Parra2012} realizó un estudio para Ecuador donde la finalidad era simular la meteorología de un año completo de todo el país. Para esto usaron el modelo WRF con 2 dominios y con la parametrización Mellor Yamada Jajic (MYJ), y se obtuvo como resultado que las temperaturas simuladas son coherentes con los fenómenos observados en las estaciones en tierra.\\
% > Basado en estas investigaciones se evidencia la necesidad de probar varias parametrizaciones e intentar lograr la mejor combinación de ellas. Para el caso de Colombia \citet{Uribe2012} escogió 10 eventos de duración de un día. Se probaron dos parametrizaciones diferentes , con dos diferentes resoluciones espaciales y dos diferentes horizontes de pronóstico con la finalidad de encontrar la mejor combinación. Como resultado encontró que la parametrización por el método de Kain-Fritschc con una resolución de 20 km de grilla presentó la mayor subestimación para la precipitación, mientras que la parametrización de Morrison presenta los mejores resultados. Y al aumentar el horizonte de pronóstico de 36 a 48 horas se logra una mejor simulación de los valores de precipitación \citep{Uribe2012}.\\
% > En Colombia, el Instituto de Hidrología, Meteorología y Estudios Ambientales (IDEAM) ha implementado el modero WRF desde el año 2007 para la predicción del tiempo atmosférico convirtiéndose en una herramienta muy importante para esta entidad \citep{Arango2011}. Se han realizado validaciones del modelo WRF en la Sabana de Bogotá para la variable precipitación mediante una comparación con las estaciones convencionales, como la realizada por \citet{Mejia2012}. El objetivo fue identificar y establecer cuál de los modelos operacionales del IDEAM lograba identificar de manera aceptable los patrones de comportamiento de las variables de precipitación. Se encontró que el modelo WRF alimentado con los datos del modelo \textit{Global Forecast System} (GFS) presentó los mejores resultados. Este estudio presenta una metodología útil para la validación de los modelos y la determinación del mejor modelo.\\
% > %Estos estudios fueron realizados en diferentes cultivos como \citet{prabha2008} que usó el cultivo de durazno, \citep{Gomez2014} quien realizó su estudio en pasturas, \citep{Saavedra2016} quien estudio en una zona productora de papa y maíz.\\
% > %Es importante resaltar que el modelo toma como referencia datos del suelo. Pero según \citet{Castro-Romero2014} en la Región Andina se presentan cambios por el uso del paisaje que dejan como resultado tan solo el 31\% de los bosques naturales y una degradación del 53\% en arbustales secos. Se estima que para el año 1998 el 69\% de los bosques andinos habían sido talados. Uno de estos sucede en Suesca - Cundinamarca \citet{Castro-Romero2014} reporta que debido al uso agropecuario intensivo que se le ha dado a los suelos es posible observar zonas desprovistas de cobertura vegetal y de suelo con estados muy avanzados de degradación, lo cual imposibilita su posible recuperación en años próximos. Ya que los ecosistemas se usan de una forma extractiva como,lo menciona \citet{Ernesto}, uno de los ecosistemas más importantes de Colombia el Páramo es usado como una mina del que se extrae carbón, materia orgánica de los suelos y biomasa de sus páramos. Pero adicionalmente \citep{Ernesto} cuantificó la tasa a la que se extrae y la estimó en $2.49e-20 \frac{seJ}{year}$. Estos cambios nos hacen reflexionar acerca de la importancia del mantenimiento de los suelos y que se deben hacer ajustes periódicos a los modelos ya que con estos estudios se demuestra que el suelo es dinámico.
% > %Un modelo de mesoescala es un modelo numérico de predicción del tiempo atmosférico, que se usa para hacer una predicción a escala de kilómetros y horas, basado en la dinámica atmosférica \citep{Uribe2012}.\\
% > %\section{Conclusiones}
% > %El cultivo de papa es un cultivo de gran importancia en el país. El consumo interno del país es abastecido casi en su mayoría por la producción interna. Este cultivo es seriamente afectado por muchos factores agroclimáticos y uno de los más importantes es la temperatura extrema. Cundinamarca es el Departamento con mayor producción en el país y la Sabana de Bogotá es la zona de Cundinamarca que más influencia tiene en la producción de papa, por sus condiciones topográficas de sabana se convierte en una altamente suceptible a heladas. Los meses con mayor probabilidad de heladas son diciembre, enero y febrero; y la hora en la que más se presentan bajas temperaturas es a las 5 am. Las temperaturas más altas se presentan en la mayoría de los meses pero principalmente entre los meses de diciembre y abril y las horas en las que más se presentan es a las 12 m.\\
% > %Basado en los registros de las estaciones convencionales desde 1971 hasta 2016, se evidenció un aumento de la temperatura en 1.34\celc. Adicionalmente se evidencian cambios en la frecuencia de las bajas y las altas temperaturas, ya que la frecuencia de las bajas temperaturas ha venido disminuyendo y las altas temperaturas han presentado un aumento en los valores registrados.
% > %De los modelos de pronóstico revisados el modelo WRF un modelo de regional de pronóstico es el que presenta las mejores características para realizar el estudio de las temperaturas, ya que es un modelo que tiene en cuenta aspectos físicos, dinámicos y su evolución, además es un modelo que ha sido probado y es usado actualmente en Colombia.
% > %%%%%%%%%%%%%%%%%%%%%%%%%%%%%%%%%%%%%%%%%%%%%%%%%%%%%%%%%%%%%%%%%%%%%%%%%%%%%%%%%
% > %%%%%%%%%%%%%%%%%%%%%%%%%%%%%%%%%%%%%%%%%%%%%%%%%%%%%%%%%%%%%%%%%%%%%%%%%%%%%%%%%
% > Se realizó una revisión bibliográfica en los previos capítulos con la finalidad de determinar el mejor modelo para realizar el pronóstico de las temperaturas extremas. Para facilitar las fortalezas y las limitaciones se realizó una comparación , ver Tabla \ref{tab:fort_deb_mod}.
% > \begin{table}[H]
% > \resizebox{\textwidth}{!}{\begin{tabular}{p{5cm}| p{5cm} p{5cm} p{5cm}}
% > Modelo                                        & Fortaleza                                                  & Limitación                            &Fuente                                                   \\ \hline
% > \multirow{2}{*}{Modelos empíricos}            & Fácil aplicación                                                               & No posee alta precisión                                   & \citep{Gomez2014, Allen1957, Snyder2010}                   \\
% >                                               & Se puede hacer modificaciones al modelo de manera sencilla                     & Los modelos son creados para determinadas condiciones     &                                                                              \\ \hline
% > \multirow{2}{*}{Balance de energía del suelo} & Fácil aplicación siempre y cuando se tengan todas las variables                & Alta incertidumbre                                        & \citep{evett2011water,Rosenzweig2014, Rossi2002}          \\
% >                                               &                                                                                & Las variables no son fáciles de calcular                  &                                                                              \\ \hline
% > Sistemas de información geográfico            & Existe una clara relación entre los valles y las temperaturas extremas         & No  tiene en cuenta los flujos de radiación de onda larga & \citep{evett2011water, Halley2003, Blennow1998}              \\ \hline
% > Redes neuronales                              & Buenas predicciones predicciones en un corto horizonte de pronóstico (6 horas) & Pronóstico a un horizonte muy corto                       & \citep{Smith2007}                                           \\ \hline
% > Modelo \textit{Weather Research and Forecasting Model} (WRF)                   & Buenas predicciones en un plazo de 2 días                                      & Alto gasto computacional                                  & \citep{prabha2008evaluation, Arango2011, Mejia2012, Ruiz2014} \\
% >                                               &                                                                                & Es necesario poseer una alta capacidad de almacenamiento  &                                                                              \\ \hline
% > \end{tabular}}
% > \caption{Tabla resumen de las ventajas y desventajas de los modelos}
% > \label{tab:fort_deb_mod}
% > \end{table}
% > Basado en la información de la tabla \ref{tab:fort_deb_mod} y teniendo en cuenta las características de cada uno de los modelos se decidió usar el Modelo WRF, ya que es un modelo que cumple con los requerimientos para este estudio y en el país se ha usado este modelo con resultados satisfactorios como los obtenidos por \citep{Mejia2012,Arango2011,Arango2014,Ruiz2014,Uribe2012,RojasA2011,ArmentaPorras2013}.\\
% > Los pasos a seguir para realizar una modelación con el modelo WRF consiste en:
% > \begin{enumerate}
% > \item Compilar el WPS
% > \item Compilar el WRF
% > \item Determinar el área de estudio
% > \item Selección de los dominios del modelo WRF para la inicializacion del modelo
% > \item Selección de las parametrizaciones que se van a emplear
% > \item Descargar los datos que van a alimentar el modelo
% > \item Realizar el pre-proceso con el WPS
% > \item Realizar el proceso de modelación con el WRF
% > \item Realización del pos-proceso con python3.6
% > \end{enumerate}
%
% ^.


\subsection{El modelo Weather Research and Forecasting Model (WRF)}

El modelo WRF es un modelo muy usado al rededor del mundo gracias a los buenos resultados obtenidos, como lo reporta \citet{Jimenez2014}. El modelo WRF es un sistema de cálculo numérico para simulación atmosférica que fue diseñado para cumplir objetivos de investigación y pronóstico, este modelo sirve en un amplio rango de escalas espaciales, desde decenas de metros hasta miles de kilómetros. Los usuarios de este modelo pueden producir simulaciones basadas en las condiciones atmosféricas actuales o condiciones idealizadas \citep{Pielke2002}. WRF es capaz de realizar una reducción de escala de un modelo de pronóstico global como GFS. La reducción de escala toma las condiciones del modelo global y le aumenta la resolución teniendo en cuenta las características de la zona de estudio \citep{Sene2010}.\\

El modelo WRF tiene una aproximación no-hidrostática, esto quiere decir que tiene en cuenta el momento en en la dirección vertical $(w)$, en comparación con el modelo hidrostático el cual no tiene en cuenta los cambios en el momento en la dirección vertical $(w)$. Los modelos no-hidrostáticos son usado para el pronóstico de fenómenos de mesoescala o escalas menores \citep{Pielke2002, Sene2010}.\\
%. El modelo hidrostático supone una homogeneidad en la columna de aire y está dado por la densidad y la gravedad \citep{Pielke2002, Sene2010}.

En el ámbito internacional el modelo WRF se ha usado en varios países como es el caso de Perú, \citet{Saavedra2016} realizó un trabajo de modelación de fenómenos meteorológicos de mesoescala. Como resultado se obtuvo que la modelación reproduce de buena forma el ciclo diario de las temperaturas del aire a dos metros, pero las temperaturas mínimas del aire a dos metros fueron sobrestimadas en las partes altas de la zona de estudio, y subestimó la tasa de enfriamiento en el fondo de los valles, generando una mayor temperatura modelada, respecto a los valores reportados por estaciones ubicadas en las laderas.\\

\citet{Fernandez2011} realizaron un estudio en Argentina para la cuidad de Mendoza con tres dominios espaciales de 36, 12 y 4 km donde se tuvieron en cuenta las condiciones orográficas para la delimitación de los mismos. Se usó el esquema de capa límite planetaria de \textit{Yonsei University}. Como resultado se encontró que los valores máximos tanto de temperatura como de humedad son predichos correctamente.\\

\citet{Corrales2015a} usó el modelo WRF para realizar un pronóstico de las temperaturas en México usando un único dominio y con la parametrización \textit{Mellor-Yamada-Janic} para la capa límite planetaria. Y como resultado obtuvo que hay algunas zonas donde el modelo es confiable para la predicción de las temperatura, lo cual puede prevenir daños por heladas en un horizonte pronóstico de 5 días.\\

\citet{Hu2010} usaron el modelo WRF para modelar las temperaturas en el centro de Estados Unidos, en los estados de Texas, Lousiana y parte de Arkansas. Ellos probaron las diferentes parametrizaciones de capa límite (PLB) y, encontraron que algunas parametrizaciones producen temperaturas más altas de lo esperado.\\

\citet{Parra2012} realizó un estudio para Ecuador donde la finalidad era simular la meteorología de un año completo de todo el país. Para esto usaron el modelo WRF con 2 dominios y con la parametrización Mellor Yamada Jajic (MYJ), y se obtuvo como resultado que las temperaturas simuladas son coherentes con los fenómenos observados en las estaciones en tierra.\\


Basado en estas investigaciones se evidencia la necesidad de probar varias parametrizaciones e intentar lograr la mejor combinación de ellas. Para el caso de Colombia \citet{Uribe2012} escogió 10 eventos de duración de un día. Se probaron dos parametrizaciones diferentes , con dos diferentes resoluciones espaciales y dos diferentes horizontes de pronóstico con la finalidad de encontrar la mejor combinación. Como resultado encontró que la parametrización por el método de Kain-Fritschc con una resolución de 20 km de grilla presentó la mayor subestimación para la precipitación, mientras que la parametrización de Morrison presenta los mejores resultados. Y al aumentar el horizonte de pronóstico de 36 a 48 horas se logra una mejor simulación de los valores de precipitación \citep{Uribe2012}.\\

En Colombia, el Instituto de Hidrología, Meteorología y Estudios Ambientales (IDEAM) ha implementado el modero WRF desde el año 2007 para la predicción del tiempo atmosférico convirtiéndose en una herramienta muy importante para esta entidad \citep{Arango2011}. Se han realizado validaciones del modelo WRF en la Sabana de Bogotá para la variable precipitación mediante una comparación con las estaciones convencionales, como la realizada por \citet{Mejia2012}. El objetivo fue identificar y establecer cuál de los modelos operacionales del IDEAM lograba identificar de manera aceptable los patrones de comportamiento de las variables de precipitación. Se encontró que el modelo WRF alimentado con los datos del modelo \textit{Global Forecast System} (GFS) presentó los mejores resultados. Este estudio presenta una metodología útil para la validación de los modelos y la determinación del mejor modelo.\\



%Estos estudios fueron realizados en diferentes cultivos como \citet{prabha2008} que usó el cultivo de durazno, \citep{Gomez2014} quien realizó su estudio en pasturas, \citep{Saavedra2016} quien estudio en una zona productora de papa y maíz.\\





%Es importante resaltar que el modelo toma como referencia datos del suelo. Pero según \citet{Castro-Romero2014} en la Región Andina se presentan cambios por el uso del paisaje que dejan como resultado tan solo el 31\% de los bosques naturales y una degradación del 53\% en arbustales secos. Se estima que para el año 1998 el 69\% de los bosques andinos habían sido talados. Uno de estos sucede en Suesca - Cundinamarca \citet{Castro-Romero2014} reporta que debido al uso agropecuario intensivo que se le ha dado a los suelos es posible observar zonas desprovistas de cobertura vegetal y de suelo con estados muy avanzados de degradación, lo cual imposibilita su posible recuperación en años próximos. Ya que los ecosistemas se usan de una forma extractiva como,lo menciona \citet{Ernesto}, uno de los ecosistemas más importantes de Colombia el Páramo es usado como una mina del que se extrae carbón, materia orgánica de los suelos y biomasa de sus páramos. Pero adicionalmente \citep{Ernesto} cuantificó la tasa a la que se extrae y la estimó en $2.49e-20 \frac{seJ}{year}$. Estos cambios nos hacen reflexionar acerca de la importancia del mantenimiento de los suelos y que se deben hacer ajustes periódicos a los modelos ya que con estos estudios se demuestra que el suelo es dinámico.

%Un modelo de mesoescala es un modelo numérico de predicción del tiempo atmosférico, que se usa para hacer una predicción a escala de kilómetros y horas, basado en la dinámica atmosférica \citep{Uribe2012}.\\


%\section{Conclusiones}

%El cultivo de papa es un cultivo de gran importancia en el país. El consumo interno del país es abastecido casi en su mayoría por la producción interna. Este cultivo es seriamente afectado por muchos factores agroclimáticos y uno de los más importantes es la temperatura extrema. Cundinamarca es el Departamento con mayor producción en el país y la Sabana de Bogotá es la zona de Cundinamarca que más influencia tiene en la producción de papa, por sus condiciones topográficas de sabana se convierte en una altamente suceptible a heladas. Los meses con mayor probabilidad de heladas son diciembre, enero y febrero; y la hora en la que más se presentan bajas temperaturas es a las 5 am. Las temperaturas más altas se presentan en la mayoría de los meses pero principalmente entre los meses de diciembre y abril y las horas en las que más se presentan es a las 12 m.\\

%Basado en los registros de las estaciones convencionales desde 1971 hasta 2016, se evidenció un aumento de la temperatura en 1.34\celc. Adicionalmente se evidencian cambios en la frecuencia de las bajas y las altas temperaturas, ya que la frecuencia de las bajas temperaturas ha venido disminuyendo y las altas temperaturas han presentado un aumento en los valores registrados.

%De los modelos de pronóstico revisados el modelo WRF un modelo de regional de pronóstico es el que presenta las mejores características para realizar el estudio de las temperaturas, ya que es un modelo que tiene en cuenta aspectos físicos, dinámicos y su evolución, además es un modelo que ha sido probado y es usado actualmente en Colombia.
%%%%%%%%%%%%%%%%%%%%%%%%%%%%%%%%%%%%%%%%%%%%%%%%%%%%%%%%%%%%%%%%%%%%%%%%%%%%%%%%%
%%%%%%%%%%%%%%%%%%%%%%%%%%%%%%%%%%%%%%%%%%%%%%%%%%%%%%%%%%%%%%%%%%%%%%%%%%%%%%%%%




Se realizó una revisión bibliográfica en los previos capítulos con la finalidad de determinar el mejor modelo para realizar el pronóstico de las temperaturas extremas. Para facilitar las fortalezas y las limitaciones se realizó una comparación , ver Tabla \ref{tab:fort_deb_mod}.



\begin{table}[H]
\resizebox{\textwidth}{!}{\begin{tabular}{p{5cm}| p{5cm} p{5cm} p{5cm}}
Modelo                                        & Fortaleza                                                  & Limitación                            &Fuente                                                   \\ \hline
\multirow{2}{*}{Modelos empíricos}            & Fácil aplicación                                                               & No posee alta precisión                                   & \citep{Gomez2014, Allen1957, Snyder2010}                   \\
                                              & Se puede hacer modificaciones al modelo de manera sencilla                     & Los modelos son creados para determinadas condiciones     &                                                                              \\ \hline
\multirow{2}{*}{Balance de energía del suelo} & Fácil aplicación siempre y cuando se tengan todas las variables                & Alta incertidumbre                                        & \citep{evett2011water,Rosenzweig2014, Rossi2002}          \\
                                              &                                                                                & Las variables no son fáciles de calcular                  &                                                                              \\ \hline
Sistemas de información geográfico            & Existe una clara relación entre los valles y las temperaturas extremas         & No  tiene en cuenta los flujos de radiación de onda larga & \citep{evett2011water, Halley2003, Blennow1998}              \\ \hline
Redes neuronales                              & Buenas predicciones predicciones en un corto horizonte de pronóstico (6 horas) & Pronóstico a un horizonte muy corto                       & \citep{Smith2007}                                           \\ \hline
Modelo \textit{Weather Research and Forecasting Model} (WRF)                   & Buenas predicciones en un plazo de 2 días                                      & Alto gasto computacional                                  & \citep{prabha2008evaluation, Arango2011, Mejia2012, Ruiz2014} \\
                                              &                                                                                & Es necesario poseer una alta capacidad de almacenamiento  &                                                                              \\ \hline


\end{tabular}}
\caption{Tabla resumen de las ventajas y desventajas de los modelos}
\label{tab:fort_deb_mod}
\end{table}

Basado en la información de la Tabla \ref{tab:fort_deb_mod} y teniendo en cuenta las características de cada uno de los modelos se decidió usar el Modelo WRF, ya que es un modelo que cumple con los requerimientos para este estudio y en el país se ha usado este modelo con resultados satisfactorios como los obtenidos por \citep{Mejia2012,Arango2011,Arango2014,Ruiz2014,Uribe2012,RojasA2011,ArmentaPorras2013}.\\

Los pasos a seguir para realizar una modelación con el modelo WRF consiste en:
\begin{enumerate}
\item Compilar el WPS
\item Compilar el WRF
\item Determinar el área de estudio
\item Selección de los dominios del modelo WRF para la inicializacion del modelo
\item Selección de las parametrizaciones que se van a emplear
\item Descargar los datos que van a alimentar el modelo
\item Realizar el pre-proceso con el WPS
\item Realizar el proceso de modelación con el WRF
\item Realización del pos-proceso con python3.6


\end{enumerate}




%% Tabla del porcentaje de coincidencias entre la precipitación horaria y diária


\begin{table}[H]
\begin{center}
\caption{Porcentaje de coincidencia de los valores de precipitación horaria y el valor de la suma de la precipitación instantánea cada 10 minutos de un día.}
\begin{tabular}{llr}
\toprule
Estación &  Porcentaje de coincidencia (\%) \\
\midrule
0  &  21195160 &     24.85 \\
1  &  21201200 &     15.17 \\
2  &  21201580 &     33.71 \\
3  &  21202270 &     25.42 \\
4  &  21205012 &     18.12 \\
5  &  21205791 &     39.62 \\
6  &  21206600 &     39.13 \\
7  &  21206790 &     20.23 \\
8  &  21206920 &     13.25 \\
9  &  21206930 &     13.99 \\
10 &  21206940 &     20.32 \\
11 &  21206950 &     13.75 \\
12 &  21206960 &     47.23 \\
13 &  21206980 &     27.37 \\
14 &  21206990 &     61.43 \\
15 &  21209920 &     43.06 \\
16 &  23125170 &     27.50 \\
17 &  24015110 &     34.88 \\
18 &  26127010 &     27.35 \\
19 &  35025080 &     14.69 \\
20 &  35025090 &     20.11 \\
21 &  35027001 &     34.07 \\
22 &  35027002 &      0.00 \\
\bottomrule
\end{tabular}

\label{table:compar-porcentaje}
\end{center}
\end{table}


\section{Conclusiones}

Las variables máximas y mínimas de temperatura de las estaciones HYDRAS no coinciden con los valores máximos y mínimos.

Existen lapsos de tiempo en los cuáles la temperatura no es registrada. Por esta razón se sugiere realizar una unión de los valores de temperatura máxima, mínima y promedio, donde prime la temperatura promedio.

Existen diferencias de temperatura entre las estaciones convencionales y las estaciones automáticas, especialmente en condiciones de alta radiación y poca velocidad del viento.

\end{comment}
%\subsection{Selección de una fecha de estudio para realizar la configuración del WRF}

El periodo simulado inicia desde el primer día del mes de febrero del año 2007 a las 00 horas UTC hasta el quinto día del mes de febrero del año 2007 a las 00 horas UTC y las heladas comenzaron a presentarse desde el  segundo día de febrero del 2007 a las 10 horas UTC.\\

La razón para haber escogido estos días es que en estas fechas se presentó un evento de helada en la Sabana de Bogotá y fue uno de los más severos de los últimos 20 años. Adicionalmente el mes de febrero es uno de los meses con mayor probabilidad de heladas. Se seleccionó este caso porque el 4 de febrero del 2007 se presentó una de las temperaturas más bajas registradas para la Sabana de Bogotá, en la Tabla \ref{table:caso1} se evidencian los registros de las temperaturas máximas y mínimas de algunas estaciones convencionales para esa fecha y el valor reportado por la estación automática de TIBAITATÁ. Según la estación automática TIBAITATÁ la helada tuvo una duración de 5 horas comenzando a las 2 a.m. y finalizando a las 7 a.m el valor mínimo es que se registró es de -4.6\celc, esto la convierte en la helada más larga para nuestro periodo de estudio, (ver Figura \ref{gra:tmp_tiba_2007}). Adicionalmente en el mismo día presentó temperaturas sobre 20\celc que duró 5 horas iniciando a las 11 a.m. y finalizando a las 4 p.m., el valor más alto registrado para esta estación fué de 26.3\celc.

\begin{figure}[H]
    \centering
    \includegraphics[draft=false, scale=0.4]{altas_bajas_2007/21206990.png}
    \caption{Temperatura registrada por una estación automática entre los días 31 de enero del 2007 y el cinco de febrero del 2007.}
    \label{gra:tmp_tiba_2007}
\end{figure}


\begin{table}[H]
\centering

\begin{tabular}{llll}
Temperatura \celc & Código   & Nombre de la estación & Municipio \\ \hline
-8.8           & 21205980 & PROVIDENCIA GJA       & Tenjo     \\
-7.4           & 21205920 & SUASUQUE              & Sopó      \\
-7.0           & 21205880 & FLORES CHIBCHA        & Madrid    \\
-4.6           & 21205420 & TIBAITATÁ             & Mosquera  \\
-4.7           & 21205420 & TIBAITATÁ [Automática]& Mosquera  \\
 21.6 &  21205700 &     GUASCA [21205700] &       Guasca \\
 22.0 &  21205790 &        APTO EL DORADO &  Bogotá D.C. \\
 22.2 &  21206230 &         VEGAS LAS HDA &  Bogotá D.C. \\
 22.5 &  21205770 &     BASE AEREA MADRID &       Madrid \\
 23.8 &  21205980 &       PROVIDENCIA GJA &        Tenjo \\
 23.8 &  21206260 &     C.UNIV.AGROP-UDCA &  Bogotá D.C. \\
 24.0 &  21206210 &    FLORES COLOMBIANAS &        Funza \\
 24.2 &  21206660 &    COL SANTIAGO PEREZ &  Bogotá D.C. \\
 24.8 &  21205420 &  TIBAITATA [21205420] &     Mosquera \\
 26.3 &  21205420 &  TIBAITATA [Automática] &     Mosquera \\
\end{tabular}
\caption{Registros de las temperaturas debajo de 0\celc y sobre 20\celc para del día 4 de febrero del 2007}
\label{table:caso1}
\end{table}



\subsection{Corrección por altura}

Se realizó una corrección de los valores de temperatura simulada por el WRF basado en la altura real de las estaciones del IDEAM. Las estaciones del IDEAM tienen un valor asociado de altura (Ver Tabla \ref{tab:correccion_alturas} columna "Altura IDEAM") para cada una de sus estaciones, pero algunos de estos no concuerdan con la realidad. Un ejemplo de esto es el caso de la estación llamada PMO GUACHENEQUE el IDEAM reporta una altura de 2300 msnm pero según el modelo digital de elevación (DEM) ALOS-PALSAR creado por \citet{ASFDAAC2007} esta ubicación tiene una elevación de 3288, la cual presenta una mejor aproximación a la altura de la zona donde se encuentra la estación. Por esta razón no se tomaron los valores de altura sugeridos por el IDEAM; en cambio se usó la altura brindada por el DEM de ALOS-PALSAR para cada una de las estaciones estudiadas, éstos valores de altura serán usados como la altura de referencia a partir de este punto (Ver Tabla \ref{tab:correccion_alturas} columna "Altura ALOS-PALSAR"). Usando la ubicación de cada una de las estaciones se realizó una extracción de los valores de altura presentes en los archivos netCDF generados en el pre-procesamiento \textit{WRF Preprocessing System} (WPS) (Ver Tabla \ref{tab:correccion_alturas} columna "Altura modelo") a partír de la variable "HGT\_M" la cuál hace referencia a la altura topográfica en metros.\\

Los valores modelados fueron corregidos teniendo en cuenta la diferencia entre la altura de la modelación ($A_{wrf}$) y la altura real ($A_{real}$). Donde se calcula la diferencia entre el modelo y la altura real y se multiplica por un factor de 6.5\celc\ por cada 1000 metros (ver Ecuación \ref{correccion_altura}). El resultado de esta operación se le suma a los valores de temperatura modelados. Las tablas de las correcciones de las alturas se encuentran en el Anexo \ref{tab:correccion_alturas}

\begin{equation} \label{correccion_altura}
(A_{wrf} - A_{real}) \times 0.0065 \frac{^{\circ}C}{m}
\end{equation}
 Como lo sugiere \citet{ValenciaMonroy2015}, 
 \ref{tab:correccion_alturas}
 
\subsection{Valores asociados al GFS}
 
Profe para hacer los plots del GFS me demoré bastante porque no los sabía hacer pero acá están los resultados. En ellos se muestra que no había viento en esos momentos y que la nubosidad estaba en la parte sur del país, como lo muestra la carta sinóptica de superficie de la NOAA del anterior informe.

\begin{figure}[H]
	\begin{center}
		\begin{subfigure}[normla]{0.4\textwidth}
	\includegraphics[draft=false, scale=0.45]{altas_bajas_2007/barbas.png}
		\caption{Mapa de temperatura superficial y gráfica de barbas para el viento}
		\label{gra_barbas}
		\end{subfigure}
		~
				\begin{subfigure}[normla]{0.4\textwidth}
	\includegraphics[draft=false, scale=0.45]{altas_bajas_2007/precip.png}
		\caption{Imágen de los valores de agua precipitable. Las unidades son $\frac{kg}{m^2}$.}
		\label{gra_agua_precip}
		\end{subfigure}
		~
			\end{center}
	\caption{Imágenes creadas a partír de la información del GFS para el cuatro de febrero del 2007 a las 12 horas UTC. Para la realización de la Figura \ref{gra_agua_precip} se usó la variable \textit{Precipitable water} del modelo GFS. La cuál según la NOAA corresponde al agua que en teoría puede precipitar si las condiciones atmosféricas fuesen adecuadas.}
	\label{gra:taylor_totaltotal}	
\end{figure}

En la figura \ref{gra_barbas} podemos ver que los vientos estaban en calma, que sobre Colombia no estaba la ZCTI, como se vió en la carta control Figura \ref{fig:carta4}. Se puede observar que frente a Ecuador existía corrientes de vientos lo cuál coincide con la carta control. El agua precipitable de la Figura \ref{gra_agua_precip} se puede observar que hay una mancha roja más abajo del río Amazonas y si esta figura se compara con la figura \ref{fig:carta4}, podemos observar que se encuentra en la misma zona en la cuál se presenta alta nubosidad.\\


Como conlcusión podemos observar que el modelo GFS está viendo el comportamiento sinóptico de el área. Adicionalmente podemos decir que para el día 4 de febrero del 2007 se presentaban condiciones secas ya que la ZCIT no estaba sobre Colombia y una gran nubosidad se presentaba en el sur del país.



%Estos son los links de la información:
%http://www.nco.ncep.noaa.gov/pmb/docs/on388/table2.html
%https://sos.noaa.gov/datasets/gfs-forecast-model-precipitable-water-real-time/

\begin{figure}[H]
    \centering
    \includegraphics[draft=false, scale=0.4]{cartas/2007/07020402QPAA99.png}
    \caption{Carta control para el día cuatro de febrero del 2007}
    \label{fig:carta4}
\end{figure}
 
%\appendix
\clearpage
\addappheadtotoc
\appendixpage

\chapter{Gráficas de la frecuencia mensual de eventos extremos en la Sabana de Bogotá.}
\label{anexo:gra_mensual_eventos_sb}



%%% Gráfica de las fecuencias
\chapter{Anexo. }


\begin{figure}[H]
\centering
\includegraphics[draft=false, scale=0.3]{graficas_fecuencia/21205420.png}
%\caption{Distribución de las altas y bajas temperaturas de la estación cod 21205420.png.}
\end{figure}
%%%%%
\begin{comment}


%%%%%
\begin{figure}[H]
\centering
\includegraphics[draft=false, scale=0.3]{graficas_fecuencia/21205520.png}
%\caption{Distribución de las altas y bajas temperaturas de la estación cod 21205520.png.}
\end{figure}
%%%%%
%%%%%
\begin{figure}[H]
\centering
\includegraphics[draft=false, scale=0.3]{graficas_fecuencia/21205710.png}
%\caption{Distribución de las altas y bajas temperaturas de la estación cod 21205710.png.}
\end{figure}
%%%%%
%%%%%
\begin{figure}[H]
\centering
\includegraphics[draft=false, scale=0.3]{graficas_fecuencia/21205740.png}
%\caption{Distribución de las altas y bajas temperaturas de la estación cod 21205740.png.}
\end{figure}
%%%%%
%%%%%
\begin{figure}[H]
\centering
\includegraphics[draft=false, scale=0.3]{graficas_fecuencia/21205750.png}
%\caption{Distribución de las altas y bajas temperaturas de la estación cod 21205750.png.}
\end{figure}
%%%%%
%%%%%
\begin{figure}[H]
\centering
\includegraphics[draft=false, scale=0.3]{graficas_fecuencia/21205770.png}
%\caption{Distribución de las altas y bajas temperaturas de la estación cod 21205770.png.}
\end{figure}
%%%%%
%%%%%
\begin{figure}[H]
\centering
\includegraphics[draft=false, scale=0.3]{graficas_fecuencia/21205790.png}
%\caption{Distribución de las altas y bajas temperaturas de la estación cod 21205790.png.}
\end{figure}
%%%%%
%%%%%
\begin{figure}[H]
\centering
\includegraphics[draft=false, scale=0.3]{graficas_fecuencia/21205840.png}
%\caption{Distribución de las altas y bajas temperaturas de la estación cod 21205840.png.}
\end{figure}
%%%%%
%%%%%
\begin{figure}[H]
\centering
\includegraphics[draft=false, scale=0.3]{graficas_fecuencia/21205870.png}
%\caption{Distribución de las altas y bajas temperaturas de la estación cod 21205870.png.}
\end{figure}
%%%%%
%%%%%
\begin{figure}[H]
\centering
\includegraphics[draft=false, scale=0.3]{graficas_fecuencia/21205910.png}
%\caption{Distribución de las altas y bajas temperaturas de la estación cod 21205910.png.}
\end{figure}
%%%%%
%%%%%
\begin{figure}[H]
\centering
\includegraphics[draft=false, scale=0.3]{graficas_fecuencia/21205920.png}
%\caption{Distribución de las altas y bajas temperaturas de la estación cod 21205920.png.}
\end{figure}
%%%%%
%%%%%
\begin{figure}[H]
\centering
\includegraphics[draft=false, scale=0.3]{graficas_fecuencia/21205940.png}
%\caption{Distribución de las altas y bajas temperaturas de la estación cod 21205940.png.}
\end{figure}
%%%%%
%%%%%
\begin{figure}[H]
\centering
\includegraphics[draft=false, scale=0.3]{graficas_fecuencia/21205950.png}
%\caption{Distribución de las altas y bajas temperaturas de la estación cod 21205950.png.}
\end{figure}
%%%%%
%%%%%
\begin{figure}[H]
\centering
\includegraphics[draft=false, scale=0.3]{graficas_fecuencia/21205960.png}
%\caption{Distribución de las altas y bajas temperaturas de la estación cod 21205960.png.}
\end{figure}
%%%%%
%%%%%
\begin{figure}[H]
\centering
\includegraphics[draft=false, scale=0.3]{graficas_fecuencia/21205980.png}
%\caption{Distribución de las altas y bajas temperaturas de la estación cod 21205980.png.}
\end{figure}
%%%%%
%%%%%
\begin{figure}[H]
\centering
\includegraphics[draft=false, scale=0.3]{graficas_fecuencia/21206030.png}
%\caption{Distribución de las altas y bajas temperaturas de la estación cod 21206030.png.}
\end{figure}
%%%%%
%%%%%
\begin{figure}[H]
\centering
\includegraphics[draft=false, scale=0.3]{graficas_fecuencia/21206050.png}
%\caption{Distribución de las altas y bajas temperaturas de la estación cod 21206050.png.}
\end{figure}
%%%%%
%%%%%
\begin{figure}[H]
\centering
\includegraphics[draft=false, scale=0.3]{graficas_fecuencia/21206150.png}
%\caption{Distribución de las altas y bajas temperaturas de la estación cod 21206150.png.}
\end{figure}
%%%%%
%%%%%
\begin{figure}[H]
\centering
\includegraphics[draft=false, scale=0.3]{graficas_fecuencia/21206200.png}
%\caption{Distribución de las altas y bajas temperaturas de la estación cod 21206200.png.}
\end{figure}
%%%%%
%%%%%
\begin{figure}[H]
\centering
\includegraphics[draft=false, scale=0.3]{graficas_fecuencia/21206210.png}
%\caption{Distribución de las altas y bajas temperaturas de la estación cod 21206210.png.}
\end{figure}
%%%%%
%%%%%
\begin{figure}[H]
\centering
\includegraphics[draft=false, scale=0.3]{graficas_fecuencia/21206220.png}
%\caption{Distribución de las altas y bajas temperaturas de la estación cod 21206220.png.}
\end{figure}
%%%%%
%%%%%
\begin{figure}[H]
\centering
\includegraphics[draft=false, scale=0.3]{graficas_fecuencia/21206230.png}
%\caption{Distribución de las altas y bajas temperaturas de la estación cod 21206230.png.}
\end{figure}
%%%%%
%%%%%
\begin{figure}[H]
\centering
\includegraphics[draft=false, scale=0.3]{graficas_fecuencia/21206260.png}
%\caption{Distribución de las altas y bajas temperaturas de la estación cod 21206260.png.}
\end{figure}
%%%%%
%%%%%
\begin{figure}[H]
\centering
\includegraphics[draft=false, scale=0.3]{graficas_fecuencia/21206450.png}
%\caption{Distribución de las altas y bajas temperaturas de la estación cod 21206450.png.}
\end{figure}
%%%%%
%%%%%
\begin{figure}[H]
\centering
\includegraphics[draft=false, scale=0.3]{graficas_fecuencia/21206550.png}
%\caption{Distribución de las altas y bajas temperaturas de la estación cod 21206550.png.}
\end{figure}
%%%%%
%%%%%
\begin{figure}[H]
\centering
\includegraphics[draft=false, scale=0.3]{graficas_fecuencia/21206560.png}
%\caption{Distribución de las altas y bajas temperaturas de la estación cod 21206560.png.}
\end{figure}
%%%%%
%%%%%
\begin{figure}[H]
\centering
\includegraphics[draft=false, scale=0.3]{graficas_fecuencia/21206610.png}
%\caption{Distribución de las altas y bajas temperaturas de la estación cod 21206610.png.}
\end{figure}
%%%%%
%%%%%
\begin{figure}[H]
\centering
\includegraphics[draft=false, scale=0.3]{graficas_fecuencia/21206620.png}
%\caption{Distribución de las altas y bajas temperaturas de la estación cod 21206620.png.}
\end{figure}
%%%%%
%%%%%
\begin{figure}[H]
\centering
\includegraphics[draft=false, scale=0.3]{graficas_fecuencia/21206640.png}
%\caption{Distribución de las altas y bajas temperaturas de la estación cod 21206640.png.}
\end{figure}
%%%%%
%%%%%
\begin{figure}[H]
\centering
\includegraphics[draft=false, scale=0.3]{graficas_fecuencia/21206670.png}
%\caption{Distribución de las altas y bajas temperaturas de la estación cod 21206670.png.}
\end{figure}
%%%%%
%%%%%
\begin{figure}[H]
\centering
\includegraphics[draft=false, scale=0.3]{graficas_fecuencia/21206690.png}
%\caption{Distribución de las altas y bajas temperaturas de la estación cod 21206690.png.}
\end{figure}
%%%%%
%%%%%
\begin{figure}[H]
\centering
\includegraphics[draft=false, scale=0.3]{graficas_fecuencia/21206700.png}
%\caption{Distribución de las altas y bajas temperaturas de la estación cod 21206700.png.}
\end{figure}
%%%%%
%%%%%
\begin{figure}[H]
\centering
\includegraphics[draft=false, scale=0.3]{graficas_fecuencia/35025050.png}
%\caption{Distribución de las altas y bajas temperaturas de la estación cod 35025050.png.}
\end{figure}
%%%%%
%%%%%
\end{comment}

%\appendix
\clearpage
\addappheadtotoc
\appendixpage

\chapter{Gráficas de la prueba de cielo despejado para las estaciones automáticas escogidas.}


\begin{figure}[H]
\begin{subfigure}[normla]{0.45\textwidth}
\includegraphics[draft=false, scale=0.4]{g_cielo_despejado/21195160.pdf}
\caption{Estación SUBIA AUTOMATICA código 21195160.}
\end{subfigure}
~
\begin{subfigure}[normla]{0.45\textwidth}
\includegraphics[draft=false, scale=0.4]{g_cielo_despejado/21206790.pdf}
\caption{Estación HDA STA ANA AUTOM código 21206790.}
\end{subfigure}
~
\begin{subfigure}[normla]{0.45\textwidth}
\includegraphics[draft=false, scale=0.4]{g_cielo_despejado/21206940.pdf}
\caption{Estación CIUDAD BOLIVAR código 21206940.}
\end{subfigure}
~
\begin{subfigure}[normla]{0.45\textwidth}
\includegraphics[draft=false, scale=0.4]{g_cielo_despejado/21206980.pdf}
\caption{Estación STA CRUZ DE SIECHA código 21206980.}
\end{subfigure}
~
%%%%%#############
%%%%%
\end{figure}
           
\begin{figure}[]\ContinuedFloat
\centering
\begin{subfigure}[normla]{0.45\textwidth}
\includegraphics[draft=false, scale=0.4]{g_cielo_despejado/21206990.pdf}
\caption{Estación TIBAITATA AUTOMATICA código 21206990.}
\end{subfigure}
~
\begin{subfigure}[normla]{0.45\textwidth}
\includegraphics[draft=false, scale=0.4]{g_cielo_despejado/24015110.pdf}
\caption{Estación LA BOYERA AUTOMATICA código 24015110.}
\end{subfigure}
~

\begin{subfigure}[normla]{0.45\textwidth}
\includegraphics[draft=false, scale=0.4]{g_cielo_despejado/35075070.pdf}
\caption{Estación CHINAVITA AUTOMATICA código 35075070.}
\end{subfigure}
~
\begin{subfigure}[normla]{0.45\textwidth}
\includegraphics[draft=false, scale=0.4]{g_cielo_despejado/35085080.pdf}
\caption{Estación LA CAPILLA AUTOM código 35085080.}
\end{subfigure}
~
    \caption{Ejemplos de prueba de cielo despejado para las estaciones automáticas escogidas.}
    \label{fig:my_label}
\end{figure}


%\include{anexo2_cc_conve_V}
%{anexo2_gra_val}

%\texttt{
%\include{anexo3_namelist_IV}
%}
%\include{anexo3_seritiem2007_VII} % Este
%\include{anexo3_correccion_alturaVI} % Este
%\include{anexo3_ta_dominios_VIII} %este
%\include{anexo3_taylor_X} %este
%\appendix

\chapter{Anexo IX. Resultados de las comparaciones entre los estadísticos y el modelo para la selección de los tiempos.}
\begin{figure}
    \centering

\begin{subfigure}[normla]{0.4\textwidth}
\includegraphics[draft=false, scale=0.2]{graficas_taylor_tiempo/taylor_21195160.png}
\caption{Estación SUBIA AUTOMATICA  código 21195160.}
\end{subfigure}
~
\begin{subfigure}[normla]{0.4\textwidth}
\includegraphics[draft=false, scale=0.2]{graficas_taylor_tiempo/taylor_21206790.png}
\caption{Estación HDA STA ANA AUTOM  código 21206790.}
\end{subfigure}
~
\begin{subfigure}[normla]{0.4\textwidth}
\includegraphics[draft=false, scale=0.2]{graficas_taylor_tiempo/taylor_21206930.png}
\caption{Estación PMO GUERRERO  código 21206930.}
\end{subfigure}
~
\begin{subfigure}[normla]{0.4\textwidth}
\includegraphics[draft=false, scale=0.2]{graficas_taylor_tiempo/taylor_21206940.png}
\caption{Estación CIUDAD BOLIVAR  código 21206940.}
\end{subfigure}
~
\begin{subfigure}[normla]{0.4\textwidth}
\includegraphics[draft=false, scale=0.2]{graficas_taylor_tiempo/taylor_21206950.png}
\caption{Estación PMO GUACHENEQUE  código 21206950.}
\end{subfigure}
~
\begin{subfigure}[normla]{0.4\textwidth}
\includegraphics[draft=false, scale=0.2]{graficas_taylor_tiempo/taylor_21206980.png}
\caption{Estación STA CRUZ DE SIECHA  código 21206980.}
\end{subfigure}
~
\end{figure}
           
\begin{figure}[H]
\centering
\begin{subfigure}[normla]{0.4\textwidth}
\includegraphics[draft=false, scale=0.2]{graficas_taylor_tiempo/taylor_21206990.png}
\caption{Estación TIBAITATA AUTOMATICA  código 21206990.}
\end{subfigure}
~
\begin{subfigure}[normla]{0.4\textwidth}
\includegraphics[draft=false, scale=0.2]{graficas_taylor_tiempo/taylor_24015110.png}
\caption{Estación LA BOYERA AUTOMATICA  código 24015110.}
\end{subfigure}
~
\begin{subfigure}[normla]{0.4\textwidth}
\includegraphics[draft=false, scale=0.2]{graficas_taylor_tiempo/taylor_35075070.png}
\caption{Estación CHINAVITA AUTOMATICA  código 35075070.}
\end{subfigure}
~
\begin{subfigure}[normla]{0.4\textwidth}
\includegraphics[draft=false, scale=0.2]{graficas_taylor_tiempo/taylor_35085080.png}
\caption{Estación LA CAPILLA AUTOM  código 35085080.}
\end{subfigure}
~
    \caption{Estación X.}
    \label{fig:my_label}
\end{figure}
%\include{anexo3_tabla_tiempos_IX}
%\include{anexo3_taylordea5_XI}

%\texttt{•}
%\include{anexo4_XII_namelistinput_mejorparame}
%}
%\include{anexograficas_taylor_tiempos_4casos_total}
%\include{anexo4_XIII_taylor_tiempos}
%%%%%%%%%%%%%%%%%%%%%%%%%%%%%%%%%%
\bibliographystyle{apalike}

\bibliography{biblio}%Nombre del archivo que tiene las citas

\end{document}
